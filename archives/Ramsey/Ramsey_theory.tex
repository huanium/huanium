\documentclass{article}
\usepackage{amsmath}
\usepackage[english]{babel}
\usepackage[utf8]{inputenc}
\usepackage{fancyhdr}

\pagestyle{fancy}
\fancyhf{}
\rhead{Huan Q. Bui}
\lhead{Ramsey Fringes Derivation}
\rfoot{\thepage}

\begin{document}
\section{Ramsey Fringes Overview:}
\noindent 
Following a double pulse, the population of the excited state is:
\begin{equation*}
\boxed{P_2 = 4\sin^{2}\theta\sin^{2}\frac{\Omega'\tau}{2} \left\lbrace \cos \frac{\Omega'\tau}{2}\cos \frac{\Delta_0 T}{2} - \cos\theta\sin\frac{\Omega'\tau}{2} \sin \frac{\Delta_0 T}{2}\right\rbrace}
\end{equation*}
\noindent 
under the assumption that initially,
\[
\begin{pmatrix}
C_1(0)\\C_2(0)
\end{pmatrix}
=
\begin{pmatrix}
1\\0
\end{pmatrix}
\]
\noindent 
The final state vector is:
\[
\begin{pmatrix}
C_1(2\tau+T)\\C_2(2\tau+T)
\end{pmatrix}
=
\rho_2 D \rho_1
\begin{pmatrix}
C_1(0)\\C_2(0)
\end{pmatrix}
\]
where $|C_1|^2 + |C_2|^2 = 1$ for all value of time, and $\rho_1$ and $\rho_2$ are propagators associated with Pulse 1 and Pulse 2 (both with width $\tau$), respectively. $D$ is a propagator associated with the field-free evolution of duration $T$.\\

\noindent Specifically, in the interaction representation:
\[
\rho_1 = e^{-i\overline\Delta\tau}
\begin{pmatrix}
e^{-i\frac{\Delta_0}{2}\tau}\left(\cos\frac{\Omega'\tau}{2} + i\cos\theta\sin\frac{\Omega'\tau}{2}\right)  
& ie^{-i\left(\frac{\Delta_0}{2}\tau - \phi_0 \right)} \sin\theta\sin\frac{\Omega'\tau}{2} 
\\
ie^{i\left(\frac{\Delta_0}{2}\tau + \phi_0 \right)} \sin\theta\sin\frac{\Omega'\tau}{2} 
& 
e^{i\frac{\Delta_0}{2}\tau}\left(\cos\frac{\Omega'\tau}{2} - i\cos\theta\sin\frac{\Omega'\tau}{2}\right)  
\end{pmatrix}\]

\noindent 
\[
\rho_2 = e^{-i\overline\Delta\tau}
\begin{pmatrix}
e^{-i\frac{\Delta_0}{2}\tau}\left(\cos\frac{\Omega'\tau}{2} + i\cos\theta\sin\frac{\Omega'\tau}{2}\right)  
& ie^{-i\left(\frac{\Delta_0}{2}\tau + \phi_0 \right)} \sin\theta\sin\frac{\Omega'\tau}{2}e^{-i\Delta_0(\tau+T)}
\\
ie^{i\left(\frac{\Delta_0}{2}\tau + \phi_0 \right)} \sin\theta\sin\frac{\Omega'\tau}{2}e^{i\Delta_0(\tau+T)}
& 
e^{i\frac{\Delta_0}{2}\tau}\left(\cos\frac{\Omega'\tau}{2} - i\cos\theta\sin\frac{\Omega'\tau}{2}\right)  
\end{pmatrix}\]
\noindent
and 
\[D
=
\begin{pmatrix}
1 & 0 \\ 0 & 1
\end{pmatrix}
\]
\noindent
Note that $D$, in the interaction representation, is the identity matrix. This is different from Ramsey's original approach in which the state vector does evolve and change during the delay time $T$. The angle $\theta$ is defined as:
\[
\sin\theta = \frac{{\Omega_0}^*}{\Omega'}
\]
\noindent
and
\[
\cos\theta = \frac{\Delta_0 + \Delta_d}{\Omega'}
\]
\noindent
where ${\Omega_0}^*$ is the complex conjugate of the Rabi rate, and $\Omega'$ can be defined as the ``effective Rabi rate.''
\[
\Omega'= \sqrt{|{\Omega_0}^*| + (\Delta_0 + \Delta_d)^2}
\]
\newpage

\section{Detailed Derivation for $P_f$}
In the interaction representation,
\begin{align}
i\begin{pmatrix}
\dot{a}_i(t)\\
\dot{a}_f(t)
\end{pmatrix}
=
\begin{pmatrix}
\Delta_i & -\frac{\Omega*_0}{2}e^{-i\Delta_0t}\\
-\frac{\Omega_0}{2}e^{i\Delta_0t} & \Delta_f
\end{pmatrix}
\begin{pmatrix}
a_i(t)\\
a_f(t)
\end{pmatrix}
\end{align}

\end{document}
