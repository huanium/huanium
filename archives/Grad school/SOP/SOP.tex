\documentclass[12pt]{article}
\usepackage[left=0.8in,right=0.8in,top=0.7in,bottom=0.8in]{geometry}

%\usepackage{newpxtext,newpxmath}
\usepackage{amsfonts}
\usepackage{mathptmx}
\usepackage{color}
\usepackage{bold-extra}
\usepackage{setspace}








\begin{document}
	
	
\begin{center}
	\textbf{Statement of Purpose (MIT)}
\end{center}\vspace{-5pt}	
I am currently in my fourth year at Colby College studying physics, mathematics, and statistics. I am interested in quantum information science and atomic physics for they involve a strong interplay between theories and experiments. My research experience at Colby, at the Joint Quantum Institute (JQI), and most recently at the Perimeter Institute for Theoretical Physics have collectively shaped my interest not just in developing quantum algorithms for simulating complex systems but also in table-top atomic physics experiments for quantum science. As I seek to pursue research in this area in graduate school and beyond, MIT Physics is a natural step next for me because of the wide range of exciting problems being tackled here and the opportunity to learn from the leading scientists in this field.  \\ \vspace{-9pt}

Under the supervision of Dr. Timothy Hsieh at Perimeter Institute since May 2020, I have been researching simulation of non-trivial quantum states using the variational quantum approximate optimization algorithm (QAOA). My project initially explored the possibility of using the cluster-state ansatz and projective measurements in conjunction with the variational QAOA algorithm to create a protocol whose circuit depth scales sublinearly as a function of system size $(L)$. Such an algorithm would extend Dr. Hsieh's previous finding that QAOA could simulate a large class of non-trivial quantum states with a linearly-depth ($\mathcal{O}(L)$) circuit. While working on this, I found numerically that a linear-depth QAOA ansatz can perfectly simulate any ground state of the disordered one-dimensional quantum Ising model. As a result, although the sublinear circuit remains an open question, this new finding pushes our research in a slightly different trajectory towards exploring extensions and further applications of the current linear-scaling protocol. In particular, we are now interested in studying its ability to let us prepare highly excited states and understand many-body localization. Following \cite{PhysRevResearch.1.033062} and \cite{Higgott2019variationalquantum}, I showed that an $\mathcal{O}(L+1)$-circuit could perfectly target a large class of a excited states of the one-dimensional non-uniform quantum Ising model. This opens up possibilities to... \textcolor{blue}{Say something here to close this paragraph off.}\\ \vspace{-9pt}

In the summer of 2019 and January 2020, I joined Professor Steven Rolston's lab at JQI at the University of Maryland, College Park to work on his experiments studying the long-range interactions of Rb atoms that are MOT-trapped around an optical nanofiber (ONF) via measuring their collective decay. Under the supervision of Dr. Hyok Sang Han, I created a method to optimize light polarization in the ONF by imaging the light scattered from the ONF to quantify circular and elliptical polarizations. Unlike free-space, the ONF medium introduces non-uniform birefringence and cylindrically asymmetric longitudinal polarization; thus, this system is necessary to guarantee quasi-linear polarization in the ONF. Additionally, I developed a stand-alone experimental control program in Python using the NI-DAQmx libraries. This program removed the group's reliance on the less backward-compatible LabView. In January 2020, while working to improve the extinction rate of an electric-optical modulator for pulse-switching, Dr. Han and I observed a transient decay flash that was much faster than even the fastest superradiance mode of the system. It turned out that this phenomenon was only recently discovered for a 3-dimensional atom cloud and is yet fully understood. Because of the unique 1-dimensional geometry of the ONF experiment, investigating this phenomenon promises new and fascinating findings.   \textcolor{red}{finish this} \\ \vspace{-9pt}

During the academic year, I work on ultracold atom experiments at Colby under Professor Charles Conover. Applying the experimental techniques I learned from my time at JQI, I am currently working towards a Physics Honors Thesis on an experiment to measure the lifetime of certain $K$ quantum states by counting fluorescence photons from a MOT-trapped cloud of ultracold $^{39}$K. In previous years, I have constructed a variety of laboratory apparatus from external cavity diode lasers to laser frequency-stabilization electronics and carried out millimeter-wave precision measurements of energy level and quantum defect on ultracold $^{39}K$ in Rydberg states. Particularly from summer 2018 to spring 2019, I measured the $nd_j \to (n+1)d_j$ two-photon transitions for $30 \leq n \leq 35$ energies to $5\times 10^{-8}$ accuracy to determine $d$-state quantum defects and absolute energy levels of potassium. At our level of precision, the AC Stark shift caused by the millimeter-wave source is significant; consequently, we were often required to extrapolate our data to obtain an unbiased value. Applying Ramsey's separated oscillatory field technique \cite{PhysRev.78.695}, I eliminated this necessity and provided an alternative measurement scheme with comparable precision. I presented this work at Colby's summer research symposium (CUSRR 2018) and at APS DAMOP 19 in Milwaukee, Wisconsin jointly with Professor Conover. We also presented measurements of $p$-state fine structure and quantum defects in DAMOP 19 and $f$-, $g$-, and $h$-state quantum defects at APS DAMOP 20. \\ \vspace{-9pt}




Apart from my primary interests in QI and atomic physics, I also explore general relativity and mathematical physics. Having thoroughly enjoyed General Relativity with Professor Robert Bluhm at Colby in fall 2018, I took three independent studies with him in the subsequent semesters studying classical field theory with a focus to better our understanding of massive gravity. In particular, we reviewed the development of the theory and the origins of nonlinear behaviors therein, namely the Vainshtein radius and various screening mechanisms. This 2-year project resulted in a 150-page exposition of the topic which I wrote and made available on my website. Utilizing the tools acquired from my physics and mathematics training, I followed original papers and Kurt Hinterbichler's review \cite{RevModPhys.84.671} to reproduce important results, providing explicit derivations and justifications wherever necessary and filling in various knowledge gaps to make the document accessible to undergraduates like myself. These gaps range from basic quantum field theory (from Klein-Gordon to $\phi^4$ theory) to how to use the Mathematica packages \texttt{xPert} and \texttt{xAct} to symbolically study perturbative general relativity. \\ \vspace{-9pt}

In my junior year, I also began working towards my Mathematics Honors Thesis under Professor Evan Randles at Colby on iterative convolutions of a class of complex-valued functions. The correspondence between the convolution and the Fourier transform makes convolution powers a central object for generating numerical solutions for partial differential equations. Using Python for computing convolution powers and Mathematica for the Fourier transforms, I gave numerical evidence verifying our new local (central) limit conjecture related to this problem. Motivated by this finding, Professor Randles and I have recently constructed via measure theory a generalized quasi-polar-coordinate integration formula for estimating the Fourier transform of a class of functions in question and for applications to operator semigroups. While the main conjecture still stands, this new result is sufficiently important and applicable that Professor Randles and I have decided to prepare a manuscript for its publication. We will also present this work at the Joint Mathematics Meeting in January of 2021. \\ \vspace{-9pt}

I have contacted Professor Soonwon Choi, who will be starting at MIT in July 2021 and whose research on quantum many-body systems, quantum information dynamics, and their applications align very well with my experience at Perimeter and interest in a balance between analytical theory and experiments. I have also expressed interest in Professor Martin Zwierlein's research on..., particularly his upcoming experiment of rotating BEC? Finally, I also find Professor Isaac Chuang's research...  \\ \vspace{-9pt}

\noindent Thank you for your time and consideration.\\ \vspace{-9pt}

\noindent Huan Q. Bui\\
\today


\newpage
\begin{spacing}{0}
	\bibliographystyle{ieeetr}
	\bibliography{ref} 
\end{spacing}

	

















	
	
	
	
	
\end{document}