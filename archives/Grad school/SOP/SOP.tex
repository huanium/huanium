\documentclass[12pt]{article}
\usepackage[left=0.8in,right=0.8in,top=0.8in,bottom=0.8in]{geometry}

\usepackage{newpxtext,newpxmath}
\usepackage{color}


\pagenumbering{gobble}



\makeatletter
\makeatother





\begin{document}
	
	
\begin{center}
	\textbf{Statement of Purpose (MIT)}
\end{center}	
I am currently in my fourth year at Colby College studying physics, mathematics, and statistics. I am drawn to research problems in quantum information science and atomic, optical, and molecular physics, for they involve a strong interplay between theories and experiments. My research experience at Colby, at the Joint Quantum Institute, and most recently at the Perimeter Institute for Theoretical Physics have collectively shaped my interest in developing and applying quantum algorithms in simulating complex quantum systems. As I seek to continue pursuing research in this area in graduate school and beyond, MIT is a natural choice for me. \textcolor{red}{Why?}

Under the supervision of Dr. Timothy Hsieh at Perimeter Institute since May 2020, I have been researching simulation of non-trivial quantum states using the variational quantum approximate optimization algorithm (QAOA). My project to explore the possibility of using the cluster-state ansatz and projective measurements in conjunction with the variational QAOA algorithm to create a protocol whose circuit depth scales sublinearly as a function of system size $(L)$. Such an algorithm would extend Dr. Hsieh's previous finding that QAOA could simulate a large class of non-trivial quantum states with a linearly-scaling ($\mathcal{O}(L)$) circuit. However, while exploring this possibility, I found numerically that a linear-depth QAOA ansatz can perfectly simulate any ground state of the disordered one-dimensional quantum Ising model. While the sublinear circuit still an open question, this finding pushes our research in a slightly different trajectory towards exploring extensions and further applications of the current linear-scaling protocol. In particular, we are now interested in studying its ability to let us prepare highly excited states and understand many-body localization. Following the works of \textcolor{red}{cite Japanese paper} and \textcolor{blue}{cite VQE paper}, I showed that an $\mathcal{O}(L+1)$-circuit could perfectly target a large class of a excited states of the one-dimensional non-uniform quantum Ising model. This opens up possibilities to \textcolor{blue}{Say something here to close this paragraph off.}

Prior to my internship at Perimeter, I have spent more than two years working on various experimental projects on ultracold atoms at Colby under Professor Charles Conover and at the Joint Quantum Institute under Professor Steven Rolston. 

Apart from my primary interests in QI and AMO physics, I also dedicate significant efforts towards exploring mathematical physics and general relativity. Since junior year, I have been working towards my Mathematics Honors Thesis on the study of iterative convolutions of finitely-supported complex-valued functions on $\mathbb{Z}^d$ with Professor Evan Randles at Colby. Last year, I gave numerical evidence indicative of a new local (central) limit theorem related to related to this problem. This has led us to establish, via measure theory, a generalized quasi-polar-coordinate integration formula for estimating the Fourier transform of a class of functions in question and for applications to operator semigroups. This work is currently in preparation for publication and will be presented at the Joint Mathematics Meeting in January of 2021. From 2018 to May 2019, 






 


	
	
	
	
	
	
	

\newpage	
	
	
\begin{center}
	\textbf{Statement of Purpose}
\end{center}
	

\noindent I am a third-year undergraduate at Colby College majoring in physics and mathematics and minoring in statistics, and I am applying for the both the summer school and a summer research internship at the Perimeter Institute. I am drawn to problems in quantum information (QI) and atomic, molecular, and optical (AMO) physics research, for they involve a strong interplay between theories and experiments. As I wish to pursue theoretical QI research in graduate school, I believe an opportunity at the Perimeter Institute will be an excellent experience for me. My interest in QI stems from my lab work and desire to describe quantum systems with mathematics. I have been taking advanced physics and mathematics courses and actively involved in AMO physics research at Colby (since 2017) and at the Joint Quantum Institute (JQI) at the University of Maryland, College Park (summer, winter 2019). I am also interested in general relativity and mathematical physics. I am currently involved in applied mathematics research and independent studies on massive gravity at Colby.   \\


At JQI, I joined the Rolston Group where we study infinite-range interactions of Rb atoms trapped around an optical nanofiber (ONF) via their collective decay. One of our future endeavors is to have an optical dipole trap using an ONF. However, unlike in typical free-space dipole traps where no waveguide is used, a control system for light polarization state is necessary in our setup to account for non-uniform birefringence and a cylindrically asymmetric longitudinal polarization state introduced by the ONF. My contribution was building the Nd:YAG 1064 nm optical arrangement and creating a method to optimize polarization in the ONF. I was able to obtain quasi-linearly polarized light via an imaging system, which I created to quantify circular and elliptical polarizations. The system consists of two orthogonal CCD cameras equipped with polarizing filters from which the ratio of detected optical power characterizes the polarization state in the ONF. I also developed a stand-alone experimental control program in Python using the NI-DAQmx libraries, independent of LabView. At the moment, I am directly involved in the collective decay measurements. \\ 

I attribute my opportunity at JQI to more than two years of experience researching Rydberg K atoms at Colby with Professor  Conover. In his lab, I have built electronics to stabilize external cavity diode lasers' wavelengths and programmed waveform generators for various purposes including fast MOT field switching to study the dynamics of the MOT cloud in the abrupt absence of the trapping field. In previous years, the Conover Group focused on precision measurements of $d$-$d$ and $s$-$p$ Rydberg mm-wave transitions in K. My role over the summer of 2018 was to study Ramsey's separated oscillatory fields as an alternative to our conventional three-step measurement method. I modified the single-pulse excitation scheme to double-pulse and recorded Ramsey fringes in the frequency domain and Rabi oscillations in the time domain. From there, I used a simple two-level atom model to derive mathematical expressions for the observed fringes and oscillations, from which I extracted the desired measurements with only two steps. This work resulted in a poster presentation at my college's research retreat (CUSRR 2018) and another at APS DAMOP 2019.  \\


Beside experimental work, I am fascinated by theoretical physics and the applications of mathematics in QI/AMO physics and general relativity, which I have been exploring in advanced courses and independent studies. By the end of this academic year, I will have finished the required physics curriculum plus one semester on QI and four semesters on classical field theory and massive gravity (material from Hinterbichler\cite{RevModPhys.84.671}, Sean Carroll's \textit{Spacetime \& Geometry}, and Anthony Zee's \textit{Quantum Field Theory in a Nutshell}). I will also have completed two semesters of linear algebra, abstract algebra (with algebraic geometry), analysis, probability, and differential equations. For my Matrix Analysis final project of Spring 2018, I presented the construction of the tensor product and its application in a simple 2-qubit entanglement quantum circuit. Now, I am researching the convolution powers of complex-valued functions and related topics in harmonic analysis with Professor Evan Randles. I hope to turn my results into an Honors Thesis for the mathematics major. \\

  
A summer research at the Perimeter Institute will provide me with an excellent opportunity to apply my experience in experimental physics and interest in theory at large to tackle problems in mathematical or theoretical physics. As I wish to pursue physics in graduate school and academia, I believe a summer research at Perimeter will allow me to establish a strong transition.










\bibliography{ref} 
\bibliographystyle{ieeetr}
















	
	
	
	
	
\end{document}