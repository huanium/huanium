\documentclass[12pt]{article}
\usepackage[left=1in, right=1in,top=1in,bottom=1in]{geometry}

%\usepackage{newpxtext,newpxmath}
\usepackage{amsfonts}
\usepackage{tgtermes}
\usepackage{color}
\usepackage{bold-extra}
\usepackage{setspace}








\begin{document}
	
	
\begin{center}
	\textbf{Statement of Objectives (MIT)}
\end{center}\vspace{-5pt}	
I am drawn to research problems in quantum information science and atomic physics for the strong interplay between theory and experiment. My research training at Colby, the Joint Quantum Institute (JQI), and most recently the Perimeter Institute for Theoretical Physics have collectively shaped my interest in table-top quantum science experiments as well as theoretical aspects of quantum simulations of complex systems. I will soon complete my undergraduate studies with two Honors Theses on experimental atomic physics and mathematical physics. The next step for me is to define a research path and apply my training to address a challenging open problem in quantum science. I believe that MIT Physics would be a fantastic option to that end, for the wide range of novel problems being tackled here and the strong collaboration between theorists and experimentalists at MIT and at the Center for Ultracold Atoms.    \\ \vspace{-9pt}

Under the supervision of Dr. Timothy Hsieh at Perimeter Institute since May 2020, I have been researching efficient variational simulation of non-trivial quantum states using the quantum approximate optimization algorithm (QAOA) ansatz in a quantum-classical hybrid setting. Recently, Dr. Hsieh and colleagues have developed such a protocol that could target with perfect fidelity a class of non-trivial states (including the GHZ state and any arbitrary point in the phase diagram of the 1D transverse-field Ising model (TFIM)), using an L-deep circuit where L is the system size. My project initially explored the possibility of improving Dr. Hsieh's protocol: to achieve a sublinear-depth circuit by incorporating aspects of measurement-based quantum computation schemes such as the cluster-state ansatz and projective measurements. Surprisingly, while benchmarking the original linear-depth QAOA ansatz, I discovered numerically that it is capable of perfectly simulating any ground state of not just the translationally-invariant TFIM as discussed in Dr. Hsieh's paper, but also the TFIM with completely random transverse field and couplings. This new finding steered my research towards understanding exactly how the QAOA ansatz consistently gives perfect fidelity, as well as its extensions and applications in condensed matter physics. To find out how QAOA makes such a good ansatz, I will most likely study Dr. Hsieh's conjecture regarding the correspondence between the wavefunction picture and free-fermion picture of the problem under the Jordan-Wigner transformation. Regarding applications of the current protocol, I have recently made new progress by showing that an (L+1)-deep QAOA ansatz could perfectly target a large class of excited states for any TFIM with random fields and couplings. From here, the natural next objective would be to target a larger class of disordered TFIM Hamiltonians that cannot be put into a free-fermion form. Ultimately, Dr. Hsieh and I are hoping to establish some relationships between this protocol our ability to understand many-body localization. While this is an ambitious goal, I am excited to see what might unfold in the next months as the project progresses. \\ \vspace{-9pt}

In the summer of 2019 and January 2020, I joined Professor Steven Rolston's lab at JQI to work on  experiments measuring their collective decay of Rb atoms MOT-trapped around an optical nanofiber to study their long-range interactions. Under the supervision of Dr. Hyok Sang Han, I created a method to optimize light polarization in the nanofiber by imaging the light scattered from the fiber to quantify circular and elliptical polarizations. The nanofiber medium introduces non-uniform birefringence and a cylindrically asymmetric longitudinal polarization. As a result, this optimization system was necessary to guarantee at least a quasi-linear polarization. Additionally, I developed a stand-alone experimental control program in Python using the NI-DAQmx libraries, removing the group's reliance on the less backward-compatible LabView. In January 2020, while working to improve the extinction rate of an electric-optical modulator for pulse-switching, Dr. Han and I observed a transient decay flash that was much faster than even the fastest superradiance mode of the system. It turned out that this phenomenon was only recently discovered for a 3-dimensional atom cloud and is not yet fully understood. Due to the unique 1-dimensional geometry of our nanofiber experiment, the team at JQI is constructing a model to describe and explain this behavior; this promises new insights into the phenomenon.  \\ \vspace{-9pt}

At Colby, I work on ultracold atom experiments under Professor Charles Conover. Applying the experimental techniques I learned from my time at JQI, I am currently working towards a Physics Honors Thesis on lifetime measurements of certain quantum states in potassium by counting photons fluorescing from a MOT-trapped cloud of excited ultracold $^{39}$K. In previous years, I have constructed a variety of laboratory apparatus ranging from external cavity diode lasers to laser frequency-stabilization electronics and carried out millimeter-wave precision measurements of energy levels and quantum defects on potassium in Rydberg states. Particularly from summer 2018 to spring 2019, I measured the $\mbox{nd}_{j} \to \mbox{(n+1)d}_{j}$ two-photon transitions for $\mbox{30} \leq \mbox{n} \leq \mbox{35}$ to determine d-state quantum defects and a range of absolute energy levels. At our level of precision, the AC Stark shift caused by the millimeter-wave source is significant; consequently, we were required to extrapolate our data to obtain an unbiased measurement. However, by applying Ramsey's separated oscillatory field method, I eliminated this necessity and gave an alternative measurement scheme with comparable precision. I presented this work at Colby's summer research symposium (CUSRR 2018) and at DAMOP 19 in Milwaukee, Wisconsin jointly with Professor Conover. Our group also presented measurements of p-state fine structure and quantum defects in DAMOP 19 and f-, g-, and h-state quantum defects at DAMOP 20. \\ \vspace{-9pt}




Apart from my primary interests in quantum information and atomic physics, I also explore general relativity and mathematical physics. Having enjoyed my General Relativity course with Professor Robert Bluhm at Colby in fall 2018, I subsequently took three independent studies with him studying classical field theory with a goal to better our understanding of massive gravity. In particular, we reviewed the theory's development and the origins of nonlinear behaviors therein, namely the Vainshtein radius and screening mechanisms. The project resulted in a 150-page exposition of the topic which I wrote and made available on my website. Using the tools acquired from my physics and mathematics training, I followed original papers and review articles to reproduce important results, providing explicit derivations and justifications wherever necessary and filling in various knowledge gaps, from basic quantum field theory to how to use the Mathematica packages \texttt{xPert} and \texttt{xAct} to symbolically study perturbative general relativity, so as to make the document accessible to undergraduates like myself. \\ \vspace{-9pt}

Last year, I also began working towards my Mathematics Honors Thesis under Professor Evan Randles at Colby on iterative convolutions of a class of complex-valued functions. The correspondence between the convolution and the Fourier transform makes convolution powers a central object for generating numerical solutions for partial differential equations. Using Python for computing convolution powers and Mathematica for Fourier transforms, I gave numerical evidence verifying our new local (central) limit conjecture related to this problem. Motivated by this finding, Professor Randles and I have recently constructed, via measure theory, a generalized quasi-polar-coordinate integration formula for estimating the Fourier transform of a class of functions in question. While the main conjecture still stands, this new result is sufficiently important and applicable that Professor Randles and I have decided to prepare a manuscript for publication. We will also present this work at the Joint Mathematics Meeting in January of 2021. \\ \vspace{-9pt}

Having been exposed to a range of fields related to my  interest, I now wish to better define my research path by focusing on a significant open problem in physics. I believe that MIT Physics is a fantastic place for me to find and work on such a problem. I have contacted Professor Soonwon Choi, who will be starting at MIT in July 2021 and whose research on quantum many-body systems, quantum information dynamics, and their applications aligns very well with my current interests. I am also expressed interests in Professor Martin Zwierlein's research on ultracold atomic gases, particularly his upcoming experiment studying rotating quantum gases to further investigate the quantum Hall effect. Finally, I also find both the theory and experimental aspects of Professor Isaac Chuang's Quanta group very appealing.  \\ \vspace{-9pt}

\noindent Thank you for your time and consideration.\\ \vspace{-9pt}

\noindent Huan Q. Bui\\
\today


%\begin{spacing}{0}
%	\bibliographystyle{ieeetr}
%	\bibliography{ref} 
%\end{spacing}

	

















	
	
	
	
	
\end{document}