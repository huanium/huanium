\documentclass[12pt]{article}
\usepackage[left=1in,right=1in,top=1in,bottom=1in]{geometry}

\usepackage{newpxtext,newpxmath}
\usepackage{color}


% \pagenumbering{gobble}



\makeatletter
\makeatother





\begin{document}
	
	
\begin{center}
	\textbf{Statement of Purpose (MIT)}
\end{center}	
I am currently in my fourth year at Colby College studying physics, mathematics, and statistics. I am drawn to research problems in quantum information science and atomic, optical, and molecular physics, for they involve a strong interplay between theories and experiments. My research experience at Colby, at the Joint Quantum Institute, and most recently at the Perimeter Institute for Theoretical Physics have collectively shaped my interest in developing and applying quantum algorithms in simulating complex quantum systems. As I seek to continue pursuing research in this area in graduate school and beyond, MIT is a natural choice for me. \textcolor{red}{Why? -- and rewrite this opening paragraph, saying sth like... ``Having had research experience in a range of things, now I feel ready to focus one big problem...''} \textcolor{blue}{Remember to mention motivations, goals, and expectations}\\ \vspace{-7pt}

Under the supervision of Dr. Timothy Hsieh at Perimeter Institute since May 2020, I have been researching simulation of non-trivial quantum states using the variational quantum approximate optimization algorithm (QAOA). My project explores the possibility of using the cluster-state ansatz and projective measurements in conjunction with the variational QAOA algorithm to create a protocol whose circuit depth scales sublinearly as a function of system size $(L)$. Such an algorithm would extend Dr. Hsieh's previous finding that QAOA could simulate a large class of non-trivial quantum states with a linearly-scaling ($\mathcal{O}(L)$) circuit. While working on this, I found numerically that a linear-depth QAOA ansatz can perfectly simulate any ground state of the disordered one-dimensional quantum Ising model. As a result, although the sublinear circuit remains an open question, this new finding pushes our research in a slightly different trajectory towards exploring extensions and further applications of the current linear-scaling protocol. In particular, we are now interested in studying its ability to let us prepare highly excited states and understand many-body localization. Following the works of \textcolor{red}{cite Japanese paper} and \textcolor{blue}{cite VQE paper}, I showed that an $\mathcal{O}(L+1)$-circuit could perfectly target a large class of a excited states of the one-dimensional non-uniform quantum Ising model. This opens up possibilities to \textcolor{blue}{Say something here to close this paragraph off.}\\ \vspace{-7pt}

Prior to my internship at Perimeter, I have spent more than two years working on various experimental projects on ultracold atoms at Colby under Professor Charles Conover and at the Joint Quantum Institute under Professor Steven Rolston. \\ \vspace{-7pt}

Apart from my primary interests in QI and AMO physics, I also dedicate significant efforts towards exploring general relativity and mathematical physics. Having thoroughly enjoyed General Relativity with Professor Robert Bluhm at Colby in Fall 2018, I took on three consecutive independent studies in the following semesters exploring classical field theory in more detail, with a focus on helping Professor Bluhm and myself understanding massive gravity more explicitly. In particular, we were interested in understanding the origins of the nonlinear behaviors of the theory, namely the Vainshtein radius and various screening mechanisms that arise as the theory is modified through its existence. After my junior year, this independent study project resulted in a 150-page detailed exposition of the topic which I wrote and made available on my website, not only for Professor Bluhm and myself but also for whoever is interested. Utilizing the tools I have picked up from my exposure to general relativity/classical field theory and my mathematics training, I followed original papers and Kurt Hinterbichler's review paper \cite{RevModPhys.84.671}, providing explicit derivations of results and justifications wherever necessary and filling in knowledge gaps such that the document becomes more accessible to undergraduates like myself. These knowledge gaps range from basic quantum field theory (from Klein-Gordon to $\phi^4$ theory) to how to use the Mathematica packages \texttt{xPert} and \texttt{xACT} to study perturbative general relativity symbolically. \\ \vspace{-7pt}

In my junior year, I also began working towards my Mathematics Honors Thesis under Professor Evan Randles at Colby on the study of iterative convolutions of finitely-supported complex-valued functions on $\mathbb{Z}^d$. Due to the fact that the Fourier transform of convolution powers are powers of Fourier transforms, iterative convolutions are one of the central objects for generating numerical solutions for partial differential equations. Using Python for computing convolution powers and Mathematica for Fourier transforms, I first demonstrated numerical evidence verifying our conjecture for a new local (central) limit theorem related to this problem. This finding allowed Professor Randles and I ground on which to construct new theories to address the main conjecture. So far, this summer we have established, via measure theory, a generalized quasi-polar-coordinate integration formula for estimating the Fourier transform of a class of functions in question and for applications to operator semigroups. While the larger question still stands, this result is sufficiently importantly and application in its own right that Professor Randles and I are currently preparing a manuscript for its publication. We will also be presenting this work at the Joint Mathematics Meeting in January of 2021. 











 


	
	
	
	
	
	
	






\bibliography{ref} 
\bibliographystyle{ieeetr}
















	
	
	
	
	
\end{document}