\documentclass[12pt]{article}
\usepackage[left=1in, right=1in,top=1in,bottom=1in]{geometry}

%\usepackage{newpxtext,newpxmath}
\usepackage{amsfonts}
\usepackage{tgtermes}
\usepackage{color}
\usepackage{bold-extra}
\usepackage{setspace}
\pagenumbering{gobble}






\begin{document}
	
	
\begin{center}
	\textbf{Statement of Objectives (MIT)}
\end{center}\vspace{-5pt}	
I am drawn to problems in quantum information and atomic physics for the interplay between theory and experiment. My research training at Colby, the Joint Quantum Institute (JQI), and the Perimeter Institute has shaped my interest in table-top experiments as well as theoretical quantum simulations. I will soon complete my undergraduate studies with two Honors Theses on experimental atomic physics and mathematical physics. The next step for me is to define a research path and apply my training to address an open problem in quantum science. To that end, MIT Physics is a fantastic option for the range of novel problems being tackled here and the tight collaboration between theorists and experimentalists not just at MIT but also at the Center for Ultracold Atoms.    \\ \vspace{-10pt}

Under the supervision of Dr. Timothy Hsieh at Perimeter Institute, I have been researching efficient variational simulation of non-trivial quantum states using the quantum approximate optimization algorithm (QAOA) ansatz in a quantum-classical hybrid setting. Recently, Dr. Hsieh and colleagues have developed such a protocol -- one that could target with perfect fidelity a class of non-trivial states using an L-deep circuit where L is the system size. My project initially explored the possibility of improving Dr. Hsieh's protocol to achieve a sublinear-depth circuit by incorporating aspects of measurement-based quantum computation schemes such as the cluster-state formalism and projective measurements. Surprisingly, while benchmarking the original algorithm, I discovered numerically that it is capable of perfectly simulating the ground state of not just any uniform TFIM as found by Dr. Hsieh, but also any TFIM with random transverse field and couplings. This finding steered my research towards understanding how the QAOA ansatz consistently gives perfect fidelity. To this end, I will most study Dr. Hsieh's conjecture regarding the correspondence between the wavefunction picture and free-fermion picture of the problem under the Jordan-Wigner transformation. Regarding extensions and applications of the current protocol, I recently showed that an (L+1)-deep QAOA ansatz could perfectly target a large class of excited states for any TFIM with random fields and couplings. From here, I will be targeting the class of disordered TFIM's that cannot be put into a free-fermion form. Ultimately, we are hoping to establish some relationships between this protocol and understanding many-body localization. While this is an ambitious goal, I am excited to see what might will as the project progresses. \\ \vspace{-10pt}

In summer 2019, I joined the Rolston group at JQI to work on  experiments on Rb atoms trapped around an optical nanofiber to study their long-range interactions. With Dr. Hyok Sang Han's guidance and my knowledge of optics, I created an imaging system for optimizing light polarization in the nanofiber. The nanofiber medium often introduces birefringence and a non-uniform longitudinal polarization; thus, this system was necessary to guarantee at least  quasi-linearly polarized light. Additionally, I developed a Python program using the NI-DAQmx libraries for controlling the entire Rb experiment, removing the group's reliance on the less compatible LabView. In January 2020, while working on the experiment, Dr. Han and I observed a transient decay flash that was much faster than even the fastest superradiance mode of the system. Since this phenomenon was only recently discovered for a 3-dimensional atom cloud and is not yet understood particularly in the 1-dimensional geometry of our nanofiber experiment, the team at JQI is constructing a model to describe and explain this behavior. This promises new insights into the phenomenon.  \\ \vspace{-10pt}

At Colby, I work on ultracold atom experiments under Professor Charles Conover. Applying the experimental techniques I learned at JQI, I am currently working towards a Physics Honors Thesis on lifetime measurements of certain quantum states in potassium by counting photons fluorescing from a MOT-trapped cloud of excited ultracold K atoms. In previous years, I have constructed a variety of laboratory apparatus from external cavity diode lasers to laser frequency-stabilization electronics and carried out millimeter-wave precision measurements of energy levels and quantum defects on K in Rydberg states. From summer 2018 to spring 2019, I measured the $\mbox{nd}_{j} \to \mbox{(n+1)d}_{j}$ two-photon transitions for $\mbox{30} \leq \mbox{n} \leq \mbox{35}$ to determine d-state quantum defects and a range of absolute energy levels of K. At our level of precision, AC Stark shifts due to the mm-wave source is significant; this requires us to extrapolate our data to obtain an unbiased measurement. However, by applying Ramsey's separated oscillatory field method, I eliminated this necessity and gave an alternative measurement scheme with comparable precision. I presented this work at Colby's summer research symposium CUSRR 2018 and at DAMOP 19 in Milwaukee, Wisconsin jointly with Professor Conover. Our group also presented measurements of the p-state fine structure and quantum defects in DAMOP 19 and f-, g-, and h-state quantum defects at DAMOP 20. \\ \vspace{-10pt}

Besides my primary interests, I also explore general relativity and mathematical physics. Having enjoyed General Relativity with Professor Robert Bluhm at Colby in fall 2018, I subsequently took three independent studies with him on classical field theory with a goal to better our understanding of massive gravity where I reviewed the theory's development and the origins of nonlinear behaviors such as the Vainshtein radius and screening mechanisms. The project resulted in my 150-page exposition of this topic which I made available on my website. Following original works and review articles, I gave explicit derivations of important results and filled in knowledge gaps to make the document accessible to undergraduates like myself. These gaps range from elementary quantum field theory to how to use Mathematica \texttt{xPert} and \texttt{xAct} packages for perturbative general relativity. \\ \vspace{-10pt}

Last year, I began my Mathematics Honors Thesis under Professor Evan Randles at Colby on convolution powers of certain complex-valued functions. The correspondence between the convolution and the Fourier transform makes convolution powers a central object for generating solutions for partial differential equations. Using Python for computing convolution powers and Mathematica for Fourier transforms, I gave numerical evidence indicative of a new local limit conjecture related to this problem. Motivated by this, we recently made progress by constructing via measure theory a generalized quasi-polar-coordinate integration formula for estimating the Fourier transform of a class of functions. While the main conjecture still stands, Professor Randles and I are preparing a manuscript and will present this work at the Joint Mathematics Meeting in January 2021. \\ \vspace{-10pt}

I wish to continue my scientific career by strengthening and bringing forward my research knowledge and experience as a teaching assistant for a range of courses from Linear Algebra to Quantum Mechanics. MIT Physics' divisions of quantum information and atomic physics offer unmatched research and learning opportunities: I have contacted Professor Soonwon Choi, who will be starting at MIT in July 2021 and whose research on quantum many-body systems, quantum information dynamics, and applications aligns extremely well with my current interests. I also contact Professor Martin Zwierlein and expressed interest in his research on ultracold atomic gases, particularly the upcoming experiment on rotating quantum gases to investigate quantum Hall physics. Finally, I am attracted to both the theoretical and experimental aspects of Professor Isaac Chuang's research, especially his work on the quantum singular value transformation. I am excited for what lies ahead on my scientific journey, and I truly believe that MIT Physics is the ideal next step forward.   \\ 



\noindent Huan Q. Bui\\
\today


%\begin{spacing}{0}
%	\bibliographystyle{ieeetr}
%	\bibliography{ref} 
%\end{spacing}

	

















	
	
	
	
	
\end{document}