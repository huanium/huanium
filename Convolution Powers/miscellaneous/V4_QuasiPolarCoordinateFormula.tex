\documentclass[11pt]{article}
\usepackage[total={7in, 8in}]{geometry}
\usepackage{graphicx}
\usepackage{amsmath, amsthm, latexsym, amssymb, color,cite,enumerate, physics, framed}
\usepackage{caption,subcaption,verbatim}
\usepackage[inline]{enumitem}
\usepackage{mathtools}
\pagenumbering{arabic}
%\theoremstyle{theorem}
\newtheorem{theorem}{Theorem}[section]
\newtheorem{lemma}[theorem]{Lemma}
\newtheorem{definition}[theorem]{Definition}
\newtheorem{corollary}[theorem]{Corollary}
\newtheorem{proposition}[theorem]{Proposition}
\newtheorem{convention}[theorem]{Convention}
\newtheorem{conjecture}[theorem]{Conjecture}
%\theoremstyle{remark}
\newtheorem{remark}{Remark}
\newtheorem{example}{Example}
\newcommand*{\myproofname}{Proof}
\newenvironment{subproof}[1][\myproofname]{\begin{proof}[#1]\renewcommand*{\qedsymbol}{$\mathbin{/\mkern-6mu/}$}}{\end{proof}}
\renewcommand\Re{\operatorname{Re}}%%redefined Re and Im
\renewcommand\Im{\operatorname{Im}}
\newcommand\MdR{\mbox{M}_d(\mathbb{R})} %dxd matrices
\newcommand\End{\operatorname{End}} %Prefix linear transformations
\newcommand\GldR{\mbox{Gl}_d(\mathbb{R})}%dxd invertible matrices
\newcommand\Gl{\operatorname{Gl}}                     %Prefix for general linear group
\newcommand\OdR{\mbox{O}_d(\mathbb{R})} %Orthogonal matrices
\newcommand\Sym{\operatorname{Sym}}
\newcommand\Exp{\operatorname{Exp}}
%\newcommand\tr{\operatorname{tr}}
\newcommand\diag{\operatorname{diag}}
\newcommand\supp{\operatorname{Supp}}
\newcommand\Spec{\operatorname{Spec}}
\renewcommand\det{\operatorname{det}}
\newcommand\Ker{\operatorname{Ker}}
\newcommand\R{\mathbb{R}}
\newcommand{\lp}{\left(}
\newcommand{\rp}{\right)}
\newcommand{\lb}{\left[}
\newcommand{\rb}{\right]}
\newcommand{\lc}{\left\{}
\newcommand{\rc}{\right\}}
\newcommand{\p}{\partial}
\newcommand{\f}[2]{\frac{#1}{#2}}
\newcommand{\Vol}{\operatorname{Vol}}
\newcommand{\iprod}{\mathbin{\lrcorner}}
\newcommand{\al}{\alpha}
\newcommand{\be}{\beta}

\author{Huan Bui and Evan Randles}
\title{A generalized polar coordinate integration formula}
\date{}
\begin{document}
\maketitle

\abstract{This paper is about math.}

\section{Introduction}

Stuff about the standard polar coordinate formula.

In this \textcolor{red}{section}, we establish a generalized polar coordinate integration formula associated to a positive-homogeneous polynomial. In what follows, we fix a positive-homogeneous polynomial $P:\mathbb{R}^d\to [0,\infty)$ and set
\begin{equation*}
S=\{\eta\in\mathbb{R}^d:P(\eta)=1\}\hspace{1cm}\mbox{and}\hspace{1cm}B=\{\eta\in\mathbb{R}^d:P(\eta)<1\}. 
\end{equation*}
 of $\mathbb{R}^d$, $S$ is a compact hypersurface, i.e., a compact smooth manifold of dimension $d-1$, and $B$ is a bounded open region. \textcolor{red}{I'm not sure if this matters, but $S$ is connected never contains $0$ and is not necessarily convex, $B$ does contains $0$ and, I believe, is connected and simply connected. Also, we should think of $P$ being the essential thing which defines both $E$ and $S$ here -- and we can write $\mu_P=\tr E$ throughout. }

\textcolor{red}{This section needs to contain a description of the standard polar coordinate integration formula and, therein, cite Folland or St.-Sh. For reference, our description should make use of the formula in the following theorem:
\begin{theorem}
Let $\Theta=\Theta^{d-1}$ be the canonical surface measure on the unit sphere $\mathbb{S}^{d-1}\subseteq \mathbb{R}^d$, i.e., $\Theta$ is the unique rotation-invariant Radon measure on $\mathbb{S}^{d-1}$ for which $\Theta(\mathbb{S}^{d-1})=d\cdot m(\mathbb{B})=(d\cdot\pi^{d/2})/\Gamma(d/2+1)$;  here $\mathbb{B}$ denotes the unit ball in $\mathbb{R}^d$, $m$ is the Lebesgue measure on $\mathbb{R}^d$ and we will write $dm(x)=dx$. Let $f\in L^1(\mathbb{R}^d)$. Then, for $\Theta$-almost every $\eta$, $t\mapsto f(t\eta)$ is (absolutely) integrable with respect to the measure $t^{d-1}dt$ on $(0,\infty)$, $\eta\mapsto \int_0^\infty f(t\eta)t^{d-1}\,dt$ is (absolutely) integrable with respect to $\Theta$ on $\mathbb{S}^{d-1}$ and
\begin{equation}\label{eq:StandardPolarIntegrationFormula}
\int_{\mathbb{R}^d}f(x)\,dx=\int_{\mathbb{S}^{d-1}}\left(\int_0^\infty f(t\eta)t^{d-1}\,dt\right)\,\Theta(d\eta).
\end{equation}
\end{theorem}}

\section{Positive-Homogeneous Functions}

\textcolor{red}{need to define $\mathcal{M}_d$. It's first use seems to be in the statement of Theorem \ref{thm:BestIntegrationFormula}. It's first use beyond the statement appears in checking that $\Sigma_{S,E}$ is a $\sigma$-algebra.}

Our setting is $d$-dimensional Euclidean space $\mathbb{R}^d$ equipped with inner product $x\cdot y=\sum_{k=1}^dx_ky_k$ and associated Euclidean norm $|x|=\sqrt{x_1^2+x_2^2+\cdots+x_d^2}$. We will take $\mathbb{R}^d$ to be equipped with its usual topology and (oriented) smooth structure. Given $x\in\mathbb{R}^d$ and $R>0$, the open ball with center $x$ and radius $R$ is denoted by $\mathbb{B}_R(x)$ and its corresponding sphere is denoted by $\mathbb{S}_R(x)$. We write $\mathbb{B}=\mathbb{B}_1(0)$ and $\mathbb{S}=\mathbb{S}_1(0)$ to denote the standard unit ball and unit sphere, respectively, in $\mathbb{R}^d$. We shall denote by $\End(\mathbb{R}^d)$ the algebra (\textcolor{red}{Or ring}) of linear transformations on $\mathbb{R}^d$ and by $\MdR$ the corresponding set of $d\times d$ real matrices. Also, we shall denote by $\Gl(\mathbb{R}^d)$ the general linear group and by $\GldR$ the corresponding group of $d\times d$ invertible real matrices. For $E\in \End(\mathbb{R}^d)$ (or $E\in\MdR$), $\|E\|$ will denote the so-called operator norm.\\

\noindent Let $\{T_t\}_{t>0}\subseteq \Gl(\mathbb{R}^d)$ be a continuous one-parameter group, i.e., $T$ is a continuous group homomorphism from the multiplicative group of positive real numbers into the general linear group, $\Gl(\mathbb{R}^d)$. It is well-known (c.f., \cite{Randles2017,Engel2000,Engel2005}) that every continuous one-parameter group $\{T_t\}$ has the unique representation
\begin{equation*}
T_t=t^E=\exp((\ln t) E)=\sum_{k=0}^\infty \frac{(\ln t)^k}{k!}E^k
\end{equation*}
for some $E\in\End(\mathbb{R}^d)$. $E$ is called (infinitesimal) generator of $\{T_t\}$ and $\{T_t\}$ is said to be generated by $E$. This, of course, gives a one-to-one correspondence between $\End(\mathbb{R}^d)$ and the collection of continuous one-parameter groups. Some basic results on one-parameter contracting groups are presented in Appendix \ref{subsec:OneParameterGroups}. Also, we encourage the reader to look at the excellent texts \cite{Engel2005} and \cite{Engel2000} on one-parameter (semi) groups.

\begin{definition} A continuous one-parameter group $\{T_t\}$ is said to be \textit{contracting} if
\begin{equation*}
\lim_{t\to 0}\|T_t\|=0. 
\end{equation*}
\end{definition}

\noindent The notion of a contracting group (or semigroup) is closely related to notion of (Lyapunov) stability in linear (and nonlinear) dynamical systems (See \textcolor{red}{We need a reference here, preferably one that uses "Lyapunov". Braun's ODE book is a good reference, but doesn't use the name Lyapunov. We should look at R. Bellman's classic (dover) book.}) The following necessary condition will be useful for us when constructing measures on surfaces in $\mathbb{R}^d$.

\begin{proposition}\label{prop:ContractingTrace}
Let $\{T_t\}\subset\Gl(\mathbb{R}^d)$ be a continuous one-parameter group with generator $E$. If $\{T_t\}$ is  contracting, then $\tr E>0$. 
\end{proposition}
\begin{proof}
The supposition that $\{T_t\}$ is a contracting group implies that $t^E\to \mathbf{0}$ as $t\to 0$ in the operator-norm topology on $\End(\mathbb{R}^d)$; here $\mathbf{0}$ is zero transformation. Because the determinant is a continuous function from $\End(\mathbb{R}^d)$, equipped with operator-norm topology, into $\mathbb{R}$, we have
\begin{equation*}
0=\det(\mathbf{0})=\det\left(\lim_{t\to 0}t^E\right)=\lim_{t\to 0}\det\left(t^E\right)=\lim_{t\to 0}t^{\tr E}
\end{equation*}
in view of the preceding proposition. Therefore, $\tr E>0$.
\end{proof}

\noindent As we discussed in the introduction, one main goal of this article is to construct measures on certain surfaces in $\mathbb{R}^d$ (analogous to $\mathbb{S}$) with which we can formulate generalizations of \eqref{eq:StandardPolarIntegrationFormula}. As the unit sphere $\mathbb{S}$ can be defined as a level set of the Euclidean norm $|\cdot|$ on $\mathbb{R}^d$, a function which is continuous,  positive-definite\footnote{Given a function $P:\mathbb{R}^d\to\mathbb{R}$, we say that $P$ is positive-definite if $P\geq 0$ and $P(x)=0$ only when $x=0$.}, and homogeneous in the sense that $t|x|=|tx|$ for all $t>0$ and $x\in\mathbb{R}^d$, the surfaces of interest for us are level sets of certain continuous functions which we will call \textit{positive-homogeneous}. The remainder of this section is dedicated to introducing this class of functions.\\

\noindent In what follows, let $P:\mathbb{R}^d\to\mathbb{R}$ be continuous and define
\begin{equation}
S=\{\eta\in\mathbb{R}^d:P(\eta)=1\}
\end{equation}
and
\begin{equation*}
B=\{\eta\in\mathbb{R}^d:P(\eta)<1\}.
\end{equation*}
Given a continuous one-parameter group $\{T_t\}$ with infinitesimal generator $E$, we say that \textit{$P$ is homogeneous with respect to $\{T_t\}$ (and, by an abuse of language, homogeneous with respect to $E$)} if
\begin{equation}\label{eq:Homogeneous}
tP(x)=P(T_tx)=P\left(t^Ex\right)
\end{equation}
for all $x\in\mathbb{R}^d$ and $t>0$. For a given map $P:\mathbb{R}^d\to\mathbb{R}$, the set of all $E$ for which \eqref{eq:Homogeneous} is satisfied is called the \textit{exponent set of $P$} and shall be denoted by $\Exp(P)$. 


\begin{proposition}\label{prop:PositiveHomogeneousCharacterization}
Let $P:\mathbb{R}^d\to\mathbb{R}$ be continuous, positive-definite (I.e., $P(x)\geq 0$ for all $x\in\mathbb{R}^d$ and $P(x)=0$ if and only if $x=0$), and have $\Exp(P)\neq \varnothing$. The following are equivalent:
\begin{enumerate}[label=(\alph*), ref=(\alph*)]
\item\label{cond:SisCompact} $S$ is compact.
\item\label{cond:PisAboveOne} There is a positive number $M$ for which
\begin{equation*}
P(x)>1
\end{equation*}
for all $|x|\geq M$. 
\item\label{cond:Contracting} For each $E\in\Exp(P)$, $T_t=t^E$ is contracting.
\item\label{cond:ThereExistsContracting} There exists $E\in\Exp(P)$ for which $T_t=t^E$ is contracting.
\item\label{cond:InfiniteLimit} We have
\begin{equation*}
\lim_{x\to\infty}P(x)=\infty.
\end{equation*}
\end{enumerate}
\end{proposition}
\begin{proof}
In the case that $d=1$, is is easy to see that every function satisfying the hypotheses is of the form
\begin{equation*}
P(x)=\begin{cases}
P(1)x^\alpha \\
P(-1)(-x)^\alpha
\end{cases}
\end{equation*}
for some $\alpha>0$ where $P(1),P(-1)>0$ and $\Exp(P)$ consists only of the linear function $x\mapsto x/\alpha$. In this setting, it is easy to see that Conditions \ref{cond:SisCompact}--\ref{cond:InfiniteLimit} are satisfied (always and) simultaneously. We shall therefore assume that $d>1$ for the remainder of the proof.

\begin{subproof}[$\ref{cond:SisCompact}\implies\ref{cond:PisAboveOne}$]
Given that $S$ is compact, it is bounded and so we have a positive number $M$ for which $P(x)\neq 1$ for all $|x|\geq M$. Observe that, if for two points $x_1,x_2\in \mathbb{R}^d\setminus\mathbb{B}_M(0)$, $P(x_1)<1<P(x_2)$ or $P(x_2)<1<P(x_1)$, then by virtue of the path connectedness of $\mathbb{R}^d\setminus\mathbb{B}_M(0)$ and the intermediate value theorem, we would be able to find  $x_0\in\mathbb{R}^d\setminus\mathbb{B}_M(0)$ for which $P(x_0)=1$, an impossibility. Therefore, to show that Condition \ref{cond:PisAboveOne} holds, we must simply rule out the case in which $P(x)<1$ for all $|x|\geq M$. Let us therefore assume, to reach a contradiction, that this alternate condition holds. In this case, we take $E\in\Exp(P)$ and $y\in\mathbb{R}^d\setminus \{0\}$ and observe that
\begin{equation*}
\lim_{t\to\infty}P(t^Ey)=\lim_{t\to\infty}tP(y)=\infty.
\end{equation*}
By virtue of our supposition, we find that $|t^Ey|<M\leq M$ for all sufficiently large $t$. In particular, there exists a sequence $t_k\to\infty$ for which $|t_k^Ey|\leq M$ for all $k$ and 
\begin{equation*}
\lim_{k\to\infty}P(t_k^Ey)=\infty.
\end{equation*}
Because $\overline{\mathbb{B}_M(0)}$, the closure of $\mathbb{B}_M(0)$, is compact, $\{t_k^Ey\}$ has a convergent subsequence which we also denote by $\{t_k^Ey\}$ by a slight abuse of notation. In view of the continuity of $P$ at $\eta=\lim_{k\to\infty}t_k^Ey$, we have
\begin{equation*}
P(\eta)=\lim_{k\to\infty}P(t_k^Ey)=\lim_{k\to\infty}t_kP(y)=\infty,
\end{equation*}
which is impossible. Thus Condition \ref{cond:PisAboveOne} holds.
\end{subproof}
\begin{subproof}[$\ref{cond:PisAboveOne}\implies\ref{cond:Contracting}$]
We shall prove the contrapositive statement. Suppose that, for $E\in\Exp(P)$, $\{t^E\}$ is not contracting. In this case, by virtue of Proposition \ref{prop:ContractingCharacterization}, there exists $x\in\mathbb{R}^d\setminus\{0\}$ and a sequence $t_k\to 0$ for which $t_k^Ex$ does not converge to zero in $\mathbb{R}^d$. If our sequence $\{t_k^Ex\}$ is bounded, then it must have a convergent subsequence $\{t_{k_m}^Ex\}$ with non-zero subsequential limit $\eta=\lim_{m\to\infty}t_{k_m}^Ex$. By the continuity of $P$, we have
\begin{equation*}
P(\eta)=\lim_{m\to\infty}P(t_{k_m}^Ex)=\lim_{k\to\infty}t_{k_m}P(x)=0
\end{equation*}
which cannot be true for it would violate the positive-definiteness of $P$. We must therefore consider the other possibility: The sequence $\{t_k^Ex\}$ is unbounded. In particular, there must be some $k_0$ for which $t_{k_0}<1/P(x)$ and $|t_{k_0}^Ex|>M$. Upon putting $y=t_{k_0}^Ex$, we have $|y|>M$ and  $P(y)=P(t_{k_0}^Ex)=t_kP(x)<1$ which shows that Condition \ref{cond:PisAboveOne} cannot hold.
\end{subproof}
\begin{subproof}[$\ref{cond:Contracting}\implies\ref{cond:ThereExistsContracting}$] This is immediate.
\end{subproof}
\begin{subproof}[$\ref{cond:ThereExistsContracting}\implies\ref{cond:InfiniteLimit}$]
Let $E\in\Exp(P)$ be such that $\{t^E\}$ is contracting and let $\{x_n\}\subseteq\mathbb{R}^d$ be such that $\lim_{k\to\infty}|x_n|=\infty$. By virtue of Proposition \ref{prop:ScaleFromSphere}, there exist sequences $\{t_k\}\subseteq (0,\infty)$ and $\{\eta_k\}\in\mathbb{S}$ for which $t_k^E\eta_k=x_k$ for all $k$ and $\lim_{k\to\infty}t_k=\infty$. Given that $P$ is continuous and strictly positive on the compact set $\mathbb{S}$, we have $\inf_{\eta\in\mathbb{S}}P(\eta)>0$ and therefore
\begin{equation*}
\liminf_k P(x_k)=\liminf_k t_kP(\eta_k)\geq \liminf_k t_k\left(\inf_{\eta\in\mathbb{S}}P(\eta)\right)=\infty
\end{equation*}
showing that $\lim_{k\to\infty} P(x_k)=\infty$, as desired.
\end{subproof}
\begin{subproof}[$\ref{cond:InfiniteLimit}\implies\ref{cond:SisCompact}$]
Because $S$ is the preimage of the closed singleton $\{1\}$ under the continuous function $P$, it is closed. By virtue of Condition \ref{cond:InfiniteLimit}, $S$ is also be bounded and thus compact in view of the Heine-Borel theorem.
\end{subproof}
\end{proof}

\textcolor{red}{\hline}
\begin{proposition}
If $P$ is continuous, positive definite and $\Exp(P)$ contains a diagonalizable $E\in\End(\mathbb{R}^d)$, then $\{t^E\}$ is contracting and hence all of the above conditions are (simultaneously) met. 
\end{proposition}
\begin{proof}
Let's assume, to reach a contradiction, that $\{t^E\}$ is not contracting. By repeating the same argument given in ``$\ref{cond:PisAboveOne}\implies\ref{cond:Contracting}$", we are left with only one possibility: There is a non-zero $x\in\mathbb{R}^d$ and a sequence $t_k\to 0$ for which $|t_k^Ex|\to\infty$. Given that $E$ is diagonalizable, let $v_1,v_2,\dots,v_d$ be a basis of eigenvectors of $E$ and let $\lambda_1,\lambda_2,\dots,\lambda_d$ be their corresponding eigenvalues. Then
\begin{equation*}
x=\alpha_1v_1+\alpha_2v_2+\cdots+\alpha_dv_d
\end{equation*}
and
\begin{equation*}
t_k^Ex=\alpha_1t_k^{\lambda_1}v_1+\alpha_2t_k^{\lambda_2}v_2+\cdots+\alpha_dt_k^{\lambda_d}v_d
\end{equation*}
for all $k$. Given that $|t_k^Ex|\to\infty$ and $t_k\to 0$, we must have at least one negative eigenvalue. We therefore assume, without loss of generality, that $\lambda_1<0$ observe that
\begin{equation*}
\infty=\lim_{t\to \infty}tP(v_1)=\lim_{t\to\infty}P(t^E v_1)=\lim_{t\to \infty}P(t^{\lambda_1} v_1)=P(0)
\end{equation*}
which is impossible.
\end{proof}
\textcolor{red}{\begin{remark}I'm still not convinced that the conclusion of the above proposition is not always true even when $E$ is not diagonalizable. It could be true and it could be not true. We really need to either prove it or find a counterexample.
\end{remark}
\begin{remark}
In fact, I believe that you can even extend the proposition to say that, if $E$ has real spectrum then it still holds. In this case, the Jordan decomposition (diagonalizable + nilpotent) should still make the same argument go through. So, to seek a counterexample, we should probably need to look in, at least $\mathbb{R}^3$. The reason for this is that, to make things break, I believe we need at least one complex eigenvalue (which means two of them) and two eigenvalues whose real parts have opposite signs.
\end{remark}
\begin{remark}
Okay, I've found a reference that might help us generalize the Prop 2.4 here. See Theorem 1 on page 378 of Martin Braun's 4th edition of "Differential Equations and their applications". 
\end{remark}}

\begin{remark}
If we are able to show that these conditions are always satisfied, then we must change the definition below. 
\end{remark}
\textcolor{red}{\hline}

\begin{definition}
Let $P:\mathbb{R}^d\to\mathbb{R}$ be continuous, positive-definite and have $\Exp(P)\neq \varnothing$. If any one (and hence all) of the equivalent conditions in Proposition \ref{prop:PositiveHomogeneousCharacterization} are fulfilled, we say that $P$ is positive-homogeneous.
\end{definition}
\begin{example}[Examples]
\textcolor{red}{We need to build this out. Maybe multiple example environments}
\begin{enumerate}
\item The Euclidean norm $|\cdot|$ is positive-homogeneous. $\Exp(|\cdot|)=I+\mathfrak{o}(d)$ where $I$ is the identity and $\mathfrak{o}(d)$ is the Lie algebra of the orthogonal group (characterized by the set of skew-symmetric matrices).
\item For any $\alpha>0$, $x\mapsto |\cdot|^\alpha$ is positive-homogeneous. $Exp(|\cdot|^{\alpha})=\frac{1}{\alpha}I+\mathfrak{o}(d).$
\item Consider $P:\mathbb{R}^d\to \mathbb{R}$ defined by
\begin{equation*}
P(x,y)=x^2+y^4
\end{equation*}
for $(x,y)\in\mathbb{R}^2$. In this case $\Exp(P)$ is a singleton consisting of $E\in\Gl(\mathbb{R}^d)$ having standard representation $\diag(1/2,1/4)$. \textcolor{blue}{Can we prove this?}.
\item Let's do one here whose $S$ is non-convex.
\item In \cite{Randles2017}, a positive-homogeneous polynomial $P$ is, by definition, a positive-definite (multivariate) polynomial for which $\Exp(P)$ contains an element $E$ of $\End(\mathbb{R}^d)$ whose spectrum is purely real (equivalently, whose characteristic polynomial factors over $\mathbb{R}$). By virtue of Proposition 2.2 of \cite{Randles2017}, for each such polynomial $P$ and $E\in\Exp(P)$ with real spectrum, there exists $A\in\Gl(\mathbb{R}^d)$ (representing a change of basis of $\mathbb{R}^d$) and a $d$-tuple of positive integers $\mathbf{m}=(m_1,m_2,\dots,m_d)\in\mathbb{N}_+^d$ for which $A^{-1}EA$ has standard matrix representation
\begin{equation}\label{eq:DiagonalizableE}
\diag\{(2m_1)^{-1},(2m_2)^{-1},\dots,(2m_d)^{-1}\}
\end{equation}
and
\begin{equation}\label{eq:SemiElliptic}
(P\circ A)(x)=\sum_{|\beta:\mathbf{m}|=2}a_\beta x^\beta
\end{equation}
for $x=(x_1,x_2,\dots,x_d)\in\mathbb{R}^d$; here, for a multi-index $\beta=(\beta_1,\beta_2,\dots,\beta_d)\in\mathbb{N}^d$, $x^{\beta}=x_1^{\beta_1}x_2^{\beta_2}\cdots x_d^{\beta_d}$ and
\begin{equation*}
|\beta:\mathbf{m}|:=\sum_{k=1}^d\frac{\beta_k}{m_k}.
\end{equation*}
Polynomials of the form \eqref{eq:SemiElliptic} are said to be semi-elliptic in the sense of L. H\"{o}rmander and the constant-coefficient partial differential operators they define are hypoelliptic \cite{Hormander1983}. With the representation \eqref{eq:DiagonalizableE}, it is easy to see that $t^{E}$ is contracting and therefore every positive-homogeneous polynomial is a positive-homogeneous function by virtue of Condition \ref{cond:ThereExistsContracting} of Proposition \ref{prop:PositiveHomogeneousCharacterization}. It is straightforward to check that the \textcolor{red}{list all examples above} are positive-homogeneous polynomials of the form \eqref{eq:SemiElliptic}, i.e., semi-elliptic polynomials, with $A=I$. We refer to Section 7.3 of \cite{Randles2017} which presents a positive-homogeneous polynomial which is not semi-elliptic (and so $A\neq I$).

\item Let $Q$ be a positive-homogeneous function with exponent set $\Exp(Q)$ and compact level set $S_Q=\{\eta:Q(\eta)=1\}$. Given any $f\in C^0(S_Q)$ for which $f(\eta)>0$ for all $\eta\in S_Q$ and $E\in \Exp(Q)$, define $P=P_{f,Q}:\mathbb{R}^d\to\mathbb{R}$ by
\begin{equation*}
P(x)=\begin{cases}
Q(x)f\left((Q(x)^{-E}x\right) & x\neq 0\\
0 & x=0
\end{cases}
\end{equation*}
for $x\in\mathbb{R}^d$. We claim that $P$ is positive-definite and $\Exp(Q)\subseteq \Exp(P)$.

\begin{subproof}To see this, we first observe that, for any $x\in\mathbb{R}^d\setminus \{0\}$, $Q((Q(x)^{-E}x)=Q(x)/Q(x)=1$ and hence $Q(x)^{-E}x\in S_Q$ and so the above formula makes sense and ensures that $P$ is continuous on $\mathbb{R}^d\setminus\{0\}$. Futhermore, because $f$ is continuous and positive on the compact set $S_Q$, we have $0<\min Q\leq \max Q<\infty$. From this it follows that $P$ must be positive-definite and also, by virtue of the squeeze theorem, continuous at $x=0$. Now, given $E\in\Exp(Q)$,
\begin{equation*}
P(t^Ex)=Q(t^Ex)f(Q(t^Ex)^{-E}t^Ex)=tQ(x)f(Q(x)^{-E}x)=tP(x)
\end{equation*}
whenever $x\neq 0$ and $t>0$. It follows that $E\in\Exp(P)$ and, because we know that $t^E$ is contracting, we conclude that $P$ is a positive-homogeneous function in view of Proposition \ref{prop:PositiveHomogeneousCharacterization}. 
\end{subproof}
 The utility of this construction allows us to see that ``most'' positive-homogeneous functions are not smooth. To see this, take any positive-definite function $Q\in C^{\infty}(\mathbb{R}^d)$. It follows (\textcolor{red}{See Section \ref{sec:SigmaForSmoothP}}) that $S_Q$ is a smooth compact embedded hypersurface of $\mathbb{R}^d$. If $P=P_{f,Q}$ is $C^\infty(\mathbb{R}^d)$, $P\vert_{S_Q}=f$ is necessarily $C^\infty(S_Q)$. It follows that $P\notin C^\infty(\mathbb{R}^d)$ whenever $f$ is chosen from $C^0(S_Q)\setminus C^\infty(S_Q)$. By precisely the same argument, we see that $P\in C^0(\mathbb{R}^d)\setminus C^k(\mathbb{R}^d)$ whenever $f\in C^0(S_Q)\setminus C^k(S_Q)$ for each $k\in\mathbb{N}$.

We also like to point out, in view of the above construction, that $S_P=\{\eta\in\mathbb{R}^d:P(\eta)=1\}$ does not generally coincide with $S_Q$. In fact, $S_Q=S_P$ if and only if $P=Q$ and (equivalently) $f=1$.

As a straightforward example, we have chosen $Q(x_1,x_2)=|(x_1,x_2)|=\sqrt{x_1^2+x_2^2}$ on $\mathbb{R}^2$ with $S_Q=\mathbb{S}$ and 
\begin{equation*}
f(x_1,x_2)=w(\mbox{Arg}(x_1,x_2))+3
\end{equation*}
where $w:\mathbb{R}\to\mathbb{R}$ is the $2\pi$-periodic Weierstrass function. The resulting positive-homogeneous function $P$ is plotted \textcolor{red}{below}. \textcolor{blue}{Huan, I'm also happy to do a periodic sawtooth or something. We should highlight $S_Q$ and $S_P$ in the graph.}
\end{enumerate}
\end{example}




\noindent Given a positive-homogeneous function $P$, let $\Sym(P)$ be the set of $O\in\End(\mathbb{R}^d)$ for which
\begin{equation*}
P(Ox)=P(x)
\end{equation*}
for all $x\in\mathbb{R}^d$. By virtue of the positive-definiteness of $P$, it is easy to see that $\Sym(P)$ is a subgroup of $\Gl(\mathbb{R}^d)$. For this reason, $\Sym(P)$ is said to be the \textit{symmetric group associated to $P$}. 

\begin{proposition}\label{prop:SymCompact}
For each positive-homogeneous function $P$, $\Sym(P)$ is a compact subgroup of $\Gl(\mathbb{R}^d)$. In particular, it is a subgroup of the orthogonal group $O(\mathbb{R}^d)$.
\end{proposition}
\begin{proof}
By virtue of the Heine-Borel theorem (and the fact that $\Gl(\mathbb{R}^d)$ is finite dimensional), we prove that $\Sym(P)$ is closed and bounded. To this end, let $\{O_n\}\subseteq\Sym(P)$ be a sequence converging to $O\in \Gl(\mathbb{R}^d)$. For each $x\in\mathbb{R}^d$, the continuity of $P$ guarantees that
\begin{equation*}
P(Ox)=P\left(\lim_{n\to\infty}O_nx\right)=\lim_{n\to\infty}P(O_nx)=\lim_{n\to\infty}P(x)=P(x).
\end{equation*}
Hence, $O\in\Sym(P)$ and so $\Sym(P)$ is closed.

We assume, to reach a contradiction, that $\Sym(P)$ is not bounded. In this case, there is a sequence $\{\eta_n\}\subseteq \mathbb{S}$ for which $\lim_{n\to\infty}|O_n\eta_n|=\infty$. Given that $\mathbb{S}$ is compact, by passing to a subsequence if needed, we may assume without loss in generality that $\lim_{n\to\infty}\eta_n=\eta\in\mathbb{S}$. By virtue of Proposition \ref{prop:PositiveHomogeneousCharacterization} and the continuity of $P$,
\begin{equation*}
P(\eta)=\lim_{n\to\infty}P(\eta_n)=\lim_{n\to\infty}P(O_n\eta_n)=\infty
\end{equation*}
which is impossible. Hence $\Sym(P)$ is bounded.
\end{proof}

\begin{corollary}\label{cor:TraceisInvariant}
Let $P$ be a positive-homogeneous function, then
\begin{equation*}
\tr E=\tr E'>0
\end{equation*}
for all $E,E'\in\Exp(P)$.
\end{corollary}
\begin{proof}
By virtue of Propositions \ref{prop:ContractingTrace} and \ref{prop:PositiveHomogeneousCharacterization}, $\tr E>0$ for all $E\in\Exp(P)$. It remains to show that the trace map is constant on $\Exp(P)$. To this end, let $E,E'\in\Exp(P)$. Then, for all $t>0$ and $x\in\mathbb{R}^d$,
\begin{equation*}
P(x)=t(1/t)P(x)=tP((1/t)^{E'}x)=P(t^E(1/t)^{E'}x)=P(t^{E}t^{-E'}x).
\end{equation*}
Thus $O_t=t^{E}t^{-E'}\in\Sym(P)$ for each $t>0$. In view of the Propositions \ref{prop:ContinuousGroupProperties} and \ref{prop:SymCompact} and the homomorphism property of the determinant,
\begin{equation*}
1=\det(O_t)=\det(t^{E}t^{E'})=\det(t^{E})\det(t^{-E'})=t^{\tr E}t^{\tr E}=t^{\tr E-\tr E'}
\end{equation*}
for all $t>0$ and therefore $\tr E=\tr E'$.
\end{proof}

\noindent In view of the preceding corollary, to each positive-homogeneous function $P$, we define the \textit{homogeneous order of $P$} to be the unique positive number $\mu_P$ for which
\begin{equation*}
\mu_P=\tr E
\end{equation*}
for all $E\in\Exp(P)$. 

\begin{example}\textcolor{red}{Needs editing}
It is easy to see that for the $P$'s in the begging examples, we have: 1: $\mu_P=d$, 2: $\mu_P=d/\alpha$, 3: $\mu_P=3/4$, 4: I don't remember, 5: $\mu_P=(2m_1)^{-1}+(2m_2)^{-1}+\cdots+(2m_d)^{-1}$. 6: $\mu_P=\mu_Q$.
\end{example}


We finish this Section by proving a \textcolor{red}{couple of facts} concerning $\Exp(P)$, $\Sym(P)$ and $S$. Given $O\in\Sym(P)$, we write
\begin{equation*}
    OF=\{O\eta:\eta\in F\}.
\end{equation*}

\begin{proposition}\label{prop:ExpP}
For any  $O \in \Sym{(P)} $
\begin{equation*}
    \Exp(P) = O^\top \Exp(P) O.
\end{equation*}
In other words, the set $\Exp(P)$ is invariant under conjugation by $\Sym(P)$.
\end{proposition}

\begin{proof}
Since $\Sym(P)$ is a subgroup of the orthogonal group $O(\mathbb{R}^d)$, $O^\top = O^{-1} \in \Sym{P}$ whenever $O\in\Sym(P)$. For a fixed $O\in\Sym(P)$, it is easy to see that the map $\Exp(P)\ni E\mapsto E\to O^\top E O\in\Exp(P)$ is a bijection and hence $\Exp(P)=O^\top \Exp(P) O$. 
\begin{comment}
Observe that, if $O^\top E_1 O=O^\top E_2 O$, then by left multiplying by $O$ and right multiplying by $O^T=O^{-1}$ we find that $E_1=E_2$ whence $\mbox{Conj}$ is injective.
Now, if $E_i, E_j\in \Exp(P)$ and $E_i\neq E_j$, then $O^\top E_i O \neq O^\top E_j O$ since $O,O^\top\in \OdR{}$. Thus the conjugation map induced by $O$, $\varphi_O: \Exp{P}\to \Exp{P}$ defined by 
\begin{equation*}
    \varphi_O (E) = O^\top E O
\end{equation*}
is one-to-one. Further, for any $E\in \Exp(P)$, a similar argument as \eqref{eq:OOO} shows that $OEO^\top \in \Exp(P)$ and satisfies
\begin{equation*}
    \varphi_O(OEO^\top) = O^\top (OEO^\top) O = E\in \Exp{(P)}.
\end{equation*}
So, $\varphi_O$ is bijective, which means 
\begin{equation*}
    \Exp(P) = O^\top \Exp(P) O = O \Exp{(P)} O^\top, \quad \text{for all } O\in \Sym{(P)}.
\end{equation*}
\end{comment}
\end{proof}

%%%%%%%%%%%%%%%%%%%%%
% \textcolor{red}{I'm putting the ``conjugation by $t^E$'' result at the bottom of the document. Great, I commented out a lot of your original proof (as it, I believe, is fairly standard for group actions). If you have a nice reference, that would be awesome. }
%%%%%%%%%%%%%%%%%%%%%





\section{A surface measure on $S$ and a generalized polar integration formula}\label{sec:IntegrationFormula}

This section is dedicated to constructing a surface measure on the the compact level set $S$ associated to a positive-homogeneous function and using it to generalize the polar integration formula \eqref{eq:StandardPolarIntegrationFormula}. To introduce our main theorem, let $P$ be a positive-homogeneous function with homogeneous order $\mu_P>0$, exponent set $\Exp(P)$ and symmetric group $\Sym(P)$. Also, let the compact set $S=S_P=\{\eta\in\mathbb{R}^d:P(\eta)=1\}$ in $\mathbb{R}^d$ be equipped with its relative topology and denote by $\mathcal{B}(S)$ the Borel $\sigma$-algebra on $S$. We denote by $\mathcal{L}$ the $\sigma$-algebra of Lebesgue measureable sets on $(0,\infty)$ and let $\lambda_P$ be the $\sigma$-finite measure on $((0,\infty),\mathcal{L})$ with $\lambda_P(dt)=t^{\mu_P-1}\,dt$. Our main theorem is as follows. 

\textcolor{red}{This should go someplace: We refer the reader to Sections 3.6 and 9.2 of \cite{Bogachev2007} for background.}

\begin{theorem}\label{thm:BestIntegrationFormula}
There exists a $\sigma$-algebra $\Sigma$ on $S$ containing $\mathcal{B}(S)$ and a finite Radon measure $\sigma_P$ on $(S,\Sigma)$ which satisfies the following properties:
\begin{enumerate}
\item\label{property:Completion} $(S,\Sigma,\sigma_P)$ is the completion of $(S,\mathcal{B}(S),\sigma_P)$. In particular, $(S,\Sigma,\sigma_P)$ is a complete measure space.
\item\label{property:Invariance} For any $F\in\Sigma$ and $O\in\Sym(P)$, $\sigma(OF)=\sigma(F)$.
\item\label{property:DefiningConditionofsigma} For any $F\in\Sigma$ and $E\in\Exp(P)$, 
\begin{equation*}
\widetilde{F_E}:=\bigcup_{0<t<1}\left(t^E F\right)=\left\{t^E\eta\in\mathbb{R}^d\setminus\{0\}:0<t<1,\eta\in F\right\}
\end{equation*}
is a Lebesgue measurable subset of $\mathbb{R}^d\setminus \{0\}$, i.e., $\widetilde{F_E}\in\mathcal{M}_d$, and
\begin{equation*}
\sigma_P(F)=\mu_P\cdot m\left(\widetilde{F_E}\right)
\end{equation*}
where $m$ denotes the Lebesgue measure on $\mathbb{R}^d$.
\end{enumerate}
\textcolor{red}{Is Property \ref{property:Invariance} characterizing? If so, it would be nice to rephrase the above as: There is a unique Radon measure..... \\}
Further, denote by $\left((0,\infty),\times S,(\mathcal{L}\times\Sigma)',\lambda_P\times\sigma_P\right)$ the completion of the product measure space $((0,\infty\times S,\mathcal{L}\times\Sigma,\lambda_P\times\sigma_P)$. We have
\begin{enumerate}
\item\label{property:BestPointIsomorphism} Given any $E\in \Exp(P)$, the map $\psi_E:(0,\infty)\times S\to\mathbb{R}^d\setminus\{0\}$, defined by $\psi_E(t,\eta)=t^E\eta$ for $t>0$ and $\eta\in S$, is a point isomorphism of the measure spaces $\left((0,\infty)\times S,(\mathcal{L}\times\Sigma)',\lambda_P\times\sigma_P\right)$ and $(\mathbb{R}^d\setminus\{0\},\mathcal{M}_d,m)$. That is
\begin{equation*}
\mathcal{M}_d=\left\{A\subseteq \mathbb{R}^d\setminus\{0\}:\psi_E^{-1}(A)\in (\mathcal{L}\times\Sigma)'\right\}
\end{equation*}
and, for each $A\in\mathcal{M}_d$,
\begin{equation*}
m(A)=(\lambda_P\times\sigma_P)(\psi_E^{-1}(A)).
\end{equation*}
\item\label{property:BestIntegrationFormula} Given any Lebesgue measureable function $f:\mathbb{R}^d\to\mathbb{C}$ and $E\in \Exp(P)$, $f\circ \psi_E$ is $(\mathcal{L}\times\Sigma)'$-measurable and the following statements hold:
\begin{enumerate}
\item If $f\geq 0$, then
\begin{equation}\label{eq:BestIntegrationFormula}
\int_{\mathbb{R}^d}f(x)\,dx=\int_0^\infty\left(\int_S f(t^E\eta)\,\sigma_P(d\eta)\right)t^{\mu_P-1}\,dt=\int_S\left(\int_0^\infty f(t^E\eta)t^{\mu_P-1}\,dt\right)\sigma_P(d\eta).
\end{equation}
\item When $f$ is complex-valued, we have 
\begin{equation*}f\in L^1(\mathbb{R}^d)\hspace{.5cm}\mbox{ if and only if}\hspace{.5cm}f\circ\psi_E\in L^1\left((0,\infty)\times S,(\mathcal{L}\times\Sigma)',\lambda_P\times\sigma_P\right)
\end{equation*} 
and, in this case, \eqref{eq:BestIntegrationFormula} holds.
\end{enumerate}
\end{enumerate}
\end{theorem}

\textcolor{red}{ I would like the paragraphs before Subsection \ref{subsec:ConstructionofSigma} to do the following:
\begin{enumerate}
    \item Show that Theorem \ref{thm:BestIntegrationFormula} recaptures the spherical measure (and has all of the usual desired properties) \eqref{eq:StandardPolarIntegrationFormula}.
    \item Talk about how the polar integration formula is useful in answering questions related to $\sigma$.
    \item Talk about how the polar integration formula is aimed at dealing with functions of the form $f(x)=f_0(P(x))$-- the generalization of radial functions. Here is a blurb:
    \begin{quote}
     As \eqref{eq:StandardPolarIntegrationFormula} finds utility in the harmonic analysis of radial functions, i.e., those functions of the form $f(x)=f_0(|x|)$, Theorem \ref{thm:BestIntegrationFormula} will aid the analysis of functions of the form $f(x)=f_0(P(x))$. In fact, this is a primary motivation for our work.
    \end{quote}
    \item Say something about our proof only using tools from point-set topology and measure thoery
    \item Tell the reader how the theorem is proved. Here is a blurb:
    \begin{quote}
    \noindent Our construction proceeds as follows. In Subsection \ref{subsec:ConstructionofSigma}, we fix $E\in\Exp(P)$ and consider the one-parameter contracting group $\{t^E\}$. As the standard isotropic one-parameter group $t\mapsto tI=t^I$ is well-fitted to the unit sphere $\mathbb{S}$ and allows every non-zero $x\in\mathbb{R}^d$ to be written uniquely as $x=t\eta$ for $t\in (0,\infty)$ and $\eta\in \mathbb{S}$, $\{t^E\}$ is well-fitted to to $S$ and has the property that every non-zero $x\in\mathbb{R}^d$ can be written uniquely as $x=t^E\eta$ where $t\in(0,\infty)$ and $\eta\in S$. With this one-parameter group as a tool, we define a \textit{surface-carried}\footnote{\textcolor{red}{We really need to see if this is standard vocabulary}} measure $\sigma=\sigma_{P,E}$ on $S$ by taking sufficiently nice sets $F\subseteq S$, stretching them into a quasi-conical region of the associated ``ball" $B$ with the contracting group $\{t^E\}$, and computing the Lebesgue measure of the result. In Subsection \ref{subsec:ProductMeasure}, we turn our focus to an associated product measure $\lambda\times\sigma$ on $(0,\infty)\times S$ with which we are able to formulate and prove a generalization of \eqref{eq:StandardPolarIntegrationFormula}; this is Theorem \ref{thm:MainIntegrationFormula}. We then derive a number of corollaries of Theorem \ref{thm:MainIntegrationFormula}, including the result that $\sigma$ is a Radon measure on $S$. As everything done in Subsections \ref{subsec:ConstructionofSigma} and \ref{subsec:ProductMeasure} is done using the contracting group $\{t^E\}$ for a chosen $E\in\Exp(P)$, it isn't clear, a priori, exactly how $\sigma$ is dependent on the choice of $E\in\Exp(P)$, if at all. In Subsection \ref{subsec:IndepdendentofE}, we prove that $\sigma$ is, in fact, independent of the choice of $E\in \Exp(P)$ and this quickly yields a stronger version of Theorem \ref{thm:MainIntegrationFormula}; this is Theorem \ref{thm:BestIntegrationFormula}.\\
    \end{quote}
    \item Set the stage for the stuff on manifolds. Here is a blurb:
    \begin{quote}
     Our construction of $\sigma$ in the present section uses only tools from point-set topology and measure theory.  In Section \ref{sec:SigmaForSmoothP}, we study the special case in which a positive-homogeneous function $P$ is additionally smooth. In that case, we will find that $S$ is a smooth compact embedded hypersurface of $\mathbb{R}^d$ and the measure $\sigma$ is closely related to the Riemannian volume on $S$.
    \end{quote}
    \item fix notation. Here is a blurb:
    \begin{quote}
\noindent Throughout this section, $P$ denotes a fixed positive-homogeneous function on $\mathbb{R}^d$ with homogeneous order $\mu_P>0$ and exponent set $\Exp(P)$. We will take $S=S_P=\{\eta\in\mathbb{R}^d:P(\eta)=1\}$ to be equipped with the relative topology inherited from $\mathbb{R}^d$ and, given $(0,\infty)$ with its usual topology, we take $(0,\infty)\times S$ to be equipped with the product topology. 
    \end{quote}
\end{enumerate}}

  
 \begin{corollary}\label{cor:IntegrateOnS}
Given $g:S\to\mathbb{C}$ and $E\in\Exp(P)$, define $G:\mathbb{R}^d\setminus \{0\}\to\mathbb{C}$ by
\begin{equation*}
G(x)=\mu_P\cdot \chi_{(0,1)}(P(x))g(P(x)^{-E}x)
\end{equation*}
for $x\in\mathbb{R}^d\setminus\{0\}$. If $g\in L^1(S,\Sigma_P,\sigma_p)$, then $G\in L^1(\mathbb{R}^d\setminus\{0\})$, and 
\begin{equation*}
    \int_{\mathbb{R}^d}G(x)\,dx=\int_Sg(\eta)\sigma_P(d\eta).
\end{equation*}
\end{corollary}
\begin{proof}
Observe that, for the $(\mathcal{L}\times\Sigma_P)'$-measurable function $k(t,\eta)=\chi_{(0,1)}(t)g(\eta)$,
\begin{equation*}
    k\circ\psi_E^{-1}(x)=k(P(x),P(x)^{-E}x)=G(x)
\end{equation*}
for $x\in\mathbb{R}^d\setminus \{0\}$. By virtue of Theorem \ref{thm:BestIntegrationFormula}, it follows that $G$ is $\mathcal{M}_d$ measurable and 
\begin{eqnarray*}
   \int_{\mathbb{R}^d}|G(x)|\,dx&=&\int_{S}\left(\int_{(0,\infty)}|k(t,\eta)|t^{\mu_P-1}\,dt\right)\sigma_P(d\eta)\\
    &=&\left(\int_S|g(\eta)|\sigma_P(d\eta)\right)\left(\int_0^1 \mu_P t^{\mu_P-1}\,dt\right)=\|g\|_{L^1(\sigma_P)}<\infty.
\end{eqnarray*}
Therefore, by an analogous computation (for $G$ instead of $|G|$), we have
\begin{equation*}
    \int_{\mathbb{R}^d}G(x)\,dx=\int_S g(\eta)\sigma_P(d\eta),
\end{equation*}
by virtue of Property \ref{property:BestIntegrationFormula} of Theorem \ref{thm:BestIntegrationFormula}. 
\end{proof}











\subsection{Construction of $\sigma_{P,E}$}\label{subsec:ConstructionofSigma}

Throughout this section, we fix $E\in\Exp(P)$. Define $\psi_E:(0,\infty)\times S\to\mathbb{R}^d\setminus\{0\}$ by
\begin{equation}\label{eq:Homeomorphism}
\psi_E(t,\eta)=t^E\eta
\end{equation}
for $t>0$ and $\eta\in S$. As $\psi_E$ is the restriction of the continuous function $(0,\infty)\times \mathbb{R}^d\ni (t,x)\mapsto t^E x\in\mathbb{R}^d$ to $(0,\infty)\times S$, it is necessarily continuous. As the following proposition shows, $\psi_E$ is, in fact, a homeomorphism.

\begin{proposition}\label{prop:PsiHomeomorphism}
The map $\psi_E:(0,\infty)\times S\to\mathbb{R}^d\setminus\{0\}$ is a homeomorphism with continuous inverse $\psi_E^{-1}:\mathbb{R}^d\setminus\{0\}\to (0,\infty)\times S$ given by
\begin{equation*}
\psi_E^{-1}(x)=(P(x),(P(x))^{-E}x)
\end{equation*}
for $x\in\mathbb{R}^d\setminus\{0\}$.
\end{proposition}

\begin{proof}
Given that $P$ is continuous and positive-definite, $P(x)>0$ for each $x\in \mathbb{R}^d\setminus\{0\}$ and the map $\mathbb{R}^d\setminus\{0\}\ni x \mapsto (P(x))^{-E}x\in \mathbb{R}^d$ is continuous. Further, in view of the homogeneity of $P$,
\begin{equation*}
P\left((P(x))^{-E}\xi\right)=P(x)^{-1}P(x)=1
\end{equation*}
for all $x\in\mathbb{R}^d\setminus\{0\}$. It follows from these two observations that
\begin{equation*}
\rho(x)=(P(x),(P(x))^{-E}x),
\end{equation*}
defined for $x\in\mathbb{R}^d\setminus\{0\}$, is a continuous function taking $\mathbb{R}^d\setminus\{0\}$ into $(0,\infty)\times S$. We have
\begin{equation*}
(\psi_E\circ \rho)(x)=\psi_E(P(x),(P(x))^{-E}x)=(P(x))^{E}(P(x))^{-E}x=x
\end{equation*}
for every $x\in \mathbb{R}^d\setminus \{0\}$ and
\begin{equation*}
(\rho\circ\psi_E)(t,\eta)=\rho(t^E\eta)=(P(t^{E}\eta),(P(t^{E}\eta))^{-E}(t^E\eta))=(tP(\eta),(tP(\eta))^{-E}(t^{E}\eta))=(t,\eta)
\end{equation*}
for every $(t,\eta)\in (0,\infty)\times S$. Thus $\rho$ is a (continuous) inverse for $\psi_E$ and so it follows that $\psi_E$ is a homeomorphism and $\rho=\psi_E^{-1}$.
\end{proof}


\begin{remark}In Section \ref{sec:SigmaForSmoothP}, we shall treat the special case that $P\in C^{\infty}(\mathbb{R}^d)$. In this case, it is straightforward to see that $S$ is a smooth embedded hypersurface and $\psi_E$ is a diffeomorphism. \textcolor{red}{(Point the reader to these statements)}
\end{remark}

\noindent We shall now construct the $\sigma$-algebra $\Sigma_{E}$ on $S$; later, we will show that it is independent of our choice of $E$. As in the statement of Theorem \ref{thm:BestIntegrationFormula}, for each $F\subseteq S$, define
\begin{equation*}
\widetilde{F_E}=\bigcup_{0<t<1}\left(t^E F\right)=\{t^E\eta:0<t<1,\eta\in F\}. 
\end{equation*}
We shall denote by $\Sigma_{E}$ the collection of subsets $F$ of $S$ for which $\widetilde{F_E}\in\mathcal{M}_d$, i.e.,  
\begin{equation*}
\Sigma_{E}=\{F\subseteq S:\widetilde{F_E}\in\mathcal{M}_d\}.
\end{equation*}


\begin{proposition}\label{prop:BorelContainment}
$\Sigma_{E}$ is a $\sigma$-algebra on $S$ containing the Borel $\sigma$-algebra on $S$, $\mathcal{B}(S)$.
\end{proposition}

\begin{proof}
Throughout the proof, we write $\Sigma=\Sigma_{E}$ and $\widetilde{F}=\widetilde{F_E}$ for each $F\subseteq S$.
We first show that $\Sigma$ is a $\sigma$-algebra. Since $\widetilde S=B\setminus\{0\}$, it is open in $\mathbb{R}^d\setminus\{0\}$ and therefore Lebesgue measurable. Hence $S\in \Sigma$. Let $G, F\in \Sigma$ be such that $G\subseteq F$. Then,
\begin{equation*}
\widetilde{F\setminus G}=\bigcup_{0<t<1}t^E\left(F\setminus G\right)=\bigcup_{0<t<1}\left(t^EF\setminus t^E G\right)=\left(\bigcup_{0<t<1}t^E F\right)\setminus\left(\bigcup_{0<t<1}t^E G\right)=\widetilde F\setminus \widetilde G
\end{equation*}
where we have used the fact that the collection $\{t^E F\}_{0<t<1}$ is mutually disjoint to pass the union through the set difference. Consequently $\widetilde F\setminus \tilde{G}$ is Lebesgue measurable and therefore $F\setminus G\in \Sigma$.  Now, given a countable collection $\{F_n\}\subseteq \Sigma_S$, observe that
\begin{equation*}
    \widetilde{\bigcup_{n=1}^\infty F_n}= \bigcup_{0<t<1} t^E \left(\bigcup_{n=1}^\infty F_n\right)= \bigcup_{0 <t < 1}  \bigcup_{n=1}^\infty  t^E F_n =\bigcup_{n=1}^\infty \bigcup_{0 <t < 1}  t^E F_n =\bigcup_{n=1}^\infty \widetilde{F_n} \in \mathcal{M}_d
\end{equation*}
whence $\cup_n F_n\in \Sigma_S$. Thus $\Sigma$ is a $\sigma$-algebra. 

Finally, we show that
\begin{equation*}
\mathcal{B}(S)\subseteq\Sigma.
\end{equation*}
As the Borel $\sigma$-algebra is the smallest $\sigma$-algebra containing the open subsets of $S$, it suffices to show that $\mathcal{O}\in \Sigma$ whenever $\mathcal{O}$ is open in $S$. Armed with Proposition \ref{prop:PsiHomeomorphism}, this is an easy task: Given an open set $\mathcal{O}\subseteq S$, observe that
\begin{equation*}
\widetilde{\mathcal{O}}=\{t^E\eta:0<t<1,\eta\in\mathcal{O}\}=\psi_E((0,1)\times\mathcal{O}).
\end{equation*}
Upon noting that $(0,1)\times\mathcal{O}$ is an open subset of $(0,\infty)\times S$, Proposition \ref{prop:PsiHomeomorphism} guarantees that $\widetilde{\mathcal{O}}=\psi_E((0,1)\times\mathcal{O})\subseteq\mathbb{R}^d\setminus\{0\}$ is open and therefore Lebesgue measurable. Thus $\mathcal{O}\subseteq \mathcal{M}_d$.
\end{proof}

\noindent We are now ready to specify a measure on the measurable space $(S,\Sigma_{E})$. For each $F\in \Sigma_{E}$, we define
\begin{equation*}
\sigma_{P,E}(F)=\mu_P\cdot m(\widetilde{F_E})
\end{equation*}
where $m$ is the Lebesgue measure on $\mathbb{R}^d$ and $\mu_P=\tr E>0$ is the homogeneous order associated to $P$.

\begin{proposition}\label{prop:sigmaisameaure}
$\sigma_{P,E}$ is a finite measure on $(S,\Sigma_{E})$.
\end{proposition}
\begin{proof}

\noindent Throughout the proof, we will write $\sigma=\sigma_{P,E}$, $\Sigma=\Sigma_{E}$, and, $\widetilde{F}=\widetilde{F_E}$ for each $F\subseteq S$. It is clear that $\sigma$ is non-negative and $\sigma(\varnothing)=0$ because $\widetilde{\varnothing}=\varnothing$. Let $\{ F_n  \}^\infty_{n=1} \subseteq \Sigma $ be a mutually disjoint collection. We claim that $\{ \widetilde{F_n} \}_{n=1}^\infty\subseteq\mathcal{M}_d$ is also a mutually disjoint collection. To see this, suppose that $x = t_n^E \eta_n = t_m^E \eta_m\in \widetilde{F_n}\cap\widetilde{F_m}$, where $t_n,t_m \in (0,1)$, $\eta_n \in F_n$, and $\eta_m \in F_m $. Then
\begin{equation*}
    t_n = P(t_n^E \eta_n) = P(x) = P(t_m^E \eta_m) = t_m,
\end{equation*}
implying that $\eta_n = \eta_m\in F_n\cap F_m$. Because $\{F_n\}_{n=1}^\infty$ is mutually disjoint, we must have $n=m$ which verifies our claim. By virtue of the countable additivity of Lebesgue measure, we therefore have
\begin{equation*}
\sigma\left(\bigcup_{n=1}^\infty F_n\right)
    = \mu_P\cdot m\left( \widetilde{\bigcup^\infty_{n=1} F_n } \right)=\mu_P\cdot m\left( \bigcup^\infty_{n=1}\widetilde{F_n} \right)
    = \mu_P\sum^\infty_{n=1} m(\widetilde{F_n})
    = \sum^\infty_{n=1}\sigma(F_n).
\end{equation*}
Therefore $\sigma$ is a measure on $(S,\Sigma)$. In view of Condition \ref{cond:PisAboveOne} of Proposition \ref{prop:PositiveHomogeneousCharacterization}, $\widetilde{S}=B\setminus\{0\}$ is a bounded subset of $\mathbb{R}^d\setminus\{0\}$ and hence $\sigma(S)=\mu_P\cdot m(B\setminus\{0\})<\infty$ showing that $\sigma$ is finite.
\end{proof}

\noindent By virtue of the two preceding propositions, $\sigma_{P,E}$ is a finite Borel measure on $S$. \textcolor{red}{As a consequence of the main result in the next subsection, we will see that $\sigma_{P,E}$ is independent of our choice of $E\in\Exp(P)$ and is a Radon measure.}

\subsection{Product Measure and Point Isomorphism}\label{subsec:ProductMeasure}

Throughout this subsection, $E\in\Exp(P)$ will remain fixed and $(S,\Sigma_{E},\sigma_{P,E})$ will denote the finite measure space of Proposition \ref{prop:sigmaisameaure}. We recall that $\mathcal{L}$ denotes the $\sigma$-algebra of Lebesgue measurable subsets of $(0,\infty)$ and  $\lambda_P$ denotes the measure on $(0,\infty)$ with $\lambda_P(dt)=t^{\mu_P-1}\,dt$, i.e., for each $L\in\mathcal{L}$,
\begin{equation*}
\lambda_P(L)=\int_0^\infty \chi_L(t)t^{\mu_P-1}\,dt.
\end{equation*}
It is easy to see that $\lambda_P$ is $\sigma$-finite and so, in view of the finiteness of the measure $\sigma_{P,E}$, there exists a unique product measure $\lambda_P\times\sigma_{P,E}$ on $(0,\infty)\times S$ equipped with the product $\sigma$-algebra $\mathcal{L}\times\Sigma_{E}$ which satisfies
\begin{equation*}
    (\lambda_P\times\sigma_{P,E})(L\times F)=\lambda_P(L)\sigma_{P,E}(F)
\end{equation*}
for all $L\in\mathcal{L}$ and $F\in\Sigma_{E}$. We shall denote by $((0,\infty)\times S,(\mathcal{L}\times\Sigma_S)',\lambda_P\times\sigma_{P,E})$ the completion of the measure space $((0,\infty)\times S,\mathcal{L}\times\Sigma_E,\lambda_P\times\sigma_P)$. So that it is at our fingertips, we state the Fubini-Tonelli theorem associated to $\lambda_P\times \sigma_{P,E}$.
\begin{theorem}[Theorem 8.12 of \cite{Rudin1987}]\label{thm:Fubini}
Let $g:(0,\infty)\times S\to\mathbb{C}$ be $(\mathcal{L}\times\Sigma_E)'$-measurable. For each $t\in (0,\infty)$, define $g^t:S\to\mathbb{C}$ by $g^t(\eta)=g(t,\eta)$ for $\eta\in S$ and, for each $\eta\in S$, define $g_\eta:(0,\infty)\to\mathbb{C}$ by $g_\eta(t)=g(t,\eta)$ for $t\in (0,\infty)$. 
\begin{enumerate}
\item For $\lambda_P$-almost every $t$, $g^t$ is $\Sigma_E$-measurable and, for $\sigma_{P,E}$-almost every $\eta$, $g_\eta$ is $\mathcal{L}$-measurable.
\item\label{item:Fubini1} If $g\geq 0$, then:
\begin{enumerate}
\item For $\lambda_P$-almost every $t$, 
\begin{equation*}
H(t)=\int_S g^t(\eta)\,\sigma_{P,E}(d\eta)
\end{equation*}
exists as a non-negative extended real number. 
\item For $\sigma_{P,E}$-almost every $\eta$,
\begin{equation*}
G(\eta)=\int_0^\infty g_\eta(t)t^{\mu_P-1}\,dt
\end{equation*}
exists as a non-negative extended real number. 
\item We have
\begin{equation}\label{eq:Fubini1}
\int_0^\infty H(t)t^{\mu_P-1}dt=\int_{(0,\infty)\times S}g\,d(\lambda_P\times\sigma_{P,E})=\int_S G(\eta)\,\sigma_{P,E}(d\eta)
\end{equation}
and, in particular,
\begin{equation}\label{eq:Fubini2}
\int_0^\infty\left(\int_S g(t,\eta)\,\sigma_{P,E}(d\eta)\right)t^{\mu_P-1}\,dt=\int_S\left(\int_0^\infty g(t,\eta)t^{\mu_P-1}\,dt\right)\,\sigma_{P,E}(d\eta).
\end{equation}
\end{enumerate}
\item\label{item:Fubini2} If $g$ is complex valued and
\begin{equation*}
\int_S\left(\int_0^\infty |g(t,\eta)|t^{\mu_P-1}\,dt\right)\,\sigma_{P,E}(d\eta)<\infty\hspace{.5cm}\mbox{or}\hspace{.5cm}\int_0^\infty\left(\int_S|g(t,\eta)|\,\sigma_{P,E}(d\eta)\right)t^{\mu_P-1}\,dt<\infty,
\end{equation*}
then $g\in L^1((0,\infty)\times S,(\mathcal{L}\times\Sigma_E)',\lambda_P\times\sigma_{P,E})$.
\item If $g\in L^1((0,\infty)\times S,(\mathcal{L}\times\Sigma_E)',\lambda_P\times\sigma_{P,E})$, then $g^t\in L^1(S,\Sigma_E,\sigma_{P,E})$ for $\lambda_P$-almost every $t$, $g_\eta\in L^1((0,\infty),\mathcal{L},\lambda_P)$ for $\sigma_{P,E}$-almost every $\eta$, and Equations \eqref{eq:Fubini1} and \eqref{eq:Fubini2} hold.
\end{enumerate}
\end{theorem}

\noindent Our primary goal in this subsection is to prove the theorem below. We note that Properties \ref{item:MainIntegrationFormula1} and \ref{item:MainintegrationFormula2} in Theorem \ref{thm:MainIntegrationFormula} differ only from Properties \ref{property:BestPointIsomorphism} and \ref{property:BestIntegrationFormula} in Theorem \ref{thm:BestIntegrationFormula} in that, a priori, the $\sigma$-algebra $\Sigma_E$ and the measure $\sigma_{P,E}$ in Theorem \ref{thm:MainIntegrationFormula} depends on our choice of $E\in\Exp(P)$. As a consequence of Theorem \ref{thm:MainIntegrationFormula}, we shall see in Subsection \ref{subsec:IndependentofE} that $\Sigma_{E_1}=\Sigma_{E_2}$ and $\sigma_{P,E_1}=\sigma_{P,E_2}$ for all $E_1,E_2\in\Exp(P)$ and so this apparent dependence is superficial; this is Proposition \ref{prop:Endependence}. As a consequence of the proposition, we shall obtain Properties \ref{property:BestPointIsomorphism} and \ref{property:BestIntegrationFormula} of Theorem \ref{thm:BestIntegrationFormula} immediately from Properties of \ref{item:MainIntegrationFormula1} and \ref{item:MainintegrationFormula2} in Theorem \ref{thm:MainIntegrationFormula}.


\begin{theorem}\label{thm:MainIntegrationFormula}
Let $((0,\infty)\times S,(\mathcal{L}\times\Sigma_E)',\lambda_P\times\sigma_{P,E})$ be as above and let $m$ be the (restricted) Lebesgue measure on $(\mathbb{R}^d\setminus\{0\},\mathcal{M}_d)$.
\begin{enumerate}
\item\label{item:MainIntegrationFormula1} The map $\psi_E: (0,\infty)\times S\to\mathbb{R}^d\setminus\{0\}$, defined by \eqref{eq:Homeomorphism}, is a point isomorphism of the measure spaces $((0,\infty)\times S,(\mathcal{L}\times\Sigma_E)',\lambda_P\times\sigma_{P,E})$ and $(\mathbb{R}^d\setminus\{0\},\mathcal{M}_d,m)$. That is
\begin{equation*}
\mathcal{M}_d=\{A\subseteq \mathbb{R}^d\setminus\{0\}:\psi_E^{-1}(A)\in(\mathcal{L}\times\Sigma_E)'\}
\end{equation*}
and, for each $A\in\mathcal{M}_d$,
\begin{equation*}
m(A)=(\lambda_P\times\sigma_{P,E})(\psi_E^{-1}(A)).
\end{equation*}
\item\label{item:MainintegrationFormula2} If $f:\mathbb{R}^d\to\mathbb{C}$ is Lebesgue measurable, then $f\circ \psi_E$ is $(\mathcal{L}\times\Sigma_E)'$-measurable and the following statements hold:
\begin{enumerate}
\item If $f\geq 0$, then
\begin{equation}\label{eq:MainIntegrationFormula}
\int_{\mathbb{R}^d}f(x)\,dx=\int_0^\infty\left(\int_S f(t^E\eta)\,\sigma_{P,E}(d\eta)\right)t^{\mu_P-1}\,dt=\int_S\left(\int_0^\infty f(t^E\eta)t^{\mu_P-1}\,dt\right)\,\sigma_{P,E}(d\eta).
\end{equation}
\item When $f$ is complex-valued, we have 
\begin{equation*}f\in L^1(\mathbb{R}^d)\hspace{.5cm}\mbox{if and only if}\hspace{0.5cm}f\circ\psi_E\in L^1((0,\infty)\times S,(\mathcal{L}\times \Sigma_E)',\lambda_P\times\sigma_{P,E})
\end{equation*}
and, in this case, \eqref{eq:MainIntegrationFormula} holds.
\end{enumerate}
\end{enumerate}
\end{theorem}

\noindent To prove Theorem \ref{thm:MainIntegrationFormula}, we shall first treat several lemmas. These lemmas isolate and generalize several important ideas used in standard proofs of \eqref{eq:StandardPolarIntegrationFormula} (See, e.g., \cite{Folland1984} and \cite{Stein2005}). 

\begin{lemma}\label{lemma:Scaling}
Let $A\subseteq\mathbb{R}^d$ and $t>0$.  $A$ is Lebesgue measurable if and only if $t^E A=\{x=t^E a:a\in A\}$ is Lebesgue measurable and, in this case,
\begin{equation*}
m(t^E A)=t^{\mu_P}m(A).
\end{equation*}
\end{lemma}
%The linear isomorphism thing is standard. For reference it is Theorem 2.20 of \cite{Rudin1987}.
\begin{proof}
Because $x\mapsto t^E x$ is a linear isomorphism, $t^E A$ is Lebesgue measurable if and only if $A$ is Lebesgue measureable. Observe that $x\in t^E A$ if and only if $t^{-E}x\in A$ and therefore
\begin{equation*}
m(t^E A)=\int_{\mathbb{R}^d}\chi_{t^E A}(x)\,dx=\int_{\mathbb{R}^d}\chi_{A}(t^{-E}x)\,dx.
\end{equation*}
Now, by making the linear change of variables $x\mapsto t^E x$, we have
\begin{equation*}
m(t^E A)=\int_{\mathbb{R}^d}\chi_A(x)|\det(t^E)|\,dx=t^{\mu_P}m(A),
\end{equation*}
because $\det(t^E)=t^{\tr E}=t^{\mu_P}>0$ by virtue of Proposition \ref{prop:ContinuousGroupProperties} and Corollary \ref{cor:TraceisInvariant}.
\end{proof}

\begin{lemma}\label{lem:SpecialRectangle}
Let $F\in\Sigma_E$. If $I\subseteq (0,\infty)$ is open, closed, $G_\delta$, or $F_\sigma$, then $\psi_E(I\times F)\in\mathcal{M}_d$ and
\begin{equation}\label{eq:SpecialRectangle}
m(\psi_E(I\times F))=(\lambda_P\times\sigma_{P,E})(I\times F)=\lambda_P(I)\sigma_{P,E}(F).
\end{equation}
\end{lemma}
\begin{proof}
To simplify notation, we shall write $\lambda=\lambda_P$ and $\sigma=\sigma_{P,E}$ throughout the proof. We fix $F\in\Sigma_E$ and consider several cases for $I$.\\

\begin{subproof}[Case 1:]\textit{$I=(0,b)$ for $0<b\leq \infty$.} When $b$ is finite, observe that
\begin{equation*}
\psi_E(I\times F)=\{t^E\eta:0<t<b,\eta\in F\}=b^E\{t^E\eta:0<t<1,\eta \in F\}=b^E\widetilde{F_E}.
\end{equation*}
By virtue of Lemma \ref{lemma:Scaling}, it follows that $\psi_E(I\times F)\in\mathcal{M}_d$ and
\begin{eqnarray*}
(\lambda\times\sigma)(I\times F)&=&\lambda(I)\sigma(F)\\
&=&\left(\int_0^b t^{\mu_P-1}\,dt\right)\left(\mu_P\cdot m(\widetilde{F_E})\right)\\
&=&b^{\mu_P}m(\widetilde{F_E})\\
&=&m(b^{E}\widetilde{F_E})\\
&=&m(\psi_E(I\times F)).
\end{eqnarray*}
When $b=\infty$ i.e., $I=(0,\infty)$, we observe that
\begin{equation*}
I=\bigcup_{n=1}^\infty (0,n)=\bigcup_{n=1}^\infty I_n
\end{equation*}
where the open intervals $I_n=(0,n)$ are nested and increasing. In view of the result above (for finite $b=n$), we have
\begin{equation*}\psi_E(I\times F)=\psi_E\left(\bigcup_{n=1}^\infty (I_n\times F)\right)=\bigcup_{n=1}^\infty\psi_E(I_n\times F)\in\mathcal{M}_d.
\end{equation*}
Given that $\psi_E$ is a bijection, $\{\psi_E(I_n\times F)\}$ is necessarily a nested increasing sequence and so, by the continuity of the measures $\lambda\times\sigma$ and $m$,
\begin{equation*}
(\lambda\times\sigma)(I\times F)=\lim_{n\to\infty}(\lambda\times\sigma)(I_n\times F)=\lim_{n\to\infty}m(\psi_E(I_n\times F))= m(\psi_E(I\times F)). 
\end{equation*}
\end{subproof}

\begin{subproof}[Case 2:]\textit{$I=(0,a]$ for $0<a<\infty$.} We have
\begin{equation*}
I=(0,a]=\bigcap_{n=1}^\infty (0,a+1/n)=\bigcap_{n=1}^\infty I_n
\end{equation*}
where the open intervals $I_n=(0,a+1/n)$ are nested and decreasing. By reasoning analogous to that given in Case 1, we have
\begin{equation*}
\psi_E(I\times F)=\bigcap_{n=1}^\infty \psi_E(I_n\times F)\in \mathcal{M}_d
\end{equation*}
and
\begin{equation*}
(\lambda\times\sigma)(I\times F)=\lim_{n\to\infty}(\lambda\times\sigma)(I_n\times F)=\lim_{n\to\infty}m(\psi_E(I_n\times F))=m(\psi_E(I\times F)).
\end{equation*}
In particular, $m(\psi_E(I\times F))=\lambda((0,a])\sigma(F)=a^{\mu_P}\sigma(F)/\mu_P<\infty.$
\end{subproof}
\begin{subproof}[Case 3:]\textit{$I=(a,b)$ for $0<a<b\leq \infty$.} In this case, $I=(0,b)\setminus (0,a]$ and so, in view of Cases 1 and 2, $\psi_E(I\times F)=\psi_E((0,b)\times F)\setminus \psi_E((0,a]\times F)\in\mathcal{M}_d$ and
\begin{eqnarray*}
(\lambda\times\sigma)(I\times F)&=&(\lambda\times\sigma)( (0,b)\times F)-(\lambda\times\sigma)((0,a]\times F)\\
&=&m(\psi_E((0,b)\times F))-m(\psi_E((0,a]\times F))\\
&=&m(\psi_E(I\times F))
\end{eqnarray*}
where we have used the fact that $(\lambda\times\sigma)((0,a]\times F)=m(\psi_E((0,a]\times F))<\infty$.
\end{subproof}
\begin{subproof}[Case 4:]\textit{$I\subseteq (0,\infty)$ is open.} In this case, it is known that $I$ can be expressed as a countable union of disjoint open intervals $\{I_n\}$ and, by virtue of Cases 1 and 3, we have
\begin{equation*}
\psi_E(I\times F)=\bigcup_{n=1}^\infty\psi_E(I_n\times F)\in\mathcal{M}_d,
\end{equation*}
where this union is disjoint, and
\begin{eqnarray*}
\lefteqn{\hspace{-1cm}m(\psi_E(I\times F))=\sum_n m(\psi_E(I_n\times F))=\sum_n (\lambda\times\sigma)(I_n\times F)}\\
&&\hspace{2cm}=\sum_n \lambda(I_n)\sigma(F)=\left(\sum_n \lambda(I_n)\right)\sigma(F)=\lambda(I)\sigma(F)=(\lambda\times\sigma)(I\times F).
\end{eqnarray*}
\end{subproof}
\begin{subproof}[Case 5:]\textit{$I\subseteq (0,\infty)$ is closed.} In this case, we have $I=(0,\infty)\setminus O$ where $O$ is open and so
\begin{equation*}
\psi_E(I\times F)=\psi_E(F\times ((0,\infty)\setminus O))=\psi_E( (0,\infty)\times F)\setminus \psi_E(O\times F)\in\mathcal{M}_d.
\end{equation*}
At this point, we'd like to use the property that 
\begin{equation*}
m(\psi_E((0,\infty)\times F)\setminus \psi_E( O\times F))=m(\psi_E((0,\infty)\times F))-m(\psi_E(O\times F)),
\end{equation*} but this only holds when $m(\psi_E(O\times F))$ is finite. We must therefore proceed differently. For each natural number $n$, define $O_n=O\cap(0,n)$ and $I_n=(0,n)\setminus O_n$. It is straightforward to show that $\{I_n\}$ and $\{\psi_E(I_n\times F)\}$ are nested and increasing with
\begin{equation*}
I=\bigcup_{n=1}^\infty I_n\hspace{1cm}\mbox{and}\hspace{1cm}\psi_E(I\times F)=\bigcup_{n=1}^\infty \psi_E(I_n\times F). 
\end{equation*}
The results of Cases 1 and 4 guarantee that, for each $n$,
\begin{equation*}
m(\psi_E(O_n\times F))=(\lambda\times\sigma)(O_n\times F)\leq (\lambda\times\sigma)((0,n)\times F)<n^{\mu_P}m(\widetilde{F_E})<\infty
\end{equation*}
and therefore
\begin{equation*}
m(\psi_E(I_n\times F))=m(\psi_E((0,n)\times F))-m(\psi_E(O_n\times F))=(\lambda\times\sigma)((0,n)\times F)-(\lambda\times\sigma)(O_n\times F)=(\lambda\times\sigma)(I_n\times F).
\end{equation*}
Then, by virtue of the continuity of measure,
\begin{equation*}
m(\psi_E(I\times F))=\lim_{n\to\infty}m(\psi_E(I_n\times F))=\lim_{n\to\infty}(\lambda\times\sigma)(I_n\times F)=(\lambda\times\sigma)(I\times F). 
\end{equation*}
\end{subproof}
\begin{subproof}[Case 6]\textit{$I\subseteq (0,\infty)$ is $G_\delta$ or $F_\sigma$.} Depending on whether $I$ is $G_\delta$ or $F_\sigma$,  express $I$ as an intersection of nested decreasing open sets or a union of nested increasing closed sets. In both cases, by virtually the same argument given in the previous cases, we find that $\psi_E(I\times F)\in \mathcal{M}_d$,
\begin{equation*}
m(\psi_E(I\times F))=(\lambda\times\sigma)(I\times F).
\end{equation*}
\end{subproof}
\end{proof}

\begin{comment}\textcolor{blue}{\begin{remark}
Huan, the analogous result to the following lemma is in the first paragraph on Page 281 of \cite{Stein2005} and is in the sentence preceding ``So we have established (10) for all measurable rectangles..." Truthfully, I don't follow their argument and, actually, I don't quite believe it. When checking it, you should verify carefully the claims made in the first few sentences.
\end{remark}}
\end{comment}
\begin{lemma}\label{lem:AllMeasurableRectangles} For any $L\in\mathcal{L}$ and $F\in \Sigma_E$, $\psi_E(L\times F)\in\mathcal{M}_d$ and 
\begin{equation*}
m(\psi_E(L\times F))=(\lambda_P\times\sigma_{P,E})(L\times F).
\end{equation*}
\end{lemma}
\begin{proof}
Fix $L\in\mathcal{L}$ and $F\in\Sigma_E$. It is easy to see that $\lambda_P$ and the Lebesgue measure $dt$ on $(0,\infty)$ are mutually absolutely continuous. It follows that $((0,\infty), \mathcal{L},\lambda_P)$ is a complete measure space and, further, that there exists an $F_\sigma$ set $L_\sigma\subseteq (0,\infty)$ and a $G_\delta$ set $L_\delta\subseteq (0,\infty)$ for which $L_\sigma\subseteq L\subseteq L_\delta$ and $\lambda_P(L_\delta\setminus L_\sigma)=0$. Note that, necessarily, $\lambda_P(L)=\lambda_P(L_\sigma)=\lambda_P(L_\delta)$. We have
\begin{equation}\label{eq:AllMeasurableRectangles1}
\psi_E(L\times F)=\psi_E( L_\sigma\times F)\cup\psi_E((L\setminus L_\sigma)\times F)
\end{equation}
where, by virtue of the preceding lemma, $\psi_E(L_\sigma\times F)\subseteq \mathcal{M}_d$ and
\begin{equation}\label{eq:AllMeasurableRectangles2}
m(\psi_E(L_{\sigma}\times F))=(\lambda_P\times\sigma_{P,E})( L_\sigma\times F)=\lambda_P(L_\sigma)\sigma_{P,E}(F)=\lambda_P(L)\sigma_{P,E}(F)=(\lambda_P\times\sigma_{P,E})(L\times F).
\end{equation}
Observe that
\begin{equation*}
\psi_E((L\setminus L_\sigma)\times F)\subseteq \psi_E((L_{\delta}\setminus L_\sigma)\times F)
\end{equation*}
where, because $L_\delta\setminus L_\sigma$ is an $G_{\delta}$ set, the latter set is a member of $\mathcal{M}_d$ and
\begin{equation*}
m(\psi_E((L_\delta\setminus L_\sigma)\times F))=(\lambda_P\times\sigma_{P,E})((L_\delta\setminus L_\sigma)\times F)=\lambda_P(L_\delta\setminus L_\sigma))\sigma_{P,E}(F)=0
\end{equation*}
by virtue of the preceding lemma. Using the fact that $(\mathbb{R}^d\setminus\{0\},\mathcal{M}_d,m)$ is complete, we conclude that $\psi_E((L\setminus L_\sigma)\times F)\in \mathcal{M}_d$ and $m(\psi_E((L\setminus L_\sigma)\times F))=0$. It now follows from \eqref{eq:AllMeasurableRectangles1} and \eqref{eq:AllMeasurableRectangles2} that $\psi_E(L\times F)\in\mathcal{M}_d$ and
\begin{equation*}
m(\psi_E(L\times F))=m(\psi_E(L_\sigma\times F))+m(\psi_E((L\setminus L_\sigma)\times F))=(\lambda_P\times\sigma_{P,E})(L\times F),
\end{equation*}
as desired.
\end{proof}

\begin{comment}\noindent \textcolor{blue}{As it is fairly elementary, I've commented out the lemma showing that $S$, as a compact set of a metric space, necessarily contains a countably dense set. If this appears in your thesis, you should feel free to include the lemma}
\end{comment}
%\begin{lemma}
%Let $S$ be a compact subset of a metric space. Then $S$ contains a countably dense set.
%\end{lemma}
%\begin{proof}
%For each $n\in\mathbb{N}$, consider the open cover
%\begin{equation*}
%\{B_{1/n}(x)\cap S, x\in S\}
%\end{equation*}
%of $S$. Since $S$ is compact, there exists a finite subcover. Let $x_{j,n}$, $j=1,2,\dots N_n$ denote the center of each of the balls, then, we have that $S$ is covered by $\{B_{1/n}(x_{j,n})\cap S,j=1,2,\dots, N_n\}$. Thus, for each $n\in \mathbb{N}$, we have a finite set $\{x_{j,n}\}$ of centers. The countable union of these finite sets, $\bigcup^\infty_{n=1} \{ x_{j,n}\}$, is countable. It is also dense because for every point $x\in S$ and $\epsilon >0$, there is always some $n$ such that $\vert x_{j,n} - x\vert < 1/n < \epsilon$.  
%\end{proof}


\begin{lemma}\label{lem:OpenRectangle}
Every open subset $U\subseteq \mathbb{R}^d\setminus\{0\}$ can be written as a countable union of open sets of the form $\psi_E(\mathcal{U})$ where $\mathcal{U}=I\times\mathcal{O}$ is an open rectangle in $S\times(0,\infty)$.
\end{lemma}


\begin{proof}
Let $\{t_k\}_{k=1}^\infty$ be a countably dense subset of $(0,\infty)$ and, because $S$ is compact, $S$ has a countably dense set $\{\eta_j\}_{j=1}^\infty$.  For each triple of natural numbers $j,l,n\in\mathbb{N}_+$, consider the open set
\begin{equation*}
\mathcal{U}_{j,l,n}=\{ \vert t - t_j \vert < 1/n \}\times \mathcal{O}_{l,n}\subseteq (0,\infty)\times S
\end{equation*}
where
\begin{equation*}
\mathcal{O}_{l,n}=\{\eta\in S: |\eta-\eta_l|<1/n\}.
\end{equation*}
Fix $U\subseteq \mathbb{R}^d\setminus \{0\}$, an open subset of $\mathbb{R}^d\setminus\{0\}$. We will show that
\begin{equation}\label{eq:OpenRectangle}
U=\bigcup_{\substack{j,l,n\\ \psi_E(\mathcal{U}_{j,l,n})\subseteq U}}\psi_E(\mathcal{U}_{j,l,n}),
\end{equation}
where, in view of Proposition \ref{prop:PsiHomeomorphism}, each $\psi_E(\mathcal{U}_{j,l,n})$ is open. It is clear that any element of the union on the right hand side of \eqref{eq:OpenRectangle} belongs to some $\psi_E(\mathcal{U}_{j,l,n}) \subseteq U$ and so the union is a subset of $U$. To prove \eqref{eq:OpenRectangle}, it therefore suffices to prove that, for each $x\in U$, there exists a triple $j,l,n$ with
\begin{equation*}
x\in\psi_E(\mathcal{U}_{j,l,n})\subseteq U.
\end{equation*}
To this end, fix $x\in U$ and let $\delta>0$ be such that $\mathbb{B}_\delta(x)\subseteq U$. Consider $(t_x,\eta_x)=\psi_E^{-1}(x)\in (0,\infty)\times S$ and set $M=\|t_x^E\|>0$ and $C=\|E\|>0$. Observe that 
\begin{eqnarray*}
\|I-\alpha^E\|&=&\left\|\sum_{k=1}^\infty \frac{(\ln \alpha)^k}{k!} E^k\right\|\\
&\leq &\sum_{k=1}^\infty \frac{|\ln \alpha|^k}{k!} \|E\|^k=e^{(C|\ln \alpha|)}-1
\end{eqnarray*}
for all $\alpha>0$. Since $\alpha\mapsto e^{(C|\ln \alpha|)}-1$ is continuous and $0$ at $\alpha=1$, we can choose $\delta'>0$ for which
\begin{equation*}
\|I-\alpha ^E\|< \frac{\delta}{2M (  |\eta_x|+2)}
\end{equation*}
whenever $|\alpha-1|<\delta'$. Choose an integer
\begin{equation*}
n>\max \left\{\frac{1}{\delta't_x}, \frac{4 M }{\delta}\right\}.
\end{equation*}
In view of the density of the collections $\{t_j\}$ and $\{\eta_l\}$, we can find $t_j, \eta_l$ such that
\begin{equation*}
    \vert t_j - t_x \vert < \frac{1}{n},\quad \vert \eta_l - \eta_x \vert < \frac{1}{n}.
\end{equation*}
It follows that the corresponding open set $\mathcal{U}_{j,l,n}$ contains $\psi_E^{-1}(x)$, or, equivalently, $x\in \psi_E(\mathcal{U}_{j,l,n})$. Thus, it remains to show that $\psi_E(\mathcal{U}_{j,l,n}) \subseteq \mathbb{B}_\delta(x)$. To this end, let $y=\psi_E(t_y,\eta_y,)\in\psi_E(\mathcal{U}_{j,l,n})$ and observe that
\begin{eqnarray*}
| x - y | &\leq& \vert \psi_E(t_x,\eta_x,) - \psi_E(t_x,\eta_y) \vert 
    + \vert \psi_E(t_x,\eta_y) - \psi_E(t_y,\eta_y) \vert\\
    &=&  \vert t_x^E (\eta_x - \eta_y) \vert + \vert (t_x^E - t_y^E) \eta_y \vert\\
    &\leq& M\vert \eta_x - \eta_y \vert + \|{t_x^E - t_y^E}\|  \vert \eta_y \vert.
\end{eqnarray*}
Since both $(\eta_x,t_x),(\eta_y,t_y) \in \mathcal{U}_{j,l,n}$, we have
\begin{equation*}
    \vert \eta_x - \eta_y \vert \leq \vert \eta_x - \eta_j \vert + \vert \eta_j - \eta_y \vert < \frac{2}{n}
\end{equation*}
and
\begin{equation*}
    \vert \eta_y \vert \leq \vert \eta_y - \eta_x \vert + \vert \eta_x \vert < \vert \eta_x \vert + \frac{2}{n}.
\end{equation*}
Also, since $|t_x-t_y|<1/n$, it follows that $t_y=\alpha t_x$ where $|1-\alpha|<1/nt_x < \delta'$ by our choice of $n$. Consequently,
\begin{eqnarray*}
    \vert x - y \vert 
    &< & \frac{2}{n} M+ \left( \vert \eta_x \vert + \frac{2}{n} \right) \|{t_x^E -   t_x^E \alpha^E}\|   \\ 
    &<& \frac{2}{n}M + \left( \vert \eta_x \vert + 2 \right)M\| I - \alpha^E\| \\
    &<&  \frac{2M }{n} +  \frac{\delta M \left( \vert \eta_x \vert + 2\right) }{2M (| \eta_x | + 2)}  \\
    &<& \frac{\delta}{2} + \frac{\delta}{2}=\delta 
\end{eqnarray*}
and so we have established \eqref{eq:OpenRectangle}. Finally, upon noting that $\{\mathcal{U}_{j,l,n})\}$ is a countable collection of open rectangles (indexed by $(j,l,n)\in\mathbb{N}_+^3$), the union in \eqref{eq:OpenRectangle} is necessarily countable and we are done with the proof.
\end{proof}

\begin{comment}\noindent \textcolor{blue}{The following is a (graduate level) homework-exercise worthy lemma. You should try to prove it yourself before you read the proof. Though it is somewhat difficult (to state and prove -- for me, at least), it is abstract enough that I suspect it is fairly well-known and we should look for a reference.}
\end{comment}

\noindent In our final lemma preceding the proof of Theorem \ref{thm:MainIntegrationFormula}, we treat a general measure-theoretic statement which gives sufficient conditions concerning two measure spaces to ensure that their completions are isomorphic. Though we suspect that this result is well-known, we present its proof for completeness.

\
\begin{lemma}\label{lem:PushforwardLemma}
Let $(X_1,\Sigma_1,\nu_1)$ and $(X_2,\Sigma_2,\nu_2)$ be measure spaces, let $\varphi:X_1\to X_2$ be a bijection and denote by $(X_i,\Sigma_i',\nu_i')$ the completion of the measure space $(X_i,\Sigma_i,\nu_i)$ for $i=1,2$. Assume that the following two properties are satisfied:
\begin{enumerate}
\item\label{property:PushforwardLemma1} For each $A_1\in\Sigma_1$, $\varphi(A_1)\in\Sigma_2'$ and $\nu_2'(\varphi(A_1))=\nu_1(A_1).$
\item\label{property:PushforwardLemma2} For each $A_2\in\Sigma_2$, $\varphi^{-1}(A_2)\in \Sigma_1'$ and $\nu_1'(\varphi^{-1}(A_2))=\nu_2(A_2)$.
\end{enumerate}
Then the measure spaces $(X_1,\Sigma_1',\nu_1')$ and $(X_2,\Sigma_2',\nu_2')$ are isomorphic with point isomorphism $\varphi$. Specifically,
\begin{equation}\label{eq:PushforwardLemma1}
\Sigma_2'=\{A_2\subseteq X_2: \varphi^{-1}(A_2)\in\Sigma_1'\}
\end{equation}
and
\begin{equation}\label{eq:PushforwardLemma2}
\nu_2'(A_2)=\nu_1'(\varphi^{-1}(A_2))
\end{equation}
for all $A_2\in\Sigma_2'$.
\end{lemma}
\begin{proof}
Let us first assume that $A_2\in\Sigma_2'$. By definition, $A_2=G_2\cup H_2$ where $G_2\in\Sigma_2$ and $H_2\subseteq G_{2,0}\in \Sigma_2$ with $\nu_2'(A_2)=\nu_2(G_2)$ and $\nu_2'(H_2)=\nu_2(G_{2,0})=0$. Consequently, $\varphi^{-1}(A_2)=\varphi^{-1}(G_2)\cup\varphi^{-1}(H_2)$ and $\varphi^{-1}(H_2)\subseteq \varphi^{-1}(G_{2,0})$. In view of Property \ref{property:PushforwardLemma2}, $\varphi^{-1}(G_2),\varphi^{-1}(G_{2,0})\in \Sigma_1'$ and we have
\begin{equation*}
\nu_1'(\varphi^{-1}(G_2))=\nu_2(G_2)=\nu_2'(A_2)\hspace{1cm}\mbox{and}\hspace{1cm}\nu_1'(\varphi^{-1}(G_{2,0}))=\nu_2(G_{2,0})=0.
\end{equation*}
In view of the fact that $(X_1',\Sigma_1',\nu_1')$ is complete, $\varphi^{-1}(H_2)\in\Sigma_1'$ and $\nu_1'(\varphi^{-1}(H_2))=0$. Consequently, we obtain $\varphi^{-1}(A_2)=\varphi^{-1}(G_2)\cup\varphi^{-1}(H_2)\in\Sigma_1'$ and
\begin{equation*}
\nu_2'(A_2)=\nu_1'(\varphi^{-1}(G_2))\leq\nu_1'(\varphi^{-1}(A_2))\leq\nu_1'(\varphi^{-1}(G_2))+\nu_1'(\varphi^{-1}(H_2))=\nu_2(G_2)+0=\nu_2'(A_2).
\end{equation*}
From this we obtain that $\Sigma_2'\subseteq \{A_2\subseteq X_2:\varphi^{-1}(A_2)\in\Sigma_1'\}$ and, for each $A_2\in\Sigma_2'$, $\nu_2'(A_2)=\nu_1'(\varphi^{-1}(A_2))$. It remains to prove that
\begin{equation*}
\{A_2\subseteq X_2:\varphi^{-1}(A_2)\in\Sigma_1'\}\subseteq \Sigma_2'.
\end{equation*}
To this end, let $A_2$ be a subset of $X_2$ for which $\varphi^{-1}(A_2)\in\Sigma_1'$. By the definition of $\Sigma_1'$, we have $\varphi^{-1}(A_2)=G_1\cup H_1$ where $G_1\in\Sigma_1$, $H_1\subseteq G_{1,0}\in\Sigma_1$ and $\nu_1'(H_1)=\nu_1(G_{1,0})=0$. In view of Property \ref{property:PushforwardLemma1}, $\varphi(G_1)\in\Sigma_2'$, $\varphi(H_1)\subseteq\varphi(G_{1,0})\in\Sigma_2'$ and $\nu_2'(\varphi(G_{1,0}))=\nu_1(G_{1,0})=0$. Because $(X_2',\Sigma_2',\nu_2')$ is complete, we have $\varphi(H_1)\in\Sigma_2'$ and so
\begin{equation*}
A_1=\varphi(\varphi^{-1}(A_2))=\varphi(G_1)\cup\varphi(H_1)\in \Sigma_2',
\end{equation*}
as desired.
\end{proof}

\noindent We are finally in a position to prove Theorem \ref{thm:MainIntegrationFormula}.\\

\begin{comment}\noindent\textcolor{blue}{Huan, my use of the monotone class lemma below avoids the method in \cite{Stein2005} which relies on Theorem 3.3 (of \cite{Stein2005}) and sweeps some things under the carpet.}
\end{comment}

\begin{proof}[Proof of Theorem \ref{thm:MainIntegrationFormula}]
Denote by $\mathcal{C}$ the collection of sets $G\subseteq (0,\infty)\times S$ for which $\psi_E(G)\in \mathcal{M}_d$ and $m(\psi_E(G))=(\lambda_P\times\sigma_{P,E})(G).$ By virtue of Lemma \ref{lem:AllMeasurableRectangles}, it follows that $\mathcal{C}$ contains all elementary sets, i.e., finite unions of disjoint measurable rectangles. Using the continuity of measure (applied to the measures $m$ and $\lambda_P\times\sigma_{P,E}$) and the fact that $\psi_E$ is a bijection, it is straightforward to verify that $\mathcal{C}$ is a monotone class. By the monotone class lemma (Theorem 8.3 of \cite{Rudin1987}), it immediately follows that $\mathcal{L}\times\Sigma_E\subseteq\mathcal{C}$. In other words, for each $G\in\mathcal{L}\times\Sigma_E$,
\begin{equation}\label{eq:Good1}
\psi(G)\in\mathcal{M}_d\hspace{1cm}\mbox{and}\hspace{1cm}m(\psi(G))=(\lambda_P\times\sigma_{P,E})(G).
\end{equation}
We claim that, for each Borel subset $A$ of $\mathbb{R}^d\setminus\{0\}$, $\psi_E^{-1}(A)\subseteq \mathcal{L}\times\Sigma_E$. To this end, we write
\begin{equation*}
\psi_E(\mathcal{L}\times\Sigma_E)=\{\psi_E(G):G\in\mathcal{L}\times\Sigma_E\}
\end{equation*}
for the $\sigma$-algebra on $\mathbb{R}^d\setminus\{0\}$ induced by $\psi_E$. In view of Lemma \ref{lem:OpenRectangle}, $\psi_E(\mathcal{L}\times\Sigma_E)$ contains every open subset of $\mathbb{R}^d\setminus\{0\}$ and therefore
\begin{equation*}
\mathcal{B}(\mathbb{R}^d\setminus\{0\})\subseteq\psi_E(\mathcal{L}\times\Sigma_E).
\end{equation*}
where $\mathcal{B}(\mathbb{R}^d\setminus\{0\})$ denotes the $\sigma$-algebra of Borel subsets of $\mathbb{R}^d\setminus\{0\}$ thus proving our claim. 

Together, the results of the two preceding paragraphs show that, for each $A\in\mathcal{B}(\mathbb{R}^d\setminus\{0\})$, $\psi_E^{-1}(A)\subseteq \mathcal{L}\times\Sigma_E$ and $m(A)=(\lambda_P\times\sigma_{P,E})(\psi_E^{-1}(A))$. Upon noting that $\mathcal{L}\times\Sigma_E\subseteq (\mathcal{L}\times\Sigma_E)'$, we immediately obtain the following statement: For each $A\in\mathcal{B}(\mathbb{R}^d\setminus\{0\})$,
\begin{equation}\label{eq:Good2}
\psi_E^{-1}(A)\in (\mathcal{L}\times\Sigma_E)'\hspace{1cm}\mbox{and}\hspace{1cm}m(A)=(\lambda_P\times\sigma_{P,E})(\psi_E^{-1}(A)).
\end{equation}
In comparing \eqref{eq:Good1} and \eqref{eq:Good2} with Properties \ref{property:PushforwardLemma1} and \ref{property:PushforwardLemma2} of Lemma \ref{lem:PushforwardLemma} and, upon noting that $((0,\infty)\times S,(\mathcal{L}\times\Sigma_E)',\lambda_P\times\sigma_{P,E})$ is the completion of $((0,\infty)\times S,\mathcal{L}\times\Sigma_E,\lambda_P\times\sigma_{P,E})$ and $(\mathbb{R}^d\setminus\{0\},\mathcal{M}_d,m)$ is the completion of $(\mathbb{R}^d\setminus\{0\},\mathcal{B}(\mathbb{R}^d\setminus\{0\}),m)$, Property \ref{item:MainIntegrationFormula1} of Theorem \ref{thm:MainIntegrationFormula} follows immediately from Lemma \ref{lem:PushforwardLemma}.

It remains to prove Property \ref{item:MainintegrationFormula2}. To this end, let $f:\mathbb{R}^d\to\mathbb{C}$ be Lebesgue measurable. Because $\mathcal{M}_d=\{A\subseteq \mathbb{R}^d\setminus\{0\}:\psi_E^{-1}(A)\in(\mathcal{L}\times\Sigma_E)'\}$, it follows that $f\circ\psi_E$ is $(\mathcal{L}\times\Sigma_E)'$-measurable. In the case that $f\geq 0$, we may approximate $f$ monotonically by simple functions and, by invoking Property \ref{item:MainIntegrationFormula1} and the monotone convergence theorem, we find that
\begin{equation}\label{eq:ChangeofMeasure}
\int_{\mathbb{R}^d}f(x)\,dx=\int_{\mathbb{R}^d\setminus \{0\}}f(x)\,dx=\int_{(0,\infty)\times S}f\circ \psi_E\, d(\lambda_P\times\sigma_{P,E}).
\end{equation}
From this, \eqref{eq:MainIntegrationFormula} follows from Item \ref{item:Fubini1} in Theorem \ref{thm:Fubini}. Finally, by applying the above result to $|f|\geq 0$, we obtain $f\in L^1(\mathbb{R}^d)$ if and only if $f\circ \psi_E\in L^1((0,\infty)\times S,(\mathcal{L}\times\Sigma_E)',\lambda_P\times\sigma_{P,E})$. In this case, by applying \eqref{eq:ChangeofMeasure} to $\Re(f)_+,\Re(f)_-,\Im(f)_+$ and $\Im(f)_-$, we find that \eqref{eq:ChangeofMeasure} holds for our integrable $f$ and, by virtue of Item  \ref{item:Fubini2} of Theorem \ref{thm:Fubini}, the desired result follows.
\end{proof}

\noindent Our next result, Proposition \ref{prop:Regular}, guarantees that, in particular, $\sigma_{F,E}$ is a Radon measure. 

\begin{proposition}\label{prop:Regular}
We have:
\begin{enumerate}
    \item\label{item:Complete} $(S,\Sigma_E,\sigma_{P,E})$ is the completion of the measure space $(S,\mathcal{B}(S),\sigma_{P,E})$. In particular, the $(S,\Sigma_E,\sigma_{P,E})$ is complete and every $F\in \Sigma_E$ is of the form $F=G\cup H$ where $G$ is a Borel set and $H$ is a subset of a Borel set $Z$ with $\sigma_{P,E}(Z)=0$.
\item\label{item:Regular} For each $F\in\Sigma_E$,
\begin{equation}\label{eq:OuterRegular}
\sigma_{P,E}(F)=\inf\{\sigma_{P,E}(\mathcal{O}):F\subseteq\mathcal{O}\subseteq S\mbox{ and $\mathcal{O}$ is open}\}
\end{equation}
and
\begin{equation}
\sigma_{P,E}(F)=\sup\{\sigma_{P,E}(K):K\subseteq F\subseteq S\mbox{ and $K$ is compact}\}.
\end{equation}
\end{enumerate} 
\end{proposition}
\begin{remark}
This proposition can be seen as an application of Proposition \ref{prop:BorelContainment} and Theorem 2.18 of \cite{Rudin1987}. The proof we give here is distinct and, we believe, nicely illustrates the utility of \eqref{eq:MainIntegrationFormula} of Theorem \ref{thm:MainIntegrationFormula}.
\end{remark}
\begin{proof}
Throughout the proof, we shall write $\sigma=\sigma_{P,E}$, $\Sigma=\Sigma_E$ and, for each $F\subseteq S$, $\widetilde{F}=\widetilde{F_E}$. We remark that, by standard arguments using $G_\delta$ and $F_\sigma$ sets, Item \ref{item:Complete} follows immediately from Item \ref{item:Regular}. Also, given that $S$ is compact and $\sigma$ is finite, it suffices to prove \eqref{eq:OuterRegular}, i.e., it suffices to prove the statement: For each $F\in \Sigma$ and $\epsilon>0$, there is an open subset $\mathcal{O}$ of $S$ containing $F$ for which 
\begin{equation*}
\sigma(\mathcal{O}\setminus F)<\epsilon.
\end{equation*}
To this end, let $F\in \Sigma$ and $\epsilon>0$. Given that $\widetilde{F}\in\mathcal{M}_d$ and $m$ is outer regular, there exists an open set $U\subseteq \mathbb{R}^d\setminus\{0\}$ for which $\widetilde{F}\subseteq U$ and $m(U\setminus\widetilde{F})<\epsilon/(2\mu_P)$. Since $\widetilde{F}$ is a subset of the open set $B\setminus\{0\}$, we may assume without loss of generality that $U\subseteq B\setminus\{0\}$ and so $m(\widetilde{F})\leq m(U)<\infty$ and
\begin{equation}\label{eq:LebesgueOuter}
m(U\setminus \widetilde{F})=m(U)-m(\widetilde{F})<\epsilon/(2\mu_P).
\end{equation}
For each $0<t<1$, consider the open set
\begin{equation*}
\mathcal{O}_t=S\cap\left( t^{-E}U\right)
\end{equation*}
in $S$. Observe that, for each $x\in F$, $t^E x\in \widetilde{F}\subseteq U$ and therefore $x\in \mathcal{O}_t$. Hence, for each $0<t<1$, $\mathcal{O}_t$ is an open subset of $S$ containing $F$. 

We claim that there is at least one $t_0\in (0,1)$ for which 
\begin{equation}\label{eq:GoodIneq}
m(\widetilde{\mathcal{O}_{t_0}})< m(U)+\epsilon/(2\mu_P).
\end{equation}
To prove the claim, we shall assume, to reach a contradiction, that 
\begin{equation*}
m(\widetilde{\mathcal{O}_{t}})\geq m(U)+\epsilon/(2\mu_P)
\end{equation*}
for all $0<t<1$. By virtue of \eqref{eq:MainIntegrationFormula} of Theorem \ref{thm:MainIntegrationFormula},
\begin{equation*}
m(U)=\int_{0}^\infty\left(\int_S \chi_{U}(t^E\eta)\,\sigma(d\eta)\right)t^{\mu_P-1}\,dt.
\end{equation*}
Upon noting that $U\subseteq B\setminus\{0\}$, it is easy to see that
\begin{equation*}
U=\bigcup_{0<s<1}s^E\mathcal{O}_s
\hspace{1cm}\mbox{and}\hspace{1cm}
t^E\eta\in \bigcup_{0<s<1}s^E\mathcal{O}_s
\end{equation*}
if and only if $0<t<1$ and $\eta\in \mathcal{O}_t$. Consequently,
\begin{equation*}
m(U)=\int_0^1\left(\int_S\chi_{\mathcal{O}_t}(\eta)\,\sigma(d\eta)\right)\,t^{\mu_P-1}\,dt=\int_0^1\sigma(\mathcal{O}_t)t^{\mu_P-1}\,dt=\int_0^1 \mu_P\cdot m(\widetilde{\mathcal{O}_t})\,t^{\mu_P-1}\,dt.
\end{equation*}
Upon making use of our supposition, we have
\begin{equation*}
\int_0^1\mu_P\cdot m(\widetilde{\mathcal{O}_t})t^{\tr E-1}\,dt\geq \int_0^1\mu_P\cdot (m(U)+\epsilon/(2\mu_P))t^{\mu_P-1}\,dt=m(U)+\epsilon/(2\mu_P)
\end{equation*}
and so
\begin{equation*}
m(U)\geq m(U)+\epsilon/(2\mu_P),
\end{equation*}
which is impossible. Thus, the stated claim is true.

Given any such $t_0$ for which \eqref{eq:GoodIneq} holds, set $\mathcal{O}=\mathcal{O}_{t_0}$. As previously noted, $\mathcal{O}$ is an open subset of $S$ which contains $F$. In view of \eqref{eq:LebesgueOuter} and \eqref{eq:GoodIneq}, we have
\begin{equation*}
m(\widetilde{\mathcal{O}})-m(\widetilde{F})<m(U)-m(\widetilde{F})+\epsilon/(2\mu_P)<\epsilon/(2\mu_P)+\epsilon/(2\mu_P)=\epsilon/\mu_P
\end{equation*}
and therefore
\begin{equation*}
\sigma(\mathcal{O}\setminus F)=\sigma(\mathcal{O})-\sigma(F)=\mu_P(m(\widetilde{\mathcal{O}})-m(\widetilde{F}))<\epsilon,
\end{equation*}
as desired.
\end{proof}


\begin{comment}
\begin{proposition}
The completion of the measure space $(S,\mathcal{B}(S),\sigma_{P,E})$ is $(S,\Sigma_E,\sigma_{P,E})$. In particular, the latter space is complete and every $F\in \Sigma_E$ is of the form $F=G\cup H$ where $G$ is a Borel set and $H$ is a subset of a Borel set $Z$ with $\sigma_{P,E}(Z)=0$.
\end{proposition}


\begin{proof}
For notational simplicity, we shall write $\sigma=\sigma_{P,E}$, $\Sigma=\Sigma_E$ and, for $F\subseteq S$, $\widetilde{F}=\widetilde{F_E}$. Let us denote by $(S,\overline{\mathcal{B}(S)},\overline{\sigma})$ the completion of the measure space $(S,\mathcal{B}(S),\sigma)$. Our job is to show that $\overline{\mathcal{B}(S)}=\Sigma$ and $\overline{\sigma}(F)=\sigma(F)$ for all $F$ in this common $\sigma$-algebra. 

First, let $F\in\overline{\mathcal{B}(S)}$ which is, by definition, a set of the form $F=G\cup H$ where $G\in\mathcal{B}(S)$ with $\overline{\sigma}(F)=\sigma(G)$ and $H\subseteq G_0\in\mathcal{B}(S)$ with $\sigma(G_0)=0$. In view of Proposition \ref{prop:BorelContainment}, $\widetilde{G}\in \mathcal{M}_d$, $\widetilde{H}\subseteq \widetilde{G_0}\in\mathcal{M}_d$ and we have
\begin{equation*}
m(\widetilde{G_0})=\frac{1}{\mu_P}\sigma(G_0)=0
\end{equation*}
Since $(\mathbb{R}^d\setminus\{0\},\mathcal{M}_d,m)$ is complete, we conclude that $\widetilde{H}\in\mathcal{M}_d$ with $m(\widetilde{H})=0$ and therefore $H\in\Sigma$ with $\sigma(H)=(\mu_P)m(\widetilde{H})=0$. It follows that $F=G\cup H\in\Sigma$ and
\begin{equation*}
\overline{\sigma}(F)=\sigma(G)\leq \sigma(F)\leq\sigma(G)+\sigma(H)=\sigma(G)+0=\overline{\sigma}(F).
\end{equation*}
It remains only to prove that $\Sigma\subseteq\overline{\mathcal{B}(S)}$. To this end, let $F\in\Sigma$ be arbitrary but fixed. By appealing to Proposition \ref{prop:Regular}, for each integer $n\in\mathbb{N}$, there exists a compact set $F_n\subseteq F$ for which
\begin{equation*}
\sigma(F\setminus F_n)=\sigma(F)-\sigma(F_n)<1/n.
\end{equation*}
Set
\begin{equation*}
G=\bigcup_{n=1}^\infty F_n\subseteq F
\end{equation*}
and $H=F\setminus G$. 
We observe that $G$ is a Borel set (in fact, an $F_\sigma$ set) and
\begin{equation*}
\sigma(H)=\sigma(F)-\sigma(G)=\sigma(F)-\lim_{n\to\infty}\sigma(F_n)=0,
\end{equation*}
by the continuity of measure. We have shown that
\begin{equation*}
F=G\cup H
\end{equation*}
where $G\in\mathcal{B}(S)$ and $H\in\Sigma$ with $\sigma(H)=0$. It remains to find a Borel set $G_0\supseteq H$ for which $\sigma(G_0)=0$. To this end, we again appeal to Proposition \ref{prop:Regular} to form a collection of open sets $\{\mathcal{O}_n\}_{n=1}^\infty$ such that, for each $n\in\mathbb{N}$, $H\subseteq \mathcal{O}_n$ and $\sigma(\mathcal{O}_n)=\sigma(\mathcal{O}_n)-\sigma(H)<1/n$. Finally, consider
\begin{equation*}
G_0=\bigcap_{n=1}^\infty\mathcal{O}_n,
\end{equation*}
which is necessarily a Borel set (in fact, a $G_\delta$-set) containing $H$ and, by the continuity of measure, has $\sigma(G_0)=0$, as desired.
\end{proof}


\textcolor{blue}{Some questions (I think the third is easiest):
\begin{enumerate}
\item To what extent does $\sigma$ depend on $E$? In other words, if $E,E'\in\Exp(P)$ (which necessarily have $\tr E=\tr E'$ in view of Section 2 of \cite{Randles2017}), would our construction have yielded the same measure $\sigma$ had we instead used the dilation $T_t=t^{E'}$ for everything? They should be closely related if not equal. But I'm really curious about this. For a little background reading, see the material near Proposition 2.3 of \cite{Randles2017}. 
\item We discussed the fact that the surface measure on the (usual) sphere $\mathbb{S}^{d-1}$ was the unique Radon measure which was rotationally invariant and satisfied $\sigma_d(\mathbb{S}^{d-1})=d\cdot m(\mathbb{B})$. I suspect that we also have some characterization for our measure $\sigma$ on $S$ -- in fact, this is why I suspect that $\sigma$ might not depend on the choice of $E$. Here are two possible conjectures and I really don't know if they are true:
\begin{conjecture} For any $O\in\Sym(P)$ and $F\in\Sigma_S$,
\begin{equation}\label{eq:Conjec1}
\sigma(O F)=\sigma(F).
\end{equation} 
\end{conjecture}
\begin{conjecture}
If $\sigma$ does not depend on $E$ (hence the construction produces the same measure $\sigma$ regardless of which $E\in\Exp(P)$ is chosen) and \eqref{eq:Conjec1} holds, $\sigma$ is the unique Radon measure on $S$ which satisfies \eqref{eq:Conjec1} and $\sigma(S)=\tr E m(B)$.
\end{conjecture}
\item There are many polynomials (and positive-definite continuous functions) that have $S$ as their ``unital" level set. For example: For any $\alpha>0$, $Q_{\alpha}(\xi):=(P(\xi))^\alpha$ is continuous and has
\begin{equation*}
S=\{\eta\in\mathbb{R}^d:Q_{\alpha}(\xi)=1\}.
\end{equation*}
Further, $Q_\alpha$ is positive-homogeneous\footnote{and is a polynomial precisely when $\alpha=1,2,\dots$.} and has $\Exp(Q_{\alpha})=\Exp(P)/\alpha$ in the senses that $E_\alpha\in \Exp(Q_\alpha)$ if and only if $E_\alpha=E/\alpha$ for $E\in \Exp(P)$. If you use $E_{\alpha}=E/\alpha$ to construct a measure $\sigma_\alpha$ on $S$ via the above construction, how is $\sigma_\alpha$ related to $\sigma$? I think I have an idea about this, but you should try it. Another question: Are there other polynomials (which are not powers of $P$) that have $S$ as their unital level set? 
\end{enumerate}}


\end{comment}


\subsection{The construction is independent of $E\in\Exp(P)$.}\label{subsec:IndependentofE}

\noindent In this subsection, we show that the Radon measure $\sigma_{P,E}$ is independent of the choice of $E\in\Exp(P)$ and complete the proof of Theorem \ref{thm:BestIntegrationFormula}. To set the stage for our first result, let $E_1,E_2\in\Exp(P)$ and consider the associated (respective) measure spaces $(S,\Sigma_{E_1},\sigma_{P,E_1})$ and $(S,\Sigma_{E_2},\sigma_{P,E_2})$ produced via the construction in Subsection \ref{subsec:ConstructionofSigma}. 

\begin{proposition}\label{prop:Endependence}
These measure spaces are the same, i.e., $\Sigma_{E_1}=\Sigma_{E_2}$ and $\sigma_{P,E_1}=\sigma_{P,E_2}$.
\end{proposition}
\begin{proof}
Throughout the proof, we will write $\Sigma_i=\Sigma_{E_i}$ and $\sigma_i=\sigma_{P,E_i}$ for $i=1,2$. In view of the Proposition \ref{prop:Regular}, it suffices to show that 
\begin{equation*}
\sigma_1(F)=\sigma_2(F)
\end{equation*}
for all $F\in \mathcal{B}(S)\subseteq \Sigma_{1}\cap\Sigma_{2}$. To this end, we let $F\in\mathcal{B}(S)$ be arbitrary but fixed. 

Given $n\in\mathbb{N}$, using the regularity of the measures $\sigma_1$ and $\sigma_2$, select open sets $\mathcal{O}_{n,1},\mathcal{O}_{n,2}$ and compact sets $K_{n,1},K_{n,2}$ for which
\begin{equation*}
K_{n,j}\subseteq F\subseteq \mathcal{O}_{n,j}\hspace{1cm}\mbox{and}\hspace{1cm}\sigma_j(\mathcal{O}_{n,j}\setminus K_{n,j})<1/n
\end{equation*}
for $j=1,2$. Observe that $K_n=K_{n,1}\cup K_{n,2}$ is a compact set, $\mathcal{O}_n=\mathcal{O}_{n,1}\cap\mathcal{O}_{n,2}$ is an open set and $K_n\subseteq F\subseteq \mathcal{O}_n$. Furthemore, 
\begin{equation*}
\sigma_j(\mathcal{O}_n\setminus K_n)\leq \sigma_j(\mathcal{O}_{n,j}\setminus K_{n,j})<1/n
\end{equation*}
for $j=1,2$. Given that $\mathcal{O}_n$ is open in $S$, $\mathcal{O}_n=S\cup U_n$ where $U_n$ is an open subset of $\mathbb{R}^d$ and, because that $S$ is compact, $K_n=K_n\cap S$ is a compact subset of $\mathbb{R}^d$. By virtue of Urysohn's lemma, let $\phi_n:\mathbb{R}^d\to [0,1]$ be a continuous function which is compactly supported in $U_n$ and for which $\phi_n(x)=1$ for all $x\in K_n$. Using this sequence of functions $\{\phi_n\}$, we establish the following useful lemma.

\begin{lemma}\label{lem:IndepProof}
For $j=1,2$ and $n\in\mathbb{N}$, define $g_{n,j}:(0,\infty)\to\mathbb{R}$ by
\begin{equation*}
g_{n,j}(t)=\int_S\phi_n(t^{E_j}\eta)\,\sigma_j(d\eta).
\end{equation*}
for $t>0$. Then $g_{n,j}$ is continuous for each $n\in\mathbb{N}$ and $j=1,2$ and
\begin{equation*}
    \sigma_j(F)=\lim_{n\to\infty}g_{n,j}(1)
\end{equation*}
for $j=1,2$.
\end{lemma}
\begin{subproof}
First, we note that, for each $t\in (0,\infty)$, the above integral makes sense because $\eta\mapsto \phi_n(t^{E_j}\eta)$ is Borel measurable (because it's continuous on $S$) and non-negative. Let $\epsilon>0$ and $t_0\in (0,\infty)$ be arbitrary but fixed. It is clear that the function $(0,\infty)\times S\ni (t,\eta)\mapsto \phi_n(t^{E_j}\eta)$ is continuous on its domain and therefore, in view of the compactness of $S$, we can find a $\delta>0$ for which
\begin{equation*}
|\phi_n(t^{E_j}\eta)-\phi_n(t_0^{E_j}\eta)|\leq\frac{\epsilon}{2\sigma_j(S)}\hspace{1cm}\mbox{whenever}\hspace{1cm}|t-t_0|<\delta
\end{equation*}
for all $\eta\in S$. The triangle inequality guarantees that
\begin{equation*}
|g_{n,j}(t)-g_{n,j}(t_0)|\leq \int_S|\phi_n(t^{E_j}\eta)-\phi_n(t_0^{E_j}\eta)|\,\sigma_j(d \eta)\leq\epsilon/2<\epsilon
\end{equation*}
whenever $|t-t_0|<\delta$. Thus, $g_{n,j}$ is continuous.

We observe that
\begin{equation*}
g_{n,j}(1)=\int_{S}\phi_n(\eta)\,\sigma_j(d\eta)
\end{equation*}
because $1^{E_j}=I$. By construction, we have $\chi_{K_n}(\eta)\leq\phi_{n}(\eta)\leq \chi_{\mathcal{O}_n}(\eta)$ for all $\eta\in S$ and $n\in\mathbb{N}_+$ and therefore
\begin{equation*}
\sigma_j(K_n)\leq g_{n,j}(1)\leq \sigma_{j}(\mathcal{O}_n)
\end{equation*}
by the monotonicity of the integral. Since
\begin{equation*}
\sigma_j(F)=\lim_{n\to\infty}\sigma_j(K_n)=\lim_{n\to\infty}\sigma_j(\mathcal{O}_n)
\end{equation*}
in view of our choice of $\mathcal{O}_n$ and $K_n$, the remaining result follows immediately from the preceding inequality (and the squeeze theorem).
\end{subproof}
\noindent Let us now complete the proof of Proposition \ref{prop:Endependence}. Given any $0<r<1< s$ and $n\in\mathbb{N}$, consider the function $f=f_{n,r,s}:\mathbb{R}^d\to [0,1]$ given by
\begin{equation*}
f(x)=\phi_n(x)\chi_{[r,s]}(P(x))
\end{equation*}
for $x\in\mathbb{R}^d$. It is clear that $f$ is Lebesgue measurable on $\mathbb{R}^d$ and non-negative. By virtue of Theorem \ref{thm:MainIntegrationFormula} (applied to the two measures $\sigma_1$ and $\sigma_2$), we have
\begin{equation}\label{eq:SameMeasure1}
\int_0^\infty \int_Sf(t^{E_1}\eta)\,\sigma_1(d\eta)t^{\mu_P-1}\,dt=\int_{\mathbb{R}^d}f(x)\,dx=\int_0^\infty \int_Sf(t^{E_2}\eta)\,\sigma_2(d\eta)t^{\mu_P-1}\,dt
\end{equation}
Upon noting that
\begin{equation*}
f(t^{E_j}\eta)=\phi_n(t^{E_j}\eta)\chi_{[r,s]}\left((P(t^{E_j}\eta)\right)=\phi_n(t^{E_j}\eta)\chi_{[r,s]}(tP(\eta))=\chi_{[r,s]}(t)\phi_n(t^{E_j}\eta)
\end{equation*}
for $t\in (0,\infty)$, $\eta\in S$, and $j=1,2$, we have
\begin{equation*}
\int_0^\infty\int_S f(t^{E_j}\eta)\,\sigma_j(d\eta)t^{\mu_P-1}\,dt=\int_{[r,s]}\int_S\phi_n(t^{E_j}\eta)\,\sigma_j(d\eta) t^{\mu_P-1}\,dt=\int_{[r,s]}g_{n,j}(t)t^{\mu_P-1}\,dt
\end{equation*}
for $j=1,2$. By virtue of the Lemma \ref{lem:IndepProof}, $t\mapsto g_{n,j}(t)t^{\mu_P-1}$ is continuous and necessarily bounded on $[r,s]$ and so the final integral above can be interpreted as a Riemann integral. In this interpretation, we have
\begin{equation}\label{eq:SameMeasure2}
\int_r^sg_{n,j}(t)t^{\mu_P-1}\,dt=\int_0^\infty \int_S f(t^{E_j}\eta)\,\sigma_j(d\eta)t^{\mu_P-1}\,dt
\end{equation}
for $j=1,2$ and $0<r<1<s$. In view of \eqref{eq:SameMeasure1} and \eqref{eq:SameMeasure2}, we conclude that
\begin{equation*}
\int_r^s g_{n,1}(t)t^{\mu_P-1}\,dt=\int_r^s g_{n,2}(t)t^{\mu_P-1}\,dt
\end{equation*}
for all $0<r<1<s$. In view of continuity of the integrands, an application of the fundamental theorem of calculus now guarantees that $g_{n,1}(1)=g_{n,2}(1)$ for each $n\in\mathbb{N}$. Therefore
\begin{equation*}
\sigma_1(F)=\lim_{n\to\infty}g_{n,1}(1)=\lim_{n\to\infty}g_{n,2}(1)=\sigma_2(F)
\end{equation*}
by virtue of Lemma \ref{lem:IndepProof}.
\end{proof}

\noindent In view of Proposition \ref{prop:Endependence}, we will denote by $\Sigma_P$ and $\sigma_P$ the unique $\sigma$-algebra and measure on $S$ which, respectively, satisfy
\begin{equation*}
    \Sigma_P=\Sigma_E\hspace{1cm}\mbox{and}\hspace{1cm}\sigma_P=\sigma_{P,E}
\end{equation*}
for all $E\in\Exp(P)$. We will henceforth assume this notation. 



\begin{proposition}\label{prop:SymInvariance}
For any $O\in\Sym(P)$ and $F\in\Sigma_P$,
\begin{equation*}
\sigma_P(O F)=\sigma_P(F).
\end{equation*} 
That is, the measure $\sigma_P$ is invariant under the action by $\Sym(P)$. 
\end{proposition}



\begin{proof}
Let $O\in\Sym(P)$, $F\in\Sigma_P$ and, for $E\in \Exp(P)$, define $E'=O^\top EO$. In view of Proposition \ref{prop:ExpP}, we note that $E'\in \Exp(P)$. Observe that
\begin{equation}\label{eq:TildeConjugation}
    \widetilde{(OF)_E}=\bigcup_{0<t<1}t^E (OF)=\bigcup_{0<t<1}O\left(O^\top t^E O F\right)=O\left(\bigcup_{0<t<1} t^{E'}F\right)=O \widetilde{F_{E'}}
\end{equation}
thanks to Proposition \ref{prop:ContinuousGroupProperties}.
In view of Proposition \ref{prop:Endependence}, we have $O\widetilde{F_{E'}}\in \mathcal{M}_d$ because $F\in \Sigma_P=\Sigma_{E'}$ and $O$ is linear. Using \eqref{eq:TildeConjugation}, we find that $\widetilde{(OF)_E}\in\mathcal{M}_d$ and therefore  $OF\in\Sigma_E=\Sigma_P$ by virtue of Proposition \ref{prop:Endependence}. In view of the fact that $O$ is orthogonal,
\begin{equation*}
\sigma_{P,E}(OF)=\mu_P\cdot m\left(\widetilde{(OF)_E}\right)=\mu_P\cdot m\left(O \widetilde{F_{E'}}\right)=\mu_P\cdot m\left(\widetilde{F}_{E'}\right)=\sigma_{P,E'}(F)
\end{equation*}
and therefore, a final appeal to Proposition \ref{prop:Endependence} guarantees that
\begin{equation*}
    \sigma_P(OF)=\sigma_{P,E}(OF)=\sigma_{P,E'}(F)=\sigma_P(F),
\end{equation*}
as desired.
\begin{comment}
Consider the measure spaces constructed from $E$ and $E'$ respectively: $(S, \Sigma_{S,E},\sigma_E)$ and $(S,\Sigma_{S,E'},\sigma_{E'})$. By virtue of Proposition \ref{prop:Endependence}, these measure spaces are the same, i.e., $\Sigma_{S,E} = \Sigma_{S,E'}$ and $\sigma_{E} = \sigma_{E'}$. For convenience, let us call these equivalent measure spaces the triple $(S,\Sigma_S,\sigma)$. 

Let $F\in \Sigma_S$ be given. We will show that $O(F)\in \Sigma_S$. To this end, we first notice that since $O\in \Sym(P) < \OdR$, we have $O O^\top = I$. As a result, we can write
\begin{equation*}
    \widetilde{O(F)} = \bigcup_{0<t<1}t^E O(F) = \bigcup_{0<t<1} O O^\top t^E O(F) = \bigcup_{0<t<1}O t^{O^\top E O} F = O\lp \bigcup_{0<t<1}t^{E'} F\rp.
\end{equation*}
We observe that the set $\bigcup_{0<t<1} t^{E'}F$ is Lebesgue measurable since $F\in \Sigma_S = \Sigma_{S,E'}$. By the orthogonal invariance of the Lebesgue measure, $\widetilde{O(F)}$ is also Lebesgue measurable, with
\begin{equation*}
    m (\widetilde{O(F)} ) = m\lb O \lp \bigcup_{0<t<1}t^{E'}F \rp \rb =  m\lp \bigcup_{0<t<1}t^{E'}F \rp.
\end{equation*}
Therefore, $O(F)\in \Sigma_S$, with which it makes sense to ask for the measure $\sigma$ of $O(F)$:
\begin{equation*}
    \sigma(O(F)) = (\tr E)m(\widetilde{O(F)}).
\end{equation*}
Now, in view of Proposition \ref{prop:Endependence} and the fact that $\tr E = \tr E'$ for $E,E'\in \Exp(P)$, we have that
\begin{equation*}
    m\lp \bigcup_{0<t<1}t^{E'}F \rp = \frac{\sigma_{E'}(F)}{\tr E' }  
    = 
    \frac{\sigma_E(F)}{\tr E'} =  \frac{\sigma_E(F)}{\tr E}   = m\lp \bigcup_{0<t<1} t^E F  \rp.
\end{equation*}
From the last three equations, we conclude that 
\begin{equation*}
    \sigma(O(F)) = \sigma(F).
\end{equation*}
Therefore, the measure $\sigma$ is invariant under symmetry group $\Sym(P)$ of $P$.  
\end{comment}
\end{proof}

\begin{proof}[Proof of Theorem \ref{thm:BestIntegrationFormula}]
Together, the results of Propositions \ref{prop:Regular}, \ref{prop:Endependence} and \ref{prop:SymInvariance}, guarantee that $\sigma_P$ is a Radon measure satisfying Properties \ref{property:Completion} and \ref{property:Invariance}. Property \ref{property:DefiningConditionofsigma} follows directly from Proposition \ref{prop:Endependence} and the definition of $\sigma_P$ in terms of $\sigma_{P,E}$ for any $E\in\Exp(P)$. Similarly, Properties \ref{property:BestPointIsomorphism} and \ref{property:BestIntegrationFormula} follow from Theorem \ref{thm:MainIntegrationFormula} by virtue of Proposition \ref{prop:Endependence}. 
\end{proof}


\section{Using a smooth structure on $S$ to compute $\sigma$.}\label{sec:SigmaForSmoothP}

\noindent In this section, we shall study the special case in which a positive-homogeneous function $P$ on $\mathbb{R}^d$ is smooth\footnote{Many of the results in this section remain valid (with appropriate modification) under the weaker assumption that $P\in C^k(\mathbb{R}^d)$ for $k=1,2,\dots$. In this setting, $S$ is easily seen to be a $C^k$ manifold. Because working in the smooth category is sufficient for our purposes, we shall not pursue the greater level of generality but invite the reader to do so.}\footnote{\textcolor{red}{To avoid trivialities, we assume that $d>1$ throughout.}}.
Under this additional assumption, we shall find that $\grad P$ is everywhere non-vanishing on $S$ and so $S$ is a smooth compact embedded hypersurface in $\mathbb{R}^d$.\\

\noindent We first set up some notation: For a smooth manifold $M$, we shall denote by $\mathcal{A}(M)$ its unique maximal atlas. Also on $M$, the collection of smooth vector fields is denoted by $\mathfrak{X}(M)$ and, for each $k=1,2,\dots$, the set of (smooth) differential $k$-forms on $M$ will be denoted by $\Omega^k(M)$. We shall use Einstein's summation convention without explicit mention.

We view $\mathbb{R}^d$ as smooth oriented Riemannian manifold with its standard Euclidean metric $\overline{g}$, oriented smooth atlas $\mathcal{A}_+(\mathbb{R}^d)$, and Riemannian volume form $ d\Vol_{\mathbb{R}^d}$. Given any $E\in\Exp(P)$, consider $\mathcal{E}_E\in \mathfrak{X}(\mathbb{R}^d)$ defined, at each $x\in\mathbb{R}^d$, by
\begin{equation*}
    (\mathcal{E}_E)_x(f)=\frac{d}{dt}f(x+t(Ex))\big\vert_{t=0}\hspace{1cm}
\end{equation*}
for $f\in C^\infty(\mathbb{R}^d)$. In the standard (global) chart with coordinates $x=(x^\alpha)$, $(\mathcal{E}_E)_{x}\in T_{x}(\mathbb{R}^d)$ is given by
\begin{equation*}
    (\mathcal{E}_E)_{x}=(Ex)^{\alpha}\partial_{x^\alpha}=E^\alpha_\beta x^{\beta}\partial_{x^\alpha}
\end{equation*}
where $(E_\alpha^\beta)$ is the standard matrix representation for $E$ and  $\partial_{x^{\alpha}}=\partial/\partial x^\alpha$. By an abuse of notation, we shall write $\grad P$ to denote both the function
\begin{equation*}
\mathbb{R}^d\ni x\mapsto \grad P(x)=\left(\frac{\partial P}{\partial x^1},\frac{\partial P}{\partial x^2},\dots,\frac{\partial P}{\partial x^d}\right)^{\top}\in\mathbb{R}^d,
\end{equation*}
where $\frac{\partial P}{\partial x^{\alpha}}=\frac{\partial P}{\partial x^{\alpha}}\vert_{x}$ for $\alpha=1,2\dots,d$, and its canonical identification $\grad P\in \mathfrak{X}(\mathbb{R}^d)$ given by \begin{equation*}
    \grad P_x=\overline{g}^{\alpha\beta}\frac{\partial P}{\partial x^{\alpha}}\partial_{x^{\beta}}=\delta^{\alpha\beta}\frac{\partial P}{\partial x^{\alpha}}\partial_{x^{\beta}}=\sum_{\alpha=1}^d\frac{\partial P}{\partial x^{\alpha}}\partial_{x^\alpha}
\end{equation*}
in standard Euclidean coordinates $x=(x^\alpha)$. Of course, for each $x\in\mathbb{R}^d$, the Riemannian norm $|\grad P_x|_\overline{g}$ of $\grad P_x\in T_x(\mathbb{R}^d)$ coincides with the Euclidean norm $|\grad P(x)|$ of $\grad P(x)\in\mathbb{R}^d$. These equivalent quantities (functions) will be henceforth denoted by $|\grad P|$.

\begin{proposition}\label{prop:InnerProdIsOne}
For each $\eta\in S$, 
\begin{equation*}
    \overline{g}(\grad P,\mathcal{E})_{\eta}=\grad P(\eta)\cdot (E\eta)=1.
\end{equation*}
In particular, $\grad P$ (and $dP$) never vanishes on $S$ and so $S$ is a compact embedded hypersurface in $\mathbb{R}^d$.
\end{proposition}
\begin{proof}
Given that $E\in\Exp(P)$ and $P\in C^\infty(\mathbb{R}^d)$, we differentiate the identity $tP(x)=P(t^Ex)$ to find that
\begin{equation*}
    P(x)=\frac{d}{dt}P(t^Ex)=\grad P(t^Ex)\cdot\left(\left(t^{E-I}E\right)x\right)
\end{equation*}
for $t>0$ and $x\in\mathbb{R}^d$. In particular, when $t=1$ and $x=\eta\in S$, we have
\begin{equation*}
    1=\grad P(\eta)\cdot \left(E\eta\right)=\frac{\partial P}{\partial x^\alpha}(E\eta)^\alpha=\overline{g}_{\alpha \beta}\left(\grad P_\eta\right)^{\alpha}(\mathcal{E}_\eta)^\beta=\overline{g}(\grad P,\mathcal{E})_\eta.
\end{equation*}
Thus, our (necessarily) compact level set $S$ is a smooth embedded hypersurface of $\mathbb{R}^d$ in view of the regular level set theorem.
\end{proof}
\noindent We shall denote by $\iota:  S \hookrightarrow \mathbb{R}^d$ the canonical inclusion map and set $d'=d-1$. As an embedded submanifold of $\mathbb{R}^d$, $S$ is a Riemannian submanifold of $\mathbb{R}^d$ with metric $g^S$ given by
\begin{equation*}
    g^S(X,Y)=\overline{g}(\iota_*(X),\iota_*(Y))
\end{equation*}
for $X,Y\in\mathfrak{X}(S)$; here, for each $\eta\in S$,  $\iota_*:T_\eta(S)\to T_\eta(\mathbb{R}^d)$ is the pushforward of $\iota$.  In view of the preceding proposition, $N:=\grad P/|\grad P|\in \mathfrak{X}(\mathbb{R}^d)$ is a smooth unit normal vector field along $S$ and it determines an orientation on the Riemannian manifold $S$. Equipped with this orientation, $(S,g^S)$ is an oriented Riemannian manifold and we shall denote by $d\Vol_S$ Riemannian volume form and by $\mathcal{A}_+(S)$ its corresponding (maximal) oriented atlas. By virtue of Proposition 15.21 of \cite{lee2013smooth}, $d\Vol_S=(N\iprod d\Vol_{\mathbb{R}^d})\vert_S$, i.e.,
\begin{eqnarray*}
    d\Vol_S(X_1,X_2,\dots,X_{d'})&=&d\Vol_{\mathbb{R}^d}(N,\iota_*(X_1),\iota_*(X_2),\dots,\iota_*(X_{d'}))\\
    &=&\frac{1}{|\grad P|}d\Vol_{\mathbb{R}^d}(\grad P,\iota_*(X_1),\iota_*(X_2),\dots,\iota_*(X_{d'}))
\end{eqnarray*}
for any collection $\{X_1,X_2,\dots,X_{d'}\}\in \mathfrak{X}(S)$. Beyond $d\Vol_S\in \Omega^{d'}(S)$, we consider the following smooth $d'$-form(s): Given $E\in \Exp(P)$, define $d\sigma_{P,E}\in\Omega^{d'}(S)$ by
\begin{equation*}
    d\sigma_{P,E}(X_1,X_2,\dots,X_{d'})=d\Vol_{\mathbb{R}^d}(\mathcal{E}_E,\iota_*(X_1),\iota_*(X_2),\dots,\iota_*(X_{d'}))
\end{equation*}
for $X_1,X_2,\dots,X_{d'}\in\mathfrak{X}(S)$. We have
\begin{proposition}\label{prop:FormRiemannRelation}
For any $E\in\Exp(P)$,
\begin{equation*}
    d\sigma_{P,E}=\frac{1}{|\grad P|}d\Vol_S.
\end{equation*}
In particular, $d\sigma_{P,E}$ is positively oriented and is independent of $E\in\Exp(P)$.
\end{proposition}
\noindent Before proving the proposition, we first treat a lemma of a purely linear algebraic nature. 
\begin{lemma}\label{lem:determinants}
Let $v_1,v_2,\dots,v_{d'}$ be linearly independent vectors in $\mathbb{R}^d$ and suppose that $w\in\mathbb{R}^d \setminus\{0\}$ is such that $w\perp v_i$ for all $i$. Then, for any $z\in\mathbb{R}^d$ for which $z\cdot w=1$,
\begin{equation*}
\det(z, v_1,v_2,\dots,v_{d'})=\frac{1}{|w|}\det(n,v_1,v_2,\dots,v_{d'})=\frac{1}{|w|^2}\det(w,v_1,v_2,\dots,v_{d'}).
\end{equation*}
where $n:=w/|w|$.
\end{lemma}

\begin{proof}
Given $z\in\mathbb{R}^d$ such that $z\cdot w=1$, it follows that 
\begin{equation*}
z=\frac{1}{|w|}n+a_1v_1+a_2v_2+\cdots a_{d'}v_{d'}.
\end{equation*}
By the multilinearity of the determinant map, we have
\begin{eqnarray*}
\det(z,v_1,v_2,\dots,v_{d'}) &=&\det\lp \frac{1}{|w|}n+a_1v_2+a_2v_2+\cdots a_{d'} v_{d'},v_1,v_2,\dots,v_{d'}\rp\\
    &=&\frac{1}{|w|}\det(n,v_1,v_2,\dots,v_{d'})+\det(a_1v_1+\cdots+a_{d'} v_{d'}, v_1,v_2,\dots,v_{d'})\\
&=&\frac{1}{|w|}\det(n, v_1,v_2,\dots,v_{d'})+0
\end{eqnarray*}
where we have used the fact that the columns of the matrix $(a_1v_1+\cdots+a_{d'} v_{d'}, v_1,v_2,\dots,v_{d'})$ are linearly dependent to conclude that the final determinant is zero.
\end{proof}
\begin{proof}[Proof of Proposition \ref{prop:FormRiemannRelation}]
We fix $E\in\Exp(P)$ and note that the assertion at hand is a local one. Thus, it suffices to verify that, for any $\eta\in S$ and $X_1,X_2,\dots,X_{d'}\in T_\eta(S)$, 
\begin{equation*}
    d\sigma_{P,E}(X_1,X_2,\dots,X_{d'})=\frac{1}{|\grad P_{\eta}|}d\Vol_{\mathbb{R}^d}(N_\eta,\iota_*(X_1),\iota_*(X_2),\dots,\iota_*(X_{d'})).
\end{equation*}
Fix $\eta\in S$ and let $(\mathcal{O},\varphi)$ be a coordinate chart centered at $\eta$ with local coordinates $u=(u^{\alpha})$. As usual, denote by $x=(x^{\alpha})$ the Euclidean coordinates on $\mathbb{R}^d$.  For each $i=1,2,\dots,{d'}$, \begin{equation*}
X_i=X_i^\alpha \partial_{u^{\alpha}}\hspace{1cm}\mbox{and}\hspace{1cm}\iota_*(X_i)=v_i^\beta\partial_{x^{\beta}}
\end{equation*}
where
\begin{equation*}
v_i^\beta =X_i^\alpha\frac{\partial x^\beta}{\partial u^\alpha}.
\end{equation*}
For each $i=1,2,\dots,d'$, we set $v_i=(v_i^1,v_i^2,\dots,v_i^{d'})^\top\in\mathbb{R}^d$. Also, let $w=\grad P(\eta)\in\mathbb{R}^d$ with $|w|=|\grad P(\eta)|=|\grad P_\eta|$, set $n=w/|w|\in\mathbb{R}^d$ and note that
\begin{equation*}
    N_\eta=\frac{1}{|\grad P_\eta|}\sum_{k=1}^d\frac{\partial P}{\partial x^{k}}\partial_{x^k}=n^{\mu}\partial_{x^\mu}.
\end{equation*}Given that $\grad P$ is normal to $S$, we have
\begin{equation*}
    v_i\cdot w=\overline{g}_{\mu,\nu}v_i^\mu w^\nu= \overline{g}(\iota_*(X_i),\grad P)_\eta=0
\end{equation*}
and hence $w\perp v_i$ for each $i=1,2,\dots,{d'}$. Recalling that $(\mathcal{E}_E)_\eta=(E\eta)^\alpha\partial_{x^{\alpha}}$, set $z=((E\eta)^1,(E\eta)^2,\dots,(E\eta)^d)^\top\in\mathbb{R}^d$ and observe that $z\cdot w=1$ by virtue of Proposition \ref{prop:InnerProdIsOne}. An appeal to the lemma guarantees that
\begin{eqnarray*}
d\sigma_{P,E}(X_1,X_2,\dots,X_{d'})&=&d\Vol_{\mathbb{R}^d}(\mathcal{E}_E,\iota_*(X_1),\iota_*(X_2),\dots,\iota_*(X_{d'}))\\
&=&\det(z,v_1,v_2,\dots,v_{d'})\\
&=&\frac{1}{|w|}\det(n,v_1,v_2,\dots,v_{d'})\\
&=&\frac{1}{|\grad P_\eta|}d\Vol_{\mathbb{R}^d}(N_\eta,\iota_*(X_1),\iota_*(X_2),\dots,\iota_*(X_{d'}))\\
&=&\frac{1}{|\grad P|}d\Vol_S(X_1,X_2,\dots,X_{d'}).
\end{eqnarray*}
\end{proof}
By virtue of the preceding proposition, we shall denote by $d\sigma_P$ the unique smooth $d'$-form on $S$ which satisfies
\begin{equation}\label{eq:sigmaForm}
    d\sigma_P=d\sigma_{P,E}=\frac{1}{|\grad P|}d\Vol_S
\end{equation}
for all $E\in\Exp(P)$. In this notation, we have this section's central result. \textcolor{red}{A note is needed about following the notation in \cite{Amann2009}}




\begin{theorem}\label{thm:RiemannLebesgue}
Let $P$ be a smooth positive-homogeneous function and let $S=\{\eta\in\mathbb{R}^d:P(\eta)=1\}$. Then $S$ is a compact smooth embedded hypersurface of $\mathbb{R}^d$. Viewing $\mathbb{R}^d$ as an oriented Riemannian manifold with its usual orientation and metric $\overline{g}$, $N=\grad P/|\grad P|$ is a smooth unit normal vector field along $S$. As a submanifold of $\mathbb{R}^d$, $S$ is a oriented Riemannian manifold of dimension $d'=d-1$ with its induced Riemannian metric $g^S$, volume form $d\Vol_S\in\Omega^{d'}(S)$ and orientation determined by $N$. The smooth form $d\sigma_P\in\Omega^{d'}(S)$, defined by \eqref{eq:sigmaForm}, coincides with the measure $\sigma_P$ in the sense that
\begin{equation*}
\int_S g(\eta)\,\sigma_P(d\eta)=\int_S g\,d\sigma_P
\end{equation*}
for all $g\in C^0(S)$ \textcolor{red}{This is actually much more general}. Furthermore, the measure $d\sigma_P$ and the canonical Riemannian volume measure $\Vol_S$ on $S$ are mutually absolutely continuous.
\end{theorem} 

\begin{proof}

Consider a chart $(\mathcal{U},\varphi) \in \mathcal{A}(S)$. Let $U = \varphi(\mathcal{U}) \subseteq \R^{d'}$, and $V = (0, 1) \times U\in \mathbb{R}^{1+d'}=\mathbb{R}^d$. Define the map $\rho_E : V \to B\setminus \{ 0 \}\subseteq \mathbb{R}^d$ by 
\begin{equation*}
    \rho_{E} (t,u) = \psi_E(t,\varphi^{-1}(u)) = t^E \varphi^{-1}(u)
\end{equation*}
for $(t,u)\in V$; this is necessarily smooth because $\varphi^{-1}$ is smooth. Also, define $\widetilde{\varphi_E}:\rho_E(V) \to V$ by
\begin{equation*}
    \widetilde{\varphi_E}(x) = (P(x), \varphi(P(x)^{-E} x))
\end{equation*}
for $x\in\rho_E(V)$. By virtually the same arguments given in the proof of Proposition \ref{prop:PsiHomeomorphism}, we find that $\widetilde{\varphi_E}$ is invertible with $\widetilde{\varphi_E}^{-1}=\rho_E$ and, since $P\in C^\infty(\mathbb{R}^d)$, we conclude that $\widetilde{\varphi_E}$ is a diffeomorphism. 

Given $F\in\Sigma_P$ for which $F\subseteq\mathcal{U}$, it is straightforward to verify that
\begin{equation*}
    \widetilde{F_E}=\rho_E\left((0,1)\times\varphi_E(F)\right).
\end{equation*}
Since $\widetilde{F_E}$ is Lebesgue measurable, an appeal to Theorem 15.11 of \cite{Apostol1974} guarantees that
\begin{equation}\label{eq:RiemannLebesgue1}
    V\ni y\mapsto \chi_{\widetilde{F_E}}(\rho_E(y))|\det(D\rho_E)(y)|
\end{equation}
is a Lebesgue measureable function on $\mathbb{R}^d$ and
\begin{equation*}
    m(\widetilde{F_E})=\int_{\mathcal{U}}\chi_{\widetilde{F_E}}\,dx=\int_{V}\chi_{\widetilde{F_E}}(\rho_E(y))|\det(D\rho_E)(y)|\,dy
\end{equation*}
where $D\rho_E$ is the Jacobian matrix of $\rho_E$. Our immediate goal to simplify the integral above. To this end, we will break down the integrand and use Fubini's theorem to conclude that $\varphi(F)\subseteq \mathbb{R}^{d'}$ is Lebesgue measurable and write the integral as an iterated integral with $dy=du\,dt$.

Firstly, it is straightforward to verify that $y=(t,u)\in (0,1)\times \varphi_E(F)$ if and only if $\rho_E(t,u)\in\widetilde{F_E}$. From this we find that
\begin{equation}\label{eq:RiemannLebesgue2}
    \chi_{\widetilde{F_E}}(\rho_E(y))=\chi_{(0,1)}(t)\chi_{\varphi(F)}(u)=\chi_{\varphi(F)}(u)
\end{equation}
for all $y=(t,u)\in V$. Now, for $y=(t,u)\in V$, observe that
\begin{eqnarray*}
D\rho_E(y)&=&\left.\left(\frac{d}{dt}(t^E\varphi^{-1}(u)) \right\vert D_u\left[t^E\varphi^{-1}(u)\right]\right)\\
&=&\left.\left(t^{E-I}E\varphi^{-1}(u)\right\vert t^E D_u\varphi^{-1}(u)\right)\\
&=&t^E\left.\left(\frac{1}{t}E\varphi^{-1}(u)\right\vert D_u\varphi^{-1}(u)\right)
\end{eqnarray*}
where the vertical bar separates the first column in $D\rho_E$ from the rightmost $d\times d'$ submatrix and $D_u$ denotes the Jacobian in the coordinates $u=(u^1,u^2,\dots,u^{d'})\in\mathbb{R}^{d'}$. Using properties of the determinant, we find that
\begin{eqnarray}\label{eq:RiemannLebesgue3}\nonumber
    \det(D\rho_E(y))&=&\det(t^E)\det\left.\left(\frac{1}{t}E\varphi^{-1}(u)\right\vert D_u\varphi^{-1}(u)\right)\\\nonumber
    &=&t^{\tr E}t^{-1}\det\left.\left(E\varphi^{-1}(u)\right\vert D_u\varphi^{-1}(u)\right)\\
    &=&t^{\mu_P-1}h(u)
\end{eqnarray}
for all $y=(t,u)\in V$ where $h(u):=\det\left.\left(E\varphi^{-1}(u)\right\vert D_u\varphi^{-1}(u)\right)$. In view of Proposition \ref{prop:FormRiemannRelation} (and \eqref{eq:sigmaForm}), we find that
\begin{eqnarray}\label{eq:RiemannLebesgue4}\nonumber
    h(u)&=&
    \det\left.\left(E\varphi^{-1}(u)\right\vert D_u\varphi^{-1}(u)\right)\\ \nonumber
    &=&
    (d\Vol_{\mathbb{R}^d})_{\varphi^{-1}(u)}(\mathcal{E}_E,\iota_*(\partial_{u^1}),\iota_*(\partial_{u^2}),\dots,\iota_*(\partial_{u^{d'}}))\\ \nonumber
    &=&\left(d\sigma_P\right)_{\varphi^{-1}(u)}(\partial_{u^1},\partial_{u^2},\dots,\partial_{u^{d'}})\\
    &=&\frac{1}{|\grad P(\varphi^{-1}(u))|} (d\Vol_S)_{\varphi^{-1}(u)} (\partial_{u^1}, \partial_{u^2},\dots,\partial_{u^d})
\end{eqnarray}
for $u\in U$. Given that $\{\partial_{u^1},\partial_{u^2},\dots,\partial_{u^{d'}}\}$ is a basis of $T_{\varphi^{-1}(u)}(S)$ and $|\grad P|$ is continuous (and non-zero) on $S$, \eqref{eq:RiemannLebesgue4} ensures that $h$ is non-zero and continuous (and smooth) on $U$. By combining \eqref{eq:RiemannLebesgue1}, \eqref{eq:RiemannLebesgue2} and \eqref{eq:RiemannLebesgue3}, we may therefore conclude that
\begin{equation*}  y=(t,u)\mapsto\chi_{\varphi(F)}(u)=\frac{\chi_{\widetilde{F_E}}(\rho(y))|\det(D_{\rho_E})(y)|}{t^{\mu_P-1}|h(u)|}
\end{equation*}
is a Lebesgue measurable function on $V\subseteq\mathbb{R}^d$. Let us now view $\mathbb{R}^d=\mathbb{R}\times\mathbb{R}^{d'}$ the Lebesgue measure $dy$ as the completion of the product measure $dt\times du$. By virtue of Fubini's theorem, for $dt$-almost every $t\in (0,1)$, the $t$-section of $V\ni(t,u)\mapsto\chi_{\varphi(F)}(u)$ is Lebesgue measurable on $U$. Of course, this ensures that $U\ni u\mapsto \chi_{\varphi(F)}(u)$ is a Lebesgue measurable function (on $U\subseteq\mathbb{R}^{d'}$) and therefore $\varphi(F)$ is Lebesgue measurable. With this, we are now able to compute
\begin{eqnarray*}
    m(\widetilde{F_E})
    &=&\int_{V}\chi_{\widetilde{F_E}}(\rho_E(y))|\det(D\rho_E)(y)|\,dy\\
    &=&\int_{(0,1)}\int_U \chi_{\varphi(F)}(u)t^{\mu_P-1}|h(u)|\,du\,dt\\
    &=&\left(\int_{(0,1)}t^{\mu_P-1}\,dt\right)\left(\int_{\varphi(F)}|h(u)|\,du\right)\\
    &=&\frac{1}{\mu_P}\int_{\varphi(F)}|h(u)|\,du\\
\end{eqnarray*}
and therefore
\begin{equation*}
    \sigma_P(F)=\int_{\varphi(F)}|h(u)|\,du. 
\end{equation*}
Upon noting that $h(u)=(d\sigma_P)_{\varphi^{-1}}(\partial_{u^1},\partial_{u^2},\dots,\partial_{u^{d'}})>0$ provided that $(\mathcal{U},\varphi)\in\mathcal{A}_+(S)$, we have proved the following statement: Given any $F\in\Sigma_P$ and chart $(\mathcal{U},\varphi)\in\mathcal{A}_+(F)$ for which $F\subseteq\mathcal{U}$, $\varphi(F)$ is a Lebesgue measurable subset of $\mathbb{R}^{d'}$, and
\begin{equation}\label{eq:RiemannLebesgue5}
    \sigma_P(F)=\int_{\varphi(F)}\varphi_*(d\sigma_P)
\end{equation}
The above identity guarantees, in particular, that 
\begin{equation*}
    \int_{\varphi(F)}\varphi_*(d\sigma_P)=\int_{\varphi'(F)}\varphi_*'(d\sigma_P)
\end{equation*}
for any $(\mathcal{U},\varphi),(\mathcal{U}',\varphi')\in\mathcal{A}_+(S)$ for which $F\subseteq \mathcal{U}\cap\mathcal{U}'$ (Another argument for this fact can be found in the proof of Lemma 1.3 of). 

Let $f=\sum_{j}c_j\chi_{F_j}$ be a simple function on $(S,\Sigma_P)$ which is supported in the domain of some chart $(\mathcal{U},\varphi)\in\mathcal{A}_+(S)$. By virtue of \eqref{eq:RiemannLebesgue5}, it follows that $\varphi_*(f)=f\circ\varphi^{-1}$ is Lebesgue measurable on $\mathbb{R}^{d'}$ and hence $f$ is $\mathcal{L}(S)$. We have
\begin{eqnarray*}
    \int_S f\,d\sigma_P&=&\int_S \frac{f}{|\grad P|}d\Vol_S\\
    &=&\int_U \varphi_*(f/\abs{\grad P})\varphi_*(d\Vol_S)\\
    &=&\int_U\varphi_*(f)\varphi^*(d\sigma_P)\\
    &=&\int_U\left(\sum_j c_j\chi_{F_j}(\varphi^{-1}(u))\right)\varphi_*(d\sigma_P)\\
    &=&\sum_jc_j\int_{\varphi(F_j)}\varphi_*(d\sigma_P)\\
    &=&\sum_jc_j\sigma_P(F_j)\\
    &=&\int_S f(\eta)\sigma_P(d\eta)
\end{eqnarray*}
where the first equality is the definition of integration with respect to the $d'$-form $f\,d\sigma_P$ (c.f., (2.1) on pg. 407 of \cite{Amann2009}), the second equality is by virtue of Theorem 1.10 of \cite{Amann2009} and the penultimate ultimate equality is by virtue of \eqref{eq:RiemannLebesgue5}. Now, let $f\in L^1(S,\Sigma_P,\sigma_P)$ be non-negative and, by virtue of the fact that $S$ is compact, let $f_n$ be a sequence of simple functions which converges uniformly to $f$. We have 

So, we have the statement that, for any $F\in\Sigma_P$ in the domain of some chart $\varphi(F)$ is Lebesgue measurable and
\begin{equation*}
    \sigma_P(F)=\int_{\varphi(F)}\varphi_*(d\sigma_P).
\end{equation*}
By linearity we find that, for any simple function $g$ (from $(S,\Sigma_P,\sigma_P)$ and any chart $(\mathcal{U},\varphi)$ for which $\supp(g)\subseteq \mathcal{U}$, $\varphi_*(g)$ is Lebesgue measurable and
\begin{equation*}
  \int_S g(\eta)\,\sigma_P(d\eta)=\int_{U}\varphi_*(g\cdot\,d\sigma_P).  
\end{equation*}
Let us define $\Gamma:C^\infty(S)\to\mathbb{R}$ by
\begin{equation*}
   \Gamma(g)= \int_S g(\eta)\sigma_P(d\eta)
\end{equation*}
for $g\in C^\infty(S)$. We aim to use the uniqueness to show that
\begin{equation*}
    \int_S g(\eta)\sigma_P(d\eta)=\Gamma(g)=\int_S (g\cdot\,d\sigma_P)
\end{equation*}
for all $g\in C^0(S)$ where the right-hand-side is the integral (defined via forms, i.e., Proposition 9.2 of the UPenn notes). We begin with a lemma:
\begin{lemma}
Let $g\in C^\infty(S)$ be non-negative and whose support is compactly supported in the domain of a chart and let and $g_n$ be a sequence of simple functions for which $g_n\nearrow g$ uniformly (which exists by a sharp form of the simple approximation lemma), then for any $(\mathcal{U},\varphi)\in\mathcal{A}_+(S)$ for which $\supp(g)\subseteq \mathcal{U}$,
\begin{equation*}
    \int_S(g\cdot \,d\sigma_P)=\lim_{n\to\infty}\int_{\varphi(\mathcal{U})}\varphi_*(g_n\sigma_P).
\end{equation*}
\end{lemma}
\begin{proof}
Let $\epsilon>0$ and let $(\mathcal{U},\varphi)$ be a chart for which $\supp(g)\subseteq\mathcal{U}$. Given that $h(u)$ is a smooth function on $U$, is is bounded on the compact set $G=\varphi(\supp(g))\subseteq U$. With this in mind, set
\begin{equation*}
    M=\sup_{u\in \varphi(\supp(g))}|h(u)|,
\end{equation*}
$K$ be the $d'$-dimensional Lebesgue measure of $G$ and let $N$ be a natural number for which
\begin{equation*}
    |g(\eta)-g_n(\eta)|<\frac{\epsilon}{M m_{d'}(\varphi(\supp(g)))}
\end{equation*}
for all $\eta\in S$ and $n\geq N$ (here we are using the fact that $\supp(g_n)\subseteq\supp(g)$ for all $n$). Then, for any $n\geq N$,
\begin{eqnarray*}
\left|\int_{S}g\cdot\,d\sigma_P-\int_{\varphi(\mathcal{U})}\varphi_*(g_n\cdot\,d\sigma_P)\right|&=&
\left|\int_{\varphi(\mathcal{U})}\varphi_*(g\cdot\,d\sigma_P)-\int_{\varphi(\mathcal{U})}\varphi_*(g_n\cdot\,d\sigma_P)\right|\\
&\leq & \left|\int_{\varphi(\mathcal{U})}\varphi_*(g-g_n)\cdot\,\varphi_*(d\sigma_P)\right|\\
&=&\left|\int_{\varphi(\supp(g))}\varphi_*(g-g_n)|h(u)|\,du\right|\\
&\leq& \int_{G}|\varphi_*(g-g_n)||h(u)|\,du\\
&<& \epsilon.
\end{eqnarray*}
\end{proof}
Okay, so given any $g\in C^\infty(S)$ which is compactly supported in the domain of a chart, we have: For any chart $(\mathcal{U},\varphi)\in\mathcal{A}_+(S)$ for which $\supp(g)\subseteq \mathcal{U}$,
\begin{equation*}
    \int g(\eta)\sigma_P(d\eta)=\lim_{n\to\infty}\int_S g_n(\eta)\sigma_P(d\eta)=\lim_{n\to\infty}\int_{\varphi(\mathcal{U})}\varphi_*(g_n\cdot\,d\sigma_P)=\int_{\varphi(\mathcal{U})}\varphi_*(g\cdot d\sigma_P)=\int_S (g\cdot d\sigma_P).
\end{equation*}
We have therefore shown that, for any $g\in C^\infty(S)$ whose domain is compactly supported in the domain of some chart,
\begin{equation*}
    \int_S g(\eta)\sigma_P(d\eta)=\int_{\varphi(\mathcal{U})}\varphi_*(g\cdot d\sigma_P)=\int_S g\cdot d\sigma_P
\end{equation*}
for any $(\mathcal{U},\varphi)\in\mathcal{A}_+(S)$ where $\supp(g)\subseteq \mathcal{U}$. Of course, 



Finally, by virtue of the fact that $S$ is a Lindel\"{o}f space, let $\{(\mathcal{U}_j,\varphi_j)\}\subseteq\mathcal{A}_+(S)$ be a countable atlas and let $\{k_j\}\subseteq C^\infty(S)$ be a smooth partition of unity subordinate to the cover $\{\mathcal{U}_j\}$. Given $F\in \Sigma_P$, the above argument guarantees that $\varphi_j(F\cap\mathcal{U}_j)$ is a Lebesgue measurable subset of $\mathbb{R}^{d'}$ and
\begin{eqnarray*}
    \sigma_P(F)&=&\int_S \chi_F(\eta)\sigma_P(d\eta)\\
    &=&\int_S \left(\sum_{j=1}^\infty k_j(\eta)\chi_F(\eta)\right)\sigma_P(d\eta)\\
    &=&
\end{eqnarray*}











Now, we claim that there exist positive constants $C_1, C_2$ such that for all $\eta\in S$, 
\begin{equation*}
    C_1 \leq \abs{\grad P(\eta)} \leq C_2.
\end{equation*}
\begin{subproof}[Subproof]
Observe that since $\grad P(\eta) \cdot E\eta = 1$ for all $\eta\in S$, $\grad P$ never vanishes on $S$. Thus, $\abs{\grad P} > 0$. Further, since $P$ is smooth, $\abs{\grad P} = \grad P \cdot \grad P$ is also smooth, hence continuous. By the extreme value theorem and the fact that $S$ is compact, $\abs{\grad P(S) }$ is a compact (hence closed and bounded) subset of $\R_+ = \{ x \in \R : x > 0 \}$ and two positive constants $C_1,C_2$ for which $C_1 \leq \abs{\grad P} \leq C_2$ are
\begin{equation*}
    C_1 = \inf_{\eta\in S}\abs{\grad P(\eta) }, \quad C_2 = \sup_{\eta\in S}\abs{\grad P(\eta) }.
\end{equation*}
\end{subproof}

With these results, we are ready to prove the last item in the theorem. Let the positive constants $C_1,C_2$ be defined as above and let $F\in \mathcal{B}(S)$. If $\Vol_S(F) = 0$, then we have that
\begin{equation*}
    0 = \Vol_S(F) = \int_S \chi_F \,d\Vol_S = \int_S \chi_F \abs{\grad P }\,d\sigma \geq C_1 \int_S \chi_F \,d\sigma = C_1 \sigma(F) \geq 0,
\end{equation*}
which implies that $\sigma(F) = 0$. Thus, $\sigma \ll \Vol_S$. On the other hand, if $\sigma(F) = 0$, then
\begin{equation*}
    0 = \sigma(F) = \int_S \chi_F \,d\sigma= \int_S \frac{\chi_F }{ \abs{\grad P } }\,d\Vol_S \geq \f{1}{C_2} \int_S \chi_F \,d\Vol_S = \f{1}{C_2} \Vol_S(F) \geq 0,
\end{equation*}
which implies that $\Vol_S(F) = 0$. Thus, $\Vol_S \ll \sigma$. We therefore conclude that $\Vol_S$ and $\sigma$ are mutually absolutely continuous.










\end{proof}














% \begin{lemma}
% For each $F\in\Sigma_S$ which is in the domain of some chart $(\mathcal{O}_\alpha,\varphi_\alpha)$, 
% \begin{equation*}
% \mathbf{x}\mapsto \chi_{\varphi_\alpha(F)}|\det(\Omega_\alpha (\mathbf{x}))|
% \end{equation*}
% is Lebesgue measurable on $\mathbb{R}^{d-1}$ and
% \begin{equation*}
% \sigma(F)=\int_{\varphi_\alpha(F)}|\det(\Omega_\alpha (\mathbf{x}))|\,dx.
% \end{equation*}
% \end{lemma}
% \begin{proof}
% Given $F\in\Sigma_S$ for which $F\subseteq\mathcal{O}_\alpha$, we invoke the outer regularity of $\sigma$ to produce a sequence of nested open sets $\{\mathcal{U}_n\}$ for which
% \begin{equation*}
% F\subseteq \mathcal{U}_{n+1}\subseteq\mathcal{U}_n\subseteq\mathcal{O}_\alpha
% \end{equation*}
% for all $n\in\mathbb{N}_+$ and for which
% \begin{equation*}
% \sigma(F)=\lim_{n\to\infty}\sigma(\mathcal{U}_n).
% \end{equation*}
% For each $n\in\mathbb{N}_+$, set
% \begin{equation*}
% f_n(\mathbf{x})=\chi_{\varphi_\alpha(\mathcal{U}_n)}(\mathbf{x})|\det(\Omega_\alpha(\mathbf{x}))|
% \end{equation*}
% for $\mathbf{x}\in\mathbb{R}^{d-1}$ \textcolor{red}{This has got to be Borel measurable}. By virtue of the computations preceding the lemma, we have
% \begin{equation*}
% \sigma(\mathcal{U}_n)=\int_{\varphi_\alpha(\mathcal{U}_n)}|\det(\Omega_\alpha(\mathbf{x}))|\,dx=\int_{\mathbb{R}^{d-1}}f_n(\mathbf{x})\,dx
% \end{equation*}
% for each $n\in\mathbb{N}_+$. Given that the sequence $\mathcal{U}_n$ is nested and decreasing, it follows that $\{f_n\}$ is a non-negative decreasing sequence of of measurable functions and consequently
% \begin{equation*}
% f(\mathbf{x})=\inf_{n\to\infty}f_n(\mathbf{x})
% \end{equation*}
% exits for each $\mathbf{x}\in\mathbb{R}^{d-1}$ and is Lebesgue measurable. Further, by the monotone convergence theorem, we have
% \begin{equation*}
% \sigma(F)=\lim_{n\to\infty}\sigma(\mathcal{U}_n)=\lim_{n\to\infty}\int_{\mathbb{R}^{d-1}}f_n(\mathbf{x})\,dx=\int_{\mathbb{R}^{d-1}}f(\mathbf{x})\,dx.
% \end{equation*}
% We then see that the lemma will be proved if we can show that
% \begin{equation*}
% f(\mathbf{x})=\chi_{\varphi_{\alpha}(F)}(\mathbf{x})|\det(\Omega_\alpha(\mathbf{x}))|
% \end{equation*}
% almost everywhere \textcolor{blue}{with respect to the $d-1$ Lebesgue measure}. \textcolor{red}{ahhhh -- I think we need inner regularity to prove this}
% \end{proof}



% \begin{theorem}\label{thm:IntegrationFormulaWithSigma}
% Let $\mathcal{A}=\{\mathcal{O}_\alpha,\varphi_\alpha\}$ be an atlas on $S$ and let $\{k_\alpha\}$ be a $C^\infty$ partition of unity subordinate to the cover $\{\mathcal{O}_\alpha\}$. Given a $\Sigma_S$-measurable function $g$, assume that $g\geq 0$ or $g\in L^1(S,\sigma)$.
% \begin{enumerate}
% \item If $\supp(g)\subseteq \mathcal{O}_\alpha$ for some $\alpha$, then
% \begin{equation*}
% \int_S g(\eta)\,\sigma(d\eta)=\int_{\mathcal{O}_\alpha}(g\circ\varphi_{\alpha}^{-1})(\mathbf{u})|\det(\Omega_\alpha(\mathbf{u}))|\,du^1du^2\dots du^{d-1}
% \end{equation*}
% \item In general,
% \begin{equation*}
% \int_S g(\eta)\,\sigma(d\eta)=\sum_{\alpha}\int_{\mathcal{O}_\alpha}(g\circ \varphi_\alpha^{-1})(\mathbf{u})(k_\alpha\circ\varphi_{\alpha}^{-1})(\mathbf{u})|\det(\Omega_\alpha(\mathbf{u}))|\,du^1du^2\dots du^{d-1}
% \end{equation*}
% \end{enumerate}
% \end{theorem}
% \begin{proof}
% Thoughts: First, I don't think the stuff done before the proof can be done for arbitrary measurable sets. So, we should do it for open sets and then use the regularity of $\sigma$ to get the desired statement. Maybe this should be a lemma. 

% Now, think about some statement that says: Given a chart $(\mathcal{O}_\alpha,\varphi_\alpha)$ can we use the above formula (and perhaps the fact that it shows absolute continuity with respect to Lebesgue measure to show that if $F\in \Sigma_S$ then $U_\alpha=\varphi_{\alpha}(F)$ is Lebesgue measurable?

% Okay, now that that's done: Assume that $g\geq 0$. Assume first that $g$ is supported on the neighborhood of a chart. Given that $g$ is $\Sigma_S$ measurable, if we can prove the above paragraph, we obtain that $g\circ\varphi^{-1}_\alpha$ is Lebesgue measurable. So, now approximate by simple functions.
% \end{proof}




%%%%%%%%%%%%%%%%%%%%%%%%%%%%%%%%%%%%%%%%%%%%%%%%
%%%%%%%%%%%%%%%%%%%%%%%%%%%%%%%%%%%%%%%%%%%%%%%%





\appendix

%%%%%%%%%%%%%%%%%%%%%%%%%%%%%%%%%%%%%%%%%%%%%%%%
%%%%%%%%%%%%%%%%%%%%%%%%%%%%%%%%%%%%%%%%%%%%%%%%


\section{Appendix}\label{sec:Appendix}
\subsection{Continuous one-parameter subgroups of $\Gl(\mathbb{R}^d)$.}\label{subsec:OneParameterGroups}

\begin{proposition}[See Section 8 of \cite{Randles2017}]\label{prop:ContinuousGroupProperties}
Let $E,G\in\End(\mathbb{R}^d)$ and $A\in\Gl(\mathbb{R}^d)$. Also, let $E^*$ denote the adjoint of $E$. Then, for all $t,s>0$, the following statements hold:

\vspace{.3cm}
\begin{tabular}{lllll}
$\bullet$ $1^E=I$ &  $\bullet$ $t^{E^*}=(t^E)^*$ & $\bullet$ $(t^E)^{-1}=t^{-E}$ &   $\bullet$ If $EG=GE$, then $t^Et^G=t^{E+G}$\\
\vspace{.1cm}\\
$\bullet$ $(st)^E=s^Et^E$ & $\bullet$ $At^EA^{-1}=t^{AEA^{-1}}$&  $\bullet$ $\det\left(t^E\right)=t^{\tr E}$\\

\end{tabular}
\end{proposition}

\begin{definition} A continuous one-parameter group $T_t=t^E$ is said to be \textit{contracting} if
\begin{equation*}
\lim_{t\to 0}\|T_t\|=0. 
\end{equation*}
\end{definition}
\noindent By virtue of the Banach-Steinhaus theorem, we have the following useful characterization.
\begin{proposition}\label{prop:ContractingCharacterization}
Let $\{T_t\}$ be a continuous one-parameter group. Then $\{T_t\}$ is contracting if and only if
\begin{equation}\label{eq:ContractingSufficient}
\lim_{t\to\infty}|T_tx|=0
\end{equation}
for all $x\in\mathbb{R}^d$.
\end{proposition}

\begin{proof}It is clear that \eqref{eq:ContractingSufficient} is a necessary condition for $\{T_t\}$ to be contracting. We must therefore prove \eqref{eq:ContractingSufficient} also sufficient. To this end, we assume that the continuous one-parameter group $\{T_t\}$ satisfies \eqref{eq:ContractingSufficient}. By virtue of the continuity of $\{T_t\}$ and \eqref{eq:ContractingSufficient}, we have
\begin{equation*}
\sup_{0<t\leq 1}|T_t x|<\infty
 \end{equation*}
for each $x\in\mathbb{R}^d$. From the Banach-Steinhaus theorem, it follows that $\|T_t\|\leq C$ for all $0<t\leq 1$. Now, suppose that $\{T_t\}$ is not contracting. In this case, one can find a sequence $\{\eta_n\}\subseteq\mathbb{S}$ and a sequence $t_n\rightarrow 0$ for which $\lim_n|T_{t_n}\eta_n|>0$. But because the unit sphere is compact, $\{\eta_n\}$ has a convergence subsequence $\eta_{n_k}\rightarrow \eta$ with $|\eta|=1$ . Observe that, for all $n$,
 \begin{equation*}
 |T_{t_n}(\eta-\eta_n)|\leq C|\eta-\eta_n|
 \end{equation*}
 and so it follows that
 \begin{equation*}
 \lim_{k\rightarrow\infty}|T_{t_{n_k}}\eta|=\lim_{k\rightarrow\infty}|T_{t_{n_k}}\eta_{n_k}|>0,
 \end{equation*}
 a contradiction.
 \end{proof}




\begin{lemma}\label{lem:OperatorBoundsforContractingGroup}
Let $\{T_t\}\subseteq\GldR$ be a continuous one-parameter group and let $E\in\End(\mathbb{R}^d)$ be its generator, i.e., $T_t=t^E$ for all $t>0$.
If $\{T_t\}$ is contracting, then $E\in\Gl(\mathbb{R}^d)$ and there is a positive constant $C$ for which
\begin{equation*}
\|T_t\|\leq C+t^{\|E\|}
\end{equation*}
for all $t>0$.
\end{lemma}
\begin{proof}
If for some non-zero vector $\eta$, $E\eta=0$, then $t^E\eta=\eta$ for all $t>0$ and this would contradict our assumption that $\{T_t\}$ is contracting. Hence $E\in\Gl(\mathbb{R}^d)$ and, in particular, $\|E\|>0$. From the representation $T_t=t^E$, it follows immediately that $\|T_t\|\leq t^{\|E\|}$ for all $t\geq 1$ and so it remains to estimate $\|T_t\|$ for $t<1$. Given that $\{T_t\}$ is continuous and contracting, the map $t\mapsto \|T_t\|$ is continuous and approaches $0$ as $t\rightarrow 0$ and so it is necessarily bounded for $0<t\leq 1$.
\end{proof}

\textcolor{red}{There is a result about diagonalizable contracting groups commented out here}
\begin{comment}
\begin{lemma}\label{lem:SpectralEstimateforContractingGroup}
Let $E\in\GldR$ be diagonalizable with strictly positive spectrum. Then $\{t^E\}$ is a continuous one-parameter contracting group. Moreover, there is a positive constant $C$ such that
\begin{equation*}
\|t^E\|\leq Ct^{\lambda_{\mbox{\tiny{max}}}}
\end{equation*}
for all $t\geq 1$ and
\begin{equation*}
\|t^E\|\leq Ct^{\lambda_{\mbox{\tiny{min}}}}
\end{equation*}
for all $0<t<1$, where $\lambda_{\mbox{\tiny{max}}}=\max(\Spec(E))$ and $\lambda_{\mbox{\tiny{min}}}=\min(\Spec(E))$.
\end{lemma}
\begin{proof}
Let $A\in\GldR$ be such that $A^{-1}EA=D=\diag(\lambda_1,\lambda_2,\dots,\lambda_d)$ where necessarily $\Spec(E)=\Spec(D)=\{\lambda_1,\lambda_2,\dots,\lambda_d\}\subseteq (0,\infty)$. It follows from the spectral mapping theorem that $\Spec(t^D)=\{t^{\lambda_1},t^{\lambda_2},\dots,t^{\lambda_d}\}$ for all $t>0$ and moreover, because $t^D$ is symmetric,
\begin{equation*}
\|t^D\|\leq \max(\{t^{\lambda_1},t^{\lambda_2},\dots,t^{\lambda_d}\})
=\begin{cases}
t^{\lambda_{\mbox{\tiny{max}}}} & \mbox{if }t\geq 1\\
t^{\lambda_{\mbox{\tiny{min}}}} & \mbox{if }t<1.
\end{cases}
\end{equation*}
By virtue of Proposition \ref{prop:tEProperties}, we have
\begin{equation*}
\|t^E\|=\|At^DA^{-1}\|\leq \|A\|\|t^D\|\|A^{-1}\|\leq C\|t^D\|=C\times
\begin{cases}
t^{\lambda_{\mbox{\tiny{max}}}} & \mbox{if }t\geq 1\\
t^{\lambda_{\mbox{\tiny{min}}}} & \mbox{if }t<1
\end{cases}
\end{equation*}
for $t>0$ where $C=\|A\|\|A^{-1}\|$; in particular, $\{t^E\}$ is contracting because $\lambda_{\mbox{\tiny{min}}}>0$.
\end{proof}
\end{comment}


\begin{proposition}\label{prop:ContractingLimits}
Let $\{T_t\}_{t>0}\subseteq\Gl(\mathbb{R}^d)$ be a continuous one-parameter contracting group.  Then, for all non-zero $x\in\mathbb{R}^d$,
\begin{equation*}
\lim_{t\rightarrow 0}|T_t x|=0\hspace{.5cm}\mbox{ and }\hspace{.5cm}\lim_{t\rightarrow\infty}|T_t x|=\infty.
\end{equation*}
\end{proposition}
\begin{proof}
The validity of the first limit is clear. Upon noting that $|x|=|T_{1/t}T_tx|\leq \|T_{1/t}\||T_t x|$ for all $t>0$, the second limit follows at once.
\end{proof}
\begin{proposition}\label{prop:ScaleFromSphere}
Let $\{T_t\}_{t>0}$ be a continuous one-parameter contracting group. There holds the following:
\begin{enumerate}[label=(\alph*), ref=(\alph*)]
\item\label{item:ScaleFromSphere_1} For each non-zero $x\in \mathbb{R}^d$, there exists $t>0$ and $\eta\in \mathbb{S}$ for which $T_t\eta=x$. Equivalently,
\begin{equation*}
\mathbb{R}^d\setminus\{0\}=\{T_t\eta:t>0\mbox{ and }\eta\in \mathbb{S}\}.
\end{equation*}
\item\label{item:ScaleFromSphere_2} For each sequence $\{x_n\}\subseteq\mathbb{R}^d$ such that $\lim_n|x_n|=\infty$, $x_n=T_{t_n}\eta_n$ for each $n$, where $\{\eta_n\}\subseteq \mathbb{S}$ and $t_n\rightarrow\infty$ as $n\rightarrow\infty$.
\item\label{item:ScaleFromSphere_3} For each sequence $\{x_n\}\subseteq\mathbb{R}^d$ such that $\lim_n|x_n|=0$, $x_n=T_{t_n}\eta_n$ for each $n$, where $\{\eta_n\}\subseteq \mathbb{S}$ and $t_n\rightarrow 0$ as $n\rightarrow\infty$.
\end{enumerate}
\end{proposition}
\begin{proof}
In view of Proposition \ref{prop:ContractingLimits}, the assertion \ref{item:ScaleFromSphere_1} is a straightforward application of the intermediate value theorem. For \ref{item:ScaleFromSphere_2}, suppose that $\{x_n\}\subseteq\mathbb{R}^d$ is such that $|x_n|\rightarrow \infty$ as $n\rightarrow\infty$. In view of \ref{item:ScaleFromSphere_1}, take $\{\eta_n\}\subseteq S$ and $\{t_n\}\subseteq (0,\infty)$ for which $x_n=T_{t_n}\eta_n$ for each $n$. In view of Lemma \ref{lem:OperatorBoundsforContractingGroup},
\begin{equation*}
\infty=\liminf_n |x_n|\leq\liminf_n \left(C+t_n^M\right)|\eta_n|\leq C+\liminf_nt_n^M,
\end{equation*}
where $C,M>0$ and therefore $t_n\rightarrow\infty$. If instead $\lim_n x_n=0$,
\begin{equation*}
\infty=\lim_{n\rightarrow\infty}\frac{|\eta_n|}{|x_n|}=\lim_{n\rightarrow\infty}\frac{|T_{1/t_n}x_n|}{|x_n|}\leq\limsup_n\|T_{1/t_n}\|\leq\limsup_n(C+(1/t_n)^M)
\end{equation*}
from which we see that $t_n\rightarrow 0$, thus proving \ref{item:ScaleFromSphere_3}.
\end{proof}

\textcolor{red}{I'm not sure we need this}
\begin{proposition}\label{prop:ContractingCapturesCompact}
Let $\{T_t\}$ be a continuous contracting one-parameter group. Then for any open neighborhood $\mathcal{O}\subseteq\mathbb{R}^d$ of the origin and any compact set $K\subseteq\mathbb{R}^d$, $K\subseteq T_t(\mathcal{O})$ for sufficiently large $t$.
\end{proposition}
\begin{proof}
Assume, to reach a contradiction, that there are sequences $\{x_n\}\subseteq K$ and $t_n\rightarrow\infty$ for which $x_n\notin T_{t_n}(\mathcal{O})$ for all $n$. Because $K$ is compact, $\{x_n\}$ has a subsequential limit and so by relabeling, let us take sequences $\{\zeta_k\}\subseteq K$ and $\{r_k\}\subseteq (0,\infty)$ for which $\zeta_k\rightarrow \zeta$, $r_k\rightarrow\infty$ and $\zeta_k\notin T_{r_k}(\mathcal{O})$ for all $k$. Setting $s_k=1/r_k$ and using the fact that $\{T_t\}$ is a one-parameter group, we have $T_{s_k}\zeta_k\notin\mathcal{O}$ for all $k$ and so $\liminf_{k}|T_{s_k}\zeta_k|>0$, where $s_k\rightarrow 0$. This is however impossible because $\{T_t\}$ is contracting and so
\begin{equation*}
\lim_{k\rightarrow\infty}|T_{s_k}\zeta_k|\leq\lim_{k\rightarrow \infty}|T_{s_k}(\zeta_k-\zeta)|+\lim_{k\rightarrow\infty}|T_{s_k}\zeta|\leq C\lim_{k\rightarrow\infty}|\zeta_k-\zeta|+0=0
\end{equation*}
in view of Lemma \ref{lem:OperatorBoundsforContractingGroup}.
\end{proof}





%%%%%%%%%%%%%%%%%%%%%%%%%%%%%%%%%%%%%%%%%%%%%%%%
%%%%%%%%%%%%%%%%%%%%%%%%%%%%%%%%%%%%%%%%%%%%%%%%





\begin{thebibliography}{99}

\bibitem{Amann2009}
H. Amann and J. Escher
\newblock {\em {Analysis III} }
\newblock {Birkh\"{a}user Basel, 2009}

\bibitem{Apostol1974} 
T.M. Apostol
\newblock {\em {Mathamtical Analysis} 2nd Edition}
\newblock {Addison-Wesley, 1974}

\bibitem{Bogachev2007}
V. I. Bogachev.
\newblock {\em {Measure Theory} Vol. 1}.
\newblock Springer-Verlag, 2007.

\bibitem{Engel2005}
Klaus-Jochen Engel and Rainer Nagel
\newblock {\em {A Short Course on Operator Semigroups}}.
\newblock Springer-Verlag, 2005

\bibitem{Engel2000}
K.-J. Engle and R. Nagel
\newblock {\em {One-Parameter Semigroups for Linear Evolution Equations}}, {\em Graduate Texts in Mathematics., Vol. 194,}
\newblock Springer-Verlag, 2000. 

\bibitem{Folland1984}
Gerald Folland.
\newblock {\em {Real Analysis: Modern Techniques and Their Applications}}, {\em Pure \& Applied Mathematics}.
\newblock Wiley-Interscience, 1984

\bibitem{Hormander1983}
Lars H{\"{o}}rmander.
\newblock {\em {The Analysis of Linear Partial Differential Operators II}}.
\newblock Springer-Verlag Berlin Heidelberg, Berlin, 1983.

\bibitem{Randles2017}
Evan Randles and Laurent Saloff-Coste. 
\newblock {\em {``Convolution powers of complex functions on $\mathbb{Z}^d$."}
\newblock Revista Matem\'{a}tica Iberoamericana, vol. 33, no. 3, 2017, pp. 1045-1121.}


\bibitem{Rudin1987}
Walter Rudin.
\newblock {\em {Real and Complex Analysis}}, {\em McGraw-Hill Series in Higher Mathematics}.
\newblock WCB/McGraw-Hill, 1987.

\bibitem{Stein2005}
Elias Stein and Rami Shakarchi.
\newblock {\em {Real Analysis} Measure Theorem, Integration, \& Hilbert Spaces}, {\em Princeton Lectures in Analysis III}.
\newblock Princeton University Press, 2005.


\bibitem{lee2013smooth}
Lee, John M.
\newblock{\em Introduction to Smooth Manifolds}.
\newblock{\em Springer, 2013}


\bibitem{absil2009optimization}
Absil, P-A and Mahony, Robert and Sepulchre, Rodolphe.
\newblock{\em Optimization algorithms on matrix manifolds}.
\newblock{\em Princeton University Press, 2009}



\end{thebibliography}

\section{Stuff to think about}
This subsection amasses some notes about thinks we might want to think about adding:
\begin{enumerate}
    \item Can we prove that $\Sym(P)$ is the symmetry group of $S$? 
    
    \item Can we prove that $\Sym(P)$ is the symmetry group of $B$?
    
    
    \item Can we prove that $\sigma_P$ is the unique Radon measure on $S$ for which 
    \begin{equation*}
        \sigma_P(OF)=\sigma_P(F)
    \end{equation*}
    for all $F\in\Sigma$ and $O\in\Sym(P)$? If so, this would make a very nice addition to Theorem \ref{thm:BestIntegrationFormula}. We could say: Moreover, $\sigma_P$ is the unique Radon measure satisfying these properties. 
    \item One thing I've been thinking about is that the proof of Proposition \ref{prop:Endependence} provides a method by which $\sigma_P$ might be able to be characterized without making any reference whatsoever to $E$. I'm not completely sure about this, but if we follow the construction, we've essentially shown that
\begin{equation*}
\sigma(F)=\lim_{n\to\infty}F_n'(1)
\end{equation*}
where
\begin{equation*}
F_n(s)=\int_{\mathbb{R}^d}f_{n,s}(x)\,dx
\end{equation*}
where
\begin{equation*}
f_{n,s}=\phi_n(x)\chi_{[1/2,s]}(P(x)).
\end{equation*}
I wonder if this provides us with a way to write (and I'm being very loose about this): Denote by $\mathcal{A}_F$ the set of continuous and compactly supported functions $\phi:\mathbb{R}^d\to [0,1]$ with $\phi(x)=1$ for all $x\in F$. Then
\begin{equation*}
\sigma(F)=(?)\inf\left\{\frac{d}{ds}\left(\int_{\mathbb{R}^d}\phi(x)\chi_{[1/2,s]}(P(x))\,dx\right)\bigg\vert_{s=1}:\phi\in \mathcal{A}_F\right\}
\end{equation*}
or maybe
\begin{equation*}
\sigma(F)=(?)\inf\left\{\lim_{h\to 0}\frac{1}{h}\int_{\mathbb{R}^d}\phi(x)\chi_{[1,1+h]}(P(x))\,dx:\phi\in \mathcal{A}_F\right\}
\end{equation*}
\item Should we include the result about conjugation by $t^E$ in section 2? Here is the statement and proof:

\begin{lemma}[$\Sym(P)$ is invariant under quasi-conjugation by $\{t^E,t^{E'}\}$]\label{lem:SymP}
\begin{equation*}
    \Sym{(P)} = t^E \Sym{(P)} t^{-E'}, \quad \text{for all } t>0,E,E'\in \Exp{(P)}
\end{equation*}
\end{lemma}


\begin{proof}
Let $t>0, E,E'\in \Exp{(P)}, O\in \Sym(P)$ be given. Observe that
\begin{equation}\label{eq:EE}
    P(t^E O t^{-E'} \eta) = tP(Ot^{-E'}\eta) = tP(t^{-E'}\eta) = \f{t}{t}P(\eta)= P(\eta)
\end{equation}
for all $\eta\in \R^d$. So, $t^E O t^{-E'} \in \Sym(P)$. Now, if $O_i,O_j\in \Sym(P)$ and $O_i\neq O_j$, then $t^E O_i t^{-E'} \neq t^E O_j t^{-E'}$ since both $t^{E}, t^{-E'}$ are bijective maps. Thus, the conjugation map induced by $t,E,E'$: $\Gamma_{t,E,E'} : \Sym{P} \to \Sym{(P)}$ defined by
\begin{equation*}
    \Gamma_{t,E,E'}(O) = t^E O t^{-E'}
\end{equation*}
is one-to-one. Further, for any $O\in \Sym(P)$, a similar argument as $\eqref{eq:EE}$ shows that $t^{-E}Ot^{E'}\in \Sym(P)$ and satisfies 
\begin{equation*}
    \Gamma_{t,E,E'}(t^{-E}Ot^E) = t^E(t^{-E}O t^{E'})t^{-E'} = O\in \Sym{(P)}.
\end{equation*}
So, $\Gamma_{t,E,E'}$ is bijective, which means 
\begin{equation*}
    \Sym{(P)} = t^{E} \Sym{(P)} t^{-E'} = t^{E'} \Sym{(P)} t^{-E}
\end{equation*}
for all $t>0$ and $E,E'\in \Exp{P}$.
\end{proof}






\item 



\end{enumerate}
\end{document}