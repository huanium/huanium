\documentclass[11pt]{article}
\usepackage[total={7in, 8in}]{geometry}
\usepackage{graphicx}
\usepackage{amsmath, amsthm, latexsym, amssymb, color,cite,enumerate}
\usepackage{caption,subcaption,}
\usepackage{mathtools}
\pagenumbering{arabic}
\theoremstyle{theorem}
\newtheorem{theorem}{Theorem}[section]
\newtheorem{lemma}[theorem]{Lemma}
\newtheorem{definition}[theorem]{Definition}
\newtheorem{corollary}[theorem]{Corollary}
\newtheorem{proposition}[theorem]{Proposition}
\newtheorem{convention}[theorem]{Convention}
\newtheorem{conjecture}[theorem]{Conjecture}
%\theoremstyle{remark}
\newtheorem{remark}{Remark}
\newtheorem{example}{Example}
\newcommand*{\myproofname}{Proof}
\newenvironment{subproof}[1][\myproofname]{\begin{proof}[#1]\renewcommand*{\qedsymbol}{$\mathbin{/\mkern-6mu/}$}}{\end{proof}}
\renewcommand\Re{\operatorname{Re}}%%redefined Re and Im
\renewcommand\Im{\operatorname{Im}}
\newcommand\MdR{\mbox{M}_d(\mathbb{R})}
\newcommand\GldR{\mbox{Gl}_d(\mathbb{R})}
\newcommand\OdR{\mbox{O}_d(\mathbb{R})}
\newcommand\Sym{\operatorname{Sym}}
\newcommand\Exp{\operatorname{Exp}}
\newcommand\tr{\operatorname{tr}}
\newcommand\diag{\operatorname{diag}}
\newcommand\supp{\operatorname{Supp}}
\newcommand\Spec{\operatorname{Spec}}
\renewcommand\det{\operatorname{det}}
\newcommand\Ker{\operatorname{Ker}}
\newcommand\R{\mathbb{R}}

\newcommand{\lp}{\left(}
\newcommand{\rp}{\right)}

\newcommand{\lb}{\left[}
\newcommand{\rb}{\right]}

\newcommand{\lc}{\left\{}
\newcommand{\rc}{\right\}}

\author{Huan Bui and Evan Randles}
\title{A generalized polar coordinate integration formula}
\date{}
\begin{document}
\maketitle

In this \textcolor{red}{section}, we establish a generalized polar coordinate integration formula associated to a positive-homogeneous polynomial. In what follows, we fix a positive-homogeneous polynomial $P:\mathbb{R}^d\to [0,\infty)$ and set
\begin{equation*}
S=\{\eta\in\mathbb{R}^d:P(\eta)=1\}\hspace{1cm}\mbox{and}\hspace{1cm}B=\{\eta\in\mathbb{R}^d:P(\eta)<1\}. 
\end{equation*}
For $E\in\Exp(P)$, the results of \cite{Randles2017} guarantee that, $T_t=t^E$ is a dilation\footnote{\textcolor{red}{We should really think hard about how much our construction depends on $E$}} of $\mathbb{R}^d$, $S$ is a compact hypersurface, i.e., a compact smooth manifold of dimension $d-1$, and $B$ is a bounded open region. \textcolor{red}{I'm not sure if this matters, but $S$ is connected never contains $0$ and is not necessarily convex, $B$ does contains $0$ and, I believe, is connected and simply connected. Also, we should think of $P$ being the essential thing which defines both $E$ and $S$ here -- and we can write $\mu_P=\tr E$ throughout. }

\textcolor{red}{This section needs to contain a description of the standard polar coordinate integration formula and, therein, cite Folland or St.-Sh. For reference, our description should make use of the formula in the following theorem:
\begin{theorem}
Let $\sigma_d$ be the canonical surface measure on the unit sphere $\mathbb{S}^{d-1}\subseteq \mathbb{R}^d$, i.e., $\sigma_d$ is the unique rotation-invariant Radon measure on $\mathbb{S}^{d-1}$ for which $\sigma(\mathbb{S}^{d-1})=d\cdot m(\mathbb{B})=(d\cdot\pi^{d/2})/\Gamma(d/2+1)$;  here $\mathbb{B}$ denotes the unit ball in $\mathbb{R}^d$, $m$ is the Lebesgue measure on $\mathbb{R}^d$ and we will write $dm(x)=dx$. Let $f\in L^1(\mathbb{R}^d)$. Then, for $\sigma_d$-almost every $\eta$, $t\mapsto f(t\eta)$ is (absolutely) integrable with respect to the measure $t^{d-1}dt$ on $(0,\infty)$, $\eta\mapsto \int_0^\infty f(t\eta)t^{d-1}\,dt$ is (absolutely) integrable with respect to $\sigma_d$ on $\mathbb{S}^{d-1}$ and
\begin{equation}\label{eq:StandardPolarIntegrationFormula}
\int_{\mathbb{R}^d}f(x)\,dx=\int_{\mathbb{S}^{d-1}}\left(\int_0^\infty f(t\eta)t^{d-1}\,dt\right)\,d\sigma_d(\eta).
\end{equation}
\end{theorem}}

\section{A surface measure on $S$ and an $S$-adapted integration formula for $\mathbb{R}^d$}\label{sec:IntegrationFormula}

Our goal in this section is to define a measure $\sigma$ on the surface $S$ \textcolor{red}{(which is adapted to $P$ and $E$)} and establish an analogue of the the formula \eqref{eq:StandardPolarIntegrationFormula} to integrate functions on $\mathbb{R}^d$. \textcolor{red}{We believe this will be useful in proving local limit theorems.} Our construction if purely measure-theoretic. It proceeds by replacing the standard isotropic group of dilations on $\mathbb{R}^d$ by a generally anisotropic group of dilations $t\mapsto T_t=t^E$ where $E\in\Exp(P)$. As the standard isotropic one-paramter group of dilation $(0,\infty)\ni t\mapsto tI\in \GldR$ is well-fitted to the unit sphere $\mathbb{S}^{d-1}$ and allows every non-zero $x\in\mathbb{R}^d$ to be written uniquely as $x=t\eta$ for $t\in (0,\infty)$ and $\eta\in \mathbb{S}^{d-1}$, $(0,\infty)\ni t\mapsto T_t=t^E\in\GldR$ is well-fitted to to $S$ and, as we see below, has the property that every non-zero $x\in\mathbb{R}^d$ can be written uniquely as $x=t^E\eta$ where $t\in(0,\infty)$ and $\eta\in S$. With this group of dilations as a tool, we define a surface measure $\sigma$ on $S$ by taking sufficiently nice sets $F\subseteq S$, stretching them into a quasi-conical region of the associated ``ball" $B$, and computing the Lebesgue measure of the result. Once the measure $\sigma$ is constructed, we turn our focus to an associated product measure and establish our analogue of \eqref{eq:StandardPolarIntegrationFormula}; this is Theorem \ref{thm:MainIntegrationFormula}. As a consequence of the theorem, we are able to show that $\sigma$ is a Radon measure on $S$. 

We shall take $S$ to be equipped with the relative topology inherited from $\mathbb{R}^d$ and, given $(0,\infty)$ with its usual topology, we take $S\times (0,\infty)$ to be equipped with the product topology. Consider the map $\psi:S\times (0,\infty)\to\mathbb{R}^d\setminus\{0\}$ defined by
\begin{equation}\label{eq:Homeomorphism}
\psi(\eta,t)=t^E\eta
\end{equation}
for $\eta\in S$ and $t>0$. As $\psi$ is the restriction of the continuous function $\mathbb{R}^d\times (0,\infty)\ni (\xi,t)\mapsto t^E\xi\in\mathbb{R}^d$ to $S\times (0,\infty)$, it is necessarily continuous. As the following proposition shows, $\psi$ is, in fact, a homeomorphism.

\begin{proposition}\label{prop:PsiHomeomorphism}
The map $\psi:S\times (0,\infty)\to\mathbb{R}^d\setminus\{0\}$, defined by \eqref{eq:Homeomorphism} is a homeomorphism with continuous inverse $\psi^{-1}:\mathbb{R}^d\setminus\{0\}\to S\times (0,\infty)$ given by
\begin{equation*}
\psi^{-1}(\xi)=((P(\xi)^{-E}\xi,P(\xi))
\end{equation*}
for $\xi\in\mathbb{R}^d\setminus\{0\}$.
\end{proposition}

\begin{proof}
Given that $P$ is continuous and positive-definite, $P(\xi)>0$ for each $\xi\in \mathbb{R}^d\setminus\{0\}$ and the map $\mathbb{R}^d\setminus\{0\}\ni \xi \mapsto P(\xi)^{-E}\xi\in \mathbb{R}^d$ is continuous. Further, in view of the homogeneity of $P$,
\begin{equation*}
P\left(P(\xi)^{-E}\xi\right)=P(\xi)^{-1}P(\xi)=1
\end{equation*}
for all $\xi\in\mathbb{R}^d\setminus\{0\}$. It follows from these two observations that
\begin{equation*}
\rho(\xi)=(P(\xi)^{-E}\xi,P(\xi),
\end{equation*}
defined for $\xi\in\mathbb{R}^d\setminus\{0\}$, is a continuous function taking $\mathbb{R}^d\setminus\{0\}$ into $S\times (0,\infty)$. We have
\begin{equation*}
(\psi\circ \rho)(\xi)=\psi((P(\xi)^{-E}\xi,P(\xi))=P(\xi)^{E}(P(\xi)^{-E}\xi)=\xi
\end{equation*}
for every $\xi\in \mathbb{R}^d\setminus \{0\}$ and
\begin{equation*}
(\rho\circ\psi)(\eta,t)=\rho(t^E\eta)=P(t^{E}\eta)^{-E}t^{E}\eta,P(t^{E}\eta))=((tP(\eta))^{-E}t^E\eta,tP(\eta)=(\eta,t)
\end{equation*}
for every $(\eta,t)\in S\times (0,\infty)$. Thus $\rho$ is a (continuous) inverse for $\psi$ and so it follows that $\psi$ is a homeomorphism and $\rho=\psi^{-1}$.
\end{proof}


\begin{remark}We shall later \textcolor{red}{(In the next section -- should have a label)} discuss manifold structures on $S$ and $S\times (0,\infty)$ at which point we'll see that, in fact, $\psi$ is a diffeomorphism.
\end{remark}

\noindent As our immediate goal is to construct a measure on the surface $S$, we shall first construct an appropriate $\sigma$-algebra on $S$. To this end, a set $F\subseteq S$ is said to be \textit{measurable} if
\begin{equation*}
\widetilde F:=\bigcup_{0<t<1}t^E F=\{t^E\eta\in\mathbb{R}^d:\eta\in F,0<t<1\}
\end{equation*}
is Lebesgue measurable and the collection of all such subsets $F$ of $S$ shall be denoted by $\Sigma_S$. In other words, if $\mathcal{M}_d$ denotes the $\sigma$-algebra of Lebesgue measurable sets in $\mathbb{R}^d\setminus\{0\}$, then
\begin{equation*}
\Sigma_S=\{F\subseteq S:\widetilde{F}\in\mathcal{M}_d\}.
\end{equation*}


\begin{proposition}\label{prop:BorelContainment}
$\Sigma_S$ is a $\sigma$-algebra on $S$ and contains the Borel $\sigma$-algebra on $S$, $\mathcal{B}(S)$.
\end{proposition}

\begin{proof}
We first show that $\Sigma_S$ is a $\sigma$-algebra. Since $\widetilde S=B\setminus\{0\}$, it is open and therefore Lebesgue measurable. Hence $S\in \Sigma_S$. Let $G, F\in \Sigma_S$ be such that $G\subseteq F$. Then,
\begin{eqnarray*}
\lefteqn{\hspace{-2cm}\widetilde{F\setminus G}=\bigcup_{0<t<1}t^E\left(F\setminus G\right)=\bigcup_{0<t<1}\left(t^EF\setminus t^E G\right)}\\
&&\hspace{4cm}=\left(\bigcup_{0<t<1}t^E F\right)\setminus\left(\bigcup_{0<t<1}t^E G\right)=\widetilde F\setminus \widetilde G
\end{eqnarray*}
where we have used the fact that the collection $\{t^E F\}_{0<t<1}$ is mutually disjoint to pass the union through the set difference. Consequently $\widetilde F\setminus \tilde{G}$ is Lebesgue measurable and therefore $F\setminus G\in \Sigma_S$.  Now, let $\{F_n\}_{n\in\mathbb{N}}$ be a countable collection of measurable sets on $S$, i.e., $\{F_n\}\subseteq \Sigma_S$. Then
\begin{equation*}
    \widetilde{\bigcup_{n=1}^\infty F_n}= \bigcup_{0<t<1} t^E \left(\bigcup_{n=1}^\infty F_n\right)= \bigcup_{0 <t < 1}  \bigcup_{n=1}^\infty  t^E F_n =\bigcup_{n=1}^\infty \bigcup_{0 <t < 1}  t^E F_n =\bigcup_{n=1}^\infty \widetilde{F_n} \in \mathcal{M}_d
\end{equation*}
and so $\bigcup_n F_n\in \Sigma_S$. Thus $\Sigma_S$ is a $\sigma$-algebra. 

Finally, we show that
\begin{equation*}
\mathcal{B}(S)\subseteq\Sigma_S.
\end{equation*}
As the Borel $\sigma$-algebra is the smallest $\sigma$-algebra containing the open subsets of $S$, it suffices to show that $\mathcal{O}\in \Sigma_S$ whenever $\mathcal{O}$ is open in $S$. Armed with Proposition \ref{prop:PsiHomeomorphism}, this is an easy task: Given an open set $\mathcal{O}\subseteq S$, observe that
\begin{equation*}
\widetilde{O}=\{t^E\eta:0<t<1,\eta\in\mathcal{O}\}=\psi(\mathcal{O}\times (0,1)).
\end{equation*}
Upon noting that $\mathcal{O}\times (0,1)$ is an open subset of $S\times (0,\infty)$, Proposition \ref{prop:PsiHomeomorphism} guarantees that $\psi(\mathcal{O}\times (0,1))=\widetilde{\mathcal{O}}$ is an open subset of $\mathbb{R}^d$ and is therefore Lebesgue measurable. 
\end{proof}

\noindent We are now ready to specify a measure on the measurable space $(S,\Sigma_S)$. For each $F\in \Sigma_S$, we define
\begin{equation*}
\sigma(F)=(\tr E) m(\widetilde F)
\end{equation*}
where $m$ is the Lebesgue measure on $\mathbb{R}^d$. We have:

\begin{proposition}\label{prop:sigmaisameaure}
$\sigma$ is a finite measure on $(S,\Sigma_S)$.
\end{proposition}
\begin{proof}

\noindent It is clear that $\widetilde{\varnothing}=\varnothing$ and therefore
\begin{equation*}
\sigma(\varnothing)=(\tr E)m(\varnothing)=0.
\end{equation*}
Second, for any $F \in \Sigma_S$, 
\begin{equation*}\sigma(F) = (\tr E) m(\tilde{F}) \geq 0
\end{equation*}
because $m$ is a measure and $\tr E\geq 0$. Now, let $\{ F_n  \}^\infty_{n=1} \subseteq \Sigma_S $ be a mutually disjoint collection. We claim that $\{ \widetilde{F_n} \}_{n=1}^\infty\subseteq\mathcal{M}_d$ is also a mutually disjoint collection. To see this, suppose that $x = t_n^E \eta_n = t_m^E \eta_m\in \widetilde{F_n}\cap\widetilde{F_m}$, where $t_n,t_m \in (0,1)$, $\eta_n \in F_n$, and $\eta_m \in F_m $. Then
\begin{equation*}
    t_n = P(t_n^E \eta_n) = P(x) = P(t_m^E \eta_m) = t_m,
\end{equation*}
implying that $\eta_n = \eta_m\in F_n\cap F_m$. Because $\{F_n\}_{n=1}^\infty$ is mutually disjoint, we must have $n=m$ which verifies our claim. By virtue of the countable additivity of Lebesgue measure, we therefore have
\begin{equation*}
\sigma\left(\bigcup_{n=1}^\infty F_n\right)
    = (\tr E)m\left( \widetilde{\bigcup^\infty_{n=1} F_n } \right)=(\tr E)m\left( \bigcup^\infty_{n=1}\widetilde{F_n} \right)
    = \tr E \sum^\infty_{n=1} m(\widetilde{F_n})
    = \sum^\infty_{n=1}\sigma(F_n).
\end{equation*}
Therefore $\sigma$ is a measure on $(S,\Sigma_S)$. Finally, because $\widetilde{S} = B\setminus\{ 0 \}$ is a bounded open region in $\mathbb{R}^d$, $m(\widetilde{S}) < \infty$ and so $\sigma(S) = (\tr E) m(\widetilde{S}) < \infty$ showing that $\sigma$ is finite.
\end{proof}

\noindent By virtue of the two preceding propositions, $\sigma$ is a finite Borel measure on $S$. In fact, as a consequence of our main result in this section, Theorem \ref{thm:MainIntegrationFormula}, we will see that $(S,\Sigma_S,\sigma)$ is the completion of the measure space $(S,\mathcal{B}(S),\sigma)$ and $\sigma$ is a Radon measure, i.e., it is both inner and outer regular. For now, we turn our focus to the product measure which will appear in our generalized integration formula \eqref{eq:MainIntegrationFormula}. Consider the measure spaces $(S,\Sigma_S,\sigma)$ and $((0,\infty),\mathcal{L},\lambda)$ where $\mathcal{L}$ is the $\sigma$-algebra of Lebesgue measurable sets on $(0,\infty)$ and $d\lambda(t)=t^{\tr E-1}\,dt$, i.e., for each $L\in\mathcal{L}$,
\begin{equation*}
\lambda(L)=\int_0^\infty \chi_{L}(t)t^{\tr E-1}\,dt.
\end{equation*}
It is easy to see that $\lambda$ is $\sigma$-finite. Given this and in view of the finiteness of the measure $\sigma$, there is a unique product measure $\mu=\sigma\times \lambda$ on $S\times (0,\infty)$ which is defined on the product $\sigma$-algebra $\Sigma_S\times \mathcal{L}$ and satisfies
\begin{equation*}
\mu(F\times L)=\sigma(F)\lambda(L)
\end{equation*}
for all $F\in \Sigma_S$ and $L\in\mathcal{L}$. We shall denote by $(S\times (0,\infty),\Sigma,\mu)$ the completion of the measure space $(S\times (0,\infty),\Sigma_S\times\mathcal{L},\mu)$. For this measure space, the following formulation of the Fubini/Tonelli theorem is applicable:
\begin{theorem}[Theorem 8.12 of \cite{Rudin1987}]\label{thm:Fubini}
Let $g:S\times (0,\infty)\to\mathbb{C}$ be $\Sigma$-measurable. For each $\eta\in S$, define $g_\eta:(0,\infty)\to\mathbb{C}$ by $g_\eta(t)=g(\eta,t)$ for $t\in (0,\infty)$ and, for each $t\in (0,\infty)$, define $g^t:S\to\mathbb{C}$ by $g^t(\eta)=g(\eta,t)$ for $\eta\in S$. 
\begin{enumerate}
\item For $\sigma$-almost every $\eta$, $g_\eta$ is $\mathcal{L}$-measurable and, for $\lambda$-almost every $t$, $g^t$ is $\Sigma_S$-measurable.
\item\label{item:Fubini1} If $g\geq 0$, then:
\begin{enumerate}
\item For $\sigma$-almost every $\eta$,
\begin{equation*}
G(\eta)=\int_0^\infty g_\eta(t)t^{\tr E-1}\,dt
\end{equation*}
exists as a non-negative extended real number. 
\item For $\lambda$-almost every $t$, 
\begin{equation*}
H(t)=\int_S g^t(\eta)\,d\sigma(\eta)
\end{equation*}
exists as a non-negative extended real number. 
\item We have
\begin{equation*}
\int_S G(\eta)\,d\sigma(\eta)=\int_{S\times (0,\infty)}g\,d\mu=\int_0^\infty H(t)t^{\tr E-1}dt
\end{equation*}
or, equivalently,
\begin{equation}\label{eq:Fubini}
\int_S\left(\int_0^\infty g(\eta,t)t^{\tr E-1}\,dt\right)\,d\sigma(\eta)=\int_{S\times(0,\infty)}g\,d\mu=\int_0^\infty\left(\int_S g(\eta,t)\,d\sigma(\eta)\right)t^{\tr E-1}\,dt.
\end{equation}
\end{enumerate}
\item\label{item:Fubini2} If $g$ is complex-valued and
\begin{equation*}
\int_S\left(\int_0^\infty |g(\eta,t)|t^{\tr E-1}\,dt\right)\,d\sigma(\eta)<\infty\hspace{1cm}\mbox{or}\hspace{1cm}\int_0^\infty\left(\int_S|g(\eta,t)|\,d\sigma(\eta)\right)t^{\tr E-1}\,dt<\infty,
\end{equation*}
then $g\in L^1(S\times (0,\infty),\mu)$.
\item If $g\in L^1(S\times (0,\infty),\mu)$, then $g_\eta\in L^1((0,\infty),\lambda)$ for $\Sigma$-almost every $\eta$, $g^t\in L^1(S,\sigma)$ for $\lambda$-almost every $t$, and \eqref{eq:Fubini} holds.
\end{enumerate}
\end{theorem}

\noindent Our main theorem of this section is stated as follows. 


\begin{theorem}\label{thm:MainIntegrationFormula}
Let $(S\times (0,\infty),\Sigma,\mu)$ be as above and let $m$ be the (restricted) Lebesgue measure on $(\mathbb{R}^d\setminus\{0\},\mathcal{M}_d,m)$.
\begin{enumerate}
\item\label{item:MainIntegrationFormula1} $m$ is the pushforward of the measure $\mu=\sigma\times\lambda$ by $\psi$ \textcolor{red}{(I believe this is the same as the statement: The measure spaces $(S\times (0,\infty),\Sigma,\mu)$ and $(\mathbb{R}^d\setminus\{0\},\mathcal{M}_d,m)$ are isomorphic with (point) isomorphism $\psi$)}. That is
\begin{equation*}
\mathcal{M}_d=\{A\subseteq \mathbb{R}^d\setminus\{0\}:\psi^{-1}(A)\in\Sigma\}
\end{equation*}
and, for each $A\in\mathcal{M}_d$,
\begin{equation*}
m(A)=\psi_*\mu(A)=\mu(\psi^{-1}(A)).
\end{equation*}
\item\label{item:MainintegrationFormula2} If $f:\mathbb{R}^d\to\mathbb{C}$ is Lebesgue measurable, then $f\circ \psi$ is $\Sigma$-measurable and the following statements hold:
\begin{enumerate}
\item If $f\geq 0$, then
\begin{equation}\label{eq:MainIntegrationFormula}
\int_{\mathbb{R}^d}f(x)\,dx=\int_0^\infty\left(\int_S f(t^E\eta)\,d\sigma(\eta)\right)t^{\tr E-1}\,dt=\int_S\left(\int_0^\infty f(t^E\eta)t^{\tr E-1}\,dt\right)\,d\sigma(\eta).
\end{equation}
\item When $f$ is complex-valued, we have $f\in L^1(\mathbb{R}^d)$ if and only if $f\circ\psi\in L^1(S\times (0,\infty),\Sigma,\mu)$ and, in this case, \eqref{eq:MainIntegrationFormula} holds.
\end{enumerate}
\end{enumerate}
\end{theorem}

\begin{remark}
\textcolor{red}{Slight abuse of notation going on with $m$ and $m$ and $dx$ -- should say when we mean the restriction and when we don't?}.
\end{remark}

\noindent To prove this theorem, we shall first treat several lemmas. These lemmas isolate and generalize (and hopefully clarify) several important ideas used in standard proofs of \eqref{eq:StandardPolarIntegrationFormula} (see, e.g., \cite{Folland1984} and \cite{Stein2005}). 

\begin{lemma}\label{lemma:Scaling}
Let $A\subseteq\mathbb{R}^d$ and $t>0$. The set $A$ is Lebesgue measurable if and only if $t^E A=\{x=t^E a:a\in A\}$ is Lebesgue measurable and, in this case,
\begin{equation*}
m(t^E A)=t^{\tr E}m(A).
\end{equation*}
\end{lemma}
\begin{proof}
Because $x\mapsto t^E x$ is a linear isomorphism, $t^E A$ is Lebesgue measurable if and only if $A$ is Lebesuge measureable (See Theorem 2.20 of \cite{Rudin1987}). Observe that $x\in t^E A$ if and only if $t^{-E}x\in A$ and therefore
\begin{equation*}
m(t^E A)=\int_{\mathbb{R}^d}\chi_{t^E A}(x)\,dx=\int_{\mathbb{R}^d}\chi_{A}(t^{-E}x)\,dx.
\end{equation*}
Now, by making the linear change of variables $x\mapsto t^E x$, we have
\begin{equation*}
m(t^E A)=\int_{\mathbb{R}^d}\chi_A(x)|\det(t^E)|\,dx=t^{\tr E}m(A),
\end{equation*}
because $\det(t^E)=t^{\tr E}>0$.
\end{proof}

\begin{lemma}\label{lem:SpecialRectangle}
Let $F\in\Sigma_S$. If $I\subseteq (0,\infty)$ is open, closed, $G_\delta$, or $F_\sigma$, then $\psi(F\times I)\in\mathcal{M}_d$ and
\begin{equation}\label{eq:SpecialRectangle}
m(\psi(F\times I))=\mu(F\times I)=\sigma(F)\lambda(I).
\end{equation}
\end{lemma}
\begin{proof}
We fix $F\in\Sigma_S$ and consider several cases for $I$.\\

\begin{subproof}[Case 1:]\textit{$I=(0,b)$ for $0<b\leq \infty$.} When $b$ is finite, observe that
\begin{equation*}
\psi(F\times I)=\{t^E\eta:0<t<b,\eta\in F\}=b^E\{t^E\eta:0<t<1,\eta \in F\}=b^E\widetilde{F}.
\end{equation*}
By virtue of Lemma \ref{lemma:Scaling}, it follows that $\psi(F\times I)\in\mathcal{M}_d$ and
\begin{eqnarray*}
\lefteqn{\hspace{-2.5cm}\mu(F\times I)=\sigma(F)\lambda(I)}\\
&=&\left((\tr E) m(\widetilde{F})\right)\left(\int_0^b t^{\tr E-1}\,dt\right)=b^{\tr E}m(\widetilde{F})=m(b^{E}\widetilde{F})\\
&&\hspace{8cm}=m(\psi(F\times I)).
\end{eqnarray*}
When $b=\infty$ i.e., $I=(0,\infty)$, we observe that
\begin{equation*}
I=\bigcup_{n=1}^\infty (0,n)=\bigcup_{n=1}^\infty I_n
\end{equation*}
where the open intervals $I_n=(0,n)$ are nested and increasing. In view of the result above (for finite $b=n$), we have
\begin{equation*}\psi(F\times I)=\psi\left(\bigcup_{n=1}^\infty (F\times I_n)\right)=\bigcup_{n=1}^\infty\psi(F\times I_n)\in\mathcal{M}_d.
\end{equation*}
Given that $\psi$ is a bijection, $\{\psi(F\times I_n)\}$ is necessarily a nested increasing sequence and so, by the continuity of the measures $\mu$ and $m$,
\begin{equation*}
\mu(F\times I)=\lim_{n\to\infty}\mu(F\times I_n)=\lim_{n\to\infty}m(\psi(F\times I_n))= m(\psi(F\times I)). 
\end{equation*}
\end{subproof}

\begin{subproof}[Case 2:]\textit{$I=(0,a]$ for $0<a<\infty$.} We have
\begin{equation*}
I=(0,a]=\bigcap_{n=1}^\infty (0,a+1/n)=\bigcap_{n=1}^\infty I_n
\end{equation*}
where the open intervals $I_n=(0,a+1/n)$ are nested and decreasing. By reasoning analogous to that given in Case 1, we have
\begin{equation*}
\psi(F\times I)=\bigcap_{n=1}^\infty \psi(F\times I_n)\in \mathcal{M}_d
\end{equation*}
and
\begin{equation*}
\mu(F\times I)=\lim_{n\to\infty}\mu(F\times I_n)=\lim_{n\to\infty}m(\psi(F\times I_n))=m(\psi(F\times I)).
\end{equation*}
In particular, $m(\psi(F\times I))=\sigma(F)\lambda((0,a])=\sigma(F)a^{\tr E}/\tr E<\infty.$
\end{subproof}
\begin{subproof}[Case 3:]\textit{$I=(a,b)$ for $0<a<b\leq \infty$.} In this case, $I=(0,b)\setminus (0,a]$ and so, in view of Cases 1 and 2, $\psi(F\times I)=\psi(F\times (0,b))\setminus \psi(F\times(0,a])\in\mathcal{M}_d$ and
\begin{equation*}
\mu(F\times I)=\mu(F\times (0,b))-\mu(F\times(0,a])=m(\psi(F\times (0,b)))-m(\psi(F\times (0,a]))=m(\psi(F\times I))
\end{equation*}
where we have used the fact that $\mu(F\times(0,a])=m(\psi(F\times (0,a]))<\infty$.
\end{subproof}
\begin{subproof}[Case 4:]\textit{$I\subseteq (0,\infty)$ is open.} In this case, it is known that $I$ can be expressed as a countable union of disjoint open intervals $\{I_n\}$ and, by virtue of Cases 1 and 3, we have
\begin{equation*}
\psi(F\times I)=\bigcup_{n=1}^\infty\psi(F\times I_n)\in\mathcal{M}_d,
\end{equation*}
where this union is disjoint, and
\begin{eqnarray*}
\lefteqn{\hspace{-1cm}m(\psi(F\times I))=\sum_n m(\psi(F\times I_n))=\sum_n \mu(F\times I_n)}\\
&&\hspace{2cm}=\sum_n \sigma(F)\lambda(I_n)\sigma(F)\left(\sum_n \lambda(I_n)\right)=\sigma(F)\lambda(I)=\mu(F\times I).
\end{eqnarray*}
\end{subproof}
\begin{subproof}[Case 5:]\textit{$I\subseteq (0,\infty)$ is closed.} In this case, we have $I=(0,\infty)\setminus O$ where $O$ is open and so
\begin{equation*}
\psi(F\times I)=\psi(F\times ((0,\infty)\setminus O))=\psi(F\times (0,\infty))\setminus \psi(F\times O)\in\mathcal{M}_d.
\end{equation*}
At this point, we'd like to use the property that 
\begin{equation*}
m(\psi(F\times (0,\infty))\setminus \psi(F\times O))=m(\psi(F\times (0,\infty)))-m(\psi(F\times O)),
\end{equation*} but this only holds when $m(\psi(F\times O)$ is finite. We must therefore proceed differently. For each natural number $n$, define $O_n=O\cap(0,n)$ and $I_n=(0,n)\setminus O_n$. It is straightforward to show that $\{I_n\}$ and $\{\psi(F\times I_n)\}$ are nested and increasing with
\begin{equation*}
I=\bigcup_{n=1}^\infty I_n\hspace{1cm}\mbox{and}\hspace{1cm}\psi(F\times I)=\bigcup_{n=1}^\infty \psi(F\times I_n). 
\end{equation*}
The results of Cases 1 and 4 guarantee that, for each $n$,
\begin{equation*}
m(\psi(F\times O_n)))=\mu(F\times O_n)\leq \mu(F\times (0,n))<n^{\tr E}m(\tilde F)<\infty
\end{equation*}
and therefore
\begin{equation*}
m(\psi(F\times I_n))=m(\psi(F\times (0,n)))-m(\psi(F\times O_n))=\mu(F\times (0,n))-\mu(F\times O_n)=\mu(F\times I_n).
\end{equation*}
Then, by virtue of the continuity of measure,
\begin{equation*}
m(\psi(F\times I))=\lim_{n\to\infty}m(\psi(F\times I_n))=\lim_{n\to\infty}\mu(F\times I_n)=\mu(F\times I). 
\end{equation*}
\end{subproof}
\begin{subproof}[Case 6]\textit{$I\subseteq (0,\infty)$ is $G_\delta$ or $F_\sigma$.} Depending on whether $I$ is $G_\delta$ or $F_\sigma$,  express $I$ as an intersection of nested decreasing open sets or a union of nested increasing closed sets. In both cases, by virtually the same argument given in the previous cases, we find that $\psi(F\times I)\in \mathcal{M}_d$,
\begin{equation*}
m(\psi(F\times I))=\mu(F\times I).
\end{equation*}
\end{subproof}
\end{proof}

\textcolor{blue}{\begin{remark}
Huan, the analogous result to the following lemma is in the first paragraph on Page 281 of \cite{Stein2005} and is in the sentence preceding ``So we have established (10) for all measurable rectangles..." Truthfully, I don't follow their argument and, actually, I don't quite believe it. 
\end{remark}}
\begin{lemma}\label{lem:AllMeasurableRectangles} For any $F\in \Sigma_S$ and $L\in\mathcal{L}$, $\psi(F\times L)\in\mathcal{M}_d$ and 
\begin{equation*}
m(\psi(F\times L))=\mu(F\times L).
\end{equation*}
\end{lemma}
\begin{proof}
Fix $F\in\Sigma_S$ and $L\in\mathcal{L}$. It is easy to see that $\lambda$ and the Lebesgue measure $dt$ on $(0,\infty)$ are mutually absolutely continuous (\textcolor{blue}{Huan, you should check this and the following sentence.}). It follows that $((0,\infty), \mathcal{L},\lambda)$ is a complete measure space and, further, that there exists an $F_\sigma$ set $L_\sigma\subseteq (0,\infty)$ and a $G_\delta$ set $L_\delta\subseteq (0,\infty)$ for which $L_\sigma\subseteq L\subseteq L_\delta$ and $\lambda(L_\delta\setminus L_\sigma)=0$. Note that, necessarily, $\lambda(L)=\lambda(L_\sigma)=\lambda(L_\delta)$. We have
\begin{equation}\label{eq:AllMeasurableRectangles1}
\psi(F\times L)=\psi(F\times L_\sigma)\cup\psi(F\times (L\setminus L_\sigma))
\end{equation}
where, by virtue of the preceding lemma, $\psi(F\times L_\sigma)\subseteq \mathcal{M}_d$ and
\begin{equation}\label{eq:AllMeasurableRectangles2}
m(\psi(F\times L_{\sigma}))=\mu(F\times L_\sigma)=\sigma(F)\lambda(L_\sigma)=\sigma(F)\lambda(L)=\mu(F\times L).
\end{equation}
Observe that
\begin{equation*}
\psi(F\times (L\setminus L_\sigma))\subseteq \psi(F\times (L_{\delta}\setminus L_\sigma))
\end{equation*}
where, because $L_\delta\setminus L_\sigma$ is an $G_{\delta}$ set, the latter set is a member of $\mathcal{M}_d$ and
\begin{equation*}
m(\psi(F\times (L_\delta\setminus L_\sigma)))=\mu(F\times (L_\delta\setminus L_\sigma))=\sigma(F)\lambda(L_\delta\setminus L_\sigma))=0
\end{equation*}
by virtue of the preceding lemma. Using the fact that $(\mathbb{R}^d\setminus\{0\},\mathcal{M}_d,m)$ is complete, we conclude that $\psi(F\times (L\setminus L_\sigma))\in \mathcal{M}_d$ and $m(\psi(F\times (L\setminus L_\sigma)))=0$. It now follows from \eqref{eq:AllMeasurableRectangles1} and \eqref{eq:AllMeasurableRectangles2} that $\psi(F\times L)\in\mathcal{M}_d$ and
\begin{equation*}
m(\psi(F\times L))=m(\psi(F\times L_\sigma))+m(\psi(F\times (L\setminus L_\sigma))=\mu(F\times L),
\end{equation*}
as desired.
\end{proof}

\noindent \textcolor{blue}{As it is fairly elementary, I've commented out the lemma showing that $S$, as a compact set of a metric space, necessarily contains a countably dense set. If this appears in your thesis, you should feel free to include the lemma}
%\begin{lemma}
%Let $S$ be a compact subset of a metric space. Then $S$ contains a countably dense set.
%\end{lemma}
%\begin{proof}
%For each $n\in\mathbb{N}$, consider the open cover
%\begin{equation*}
%\{B_{1/n}(x)\cap S, x\in S\}
%\end{equation*}
%of $S$. Since $S$ is compact, there exists a finite subcover. Let $x_{j,n}$, $j=1,2,\dots N_n$ denote the center of each of the balls, then, we have that $S$ is covered by $\{B_{1/n}(x_{j,n})\cap S,j=1,2,\dots, N_n\}$. Thus, for each $n\in \mathbb{N}$, we have a finite set $\{x_{j,n}\}$ of centers. The countable union of these finite sets, $\bigcup^\infty_{n=1} \{ x_{j,n}\}$, is countable. It is also dense because for every point $x\in S$ and $\epsilon >0$, there is always some $n$ such that $\vert x_{j,n} - x\vert < 1/n < \epsilon$.  
%\end{proof}


\begin{lemma}\label{lem:OpenRectangle}
Every open subset $U\subseteq \mathbb{R}^d\setminus\{0\}$ can be written as a countable union of open sets of the form $\psi(\mathcal{U})$ where $\mathcal{U}=\mathcal{O}\times I$ is an open rectangle in $S\times (0,\infty)$.
\end{lemma}


\begin{proof}
In what follows, $|\cdot|$ denotes the Euclidean norm on $\mathbb{R}^d$, $N_\delta(x)$ denotes the associated open ball of radius $\delta$ and center $x\in\mathbb{R}^d$ and, for each linear transformation $T:\mathbb{R}^d\to\mathbb{R}^d$, $\|T\|$ denotes the operator norm of $T$ associated to the Euclidean norm on $\mathbb{R}^d$. Given that $S$ is compact (as a subspace of the metric space $\mathbb{R}^d$), $S$ has a countably dense set $\{\eta_j\}_{j=1}^\infty$. Let $\{t_k\}_{k=1}^\infty$ be a countably dense subset of $(0,\infty)$. For each triple of natural numbers $j,l,n\in\mathbb{N}_+$, consider the open set
\begin{equation*}
\mathcal{U}_{j,l,n}=\mathcal{O}_{j,n}\times \{ \vert t - t_l \vert < 1/n \}\subseteq S\times (0,\infty)
\end{equation*}
where
\begin{equation*}
\mathcal{O}_{j,n}=\{\eta\in S: |\eta-\eta_j|<1/n\}.
\end{equation*}
Here, $|\cdot|$ denotes the Euclidean norm on $\mathbb{R}^d$. Fix $U\subseteq \mathbb{R}^d\setminus \{0\}$, an open subset of $\mathbb{R}^d\setminus\{0\}$. We will show that
\begin{equation}\label{eq:OpenRectangle}
U=\bigcup_{\substack{j,l,n\\ \psi(\mathcal{U}_{j,l,n})\subseteq U}}\psi(\mathcal{U}_{j,l,n}),
\end{equation}
where each $\psi(\mathcal{U}_{j,l,n})$ is open because $\psi$ is a homeomorphism. It is clear that any element of the union on the right hand side of \eqref{eq:OpenRectangle} belongs to some $\psi(\mathcal{U}_{j,l,n}) \subseteq U$ and so the union is a subset of $U$. To prove \eqref{eq:OpenRectangle}, it therefore suffices to prove that, for each $x\in U$, there exists a triple $j,l,n$ with
\begin{equation*}
x\in\psi(\mathcal{U}_{j,l,n})\subseteq U.
\end{equation*}
To this end, fix $x\in U$ and, because $U$ is open, let $\delta>0$ be such that $N_{\delta}(x)\subseteq U$. Consider $(\eta_x,t_x)=\psi^{-1}(x)\in S\times (0,\infty)$ and set $M=\|t_x^E\|>0$ and $C=\|E|>0$. Observe that 
\begin{eqnarray*}
\|I-\alpha^E\|&=&\left\|\sum_{k=1}^\infty \frac{(\ln \alpha)^k}{k!} E^k\right\|\\
&\leq &\sum_{k=1}^\infty \frac{|\ln \alpha|^k}{k!} \|E\|^k=e^{(C|\ln \alpha|)}-1
\end{eqnarray*}
for all $\alpha>0$. Since $\alpha\mapsto e^{(C|\ln \alpha|)}-1$ is continuous and $0$ at $\alpha=1$, we can choose $\delta'>0$ for which
\begin{equation*}
\|I-\alpha ^E\|\leq \frac{\delta}{2M (  |\eta_x|+2)}
\end{equation*}
whenever $|\alpha-1|<\delta'$. Choose an integer
\begin{equation*}
n>\max \left\{\frac{1}{t_x},\frac{1}{\delta't_x}, \frac{4 M }{\delta}\right\}.
\end{equation*}
In view of the density of the collections $\{t_l\}$ and $\{\eta_j\}$, we can find $t_l, \eta_j$ such that
\begin{equation*}
    \vert t_l - t_x \vert < \frac{1}{n},\quad \vert \eta_j - \eta_x \vert < \frac{1}{n}.
\end{equation*}
It follows that the corresponding open set $\mathcal{U}_{j,l,n}$ contains $\psi^{-1}(x)$, or, equivalently, $x\in \psi(\mathcal{U}_{j,l,n})$ since $\psi$ is bijective. Thus, it remains to show that $\psi(\mathcal{U}_{j,l,n}) \subseteq N_\delta(x)$. To this end, let $y=\psi(\eta_y,t_y)\in\psi(\mathcal{U}_{j,l,n})$ and consider 
\begin{equation*}
    z = \psi(\eta_y,t_x).
\end{equation*}
By the triangle inequality, we have
\begin{eqnarray*}
    | x - y | 
    &\leq& |x-z | + |z-y| \\
    &\leq& \vert \psi(\eta_x,t_x) - \psi(\eta_y,t_x) \vert 
    + \vert \psi(\eta_y,t_x) - \psi(\eta_y,t_y) \vert\\
    &=& \vert t_x^E \eta_x - t_x^E \eta_y \vert 
    + \vert t_x^E \eta_y - t_y^E \eta_y \vert\\
    &=& \vert t_x^E (\eta_x - \eta_y) \vert + \vert (t_x^E - t_y^E) \eta_y \vert\\
    &\leq& M\vert \eta_x - \eta_y \vert + \|{t_x^E - t_y^E}\|  \vert \eta_y \vert.
\end{eqnarray*}
Since both $(\eta_x,t_x),(\eta_y,t_y) \in \mathcal{U}_{j,l,n}$, we have
\begin{equation*}
    \vert \eta_x - \eta_y \vert \leq \vert \eta_x - \eta_j \vert + \vert \eta_j - \eta_y \vert < \frac{2}{n}
\end{equation*}
and
\begin{equation*}
    \vert \eta_y \vert \leq \vert \eta_y - \eta_x \vert + \vert \eta_x \vert < \vert \eta_x \vert + \frac{2}{n}.
\end{equation*}
Also, since $|t_x-t_y|<1/n$, it follows that $t_y=\alpha t_x$ where
\begin{equation*}
|1-\alpha|<\frac{1}{nt_x} < \delta'
\end{equation*}
by our choice of $n$. Consequently,
\begin{eqnarray*}
    \vert x - y \vert 
    &< & \frac{2}{n} M+ \left( \vert \eta_x \vert + \frac{2}{n} \right) \|{t_x^E -   t_x^E \alpha^E}\|   \\ 
    &<& \frac{2}{n}M + \left( \vert \eta_x \vert + 2 \right)M\| I - \alpha^E\| \\
    &\leq&  \frac{2M }{n} +  \frac{\delta M \left( \vert \eta_x \vert + 2\right) }{2M (| \eta_x | + 2)}  \\
    &<& \frac{\delta}{2} + \frac{\delta}{2} \\
    &=& \delta,
\end{eqnarray*}
and so we have established \eqref{eq:OpenRectangle}. Finally, upon noting that $\{\mathcal{U}_{j,l,n})\}$ is a countable collection of open rectangles (indexed by $(j,l,n)\in\mathbb{N}_+^3$), the union in \eqref{eq:OpenRectangle} is necessarily countable and we are done with the proof.
\end{proof}

\noindent \textcolor{blue}{The following is a (graduate level) homework-exercise worthy lemma. You should try to prove it yourself before you read the proof. Though it is somewhat difficult (to state and prove -- for me, at least), it is abstract enough that I suspect it is fairly well-known and we should look for a reference.}\\

\noindent In our final lemma preceding the proof of Theorem \ref{thm:MainIntegrationFormula}, we treat a general measure-theoretic statement which gives sufficient conditions concerning two measure spaces to ensure that their completions are isomorphic \textcolor{red}{Check this against the pushforward -- I don't like the Bogachev reference \cite{Bogachev2007})}. Though we suspect that this result is well-known, we present its proof for completeness.

\
\begin{lemma}\label{lem:PushforwardLemma}
Let $(X_1,\Sigma_1,\mu_1)$ and $(X_2,\Sigma_2,\mu_2)$ be measure spaces, let $\varphi:X_1\to X_2$ be a bijection and denote by $(X_i',\Sigma_i',\mu_i')$ the completion of the measure space $(X_i,\Sigma_i,\mu_i)$ for $i=1,2$. Assume that the following two properties are satisfied:
\begin{enumerate}
\item\label{property:PushforwardLemma1} For each $A_1\in\Sigma_1$, $\varphi(A_1)\in\Sigma_2'$ and $\mu_2'(\varphi(A_1))=\mu_1(A_1).$
\item\label{property:PushforwardLemma2} For each $A_2\in\Sigma_2$, $\varphi^{-1}(A_2)\in \Sigma_1'$ and $\mu_1'(\varphi^{-1}(A_2))=\mu_2(A_2)$.
\end{enumerate}
Then the measure spaces $(X_1',\Sigma_1',\mu_1')$ and $(X_2',\Sigma_2',\mu_2')$ are isomorphic with (point) isomorphism $\varphi$. \textcolor{red}{I believe this is another way to state that $\mu_2'$ is precisely the pushforward of $\mu_1'$. We should check this.} Specifically,
\begin{equation}\label{eq:PushforwardLemma1}
\Sigma_2'=\{A_2\subseteq X_2: \varphi^{-1}(A_2)\in\Sigma_1'\}
\end{equation}
and
\begin{equation}\label{eq:PushforwardLemma2}
\mu_2'(A_2)=\mu_1'(\varphi^{-1}(A_2))
\end{equation}
for all $A_2\in\Sigma_2'$.
\end{lemma}
\begin{proof}
Let us first assume that $A_2\in\Sigma_2'$. By definition, $A_2=G_2\cup H_2$ where $G_2\in\Sigma_2$ and $H_2\subseteq G_{2,0}\in \Sigma_2$ with $\mu_2'(A_2)=\mu_2(G_2)$ and $\mu_2'(H_2)=\mu_2(G_{2,0})=0$. Consequently, $\varphi^{-1}(A_2)=\varphi^{-1}(G_2)\cup\varphi^{-1}(H_2)$ and $\varphi^{-1}(H_2)\subseteq \varphi^{-1}(G_{2,0})$. In view of Property \ref{property:PushforwardLemma2}, $\varphi^{-1}(G_2),\varphi^{-1}(G_{2,0})\in \Sigma_1'$ and we have
\begin{equation*}
\mu_1'(\varphi^{-1}(G_2))=\mu_2(G_2)=\mu_2'(A_2)\hspace{1cm}\mbox{and}\hspace{1cm}\mu_1'(\varphi^{-1}(G_{2,0}))=\mu_2(G_{2,0})=0.
\end{equation*}
In view of the fact that $(X_1',\Sigma_1',\mu_1')$ is complete, $\varphi^{-1}(H_2)\in\Sigma_1'$ and $\mu_1'(\varphi^{-1}(H_2))=0$. Consequently, we obtain $\varphi^{-1}(A_2)=\varphi^{-1}(G_2)\cup\varphi^{-1}(H_2)\in\Sigma_1'$ and
\begin{equation*}
\mu_2'(A_2)=\mu_1'(\varphi^{-1}(G_2))\leq\mu_1'(\varphi^{-1}(A_2))\leq\mu_1'(\varphi^{-1}(G_2))+\mu_1'(\varphi^{-1}(H_2))=\mu_2(G_2)+0=\mu_2'(A_2).
\end{equation*}
From this we obtain that $\Sigma_2'\subseteq \{A_2\subseteq X_2:\varphi^{-1}(A_2)\in\Sigma_1'\}$ and, for each $A_2\in\Sigma_2'$, $\mu_2'(A_2)=\mu_1'(\varphi^{-1}(A_2))$. It remains to prove that
\begin{equation*}
\{A_2\subseteq X_2:\varphi^{-1}(A_2)\in\Sigma_1'\}\subseteq \Sigma_2'.
\end{equation*}
To this end, let $A_2$ be a subset of $X_2$ for which $\varphi^{-1}(A_2)\in\Sigma_1'$. By the definition of $\Sigma_1'$, we have $\varphi^{-1}(A_2)=G_1\cup H_1$ where $G_1\in\Sigma_1$, $H_1\subseteq G_{1,0}\in\Sigma_1$ and $\mu_1'(H_1)=\mu_1(G_{1,0})=0$. In view of Property \ref{property:PushforwardLemma1}, $\varphi(G_1)\in\Sigma_2'$, $\varphi(H_1)\subseteq\varphi(G_{1,0})\in\Sigma_2'$ and $\mu_2'(\varphi(G_{1,0}))=\mu_1(G_{1,0})=0$. Because $(X_2',\Sigma_2',\mu_2')$ is complete, we have $\varphi(H_1)\in\Sigma_2'$ and so
\begin{equation*}
A_1=\varphi(\varphi^{-1}(A_2))=\varphi(G_1)\cup\varphi(H_1)\in \Sigma_2',
\end{equation*}
as desired.
\end{proof}

\noindent We are finally in a position to prove Theorem \ref{thm:MainIntegrationFormula}.\\

\noindent\textcolor{blue}{Huan, my use of the monotone class lemma below avoids the method in \cite{Stein2005} which relies on Theorem 3.3 (of \cite{Stein2005}) and sweeps some things under the carpet.}

\begin{proof}[Proof of Theorem \ref{thm:MainIntegrationFormula}]
Denote by $\mathcal{C}$ the collection of sets $G\subseteq S\times (0,\infty)$ for which $\psi(G)\in \mathcal{M}_d$ and $m(\psi(G))=\mu(G).$ By virtue of Lemma \ref{lem:AllMeasurableRectangles}, it follows that $\mathcal{C}$ contains all elementary sets, i.e., finite unions of disjoint measurable rectangles. Using the continuity of measure (applied to the measures $m$ and $\mu$) and the fact that $\psi$ is a bijection, it is straightforward to verify that $\mathcal{C}$ is a monotone class. By the so-called monotone class lemma (Theorem 8.3 of \cite{Rudin1987}), it immediately follows that $\Sigma_S\times\mathcal{L}\subseteq\mathcal{C}$. In other words, for each $G\in\Sigma_S\times\mathcal{L}$,
\begin{equation}\label{eq:Good1}
\psi(G)\in\mathcal{M}_d\hspace{1cm}\mbox{and}\hspace{1cm}m(\psi(G))=\mu(G).
\end{equation}
We claim that, for each Borel subset $A$ of $\mathbb{R}^d\setminus\{0\}$, $\psi^{-1}(A)\subseteq \Sigma_S\times \mathcal{L}$. To this end, we write
\begin{equation*}
\psi(\Sigma_S\times\mathcal{L})=\{\psi(G):G\in\Sigma\times \mathcal{L}\}
\end{equation*}
for the $\sigma$-algebra on $\mathbb{R}^d\setminus\{0\}$ induced by $\psi$. In view of Lemma \ref{lem:OpenRectangle}, $\psi(\Sigma_S\times\mathcal{L})$ contains every open subset of $\mathbb{R}^d\setminus\{0\}$ and therefore
\begin{equation*}
\mathcal{B}(\mathbb{R}^d\setminus\{0\})\subseteq\psi(\Sigma_S\times\mathcal{L}).
\end{equation*}
where $\mathcal{B}(\mathbb{R}^d\setminus\{0\})$ denotes the $\sigma$-algebra of Borel subsets of $\mathbb{R}^d\setminus\{0\}$ thus proving our claim. 

Together, the results of the two preceding paragraphs show that, for each $A\in\mathcal{B}(\mathbb{R}^d\setminus\{0\})$, $\psi^{-1}(A)\subseteq \Sigma_S\times\mathcal{L}$ and $m(A)=\mu(\psi^{-1}(A))$. Upon noting that $\Sigma_S\times\mathcal{L}\subseteq \Sigma$, we immediately obtain the following statement. For each $A\in\mathcal{B}(\mathbb{R}^d\setminus\{0\})$,
\begin{equation}\label{eq:Good2}
\psi^{-1}(A)\subseteq \Sigma\hspace{1cm}\mbox{and}\hspace{1cm}m(A)=\mu(\psi^{-1}(A)).
\end{equation}
In comparing \eqref{eq:Good1} and \eqref{eq:Good2} with Properties \ref{property:PushforwardLemma1} and \ref{property:PushforwardLemma2} of Lemma \ref{lem:PushforwardLemma} and upon noting that $(S\times(0,\infty),\Sigma,\mu)$ is the completion of $(S\times(0,\infty),\Sigma_S\times\mathcal{L},\mu)$ and $(\mathbb{R}^d\setminus\{0\},\mathcal{M}_d,m)$ is the completion of $(\mathbb{R}^d\setminus\{0\},\mathcal{B}(\mathbb{R}^d\setminus\{0\}),m)$, Item \ref{item:MainIntegrationFormula1} of the theorem follows immediately from Lemma \ref{lem:PushforwardLemma}.

It remains to prove Item \ref{item:MainintegrationFormula2}. To this end, let $f:\mathbb{R}^d\to\mathbb{C}$ be Lebesgue measurable. Because $\mathcal{M}_d=\{A\subseteq \mathbb{R}^d\setminus\{0\}:\psi^{-1}(A)\in\Sigma\}$, it follows immediately that $f\circ\psi$ is $\Sigma$-measurable \textcolor{blue}{Add the details here: We can use the statement that $f:\mathbb{R}^d\to\mathbb{C}$ is Lebesgue measurable if and only if $f^{-1}(B)$ is a Lebesgue measurable subset of $\mathbb{R}^d$ for every $B\in\mathcal{B}(\mathbb{C})$}. In the case that $f\geq 0$, Item \ref{item:MainIntegrationFormula1} guarantees that
\begin{equation*}
\int_{\mathbb{R}^d}f(x)\,dx=\int_{\mathbb{R}^d\setminus \{0\}}f(x)\,dx=\int_{S\times (0,\infty)}(f\circ \psi)(\eta,t) \,d\mu(\eta,t)
\end{equation*}
where we have used the fact that the $\{0\}\subseteq\mathbb{R}^d$ has Lebesgue measure $0$  (\textcolor{red}{It would be nice to have a good statement of the change of variables formula we're using} -- \textcolor{blue}{is this in Folland?}). From this, \eqref{eq:MainIntegrationFormula} follows from Item \ref{item:Fubini1} in Theorem \ref{thm:Fubini}. Finally, by applying the above result to $|f|\geq 0$, we obtains $f\in L^1(\mathbb{R}^d)$ if and only if $f\circ \psi\in L^1(S\times (0,\infty),\Sigma,\mu)$ and, in this case, \eqref{eq:MainIntegrationFormula} follows directly from Item \ref{item:Fubini2} of Theorem \ref{thm:Fubini},
\end{proof}

\noindent Using Theorem \ref{thm:MainIntegrationFormula}, we are able to establish the following proposition. 

\begin{proposition}\label{prop:Regular}
Consider the finite Borel measure $\sigma$ on the measure space $(S,\Sigma_S,\sigma)$. For each $F\in\Sigma_S$,
\begin{equation}\label{eq:OuterRegular}
\sigma(F)=\inf\{\sigma(\mathcal{O}):F\subseteq\mathcal{O}\subseteq S\mbox{ and $\mathcal{O}$ is open}\}
\end{equation}
and
\begin{equation}
\sigma(F)=\sup\{\sigma(K):K\subseteq F\subseteq S\mbox{ and $K$ is compact}\}.
\end{equation}
In particular, $\sigma$ is a Radon measure \textcolor{red}{(Need a citation -- Folland)}.
\end{proposition}
\begin{proof}
Given that $S$ is compact and $\sigma$ is finite, it suffices to prove \eqref{eq:OuterRegular}, i.e., it suffices to prove the statement: For each $F\in \Sigma_S$ and $\epsilon>0$, there is an open subset $\mathcal{O}$ of $S$ containing $F$ for which 
\begin{equation*}
\sigma(\mathcal{O}\setminus F)<\epsilon.
\end{equation*}
To this end, let $F\in \Sigma_S$ and $\epsilon>0$. Given that $\widetilde{F}$ is a Lebesgue measurable subset of $\mathbb{R}^d$ and the Lebesgue measure $m$ is outer regular, there exists an open set $U\subseteq \mathbb{R}^d$ for which $\widetilde{F}\subseteq U$ and
\begin{equation}\label{eq:LebesgueOuter}
m(U\setminus \widetilde{F})=m(U)-m(\widetilde{F})<\epsilon/(2\tr E).
\end{equation}
Since $\widetilde{F}$ is necessarily a subset of the open set $B\setminus\{0\}$, we assume without loss of generality that $U\subseteq B\setminus\{0\}$.
For each $0<t<1$, consider the open set
\begin{equation*}
\mathcal{O}_t=S\cap\left( t^{-E}U\right)
\end{equation*}
in $S$. Observe that, for each $x\in F$, $t^E x\in \widetilde{F}\subseteq U$ and therefore $x\in \mathcal{O}_t$. Hence, for each $0<t<1$, $\mathcal{O}_t$ is an open subset of $S$ containing $F$. 

We claim that there is at least one $t_0\in (0,1)$ for which 
\begin{equation}\label{eq:GoodIneq}
m(\widetilde{\mathcal{O}_{t_0}})< m(U)+\epsilon/(2\tr E).
\end{equation}
To prove the claim, we shall assume, to reach a contradiction, that 
\begin{equation*}
m(\widetilde{\mathcal{O}_{t}})\geq m(U)+\epsilon/(2\tr E)
\end{equation*}
for all $0<t<1$. By virtue of Theorem \ref{thm:MainIntegrationFormula},
\begin{equation*}
m(U)=\int_{0}^\infty\left(\int_S \chi_{U}(t^E\eta)\,d\sigma(\eta)\right)t^{\tr E-1}\,dt.
\end{equation*}
Upon noting that $U\subseteq B\setminus\{0\}$, it is easy to see that
\begin{equation*}
U=\bigcup_{0<s<1}s^E\mathcal{O}_s
\end{equation*}
and
\begin{equation*}
t^E\eta\in \bigcup_{0<s<1}s^E\mathcal{O}_s
\end{equation*}
if and only if $0<t<1$ and $\eta\in \mathcal{O}_t$. Consequently,
\begin{eqnarray*}
m(U)&=&\int_0^1\left(\int_S\chi_{\mathcal{O}_t}(\eta)\,d\sigma(\eta)\right)t^{\tr E-1}\,dt\\
&=&\int_0^1\sigma(\mathcal{O}_t)t^{\tr E-1}\,dt\\
&=&\int_0^1 (\tr E)m(\widetilde{\mathcal{O}_t})t^{\tr E-1}\,dt.
\end{eqnarray*}
Upon making use of our supposition, we have
\begin{equation*}
\int_0^1(\tr E) m(\widetilde{\mathcal{O}_t})t^{\tr E-1}\,dt\geq \int_0^1(\tr E)(m(U)+\epsilon/(2\tr E))t^{\tr E-1}\,dt=m(U)+\epsilon/(2\tr E)
\end{equation*}
and so
\begin{equation*}
m(U)\geq m(U)+\epsilon/(2\tr E),
\end{equation*}
which is impossible. Thus, the stated claim is true.

Given any such $t_0$ for which \eqref{eq:GoodIneq} holds, set $\mathcal{O}=\mathcal{O}_{t_0}$. As previously noted, $\mathcal{O}$ is an open subset of $S$ which contains $F$. In view of \eqref{eq:LebesgueOuter} and \eqref{eq:GoodIneq}, we have
\begin{equation*}
m(\widetilde{\mathcal{O}})-m(\widetilde{F})<m(U)-m(\widetilde{F})+\epsilon/(2\tr E)<\epsilon/(2\tr E)+\epsilon/(2\tr E)=\epsilon/\tr E
\end{equation*}
and therefore
\begin{equation*}
\sigma(\mathcal{O}\setminus F)=\sigma(\mathcal{O})-\sigma(F)=\tr E(m(\widetilde{\mathcal{O}})-m(\widetilde{F}))<\epsilon,
\end{equation*}
as desired.
\end{proof}

\begin{corollary}
The completion of the measure space $(S,\mathcal{B}(S),\sigma)$ is $(S,\Sigma_S,\sigma)$. In particular, the latter space is complete and every $F\in \Sigma_S$ is of the form $F=G\cup H$ where $G$ is a Borel set and $H$ is a subset of a Borel set $Z$ with $\sigma(Z)=0$.
\end{corollary}

\textcolor{red}{We should think about maybe calling this a proposition. }
\begin{proof}
Let us denote by $(S,\overline{\mathcal{B}(S)},\overline{\sigma})$ the completion of the measure space $(S,\mathcal{B}(S),\sigma)$. Our job is to show that $\overline{\mathcal{B}(S)}=\Sigma_S$ and $\overline{\sigma}(F)=\sigma(F)$ for all $F$ in this common $\sigma$-algebra. 

First, let $F\in\overline{\mathcal{B}(S)}$ which is, by definition, a set of the form $F=G\cup H$ where $G\in\mathcal{B}(S)$ with $\overline{\sigma}(F)=\sigma(G)$ and $H\subseteq G_0\in\mathcal{B}(S)$ with $\sigma(G_0)=0$. In view of Proposition \ref{prop:BorelContainment}, $\widetilde{G}\in \mathcal{M}_d$, $\widetilde{H}\subseteq \widetilde{G_0}\in\mathcal{M}_d$ and we have
\begin{equation*}
m(\widetilde{G_0})=\frac{1}{\tr E}\sigma(G_0)=0
\end{equation*}
Since $(\mathbb{R}^d\setminus\{0\},\mathcal{M}_d,m)$ is complete, we conclude that $\widetilde{H}\in\mathcal{M}_d$ with $m(\widetilde{H})=0$ and therefore $H\in\Sigma_S$ with $\sigma(H)=(\tr E)m(\widetilde{H})=0$. It follows that $F=G\cup H\in\Sigma_S$ and
\begin{equation*}
\overline{\sigma}(F)=\sigma(G)\leq \sigma(F)\leq\sigma(G)+\sigma(H)=\sigma(G)+0=\overline{\sigma}(F).
\end{equation*}
It remains only to prove that $\Sigma_S\subseteq\overline{\mathcal{B}(S)}$. To this end, let $F\in\Sigma_S$ be arbitrary but fixed. By appealing to Proposition \ref{prop:Regular}, for each integer $n\in\mathbb{N}$, there exists a compact set $F_n\subseteq F$ for which
\begin{equation*}
\sigma(F\setminus F_n)=\sigma(F)-\sigma(F_n)<1/n.
\end{equation*}
Set
\begin{equation*}
G=\bigcup_{n=1}^\infty F_n\subseteq F
\end{equation*}
and $H=F\setminus G$. 
We observe that $G$ is a Borel set (in fact, an $F_\sigma$ set) and
\begin{equation*}
\sigma(H)=\sigma(F)-\sigma(G)=\sigma(F)-\lim_{n\to\infty}\sigma(F_n)=0,
\end{equation*}
by the continuity of measure. We have shown that
\begin{equation*}
F=G\cup H
\end{equation*}
where $G\in\mathcal{B}(S)$ and $H\in\Sigma_S$ with $\sigma(H)=0$. It remains to find a Borel set $G_0\supseteq H$ for which $\sigma(G_0)=0$. To this end, we again appeal to Proposition \ref{prop:Regular} to form a collection of open sets $\{\mathcal{O}_n\}_{n=1}^\infty$ such that, for each $n\in\mathbb{N}$, $H\subseteq \mathcal{O}_n$ and $\sigma(\mathcal{O}_n)=\sigma(\mathcal{O}_n)-\sigma(H)<1/n$. Finally, consider
\begin{equation*}
G_0=\bigcap_{n=1}^\infty\mathcal{O}_n,
\end{equation*}
which is necessarily a Borel set (in fact, a $G_\delta$-set) containing $H$ and, by the continuity of measure, has $\sigma(G_0)=0$, as desired.
\end{proof}

\textcolor{blue}{Some questions (I think the third is easiest):
\begin{enumerate}
\item To what extent does $\sigma$ depend on $E$? In other words, if $E,E'\in\Exp(P)$ (which necessarily have $\tr E=\tr E'$ in view of Section 2 of \cite{Randles2017}), would our construction have yielded the same measure $\sigma$ had we instead used the dilation $T_t=t^{E'}$ for everything? They should be closely related if not equal. But I'm really curious about this. For a little background reading, see the material near Proposition 2.3 of \cite{Randles2017}. 
\item We discussed the fact that the surface measure on the (usual) sphere $\mathbb{S}^{d-1}$ was the unique Radon measure which was rotationally invariant and satisfied $\sigma_d(\mathbb{S}^{d-1})=d\cdot m(\mathbb{B})$. I suspect that we also have some characterization for our measure $\sigma$ on $S$ -- in fact, this is why I suspect that $\sigma$ might not depend on the choice of $E$. Here are two possible conjectures and I really don't know if they are true:
\begin{conjecture} For any $O\in\Sym(P)$ and $F\in\Sigma_S$,
\begin{equation}\label{eq:Conjec1}
\sigma(O F)=\sigma(F).
\end{equation} 
\end{conjecture}
\begin{conjecture}
If $\sigma$ does not depend on $E$ (hence the construction produces the same measure $\sigma$ regardless of which $E\in\Exp(P)$ is chosen) and \eqref{eq:Conjec1} holds, $\sigma$ is the unique Radon measure on $S$ which satisfies \eqref{eq:Conjec1} and $\sigma(S)=\tr E m(B)$.
\end{conjecture}
\item There are many polynomials (and positive-definite continuous functions) that have $S$ as their ``unital" level set. For example: For any $\alpha>0$, $Q_{\alpha}(\xi):=(P(\xi))^\alpha$ is continuous and has
\begin{equation*}
S=\{\eta\in\mathbb{R}^d:Q_{\alpha}(\xi)=1\}.
\end{equation*}
Further, $Q_\alpha$ is positive-homogeneous\footnote{and is a polynomial precisely when $\alpha=1,2,\dots$.} and has $\Exp(Q_{\alpha})=\Exp(P)/\alpha$ in the senses that $E_\alpha\in \Exp(Q_\alpha)$ if and only if $E_\alpha=E/\alpha$ for $E\in \Exp(P)$. If you use $E_{\alpha}=E/\alpha$ to construct a measure $\sigma_\alpha$ on $S$ via the above construction, how is $\sigma_\alpha$ related to $\sigma$? I think I have an idea about this, but you should try it. Another question: Are there other polynomials (which are not powers of $P$) that have $S$ as their unital level set? 
\end{enumerate}}




\textcolor{blue}{{\Large Okay!}, I have an answer to Question 1. It's given below. I still haven't addressed the invariance under $\Sym(P)$ which, as you know, is a stronger statement. }
\subsection{Does this construction depend on $E$?}

Let $P$ be a positive-homogeneous polynomial and let $S=\{\eta\in\mathbb{R}^d:P(\eta)=1\}$. In the preceding section, we selected $E\in\Exp(P)$ and used it (via its associated one-parameter group of dilations $T_t=t^E$) to construct the measure $\sigma$ on $S$ for which \eqref{eq:MainIntegrationFormula} holds. As mentioned \textcolor{red}{earlier}, $\Exp(P)$ isn't necessarily a singleton and so we are led to ask: 
\begin{quote}\textit{In what sense, if any, does the above construction depend on $E$?} 
\end{quote}
To answer this question, let $E_1,E_2\in\Exp(P)$  (which are possibly different) and consider the associated respective measure spaces $(S,\Sigma_{S,1},\sigma_1)$ and $(S,\Sigma_{S,2},\sigma_2)$ produced via the above construction. 


\begin{proposition}\label{prop:Endependence}
These measure spaces are the same, i.e., $\Sigma_{S,1}=\Sigma_{S,2}$ and $\sigma_1=\sigma_2$. In other words, the surface measure $\sigma$ from the previous section depends only on $P$. 
\end{proposition}
\begin{proof}
In view of the two preceding \textcolor{red}{results}, it suffices to show that 
\begin{equation*}
\sigma_1(F)=\sigma_2(F)
\end{equation*}
for all $F\in \mathcal{B}(S)\subseteq \Sigma_{S,1}\cap\Sigma_{S,2}$. To this end, we let $F\in\mathcal{B}(S)$ be arbitrary but fixed. 

Given $n\in\mathbb{N}$, using the regularity of the measures $\mu_1$ and $\mu_2$, select open sets $\mathcal{O}_{n,1},\mathcal{O}_{n,2}$ and compact sets $K_{n,1},K_{n,2}$ for which
\begin{equation*}
K_{n,j}\subseteq F\subseteq \mathcal{O}_{n,j}\hspace{1cm}\mbox{and}\hspace{1cm}\sigma_j(\mathcal{O}_{n,j}\setminus K_{n,j})<1/n
\end{equation*}
for $j=1,2$. Observe that $K_n=K_{n,1}\cup K_{n,2}$ is a compact set, $\mathcal{O}_n=\mathcal{O}_{n,1}\cap\mathcal{O}_{n,2}$ is an open set and $K_n\subseteq F\subseteq \mathcal{O}_n$. Further, 
\begin{equation*}
\sigma_j(\mathcal{O}_n\setminus K_n)\leq \sigma_j(\mathcal{O}_{n,j}\setminus K_{n,j})<1/n
\end{equation*}
for $j=1,2$ Given that $\mathcal{O}_n$ is open in $S$, $\mathcal{O}_n=S\cup U_n$ where $U_n$ is an open subset of $\mathbb{R}^d$ and, because that $S$ is compact, $K_n=K_n\cap S$ is a compact subset of $\mathbb{R}^d$. By virtue of Urysohn's lemma, let $\phi_n:\mathbb{R}^d\to [0,1]$ be a continuous function which is compactly supported in $U_n$ and for which $\phi_n(x)=1$ for all $x\in K_n$. Using this sequence of functions $\{\phi_n\}$, we establish the following two facts: 

\begin{lemma}
For $n\in\mathbb{N}_+$ and $j=1,2$, $g_{n,j}:(0,\infty)\to\mathbb{R}$ defined by
\begin{equation*}
g_{n,j}(t)=\int_S\phi_n(t^{E_j}\eta)\,d\sigma_j(\eta).
\end{equation*}
is continuous.
\end{lemma}
\begin{subproof}
First, we note that, for each $t\in (0,\infty)$, the above integral makes sense because $\eta\mapsto \phi_n(t^{E_j}\eta)$ is Borel measurable (because it's continuous on $S$) and non-negative. Let $\epsilon>0$ and $t_0\in (0,\infty)$ be arbitrary but fixed. It is clear that the function $S\times (0,\infty)\ni (\eta,t)\mapsto \phi_n(t^{E_j}\eta)$ is continuous on its domain and therefore, in view of the compactness of $S$, we can find a $\delta>0$ for which
\begin{equation*}
|\phi_n(t^{E_j}\eta)-\phi_n(t_0^{E_j}\eta)|\leq\frac{\epsilon}{2\sigma_j(S)}\hspace{1cm}\mbox{whenever}\hspace{1cm}|t-t_0|<\delta
\end{equation*}
for all $\eta\in S$. \textcolor{blue}{This is a fairly standard ``uniform" continuity compactness argument. You should try to prove it.} The triangle inequality guarantees that
\begin{equation*}
|g_{n,j}(t)-g_{n,j}(t_0)|\leq \int_S|\phi_n(t^{E_j}\eta)-\phi_n(t_0^{E_j}\eta)|\,d\sigma_j(\eta)\leq\epsilon/2<\epsilon
\end{equation*}
whenever $|t-t_0|<\delta$.
\end{subproof}
\begin{lemma}
For $j=1,2$,
\begin{equation*}
\sigma_j(F)=\lim_{n\to\infty}g_{n,j}(1).
\end{equation*}
\end{lemma}
\begin{subproof}
For each $n\in\mathbb{N}$,
\begin{equation*}
g_{n,j}(1)=\int_{S}\phi_n(\eta)\,d\sigma_j(\eta)
\end{equation*}
because $1^{E_j}=I$. By construction, we have $\chi_{K_n}(\eta)\leq\phi_{n}(\eta)\leq \chi_{\mathcal{O}_n}(\eta)$ for all $\eta\in S$ and $n\in\mathbb{N}_+$ and therefore
\begin{equation*}
\sigma_j(K_n)\leq g_{n,j}(1)\leq \sigma_{j}(\mathcal{O}_n)
\end{equation*}
by the monotonicity of the integral. Since
\begin{equation*}
\sigma_j(F)=\lim_{n\to\infty}\sigma_j(K_n)=\lim_{n\to\infty}\sigma_j(\mathcal{O}_n)
\end{equation*}
in view of our choice of $\mathcal{O}_n$ and $K_n$, the desired result follows immediately from the preceding inequality (and the squeeze theorem).
\end{subproof}
\noindent Given any $0<r<1< s$ and $n\in\mathbb{N}$, consider the function $f=f_{n,r,s}:\mathbb{R}^d\to [0,1]$ given by
\begin{equation*}
f(x)=\phi_n(x)\chi_{[r,s]}(P(x))
\end{equation*}
for $x\in\mathbb{R}^d$. It is clear that $f$ is Lebesgue measurable on $\mathbb{R}^d$ and non-negative. By virtue of Theorem \ref{thm:MainIntegrationFormula} (applied to the two measures $\sigma_1$ and $\sigma_2$), we have
\begin{equation}\label{eq:SameMeasure1}
\int_0^\infty \int_Sf(t^{E_1}\eta)\,d\sigma_1(\eta)t^{\tr E_1-1}\,dt=\int_{\mathbb{R}^d}f(x)\,dx=\int_0^\infty \int_Sf(t^{E_2}\eta)\,d\sigma_2(\eta)t^{\tr E_2-1}\,dt
\end{equation}
Upon noting that
\begin{equation*}
f(t^{E_j}\eta)=\phi_n(t^{E_j}\eta)\chi_{[r,s]}\left((P(t^{E_j}\eta)\right)=\phi_n(t^{E_j}\eta)\chi_{[r,s]}(tP(\eta))=\chi_{[r,s]}(t)\phi_n(t^{E_j}\eta)
\end{equation*}
for $\eta\in S$, $t\in (0,\infty)$, and $j=1,2$, we have
\begin{equation*}
\int_0^\infty\int_S f(t^{E_j}\eta)\,d\sigma_j(\eta)t^{\tr E_j-1}\,dt=\int_{[r,s]}\int_S\phi_n(t^{E_j}\eta)\,d\sigma_j(\eta) t^{\tr E_j-1}\,dt=\int_{[r,s]}g_{n,j}(t)t^{\tr E_j-1}\,dt
\end{equation*}
for $j=1,2$. Given that $t\mapsto g_{n,j}(t)t^{\tr E_j-1}$ is continuous and bounded on $[r,s]$ (which is in part due to our lemma), the final integral above can be interpreted as a Riemann integral. That is, for $j=1,2$ and $0<r<1<s$,
\begin{equation}\label{eq:SameMeasure2}
\int_0^\infty \int_S f(t^{E_j}\eta)\,d\sigma_j(\eta)t^{\tr E_j-1}\,dt=\int_r^sg_{n,j}(t)t^{\tr E_j-1}\,dt.
\end{equation}
In view of \eqref{eq:SameMeasure1} and \eqref{eq:SameMeasure2}, we conclude that
\begin{equation*}
\int_r^s g_{n,1}(t)t^{\tr E_j-1}\,dt=\int_r^s g_{n,2}(t)t^{\tr E_2-1}\,dt
\end{equation*}
for all $0<r<1<s$. By virtue of the continuity of the integrands, the fundamental theorem of calculus guarantees that
\begin{equation*}
g_{n,1}(t)t^{\tr E_1-1}=g_{n,2}(t)t^{\tr E_2-1}
\end{equation*}
for all $t\in [r,s]$. In particular, at $t=1$, we have $g_{n,1}(1)=g_{n,2}(1)$ \textcolor{blue}{We can actually conclude that the $g$'s are the same everywhere because the traces are the same}. In view of the second lemma, it follows that
\begin{equation*}
\sigma_1(F)=\lim_{n\to\infty}g_{n,1}(1)=\lim_{n\to\infty}g_{n,2}(1)=\sigma_2(F).
\end{equation*}
\end{proof}

\noindent \textcolor{blue}{One thing I've been thinking about is that the above proof provides a method by which $\sigma$ might be able to be characterized without making any reference whatsoever to $E$. I'm not completely sure about this, but if we follow the construction, we've essentially shown that
\begin{equation*}
\sigma(F)=\lim_{n\to\infty}F_n'(1)
\end{equation*}
where
\begin{equation*}
F_n(s)=\int_{\mathbb{R}^d}f_{n,s}(x)\,dx
\end{equation*}
where
\begin{equation*}
f_{n,s}=\phi_n(x)\chi_{[1/2,s]}(P(x)).
\end{equation*}
I wonder if this provides us with a way to write (and I'm being very loose about this): Denote by $\mathcal{A}_F$ the set of continuous and compactly supported functions $\phi:\mathbb{R}^d\to [0,1]$ with $\phi(x)=1$ for all $x\in F$. Then
\begin{equation*}
\sigma(F)=(?)\inf\left\{\frac{d}{ds}\left(\int_{\mathbb{R}^d}\phi(x)\chi_{[1/2,s]}(P(x))\,dx\right)\bigg\vert_{s=1}:\phi\in \mathcal{A}_F\right\}
\end{equation*}
or maybe
\begin{equation*}
\sigma(F)=(?)\inf\left\{\lim_{h\to 0}\frac{1}{h}\int_{\mathbb{R}^d}\phi(x)\chi_{[1,1+h]}(P(x))\,dx:\phi\in \mathcal{A}_F\right\}
\end{equation*}
}







\begin{convention}
For $O\in \Sym(P)$ and $F\subseteq S$, we define
\begin{equation*}
    O(F) \coloneqq \{ O \eta : \eta \in F \}.
\end{equation*}
\end{convention}








\begin{lemma}\label{lem:ExpP}
$\Exp(P)$ is invariant under conjugation by $\Sym(P)$. That is, for any  $O \in \Sym{(P)} $
\begin{equation*}
    \Exp(P) = O^\top \Exp(P) O.
\end{equation*}
\end{lemma}

\begin{proof}
Since $\Sym(P) < \OdR{}$, if $O\in \Sym{P}$ then $O^\top = O^{-1} \in \Sym{P}$. Thus,
\begin{equation}\label{eq:OOO}
    P(t^E \eta) = tP(\eta) = tP(O\eta) = P(t^E O\eta) = P(O^\top t^E O \eta) = P(t^{O^\top E O }\eta)
\end{equation}
for all $t>0,\eta\in \R^d$. It follows that $O^\top E O \in \Exp(P)$ for every $E\in \Exp(P)$ and $O\in \Sym(P)$.  
Now, if $E_i, E_j\in \Exp(P)$ and $E_i\neq E_j$, then $O^\top E_i O \neq O^\top E_j O$ since $O,O^\top\in \OdR{}$. Thus the conjugation map induced by $O$, $\varphi_O: \Exp{P}\to \Exp{P}$ defined by 
\begin{equation*}
    \varphi_O (E) = O^\top E O
\end{equation*}
is one-to-one. Further, for any $E\in \Exp(P)$, a similar argument as \eqref{eq:OOO} shows that $OEO^\top \in \Exp(P)$ and satisfies
\begin{equation*}
    \varphi_O(OEO^\top) = O^\top (OEO^\top) O = E\in \Exp{(P)}.
\end{equation*}
So, $\varphi_O$ is bijective, which means 
\begin{equation*}
    \Exp(P) = O^\top \Exp(P) O = O \Exp{(P)} O^\top, \quad \text{for all } O\in \Sym{(P)}.
\end{equation*}
\end{proof}



\begin{corollary}
For any $O\in\Sym(P)$ and $F\in\Sigma_S$,
\begin{equation*}
\sigma(O (F))=\sigma(F).
\end{equation*} 
That is, the measure $\sigma$ is invariant under the symmetry group $\Sym(P)$ of $P$. 
\end{corollary}



\begin{proof}
Let $E\in \Exp(P)$ and $O\in \Sym(P)$ be given. In view of Lemma \ref{lem:ExpP}, we have 
\begin{equation*}
    E' = O^\top E O \in \Exp(P).
\end{equation*}
Consider the measure spaces constructed from $E$ and $E'$ respectively: $(S, \Sigma_{S,E},\sigma_E)$ and $(S,\Sigma_{S,E'},\sigma_{E'})$. By virtue of Proposition \ref{prop:Endependence}, these measure spaces are the same, i.e., $\Sigma_{S,E} = \Sigma_{S,E'}$ and $\sigma_{E} = \sigma_{E'}$. For convenience, let us call these equivalent measure spaces the triple $(S,\Sigma_S,\sigma)$. 

Let $F\in \Sigma_S$ be given. We will show that $O(F)\in \Sigma_S$. To this end, we first notice that since $O\in \Sym(P) < \OdR$, we have $O O^\top = I$. As a result, we can write
\begin{equation*}
    \widetilde{O(F)} = \bigcup_{0<t<1}t^E O(F) = \bigcup_{0<t<1} O O^\top t^E O(F) = \bigcup_{0<t<1}O t^{O^\top E O} F = O\lp \bigcup_{0<t<1}t^{E'} F\rp.
\end{equation*}
We observe that the set $\bigcup_{0<t<1} t^{E'}F$ is Lebesgue measurable since $F\in \Sigma_S = \Sigma_{S,E'}$. By the orthogonal invariance of the Lebesgue measure, $\widetilde{O(F)}$ is also Lebesgue measurable, with
\begin{equation*}
    m (\widetilde{O(F)} ) = m\lb O \lp \bigcup_{0<t<1}t^{E'}F \rp \rb =  m\lp \bigcup_{0<t<1}t^{E'}F \rp.
\end{equation*}
Therefore, $O(F)\in \Sigma_S$, with which it makes sense to ask for the measure $\sigma$ of $O(F)$:
\begin{equation*}
    \sigma(O(F)) = (\tr E)m(\widetilde{O(F)}).
\end{equation*}
Now, in view of Proposition \ref{prop:Endependence} and the fact that $\tr E = \tr E'$ for $E,E'\in \Exp(P)$, we have that
\begin{equation*}
    m\lp \bigcup_{0<t<1}t^{E'}F \rp = \frac{\sigma_{E'}(F)}{\tr E' }  
    = 
    \frac{\sigma_E(F)}{\tr E'} =  \frac{\sigma_E(F)}{\tr E}   = m\lp \bigcup_{0<t<1} t^E F  \rp.
\end{equation*}
From the last three equations, we conclude that 
\begin{equation*}
    \sigma(O(F)) = \sigma(F).
\end{equation*}
Therefore, the measure $\sigma$ is invariant under symmetry group $\Sym(P)$ of $P$.  
\end{proof}


\section{Using a smooth structure on $S$ to compute $\sigma$.}

Up to this point, we have not used anything about the fine (smooth) structure of $S$. We've relied only on results from basic point-set topology (including the compactness of $S$, the continuity of $P$ and $\psi$) and measure theory to construct and develop basic results concerning the measure $\sigma$ and the resulting polar coordinate integration formula \eqref{eq:MainIntegrationFormula}. In this section, we shall use our additional knowledge of $S$, particularly that is is a smooth manifold, to get a handle on computing integrals of the form
\begin{equation*}
\int_S g(\eta)\,d\sigma(\eta)
\end{equation*}
for (sufficiently nice) $\Sigma_S$-measurable functions $g$. 

To this end, let us view $S$ through the eyes of smooth manifold theory. To avoid some trivialities, we assume henceforth that $d>1$. Given $E\in\Exp(P)$, observe that
\begin{equation*}
P(\xi)=\frac{d}{dt}\left(tP(\xi)\right)=\frac{d}{dt}(P(t^E\xi))=\left(\nabla P\right)(t^E\xi)\cdot \left(t^{E-I}E\xi\right)
\end{equation*}
for $t>0$ and $\xi\in\mathbb{R}^d$. By evaluating this expression at $t=1$, we see that
\begin{equation*}
P(\xi)=\nabla P(\xi)\cdot(E\xi)
\end{equation*}
for all $\xi\in\mathbb{R}^d$. In particular, $\eta\in S$ if and only if  and 
\begin{equation*}
\nabla P(\eta)\cdot (E\eta)=1
\end{equation*}
from which we see that $P$'s gradient, $\nabla P$, is non-vanishing on $S$. As a consequence, $S$ is a smooth embedded compact hypersurface of $\mathbb{R}^d$. (\textcolor{blue}{Huan, I'm having a hard time finding readable references for these words and I'm not sure what depth you should understand them. The reference I'm using is John M. Lee \textit{Introduction to Smooth Manifolds} -- though it is pretty thorough, it is NOT an easy reference. Specifically, the conclusion I made above is a consequence of Corollary 8.10 therein.)} 


Let $\mathcal{A}=\{(\mathcal{O}_\alpha,\varphi_{\alpha})\}$ be a smooth atlas on $S$. Given that $S$ is compact, we will assume (without loss of generality) that this is a finite atlas \textcolor{red}{I'm not sure this is what we want to do, it just makes all the partition of unity business easier}. Using $\mathcal{A}$, we shall construct an atlas $\widetilde{\mathcal{A}}$ on $B_0:=B\setminus \{0\}$ in the following way: For each index $\alpha$, consider the map $\rho_{\alpha}:V_\alpha \to \widetilde{\mathcal{O}_\alpha}= \psi(\mathcal{O}_\alpha\times(0,1))$ defined by
\begin{equation*}
\rho_{\alpha}(\mathbf{x},t)=\psi(\varphi_\alpha^{-1}(\mathbf{x}),t)=t^E\varphi^{-1}_{\alpha}(\mathbf{x})
\end{equation*}
for all
\begin{equation*}
(\mathbf{x},t)=(x_1,x_2,\dots,x_{d-1},t)\in V_\alpha:=\varphi_{\alpha}(\mathcal{O}_\alpha)\times (0,1)\subseteq\mathbb{R}^d.
\end{equation*}
Given that $\varphi_\alpha$ and $\psi$ are homeomorphisms, it is easy to see that $\rho_{\alpha}$ is a homeomorphism with inverse $\rho_{\alpha}^{-1}=\widetilde{\varphi}_\alpha$ given by
\begin{equation*}
\widetilde{\varphi}_\alpha(\xi)=(\varphi_{\alpha}(P(\xi)^{-E}\xi),P(\xi))
\end{equation*}
for $\xi\in \widetilde{\mathcal{O}_a}$.
\begin{proposition}
\begin{equation*}
\widetilde{\mathcal{A}}=\{(\widetilde{\mathcal{O}_\alpha},\widetilde{\varphi}_\alpha)\}
\end{equation*}
is an atlas on the open submanifold $B_0=B\setminus\{0\}$ of $\mathbb{R}^d$.
\end{proposition}
\begin{proof}
Given that $\mathcal{A}$ is an atlas on $S$, we have
\begin{equation*}
\bigcup_\alpha \widetilde{\mathcal{O}_\alpha}=\bigcup_{\alpha}\psi(\mathcal{O}_\alpha\times (0,1))=\psi\left(\left(\bigcup_\alpha \mathcal{O}_\alpha\right)\times (0,1)\right)=\psi(S\times (0,1))=B_0.
\end{equation*}
It remains to show that every pair of charts in $\widetilde{\mathcal{A}}$ are smoothly compatible. To this end, set $U_\alpha=\widetilde{\varphi}_{\alpha}(\widetilde{\mathcal{O}_\alpha}\cap\widetilde{\mathcal{O}_\beta})\subseteq \mathbb{R}^d$ and $U_\beta=\widetilde{\varphi}_{\beta}(\widetilde{\mathcal{O}_\alpha}\cap\widetilde{\mathcal{O}_\beta})\subseteq \mathbb{R}^d$ and observe that $\widetilde{\varphi}_\beta\circ \widetilde{\varphi}_\alpha^{-1}:U_\alpha\to U_\beta$ is given by
\begin{eqnarray*}
\left(\widetilde{\varphi}_\beta\circ \widetilde{\varphi}_\alpha^{-1}\right)(\mathbf{x},t)&=&\widetilde{\varphi}_\beta\left(t^{E}\varphi_\alpha^{-1}(\mathbf{x})\right)\\
&=&\left(\varphi_{\beta}\left(P(t^E\varphi_{\alpha}^{-1}(\mathbf{x}))^{-E}\left(t^E\varphi_{\alpha}^{-1}(\mathbf{x})\right)\right),P\left(t^E\varphi_{\alpha}^{-1}(\mathbf{x})\right)\right)\\
&=&\left(\left(\varphi_\beta^{-1}\circ\varphi_\alpha\right)(\mathbf{x}),t\right)
\end{eqnarray*}
for $(\mathbf{x},t)\in U_\alpha$. By virtue of the fact that $\mathcal{A}$ is an atlas,  $\varphi_\beta^{-1}\circ\varphi_\alpha$ is necessarily a diffeomorphism and so it immediately follows that $\widetilde{\varphi}_\beta\circ \widetilde{\varphi}_\alpha^{-1}$ is a diffeomorphism.
\end{proof}
With the above atlas in mind, let $F\in\Sigma_S$ \textcolor{red}{Do we want to assume that $F$ is open and then use the regularity of $\sigma$?} be such that $F\subseteq \mathcal{O}_\alpha$ for some chart $\mathcal{O}_\alpha$ and set $U_\alpha=\varphi_{\alpha}(F)\subseteq \mathbb{R}^{d-1}$. It is easy to see that
\begin{equation*}
\widetilde{F}=\psi(F\times (0,1))=\widetilde{\varphi}_\alpha^{-1}(U\times (0,1))
\end{equation*}
Consequently, \textcolor{red}{Need a reference for doing this on an open submanifold of Euclidean space. Is this in Wade? Is the determinant always non-vanishing?}
\begin{equation}\label{eq:CoordinateFormulaTilde1}
m(\widetilde{F})=\int_{\widetilde{\varphi}_\alpha(\widetilde{F})} \left|\det(D\widetilde{\varphi}^{-1}_\alpha(\mathbf{u}))\right| \,d\mathbf{u}\\
=\int_0^1\left(\int_{U_\alpha}\left|\det(D(\rho_\alpha)(\mathbf{x},t))\right|\,dx\right)\,dt
\end{equation}
where we have written $\mathbf{u}=(\mathbf{x},t)$ and $d\mathbf{u}=dx\,dt$. For notational simplicity, we write $h=\varphi_{\alpha}^{-1}$and observe that
\begin{eqnarray*}
D(\rho_\alpha)(\mathbf{x},t)=D(t^Eh(\mathbf{x}))&=&\left(D_{\mathbf{x}}\left(t^Eh(\mathbf{x})\right)\big\vert|D_t\left(t^Eh(\mathbf{x})\right)\right)\\
&=&\left(t^E D_x h(\mathbf{x})\big\vert t^{E-I}(E(h(\mathbf{x})))\right)\\
&=&t^E\left(D_x h(\mathbf{x})\big\vert t^{-1}(E(h(\mathbf{x})))\right)
\end{eqnarray*}
for $(\mathbf{x},t)\in \varphi_\alpha(\mathcal{O}_\alpha)\times (0,1)$ where \textcolor{red}{Perhaps too many parentheses? Can we leave some notation suppressed? Also, maybe using $\mathbf{x}$ is a bad idea. What about $\mathbf{u}$?}
\begin{equation*}
D_x h(\mathbf{x})=D_x h(x_1,x_2,\dots,x_{d-1})=
\begin{pmatrix}
\frac{\partial h_1}{\partial x_1} & \frac{\partial h_1}{\partial x_2} & \cdots & \frac{\partial h_1}{\partial x_{d-1}}\\
\frac{\partial h_2}{\partial x_1} & \frac{\partial h_2}{\partial x_2} & \cdots & \frac{\partial h_2}{\partial x_{d-1}}\\
 \vdots & \vdots &\ddots & \vdots\\
\frac{\partial h_d}{\partial x_1} & \frac{\partial h_d}{\partial x_2} & \cdots & \frac{\partial h_d}{\partial x_{d-1}}\\
\end{pmatrix}.
\end{equation*}
Given $\varphi_\alpha^{-1}$ and $E$, define $\Omega_\alpha:\varphi_{\alpha}(\mathcal{O}_\alpha)\to \GldR$ by
\begin{equation*}
\Omega_\alpha(\mathbf{x})=\left(D_x h(\mathbf{x})\big\vert Eh(\mathbf{x})\right)=
\begin{pmatrix}
\frac{\partial h_1}{\partial x_1} & \frac{\partial h_1}{\partial x_2} & \cdots & \frac{\partial h_1}{\partial x_{d-1}} & (Eh)_1\\
\frac{\partial h_2}{\partial x_1} & \frac{\partial h_2}{\partial x_2} & \cdots & \frac{\partial h_2}{\partial x_{d-1}} & (Eh)_2\\
 \vdots & \vdots &\ddots & \vdots & \vdots \\
\frac{\partial h_d}{\partial x_1} & \frac{\partial h_d}{\partial x_2} & \cdots & \frac{\partial h_d}{\partial x_{d-1}} & (Eh)_d\\
\end{pmatrix}.
\end{equation*}
and, using properties of the determinant, we find
\begin{eqnarray*}
\lefteqn{\hspace{-1cm}\det \left((D\rho_\alpha)(\mathbf{x},t)\right)=\det(t^E)\det\left(D_x h(\mathbf{x})\big\vert t^{-1}E(h(\mathbf{x}))\right)}\\
&&\hspace{3cm}=t^{\tr E}t^{-1}\det (D_xh(\mathbf{x})\big\vert E(h(\mathbf{x})))=t^{\tr E-1}\det (\Omega_\alpha(\mathbf{x}))
\end{eqnarray*}
for $(\mathbf{x},t)\in U_\alpha\times (0,1)$. In view of \eqref{eq:CoordinateFormulaTilde1}, we have
\begin{equation*}
m\left(\widetilde{F}\right)=\int_0^1 t^{\tr E-1}\left(\int_{U_\alpha}|\det(\Omega_\alpha(\mathbf{x}))|\,dx\right)\,dt=\frac{1}{\tr E}\int_{U_\alpha}|\det(\Omega_\alpha(\mathbf{x}))|\,dx
\end{equation*}
and therefore
\begin{equation*}
\sigma(F)=(\tr E)m\left(\widetilde{F}\right)=\int_{U_\alpha}|\det(\Omega_\alpha(\mathbf{x}))|\,dx.
\end{equation*}
This computation allows us to prove the following result.
\begin{lemma}
For each $F\in\Sigma_S$ which is in the domain of some chart $(\mathcal{O}_\alpha,\varphi_\alpha)$, 
\begin{equation*}
\mathbf{x}\mapsto \chi_{\varphi_\alpha(F)}|\det(\Omega_\alpha (\mathbf{x}))|
\end{equation*}
is Lebesgue measurable on $\mathbb{R}^{d-1}$ and
\begin{equation*}
\sigma(F)=\int_{\varphi_\alpha(F)}|\det(\Omega_\alpha (\mathbf{x}))|\,dx.
\end{equation*}
\end{lemma}
\begin{proof}
Given $F\in\Sigma_S$ for which $F\subseteq\mathcal{O}_\alpha$, we invoke the outer regularity of $\sigma$ to produce a sequence of nested open sets $\{\mathcal{U}_n\}$ for which
\begin{equation*}
F\subseteq \mathcal{U}_{n+1}\subseteq\mathcal{U}_n\subseteq\mathcal{O}_\alpha
\end{equation*}
for all $n\in\mathbb{N}_+$ and for which
\begin{equation*}
\sigma(F)=\lim_{n\to\infty}\sigma(\mathcal{U}_n).
\end{equation*}
For each $n\in\mathbb{N}_+$, set
\begin{equation*}
f_n(\mathbf{x})=\chi_{\varphi_\alpha(\mathcal{U}_n)}(\mathbf{x})|\det(\Omega_\alpha(\mathbf{x}))|
\end{equation*}
for $\mathbf{x}\in\mathbb{R}^{d-1}$ \textcolor{red}{This has got to be Borel measurable}. By virtue of the computations preceding the lemma, we have
\begin{equation*}
\sigma(\mathcal{U}_n)=\int_{\varphi_\alpha(\mathcal{U}_n)}|\det(\Omega_\alpha(\mathbf{x}))|\,dx=\int_{\mathbb{R}^{d-1}}f_n(\mathbf{x})\,dx
\end{equation*}
for each $n\in\mathbb{N}_+$. Given that the sequence $\mathcal{U}_n$ is nested and decreasing, it follows that $\{f_n\}$ is a non-negative decreasing sequence of of measurable functions and consequently
\begin{equation*}
f(\mathbf{x})=\inf_{n\to\infty}f_n(\mathbf{x})
\end{equation*}
exits for each $\mathbf{x}\in\mathbb{R}^{d-1}$ and is Lebesgue measurable. Further, by the monotone convergence theorem, we have
\begin{equation*}
\sigma(F)=\lim_{n\to\infty}\sigma(\mathcal{U}_n)=\lim_{n\to\infty}\int_{\mathbb{R}^{d-1}}f_n(\mathbf{x})\,dx=\int_{\mathbb{R}^{d-1}}f(\mathbf{x})\,dx.
\end{equation*}
We then see that the lemma will be proved if we can show that
\begin{equation*}
f(\mathbf{x})=\chi_{\varphi_{\alpha}(F)}(\mathbf{x})|\det(\Omega_\alpha(\mathbf{x}))|
\end{equation*}
almost everywhere \textcolor{blue}{with respect to the $d-1$ Lebesgue measure}. \textcolor{red}{ahhhh -- I think we need inner regularity to prove this}
\end{proof}
\begin{theorem}
Let $\mathcal{A}=\{\mathcal{O}_\alpha,\varphi_\alpha\}$ be an atlas on $S$ and let $\{k_\alpha\}$ be a $C^\infty$ partition of unity subordinate to the cover $\{\mathcal{O}_\alpha\}$. Given a $\Sigma_S$-measurable function $g$, assume that $g\geq 0$ or $g\in L^1(S,\sigma)$.
\begin{enumerate}
\item If $\supp(g)\subseteq \mathcal{O}_\alpha$ for some $\alpha$, then
\begin{equation*}
\int_S g(\eta)\,d\sigma(\eta)=\int_{\mathcal{O}_\alpha}(g\circ\varphi_{\alpha}^{-1})(\mathbf{x})|\det(\Omega_\alpha(\mathbf{x}))|\,dx
\end{equation*}
\item In general,
\begin{equation*}
\int_S g(\eta)\,d\sigma(\eta)=\sum_{\alpha}\int_{\mathcal{O}_\alpha}(g\circ \varphi_\alpha^{-1})(\mathbf{x})(k_\alpha\circ\varphi_{\alpha}^{-1})(\mathbf{x})|\det(\Omega_\alpha(\mathbf{x}))|\,dx
\end{equation*}
\end{enumerate}
\end{theorem}
\begin{proof}
Thoughts: First, I don't think the stuff done before the proof can be done for arbitrary measurable sets. So, we should do it for open sets and then use the regularity of $\sigma$ to get the desired statement. Maybe this should be a lemma. 

Now, think about some statement that says: Given a chart $(\mathcal{O}_\alpha,\varphi_\alpha)$ can we use the above formula (and perhaps the fact that it shows absolute continuity with respect to Lebesgue measure to show that if $F\in \Sigma_S$ then $U_\alpha=\varphi_{\alpha}(F)$ is Lebesgue measurable?

Okay, now that that's done: Assume that $g\geq 0$. Assume first that $g$ is supported on the neighborhood of a chart. Given that $g$ is $\Sigma_S$ measurable, if we can prove the above paragraph, we obtain that $g\circ\varphi^{-1}_\alpha$ is Lebesgue measurable. So, now approximate by simple functions.
\end{proof}










































\begin{thebibliography}{99}

\bibitem{Bogachev2007}
V. I. Bogachev.
\newblock {\em {Measure Theory} Vol. 1}.
\newblock Springer-Verlag, 2007.

\bibitem{Folland1984}
Gerald Folland.
\newblock {\em {Real Analysis: Modern Techniques and Their Applications}}, {\em Pure \& Applied Mathematics}.
\newblock Wiley-Interscience, 1984

\bibitem{Randles2017}
Evan Randles and Laurent Saloff-Coste. 
\newblock {\em {``Convolution powers of complex functions on $\mathbb{Z}^d$."}
\newblock Revista Matem\'{a}tica Iberoamericana, vol. 33, no. 3, 2017, pp. 1045-1121.}


\bibitem{Rudin1987}
Walter Rudin.
\newblock {\em {Real and Complex Analysis}}, {\em McGraw-Hill Series in Higher Mathematics}.
\newblock WCB/McGraw-Hill, 1987.

\bibitem{Stein2005}
Elias Stein and Rami Shakarchi.
\newblock {\em {Real Analysis} Measure Theorem, Integration, \& Hilbert Spaces}, {\em Princeton Lectures in Analysis III}.
\newblock Princeton University Press, 2005.

\end{thebibliography}
\end{document}V