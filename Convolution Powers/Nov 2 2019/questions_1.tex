\documentclass{book}
\usepackage{physics}
\usepackage{graphicx}
\usepackage{caption}
\usepackage{amsmath}
\usepackage{bm}
\usepackage{authblk}
\usepackage{empheq}
\usepackage{amsfonts}
\usepackage{esint}
\usepackage[makeroom]{cancel}
\usepackage{dsfont}
\usepackage{centernot}
\usepackage{mathtools}
\usepackage{bigints}
\usepackage{amsthm}
\theoremstyle{definition}
\newtheorem{defn}{Definition}[section]
\newtheorem{prop}{Proposition}[section]
\newtheorem{rmk}{Remark}[section]
\newtheorem{thm}{Theorem}[section]
\newtheorem{exmp}{Example}[section]
\newtheorem{prob}{Problem}[section]
\newtheorem{sln}{Solution}[section]
\newtheorem*{prob*}{Problem}
\newtheorem{exer}{Exercise}[section]
\newtheorem*{exer*}{Exercise}
\newtheorem*{sln*}{Solution}
\usepackage{empheq}
\usepackage{hyperref}
\usepackage{tensor}
\usepackage{xcolor}
\hypersetup{
	colorlinks,
	linkcolor={black!50!black},
	citecolor={blue!50!black},
	urlcolor={blue!80!black}
}


\newcommand*\widefbox[1]{\fbox{\hspace{2em}#1\hspace{2em}}}

\newcommand{\p}{\partial}
\newcommand{\R}{\mathbb{R}}
\newcommand{\C}{\mathbb{C}}
\newcommand{\lag}{\mathcal{L}}
\newcommand{\nn}{\nonumber}
\newcommand{\ham}{\mathcal{H}}
\newcommand{\M}{\mathcal{M}}
\newcommand{\I}{\mathcal{I}}
\newcommand{\K}{\mathcal{K}}
\newcommand{\F}{\mathcal{F}}
\newcommand{\w}{\omega}
\newcommand{\lam}{\lambda}
\newcommand{\al}{\alpha}
\newcommand{\be}{\beta}
\newcommand{\x}{\xi}

\newcommand{\G}{\mathcal{G}}

\newcommand{\f}[2]{\frac{#1}{#2}}

\newcommand{\ift}{\infty}

\newcommand{\lp}{\left(}
\newcommand{\rp}{\right)}

\newcommand{\lb}{\left[}
\newcommand{\rb}{\right]}

\newcommand{\lc}{\left\{}
\newcommand{\rc}{\right\}}


\newcommand{\V}{\mathbf{V}}
\newcommand{\U}{\mathcal{U}}
\newcommand{\Id}{\mathcal{I}}
\newcommand{\D}{\mathcal{D}}
\newcommand{\Z}{\mathcal{Z}}

%\setcounter{chapter}{-1}


\makeatletter
\renewcommand{\@chapapp}{Part}
%\renewcommand\thechapter{$\bf{\ket{\arabic{chapter}}}$}
%\renewcommand\thesection{$\bf{\ket{\arabic{section}}}$}
%\renewcommand\thesubsection{$\bf{\ket{\arabic{subsection}}}$}
%\renewcommand\thesubsubsection{$\bf{\ket{\arabic{subsubsection}}}$}
\makeatother



\usepackage{subfig}
\usepackage{listings}
\captionsetup[lstlisting]{margin=0cm,format=hang,font=small,format=plain,labelfont={bf,up},textfont={it}}
\renewcommand*{\lstlistingname}{Code \textcolor{violet}{\textsl{Mathematica}}}
\definecolor{gris245}{RGB}{245,245,245}
\definecolor{olive}{RGB}{50,140,50}
\definecolor{brun}{RGB}{175,100,80}
\lstset{
	tabsize=4,
	frame=single,
	language=mathematica,
	basicstyle=\scriptsize\ttfamily,
	keywordstyle=\color{black},
	backgroundcolor=\color{gris245},
	commentstyle=\color{gray},
	showstringspaces=false,
	emph={
		r1,
		r2,
		epsilon,epsilon_,
		Newton,Newton_
	},emphstyle={\color{olive}},
	emph={[2]
		L,
		CouleurCourbe,
		PotentielEffectif,
		IdCourbe,
		Courbe
	},emphstyle={[2]\color{blue}},
	emph={[3]r,r_,n,n_},emphstyle={[3]\color{magenta}}
}


\begin{document}
\begin{center}
	\huge{QUESTIONS \#1}\\
	$\,$\\
	\normalsize{\today}\\
	\normalsize{Huan Bui}
\end{center}


Hi Evan, I tried to evaluate some (seemingly doable) integrals for the $d$-dimensional problem after our discussion and came up with some questions. 
\begin{enumerate}
	\item When we write the integral 
	\begin{align*}
	\int_{\mathbb{R}^d} e^{-iP(\xi) - ix\cdot \xi}\,d\xi,
	\end{align*}
	how is the measure $d\xi$ defined? When, say, $d=2$, is $d\xi$ just equal to $d x dy$ in the euclidean basis? 
	
	\item Under some change of basis, say
	\begin{align*}
	\xi \to t^E \eta,
	\end{align*}  
	does this measure change as
	\begin{align*}
	d\xi \to t^{\tr E}d\eta \sim t^{\tr E}\,dtd\Omega(\eta)
	\end{align*}
	as I would expect?
	
	\item I looked at your ``Some thoughts/questions in algebraic geometry'' and found an example of a nondegenerate homogeneous $P(\xi): \mathbb{R}^2 \to \mathbb{C}$ given by
	\begin{align*}
	P(\xi) = -i\xi_1 + \xi_2^2
	\end{align*}
	with respect to the euclidean basis with $\text{diag}(1,1/2)\in \text{Exp}(P)$. Assuming $d\xi = d\xi_1d\xi_2$ in the euclidean basis and letting $x = 0$, I attempted to evaluate
	\begin{align*}
	\int_{\mathbb{R}^d} e^{-iP(\xi)}\,d\xi = \int_{\mathbb{R}^d} e^{-i(i\xi_1 + \xi_2^2)}\,d\xi_1d\xi_2.
	\end{align*} 
	I recognized that we don't need to worry about ``integrating out the angular element'' since $P(\xi)$ doesn't have any cross terms. So I rewrote this as
	\begin{align*}
	\int_\mathbb{R}e^{\xi_1}\,d\xi_1\,\underbrace{\int_\mathbb{R}e^{-i\xi_2^2}\,d\xi_2}_{(1-i)\sqrt{\pi/2}}.
	\end{align*}
	The first integral doesn't converge, even as an improper Riemann integral. Does this mean there must be other restrictions other than $P(\xi)$ being nondegenerate homogeneous for $H_P^1(\xi_1,\xi_2)$ to exist?
	
	\item I guess this is not a question. I tried to evaluate the following integral after the change of variables from $\xi \to t^E\eta$, where $\eta$ is such that $P(\eta) = 1$:
	\begin{align*}
	\int_0^\infty e^{-iP(\eta)t}t^{\tr E}\,dt.
	\end{align*}
	where I'm leaving out the ``angular integration $\Omega$'' for now. It turned out that
	\begin{align*}
	\int_0^\infty e^{-it}t^{\tr E}\,dt =  -i e^{\lp-\frac{1}{2}i\pi\tr E\rp} \Gamma (\tr E+1)
	\end{align*}
	so long as $\tr E \neq 0$ and $-1 < \tr E$. This all assumes that $x=0$ in
	\begin{align*}
	\int_{\mathbb{R}^d}e^{-iP(\xi) - ix\cdot \xi}\,d\xi = \int_{\mathbb{R}^d}e^{-it - ix\cdot t^E \eta}t^{\tr E}\,dtd\Omega(\eta).
	\end{align*}
	I'm running some test cases with $x\neq 0$, assuming $x\cdot t^E \eta$ to be just some linear combination of the powers of $t$. Most of them seem to converge. But of course I'll have to worry about integrating over all $\eta$ as well.   
\end{enumerate}


	
	
	
\end{document}