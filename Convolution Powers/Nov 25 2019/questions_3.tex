\documentclass{article}
\usepackage[margin=1in]{geometry}
\usepackage{physics}
\usepackage{graphicx}
\usepackage{caption}
\usepackage{amsmath}
\usepackage{bm}
\usepackage{authblk}
\usepackage{empheq}
\usepackage{amsfonts}
\usepackage{esint}
\usepackage[makeroom]{cancel}
\usepackage{dsfont}
\usepackage{centernot}
\usepackage{mathtools}
\usepackage{bigints}
\usepackage{amsthm}
\theoremstyle{definition}
\newtheorem{defn}{Definition}[section]
\newtheorem{prop}{Proposition}[section]
\newtheorem{rmk}{Remark}[section]
\newtheorem{thm}{Theorem}[section]
\newtheorem{exmp}{Example}[section]
\newtheorem{prob}{Problem}[section]
\newtheorem{sln}{Solution}[section]
\newtheorem*{prob*}{Problem}
\newtheorem{exer}{Exercise}[section]
\newtheorem*{exer*}{Exercise}
\newtheorem*{sln*}{Solution}
\usepackage{empheq}
\usepackage{hyperref}
\usepackage{tensor}
\usepackage{xcolor}
\hypersetup{
	colorlinks,
	linkcolor={black!50!black},
	citecolor={blue!50!black},
	urlcolor={blue!80!black}
}


\newcommand*\widefbox[1]{\fbox{\hspace{2em}#1\hspace{2em}}}

\newcommand{\p}{\partial}
\newcommand{\R}{\mathbb{R}}
\newcommand{\C}{\mathbb{C}}
\newcommand{\lag}{\mathcal{L}}
\newcommand{\nn}{\nonumber}
\newcommand{\ham}{\mathcal{H}}
\newcommand{\M}{\mathcal{M}}
\newcommand{\I}{\mathcal{I}}
\newcommand{\K}{\mathcal{K}}
\newcommand{\F}{\mathcal{F}}
\newcommand{\w}{\omega}
\newcommand{\lam}{\lambda}
\newcommand{\al}{\alpha}
\newcommand{\be}{\beta}
\newcommand{\x}{\xi}

\newcommand{\G}{\mathcal{G}}

\newcommand{\f}[2]{\frac{#1}{#2}}

\newcommand{\ift}{\infty}

\newcommand{\lp}{\left(}
\newcommand{\rp}{\right)}

\newcommand{\lb}{\left[}
\newcommand{\rb}{\right]}

\newcommand{\lc}{\left\{}
\newcommand{\rc}{\right\}}


\newcommand{\V}{\mathbf{V}}
\newcommand{\U}{\mathcal{U}}
\newcommand{\Id}{\mathcal{I}}
\newcommand{\D}{\mathcal{D}}
\newcommand{\Z}{\mathcal{Z}}

%\setcounter{chapter}{-1}


%\makeatletter
%\renewcommand{\@chapapp}{Part}
%\renewcommand\thechapter{$\bf{\ket{\arabic{chapter}}}$}
%\renewcommand\thesection{$\bf{\ket{\arabic{section}}}$}
%\renewcommand\thesubsection{$\bf{\ket{\arabic{subsection}}}$}
%\renewcommand\thesubsubsection{$\bf{\ket{\arabic{subsubsection}}}$}
%\makeatother



\usepackage{subfig}
\usepackage{listings}
\captionsetup[lstlisting]{margin=0cm,format=hang,font=small,format=plain,labelfont={bf,up},textfont={it}}
\renewcommand*{\lstlistingname}{Code \textcolor{violet}{\textsl{Mathematica}}}
\definecolor{gris245}{RGB}{245,245,245}
\definecolor{olive}{RGB}{50,140,50}
\definecolor{brun}{RGB}{175,100,80}
\lstset{
	tabsize=4,
	frame=single,
	language=mathematica,
	basicstyle=\scriptsize\ttfamily,
	keywordstyle=\color{black},
	backgroundcolor=\color{gris245},
	commentstyle=\color{gray},
	showstringspaces=false,
	emph={
		r1,
		r2,
		epsilon,epsilon_,
		Newton,Newton_
	},emphstyle={\color{olive}},
	emph={[2]
		L,
		CouleurCourbe,
		PotentielEffectif,
		IdCourbe,
		Courbe
	},emphstyle={[2]\color{blue}},
	emph={[3]r,r_,n,n_},emphstyle={[3]\color{magenta}}
}


\begin{document}
\begin{center}
	\huge{Questions/Ideas \#3}\\
	$\,$\\
	\normalsize{\today}\\
	\normalsize{Huan Bui}
\end{center}


The goal here is see whether the ``volume'' of the bounded set $\mathcal{S} := \{ s \in \mathbb{R}^d \vert P(s) = 1 \}$ where
\begin{align}
P(s) = s_1^{k_1} + s_2^{k_2} + \dots + s_d^{k_d}.
\end{align}
The powers are even numbers, as usual. \\

I looked at the $\det{A^\top A}$ approach to finding volumes, but it turns out that
\begin{align}
\sqrt{\det(J^\top J)} = \abs{\det(J)}
\end{align}
where $J$ is the Jacobian matrix. So this doesn't help at all. \\

Okay, I considered the general $d=2$ case:
\begin{align}
P(s) = s_1^{k_2} + s_2^{k_2}
\end{align}
where $k_1,k_2$ are even numbers greater than 0. The parameterization (setting $t=1$) is 
\begin{align}
\vec{s} = \begin{bmatrix}
s_1 \\ \lp 1-s_1^{k_1} \rp^{1/k_2}
\end{bmatrix}
\end{align}
The Jacobian is then
\begin{align}
\sqrt{\frac{t^{2 \left(\frac{1}{m}+\frac{1}{n}-1\right)}
		\left(1-x^n\right)^{\frac{2}{m}-2}}{m^2}} = {\frac{
		\left(1-x^n\right)^{\frac{1}{m}-1}}{m}}.
\end{align}
Integrating over $[-1,1]$ gives the ``volume'' of the top half of the space:
\begin{align}
_2F_1\left(1-\frac{1}{ m},\frac{1}{ n};1+\frac{1}{
	n};(-1)^{n}\right)+\frac{\Gamma \left(\frac{1}{ m}\right) \Gamma \left(1+\frac{1}{
		n}\right)}{\Gamma \left(\frac{m+n}{ m n}\right)}
\end{align}
So this integral \textbf{converges} in general. \\

What about when $d=3$? How do we deal with the $d=3$ case? My strategy is to do the $ds_1ds_2$ integral on $1 - s_3^{k_3}$ first, obtain something in terms of $s_3$, then to the $ds_3$ integral on $[-1,1]$. \\

The parameterization for $s_1^{k_1} + s_2^{k_2} = 1- s_3^{k_3}$ after setting $t=1$ is 
\begin{align}
\vec{s} = \begin{bmatrix}
s_1 \\ (R_3 - s_1^{k_1})^{1/k_2}
\end{bmatrix}
\end{align}
where I have defined $R_3 = 1 - s_3^{k_3}$. The associated Jacobian with $(t=1)$ is
\begin{align}
{\frac{R_3 
		\left(-{s_1}^{{k_1}}-{s_3}^{{k_3}}+1\right)^{\frac{1}{{k_2}}-1}}{{k_2}}} = {\frac{R_3 
		\left(R_3-{s_1}^{{k_1}}\right)^{\frac{1}{{k_2}}-1}}{{k_2}}}
\end{align}
Now, we wish to evaluate the integral
\begin{align}
\iint \,ds_1ds_2ds_3 = \int_{-1}^1\,ds_3 \iint_{\p,s_3} ds_1ds_2.
\end{align}
The $ds_1ds_2$ integral can be handled using the same procedure as before, except we're no longer integrating from $-1$ to $1$, but rather from $-R_3^{1/k_1}$ to $R_3^{1/k_1}$. To this end,
\begin{align}
\iint ds_1ds_2 = \int_{-R_3^{1/k_1}}^{R_3^{1/k_1}} \lp R_3 - s_1^{k_1}\rp^{-1+1/k_2}\,ds_1
\end{align}
After a change of variables this integral is equal to
\begin{align}
R_3^{-1+1/k_1+1/k_2}\int^1_{-1}(1-x^{k_1})^{-1+1/k_2}\,dx.
\end{align}
This $x$ integral converges and gives something similar to what we have before. In the ends of the $ds_3$ integral, the $x$ integral will be a constant, so we don't worry about it. To find the total integral, we will multiply this $x$ integral with the $ds_3$ integral. The $ds_3$ integral will now be
\begin{align}
\int^1_{-1}R_3\cdot R_3^{-1+1/k_1+1/k_2}\,ds_3 &= \int^1_{-1} R_3^{1/k_1+1/k_2}\,ds_3 = \int^1_{-1} \lp 1- s_3^{k_3} \rp^{1/k_1+1/k_2}\,ds_3 \nn\\
&= _2F_1\left(-\frac{{k_1}+{k_2}}{{k_1}
	{k_2}},\frac{1}{{k_3}};1+\frac{1}{{k_3}};(-1)^{{k_3}}\right)+\frac{\Gamma
	\left(1+\frac{1}{{k_3}}\right) \Gamma
	\left(1+\frac{1}{{k_1}} + \f{1}{k_2}\right)}{\Gamma
	\left(\frac{1}{{k_2}}+\frac{1}{{k_3}}+1+\frac{1}{{k_1}}\right)}
\end{align}
where the extra factor of $R_3$ comes from the Jacobian. So we see that this 3-d integral also \textbf{converges} in general. \\

\textbf{Example:} We shall verify this with the 3-sphere where $x^2 + y^2 + z^2 = 1$. The $ds_1ds_2$ integral is 
\begin{align}
\int^1_{-1} (1-s_1^2)^{-1+1/2}\,ds_1 = \pi
\end{align}
The $ds_3$ integral is 
\begin{align}
\int_{-1}^{1} (1-s_3^2)^{(1/2+1/2)}\,ds_3 = \f{4}{3}.
\end{align}	
Multiplying everything together we get $4\pi / 3$, which is the volume is a 3-sphere of radius 1. \qedhere \\


Next, can we use induction to get convergence at higher orders? Let's see if we can figure out any pattern. Suppose $R_3 = (1-s_3^{k_3})$ now becomes a bit more complicated: involves another $s_4^{k_4}$: $R_{3} \to R_{34} = 1 - s_3^{k_3} - s_4^{k_4}$. With this, we have a change in integrals:
\begin{align}
\int^1_{-1}R_{34}\cdot R_{34}^{-1+1/k_1+1/k_2}\,ds_3ds_4 \to \int^1_{-1}\,ds_4 \int^{R_4^{1/k_4}}_{-R_4^{1/k_4}}(R_4 - s_3^{k_3})^{1/k_1+1/k_2}\,ds_3
\end{align}
where $R_4 = 1 - s_4^{1/k_4}$. This integral looks familiar! We can repeat what we've done before to get
\begin{align}
\int^1_{-1}\,ds_4 R_4^{-1+1/k_1+1/k_2+1/k_3}\int^{1}_{-1}(1 - x^{k_3})^{1/k_1+1/k_2}\,ds_3.
\end{align}
I tried to evaluated this integral in Mathematica and it converges. I also tested with $k_1 = k_2 = k_3 = k_4 = 2$ and found the volume of the 4-sphere, off by a factor of 2, which is good enough for me... \\

I think we can carry on with this, to show the ``volume'' of $\mathcal{S}$ is bounded, i.e., the angular integrals converge.  

	
	
	
\end{document}