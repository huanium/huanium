\documentclass{article}
\usepackage{physics}
\usepackage{graphicx}
\usepackage{caption}
\usepackage{amsmath}
\usepackage{bm}
\usepackage{framed}
\usepackage{authblk}
\usepackage{empheq}
\usepackage{amsfonts}
\usepackage{esint}
\usepackage[makeroom]{cancel}
\usepackage{dsfont}
\usepackage{centernot}
\usepackage{mathtools}
\usepackage{subcaption}
\usepackage{bigints}
\usepackage{amsthm}
\theoremstyle{definition}
\newtheorem{lemma}{Lemma}
\newtheorem{defn}{Definition}[section]
\newtheorem{prop}{Proposition}[section]
\newtheorem{rmk}{Remark}[section]
\newtheorem{thm}{Theorem}[section]
\newtheorem{exmp}{Example}[section]
\newtheorem{prob}{Problem}[section]
\newtheorem{sln}{Solution}[section]
\newtheorem*{prob*}{Problem}
\newtheorem{exer}{Exercise}[section]
\newtheorem*{exer*}{Exercise}
\newtheorem*{sln*}{Solution}
\usepackage{empheq}
\usepackage{tensor}
\usepackage{xcolor}
%\definecolor{colby}{rgb}{0.0, 0.0, 0.5}
\definecolor{MIT}{RGB}{163, 31, 52}
\usepackage[pdftex]{hyperref}
%\hypersetup{colorlinks,urlcolor=colby}
\hypersetup{colorlinks,linkcolor={MIT},citecolor={MIT},urlcolor={MIT}}  
\usepackage[left=1in,right=1in,top=1in,bottom=1in]{geometry}

\usepackage{newpxtext,newpxmath}
\newcommand*\widefbox[1]{\fbox{\hspace{2em}#1\hspace{2em}}}

\newcommand{\p}{\partial}
\newcommand{\R}{\mathbb{R}}
\newcommand{\C}{\mathbb{C}}
\newcommand{\lag}{\mathcal{L}}
\newcommand{\nn}{\nonumber}
\newcommand{\ham}{\mathcal{H}}
\newcommand{\M}{\mathcal{M}}
\newcommand{\I}{\mathcal{I}}
\newcommand{\K}{\mathcal{K}}
\newcommand{\F}{\mathcal{F}}
\newcommand{\w}{\omega}
\newcommand{\lam}{\lambda}
\newcommand{\al}{\alpha}
\newcommand{\be}{\beta}
\newcommand{\x}{\xi}

\newcommand{\G}{\mathcal{G}}

\newcommand{\f}[2]{\frac{#1}{#2}}

\newcommand{\ift}{\infty}

\newcommand{\lp}{\left(}
\newcommand{\rp}{\right)}

\newcommand{\lb}{\left[}
\newcommand{\rb}{\right]}

\newcommand{\lc}{\left\{}
\newcommand{\rc}{\right\}}


\newcommand{\V}{\mathbf{V}}
\newcommand{\U}{\mathcal{U}}
\newcommand{\Id}{\mathcal{I}}
\newcommand{\D}{\mathcal{D}}
\newcommand{\Z}{\mathcal{Z}}

%\setcounter{chapter}{-1}


\usepackage{enumitem}



\usepackage{listings}
\captionsetup[lstlisting]{margin=0cm,format=hang,font=small,format=plain,labelfont={bf,up},textfont={it}}
\renewcommand*{\lstlistingname}{Code \textcolor{violet}{\textsl{Mathematica}}}
\definecolor{gris245}{RGB}{245,245,245}
\definecolor{olive}{RGB}{50,140,50}
\definecolor{brun}{RGB}{175,100,80}

%\hypersetup{colorlinks,urlcolor=colby}
\lstset{
	tabsize=4,
	frame=single,
	language=mathematica,
	basicstyle=\scriptsize\ttfamily,
	keywordstyle=\color{black},
	backgroundcolor=\color{gris245},
	commentstyle=\color{gray},
	showstringspaces=false,
	emph={
		r1,
		r2,
		epsilon,epsilon_,
		Newton,Newton_
	},emphstyle={\color{olive}},
	emph={[2]
		L,
		CouleurCourbe,
		PotentielEffectif,
		IdCourbe,
		Courbe
	},emphstyle={[2]\color{blue}},
	emph={[3]r,r_,n,n_},emphstyle={[3]\color{magenta}}
}




\begin{document}
\begin{framed}
\noindent Name: \textbf{Huan Q. Bui}\\
\today \\
BEC1 mask fabrication proposal \\
PI: Prof. Martin Zwierlein\\
Version: 2
\end{framed}


\noindent \textbf{1. OBJECTIVE}\\

\noindent The goal of this fabrication is to make an array of beam blockers for CW-laser application. The beam blockers are silver (Ag) circular masks of radii $\sim$ 100 $\mu$m that are deposited on a Thorlabs AR-coated glass window (N-BK7). The masks are distributed on the glass window according to the following pattern:

\begin{figure}[!htb]
\centering
\includegraphics[scale=0.35]{masks.png}
\end{figure}

\noindent \textbf{Note:} the pattern sent to the MLA machine will be inverted due to the use of negative photoresist.\\

\noindent The BEC1 experiment creates an optical trap for ultracold atoms using 532 nm light in the form of a "coke can." The trap consists of one hollow beam and two end cap beams. The hollow beam is generated by using one of the masks in the pattern to block the central region of a Bessel beam. The diameter of the coke can be controlled by selecting a circular mask of a particular size from the pattern. The setup in BEC1 is similar, but not identical, to the following schematic:

\begin{figure}[!htb]
\centering
\includegraphics[scale=0.45]{hollow_beam.png}
\end{figure}






\newpage



\noindent \textbf{2. TENTATIVE PROCEDURE}

\begin{enumerate}[label=(\alph*)]

\item Acquire Thorlabs' \href{https://www.thorlabs.com/thorproduct.cfm?partnumber=WL11050-C7}{WL11050-C7} glass window. Material N-BK7. Diameter: 1 inch. The glass window is AR-coated, suitable for 400-700 nm. The mask will be created on this glass window blank. 



\begin{figure}[!htb]
\centering
\includegraphics[scale=0.6]{thorlabs_window.png}
\end{figure}

Other specifications:
\begin{itemize}
\item Surface flatness: $\leq \lambda/10$ at 633 nm

\item Surface quality: 10-5 scratch-dig

\item Thickness tolerance: $\pm 0.3$ mm 

\item Parallelism: $\leq 5$ arcsec
\end{itemize}


\item Clean the N-BK7 window with acetone, followed by methanol, followed by IPA. Blow dry with N$_2$.

\item Spin AZ nLOF 2020 (negative) photoresist to approximately 2 \textmu m at approximately 3000 rpm for 30-60 seconds. The optimal spin parameters will be determined during machine training/once the procedure starts.

\item Bake for 110$^\circ$C for 90 seconds. Exact temperature and time to be determined. 

\item Expose pattern with the MLA150 machine. Dosage: approximately 200 mJ/cm$^2$. The optimal dosage is determined by the age of the nLOF photoresist and is calibrated by MIT.nano's Dr. Hanqing Li. 

\item Post-exposure baking (PEB). This step is required due to the use of negative photoresist. Bake at 110$^\circ$C for 90 seconds. 

\item Develop substrate. Time and temperature: to be determined.

\item Deposit a layer of silver (Ag) with thickness of 150 nm by electron-beam evaporation using the E-beam-Aja machine.

\item Silver liftoff: remove the photoresist and the silver that was deposited on the resist in the Microposit Remover 1165 overnight. 

\item Solvent clean using acetone, followed by methanol, followed by IPA. Blow dry with N$_2$.

\item Deposit a 20 nm-thick layer of SiO$_2$ on the entire substrate using CVD.

\end{enumerate}


\end{document}











