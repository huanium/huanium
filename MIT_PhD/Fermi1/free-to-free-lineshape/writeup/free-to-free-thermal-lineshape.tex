\documentclass{article}
\usepackage{physics}
\usepackage{graphicx}
\usepackage{caption}
\usepackage{amsmath}
\usepackage{bm}
\usepackage{framed}
\usepackage{authblk}
\usepackage{empheq}
\usepackage{amsfonts}
\usepackage{esint}
\usepackage[makeroom]{cancel}
\usepackage{dsfont}
\usepackage{centernot}
\usepackage{mathtools}
\usepackage{subcaption}
\usepackage{bigints}
\usepackage{amsthm}
\theoremstyle{definition}
\newtheorem{lemma}{Lemma}
\newtheorem{defn}{Definition}[section]
\newtheorem{prop}{Proposition}[section]
\newtheorem{rmk}{Remark}[section]
\newtheorem{thm}{Theorem}[section]
\newtheorem{exmp}{Example}[section]
\newtheorem{prob}{Problem}[section]
\newtheorem{sln}{Solution}[section]
\newtheorem*{prob*}{Problem}
\newtheorem{exer}{Exercise}[section]
\newtheorem*{exer*}{Exercise}
\newtheorem*{sln*}{Solution}
\usepackage{empheq}
\usepackage{tensor}
\usepackage{xcolor}
%\definecolor{colby}{rgb}{0.0, 0.0, 0.5}
\definecolor{MIT}{RGB}{163, 31, 52}
\usepackage[pdftex]{hyperref}
%\hypersetup{colorlinks,urlcolor=colby}
\hypersetup{colorlinks,linkcolor={MIT},citecolor={MIT},urlcolor={MIT}}  
\usepackage[left=1in,right=1in,top=1in,bottom=1in]{geometry}
\usepackage{newpxtext,newpxmath}
\newcommand*\widefbox[1]{\fbox{\hspace{2em}#1\hspace{2em}}}
\newcommand{\p}{\partial}
\newcommand{\R}{\mathbb{R}}
\newcommand{\C}{\mathbb{C}}
\newcommand{\lag}{\mathcal{L}}
\newcommand{\nn}{\nonumber}
\newcommand{\ham}{\mathcal{H}}
\newcommand{\M}{\mathcal{M}}
\newcommand{\I}{\mathcal{I}}
\newcommand{\K}{\mathcal{K}}
\newcommand{\F}{\mathcal{F}}
\newcommand{\w}{\omega}
\newcommand{\lam}{\lambda}
\newcommand{\al}{\alpha}
\newcommand{\be}{\beta}
\newcommand{\x}{\xi}
\newcommand{\G}{\mathcal{G}}
\newcommand{\f}[2]{\frac{#1}{#2}}
\newcommand{\ift}{\infty}
\newcommand{\lp}{\left(}
\newcommand{\rp}{\right)}
\newcommand{\lb}{\left[}
\newcommand{\rb}{\right]}
\newcommand{\lc}{\left\{}
\newcommand{\rc}{\right\}}
\newcommand{\V}{\mathbf{V}}
\newcommand{\U}{\mathcal{U}}
\newcommand{\Id}{\mathcal{I}}
\newcommand{\D}{\mathcal{D}}
\newcommand{\Z}{\mathcal{Z}}

%\setcounter{chapter}{-1}

\usepackage{enumitem}
\usepackage{listings}
\captionsetup[lstlisting]{margin=0cm,format=hang,font=small,format=plain,labelfont={bf,up},textfont={it}}
\renewcommand*{\lstlistingname}{Code \textcolor{violet}{\textsl{Mathematica}}}
\definecolor{gris245}{RGB}{245,245,245}
\definecolor{olive}{RGB}{50,140,50}
\definecolor{brun}{RGB}{175,100,80}

%\hypersetup{colorlinks,urlcolor=colby}
\lstset{
	tabsize=4,
	frame=single,
	language=mathematica,
	basicstyle=\scriptsize\ttfamily,
	keywordstyle=\color{black},
	backgroundcolor=\color{gris245},
	commentstyle=\color{gray},
	showstringspaces=false,
	emph={
		r1,
		r2,
		epsilon,epsilon_,
		Newton,Newton_
	},emphstyle={\color{olive}},
	emph={[2]
		L,
		CouleurCourbe,
		PotentielEffectif,
		IdCourbe,
		Courbe
	},emphstyle={[2]\color{blue}},
	emph={[3]r,r_,n,n_},emphstyle={[3]\color{magenta}}
}

\begin{document}

\begin{center}
	\Large{Free-to-free thermal RF lineshape}
\end{center}	

\begin{center}
	\today
\end{center}

\section{Without interactions}

In the absence of interactions, the ideal lineshape is due to Rabi:
\begin{equation}\label{eq:Rabi}
I(\delta) = \f{\omega_R^2}{\omega_R^2 + \delta^2} \sin^2 \lp  \f{\sqrt{\omega_R^2 + \delta^2}}{2} T \rp
\end{equation}
For weak and long drive with dephasing and broadening (due to magnetic field inhomogeneity, say), the lineshape is better-described by a Lorentzian:
\begin{equation}\label{eq:no-interaction-lineshape}
I(\delta) = \f{1}{2}\f{\Gamma^2/4}{\delta^2 + \Gamma^2/4}
\end{equation}
which peaks at $1/2$. 

\subsection{RF spectrum from Fermi's Golden Rule}


\subsection{RF spectrum from Green's function method}



\section{With interactions}

The asymptotic form of the reduced radial wavefunction for the relative motion between an Na atom and a K atom is 
\begin{equation}\label{eq:wfn}
\psi(r) = \sqrt{\frac{2\mu}{\pi \hbar^2 k}} \sin(kr + \delta),
\end{equation}
where $\delta$ is the phase shift due to the interaction between Na and K. This phase shift, for low-energy $s$-wave interaction, is completely determined by the scattering length $a$ and the relative momentum $k$:
\begin{equation} \label{eq:phase_shift}
\cos \delta = \sqrt{\f{1}{1 + a^2 k^2}}.
\end{equation}
This also means $\psi(r)$ is determined by $k$ and $a$. From now on, all $\psi$'s have the same $a$ unless stated otherwise. In \eqref{eq:wfn}, the $1/\sqrt{k}$ prefactor is due to the energy normalization
\begin{equation}\label{eq:norm}
\int_0^\infty \psi_k^*(r) \psi_{k'}(r) \,dr = \delta(E_k - E_{k'}),
\end{equation}
where $E_k = \hbar^2 k^2 / 2\mu$.\\

Now consider doing RF spectroscopy on a K atom in a thermal bath of Na atoms. The RF photon is absorbed and can flip the spin of the $K$ atom. In this process, the wavefunction of the relative motion can transition to one with different relative momentum $k_i \to k_f$ and phase shift $\delta_i \to \delta_f$. The RF photon itself does not transfer momentum, but it cannot states with different relative momenta. The change in the phase shift is due to the fact that the Na-K scattering length is spin-dependent. 
\begin{eqnarray}\label{eq:wfns}
\psi_{i,k_i}(r) &=&  \sqrt{\frac{2\mu}{\pi \hbar^2 k}} \sin(k_ir + \delta_i),\\
\psi_{f,k_f}(r) &=&  \sqrt{\frac{2\mu}{\pi \hbar^2 k}} \sin(k_fr + \delta_f),
\end{eqnarray}

\subsection{RF spectrum from Fermi's Golden Rule}

In the limit of weak and long drive, the RF spectrum is given by Fermi's Golden Rule:
\begin{equation}\label{eq:FGR}
I(\omega_{RF}) = \f{2\pi}{\hbar} \sum_{k_f} \abs{\bra{\psi_{i,k_i}}  V_{RF}  \ket{\psi_{f,k_f}}}^2  \delta(\mathcal{E}_f - \mathcal{E}_i) = \f{2\pi}{\hbar} \sum_{k_f} \abs{\bra{\psi_{i,k_i}}   \ket{\psi_{f,k_f}}}^2 V_{RF}^2  \delta(\mathcal{E}_f - \mathcal{E}_i) 
\end{equation}
where 
\begin{eqnarray}\label{eq:energies}
\mathcal{E}_{i} &=& E_{k_i} + \Delta_i + \hbar \omega_{RF}\\
\mathcal{E}_{f} &=& E_{k_f} + \Delta_f.
\end{eqnarray}
As state before, the RF photon is absorbed. Some of energy carried by the RF photon is used to flip the spin. The rest is to ensure energy conservation when transitioning to a final state with different interaction than the initial state. The quantities $\Delta_{i,f}$ denote the mean-field shifts of the K atoms propagating in the Na medium:
\begin{equation}
\Delta_{i,f} = \f{4\pi \hbar^2}{m_{Na}} n_{Na} a_{Na-K_{i,f}}.
\end{equation}
\textcolor{red}{Find out what this mass is. Probably mass of Na?}\\

We now evaluate the overlap $\bra{\psi_{i,k_i}} \ket{\psi_{f,k_f}}$:
\begin{eqnarray}
\bra{\psi_{i,k_i}} \ket{\psi_{f,k_f}} 
&=& \f{2\mu}{\pi \hbar^2} \f{1}{\sqrt{k_ik_f}} \int_0^\infty  \sin(k_ir + \delta_i) \sin(k_f r + \delta_f) \,dr \\ 
&\stackrel{\mathclap{\normalfont{R\to \infty}}}{=}& 
\f{2\mu}{\pi \hbar^2} \f{1}{\sqrt{k_ik_f}} \int_0^R  \sin(k_ir + \delta_i) \sin(k_f r + \delta_f) \,dr \\ 
&\stackrel{\mathclap{\normalfont{R\to \infty}}}{=}&  
\f{\mu}{\pi \hbar^2 \sqrt{k_ik_f}} 
\left[ \f{-\sin(\delta_i - \delta_f) + \sin(\delta_i - \delta_f + (k_i - k_f)R)}{k_i - k_f} \right. \nonumber \\
&&\qquad \qquad \left.+ \f{\sin(\delta_i + \delta_f) - \sin(\delta_i + \delta_f + (k_i + k_f)R)}{k_i + k_f} \right].
\end{eqnarray}

How to deal with the part that depends on $R$? One way is to say that there is a box boundary condition for which $k_{i,f} R = n \pi$, so that all terms with $R$ vanish. \textcolor{red}{Is the boundary condition forced?} Another way is to see what happens for $k_i \neq k_f$: as $R$ goes to $\infty$, both terms that depend on $R$ vanish. \textcolor{red}{What about when $k_i = k_f$?}\\

Assuming that $k_i \neq k_f$, we find 
\begin{eqnarray}
\abs{\bra{\psi_{i,k_i}} \ket{\psi_{f,k_f}} }^2
&=& \abs{ \f{\mu}{\pi \hbar^2 \sqrt{k_ik_f}} \lb \f{-\sin(\delta_i - \delta_f)}{k_i - k_f} + \f{\sin(\delta_i + \delta_f)}{k_i + k_f}\rb }^2 \\
&=& \hdots \\
&=& \f{\mu^2 (a_i - a_f)^2}{\pi^2 \hbar^4}  \f{k_i k_f}{(1 - a_i^2k_i^2) (1 - a_f^2 k_f^2)} \f{1}{(k_i^2 - k_f^2)^2},
\end{eqnarray}
where we have used \eqref{eq:phase_shift} to simplify to the last line. \\

For a given $\omega_{RF}$, a nontrivial contribution to $I(\omega_{RF})$ comes from a $k_f$ for which $\mathcal{E}_i = \mathcal{E}_f$, or 
\begin{equation}
\f{\hbar^2 k_i^2}{2\mu} + \Delta_i + \hbar \omega_{RF} = \f{\hbar^2 k_f^2}{2\mu} + \Delta_f \iff \hbar \omega_{RF} = \f{\hbar^2 (k_f^2 - k_i^2)}{2\mu} + (\Delta_f - \Delta_i).
\end{equation}


\subsection{RF spectrum from Green's function method}




\bibliography{free-to-free-lineshape} 
\bibliographystyle{unsrt}	

\end{document}