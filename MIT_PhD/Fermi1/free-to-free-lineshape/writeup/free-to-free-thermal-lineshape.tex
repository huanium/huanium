\documentclass{article}
\usepackage{physics}
\usepackage{graphicx}
\usepackage{caption}
\usepackage{amsmath}
\usepackage{bm}
\usepackage{framed}
\usepackage{authblk}
\usepackage{empheq}
\usepackage{amsfonts}
\usepackage{esint}
\usepackage[makeroom]{cancel}
\usepackage{dsfont}
\usepackage{centernot}
\usepackage{mathtools}
\usepackage{subcaption}
\usepackage{bigints}
\usepackage{amsthm}
\theoremstyle{definition}
\newtheorem{lemma}{Lemma}
\newtheorem{defn}{Definition}[section]
\newtheorem{prop}{Proposition}[section]
\newtheorem{rmk}{Remark}[section]
\newtheorem{thm}{Theorem}[section]
\newtheorem{exmp}{Example}[section]
\newtheorem{prob}{Problem}[section]
\newtheorem{sln}{Solution}[section]
\newtheorem*{prob*}{Problem}
\newtheorem{exer}{Exercise}[section]
\newtheorem*{exer*}{Exercise}
\newtheorem*{sln*}{Solution}
\usepackage{empheq}
\usepackage{tensor}
\usepackage{xcolor}
%\definecolor{colby}{rgb}{0.0, 0.0, 0.5}
\definecolor{MIT}{RGB}{163, 31, 52}
\usepackage[pdftex]{hyperref}
%\hypersetup{colorlinks,urlcolor=colby}
\hypersetup{colorlinks,linkcolor={MIT},citecolor={MIT},urlcolor={MIT}}  
\usepackage[left=1in,right=1in,top=1in,bottom=1in]{geometry}
\usepackage{newpxtext,newpxmath}
\newcommand*\widefbox[1]{\fbox{\hspace{2em}#1\hspace{2em}}}
\newcommand{\p}{\partial}
\newcommand{\R}{\mathbb{R}}
\newcommand{\C}{\mathbb{C}}
\newcommand{\lag}{\mathcal{L}}
\newcommand{\nn}{\nonumber}
\newcommand{\ham}{\mathcal{H}}
\newcommand{\M}{\mathcal{M}}
\newcommand{\I}{\mathcal{I}}
\newcommand{\K}{\mathcal{K}}
\newcommand{\F}{\mathcal{F}}
\newcommand{\w}{\omega}
\newcommand{\lam}{\lambda}
\newcommand{\al}{\alpha}
\newcommand{\be}{\beta}
\newcommand{\x}{\xi}
\newcommand{\G}{\mathcal{G}}
\newcommand{\f}[2]{\frac{#1}{#2}}
\newcommand{\ift}{\infty}
\newcommand{\lp}{\left(}
\newcommand{\rp}{\right)}
\newcommand{\lb}{\left[}
\newcommand{\rb}{\right]}
\newcommand{\lc}{\left\{}
\newcommand{\rc}{\right\}}
\newcommand{\V}{\mathbf{V}}
\newcommand{\U}{\mathcal{U}}
\newcommand{\Id}{\mathcal{I}}
\newcommand{\D}{\mathcal{D}}
\newcommand{\Z}{\mathcal{Z}}

%\setcounter{chapter}{-1}

\usepackage{enumitem}
\usepackage{listings}
\captionsetup[lstlisting]{margin=0cm,format=hang,font=small,format=plain,labelfont={bf,up},textfont={it}}
\renewcommand*{\lstlistingname}{Code \textcolor{violet}{\textsl{Mathematica}}}
\definecolor{gris245}{RGB}{245,245,245}
\definecolor{olive}{RGB}{50,140,50}
\definecolor{brun}{RGB}{175,100,80}

%\hypersetup{colorlinks,urlcolor=colby}
\lstset{
	tabsize=4,
	frame=single,
	language=mathematica,
	basicstyle=\scriptsize\ttfamily,
	keywordstyle=\color{black},
	backgroundcolor=\color{gris245},
	commentstyle=\color{gray},
	showstringspaces=false,
	emph={
		r1,
		r2,
		epsilon,epsilon_,
		Newton,Newton_
	},emphstyle={\color{olive}},
	emph={[2]
		L,
		CouleurCourbe,
		PotentielEffectif,
		IdCourbe,
		Courbe
	},emphstyle={[2]\color{blue}},
	emph={[3]r,r_,n,n_},emphstyle={[3]\color{magenta}}
}

\begin{document}

\begin{center}
	\Large{Free-to-free thermal RF lineshape}
\end{center}	

\begin{center}
	\today
\end{center}

\section{Context}

Consider an impurity particle $F$ with spin-$1/2$ in a bath of particles $B$ with density $n_B$. In our case, the impurity particle is a fermion and the bath is bosons, but for now we focus on the case of a thermal gas, where quantum statistics do not play a role. The $F_{\uparrow/\downarrow}-B$ interaction is parameterized by the scattering length $a_{\uparrow/\downarrow}$. The state $\ket{F_\uparrow}$ is Rabi-coupled to the state $\ket{F_\downarrow}$ with strength $\Omega_0$ and detuning $\Delta_0$. Here, we are interested in the spectral function for the particle $F$ in both spin states $\uparrow$ and $\downarrow$. 


\section{Lineshape from Fermi's Golden Rule}

\subsection{Non-interacting case}

In the absence of interactions, the bath can be ignored, and the problem is the well-known Rabi problem for a two-level system plus the kinetic degree of freedom for the (free) impurity. The eigenstates in the presence of the Rabi coupling are dressed states, with energies
\begin{align}
E^{\pm}_\mathbf{k} = \epsilon_\mathbf{k} + \f{1}{2} \lp \Delta_0 \pm \sqrt{\Omega_0^2 + \Delta_0^2} \rp
\end{align}
where $\epsilon_\mathbf{k} = \hbar^2 k^2 / 2m_F$ is the kinetic energy of the impurity. The spectral function in this case has two delta-function peaks, one at $E^-$ and one at $E^+$. The lineshape going from one spin state to another thus has a single delta-peak.

\subsection{Interacting case}

When interaction is present, it is unclear how to treat this problem. We can try to obtain the spectral function from Fermi's Golden Rule. For this, we must be able to evaluate the overlap between the initial and final states. In our case, both the initial and final states are continuum states, whose radial wavefunctions are wavefunctions for the relative motion between a $B$ particle and an $F$ particle, and will be taken to be energy-normalized wavefunctions. Computing the overlap boils down to finding the overlap between these continuum wavefunctions. We start by considering their asymptotic form:
\begin{equation}\label{eq:wfn}
\psi(r) = \sqrt{\frac{2\mu}{\pi \hbar^2 k}} \sin(kr + \delta),
\end{equation}
where $\delta$ is the phase shift due to the interaction between $B$ and $F$. This phase shift, for low-energy $s$-wave interaction, is completely determined by the scattering length $a_{\uparrow/\downarrow}$ and the relative momentum $k$:
\begin{equation} \label{eq:phase_shift}
\cos \delta = \sqrt{\f{1}{1 + a^2 k^2}}.
\end{equation}
This also means $\psi(r)$ is determined by $k$ and $a$. From now on, all $\psi$'s have the same $a$ unless stated otherwise. In \eqref{eq:wfn}, the $1/\sqrt{k}$ prefactor is due to the energy normalization
\begin{equation}\label{eq:norm}
\int_0^\infty \psi_k^*(r) \psi_{k'}(r) \,dr = \delta(E_k - E_{k'}),
\end{equation}
where $E_k = \hbar^2 k^2 / 2\mu$, where $\mu$ is the reduced mass of the $B$-$F$ two-body system. \\

\noindent  Now consider doing RF spectroscopy on an $F$ atom in a thermal bath of $B$ atoms. The RF photon is absorbed and can flip the spin of the $F$ atom. In this process, the wavefunction of the relative motion can transition to one with different relative momentum $k_i \to k_f$ and phase shift $\delta_i \to \delta_f$. The RF photon itself does not transfer momentum, but it connect states with different relative momenta. The change in the phase shift is due to the fact that the $B$-$F$ scattering length is spin-dependent. 
\begin{eqnarray}\label{eq:wfns}
\psi_{i,k_i}(r) &=&  \sqrt{\frac{2\mu}{\pi \hbar^2 k}} \sin(k_ir + \delta_i),\\
\psi_{f,k_f}(r) &=&  \sqrt{\frac{2\mu}{\pi \hbar^2 k}} \sin(k_fr + \delta_f),
\end{eqnarray}
In the limit of weak and long drive, the RF spectrum is given by Fermi's Golden Rule:
\begin{equation}\label{eq:FGR}
I(\omega_{RF}) = \f{2\pi}{\hbar} \sum_{k_f} \abs{\bra{\psi_{i,k_i}}  V_{RF}  \ket{\psi_{f,k_f}}}^2  \delta(\mathcal{E}_f - \mathcal{E}_i) = \f{2\pi}{\hbar} \sum_{k_f} \abs{\bra{\psi_{i,k_i}}   \ket{\psi_{f,k_f}}}^2 V_{RF}^2  \delta(\mathcal{E}_f - \mathcal{E}_i) 
\end{equation}
where 
\begin{eqnarray}\label{eq:energies}
\mathcal{E}_{i} &=& E_{k_i} + \Delta_i + \hbar \omega_{RF}\\
\mathcal{E}_{f} &=& E_{k_f} + \Delta_f.
\end{eqnarray}
As state before, the RF photon is absorbed. Some of energy carried by the RF photon is used to flip the spin. The rest is to ensure energy conservation when transitioning to a final state with different interaction than the initial state. The quantities $\Delta_{i,f}$ denote the mean-field shifts, which we can take to be zero for simplicity. We now evaluate the overlap $\bra{\psi_{i,k_i}} \ket{\psi_{f,k_f}}$:
\begin{eqnarray}
\bra{\psi_{i,k_i}} \ket{\psi_{f,k_f}} 
&=& \f{2\mu}{\pi \hbar^2} \f{1}{\sqrt{k_ik_f}} \int_0^\infty  \sin(k_ir + \delta_i) \sin(k_f r + \delta_f) \,dr \\ 
&\stackrel{\mathclap{\normalfont{R\to \infty}}}{=}& 
\f{2\mu}{\pi \hbar^2} \f{1}{\sqrt{k_ik_f}} \int_0^R  \sin(k_ir + \delta_i) \sin(k_f r + \delta_f) \,dr \\ 
&\stackrel{\mathclap{\normalfont{R\to \infty}}}{=}&  
\f{\mu}{\pi \hbar^2 \sqrt{k_ik_f}} 
\left[ \f{-\sin(\delta_i - \delta_f) + \sin(\delta_i - \delta_f + (k_i - k_f)R)}{k_i - k_f} \right. \nonumber \\
&&\qquad \qquad \left.+ \f{\sin(\delta_i + \delta_f) - \sin(\delta_i + \delta_f + (k_i + k_f)R)}{k_i + k_f} \right].
\end{eqnarray}

How to deal with the part that depends on $R$? One way is to say that there is a box boundary condition for which $k_{i,f} R = n \pi$, so that all terms with $R$ vanish. \textcolor{red}{Is the boundary condition forced?} Another way is to see what happens for $k_i \neq k_f$: as $R$ goes to $\infty$, both terms that depend on $R$ vanish. \textcolor{red}{What about when $k_i = k_f$?} Assuming that $k_i \neq k_f$, we find 
\begin{eqnarray}
\abs{\bra{\psi_{i,k_i}} \ket{\psi_{f,k_f}} }^2
&=& \abs{ \f{\mu}{\pi \hbar^2 \sqrt{k_ik_f}} \lb \f{-\sin(\delta_i - \delta_f)}{k_i - k_f} + \f{\sin(\delta_i + \delta_f)}{k_i + k_f}\rb }^2 \\
&=& \hdots \\
&=& \f{\mu^2 (a_i - a_f)^2}{\pi^2 \hbar^4}  \f{k_i k_f}{(1 - a_i^2k_i^2) (1 - a_f^2 k_f^2)} \f{1}{(k_i^2 - k_f^2)^2},
\end{eqnarray}
where we have used \eqref{eq:phase_shift} to simplify to the last line. \\

\noindent While this overlap produces what we might have expected in terms of the scattering amplitudes, there is a problem. When $k_f \to k_i$, the overlap diverges. One might be tempted to fix it by thinking about the overlap as a delta-function, \textcolor{red}{but I don't this would be correct.}


\section{Lineshape from Green's function method}


\subsection{Non-interacting case}

In the absence of interactions and Rabi coupling, the impurity Green's function is just 
\begin{align}
G^{(0)}_{\uparrow/\downarrow} (\mathbf{k}, \epsilon) = \f{1}{ \epsilon - \epsilon_\mathbf{k}},
\end{align}
where $\epsilon_\mathbf{k} = \hbar^2 k^2 / 2m_F$, as before. Here, we do not yet make the distinction between advanced and retarded Green's function for brevity. In any case, turning on the Rabi coupling mixes bare states $\ket{F_\uparrow}$ and $\ket{\F_downarrow}$, and the resulting Green's function is a matrix. In the $\ket{F_\uparrow}, \ket{F_\downarrow}$ basis, this matrix is 
\begin{align}
\mathbf{G}_\Omega (\mathbf{k}, \epsilon) 
=& 
\begin{pmatrix}
G^{(\Omega)}_\uparrow & G^{(\Omega)}_{\uparrow\downarrow}  \\
G^{(\Omega)}_{\downarrow\uparrow}  & G^{(\Omega)}_\downarrow 
\end{pmatrix} \\
=&
\begin{pmatrix}
[G^{(0)}_{\uparrow} (\mathbf{k}, \epsilon)]^{-1} & -\Omega_0/2 \\ 
-\Omega_0/2  & [G^{(0)}_{\downarrow} (\mathbf{k}, \epsilon)]^{-1}
\end{pmatrix}^{-1}.
\end{align}
Inverting this matrix, we find
\begin{align}
G^{(\Omega)}_\uparrow (\mathbf{k}, \epsilon) 
&= \f{v^2}{\epsilon - E_\mathbf{k}^+}  + \f{u^2}{\epsilon - E_\mathbf{k}^-},\\
G^{(\Omega)}_{\uparrow\downarrow} (\mathbf{k}, \epsilon) 
&= \f{uv}{\epsilon - E_\mathbf{k}^+}  - \f{uv}{\epsilon - E_\mathbf{k}^-},\\
G^{(\Omega)}_{\downarrow\uparrow} (\mathbf{k}, \epsilon) 
&= \f{uv}{\epsilon - E_\mathbf{k}^+}  - \f{uv}{\epsilon - E_\mathbf{k}^-},\\
G^{(\Omega)}_\downarrow (\mathbf{k}, \epsilon) &= \f{u^2}{\epsilon - E_\mathbf{k}^+}  + \f{v^2}{\epsilon - E_\mathbf{k}^-},
\end{align}
where 
\begin{align}
u^2 = \f{1}{2}\lp 1 + \f{\Delta_0}{\sqrt{\Omega_0^2 + \Delta_0^2}} \rp,
\quad\quad
v^2 = \f{1}{2}\lp 1 - \f{\Delta_0}{\sqrt{\Omega_0^2 + \Delta_0^2}} \rp,
\quad\quad
uv = \f{1}{2}\f{\Omega_0}{\sqrt{\Omega_0^2 + \Delta_0^2}}.
\end{align}
Solving for the poles give the Rabi-coupled single-particle energies:
\begin{align}
E^{\pm}_\mathbf{k} = \epsilon_\mathbf{k} + \f{1}{2} \lp \Delta_0 \pm \sqrt{\Omega_0^2 + \Delta_0^2} \rp.
\end{align}
The spectral function, for the Green's functions here, can be obtained by inspection:
\begin{align}
A^{(\Omega)}_\uparrow (\mathbf{k}, \epsilon) &= v^2 \delta(\epsilon - E_\mathbf{k}^+) 
+ u^2 \delta(\epsilon - E_\mathbf{k}^-) \\ 
A^{(\Omega)}_\downarrow (\mathbf{k}, \epsilon) &= u^2 \delta(\epsilon - E_\mathbf{k}^+) 
+ v^2 \delta(\epsilon - E_\mathbf{k}^-).
\end{align}
Correspondingly the lineshape for the RF transfer from one spin state to another is simply a delta-function. 


\subsection{Interacting case}

Now we turn on the interaction between the impurity $F$ and the medium $B$, the Green's function is modified due to a nontrivial self-energy term:
\begin{align}
\mathbf{G}(\mathbf{p}, \epsilon) = \lb \mathbf{G}_\Omega (\mathbf{p}, \epsilon)^{-1} - \mathbf{\Sigma}(\mathbf{p}, \epsilon) \rb^{-1}. 
\end{align}
Since $\ket{F_\uparrow}$ and $\ket{F_\downarrow}$ interact with the bath $B$ and not with each other, the self-energy is 
\begin{align}
\mathbf{\Sigma}(\mathbf{p}, \epsilon) = 
\begin{pmatrix}
\Sigma_\uparrow(\mathbf{p}, \epsilon) & 0 \\ 
0 & \Sigma_\downarrow(\mathbf{p}, \epsilon)
\end{pmatrix}.
\end{align}
The impurity Green's functions are then 
\begin{align}
G_\uparrow(\mathbf{p}, \epsilon) 
&=  \f{1}{ \epsilon - \epsilon_\mathbf{p} - \Sigma_\uparrow(\mathbf{p}, \epsilon) - \f{\Omega_0^2}{4} \f{1}{\epsilon - \epsilon_\mathbf{p} - \Delta_0 - \Sigma_\downarrow(\mathbf{p}, \epsilon)}}\\ 
G_\downarrow(\mathbf{p}, \epsilon) 
&=  \f{1}{\epsilon - \epsilon_\mathbf{p} -\Delta_0 - \Sigma_\downarrow(\mathbf{p}, \epsilon) - \f{\Omega^2}{4} \f{1}{\epsilon - \epsilon_\mathbf{p} - \Sigma_\uparrow(\mathbf{p}, \epsilon)}}.
\end{align}
Here, the self-energy implicitly depends on the Rabi frequency $\Omega_0$. In the case that $\Omega_0 \to 0$, the impurity Green's functions take a more familiar form
\begin{align}\label{eq:zero-omega}
G_\uparrow(\mathbf{k}, \epsilon) 
=  \f{1}{ \epsilon - \epsilon_\mathbf{k} - \Sigma_\uparrow(\mathbf{k}, \epsilon) }
\quad\quad\quad
G_\downarrow(\mathbf{k}, \epsilon) 
=  \f{1}{\epsilon - \epsilon_\mathbf{k} -\Delta_0 - \Sigma_\downarrow(\mathbf{k}, \epsilon) }.
\end{align}

\subsubsection{The self-energy $\Sigma_{\uparrow/\downarrow} (\mathbf{k}, \epsilon)$ and the spectral function $A_{\uparrow/\downarrow}(\mathbf{k}, \epsilon)$} 

\textcolor{red}{See Mahan, Section 4.1.4: Potential Scattering: Impurity scattering}\\


\noindent Let's assume that $\Re[\Sigma (\mathbf{k}, \epsilon)] = 0$ for now and only care the imaginary part of the self-energy. It turns out that there is a simple formula:
\begin{align}
-2 \Im [\Sigma (\mathbf{k}, \epsilon)] = 2 \pi n_B \int \f{d^3 p }{(2\pi)^3} \abs{T_{\mathbf{k}\mathbf{p}}}^2 \delta(\epsilon - \epsilon_\mathbf{p}),
\end{align}
where $n_B$ is the density of the bath of $B$ particles and $T_{\mathbf{k} \mathbf{p}}$ represents the $T$-matrix. In the special case where $\epsilon = \epsilon_\mathbf{k}$, the diagonal element $T_{\mathbf{kk}}$ has a simple expression, which gives 
\begin{align}
2 \Im [\Sigma (\mathbf{k}, \epsilon_\mathbf{k})] = -2 n_B \Im[T_\mathbf{kk}] = n_B v_F \sigma(k) = n_B \sqrt{2m_F \epsilon_\mathbf{k}} \sigma(k),
\end{align}
where $\sigma(k) \sim 4 \pi a_{B-F}^2$ is the scattering cross-section. \\

\noindent In general, the $T$-matrix is given by the integral equation:
\begin{align}
T_{\mathbf{k}'\mathbf{k}} = V_{\mathbf{k}'\mathbf{k}} + \mathcal{P}\sum_{\mathbf{k}_1} \f{ V_{\mathbf{k}'\mathbf{k}_1}  T_{\mathbf{k}_1\mathbf{k}}  }{\epsilon_\mathbf{k} - \epsilon_{\mathbf{k}_1} + i\delta}
\end{align}
where
\begin{align}
V_{\mathbf{k}'\mathbf{k}} = \int d^3 r V(r) e^{-i \mathbf{r} \cdot (\mathbf{k}' - \mathbf{k})} = V(\mathbf{k}' - \mathbf{k}). 
\end{align}
Suppose that $V(r) = (4\pi \hbar^2 a/m) \delta(r)$, then $V_{\mathbf{k k'}}$ is identically $4\pi \hbar^2 a/m$. 



\subsubsection{RF response}

In experiments, the RF field transfers atoms from an occupied spin state into an empty final state. From the previous sections, we have found the spectral functions for both. Let's assume that the rf field flips the spin $\uparrow$ to $\downarrow$. Within linear response, the rate of transitions out of the initial state induced by the rf field with frequency $\omega$ and wave vector $\mathbf{q}$ is given by the convolution 
\begin{align}
I(\mathbf{q}, \omega)
\propto 
\int \f{d^3 k}{(2\pi)^3} \int d\epsilon \, A_\downarrow(\mathbf{k} + \mathbf{q}, \epsilon + \hbar \omega) A_\uparrow(\mathbf{k}, \epsilon) e^{-\be \epsilon}.
\end{align}
Here, unwritten is a prefactor that depends on the coupling strength to the rf field. Since the momentum transfer due to the rf field is very small, we can simply set $\mathbf{q} = 0$. With this simplification, the rf spectrum is 
\begin{align}
I(\omega) 
\propto
\int \f{d^3k}{(2\pi)^3} 
\int d\epsilon \, A_\downarrow(\mathbf{k}, \epsilon + \hbar \omega) A_\uparrow(\mathbf{k}, \epsilon)
\end{align}
If the final state is non-interacting, its spectral function is a delta-function:
\begin{align}
A_{\downarrow, \text{non-interacting}} (\mathbf{k},\epsilon) 
\propto
\delta(\epsilon - \epsilon_\mathbf{k}).
\end{align}
This gives the rf response function for this case:
\begin{align}
I_{\downarrow \text{ non-interacting}}(\omega) \propto 
\hbar \int \f{d^3k}{(2\pi)^3}  A_\uparrow(\mathbf{k}, \epsilon_\mathbf{k} - \hbar \omega).
\end{align}





\bibliography{free-to-free-lineshape} 
\bibliographystyle{unsrt}	

\end{document}