\documentclass{article}
\usepackage{physics}
\usepackage{graphicx}
\usepackage{caption}
\usepackage{amsmath}
\usepackage{bm}
\usepackage{framed}
\usepackage{authblk}
\usepackage{empheq}
\usepackage{amsfonts}
\usepackage{esint}
\usepackage[makeroom]{cancel}
\usepackage{dsfont}
\usepackage{centernot}
\usepackage{mathtools}
\usepackage{subcaption}
\usepackage{bigints}
\usepackage{amsthm}
\theoremstyle{definition}
\newtheorem{lemma}{Lemma}
\newtheorem{defn}{Definition}[section]
\newtheorem{prop}{Proposition}[section]
\newtheorem{rmk}{Remark}[section]
\newtheorem{thm}{Theorem}[section]
\newtheorem{exmp}{Example}[section]
\newtheorem{prob}{Problem}[section]
\newtheorem{sln}{Solution}[section]
\newtheorem*{prob*}{Problem}
\newtheorem{exer}{Exercise}[section]
\newtheorem*{exer*}{Exercise}
\newtheorem*{sln*}{Solution}
\usepackage{empheq}
\usepackage{tensor}
\usepackage{xcolor}
%\definecolor{colby}{rgb}{0.0, 0.0, 0.5}
\definecolor{MIT}{RGB}{163, 31, 52}
\usepackage[pdftex]{hyperref}
%\hypersetup{colorlinks,urlcolor=colby}
\hypersetup{colorlinks,linkcolor={MIT},citecolor={MIT},urlcolor={MIT}}  
\usepackage[left=1in,right=1in,top=1in,bottom=1in]{geometry}

\usepackage{newpxtext,newpxmath}
\newcommand*\widefbox[1]{\fbox{\hspace{2em}#1\hspace{2em}}}

\newcommand{\p}{\partial}
\newcommand{\R}{\mathbb{R}}
\newcommand{\C}{\mathbb{C}}
\newcommand{\lag}{\mathcal{L}}
\newcommand{\nn}{\nonumber}
\newcommand{\ham}{\mathcal{H}}
\newcommand{\M}{\mathcal{M}}
\newcommand{\I}{\mathcal{I}}
\newcommand{\K}{\mathcal{K}}
\newcommand{\F}{\mathcal{F}}
\newcommand{\w}{\omega}
\newcommand{\lam}{\lambda}
\newcommand{\al}{\alpha}
\newcommand{\be}{\beta}
\newcommand{\x}{\xi}

\newcommand{\G}{\mathcal{G}}

\newcommand{\f}[2]{\frac{#1}{#2}}

\newcommand{\ift}{\infty}

\newcommand{\lp}{\left(}
\newcommand{\rp}{\right)}

\newcommand{\lb}{\left[}
\newcommand{\rb}{\right]}

\newcommand{\lc}{\left\{}
\newcommand{\rc}{\right\}}


\newcommand{\V}{\mathbf{V}}
\newcommand{\U}{\mathcal{U}}
\newcommand{\Id}{\mathcal{I}}
\newcommand{\D}{\mathcal{D}}
\newcommand{\Z}{\mathcal{Z}}

%\setcounter{chapter}{-1}


\usepackage{enumitem}



\usepackage{listings}
\captionsetup[lstlisting]{margin=0cm,format=hang,font=small,format=plain,labelfont={bf,up},textfont={it}}
\renewcommand*{\lstlistingname}{Code \textcolor{violet}{\textsl{Mathematica}}}
\definecolor{gris245}{RGB}{245,245,245}
\definecolor{olive}{RGB}{50,140,50}
\definecolor{brun}{RGB}{175,100,80}

%\hypersetup{colorlinks,urlcolor=colby}
\lstset{
	tabsize=4,
	frame=single,
	language=mathematica,
	basicstyle=\scriptsize\ttfamily,
	keywordstyle=\color{black},
	backgroundcolor=\color{gris245},
	commentstyle=\color{gray},
	showstringspaces=false,
	emph={
		r1,
		r2,
		epsilon,epsilon_,
		Newton,Newton_
	},emphstyle={\color{olive}},
	emph={[2]
		L,
		CouleurCourbe,
		PotentielEffectif,
		IdCourbe,
		Courbe
	},emphstyle={[2]\color{blue}},
	emph={[3]r,r_,n,n_},emphstyle={[3]\color{magenta}}
}


\begin{document}
\begin{framed}
\noindent Name: \textbf{Huan Q. Bui}\\
Title: Franck-Condon calculation for the RF association lineshape of Feshbach molecules\\
Problem set: \textbf{\#9}\\
Due: Friday, April 21, 2022\\
Collaborators:  
\end{framed}
	

\noindent \textbf{1. Non-perturbative calculation of the rf response of a Feshbach molecule.} In this problem we calculate the response of a Feshbach molecule to an external rf drive non-perturbatively. The setup is as follows:



\begin{enumerate}[label=(\alph*)]

% a
\item Here we calculate the coupling strength. What does the rf photon do? It flips the spin of one of one of the atoms in the molecule into another hyperfine state where it is free. The result is one transferred atom and its leftover partner. The coupling strength depends not only on the Rabi frequency $\Omega_R$ of the rf drive but also on the (Franck-Condon) overlap between the initial state, which is the bound molecular wavefunction, and the final state, which is the free two-atom wavefunction:
\begin{align*}
V_{mk} = 
\f{\hbar \Omega_R}{2} \int_0^\infty \psi_k(r)^* \psi_m(r)\,dr =  
\f{\hbar \Omega_R}{2} \int_0^\infty \sqrt{\f{2}{R}} \sin(kr) \sqrt{\f{2}{a}} e^{-r/a}\,dr = 
\f{\hbar \Omega_R}{ \sqrt{aR}} \f{k}{1/a^2 + k^2},
\end{align*}
which attains a maximum of $V_{mk,\text{max}} = (\hbar \Omega_R/2)\sqrt{a/R}$ at $k=1/a$, and is significant for $k \sim 1/a$. 

We can also find the $k$-dependence of $V_{mk}$ at low and high $k$'s:
\begin{align*}
&\text{Low } \,\, k: \quad\quad V_{mk} \to \f{\hbar \Omega_R}{\sqrt{R}} a^{3/2} k \\ 
&\text{High } k: \quad\quad V_{mk} \to \f{\hbar \Omega_R}{\sqrt{aR}} \f{1}{k}.
\end{align*}
So, $V_{mk} \propto k$ for low $k$ and $V_{mk} \propto 1/k$ for high $k$.




\end{enumerate}




\end{document}








