\documentclass{beamer}
 
\usepackage[utf8]{inputenc}


\usetheme{Madrid}
\usecolortheme{default}
\usepackage{caption}
\usepackage{subcaption}
\usepackage{hhline}
\usepackage{graphicx}
\usepackage{physics}
\usepackage{amsmath}
\usepackage{amsfonts}
\usepackage{esint}
\usepackage{bbold}
\usepackage{mathtools}
\usepackage{dsfont}
\usepackage{amsthm}
\usepackage{bbm}
\usepackage{amssymb}
\theoremstyle{definition}
\newtheorem{defn}{Definition}[section]
\newtheorem{prop}{Properties}[section]
\newtheorem{rmk}{Remark}[section]
\newtheorem{exmp}{Example}[section]
\newtheorem{prob}{Problem}[section]
\newtheorem{proposition}{Proposition}
\newtheorem{thm}{Theorem}[section]
\newtheorem*{prob*}{Problem}
\newtheorem*{sln*}{Solution}
\usepackage{empheq}
\usepackage{tensor}
\usepackage{MnSymbol,wasysym}

\newcommand{\lag}{\mathcal{L}}
\newcommand{\pOne}{\text{5p}_\text{1/2}}
\newcommand{\pThree}{\text{5p}_\text{3/2}}
\newcommand{\potassium}{^\text{39}\text{K}}
\newcommand{\R}{\mathbb{R}}
\newcommand{\lp}{\left(}
\newcommand{\rp}{\right)}
\newcommand{\lb}{\left[}
\newcommand{\rb}{\right]}
\newcommand{\lc}{\left\{}
\newcommand{\rc}{\right\}}
\newcommand{\p}{\partial}
\newcommand{\f}[2]{\frac{#1}{#2}}
\newcommand{\Vol}{\operatorname{Vol}}
\newcommand{\iprod}{\mathbin{\lrcorner}}
\newcommand{\al}{\alpha}
\newcommand{\be}{\beta}
\newcommand{\FT}{\mathcal{F}}
\newcommand{\LT}{\mathcal{L}}
\usepackage{hyperref}
\usepackage{tensor}
\usepackage{xcolor}
\hypersetup{
	colorlinks,
	linkcolor={black!50!black},
	citecolor={blue!50!black},
	urlcolor={blue!80!black}
}

% 3j symbol
\newcommand{\tj}[6]{ \begin{pmatrix}
		#1 & #2 & #3 \\
		#4 & #5 & #6 
\end{pmatrix}}
% 6j symbol
\newcommand{\Gj}[6]{ \begin{Bmatrix}
		#1 & #2 & #3 \\
		#4 & #5 & #6 
\end{Bmatrix}}

\setbeamerfont{title}{size=\large}

 
 
%Information to be included in the title page:
\title[\textcolor{white}{{}}]
{
	How to truncate big Hilbert spaces? 
}



\author[Bui] % (optional)
{Huan Q. Bui
	}
\institute[MIT] % (optional)
{
}
\date{ZGS, Jan 28, 2022}
 
%\logo{\includegraphics[height=0.3cm]{colby.png}}
 
\begin{document}
 
\frame{\titlepage}


\begin{frame}
	\frametitle{Outline}
	\begin{itemize}
		\item Motivation
		\item Compressing $\ket{\Psi}$ with SVD
		\item MPS and DMRG
	\end{itemize}
\end{frame}


\begin{frame}
	\frametitle{Motivation}
	
	$N$ sites, each is a spin-$1/2$. Find ground state of:
	\begin{align*}
		\mathcal{H} = - J \sum_{i=1}^N \sigma_i^z \sigma_{i+1}^z - h \sum_{i=1}^N \sigma^x_i
	\end{align*}	
	
	
	Hilbert space dimension $\sim 2^N$\\
	
	\vspace{8pt}
	
	Exact diagonalization O.K. for $N \lessapprox 20$ on laptop\\
	
	\vspace{8pt}
	
	$N \to \infty$ (thermodynamic limit): needle in the haystack\\
	
	\vspace{8pt}
	
	{$\boxed{!!}$ For most relevant Hamiltonians, haystack $\ll$ full Hilbert space }\\
	\vspace{2pt}
	{$\quad\,\,\,\,$e.g. haystack $\sim$ subspace of states with low entanglement entropy }\\
	\vspace{2pt}
	{$\quad\,\,\implies$ Clever parameterization + efficient algorithms = \smiley{}?}
	
		
	
\end{frame}



\begin{frame}
	\frametitle{Compressing $\ket{\Psi}$ with SVD}
	
	\begin{theorem}[Singular value decomposition]
		blah
	\end{theorem}
	
	
	
	Low-rank approximation: 
	\begin{theorem}[Low-rank approximation]
		
	\end{theorem}
	
	
	Applications: image compression
\end{frame}

\begin{frame}
	\frametitle{Compressing $\ket{\Psi}$ with SVD}
	\underline{Idea}: represent $\ket{\Psi}$ as a matrix, then SVD\\
	
	\vspace{8pt}
	
	Split $N$ spins on a 1d chain into $L+R$: 
	\begin{align*}
		\ket{\Psi} = \sum_{l,r} \psi_{lr} \ket{l}\ket{r}, \quad\quad\quad \ket{l}\in \mathbb{H}_L
	\end{align*}
	$\psi_{lr}$ has two indices $\implies$ treat as a matrix (NOT an operator!) \\ 
	
	\vspace{8pt} 
	
	Apply SVD: $\quad\psi_{lr} = [\mathbf{U}\,\,\mathbf{D}\,\,\mathbf{V}]_{lr}$\\
	
	\vspace{8pt}
	
	$\mathbf{U}, \mathbf{V}$ are unitary. $\mathbf{D} = \text{diag}(\lambda_1,\lambda_2,\dots)$:
	\begin{align*}
		\lambda_i\text{'s} 
		&= \text{singular values of } \psi_{lr} \\
		&= \text{eigenvalues of } \sqrt{\psi^\dagger\psi} = \sqrt{\rho} \implies \lambda_i^2 = \text{eigenvalues of } \rho
	\end{align*}
	
\end{frame}


\begin{frame}
	\frametitle{Compressing $\ket{\Psi}$ with SVD}
	
	After SVD:
	\begin{align*}
		\ket{\Psi} &= \sum_{l,r} \sum_{i} \mathbf{U}_{li}\mathbf{D}_{ii} \mathbf{V}_{ir} \ket{l}\ket{r}\\
		&= \sum_{i} \sum_{\textcolor{blue}{l},\textcolor{red}{r}}  \textcolor{blue}{\mathbf{U}_{li}}\mathbf{D}_{ii} \textcolor{red}{\mathbf{V}_{ir} } \textcolor{blue}{\ket{l}}\textcolor{red}{\ket{r}}\\
		&= \sum_i \lambda_i \ket{i}_L \ket{i}_R \,\,\,\leftarrow \text{ Schmidt decomposition}
	\end{align*}
	Normalization:
	\begin{align*}
		\Tr(\psi^\dagger\psi) = \sum_i \lambda_i^2 = 1 \implies \lambda_i^2 \text{: probability for i$^\text{th}$ Schmidt state pair}
	\end{align*}
\end{frame}


\begin{frame}
	\frametitle{Compressing $\ket{\Psi}$ with SVD}
	
	Why SVD and Schmidt decomposition? \\
	
	\vspace{8pt}
	
	$\quad$ SVD compression $\equiv$ make states with low entanglement entropy\\
	
	\vspace{8pt}
	
	von Neumann entanglement entropy between $L$ and $R$:
	\begin{align*}
		S(\rho_L) = -\Tr[\rho_L \ln \rho_L] = -\Tr[\rho_R \ln \rho_R] = S(\rho_R) 
	\end{align*}
	\begin{align*}
		[\rho_L]_{ll'} = \sum_r \psi^*_{lr}\psi_{l'r} \quad\quad [\rho_R]_{rr'} = \sum_l \psi^*_{lr}\psi_{lr'}
	\end{align*}
	Fact: Eigenvalues of $\rho_L, \rho_R$ are exactly the $\lambda_i$'s. So,
	\vspace{-7pt}
	\begin{align*}
		&\quad\quad \quad \,\, S = S(\rho_L) = S(\rho_R) = -\sum_i^{\sim 2^{N/2}} \lambda_i^2 \ln \lambda_i^2 \to -\sum_i^m \lambda_i^2 \ln \lambda_i^2\\
		&\text{\textcolor{blue}{Drop small $\lambda_i$'s $\implies$ reduce $S$ and exponential compression, $m\sim \mathcal{O}(100)$}}
	\end{align*}
	
\end{frame}


\begin{frame}
	\frametitle{Compressing $\ket{\Psi}$ with SVD}
	
	\underline{Example}:  
	\begin{align*}
		\ket{\Psi} = \f{1}{\sqrt{2}}\lp \ket{\uparrow\uparrow} - \ket{\uparrow\downarrow} \rp
	\end{align*}
	Matrixify and SVD:
	\begin{align*}
	&\ket{\Psi} = \sum_{ij}\psi_{ij}\ket{i}\ket{j} \quad \text{ with } \quad   [\psi_{ij}] = \begin{pmatrix}
			\f{1}{\sqrt{2}} & -\f{1}{\sqrt{2}} \\ 0 & 0
		\end{pmatrix}\\
	&\quad\quad \,\,\,[\psi_{ij}] = \begin{pmatrix}
			1 & 0 \\ 0 & 1
		\end{pmatrix}
	\underbrace{\begin{pmatrix}
		1 & 0 \\ 0 & 0 
	\end{pmatrix}}_D
	\begin{pmatrix}
		\f{1}{\sqrt{2}} & -\f{1}{\sqrt{2}} \\ 	\f{1}{\sqrt{2}} & \f{1}{\sqrt{2}}
	\end{pmatrix}
	\end{align*}
	Entanglement entropy is 0 $\implies$ not entangled (makes sense)
\end{frame}



\begin{frame}
	\frametitle{But wait...}
	
	We need $\ket{\Psi}$ to compress. But we want to find/approximate such a $\ket{\Psi}$. \\
	
	\vspace{8pt}
	
	Insert chicken and egg here.\\
	
	\vspace{8pt}
	
	Solution:
	\begin{itemize}
		\item 
	\end{itemize}
\end{frame}


\begin{frame}
	\frametitle{MPS and DMRG}
\end{frame}


%%%%%%%%%%%%%%%%%%%%%%%%%%%%%%%%%%%%%%%%%%%%%%%%%%%%%%%%%%%%%%%%%%%%%%%%%

 
%\begin{frame}
%\frametitle{Layout}
%\tableofcontents
%\end{frame}

%%%%%%%%%%%%%%%%%%%%%%%%%%%%%%%%%%%%%%%%%%%%%%%%%%%%%%%%%%%%%%%%%%%%%%%%%




\begin{frame}
\textcolor{purple}{{Phys. Rev. Lett. \textbf{127}, 073604 – Published 13 August 2021}}
\end{frame}





\end{document}