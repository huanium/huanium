\documentclass{article}
\usepackage{physics}
\usepackage{graphicx}
\usepackage{caption}
\usepackage{amsmath}
\usepackage{bm}
\usepackage{framed}
\usepackage{authblk}
\usepackage{empheq}
\usepackage{amsfonts}
\usepackage{esint}
\usepackage[makeroom]{cancel}
\usepackage{dsfont}
\usepackage{centernot}
\usepackage{mathtools}
\usepackage{bigints}
\usepackage{amsthm}
\theoremstyle{definition}
\newtheorem{defn}{Definition}[section]
\newtheorem{prop}{Proposition}[section]
\newtheorem{rmk}{Remark}[section]
\newtheorem{thm}{Theorem}[section]
\newtheorem{exmp}{Example}[section]
\newtheorem{prob}{Problem}[section]
\newtheorem{sln}{Solution}[section]
\newtheorem*{prob*}{Problem}
\newtheorem{exer}{Exercise}[section]
\newtheorem*{exer*}{Exercise}
\newtheorem*{sln*}{Solution}
\usepackage{empheq}
\usepackage{tensor}
\usepackage{xcolor}
%\definecolor{colby}{rgb}{0.0, 0.0, 0.5}
\definecolor{MIT}{RGB}{163, 31, 52}
\usepackage[pdftex]{hyperref}
%\hypersetup{colorlinks,urlcolor=colby}
\hypersetup{colorlinks,linkcolor={MIT},citecolor={MIT},urlcolor={MIT}}  
\usepackage[left=1in,right=1in,top=1in,bottom=1in]{geometry}

\usepackage{newpxtext,newpxmath}
\newcommand*\widefbox[1]{\fbox{\hspace{2em}#1\hspace{2em}}}

\newcommand{\p}{\partial}
\newcommand{\R}{\mathbb{R}}
\newcommand{\C}{\mathbb{C}}
\newcommand{\lag}{\mathcal{L}}
\newcommand{\nn}{\nonumber}
\newcommand{\ham}{\mathcal{H}}
\newcommand{\M}{\mathcal{M}}
\newcommand{\I}{\mathcal{I}}
\newcommand{\K}{\mathcal{K}}
\newcommand{\F}{\mathcal{F}}
\newcommand{\w}{\omega}
\newcommand{\lam}{\lambda}
\newcommand{\al}{\alpha}
\newcommand{\be}{\beta}
\newcommand{\x}{\xi}

\newcommand{\G}{\mathcal{G}}

\newcommand{\f}[2]{\frac{#1}{#2}}

\newcommand{\ift}{\infty}

\newcommand{\lp}{\left(}
\newcommand{\rp}{\right)}

\newcommand{\lb}{\left[}
\newcommand{\rb}{\right]}

\newcommand{\lc}{\left\{}
\newcommand{\rc}{\right\}}


\newcommand{\V}{\mathbf{V}}
\newcommand{\U}{\mathcal{U}}
\newcommand{\Id}{\mathcal{I}}
\newcommand{\D}{\mathcal{D}}
\newcommand{\Z}{\mathcal{Z}}

%\setcounter{chapter}{-1}


\usepackage{enumitem}



\usepackage{subfig}
\usepackage{listings}
\captionsetup[lstlisting]{margin=0cm,format=hang,font=small,format=plain,labelfont={bf,up},textfont={it}}
\renewcommand*{\lstlistingname}{Code \textcolor{violet}{\textsl{Mathematica}}}
\definecolor{gris245}{RGB}{245,245,245}
\definecolor{olive}{RGB}{50,140,50}
\definecolor{brun}{RGB}{175,100,80}

%\hypersetup{colorlinks,urlcolor=colby}
\lstset{
	tabsize=4,
	frame=single,
	language=mathematica,
	basicstyle=\scriptsize\ttfamily,
	keywordstyle=\color{black},
	backgroundcolor=\color{gris245},
	commentstyle=\color{gray},
	showstringspaces=false,
	emph={
		r1,
		r2,
		epsilon,epsilon_,
		Newton,Newton_
	},emphstyle={\color{olive}},
	emph={[2]
		L,
		CouleurCourbe,
		PotentielEffectif,
		IdCourbe,
		Courbe
	},emphstyle={[2]\color{blue}},
	emph={[3]r,r_,n,n_},emphstyle={[3]\color{magenta}}
}






\begin{document}
\begin{framed}
	\noindent Name: \textbf{Huan Q. Bui}\\
	Course: \textbf{8.309 - Classical Mechanics III}\\
	Problem set: \textbf{\#1}\\
	Re-grade request: Problem 2(b)
\end{framed}
	
	
	
\noindent \textbf{2. } \textbf{Double Pendulum in a Plane with Gravity}\\

	


\noindent \textcolor{blue}{Re-grade justification:} In my original write-up, my Hamiltonian has a small typo, which is a missing parenthesis highlighted in red in the expression below. However, this missing parenthesis is a genuine typo because, as I will show below, the rest of my solution (equations of motion involving $\dot{p}_{\theta_1}$ and $\dot{p}_{\theta_2}$) is otherwise correct. Moreover, since the computations were carried out in Mathematica (the code is in my write-up), there not should not be inaccuracies beyond typos, so long as the setup is correct (and mine is).  \\


\noindent First, I will simplify my Hamiltonian so that it matches the solution:
\begin{align*}
\ham &= 
-\frac{l_1^2 \left(g{l_2}^2 m^2 [\cos (2 (\theta_1-\theta_2))-3] 
	(2l_1 \cos \theta_1+l_2 \cos \theta_2 \textcolor{red}{)} + 2p_{\theta_2}^2\right)
	-2l_1l_2p_{\theta_1}p_{\theta_2} \cos (\theta_1-\theta_2)+ l_2^2 p_{\theta_1}^2}
{l_1^2l_2^2 m [\cos (2 (\theta_1-\theta_2))-3]}\\
&= -2l_1gm\cos\theta_1 - l_2gm\cos\theta_2 - \f{2l_1^2p_{\theta_2}^2 - 2l_1l_2 p_{\theta_1}p_{\theta_2}\cos(\theta_1 - \theta_2) + l_2^2p_{\theta_1}^2}{l_1^2l_2^2 m [\cos (2 (\theta_1-\theta_2))-3]}\\
&= -2l_1gm\cos\theta_1 - l_2gm\cos\theta_2 + 
\f{2l_1^2p_{\theta_2}^2 - 2l_1l_2p_{\theta_1}p_{\theta_2}\cos(\theta_1-\theta_2) + l_2^2p_{\theta_1}^2}{2l_1^2l_2^2m(1+\sin^2(\theta_1-\theta_2))}.
\end{align*}\qed



\noindent Next, I will show that my equations of motion for the $\dot{p}$'s are also the same as the solution's.
\begin{align*}
\dot{p}_{\theta_1} &= -\f{\p \ham}{\p \theta_1}  = 
\frac{-2 g l_1^3 l_2^2 m^2 \sin \theta_1[\cos(2 (\theta_1-\theta_2))-3]^2+2 \sin (2(\theta_1-\theta_2)) \left(2 l_1^2p_{\theta_2}^2+l_2^2 p_{\theta_1}^2\right)}{l_1^2 l_2^2 m [\cos (2(\theta_1-\theta_2))-3]^2} \\
&\quad\quad\quad\quad\quad+ \frac{-2l_1l_2 p_{\theta_1}p_{\theta_2} \sin(\theta_1-\theta_2) [\cos (2 (\theta_1-\theta_2))+5]}{l_1^2 l_2^2 m [\cos (2(\theta_1-\theta_2))-3]^2}\\
&= -2gl_1m \sin\theta_1 + \f{4\sin(\theta_1-\theta_2)\cos(\theta_1-\theta_2)\left(2 l_1^2p_{\theta_2}^2+l_2^2 p_{\theta_1}^2\right)}{4l_1^2l_2^2m[1+\sin^2(\theta_1-\theta_2)]^2}
- \frac{2l_1l_2 p_{\theta_1}p_{\theta_2} \sin(\theta_1-\theta_2) [2\cos^2 (\theta_1-\theta_2)+4]}{4l_1^2l_2^2m[1+\sin^2(\theta_1-\theta_2)]^2}\\
&= -2gl_1m \sin\theta_1 + \f{\sin(\theta_1-\theta_2)\cos(\theta_1-\theta_2)\left(2 l_1^2p_{\theta_2}^2+l_2^2 p_{\theta_1}^2\right)}{l_1^2l_2^2m[1+\sin^2(\theta_1-\theta_2)]^2}+ \frac{-l_1l_2 p_{\theta_1}p_{\theta_2} \sin(\theta_1-\theta_2) [\cos^2 (\theta_1-\theta_2)+2]}{l_1^2l_2^2m[1+\sin^2(\theta_1-\theta_2)]^2}\\
&= -2gl_1m \sin\theta_1 + \f{\sin(\theta_1-\theta_2)\lb \cos(\theta_1-\theta_2)\left(2 l_1^2p_{\theta_2}^2+l_2^2 p_{\theta_1}^2\right) -l_1l_2 p_{\theta_1}p_{\theta_2}(\cos^2(\theta_1-\theta_2)+2) \rb}{l_1^2l_2^2m[1+\sin^2(\theta_1-\theta_2)]^2}\\
&= -2gl_1m \sin\theta_1 + \f{\sin(\theta_1-\theta_2)\lb l_2p_{\theta_1}\cos(\theta_1-\theta_2) -2l_1p_{\theta_2} \rb 
\lb l_2p_{\theta_1} - l_1p_{\theta_2}\cos(\theta_1-\theta_2) \rb}
{l_1^2l_2^2m[1+\sin^2(\theta_1-\theta_2)]^2}.
\end{align*}\qed


\noindent This matches the solution. And finally, 



\begin{align*}
\dot{p}_{\theta_2} = &-\f{\p \ham}{\p \theta_2}
= - g l_2 m \sin\theta_2 \\
&+\frac{2 \sin (\theta_1-\theta_2) \left(-4
	l_1^2 p_{\theta_2}^2 \cos (\theta_1-\theta_2)+l_1 l_2 p_{\theta_1}p_{\theta_2} [\cos (2 (\theta_1-\theta_2))+5]-2 l_2^2 p_{\theta_1}^2 \cos (\theta_1-\theta_2)\right)}{l_1^2 l_2^2 m [\cos (2 (\theta_1-\theta_2))-3]^2}\\
&= - g l_2 m \sin\theta_2 + \f{2\sin(\theta_1-\theta_2)
	\lb -4
	l_1^2 p_{\theta_2}^2 \cos (\theta_1-\theta_2)+l_1 l_2 p_{\theta_1}p_{\theta_2} [2\cos^2( \theta_1-\theta_2)+4]-2 l_2^2 p_{\theta_1}^2 \cos (\theta_1-\theta_2) \rb
}{4l_1^2l_2^2m[1+\sin^2(\theta_1-\theta_2)]^2}\\
&= - g l_2 m \sin\theta_2 + \f{\sin(\theta_1-\theta_2)
	\lb -2
	l_1^2 p_{\theta_2}^2 \cos (\theta_1-\theta_2)+l_1 l_2 p_{\theta_1}p_{\theta_2} [\cos^2( \theta_1-\theta_2)+2]- l_2^2 p_{\theta_1}^2 \cos (\theta_1-\theta_2) \rb
}{l_1^2l_2^2m[1+\sin^2(\theta_1-\theta_2)]^2}\\
&= - g l_2 m \sin\theta_2 + \f{-\sin(\theta_1-\theta_2)
	\lb l_2p_{\theta_1}\cos(\theta_1-\theta_2) -2l_1p_{\theta_2} \rb 
	\lb l_2p_{\theta_1} - l_1p_{\theta_2}\cos(\theta_1-\theta_2) \rb
}{l_1^2l_2^2m[1+\sin^2(\theta_1-\theta_2)]^2}\\
&= - g l_2 m \sin\theta_2 + \f{\sin(\theta_2-\theta_1)
	\lb l_2p_{\theta_1}\cos(\theta_1-\theta_2) -2l_1p_{\theta_2} \rb 
	\lb l_2p_{\theta_1} - l_1p_{\theta_2}\cos(\theta_1-\theta_2) \rb
}{l_1^2l_2^2m[1+\sin^2(\theta_1-\theta_2)]^2},
\end{align*}
which also matches the solution. \qed


	
	
	
\end{document}



