\documentclass{article}
\usepackage{physics}
\usepackage{graphicx}
\usepackage{caption}
\usepackage{amsmath}
\usepackage{bm}
\usepackage{framed}
\usepackage{authblk}
\usepackage{empheq}
\usepackage{amsfonts}
\usepackage{esint}
\usepackage[makeroom]{cancel}
\usepackage{dsfont}
\usepackage{centernot}
\usepackage{mathtools}
\usepackage{bigints}
\usepackage{amsthm}
\theoremstyle{definition}
\newtheorem{lemma}{Lemma}
\newtheorem{defn}{Definition}[section]
\newtheorem{prop}{Proposition}[section]
\newtheorem{rmk}{Remark}[section]
\newtheorem{thm}{Theorem}[section]
\newtheorem{exmp}{Example}[section]
\newtheorem{prob}{Problem}[section]
\newtheorem{sln}{Solution}[section]
\newtheorem*{prob*}{Problem}
\newtheorem{exer}{Exercise}[section]
\newtheorem*{exer*}{Exercise}
\newtheorem*{sln*}{Solution}
\usepackage{empheq}
\usepackage{tensor}
\usepackage{xcolor}
%\definecolor{colby}{rgb}{0.0, 0.0, 0.5}
\definecolor{MIT}{RGB}{163, 31, 52}
\usepackage[pdftex]{hyperref}
%\hypersetup{colorlinks,urlcolor=colby}
\hypersetup{colorlinks,linkcolor={MIT},citecolor={MIT},urlcolor={MIT}}  
\usepackage[left=1in,right=1in,top=1in,bottom=1in]{geometry}

\usepackage{newpxtext,newpxmath}
\newcommand*\widefbox[1]{\fbox{\hspace{2em}#1\hspace{2em}}}

\newcommand{\p}{\partial}
\newcommand{\R}{\mathbb{R}}
\newcommand{\C}{\mathbb{C}}
\newcommand{\lag}{\mathcal{L}}
\newcommand{\nn}{\nonumber}
\newcommand{\ham}{\mathcal{H}}
\newcommand{\M}{\mathcal{M}}
\newcommand{\I}{\mathcal{I}}
\newcommand{\K}{\mathcal{K}}
\newcommand{\F}{\mathcal{F}}
\newcommand{\w}{\omega}
\newcommand{\lam}{\lambda}
\newcommand{\al}{\alpha}
\newcommand{\be}{\beta}
\newcommand{\x}{\xi}

\newcommand{\G}{\mathcal{G}}

\newcommand{\f}[2]{\frac{#1}{#2}}

\newcommand{\ift}{\infty}

\newcommand{\lp}{\left(}
\newcommand{\rp}{\right)}

\newcommand{\lb}{\left[}
\newcommand{\rb}{\right]}

\newcommand{\lc}{\left\{}
\newcommand{\rc}{\right\}}


\newcommand{\V}{\mathbf{V}}
\newcommand{\U}{\mathcal{U}}
\newcommand{\Id}{\mathcal{I}}
\newcommand{\D}{\mathcal{D}}
\newcommand{\Z}{\mathcal{Z}}

%\setcounter{chapter}{-1}


\usepackage{enumitem}



\usepackage{subfig}
\usepackage{listings}
\captionsetup[lstlisting]{margin=0cm,format=hang,font=small,format=plain,labelfont={bf,up},textfont={it}}
\renewcommand*{\lstlistingname}{Code \textcolor{violet}{\textsl{Mathematica}}}
\definecolor{gris245}{RGB}{245,245,245}
\definecolor{olive}{RGB}{50,140,50}
\definecolor{brun}{RGB}{175,100,80}

%\hypersetup{colorlinks,urlcolor=colby}
\lstset{
	tabsize=4,
	frame=single,
	language=mathematica,
	basicstyle=\scriptsize\ttfamily,
	keywordstyle=\color{black},
	backgroundcolor=\color{gris245},
	commentstyle=\color{gray},
	showstringspaces=false,
	emph={
		r1,
		r2,
		epsilon,epsilon_,
		Newton,Newton_
	},emphstyle={\color{olive}},
	emph={[2]
		L,
		CouleurCourbe,
		PotentielEffectif,
		IdCourbe,
		Courbe
	},emphstyle={[2]\color{blue}},
	emph={[3]r,r_,n,n_},emphstyle={[3]\color{magenta}}
}






\begin{document}
\begin{framed}
\noindent Name: \textbf{Huan Q. Bui}\\
Course: \textbf{8.321 - Quantum Theory I}\\
Problem set: \textbf{\#8}
\end{framed}
	



\noindent \textbf{1. }
\begin{enumerate}[label=(\alph*)]
	\item By virtue of separation of variables, the energy must be given by 
	\begin{align*}
	E = E_{z} + E_{xy}
	\end{align*}
	where $E_z$ is the energy from the infinite square well of length $L$ and $E_{xy}$ is the energy due to confinement in the annulus. We thus have
	\begin{align*}
	E_z = \f{\hbar^2 \pi^2 l^2}{2m L^2} = \f{\hbar^2}{2m} \lp \f{\pi l }{L} \rp^2, \quad\quad\quad l = 1,2,3,\dots
	\end{align*}
	The Schr\"{o}dinger equation for the radial confinement is 
	\begin{align*}
	-\f{\hbar^2}{2m } \laplacian \psi = E_{xy} \psi
	\end{align*}
	By separation of variables we may say $\psi = \psi(\rho,\phi) = R(\rho) \Phi(\phi)$, so that 
	\begin{align*}
	\f{1}{\rho}\p_\rho(\rho \p_\rho (R\Phi)) + \f{1}{\rho^2}\p^2_\phi (R\Phi) = -\f{2mE_{xy}}{\hbar^2}R\Phi \implies \f{\rho}{R}\p_\rho(\rho \p_\rho R) + \f{1}{\Phi}\p^2_\phi (\Phi) = -\f{2mE}{\hbar^2}\rho^2.
	\end{align*}
	After putting
	\begin{align*}
	\f{1}{\Phi}\p^2_\phi \Phi = - m^2 
	\end{align*}
	where $m$ is a natural number due to the single-valuedness of $\Phi$, we have an equation for $R$:
	\begin{align*}
	\rho\p_\rho(\rho \p_\rho R) = \lp -\f{2mE_{xy}}{\hbar^2}\rho^2 + m^2\rp R \implies \rho^2 R'' + \rho R' + \lp \f{2mE_{xy}}{\hbar^2}\rho^2 - m^2 \rp R = 0
	\end{align*}
	whose solution are provided as a linear combination of Bessel functions of the first and second kind:
	\begin{align*}
	R = c_1 J_m\lp \sqrt{\f{2mE_{xy}}{\hbar^2}}\rho\rp + c_2 N_m\lp \sqrt{\f{2mE_{xy}}{\hbar^2}}\rho\rp.
	\end{align*}
	Using the boundary conditions $R(\rho_a) = R(\rho_b) =0$ we may find $c_1,c_2$ from solving the system 
	\begin{align*}
	& c_1 J_m\lp \sqrt{\f{2mE_{xy}}{\hbar^2}}\rho_a\rp + c_2 N_m\lp \sqrt{\f{2mE_{xy}}{\hbar^2}}\rho_a\rp = 0\\
	& c_1 J_m\lp \sqrt{\f{2mE_{xy}}{\hbar^2}}\rho_b\rp + c_2 N_m\lp \sqrt{\f{2mE_{xy}}{\hbar^2}}\rho_b\rp = 0
	\end{align*}
	The result are two equal ratios which relate the $J,N$'s:
	\begin{align*}
	J_m\lp\sqrt{\f{2mE_{xy}}{\hbar^2}}\rho_b\rp N_m\lp \sqrt{\f{2mE_{xy}}{\hbar^2}}\rho_a \rp - N_m\lp \sqrt{\f{2mE_{xy}}{\hbar^2}}\rho_a \rp J_m\lp \sqrt{\f{2mE_{xy}}{\hbar^2}}\rho_b \rp = 0.
	\end{align*}
	Let $E_{xy} = E_{mn}$ and $k_{mn} = \sqrt{\f{2mE_{mn}}{\hbar^2}}$ be the $n$th root of the equation above. Then we have the full energy spectrum:
	\begin{align*}
	E = E_{xy} + E_z = \f{\hbar^2}{2m} \lp k_{mn}^2 + \lp \f{\pi l}{L} \rp^2 \rp
	\end{align*}
	where $l=1,2,3,\dots$ and $m=0,1,2,\dots$, as desired. 
	
	
	\item In the presence of $\vec{B} = B \hat{z}$, we have that
	\begin{align*}
	-i\hbar \grad  \to -i\hbar \grad - \f{e}{c}\vec{A} \implies \grad \to \grad - \lp \f{ie}{\hbar c} \rp \vec{A} 
	\end{align*}
	where
	\begin{align*}
	\vec{A} = \lp \f{B\rho_a^2}{\rho} \rp \hat{\phi}
	\end{align*}
	by virtue of Stokes's Theorem, as presented in the textbook. Here, the vector potential is such that $\curl \vec{A} = \vec{B} = 0$ in the annulus region.  Since $\vec{A}$ only has a nontrivial component in $\phi$, the partial derivative with respect to $\phi$ now changes as
	\begin{align*}
	\p_\phi \to \p_\phi - \f{ie}{\hbar c} \f{B\rho_a^2}{2}
	\end{align*}
	which modifies the $\Phi(\phi)$ equation to
	\begin{align*}
	\p^2_\phi \Phi = - m^2 \Phi \to \p_\phi^2 \Phi - \lp \f{ie}{\hbar c} \rp B\rho_a^2 \p_\phi \Phi + \lb m^2 - \lp \f{eB \rho_a^2}{2\hbar c} \rp^2 \rb \Phi = 0
	\end{align*}
	Due to the single-valuedness of $\Phi$, $m$ in this case is not necessarily an integer. Letting 
	\begin{align*}
	m^2 - \lp \f{eB \rho_a^2}{2\hbar c} \rp^2  = m'^2,
	\end{align*}
	we may repeat what we did before to find 
	\begin{align*}
	J_{m'}\lp\sqrt{\f{2mE_{xy}}{\hbar^2}}\rho_b\rp N_{m'}\lp \sqrt{\f{2mE_{xy}}{\hbar^2}}\rho_a \rp - N_{m'}\lp \sqrt{\f{2mE_{xy}}{\hbar^2}}\rho_a \rp J_{m'}\lp \sqrt{\f{2mE_{xy}}{\hbar^2}}\rho_b \rp = 0.
	\end{align*}
	Let $E_{xy} = E_{m'n}$ and $k_{m'n} = \sqrt{\f{2mE_{m'n}}{\hbar^2}}$ be the $n$th root of the equation above. Then we have the full energy spectrum:
	\begin{align*}
	E = E_{xy} + E_z = \f{\hbar^2}{2m} \lp k_{m'n}^2 + \lp \f{\pi l}{L} \rp^2 \rp
	\end{align*}
	like before. 
	

	\item Consider the ground state of both problems. In particular we look at the $\Phi$ solution. Ground state implies $m=0$ and $m' = 0$. The normalized $\Phi$ solution when $B=0$ is 
	\begin{align*}
	\Phi(\phi) = 1
	\end{align*}
	while the normalized $\Phi$ solution when $B\neq 0$ is 
	\begin{align*}
	\Phi(\phi) = \exp\lp i\f{ eB\rho_a^2}{2\hbar c} \phi\rp 
	\end{align*}
	Due to the single-valuedness of $\Phi$, we must have
	\begin{align*}
	\f{ eB\rho_a^2}{2\hbar c} = N
	\end{align*}
	where $N$ is an integer. So, we have ``flux quantization'':
	\begin{align*}
	\pi \rho_a^2 B = \f{2\pi N \hbar c}{e}, \quad\quad N\in \mathbb{Z}
	\end{align*}
\end{enumerate}


\noindent \textbf{2. }
\begin{enumerate}[label=(\alph*)]
	\item We this part we just compute:
	\begin{align*}
	[\Pi_x, \Pi_y] 
	&= \lb p_x - eA_x/c, p_y - eA_y/c \rb \\
	&= \lb p_x - eA_x/c, p_y \rb + \lb p_x - eA_x/c, - eA_y/c \rb  \\
	&= \cancel{\lb p_x,p_y \rb}  + (e/c)\lb - A_x, p_y  \rb + (e/c)\lb p_x, -A_y \rb + \cancel{(e/c)^2[-A_x,-A_y]}\\
	&= -i\hbar \f{e}{c}\f{\p}{\p y}A_x + i\hbar \f{e}{c} \f{\p }{ \p x} A_y \\
	&= i\hbar \f{eB}{c}
	\end{align*}
	
	\item The new Hamiltonian is 
	\begin{align*}
	\ham = \f{p_z^2}{2m} + \f{\Pi_x^2}{2m} + \f{\Pi_y^2}{2m}.
	\end{align*}
	Let us put $\widetilde{\Pi}_x = (c/eB) \Pi_x$  so that $[\widetilde{\Pi}_x, {\Pi}_y]= i\hbar$. In these new variables the Hamiltonian becomes
	\begin{align*}
	\ham = \f{p_z^2}{2m} + \f{e^2 B^2}{c^2}\f{\widetilde{\Pi}_x^2}{2m} + \f{{\Pi}_y^2}{2m}
	\end{align*}
	We may change our notation to make it more suggestive. Since $[\widetilde{\Pi}_x,\Pi_y] = i\hbar$, we may put $\widetilde{\Pi}_x = Y$, so that $[Y,\Pi_y] = i\hbar$. With this, the Hamiltonian is 
	\begin{align*}
	\ham = \f{p_z^2}{2m} + \lb \f{\Pi_y^2}{2m} + \f{m}{2} \f{e^2B^2}{m^2c^2}Y^2 + \f{\Pi_y^2}{2m} \rb
	\end{align*}
	The last two terms form a 1D QHO Hamiltonian. As a result, we immediately get the energy spectrum:
	\begin{align*}
	E = \f{\hbar^2 k^2}{2m} + \f{\hbar \abs{eB}}{mc}\lp n + \f{1}{2}\rp, \quad\quad n\in \mathbb{N}
	\end{align*}
	as desired. 
\end{enumerate}


\noindent \textbf{3. }
\begin{enumerate}[label=(\alph*)]
	\item $\vec{A} = (-yB,0,0)$ leads to $\Pi_x = \hat{p}_x + \f{eB}{c}\hat{y}$ and $\Pi_y = \hat{p}_y$. The Hamiltonian is 
	\begin{align*}
	\ham = \f{p_z^2}{2m} + \f{p_y^2}{2m} + \f{1}{2m}\lp \hat{p}_x + \f{eB}{c}\hat{y} \rp^2 =  \f{p_z^2}{2m} + \f{p_y^2}{2m} + \f{m e^2 B^2}{2m^2c^2}\lp \f{c}{eB}\hat{p}_x + \hat{y} \rp^2
	\end{align*} 
	We notice that $\hat{p}_x$ commutes with this Hamiltonian, and so we may replace $p_x$ with $\hbar k_x$ to get
	\begin{align*}
	\ham =  \f{p_z^2}{2m} + \f{p_y^2}{2m} + \f{m e^2 B^2}{2m^2c^2}\lp \f{\hbar c k_x}{eB} + \hat{y} \rp^2
	\end{align*}
	The term $p_z^2/2m$ is ancillary, and so is $p_x^2/2m$ which does not explicitly appear in the Hamiltonian above since $k_x$, same as $k_z=k$, is a constant of motion. The full wavefunction is therefore
	\begin{align*}
	\Psi_{k,n}(x,y,z) = e^{i(k_x x + k z )} \phi_n\lp y + \f{\hbar c k_x}{eB} \rp
	\end{align*}
	where $\phi_n$ are the eigenstates of the QHO with frequency $\omega = eB/mc$, $n\in \mathbb{N}$. 
	
	
	\item $\vec{A} = (-yB/2,xB/2,0)$ leads to $\Pi_x = \hat{p}_x + \f{eB}{2c}\hat{y}$ and $\Pi_y = \hat{p}_y - \f{eB}{2c} \hat{x}$. The Hamiltonian becomes
	\begin{align*}
	\ham &= \f{p_z^2}{2m} + \f{1}{2m}\lb \lp \hat{p}_x + \f{eB}{2c}\hat{y} \rp^2 + \lp \hat{p}_y - \f{eB}{2c} \hat{x} \rp^2 \rb 
	\end{align*}
	Inspired by the approach on Wikipedia, let us ignore the $z$ part for now and go dimensionless (since they're a lot of factors flying around) so that
	\begin{align*}
	\ham = \f{1}{2}\lb \lp -i\p_x - \f{y}{2} \rp^2 + \lp -i\p_y + \f{x}{2} \rp^2 \rb
	\end{align*}
	
	Let us define two new operators:
	\begin{align*}
	a_{\pm} = \f{1}{\sqrt{2}}(a_x \pm i a_y)
	\end{align*}
%	Let us define 
%	\begin{align*}
%	w = x+iy \quad\quad\text{and}\quad\quad \bar{w} = x - iy.
%	\end{align*}
%	Correspondingly we can also define 
%	\begin{align*}
%	\p = \f{1}{2}(\p_x - i\p_y) \quad\quad \text{and}\quad\quad \bar\p = \f{1}{2}(\p_x + i\p_y)
%	\end{align*}
%	so that $\p w = \bar{\p}\bar{w} = 1$ and $\p \bar{w} = \bar{\p} w = 0$, i.e. we can treat $w,\bar{w}$ as independent variables. We may check that 
%	\begin{align*}
%	\ham = \f{p_z^2}{2m} - \f{2\hbar^2}{m}\p \bar{\p} - \f{eB\hbar }{2mc}(w\p - \bar{w}\bar{\p}) + \f{m e^2B^2}{8m^2c^2}w\bar{w}
%	\end{align*}
	where
	\begin{align*}
	a_x = \f{x}{2} + \p_x \quad\text{and}\quad a_y = \f{y}{2} + \p_y
	\end{align*}
	(which can be obtained by going dimensionless with the usual definition of ladder operators). From here, we can readily check that 
	\begin{align*}
	[a_+, a_+^\dagger] = [a_-,a_-^\dagger]  =1
	\end{align*}
	and that the Hamiltonian is in fact
	\begin{align*}
	\ham = a_-^\dagger a_- + \f{1}{2}.
	\end{align*}
	\begin{proof}
		\textbf{\textcolor{purple}{While it is possible, I won't check this because I'm already extremely low on time.}}
	\end{proof}
	In any case, from these two sets of ladder operators, we see that the eigenstates are specified by two quantum numbers $n_-, n_+$.
	\begin{align*}
	&a_-^\dagger\ket{n_-, n_+} = \sqrt{n_-+1} \ket{n_-+1,n_+} \\
	&a_-\ket{n_-, n_+} = \sqrt{n_-} \ket{n_--1,n_+} \\
	&a_+^\dagger\ket{n_-, n_+} = \sqrt{n_++1} \ket{n_-,n_++1} \\
	&a_+ \ket{n_-, n_+} = \sqrt{n_+} \ket{n_-,n_+-1}  
	\end{align*}
	The eigenstates are
	\begin{align*}
	\ket{n_-,n_+} = \f{(a_-^\dagger)^{n_-}}{\sqrt{n_-!}}\f{(a_+^\dagger)^{n_+}}{\sqrt{n_+!}}\ket{0,0}
	\end{align*}

	\textbf{\textcolor{blue}{Not sure what to do from here...}}
	
	
	
	\item $\vec{A} = (0,xB,0)$ leads to $\Pi_y = \hat{p}_y - \f{eB}{c}\hat{x}$ and $\Pi_x = \hat{p}_x$. The Hamiltonian is 
	\begin{align*}
	\ham = \f{p_z^2}{2m} + \f{p_x^2}{2m} + \f{1}{2m}\lp \hat{p}_y - \f{eB}{c}\hat{x} \rp^2 =  \f{p_z^2}{2m} + \f{p_y^2}{2m} + \f{m e^2 B^2}{2m^2c^2}\lp \f{c}{eB}\hat{p}_y - \hat{x} \rp^2
	\end{align*}
	We notice that $\hat{p}_y$ commutes with this Hamiltonian, and so we may replace $p_y$ with $\hbar k_y$ to get
	\begin{align*}
	\ham =  \f{p_z^2}{2m} + \f{p_x^2}{2m} + \f{m e^2 B^2}{2m^2c^2}\lp \f{\hbar c k_y}{eB} - \hat{x} \rp^2
	\end{align*}
	The term $p_z^2/2m$ is ancillary, and so is $p_y^2/2m$ which does not explicitly appear in the Hamiltonian above since $k_y$, same as $k_z=k$, is a constant of motion. The full wavefunction is therefore
	\begin{align*}
	\Psi_{k,n}(x,y,z) = e^{i(k_y y + k z )} \phi_n\lp x - \f{\hbar c k_y}{eB} \rp
	\end{align*}
	where $\phi_n$ are the eigenstates of the QHO with frequency $\omega = eB/mc$, $n\in \mathbb{N}$. 
	
	 
\end{enumerate}


\noindent \textbf{4. \textbf{\textcolor{purple}{Not sure if I can complete this problem because I'm running super low on time...}}}


Let us pick the vector potential $\vec{A} = (0,xB,0)$ from Part (c) to do this problem. We have $\vec{A}$ reproduces $\vec{B} = (0,0,B)$, as wanted. The electric field is given by $\vec{E} = (E,0,0)$, and so we may pick the associated scalar potential to be $\phi(x,y,z) =Ex$.  The Hamiltonian is therefore, 
\begin{align*}
\ham 
&=  \f{p_z^2}{2m} + \f{p_x^2}{2m} + \f{m e^2 B^2}{2m^2c^2}\lp \f{\hbar c k_y}{eB} - \hat{x} \rp^2 + eE\hat{x}\\
&= \f{p_z^2}{2m} + \f{p_x^2}{2m} + \f{m e^2 B^2}{2m^2c^2}\lp \hat{x} - \f{\hbar c k_y}{eB}  \rp^2 + eE\lp \hat{x}- \f{\hbar c k_y}{eB} \rp + eE\f{\hbar c k_y}{eB} \\
&= \f{p_z^2}{2m} + \f{p_x^2}{2m} + 
\f{m}{2}
\lb  \f{e^2 B^2}{m^2c^2}\lp \hat{x} - \f{\hbar c k_y}{eB}  \rp^2 + \f{2eE}{m}\lp \hat{x}- \f{\hbar c k_y}{eB} \rp + \f{2}{m}\f{E\hbar c k_y}{B} \rb \\
&= \f{p_z^2}{2m} + \f{p_x^2}{2m} + 
\f{m}{2}
\lb  \f{e^2 B^2}{m^2c^2}\lp \hat{x} - \f{\hbar c k_y}{eB}  \rp^2 + 2\f{{eB}}{mc}\lp \hat{x}- \f{\hbar c k_y}{eB} \rp \f{cE}{B} + \f{c^2E^2}{B^2} - \f{c^2E^2}{B^2} + \f{2}{m}\f{E\hbar c k_y}{B} \rb \\
&= \f{p_z^2}{2m} + \f{p_x^2}{2m} + 
\f{m}{2}
\lb  \f{e^2 B^2}{m^2c^2}\lp \hat{x} - \f{\hbar c k_y}{eB}  \rp^2 + 2\f{{eB}}{mc}\lp \hat{x}- \f{\hbar c k_y}{eB} \rp \f{cE}{B} + \f{c^2E^2}{B^2}\rb + \f{m}{2}\lb - \f{c^2E^2}{B^2} + \f{2}{m}\f{E\hbar c k_y}{B} \rb.
\end{align*}
At this point we may drop the last term because we can always redefine the scalar potential $\phi$ so that they (which are constants and do not contribute to the dynamics of the problem) vanish. We therefore get
\begin{align*}
\ham = \f{p_z^2}{2m} + \f{p_x^2}{2m} + 
\f{m}{2}
\lb  \f{{eB}}{mc}\lp \hat{x}- \f{\hbar c k_y}{eB} \rp  + \f{cE}{B}  \rb^2 = \f{p_z^2}{2m} + \f{p_x^2}{2m} + 
\f{m}{2}
\lb  \f{{eB}}{mc}\lp \hat{x}- \f{\hbar c k_y}{eB} + \f{c^2 mE}{eB^2} \rp   \rb^2
\end{align*}
The resulting eigenstates are thus
\begin{align*}
\Psi_{k,n}(x,y,z) = e^{i(k_y y + k_z z )} \phi_n\lp x - \f{\hbar c k_y}{eB}+ \f{c^2 mE}{eB^2}\rp
\end{align*}
where, as before, $\phi_n$ denotes the harmonic oscillator eigenstates which frequency $\omega_c = eB/mc$.\\















\noindent \textbf{5. } We can start by using known expressions for $Y^m_l$. At the end of this problem I will solve for $Y^m_l$ explicitly (using the eigenvalue equations) and from there check that they match with what we have here. 
\begin{align*}
Y_2^m(\theta,\phi) = \sqrt{\f{5}{4\pi} \f{(2-m)!}{(2+m)!}} P^m_2(\cos\theta)e^{im\phi}, \quad\quad m = -2,-1,0,1,2
\end{align*}
With 
\begin{align*}
P^m_2(\cos\theta) = \f{(-1)^m}{8}(1-\cos^2\theta)^{m/2}\f{d^{2+m}}{d\cos^{2+m}\theta}(\cos^2\theta - 1)^2
\end{align*}
we find 
\begin{align*}
P^{-2}_2(\cos\theta) &= \f{1}{8}\sin^2\theta\\
P^{-1}_2(\cos\theta) &= \f{1}{2}\cos\theta\sin\theta\\
P^{0}_2(\cos\theta) &= \f{1}{4}(1+3\cos2\theta)\\
P^{1}_2(\cos\theta) &= -3\cos\theta\sin\theta\\
P^{2}_2(\cos\theta) &= 3\sin^2\theta
\end{align*}
From here, we find that
\begin{align*}
Y^{-2}_2(\theta,\phi) &= \frac{1}{4} \sqrt{\frac{15}{2 \pi }} e^{-2 i \phi } \sin ^2\theta  
\implies Y^{-2}_2(x,y,z) =  \frac{1}{4} \sqrt{\frac{15}{2 \pi }} e^{-2 i \arctan(y/x) } (1-z^2)\\
Y^{-1}_2(\theta,\phi) &= \frac{1}{2} \sqrt{\frac{15}{2 \pi }} e^{-i \phi } \sin \theta  \cos \theta 
\implies Y^{-1}_2(x,y,z) = \frac{1}{2} \sqrt{\frac{15}{2 \pi }} e^{-i \arctan(y/x) } z\sqrt{1-z^2} \\
Y^{0}_2(\theta,\phi) &= \frac{1}{4} \sqrt{\frac{5}{\pi }} \left(3 \cos ^2\theta -1\right)
\implies Y^{0}_2(x,y,z) =   \frac{1}{4} \sqrt{\frac{5}{\pi }} \left(3 z^2 -1\right) \\
Y^{1}_2(\theta,\phi) &= -\frac{1}{2} \sqrt{\frac{15}{2 \pi }} e^{i \phi } \sin \theta  \cos \theta 
\implies  Y^{-1}_2(x,y,z) = -\frac{1}{2} \sqrt{\frac{15}{2 \pi }} e^{i \arctan(y/x) } z\sqrt{1-z^2}
\\
Y^{2}_2(\theta,\phi) &= \frac{1}{4} \sqrt{\frac{15}{2 \pi }} e^{2 i \phi } \sin ^2\theta 
\implies Y^{-2}_2(x,y,z) =  \frac{1}{4} \sqrt{\frac{15}{2 \pi }} e^{2 i \arctan(y/x) } (1-z^2)
\end{align*}
where we have used
\begin{align*}
\theta = \arccos z\quad\quad \text{and}\quad\quad \phi = \arctan(y/x)
\end{align*}
Finally,
\begin{align*}
\sum_m \abs{Y^m_2}^2 &= \f{1}{16} \f{15}{2\pi}(1-z^2)^2 + \f{1}{4}\f{15}{2\pi} z^2(1-z^2) + \f{1}{16}\f{5}{\pi} (3z^2-1)^2 + \f{1}{4}\f{15}{2\pi} z^2(1-z^2) + \f{1}{16} \f{15}{2\pi}(1-z^2)^2 \\
&= \boxed{\f{5}{4\pi}}
\end{align*}
as expected from Uns\"{o}ld's Theorem. Mathematica code:
\begin{lstlisting}
In[60]:= 2*(1/16)*(15/(2*Pi))*(1 - z^2)^2 + (1/2)*15/2/Pi*
z^2*(1 - z^2) + (1/16)*5/Pi*(3*z^2 - 1)^2 // FullSimplify

Out[60]= 5/(4 \[Pi])
\end{lstlisting}

Now, we justify our answers above for solving for $Y^m_2$ explicitly. To this end, we put $Y^m_2(\theta,\phi) = \Theta(\theta)\Phi(\phi)$. Using separation of variables, we have a system of equations:
\begin{align*}
\Phi'' = -m^2\Phi \quad\quad \text{and}\quad\quad 
\f{1}{\sin\theta}\f{d}{d\theta}\lp \sin\theta\f{d\Theta}{d\theta} \rp + \lb 2(2+1) - \f{m^2}{\sin^2\theta} \rb\Theta = 0.
\end{align*}
We choose $\Phi = e^{im\phi}$ where $m\in\mathbb{Z}$ to guarantee single-valuedness. For the $\Theta$ equation, we may change variables to $z = \cos\theta$, so that the differential equation for $\Theta$ reduces to 
\begin{align*}
(1-z^2)\f{d^2\Theta}{dz^2} -2z \f{d\Theta}{dz} + \lb 2(2+1) - \f{m^2}{1-z^2} \rb\Theta = 0
\end{align*}
Solving in Mathematica gives
\begin{align*}
\Theta_{m=0}(z) &= C_1(3z^2-1) \\
\Theta_{m=\pm 1}(z) &= \pm C_2 z\sqrt{1 - z^2} \\
\Theta_{m=\pm 2}(z) &= C_3(1-z^2).
\end{align*}
Plugging in $z=\cos\theta$ and normalizing we find the same solution as before:
\begin{align*}
Y^{-2}_2(\theta,\phi) &= \frac{1}{4} \sqrt{\frac{15}{2 \pi }} e^{-2 i \phi } \sin ^2\theta  \\
Y^{-1}_2(\theta,\phi) &= \frac{1}{2} \sqrt{\frac{15}{2 \pi }} e^{-i \phi } \sin \theta  \cos \theta \\
Y^{0}_2(\theta,\phi) &= \frac{1}{4} \sqrt{\frac{5}{\pi }} \left(3 \cos ^2\theta -1\right)
 \\
Y^{1}_2(\theta,\phi) &= -\frac{1}{2} \sqrt{\frac{15}{2 \pi }} e^{i \phi } \sin \theta  \cos \theta \\
Y^{2}_2(\theta,\phi) &= \frac{1}{4} \sqrt{\frac{15}{2 \pi }} e^{2 i \phi } \sin ^2\theta 
\end{align*}
where the normalization condition is 
\begin{align*}
\int^{2\pi}_0 \,d\phi \int^\pi_0 \,d\theta \abs{\Theta(\theta)\Phi(\phi)}^2\sin\theta = 1
\end{align*}
Mathematica code for solving ODE's and finding normalization constants:
\begin{lstlisting}
(*m= +/- 2*)
DSolve[(1 - x^2)*D[y[x], {x, 2}] - 
2*x*D[y[x], x] + (2*(2 + 1) - 4/(1 - x^2))*y[x] == 0, y[x], x]


(*m = +/- 1*)
DSolve[(1 - x^2)*D[y[x], {x, 2}] - 
2*x*D[y[x], x] + (2*(2 + 1) - 1/(1 - x^2))*y[x] == 0, y[x], x]

(*m = 0*)
DSolve[(1 - x^2)*D[y[x], {x, 2}] - 
2*x*D[y[x], x] + (2*(2 + 1) - 0/(1 - x^2))*y[x] == 0, y[x], x]

(*Normalization*)
(*m=0*)
In[92]:= 2*Pi*Integrate[((-1 + 3 Cos[t]^2))^2*Sin[t], {t, 0, Pi}]

Out[92]= (16 \[Pi])/5

(*m= +/- 1*)
In[93]:= 2*Pi*Integrate[(Sin[t]*Cos[t])^2*Sin[t], {t, 0, Pi}]

Out[93]= (8 \[Pi])/15

(*m = =/- 2*)
In[94]:= 2*Pi*Integrate[(Sin[t]^2)^2*Sin[t], {t, 0, Pi}]

Out[94]= (32 \[Pi])/15
\end{lstlisting}


 
\noindent \textbf{6. } By separation of variables, we write
\begin{align*}
\Psi(r,\theta,\phi) = R(r)\Theta(\theta)\Phi(\phi)
\end{align*}
The 3D Schr\"{o}dinger's equation in spherical coordinates reads
\begin{align*}
-\f{\hbar^2}{2m}\lp \f{1}{r^2}\p_r (r^2 \p_r (R\Theta\Phi)) + \f{1}{r^2\sin\theta}\p_\theta (\sin\theta \p_\theta(R\Theta\Phi)) + \f{1}{r^2\sin^2\theta}\p^2_\phi(R\Theta\Phi)  \rp + (V(r) - E)R\Theta\Phi = 0
\end{align*}
Let $R = u(r)/r$ and diving both sides of the equation by $R\Theta\Phi/r^2$, we find 
\begin{align*}
-\f{\hbar^2}{2m}\lb r^2\f{u''}{u} + \lp \f{1}{\sin^2}\f{\Phi''}{\theta \Phi} + \f{\cos\theta \Theta' + \Theta''}{\sin\theta} \rp \rb = r^2(E-V(r))
\end{align*}
Rearranging gives
\begin{align*}
\f{r^2u''}{u} + \f{2m}{\hbar^2}r^2(E-V(r)) = - \f{1}{\sin^2}\f{\Phi''}{\theta \Phi} - \f{\cos\theta \Theta' + \Theta''}{\sin\theta} = \lambda
\end{align*}
We may rewrite the angular equation as 
\begin{align*}
\f{1}{Y}\f{1}{\sin\theta}\p_\theta\lp \sin\theta \p_\theta Y \rp + \f{1}{Y}\f{1}{\sin^2\theta}\p_\phi^2 Y = -\lambda
\end{align*}
where $Y(\theta,\phi) = \Theta(\theta)\Phi(\phi)$. The solutions are of course the spherical harmonics $ Y = Y^m_l(\theta,\phi)$ and $\lambda= l(l+1)$. With this, we come back to the radial equation to find 
\begin{align*}
\f{r^2 u''}{u} + \f{2m(E-V(r))}{\hbar^2}r^2 = l(l+1) 
\end{align*}
Rearranging gives
\begin{align*}
\lb -\f{\hbar^2}{2m} \f{d^2}{dr^2} + \f{\hbar^2 l(l+1)}{2mr^2} + V(r)\rb u(r) = Eu(r)
\end{align*}
as desired. \\

\noindent \textbf{7. } The Hamiltonian is 
\begin{align*}
\ham = \f{p_x^2}{2m} + \f{p_y^2}{2m} + \f{1}{2}\omega^2 x^2 + \f{1}{2}\omega^2 y^2.
\end{align*}
It is clear that the Hamiltonian can be written as
\begin{align*}
\ham = \hbar \omega \lp  a_x^\dagger a_x + a^\dagger_y a_y + 1 \rp
\end{align*}
where, as usual, 
\begin{align*}
a_i &= \sqrt{\f{m\omega}{2\hbar}}\lp \hat{r}_i + \f{i}{m\omega}\hat{p}_i \rp\\
a^\dagger_i &= \sqrt{\f{m\omega}{2\hbar}}\lp \hat{r}_i - \f{i}{m\omega} \hat{p}_i\rp
\end{align*}
Since $n_i = a^\dagger_i a_i = 0,1,2,\dots$ we can see that for each energy level $\hbar\omega(n_x+n_y+1)$ has a degeneracy of $n_x+n_y+1$. Now we wish to write $J_z$ in terms of the creation and annihilation operators. This can be done by working from the definition:
\begin{align*}
J_z 
&= xp_y - yp_x \\
&= \f{i\hbar}{2} (a_x^\dagger + a_x)(a^\dagger_y - a_y) - \f{i\hbar}{2}(a^\dagger_y + a_y)(a^\dagger_x - a_x) \\
&= i\hbar (a_xa^\dagger_y - a^\dagger_x a_y)
\end{align*}
Let us introduce
\begin{align*}
a_{\pm} = \f{1}{\sqrt{2}}(a_x \pm i a_y)
\end{align*}
which satify nice properties:
\begin{align*}
[a_\pm , a_\pm^\dagger] = \f{1}{2}[a_x \pm ia_y, a_x^\dagger \mp ia_y^\dagger] = 1 .
\end{align*}
Moreover, we also have
\begin{align*}
a^\dagger_\pm a_\pm = \f{1}{2}(a^\dagger_x \mp ia^\dagger_y)(a_x \pm ia_y) = \f{1}{2}(a^\dagger_x a_x + a^\dagger_y a_y \pm ia^\dagger_x a_y \mp ia_xa_y^\dagger)
\end{align*}
from which we find
\begin{align*}
\ham = \hbar\omega\lp a^\dagger_xa_x + a^\dagger_y a_y + 1 \rp = \hbar \omega( a^\dagger_+ a_+ + a^\dagger_- a_- +1)
\end{align*}
and 
\begin{align*}
J_z = i\hbar (a_xa^\dagger_y - a^\dagger_x a_y) =  \hbar (a^\dagger_- a_- - a^\dagger_+a_+)
\end{align*}
If we define $N_\pm  = a^\dagger_\pm a_\pm$ then we have
\begin{align*}
\ham = \hbar \omega(N_+ + N_- +1) \quad\quad \text{and}\quad\quad J_z = \hbar(N_- - N_+).
\end{align*}
Using the exact same analysis we did with $a_x$ (or $a_y$ for that matter) where we say 
\begin{align*}
\ham a^\dagger_x \ket{E} = \dots = (E+\hbar\omega)(a^\dagger_x \ket{E}) 
\end{align*}
using the commutation relations for $a,a^\dagger$, we find that the spectra of $N_\pm$ are also nonnegative integers. Moreover, since $[N_+, N_-] = 0$, they are simultaneously diagonalizable. This means that specifying a pair of eigenvalues of $N_-,N_+$ uniquely determines the simultaneous eigenvector for $N_-, N_+$ and completely specifies energy eigenstate. This basically says that $\{N_-,N_+\}$ is a CSCO. $\{\ham, J_z\}$ is a CSCO follows from the fact that $\ham, J_z$ are essentially linearly independent linear combinations of $\{N_-,N_+\}$ (there's a subslety here since $\ham$ has an offset $\hbar \omega \mathbb{I}$, but since the identity operator commutes with everything we're okay).\\


Finally, suppose that the system has energy $\hbar \omega(n+1)$. We would like to know what the possible eigenvalues of $J_z$ are. Well, since $n = n_+ + n_-$, we have $(n+1)$ cases
\begin{align*}
&n_+ = n, n_- = 0 \implies n_- - n_+ = -n\\
&n_+ = n-1, n_- = 1 \implies n_- - n_+ = -(n+2)\\
&\vdots \\
&n_+ = 0, n_- = n \implies n_- - n_+ = n
\end{align*}
This implies that there are $(n+1)$ possible eigenvalues for $J_z$ whose values are
\begin{align*}
m = n_- - n_+ \in \{-n, -n+2, \dots, n-2, n\}
\end{align*}



\end{document}








