\documentclass{article}
\usepackage{physics}
\usepackage{graphicx}
\usepackage{caption}
\usepackage{amsmath}
\usepackage{bm}
\usepackage{framed}
\usepackage{authblk}
\usepackage{empheq}
\usepackage{amsfonts}
\usepackage{esint}
\usepackage[makeroom]{cancel}
\usepackage{dsfont}
\usepackage{centernot}
\usepackage{mathtools}
\usepackage{subcaption}
\usepackage{bigints}
\usepackage{amsthm}
\theoremstyle{definition}
\newtheorem{lemma}{Lemma}
\newtheorem{defn}{Definition}[section]
\newtheorem{prop}{Proposition}[section]
\newtheorem{rmk}{Remark}[section]
\newtheorem{thm}{Theorem}[section]
\newtheorem{exmp}{Example}[section]
\newtheorem{prob}{Problem}[section]
\newtheorem{sln}{Solution}[section]
\newtheorem*{prob*}{Problem}
\newtheorem{exer}{Exercise}[section]
\newtheorem*{exer*}{Exercise}
\newtheorem*{sln*}{Solution}
\usepackage{empheq}
\usepackage{tensor}
\usepackage{xcolor}
%\definecolor{colby}{rgb}{0.0, 0.0, 0.5}
\definecolor{MIT}{RGB}{163, 31, 52}
\usepackage[pdftex]{hyperref}
%\hypersetup{colorlinks,urlcolor=colby}
\hypersetup{colorlinks,linkcolor={MIT},citecolor={MIT},urlcolor={MIT}}  
\usepackage[left=1in,right=1in,top=1in,bottom=1in]{geometry}
\setcounter{MaxMatrixCols}{20}
\usepackage{newpxtext,newpxmath}
\newcommand*\widefbox[1]{\fbox{\hspace{2em}#1\hspace{2em}}}

\newcommand{\p}{\partial}
\newcommand{\R}{\mathbb{R}}
\newcommand{\C}{\mathbb{C}}
\newcommand{\lag}{\mathcal{L}}
\newcommand{\nn}{\nonumber}
\newcommand{\ham}{\mathcal{H}}
\newcommand{\M}{\mathcal{M}}
\newcommand{\I}{\mathcal{I}}
\newcommand{\K}{\mathcal{K}}
\newcommand{\F}{\mathcal{F}}
\newcommand{\w}{\omega}
\newcommand{\lam}{\lambda}
\newcommand{\al}{\alpha}
\newcommand{\be}{\beta}
\newcommand{\x}{\xi}

\newcommand{\G}{\mathcal{G}}

\newcommand{\f}[2]{\frac{#1}{#2}}

\newcommand{\ift}{\infty}

\newcommand{\lp}{\left(}
\newcommand{\rp}{\right)}

\newcommand{\lb}{\left[}
\newcommand{\rb}{\right]}

\newcommand{\lc}{\left\{}
\newcommand{\rc}{\right\}}


\newcommand{\V}{\mathbf{V}}
\newcommand{\U}{\mathcal{U}}
\newcommand{\Id}{\mathcal{I}}
\newcommand{\D}{\mathcal{D}}
\newcommand{\Z}{\mathcal{Z}}

%\setcounter{chapter}{-1}


\usepackage{enumitem}



\usepackage{listings}
\captionsetup[lstlisting]{margin=0cm,format=hang,font=small,format=plain,labelfont={bf,up},textfont={it}}
\renewcommand*{\lstlistingname}{Code \textcolor{violet}{\textsl{Mathematica}}}
\definecolor{gris245}{RGB}{245,245,245}
\definecolor{olive}{RGB}{50,140,50}
\definecolor{brun}{RGB}{175,100,80}

%\hypersetup{colorlinks,urlcolor=colby}
\lstset{
	tabsize=4,
	frame=single,
	language=mathematica,
	basicstyle=\scriptsize\ttfamily,
	keywordstyle=\color{black},
	backgroundcolor=\color{gris245},
	commentstyle=\color{gray},
	showstringspaces=false,
	emph={
		r1,
		r2,
		epsilon,epsilon_,
		Newton,Newton_
	},emphstyle={\color{olive}},
	emph={[2]
		L,
		CouleurCourbe,
		PotentielEffectif,
		IdCourbe,
		Courbe
	},emphstyle={[2]\color{blue}},
	emph={[3]r,r_,n,n_},emphstyle={[3]\color{magenta}}
}

\newcommand{\diag}{\text{diag}}
\newcommand{\psirot}{\ket{\psi_\text{rot}(t)} }
\newcommand{\RWA}{\ham_\text{rot}^\text{RWA}}


\begin{document}
\begin{framed}
\noindent Name: \textbf{Huan Q. Bui}\\
Course: \textbf{8.421 - AMO I}\\
Problem set: \textbf{\#6}\\
Due: Friday, March 18, 2022.
\end{framed}
	
\noindent \textbf{1. Symmetries and Permanent Dipole Moments}

\begin{enumerate}[label=(\alph*)]
	\item As we know, the energy eigenstates of the rotationally invariant Hamiltonian of HCl have no ``permanent'' electric dipole moment in the lab frame. This makes sense since the orientation of the molecule is arbitrary. More precisely, The energy eigenfunctions go like $\psi_{nlm} = R_{nl}Y_{lm}$ and have parity $(-1)^l$, which gives $\bra{Jm_J} \hat{d} \ket{JmJ}= 0$.  This means that the observed ``permanent electric dipole'' only exists in the presence of a bias electric field which breaks symmetry. The the presence of an electric field, states of difference parities connect and the molecule behaves like it has a permanent electric dipole.
	
	\item  The observed linear Stark effect comes from the fact that the HCl molecule has permanent electric dipole moment. This is allowed when $l$ is large, at which point there is a degeneracy of states with different parities. 
	
	\item Naively I would think that since the orbital magnetic moments of electrons and protons in atoms do not necessarily cancel, it is possible for an atom to have a permanent magnetic dipole moment. However, this cannot be the case for electric dipole moment because this would imply that there exists a ``separation of charges'' in the atoms, i.e., the ``centers'' of the negative and positive charge distributions do not coincide. But this implies that there is a restoring force present which is possible only if the atom is polarized. 
\end{enumerate}

\noindent \textbf{2. The Stark Effect in Hydrogen}


\begin{enumerate}[label=(\alph*)]
	\item To do this problem, we must first identity the relevant two-level system. The $n=2$ level of hydrogen has a total of 8 states when both electron charge and spin are included in the Hamiltonian. Fine structure splitting raises the four $2P_{3/2}$ states about 10 GHz above the two $2P_{1/2}$ and two $2S_{1/2}$ states. From here, plus the fact that the two levels must have different $l$'s, we can ignore the $2P_{3/2}$ states. The Lamb shift raises $2S_{1/2}$ above $2P_{1/2}$ by about 1 GHz. This is a small splitting so we may as well treat $2S_{1/2}$ and $2P_{1/2}$ as ``degenerate'' in order to do degenerate perturbation theory (we'll keep the energy splitting the diagonal terms though). Both $2S_{1/2}$ and $2P_{1/2}$ have total angular momentum $J=1/2$. The substates in each level can be assumed to be degenerate and treated as one state. 
	
	The Hamiltonian is thus 
	\begin{align*}
	\ham = \begin{pmatrix}
	E_0 - \Delta/2 & 3ea_0\mathcal{E} \\ 3ea_0\mathcal{E} & E_0 + \Delta/2
	\end{pmatrix}
	\end{align*}
	where we have picked the basis with $\ket{g} = \ket{2P_{1/2}} = (1,0)^\top$ and $\ket{e} = \ket{2S_{1/2}} = (0,1)^\top$. The eigenvalues are
	\begin{align*}
	E_\pm &= E_0 \pm \sqrt{\Delta^2/4 + (3ea_0\mathcal{E})^2}\\
	\ket{+} &= \\
	\ket{-} &= 
	\end{align*}
	Here we have chosen $\Delta$ to be the Lamb shift. 	Since it doesn't change the dynamics, we may also just set $E_0  = 0$. Working in units of frequency and writing the off-diagonal term as $V$ we have
	\begin{align*}
	\ham= \begin{pmatrix}
	-\omega_0/2 & V \\ V& \omega_0 /2
	\end{pmatrix}.
	\end{align*}
	The eigenvalues and eigenvectors are
	\begin{align*}
	\omega^\pm &= \pm \sqrt{(\omega_0/2)^2 + V^2}\\
	\ket{+} &= \sin \f{\theta}{2}\ket{g} + \cos\f{\theta}{2}\ket{e}\\
	\ket{-} &= -\cos\f{\theta}{2}\ket{g} + \sin\f{\theta}{2}\ket{e}
	\end{align*}
	where we have defined $\tan\theta = 2V/\omega_0$. 
	
	\item For the Stark shift to be linear, we must have that 
	\begin{align*}
	2V\approx \omega_0 \implies 2(3ea_0 \mathcal{E}) \approx \omega_0 \hbar \implies {\mathcal{E} \approx  \f{\hbar\omega_0}{6ea_0}}
	\end{align*}
	The numerical value of this quantity is 
	\begin{align*}
	\mathcal{E}_0 \approx 2.17 \text{ kV/m} = \boxed{21.77 \text{ V/cm}}
	\end{align*}
	In this case, the linear shift may be calculated by evaluating the limit of $d\Delta E/d\mathcal{E}$ as $\mathcal{E} \to \infty$. The value obtained is 
	\begin{align*}
	\f{d\Delta \omega}{d\mathcal{E}}\bigg\vert_{\mathcal{E}\to \infty} = \f{3 e a_0}{\hbar} =\f{\omega_0/2}{\mathcal{E}_0}.
	\end{align*}
	Here $\omega_0 = 1.06 \text{ GHz}$ is the Lamb shift. So the numerical value for this quantity is 
	\begin{align*}
	\f{d\Delta \omega}{d\mathcal{E}}\bigg\vert_{\mathcal{E}\to \infty} = \f{530}{21.77} \text{ MHz/(V/cm)} \approx \boxed{ 24.35 \text{ MHz/(V/cm)} }
	\end{align*}
	
	
	
\end{enumerate}

\noindent \textbf{Note to the grader:}  My apologies in advance for the sloppiness in presentation, explanation etc. in this pset. I have not had proper sleep since Sunday, March 12, 2022 thanks to my wonderfully considerate roommate and his associates. 

\end{document}








