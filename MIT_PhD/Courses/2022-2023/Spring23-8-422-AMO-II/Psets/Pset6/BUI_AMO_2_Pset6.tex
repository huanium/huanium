\documentclass{article}
\usepackage{physics}
\usepackage{graphicx}
\usepackage{caption}
\usepackage{amsmath}
\usepackage{bm}
\usepackage{framed}
\usepackage{authblk}
\usepackage{empheq}
\usepackage{amsfonts}
\usepackage{esint}
\usepackage[makeroom]{cancel}
\usepackage{dsfont}
\usepackage{centernot}
\usepackage{mathtools}
\usepackage{subcaption}
\usepackage{bigints}
\usepackage{amsthm}
\theoremstyle{definition}
\newtheorem{lemma}{Lemma}
\newtheorem{defn}{Definition}[section]
\newtheorem{prop}{Proposition}[section]
\newtheorem{rmk}{Remark}[section]
\newtheorem{thm}{Theorem}[section]
\newtheorem{exmp}{Example}[section]
\newtheorem{prob}{Problem}[section]
\newtheorem{sln}{Solution}[section]
\newtheorem*{prob*}{Problem}
\newtheorem{exer}{Exercise}[section]
\newtheorem*{exer*}{Exercise}
\newtheorem*{sln*}{Solution}
\usepackage{empheq}
\usepackage{tensor}
\usepackage{xcolor}
%\definecolor{colby}{rgb}{0.0, 0.0, 0.5}
\definecolor{MIT}{RGB}{163, 31, 52}
\usepackage[pdftex]{hyperref}
%\hypersetup{colorlinks,urlcolor=colby}
\hypersetup{colorlinks,linkcolor={MIT},citecolor={MIT},urlcolor={MIT}}  
\usepackage[left=1in,right=1in,top=1in,bottom=1in]{geometry}

\usepackage{newpxtext,newpxmath}
\newcommand*\widefbox[1]{\fbox{\hspace{2em}#1\hspace{2em}}}

\newcommand{\p}{\partial}
\newcommand{\R}{\mathbb{R}}
\newcommand{\C}{\mathbb{C}}
\newcommand{\lag}{\mathcal{L}}
\newcommand{\nn}{\nonumber}
\newcommand{\ham}{\mathcal{H}}
\newcommand{\M}{\mathcal{M}}
\newcommand{\I}{\mathcal{I}}
\newcommand{\K}{\mathcal{K}}
\newcommand{\F}{\mathcal{F}}
\newcommand{\w}{\omega}
\newcommand{\lam}{\lambda}
\newcommand{\al}{\alpha}
\newcommand{\be}{\beta}
\newcommand{\x}{\xi}

\newcommand{\G}{\mathcal{G}}

\newcommand{\f}[2]{\frac{#1}{#2}}

\newcommand{\ift}{\infty}

\newcommand{\lp}{\left(}
\newcommand{\rp}{\right)}

\newcommand{\lb}{\left[}
\newcommand{\rb}{\right]}

\newcommand{\lc}{\left\{}
\newcommand{\rc}{\right\}}


\newcommand{\V}{\mathbf{V}}
\newcommand{\U}{\mathcal{U}}
\newcommand{\Id}{\mathcal{I}}
\newcommand{\D}{\mathcal{D}}
\newcommand{\Z}{\mathcal{Z}}

%\setcounter{chapter}{-1}


\usepackage{enumitem}



\usepackage{listings}
\captionsetup[lstlisting]{margin=0cm,format=hang,font=small,format=plain,labelfont={bf,up},textfont={it}}
\renewcommand*{\lstlistingname}{Code \textcolor{violet}{\textsl{Mathematica}}}
\definecolor{gris245}{RGB}{245,245,245}
\definecolor{olive}{RGB}{50,140,50}
\definecolor{brun}{RGB}{175,100,80}

%\hypersetup{colorlinks,urlcolor=colby}
\lstset{
	tabsize=4,
	frame=single,
	language=mathematica,
	basicstyle=\scriptsize\ttfamily,
	keywordstyle=\color{black},
	backgroundcolor=\color{gris245},
	commentstyle=\color{gray},
	showstringspaces=false,
	emph={
		r1,
		r2,
		epsilon,epsilon_,
		Newton,Newton_
	},emphstyle={\color{olive}},
	emph={[2]
		L,
		CouleurCourbe,
		PotentielEffectif,
		IdCourbe,
		Courbe
	},emphstyle={[2]\color{blue}},
	emph={[3]r,r_,n,n_},emphstyle={[3]\color{magenta}}
}


\begin{document}
\begin{framed}
\noindent Name: \textbf{Huan Q. Bui}\\
Course: \textbf{8.422 - AMO II}\\
Problem set: \textbf{\#6}\\
Due: Friday, Mar 31, 2022\\
Collaborators:  
\end{framed}
	



\noindent \textbf{1. Rayleigh and Thomson scattering using two different interaction Hamiltonians.} Here we have that $\ket{i} = \ket{a, \mathbf{k\epsilon}}$ and $\ket{f} = \ket{a, \mathbf{k' \epsilon'}}$

\begin{enumerate}[label=(\alph*)]
\item Let us first calculate $\mathcal{T}_{fi}$ for Rayleigh and Thomson scattering for the electric-dipole Hamiltonian. \\


\noindent \underline{Rayleigh scattering, electric-dipole Hamiltonian}:  There are two possible processes with different intermediate states. One possibility is that the intermediate state is $\ket{b, 0}$ and the other is $\ket{b, \mathbf{k\epsilon, k'\epsilon'}}$. The interaction Hamiltonian is:
\begin{align*}
H_{I}' =  -  \mathbf{d}\cdot \mathbf{E_\perp(0)} =  i \mathcal{E}_{\omega} (a \mathbf{\epsilon} - a^\dagger \mathbf{\epsilon} ) = i \sqrt{\f{ \hbar \omega }{2\epsilon_0 V}}  (a - a^\dagger) \mathbf{d} \cdot  \mathbf{\epsilon}. 
\end{align*}
This Hamiltonian couples the initial and final states to the intermediate states of the process. We first calculate the relevant couplings between the initial state and the intermediate states:
\begin{align*}
\bra{b,0}  H_I' \ket{a, \mathbf{k\epsilon}} 
&= -i \sqrt{\f{\hbar \omega}{ 2\epsilon_0 V }} \bra{b} \mathbf{d}\cdot \mathbf{\epsilon} \ket{a} \\
\bra{b, \mathbf{k\epsilon , k'\epsilon'}} H_i' \ket{a, \mathbf{k\epsilon,0}} 
&= i \sqrt{\f{\hbar \omega}{2\epsilon_0 V}} \bra{b} \mathbf{d} \cdot \mathbf{\epsilon'} \ket{a}. 
\end{align*}
Similar results can be obtained the couplings between the intermediate states and the final state $\ket{f}$. \\

In the perturbative expansion for $\mathcal{T}_{fi}$, the first-order term $\bra{f} H_I' \ket{i}$ vanishes since the interaction Hamiltonian does not couple the initial and final states directly. As a result, the first (potentially) non-vanishing term is second-order. With this, let us calculate 
\begin{align*}
\mathcal{T}_{fi} 
= \bra{f}  H_I'   \f{1}{ E_i - H_0 + i\eta}  H_I'   \ket{i}  
= \f{\hbar \omega}{2\epsilon_0 V} \sum_b \lb   
\f{\bra{a} \mathbf{d} \cdot \mathbf{\epsilon'} \ket{b} \bra{b} \mathbf{d} \cdot \mathbf{\epsilon} \ket{a}}{ E_a + \hbar \omega - E_b} 
+ 
\f{ \bra{a} \mathbf{d} \cdot \mathbf{\epsilon}    \ket{b} \bra{b} \mathbf{d} \cdot \mathbf{\epsilon'}    \ket{a}  }{E_a - \hbar \omega - E_b}  \rb
\end{align*} 
Since $\hbar \omega \ll \abs{E_b - E_a}$ in Rayleigh scattering, we may write
\begin{align*}
\mathcal{T}_{fi} 
&= \f{\hbar \omega}{2\epsilon_0 V} \sum_b    
\f{\bra{a} \mathbf{d} \cdot \mathbf{\epsilon'} \ket{b} \bra{b} \mathbf{d} \cdot \mathbf{\epsilon} \ket{a} + 
\bra{a} \mathbf{d} \cdot \mathbf{\epsilon}    \ket{b} \bra{b} \mathbf{d} \cdot \mathbf{\epsilon'}    \ket{a} }{ E_a - E_b}.
\end{align*}




\noindent \underline{Thomson scattering, electric-dipole Hamiltonian}: The calculation here is similar to the one above. Let's rewrite the full result here first:
\begin{align*}
\mathcal{T}_{fi} 
= \bra{f}  H_I'   \f{1}{ E_i - H_0 + i\eta}  H_I'   \ket{i}  
= \f{\hbar \omega}{2\epsilon_0 V} \sum_b \lb   
\f{\bra{a} \mathbf{d} \cdot \mathbf{\epsilon'} \ket{b} \bra{b} \mathbf{d} \cdot \mathbf{\epsilon} \ket{a}}{ E_a + \hbar \omega - E_b} 
+ 
\f{ \bra{a} \mathbf{d} \cdot \mathbf{\epsilon}    \ket{b} \bra{b} \mathbf{d} \cdot \mathbf{\epsilon'}    \ket{a}  }{E_a - \hbar \omega - E_b}  \rb
\end{align*} 
In the limit $\hbar \omega \gg \abs{E_b - E_a}$, the two terms in the expression above cancel. As a result, we have to expand them to next order of $s = (E_b - E_a)/\hbar \omega$. 
\begin{align*}
\mathcal{T}_{fi} 
&= \f{1}{2\epsilon_0 V} \sum_b \lb   
\f{
\bra{a} \mathbf{d} \cdot \mathbf{\epsilon'} \ket{b} \bra{b} \mathbf{d} \cdot \mathbf{\epsilon} \ket{a}
}{ 1-s} 
+ 
\f{ 
\bra{a} \mathbf{d} \cdot \mathbf{\epsilon}    \ket{b} \bra{b} \mathbf{d} \cdot \mathbf{\epsilon'}    \ket{a} 
 }{-1-s}  \rb  \\
&\sim \f{1}{2\epsilon_0 V} \sum_b 
\f{E_b - E_a}{\hbar \omega} \lb  
\bra{a} \mathbf{d} \cdot \mathbf{\epsilon'} \ket{b} \bra{b} \mathbf{d} \cdot \mathbf{\epsilon} \ket{a} 
+ 
\bra{a} \mathbf{d} \cdot \mathbf{\epsilon}    \ket{b} \bra{b} \mathbf{d} \cdot \mathbf{\epsilon'}    \ket{a} 
\rb
\end{align*}
By replacing $(E_a - E_b) \bra{b} \mathbf{d} \cdot \mathbf{\epsilon} \ket{a}$ by $(i\hbar q/m) \bra{b} \mathbf{p}\cdot \mathbf{\epsilon} \ket{a}$, and then using the closure relation over $b$ we find that
\begin{align*}
\mathcal{T}_{fi} 
&= \f{1}{2\epsilon_0 V} \f{-\hbar^2 q^2}{m^2} \sum_b 
\f{E_b - E_a}{\hbar \omega} \lb  
\f{ 
\bra{a} \mathbf{p} \cdot \mathbf{\epsilon'} \ket{b} \bra{b} \mathbf{p} \cdot \mathbf{\epsilon} \ket{a} 
 }{(E_a - E_b)^2}  + 
 \f{
 \bra{a} \mathbf{p} \cdot \mathbf{\epsilon}    \ket{b} \bra{b} \mathbf{p} \cdot \mathbf{\epsilon'}    \ket{a} 
 }
 {(E_a - E_b)^2}
 \rb.
\end{align*}
How do we simplify this? First we replace $E_b - E_a$ by $E_I$ the ionization energy. Then we introduce the closure relation over $b$. After this, we get terms that go like $p^2$. Taking the $1/m$ from the prefactor we get $p^2/2m$ which we set to be equal to $E_I$, so all the $E_I$'s simplify and we're left with only terms involving $\mathbf{\epsilon}$ and $\mathbf{\epsilon'}$.  After all this is done, we get a factor of $2$ in the numerator, and the expression simplifies to 
\begin{align*}
\mathcal{T}_{fi} = 
\f{q^2}{2m} \f{\hbar}{\epsilon_0 V \omega} \sum_{m,n} \delta_{m,n} (\mathbf{e}_m \cdot \mathbf{\epsilon})(\mathbf{e}_n \cdot \mathbf{\epsilon'}) = 
\f{q^2}{2m} \f{\hbar}{\epsilon_0 V \omega} (\mathbf{\epsilon}\cdot \mathbf{\epsilon'}).
\end{align*}





\noindent \underline{Rayleigh scattering, Coulomb-gauge interaction Hamiltonian}:  The Coulomb-gauge interaction Hamiltonian is 
\begin{align*}
H_I =  H_{I1} + H_{I2}  = - \f{q}{m} \mathbf{p}\cdot \mathbf{A(0)} + \f{q^2}{ 2m } \mathbf{A^2(0)}.
\end{align*}
Let us first deal with $H_{I1}$. Since the possible intermediate states are $\ket{b,0}$ and $\ket{b,\mathbf{k\epsilon, k'\epsilon'}}$, the relevant couplings are
\begin{align*}
\bra{b,0} H_{I1} \ket{a, \mathbf{k\epsilon}} 
&= -\f{q}{m} \sqrt{\f{\hbar}{2\epsilon_0 V \omega}}  \bra{b}  \mathbf{p} \cdot \mathbf{\epsilon}   \ket{a} \\ 
\bra{b, \mathbf{k\epsilon,k'\epsilon'}} H_{I1} \ket{a, \mathbf{k\epsilon,0}}
&= -\f{q}{m} \sqrt{\f{\hbar}{2\epsilon_0 V \omega}} \bra{b}  \mathbf{p}\cdot \mathbf{\epsilon'}  \ket{a}.
 \end{align*}
Since $H_{I1}$ does not directly couple $\ket{i}$ and $\ket{f}$, only the second order term of $H_{I1}$ contributes. This means
 \begin{align*}
 \mathcal{T}_{fi}^{H_{I1}} = \f{q^2}{m^2} \f{\hbar }{2\epsilon_0 V \omega} 
 \sum_b
 \lb 
 \f{\bra{a} \mathbf{p}\cdot \mathbf{\epsilon'}  \ket{b} \bra{b}  \mathbf{p}\cdot \mathbf{\epsilon} \ket{a}}{E_a + \hbar \omega - E_b}
 +
 \f{\bra{a}  \mathbf{p}\cdot \mathbf{\epsilon} \ket{b}\bra{b}  \mathbf{p}\cdot \mathbf{\epsilon'}  \ket{a}}{E_a - \hbar \omega - E_b}
 \rb.
 \end{align*}
Next we look at $H_{I2}$:
\begin{align*}
H_{I2} = \f{q^2}{2m} \mathbf{A^2(0)} = \f{q^2}{2m} \f{\hbar}{2\epsilon_0 V \omega} \sum_{\mathbf{k,\epsilon}}\sum_{\mathbf{k',\epsilon'}} 
(\mathbf{\epsilon}\cdot \mathbf{\epsilon'}) 
(a_{\mathbf{k\epsilon}} + a^\dagger_{\mathbf{k\epsilon}})
(a_{\mathbf{k'\epsilon'}} + a^\dagger_{\mathbf{k'\epsilon'}}).
\end{align*}
Since $H_{I2}$ contains terms like $a_\mathbf{k\epsilon} a^\dagger_{\mathbf{k'\epsilon'}}, a^\dagger_{\mathbf{k\epsilon}} a_\mathbf{k'\epsilon'}$,  it couples the initial and final states directly. This means it suffices to calculate the first order term in the expansion for $\mathcal{T}_{fi}^{H_{I2}}$:
\begin{align*}
\mathcal{T}_{fi}^{H_{I2}} 
&= \f{q^2}{2m} \f{\hbar }{\epsilon_0 V \omega}(\mathbf{\epsilon} \cdot \mathbf{\epsilon'}).
\end{align*}
With these, we have
\begin{align*}
\mathcal{T}_{fi} = \mathcal{T}_{fi}^{H_{I1}}  + \mathcal{T}_{fi}^{H_{I2}}.
\end{align*}
Let's look at $\mathcal{T}_{fi}^{H_{I1}}$. For Rayleigh scattering, $\hbar \omega \ll \abs{E_a - E_b}$. If we simply ignore the $\hbar \omega$ in the denominators of $\mathcal{T}_{fi}^{H_{I1}}$ then the result will be canceled out by $\mathcal{T}_{fi}^{H_{I2}}$. Here's why: \textcolor{black}{Suppose we make the naive simplification by setting $\hbar \omega \to 0$ in the denominators, then we find that
\begin{align*}
 \mathcal{T}_{fi}^{H_{I1}} = \f{q^2}{m^2} \f{\hbar }{2\epsilon_0 V \omega} 
 \sum_b
 \lb 
 \f{\bra{a} \mathbf{p}\cdot \mathbf{\epsilon'}  \ket{b} \bra{b}  \mathbf{p}\cdot \mathbf{\epsilon} \ket{a}}{E_a  - E_b}
 +
 \f{\bra{a}  \mathbf{p}\cdot \mathbf{\epsilon} \ket{b}\bra{b}  \mathbf{p}\cdot \mathbf{\epsilon'}  \ket{a}}{E_a - E_b}
 \rb.
 \end{align*}
 By replacing $\bra{b} \mathbf{p}\cdot \mathbf{\epsilon} \ket{a}$ by $(m/i\hbar) (E_b - E_a) \bra{b}  \mathbf{r}\cdot \mathbf{\epsilon}  \ket{a} $ and invoking the closure relation over $b$, we find 
\begin{align*}
 \mathcal{T}_{fi}^{H_{I1}} = -\f{q^2 m^2}{\hbar^2 m^2} \f{\hbar }{2\epsilon_0 V \omega} 
\sum_b (E_b - E_a)
 \lb 
 \bra{a} \mathbf{r}\cdot \mathbf{\epsilon'}  \ket{b} \bra{b}  \mathbf{r}\cdot \mathbf{\epsilon} \ket{a}
 +
\bra{a}  \mathbf{r}\cdot \mathbf{\epsilon} \ket{b}\bra{b}  \mathbf{r}\cdot \mathbf{\epsilon'}  \ket{a} \rb.
\end{align*}
We immediately recognize that the summation can be written in terms of the oscillator strengths, which sum to unity:
\begin{align*}
 \mathcal{T}_{fi}^{H_{I1}} = -\f{q^2 m^2}{\hbar^2 m^2} \f{\hbar }{2\epsilon_0 V \omega} 
\sum_b  2 f_{ab} \f{\hbar^2}{2m} (\mathbf{\epsilon}\cdot \mathbf{\epsilon'}) =  -\f{q^2}{2m} \f{\hbar}{\epsilon_0 V \omega} (\mathbf{\epsilon}\cdot \mathbf{\epsilon'}) 
\end{align*}
which cancels $\mathcal{T}_{fi}^{H_{I2}}$. 
}

As a result, we can't just ignore $\hbar \omega$. Rather, we have to expand the terms in $\mathcal{T}_{fi}^{H_{I1}}$ in $s = \hbar \omega / (E_b - E_a)$:
\begin{align*}
\mathcal{T}_{fi}^{H_{I1}}
&= 
\sum_b \f{1}{E_b - E_a}
 \lb 
 \bra{a} \mathbf{p}\cdot \mathbf{\epsilon'}  \ket{b} \bra{b}  \mathbf{p}\cdot \mathbf{\epsilon} \ket{a} \f{1}{s-1}
 +
 \bra{a}  \mathbf{p}\cdot \mathbf{\epsilon} \ket{b}\bra{b}  \mathbf{p}\cdot \mathbf{\epsilon'}  \ket{a}\f{1}{-s-1}
 \rb \\
 &= \sum_b \f{1}{E_b - E_a}
 \lb 
 \bra{a} \mathbf{p}\cdot \mathbf{\epsilon'}  \ket{b} \bra{b}  \mathbf{p}\cdot \mathbf{\epsilon} \ket{a} (-1-s-s^2)
 +
 \bra{a}  \mathbf{p}\cdot \mathbf{\epsilon} \ket{b}\bra{b}  \mathbf{p}\cdot \mathbf{\epsilon'}  \ket{a} (-1+s-s^2)
 \rb \\
\end{align*}
The zeroth order terms in the expansion of the denominators are cancelled by the $\mathcal{T}_{fi}^{H_{I2}}$ for the same reason why we have to perform this expansion in the first place. The first order terms in the expansion cancel each other, so we're left with only the second order terms. By replacing $\bra{b} \mathbf{p}\cdot \mathbf{\epsilon} \ket{a}$ by $(m/i\hbar) (E_b - E_a) \bra{b}  \mathbf{r}\cdot \mathbf{\epsilon}  \ket{a} $ and invoking the closure relation over $b$, we find 
\begin{align*}
\mathcal{T}_{fi} 
&= 
\f{q^2 m^2}{\hbar^2 m^2} \f{\hbar }{2\epsilon_0 V \omega} (\hbar \omega)^2
\sum_b \f{1}{E_a - E_b}
\lb 
 \bra{a} \mathbf{r}\cdot \mathbf{\epsilon'}  \ket{b} \bra{b}  \mathbf{r}\cdot \mathbf{\epsilon} \ket{a} 
 +
  \bra{a}  \mathbf{r}\cdot \mathbf{\epsilon} \ket{b}\bra{b}  \mathbf{r}\cdot \mathbf{\epsilon'}  \ket{a} 
\rb  \\
&= \f{\hbar \omega}{2\epsilon_0 V} \sum_b 
\f{
 \bra{a} \mathbf{d}\cdot \mathbf{\epsilon'}  \ket{b} \bra{b}  \mathbf{d}\cdot \mathbf{\epsilon} \ket{a} 
+
 \bra{a}  \mathbf{d}\cdot \mathbf{\epsilon} \ket{b}\bra{b}  \mathbf{d}\cdot \mathbf{\epsilon'}  \ket{a} 
}
{E_b - E_a},
\end{align*}
which is what we found before. 


\noindent \underline{Thomson scattering, Coulomb-gauge interaction Hamiltonian}:  The first few steps of the calculation is the same as above. We begin to differ when the comparison between $\hbar \omega$ and $\abs{E_b - E_a}$ comes in. For Thomson scattering, $\hbar \omega \gg (E_b - E_a)$, so we may ignore the contribution due to $\mathcal{T}_{fi}^{H_{I1}}$ since the two terms in its expression cancel each other out. If we're being careful, we will have to expand $\mathcal{T}_{fi}^{H_{I1}}$ to next order in $s = (E_b - E_a)/\hbar \omega$ to check that we can actually do this:
\begin{align*}
\mathcal{T}_{fi}^{H_{I1}}
&=
\f{q^2}{m^2} \f{\hbar }{2\epsilon_0 V \omega} \f{1}{\hbar\omega}
 \sum_b
 \lb 
 \f{\bra{a} \mathbf{p}\cdot \mathbf{\epsilon'}  \ket{b} \bra{b}  \mathbf{p}\cdot \mathbf{\epsilon} \ket{a}}{1-s}
 +
 \f{\bra{a}  \mathbf{p}\cdot \mathbf{\epsilon} \ket{b}\bra{b}  \mathbf{p}\cdot \mathbf{\epsilon'}  \ket{a}}{-1-s}
 \rb \\
 &= \f{q^2}{m^2} \f{\hbar }{2\epsilon_0 V \omega} \f{1}{(\hbar\omega)^2}
\sum_b 
(E_b - E_a)
\lb
 \bra{a} \mathbf{p}\cdot \mathbf{\epsilon'}  \ket{b} \bra{b}  \mathbf{p}\cdot \mathbf{\epsilon} \ket{a}
 + 
 \bra{a}  \mathbf{p}\cdot \mathbf{\epsilon} \ket{b}\bra{b}  \mathbf{p}\cdot \mathbf{\epsilon'}  \ket{a}
\rb
\end{align*}
Since $E_b - E_a \sim E_I$ the ionization energy, we can bring $E_b - E_a$ out of the numerator and invoke the closure relation over the intermediate states $b$. The remaining terms look like $p^2/2m$ which we set to the ionization energy as well. In the end, we find that
\begin{align*}
\mathcal{T}_{fi}^{H_{I1}} \sim \mathcal{T}_{fi}^{H_{I2}} \lp \f{E_I}{\hbar \omega} \rp^2. 
\end{align*}
Given that $\hbar \omega \gg E_I$, we see that $\mathcal{T}_{fi}^{H_{I2}} \gg \mathcal{T}_{fi}^{H_{I1}}$, so we can rightfully ignore the latter. With this result we can confidently say that
\begin{align*}
\mathcal{T}_{fi} = \mathcal{T}_{fi}^{H_{I2}} = 
 \f{q^2}{2m} \f{\hbar }{\epsilon_0 V \omega}(\mathbf{\epsilon} \cdot \mathbf{\epsilon'}).
\end{align*}



\item How do we calculate the total cross section? First we calculate the transition probability per unit time per unit solid angle $\delta W_{fi} / \delta \Omega$ via Fermi's Golden rule. Then we divide this quantity by the photon flux to obtain the differential cross section. The total cross section is then obtained by integrating over the solid angles. For Thomson scattering: 
\begin{align*} 
\f{\delta W_{fi}}{\delta \Omega} = \f{2\pi}{\hbar }  \abs{\mathcal{T}_{fi}}^2 \rho(E' = E) 
= \f{2\pi}{\hbar}
\lp \f{q^2}{2m} \f{\hbar }{\epsilon_0 V \omega}(\mathbf{\epsilon} \cdot \mathbf{\epsilon'})\rp^2 
 \f{V}{8\pi^3} \f{(\hbar c k)^2}{\hbar^3 c^3} = 
 \f{q^4}{(4\pi\epsilon_0)^2m^2 c^3 V} (\mathbf{\epsilon}\cdot \mathbf{\epsilon'})^2.
\end{align*}
Dividing this by the photon flux $c/V$, we obtain the differential cross section:
\begin{align*}
\f{d \sigma}{ d \Omega} = 
 \f{q^4}{(4\pi\epsilon_0)^2m^2 c^4} (\mathbf{\epsilon}\cdot \mathbf{\epsilon'})^2 = r_0^2  (\mathbf{\epsilon}\cdot \mathbf{\epsilon'})^2 
\end{align*}
where $r_0$ is the classical electron radius. To get the total cross section, let us pick $\mathbf{\epsilon} = \hat{z}$. We now need to sum over $\mathbf{\epsilon'} \perp \mathbf{k'}$ and do an angular average. Using Equation (55) in Complement A1 of API, we find 
\begin{align*}
\sum_{\mathbf{\epsilon'}\perp \mathbf{k'}} \abs{ \hat{z}\cdot \mathbf{\epsilon'} }^2
&= \mathbf{\epsilon'} \cdot \mathbf{\epsilon'}^* - \f{(  \mathbf{k'}\cdot \hat{z} )( \mathbf{k'}\cdot \hat{z}^* )}{k'^2} = 1-\cos^2\theta.
\end{align*}
So, the total cross section for Thomson scattering is:
\begin{align*}
\sigma = r_0^2 \int_0^{2\pi}\,d\phi \int_0^\pi \,d\theta  (1-\cos^2 \theta)\sin\theta  = \f{8\pi}{3}r_0^2. 
\end{align*}


\end{enumerate}


%%%%%%%%%


\noindent \textbf{2. Long-range (Van der Waals) interaction between ground-state atoms.} The electrostatic interaction between atoms $a$ and $b$ is described to first order by the dipole-dipole term:
\begin{align*}
H_\text{el}(R) = \f{ \mathbf{d}_a \cdot \mathbf{d}_b - 3 (\mathbf{d}_a \cdot \hat{\mathbf{R}})(\mathbf{d}_b \cdot \hat{\mathbf{R}})  }{R^3}.
\end{align*}
Here $\mathbf{R} = \mathbf{R}_{nb} - \mathbf{R}_{na}$ is a position vector pointing from the nuclei of $a$ to the nuclei of $b$. We will use time-independent perturbation theory to calculate the effect of $H_\text{el}$. 

\begin{enumerate}[label=(\alph*)]

\item Since there is no average dipole moment on either atom due to spherical symmetry, the first non-vanishing term in the series for the perturbed ground state energy of the system is given by second-order perturbation theory:
\begin{align*}
\Delta E = \sum_{i_a i_b \neq g_a g_b} \f{  \bra{g_a g_b}  H_\text{el} \ket{i_a i_b } \bra{i_a i_b}  H_\text{el} \ket{g_a g_b} }{E_{ga} + E_{gb}  - E_{ia} - E_{ib}}.
\end{align*}

\item Here we express our answer to Part (a) in terms of oscillator strengths. To start, let us put our atoms in Cartesian coordinates. Let $\mathbf{r}_A = (x_A, y_A, z_A)$ and $\mathbf{r}_B = (x_B, y_B, z_B)$ and $R = (X,Y,Z)$ where we assume atom $A$ to be at the origin. The dipole-dipole Hamiltonian thus becomes
\begin{align*}
H_\text{el}  = e^2 \lb \f{x_A x_B + y_A y_B + z_A z_B}{R^3} - 3\f{ (x_AX + y_A Y + z_A Z)(x_B X + y_B Y + z_B Z) }{R^5} \rb
\end{align*}
Let us take $\vec{R}$ to be in the $z$-axis, then $X=Y=0$, and $R = Z$. So we have
\begin{align*}
H_\text{el} =  e^2 \f{x_a x_b + y_a y_b- 2z_a z_b}{R^3}.
\end{align*}
With this, we can evaluate the ground state energy shift:
\begin{align*}
\Delta E 
&= \sum_{i_a i_b \neq g_a g_b} \f{  \bra{g_a g_b}  H_\text{el} \ket{i_a i_b } \bra{i_a i_b}  H_\text{el} \ket{g_a g_b} }{E_{ga} + E_{gb}  - E_{ia} - E_{ib}} \\
&= \f{e^2}{R^6} \sum_{i_ai_b \neq g_a b_b} 
\f{ 
 \bra{g_a g_b}  x_a x_b + y_a y_b - 2 z_a z_b   \ket{i_a i_b}   
 \bra{i_a i_b}  x_a x_b + y_a y_b - 2 z_a z_b  \ket{g_a g_b}   
}{E_{ga} + E_{gb}  - E_{ia} - E_{ib}}.
\end{align*}
How do we express this in terms of oscillator strengths? To do this, let us consider the case where the ground state wavefunction is of the orbital $s$ which is spherically symmetric. By this symmetry, the $x$ and $y$ terms drop out (can check this explicitly using the wavefunctions of Hydrogen), and we find 
\begin{align*}
\Delta E = \f{4e^2}{R^6} \sum_{i_ai_b \neq g_a g_b} \f{  \abs{\bra{g_a} z_a \ket{i_a}}^2  \abs{\bra{g_b} z_b \ket{i_b}}^2 }{E_{ga} + E_{gb} - E_{ia} - E_{ib}}.
\end{align*}
Alternatively, we could also say that the only relevant direction for photon exchange is along $z$, so we can simply drop the terms involving $x,y$. In any case, in terms of the oscillator strengths $f_{ig} = 2 m \omega_{ig} \abs{ \bra{i} z \ket{g}}^2 / \hbar $, we have
\begin{align*}
\Delta E 
&= \f{4 \hbar^2  e^2 }{4 R^6 m^2 } \sum_{i_a i_b \neq g_a g_b} 
\f{1}{E_{ga} + E_{gb} - E_{ia} - E_{ib}} \f{ f_{i_a g_a} f_{i_bg_b}}{  \omega_{i_a g_a} \omega_{i_b g_b}} \\
&= \f{\hbar^4 e^2 }{R^6 m^2} \sum_{i_a i_b \neq g_a g_b} 
\f{1}{ (E_{ga} - E_{ia}) + (E_{gb} - E_{ib})} \f{ f_{i_a g_a} f_{i_bg_b}}{  (E_{ga} - E_{ia}) (E_{gb} - E_{ib})}.
\end{align*}


Before proceeding, we shall look at how $E_{ia}, E_{ib}$ compare to the ground state energies. For Hydrogen, for instance, $E_g \sim \text{Ryd}$, while $E_i \sim \text{Ryd}/n^2$ which is typically smaller than $E_g$ for states with significant matrix elements. For $E_i \sim E_g$ or larger, the matrix elements are smaller. As a result, we can simply set the denominator a bit to get:
\begin{align*}
\Delta E = \f{\hbar^4 e^2 }{R^6 m^2} \f{1}{E_{ga} + E_{gb}} 
\sum_{i_a i_b \neq g_a g_b} 
\f{ f_{i_a g_a} f_{i_bg_b}}{  (E_{ga} - E_{ia}) (E_{gb} - E_{ib})}.
\end{align*}



\item We can estimate $C_6$ using the approximation that the oscillator strength $f_{ig}$ is large for only one transition $\ket{g} \to \ket{i}$. The $\ket{nS} \to \ket{(n+1)P}$ transitions in alkali atoms are the classic examples, with $f \approx 0.98$. In terms of the static polarizability:
\begin{align*}
\al_g = 2e^2 \sum_i \f{\abs{ \bra{i}z \ket{g} }^2}{E_i - E_g},
\end{align*}
we write, under the approximation $E_{ia}, E_{ib}$ small relative to $E_{ga}, E_{gb}$:
\begin{align*}
\Delta E 
= 
\f{ 4e^2 }{R^6} \f{1}{E_{ga} + E_{gb}} 
\sum_{ia} \abs{ \bra{g_a} z_a \ket{i_a}}^2   \sum_{ib} \abs{ \bra{g_b} z_b \ket{i_b}}^2 
=  \f{1}{e^2} \f{E_{ga}E_{gb}}{E_{ga} + E_{gb}} 
\f{\al_g^{(a)} \al_g^{(b)}}{R^6}.
\end{align*}
We note that the ground state energies are negative (which could also be defined as \textit{minus} the ionization energy, which is positive). From here, we see that the Van der Waals interaction \textit{lowers} energy of the system. 



\end{enumerate}


%%%%%%%%%

\noindent \textbf{3. Long-range interaction between an excited atom and a ground-state atom.} In this problem we consider the case where one atom is in the excited state and one is in the ground state. For simplicity we model each atom as a two level system with one ground state and one excited state. 

\begin{enumerate}[label=(\alph*)]


\item 




\item 




\item



\end{enumerate}


\end{document}








