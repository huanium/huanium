\documentclass{article}
\usepackage{physics}
\usepackage{graphicx}
\usepackage{caption}
\usepackage{amsmath}
\usepackage{bm}
\usepackage{framed}
\usepackage{authblk}
\usepackage{empheq}
\usepackage{amsfonts}
\usepackage{esint}
\usepackage[makeroom]{cancel}
\usepackage{dsfont}
\usepackage{centernot}
\usepackage{mathtools}
\usepackage{subcaption}
\usepackage{bigints}
\usepackage{amsthm}
\theoremstyle{definition}
\newtheorem{lemma}{Lemma}
\newtheorem{defn}{Definition}[section]
\newtheorem{prop}{Proposition}[section]
\newtheorem{rmk}{Remark}[section]
\newtheorem{thm}{Theorem}[section]
\newtheorem{exmp}{Example}[section]
\newtheorem{prob}{Problem}[section]
\newtheorem{sln}{Solution}[section]
\newtheorem*{prob*}{Problem}
\newtheorem{exer}{Exercise}[section]
\newtheorem*{exer*}{Exercise}
\newtheorem*{sln*}{Solution}
\usepackage{empheq}
\usepackage{tensor}
\usepackage{xcolor}
%\definecolor{colby}{rgb}{0.0, 0.0, 0.5}
\definecolor{MIT}{RGB}{163, 31, 52}
\usepackage[pdftex]{hyperref}
%\hypersetup{colorlinks,urlcolor=colby}
\hypersetup{colorlinks,linkcolor={MIT},citecolor={MIT},urlcolor={MIT}}  
\usepackage[left=1in,right=1in,top=1in,bottom=1in]{geometry}

\usepackage{newpxtext,newpxmath}
\newcommand*\widefbox[1]{\fbox{\hspace{2em}#1\hspace{2em}}}

\newcommand{\p}{\partial}
\newcommand{\R}{\mathbb{R}}
\newcommand{\C}{\mathbb{C}}
\newcommand{\lag}{\mathcal{L}}
\newcommand{\nn}{\nonumber}
\newcommand{\ham}{\mathcal{H}}
\newcommand{\M}{\mathcal{M}}
\newcommand{\I}{\mathcal{I}}
\newcommand{\K}{\mathcal{K}}
\newcommand{\F}{\mathcal{F}}
\newcommand{\w}{\omega}
\newcommand{\lam}{\lambda}
\newcommand{\al}{\alpha}
\newcommand{\be}{\beta}
\newcommand{\x}{\xi}

\newcommand{\G}{\mathcal{G}}

\newcommand{\f}[2]{\frac{#1}{#2}}

\newcommand{\ift}{\infty}

\newcommand{\lp}{\left(}
\newcommand{\rp}{\right)}

\newcommand{\lb}{\left[}
\newcommand{\rb}{\right]}

\newcommand{\lc}{\left\{}
\newcommand{\rc}{\right\}}


\newcommand{\V}{\mathbf{V}}
\newcommand{\U}{\mathcal{U}}
\newcommand{\Id}{\mathcal{I}}
\newcommand{\D}{\mathcal{D}}
\newcommand{\Z}{\mathcal{Z}}

%\setcounter{chapter}{-1}


\usepackage{enumitem}



\usepackage{listings}
\captionsetup[lstlisting]{margin=0cm,format=hang,font=small,format=plain,labelfont={bf,up},textfont={it}}
\renewcommand*{\lstlistingname}{Code \textcolor{violet}{\textsl{Mathematica}}}
\definecolor{gris245}{RGB}{245,245,245}
\definecolor{olive}{RGB}{50,140,50}
\definecolor{brun}{RGB}{175,100,80}

%\hypersetup{colorlinks,urlcolor=colby}
\lstset{
	tabsize=4,
	frame=single,
	language=mathematica,
	basicstyle=\scriptsize\ttfamily,
	keywordstyle=\color{black},
	backgroundcolor=\color{gris245},
	commentstyle=\color{gray},
	showstringspaces=false,
	emph={
		r1,
		r2,
		epsilon,epsilon_,
		Newton,Newton_
	},emphstyle={\color{olive}},
	emph={[2]
		L,
		CouleurCourbe,
		PotentielEffectif,
		IdCourbe,
		Courbe
	},emphstyle={[2]\color{blue}},
	emph={[3]r,r_,n,n_},emphstyle={[3]\color{magenta}}
}






\begin{document}
\begin{framed}
\noindent Name: \textbf{Huan Q. Bui}\\
Course: \textbf{8.422 - AMO II}\\
Problem set: \textbf{\#3}\\
Due: Friday, Mar 3, 2022.
\end{framed}
	
	
\noindent \textbf{1. Classical Coherence of Light.} \\

\noindent Consider a classical light field. The classical expressions for first-order and second-order coherence are:
\begin{align*}
g^{(1)}(\tau) &= \f{\langle E^*(t) E (t+\tau)\rangle}{\langle E^*(t) E(t) \rangle} \\
g^{(2)}(\tau) &= \f{\langle I(t) I (t+\tau) \rangle }{\langle I(t) \rangle^2} 
= \f{\langle  E^*(t) E^*(t+\tau) E(t+\tau) E(t)   \rangle }{\langle E^*(t) E(t) \rangle^2}
\end{align*}
where $\langle \cdot \rangle$ denotes a statistical averaging over many measurements, which can be interpreted as a time average. Moreover, $\langle f^*(t) g(t)\rangle$ defines a scalar product. 

\begin{enumerate}[label=\alph*)]

\item Here we show that $\vert g^{(1)}(\tau) \vert \leq 1$. In view of Cauchy's inequality, and the fact that $\langle E^*(t+\tau) E(t+\tau)\rangle = \langle E^*(t) E(t) \rangle $
\begin{align*}
\vert g^{(1)}(\tau)\vert^2 = \f{\vert \langle E^*(t) E (t+\tau)\rangle \vert^2}{\vert \langle E^*(t) E(t) \rangle\vert^2} 
\leq \f{ \langle E^*(t) E(t) \rangle \langle E^*(t+\tau) E(t+\tau) \rangle }{\vert \langle E^*(t) E(t) \rangle \vert^2}
= \f{ \langle E^*(t) E(t) \rangle \langle E^*(t) E(t) \rangle }{\vert \langle E^*(t) E(t) \rangle \vert^2} = 1.
\end{align*}
So, $\vert g^{(1)}(\tau)  \vert \leq 1$ as desired. 


\item Here we show that for zero time-delay, the second-order coherence obeys $g^{(2)}(0) \geq 1$. To this end, we notice that
\begin{align*}
0 \leq \langle [I(t) - \langle I(t) \rangle]^2 \rangle = \langle I(t)^2 - 2I(t) \langle I(t) \rangle + \langle I(t) \rangle^2 \rangle = \langle I(t)^2 \rangle -2 \langle I(t)\rangle^2 + \langle I(t) \rangle^2 = \langle I(t)^2 \rangle - \langle I(t)\rangle^2. 
\end{align*}

Thus, for $\tau=0$, we have
\begin{align*}
g^{(2)}(0) =  \f{\langle I(t)^2 \rangle}{\langle I(t) \rangle^2} \geq  \f{ \langle   I(t)  \rangle^2 }{\langle I(t) \rangle^2} = 1,
\end{align*}
as desired. This implies that light in a number state, which has $g^{(2)}(0) < 1$, has no classical analog.

\item Here we show that $g^{(2)}(\tau) \leq g^{(2)}(0)$. Viewing $\langle I(t) I(t+\tau) \rangle$ as the inner product $\langle I(t), I(t+\tau) \rangle$ (which is possible since intensities are real numbers), we see immediately that $\langle I(t) I(t+\tau) \rangle \leq \langle I(t) I(t) \rangle$, which gives
\begin{align*}
g^{(2)}(\tau) =  \f{\langle I(t) I (t+\tau) \rangle }{\langle I(t) \rangle^2}  \leq  \f{\langle I(t) I (t) \rangle }{\langle I(t) \rangle^2} = g^{(2)}(0).
\end{align*}

This implies that anti-bunched light, which has $g^{(2)}(\tau) > g^{(2)}(0)$, has no classical analog.

\item Consider chaotic classical light generated by an ensemble of $\nu$ atoms. The total electric field can be expressed as $E(t) = \sum^\nu_{i=1} E_i(t)$, where the phases of $E_i$ are random. Here we show that when $\nu$ is large, 
\begin{align*}
g^{(2)}(\tau) = 1 + \vert g^{(1)} (\tau)\vert ^2.
\end{align*}

Since statistical averages in which the random electric field phases do not cancel are zero, we have 
\begin{align*}
\langle E^*(t) E(t+\tau) \rangle = 
\end{align*}



\end{enumerate}





\noindent \textbf{2. Quantum Coherence of Light.}\\

\noindent Consider light in a single mode of the radiation field. The quantum mechanical expressions for
first-order and second-order coherence are
\begin{align*}
g^{(1)}(\mathbf{r}_1, t_1, \mathbf{r}_2, t_2) 
&= \f{\langle \hat{E}^-(\mathbf{r}_1, t_1)  \hat{E}^+(\mathbf{r}_2, t_2) \rangle }{ 
\langle \hat{E}^-(\mathbf{r}_1, t_1) \hat{E}^+(\mathbf{r}_1, t_1) \rangle^{1/2}   
\langle \hat{E}^-(\mathbf{r}_2, t_2) \hat{E}^+(\mathbf{r}_2, t_2)\rangle^{1/2}} \\
g^{(2)}(\mathbf{r}_1, t_1, \mathbf{r}_2, t_2 ; \mathbf{r}_2, t_2, \mathbf{r}_1, t_1) 
&= \f{\langle \hat{E}^-(\mathbf{r}_1, t_1) \hat{E}^-(\mathbf{r}_2, t_2) \hat{E}^+(\mathbf{r}_2, t_2) \hat{E}^+(\mathbf{r}_1, t_1)  \rangle }{
\langle \hat{E}^-(\mathbf{r}_1, t_1) \hat{E}^+(\mathbf{r}_1, t_1) \rangle 
\langle \hat{E}^-(\mathbf{r}_2, t_2) \hat{E}^+(\mathbf{r}_2, t_2)\rangle}
\end{align*}
where
\begin{align*}
\hat{E}^+(\mathbf{r},t) &= 
i \sqrt{ \f{\hbar \omega}{2 \epsilon_0 V} } a e^{-i(\omega t - \mathbf{k} \cdot \mathbf{r})} \\
\hat{E}^-(\mathbf{r},t) &= 
- i \sqrt{ \f{\hbar \omega}{2 \epsilon_0 V} } a^\dagger e^{+i(\omega t - \mathbf{k} \cdot \mathbf{r})} 
\end{align*}


\begin{enumerate}[label=\alph*)]

\item Here we show that 
\begin{align*}
g^{(2)}(0) = \f{\langle a^\dagger a^\dagger a a  \rangle}{\langle a^\dagger a  \rangle^2}   = \f{\langle n^2 \rangle - \langle n  \rangle}{\langle n \rangle^2}
\end{align*}

\item Next we show that for number states of light $\ket{n}$ where $n > 2$, 
\begin{align*}
\vert g^{(1)} \vert = 1 \quad\quad g^{(2)} = 1 - \f{1}{n}
\end{align*}
independent of space-time separation. \\

\noindent For $n=0$, we calculate $g^{(1)}$ and $g^{(2)}$:
\begin{align*}
g_{n=0}^{(1)} &= \\
g_{n=0}^{(2)} &= 
\end{align*}

\noindent For $n=1$, we calculate $g^{(1)}$ and $g^{(2)}$
\begin{align*}
g_{n=1}^{(1)} &= \\
g_{n=1}^{(2)} &= 
\end{align*}

\item Consider a coherent state $\ket{\al}$, we show that $\vert g^{(1)} \vert = \vert g^{(2)} \vert = 1$
\begin{align*}
\vert g^{(1)}_{\ket{\al}} \vert =  
\end{align*}
\begin{align*}
\vert g^{(2)}_{\ket{\al}} \vert =  
\end{align*} 

\item Finally, consider chaotic light with density matrix
\begin{align*}
\hat{\rho} = \lp  1 - e^{-\hbar \omega / k_B T} \rp \sum_n e^{-n\hbar \omega / k_B T} \ketbra{n}.
\end{align*}
Here we show that $\vert g^{(1)} \vert = 1$ and $\vert g^{(2)} \vert = 2$.\\


Compare these results to what we found in Problem 1... What about multi-mode chaotic light?

\item Consider the states
\begin{align*}
\ket{\psi_\pm} = \f{\ket{\al} \pm  \ket{-\al}}{ \sqrt{2}  \sqrt{1 \pm e^{-2\abs{\al}^2}}}.
\end{align*}
Let us compute $g^{(2)}(\tau)$ for these states as a function of $\al$. \\


Do either of these two states show non-classical second-order coherence? Why (or why not)?
(Make sure you agree with the normalization given)

\end{enumerate}



\noindent \textbf{3. The Quantum Beamsplitter.}\\

\noindent Let the beamsplitter operator $B$, acting with angle $\angle$ on modes $a$ and $b$, by defined by 
\begin{align*}
B = \exp\lb \theta \lp a^\dagger b - a b^\dagger \rp \rb .
\end{align*}

\begin{enumerate}[label=\alph*)]

\item Here we show that $B$ conserves photon number and leaves coherent states as coherent states. 

\item Let $\ket{\al}$ be a coherent state. Here we compute $B \ket{0}_b \ket{\al}_a$. \\

From here, we see that the output is a tensor product of coherent states for all $\theta$. This makes sense: the beamsplitter has well-defined transmission and reflection coefficients. What are these in terms of $\theta$? 

\item There is close connection between the Lie group $SU(2)$ and the algebra of two coupled harmonic oscillators, which is useful for understanding $B$. Define
\begin{align*}
s_z = a^\dagger a - b^\dagger b \quad\quad s_+ = a^\dagger a \quad\quad s_- = b^\dagger b,
\end{align*}
and let $s_\pm = (s_x \pm is s_y)/\sqrt{2}$. \\


What is $B(\theta)$ in spin space? \\

What is $a^\dagger a + b^\dagger b$ in spin space? \\

We now show that $s_x, s_y, s_z$ have the same commutation relations as Pauli matrices. \\

How does this explain why $a^\dagger a + b^\dagger b$ is invariant under $B$. It is the Casimir operator of the algebra. 

\item Here we look at how a beamsplitter transforms an input of a photon-number eigenstate. Let 
\begin{align*}
B(\theta) = \exp\lb \theta \lp -a^\dagger b + ab^\dagger  \rp\rb,
\end{align*}
and $B = B(\pi/4)$ be a 50/50 beamsplitter, such that
\begin{align*}
BaB^\dagger = \f{a+b}{\sqrt{2}} \quad\quad  BbB^\dagger = \f{-a + b}{\sqrt{2}}.
\end{align*}

Let us compute $B\ket{0}_b\ket{n}_a$. \\


Note that the result is NOT $\ket{n/2}\ket{n/2}$ since $\ket{n}$ is a number state and not a coherent state. \\


What photon number states have the largest amplitude? \\

How sharp is the distribution for $n = 10$, and $n = 100$, or as a function of $n$, if a general solution exists? To do this, we look at the binomial expansion for $(a^\dagger  + b^\dagger)^n$. 

\end{enumerate}


\noindent \textbf{4. The Hanbury-Brown and Twiss experiment and $g^{(2)}(\tau)$.} \\

\noindent Here we look at how the HBT experiment measures $g^{(2)}(\tau)$. To this end, let $a,a^\dagger,b,b^\dagger$ be the raising and lowering operators for the two modes of light input to the beamsplitter, and let the unitary transformation performed by the beamsplitter be defined by 
\begin{align*}
a_1 = U a U^\dagger = \f{a+b}{\sqrt{2}} \quad\quad b_1 = UbU^\dagger = \f{a-b}{\sqrt{2}}.
\end{align*}
For light input in state $\ket{\psi_a}$, suppose that the output of the coincidence circuit is a voltage
\begin{align*}
V_{\psi_a,0_b} = V_0 \bra{\psi_a,  0_b} a_1^\dagger a_1 b_1^\dagger b_1 \ket{\psi_a, 0_b},
\end{align*}
which is nothing but the average of the product of the two detected photon signals. We will show that $V_{\psi_a,0_b}$ gives measure of $g^{(2)}(\tau)$ up to an additive offset and normalization. 




\end{document}








