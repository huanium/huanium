\documentclass{article}
\usepackage{physics}
\usepackage{graphicx}
\usepackage{caption}
\usepackage{amsmath}
\usepackage{bm}
\usepackage{framed}
\usepackage{authblk}
\usepackage{empheq}
\usepackage{amsfonts}
\usepackage{esint}
\usepackage[makeroom]{cancel}
\usepackage{dsfont}
\usepackage{centernot}
\usepackage{mathtools}
\usepackage{subcaption}
\usepackage{bigints}
\usepackage{amsthm}
\theoremstyle{definition}
\newtheorem{lemma}{Lemma}
\newtheorem{defn}{Definition}[section]
\newtheorem{prop}{Proposition}[section]
\newtheorem{rmk}{Remark}[section]
\newtheorem{thm}{Theorem}[section]
\newtheorem{exmp}{Example}[section]
\newtheorem{prob}{Problem}[section]
\newtheorem{sln}{Solution}[section]
\newtheorem*{prob*}{Problem}
\newtheorem{exer}{Exercise}[section]
\newtheorem*{exer*}{Exercise}
\newtheorem*{sln*}{Solution}
\usepackage{empheq}
\usepackage{tensor}
\usepackage{xcolor}
%\definecolor{colby}{rgb}{0.0, 0.0, 0.5}
\definecolor{MIT}{RGB}{163, 31, 52}
\usepackage[pdftex]{hyperref}
%\hypersetup{colorlinks,urlcolor=colby}
\hypersetup{colorlinks,linkcolor={MIT},citecolor={MIT},urlcolor={MIT}}  
\usepackage[left=1in,right=1in,top=1in,bottom=1in]{geometry}

\usepackage{newpxtext,newpxmath}
\newcommand*\widefbox[1]{\fbox{\hspace{2em}#1\hspace{2em}}}

\newcommand{\p}{\partial}
\newcommand{\R}{\mathbb{R}}
\newcommand{\C}{\mathbb{C}}
\newcommand{\lag}{\mathcal{L}}
\newcommand{\nn}{\nonumber}
\newcommand{\ham}{\mathcal{H}}
\newcommand{\M}{\mathcal{M}}
\newcommand{\I}{\mathcal{I}}
\newcommand{\K}{\mathcal{K}}
\newcommand{\F}{\mathcal{F}}
\newcommand{\w}{\omega}
\newcommand{\lam}{\lambda}
\newcommand{\al}{\alpha}
\newcommand{\be}{\beta}
\newcommand{\x}{\xi}

\newcommand{\G}{\mathcal{G}}

\newcommand{\f}[2]{\frac{#1}{#2}}

\newcommand{\ift}{\infty}

\newcommand{\lp}{\left(}
\newcommand{\rp}{\right)}

\newcommand{\lb}{\left[}
\newcommand{\rb}{\right]}

\newcommand{\lc}{\left\{}
\newcommand{\rc}{\right\}}


\newcommand{\V}{\mathbf{V}}
\newcommand{\U}{\mathcal{U}}
\newcommand{\Id}{\mathcal{I}}
\newcommand{\D}{\mathcal{D}}
\newcommand{\Z}{\mathcal{Z}}

%\setcounter{chapter}{-1}


\usepackage{enumitem}



\usepackage{listings}
\captionsetup[lstlisting]{margin=0cm,format=hang,font=small,format=plain,labelfont={bf,up},textfont={it}}
\renewcommand*{\lstlistingname}{Code \textcolor{violet}{\textsl{Mathematica}}}
\definecolor{gris245}{RGB}{245,245,245}
\definecolor{olive}{RGB}{50,140,50}
\definecolor{brun}{RGB}{175,100,80}

%\hypersetup{colorlinks,urlcolor=colby}
\lstset{
	tabsize=4,
	frame=single,
	language=mathematica,
	basicstyle=\scriptsize\ttfamily,
	keywordstyle=\color{black},
	backgroundcolor=\color{gris245},
	commentstyle=\color{gray},
	showstringspaces=false,
	emph={
		r1,
		r2,
		epsilon,epsilon_,
		Newton,Newton_
	},emphstyle={\color{olive}},
	emph={[2]
		L,
		CouleurCourbe,
		PotentielEffectif,
		IdCourbe,
		Courbe
	},emphstyle={[2]\color{blue}},
	emph={[3]r,r_,n,n_},emphstyle={[3]\color{magenta}}
}


\begin{document}
\begin{framed}
\noindent Name: \textbf{Huan Q. Bui}\\
Course: \textbf{8.422 - AMO II}\\
Problem set: \textbf{\#3}\\
Due: Friday, Mar 3, 2022\\
References: 
\end{framed}
	
	
\noindent \textbf{1. Squeezing Hamiltonian.} \\

\noindent The dimensionless squeezing Hamiltonian is 
\begin{align*}
\ham = \f{\hbar \omega}{2} (\tilde{p}^2 - \tilde{x}^2). 
\end{align*}
With $p = (\tilde{p}- \tilde{x}) / \sqrt{2}$ and $x = (\tilde{p} + \tilde{x})/\sqrt{2}$, we have $[x,p] = \tilde{x}, \tilde{p} = i$ and the Hamiltonian becomes
\begin{align*}
\ham = \hbar \omega xp
\end{align*}
plus an offset which we ignore. 

\begin{enumerate}

\item Consider a wavefunction $\psi_0(x)$ at $t=0$. From Schr\"{o}dinger's equation we find 
\begin{align*}
i\hbar \f{\p}{\p t}\psi(x,t) = \hbar \omega x p \psi(x,t) = -i \hbar \omega x \f{\p}{\p x} \psi(x,t).
\end{align*}




\item 

\item 

\item 

\item 

\item 

\end{enumerate}



\noindent \textbf{2. Disentangling the Squeezing Operator.} \\

\noindent In this problem we show that
\begin{align*}
e^{\f{r}{2} ( {a^\dagger}^2 - a^2)}  \ket{0} = \f{1}{\sqrt{\cosh r}} \sum_{n=0}^\infty  \f{\sqrt{(2n)!}}{2^n n!} (\tanh r)^n \ket{2n}.
\end{align*}
To do this, we follow the parts below. 

\begin{enumerate}[label=(\alph*)]

\item We first calculate the following commutators:
\begin{align*}
[a^2, {a^\dagger}^2] 
&=   aa a^\dagger a^\dagger - a^\dagger a^\dagger a a \\
&= a(1+a^\dagger a) a^\dagger - a^\dagger (aa^\dagger - 1) a \\
&= 1 + a^\dagger a + (1+a^\dagger a)(1 + a^\dagger a ) - a^\dagger a a^\dagger a + a^\dagger a \\
&= 4a^\dagger a + 2. 
\end{align*}
\begin{align*}
[a^2, a^\dagger a ] 
&= aaa^\dagger a - a^\dagger a a a  \\
&= a(1+a^\dagger a) a - (aa^\dagger -1) aa \\
&= 2a^2.
\end{align*}
\begin{align*}
[{a^\dagger}^2 , a^\dagger a] 
&= a^\dagger a^\dagger a^\dagger a  - a^\dagger a a^\dagger a^\dagger \\
&= a^\dagger a^\dagger (aa^\dagger - 1) - a^\dagger (1 + a^\dagger a) a^\dagger \\
&= -2{a^\dagger}^2.
\end{align*}

From these, we conclude that the Lie algebra of operators $\{  a^2, {a^\dagger}^2, a^\dagger a + 1/2  \}$ is closed under commutation. This means we must be able to write
\begin{align*}
e^{\f{r}{2} ({a^\dagger}^2 - a^2)  } = e^{\f{u}{2} {a^\dagger}^2} e^{t(^\dagger a + 1/2)} e^{\f{v}{2}a^2}. 
\end{align*}
Our job now is to find the numbers $u,t,v$, which are functions of $r$. To do this, we find any other Lie algebra whose three operators obey the same commutation relations which allows us to more easily find $u,t,v$. It turns out that Pauli matrices work. 

\item Consider the replacement:
\begin{align*}
a^\dagger a + \f{1}{2} 
&\to \sigma_z = 
\begin{pmatrix}
1 & 0 \\ 0 & -1 
\end{pmatrix} \\
a^2 
&\to -\sigma_- 
= -\sigma_x +  i\sigma_y 
= 
\begin{pmatrix}
0 & 0 \\ -2 & 0
\end{pmatrix} \\
{a^\dagger}^2
&\to \sigma_+ = \sigma_x + i\sigma_y = 
\begin{pmatrix}
0 & 2 \\ 0 & 0
\end{pmatrix}
\end{align*}
Let us check that the commutation relations above still hold. 
\begin{align*}
[a^2, {a^\dagger}^2]
\to
[-\sigma_-, \sigma_+] 
=
[-\sigma_x, i\sigma_y]  + [i\sigma_y, \sigma_x] = 4\sigma_z \leftarrow 4a^\dagger a + \f{1}{2}  \quad \checkmark
\end{align*}
\begin{align*}
[a^2, a^\dagger a] 
\to 
[-\sigma_-, \sigma_z - 1/2] = [-\sigma_-, \sigma_z] 
= 
2i\sigma_y -2\sigma_x = 2(-\sigma_-) \leftarrow 2 a^2 \quad \checkmark
\end{align*}
\begin{align*}
[{a^\dagger}^2, a^\dagger a] 
=
[\sigma_+, \sigma_z - 1/2] = [\sigma_+, \sigma_z]
=
-2i\sigma_y -2 \sigma_x = -2 \sigma_+  \leftarrow  -2 {a^\dagger}^2 \quad \checkmark
\end{align*}


\item With
\begin{align*}
\f{r}{2}\lp {a^\dagger}^2 - a^2 \rp = \f{r}{2}(\sigma_+  + \sigma_-) = r\sigma_x = \begin{pmatrix}
0 & r \\ r& 0
\end{pmatrix},
\end{align*}
we find 
\begin{align*}
e^{\f{r}{2}\lp {a^\dagger}^2 - a^2 \rp } = \begin{pmatrix}
\cosh r & \sinh r \\ \sinh r & \cosh r
\end{pmatrix}.
\end{align*}
Mathematica code:
\begin{lstlisting}
In[7]:= MatrixExp[r*PauliMatrix[1]] // FullSimplify
Out[7]= {{Cosh[r], Sinh[r]}, {Sinh[r], Cosh[r]}}
\end{lstlisting}


\item Similarly, we have
\begin{align*}
U = e^{\f{u}{2} {a^\dagger}^2} = \exp\lb \f{u}{2} \begin{pmatrix}  
0 & 2 \\ 0 & 0 
\end{pmatrix} \rb = 
\exp\lb \begin{pmatrix}  
0 & u \\ 0 & 0 
\end{pmatrix} \rb 
= \begin{pmatrix}
1 & u \\ 0 & 1
\end{pmatrix}
\end{align*}
\begin{align*}
T = e^{t\lp a^\dagger a + 1/2 \rp} = e^{t\sigma_z} = \exp\lb \begin{pmatrix} t & 0 \\ 0 & -t  \end{pmatrix} \rb
= \begin{pmatrix}
e^t & 0 \\ 0 & e^{-t}
\end{pmatrix}
\end{align*}
\begin{align*}
V = e^{\f{v}{2}a^2} = \exp\lb \f{v}{2}\begin{pmatrix}  0 & 0 \\ -2 & 0   \end{pmatrix} \rb
= \exp\lb \begin{pmatrix}  0 & 0 \\ -v & 0  \end{pmatrix} \rb = \begin{pmatrix}
1 & 0 \\ -v & 1
\end{pmatrix}.
\end{align*}
With these,
\begin{align*}
UTV = 
\begin{pmatrix}
e^t - e^{-t} uv & e^{-t} u    \\ -e^{-t} v & e^{-t}
\end{pmatrix}.
\end{align*}
Mathematica code:
\begin{lstlisting}
In[14]:= U = MatrixExp[{{0, u}, {0, 0}}]
Out[14]= {{1, u}, {0, 1}}

In[15]:= T = MatrixExp[t*PauliMatrix[3]]
Out[15]= {{E^t, 0}, {0, E^-t}}

In[16]:= V = MatrixExp[{{0, 0}, {-v, 0}}]
Out[16]= {{1, 0}, {-v, 1}}

In[18]:= U . T . V // FullSimplify
Out[18]= {{E^t - E^-t u v, E^-t u}, {-E^-t v, E^-t}}
\end{lstlisting}

\item Comparing the results of Parts (c) and (d) we easily find that
\begin{align*}
t = -\ln \cosh r, \quad\quad u = -v = e^t \sinh r = \tanh r .
\end{align*}
so we have
\begin{align*}
e^{\f{r}{2}\lp {a^\dagger}^2 - a^2 \rp} = e^{\f{\tanh r }{2} {a^\dagger}^2} e^{-\ln \cosh r  \lp a^\dagger a + 1/2 \rp} e^{-\f{\tanh r }{2} a^2} = 
\f{1}{\sqrt{\cosh r}} 
e^{\f{\tanh r }{2} {a^\dagger}^2} e^{-\ln \cosh r  ( a^\dagger a ) } e^{-\f{\tanh r }{2} a^2}
\end{align*}


\item Applying the operator above to $\ket{0}$, we realize that the two right most operators act on $\ket{0}$ as the identity, so we end up with
\begin{align*}
e^{\f{r}{2}\lp {a^\dagger}^2 - a^2 \rp} \ket{0} 
&= \f{1}{\sqrt{\cosh r}} e^{\f{\tanh r }{2} {a^\dagger}^2}  \ket{0} \\
&= \f{1}{\sqrt{\cosh r}} \sum_{n=0}^\infty  \f{(\tanh r)^n}{2^n n!} {a^\dagger}^{2n}  \ket{0} \\
&= \f{1}{\sqrt{\cosh r}} \sum_{n=0}^\infty \f{\sqrt{(2n)!}}{2^n n!} (\tanh r)^{n} \ket{2n},
\end{align*}
as desired. 

\end{enumerate}



\noindent \textbf{3. Generation of Squeezed States by Two-Photon Interactions.}\\


\noindent Consider a mode $(\vec{k}, \vec{\epsilon})$ with wavevector $\vec{k}$ and polarization $\vec{\epsilon}$ of the EM field with frequency $\omega$ whose Hamiltonian is given by 
\begin{align*}
H = \hbar \omega a^\dagger a + i\hbar \Lambda \lb (a^\dagger)^2 e^{-2i\omega t} - a^2 e^{2i \omega t } \rb.
\end{align*}
The first term is the energy of the mode of the free field. The second term describes a
two-photon interaction process such as parametric amplification (a classical wave of frequency $2\omega$
generating two photons with frequency $\omega$). $\Lambda$ is a real quantity characterizing the strength of the interaction. In this problem, we will show that this Hamiltonian produces squeezed vacuum and explore how
it acts on coherent states.

\begin{enumerate}[label=(\alph*)]

\item The equation of motion for $a(t)$ in the Heisenberg picture is 
\begin{align*}
\f{d}{dt}a_H = \f{i}{\hbar} [H_H, a_H],
\end{align*}
where
\begin{align*}
???
\end{align*}


Now we compute the commutators:
\begin{align*}
[a^\dagger a , a] = [a,a^\dagger a] = -[a,a^\dagger] a - a^\dagger [a,a] = -a.
\end{align*}
\begin{align*}
[(a^\dagger)^2, a] = -[a,(a^\dagger)^2] = -[a,a^\dagger]a^\dagger - a^\dagger [a, a^\dagger] = -2a^\dagger
\end{align*}
 \begin{align*}
 [a^2,a] = 0
\end{align*}
From these, we find the equations of motion for $a$ and $a^\dagger$
\begin{align*}
\dot a  
&= \f{i}{\hbar} \lp - \hbar \omega a  -2 i \hbar \Lambda a^\dagger e^{-2i\omega t}  \rp =  -i \omega a + 2 \Lambda a^\dagger e^{-2i\omega t } \\
\dot a^\dagger
&= i \omega a^\dagger + 2\Lambda a e^{2i\omega t}.
\end{align*}
In matrix form, 


\item 


\item 

\item 


\item 

\end{enumerate}



\end{document}








