\documentclass[reprint,
%superscriptaddress,
%groupedaddress,
%unsortedaddress,
%runinaddress,
%frontmatterverbose, 
%preprint,
%preprintnumbers,
nofootinbib,
%nobibnotes,
%bibnotes,
amsmath,amssymb,
aps]{revtex4-1}
\usepackage{physics}
\usepackage{graphicx}
\usepackage{amsmath}
\usepackage{subfigure}
\usepackage{bm}
\usepackage{dcolumn}% Align table columns on decimal point
\usepackage{framed}
\usepackage{empheq}
\usepackage{amsfonts}
\usepackage{esint}
\usepackage[makeroom]{cancel}
\usepackage{dsfont}
\usepackage{centernot}
\usepackage{mathtools}
\usepackage{bigints}
\usepackage{empheq}
\usepackage{tensor}
\usepackage{xcolor}
\definecolor{colby}{rgb}{0.0, 0.0, 0.5}
\definecolor{MIT}{RGB}{163, 31, 52}
\usepackage[pdftex]{hyperref}
%\hypersetup{colorlinks,linkcolor={MIT},citecolor={MIT},urlcolor={MIT}}  
\hypersetup{colorlinks,linkcolor=blue,citecolor=blue,urlcolor=blue}  
%\usepackage[left=1in,right=1in,top=1in,bottom=1in]{geometry}
%\setcounter{MaxMatrixCols}{20}
% \usepackage{newpxtext,newpxmath}
%\newcommand*\widefbox[1]{\fbox{\hspace{2em}#1\hspace{2em}}}

\newcommand{\p}{\partial}
\newcommand{\R}{\mathbb{R}}
\newcommand{\C}{\mathbb{C}}
\newcommand{\lag}{\mathcal{L}}
\newcommand{\nn}{\nonumber}
\newcommand{\ham}{\mathcal{H}}
\newcommand{\M}{\mathcal{M}}
\newcommand{\I}{\mathcal{I}}
\newcommand{\K}{\mathcal{K}}
\newcommand{\F}{\mathcal{F}}
\newcommand{\w}{\omega}
\newcommand{\lam}{\lambda}
\newcommand{\al}{\alpha}
\newcommand{\be}{\beta}
\newcommand{\x}{\xi}
\newcommand{\G}{\mathcal{G}}
\newcommand{\f}[2]{\frac{#1}{#2}}
\newcommand{\ift}{\infty}
\newcommand{\lp}{\left(}
\newcommand{\rp}{\right)}
\newcommand{\lb}{\left[}
\newcommand{\rb}{\right]}
\newcommand{\lc}{\left\{}
\newcommand{\rc}{\right\}}
\newcommand{\V}{\mathbf{V}}
\newcommand{\U}{\mathcal{U}}
\newcommand{\Id}{\mathcal{I}}
\newcommand{\D}{\mathcal{D}}
\newcommand{\Z}{\mathcal{Z}}

\definecolor{gris245}{RGB}{245,245,245}
\definecolor{olive}{RGB}{50,140,50}
\definecolor{brun}{RGB}{175,100,80}

\newcommand{\diag}{\text{diag}}
\newcommand{\psirot}{\ket{\psi_\text{rot}(t)} }
\newcommand{\RWA}{\ham_\text{rot}^\text{RWA}}

% 3j symbol
\newcommand{\tj}[6]{ \begin{pmatrix}
		#1 & #2 & #3 \\
		#4 & #5 & #6 
\end{pmatrix}}



\begin{document}
	
	

\title{Linear response theory and applications in many-body quantum physics}
\author{Huan Q. Bui}
\email{huanbui@mit.edu}
\affiliation{
	MIT-Harvard Center for Ultracold Atoms, Research Laboratory of Electronics, and Department of Physics, Massachusetts Institute of Technology, Cambridge, Massachusetts 02139, USA}
\date{\today}


\begin{abstract}
	In this paper, we...
\end{abstract}

\maketitle


\section{Introduction}



It is often possible to solve for the dynamics of single-body $N$-level systems exactly, either analytically or numerically. For example, Rabi oscillations perfectly describe the dynamics of a two-level system. However, many-body systems such as non-ideal Bose-Einstein condensates or the unitary Fermi gas often exhibit complicated interactions, making exact descriptions of these systems unattainable in most cases. \textcolor{purple}{Examples sound, vortices, something... give some examples here, transport phenomena, etc. .} As a result, solutions typically come from linear response theory. 

The object of study in linear response theory is the linear response function. In the simplest case, the response function is simply a proportionality constant $\chi$ relating the response of a system $x(t)$ due to a perturbation $h(t)$ at that instant $t$. For linear, time-invariant systems with memory, the response $x(t)$ gets contributions due to past perturbations $h(t')$ and is given by the convolution of the response function with the perturbation:
\begin{align}\label{eq:linear_response}
x(t) = \int_{-\infty}^{\infty} \chi(t-t') h(t') \,dt'.
\end{align}
In general, $\chi$ solves the equation $ L \chi(t-t') = \delta(t-t')$ where $L$ is the linear differential operator associated with the system under consideration. It is identically the Green's function associated with $L$ and characterizes the system's response to an external impulse. 

As Section \textcolor{purple}{blah} will show, linear response theory provides a means for extracting rich physics. For the case of the damped harmonic oscillator, we can fully characterize its resonance behavior and energy dissipation based solely on the (complex) response function. Linear response theory also finds applications in statistical mechanics: When it is in statistical mechanics, we can understand and derive the famous fluctuation-dissipation theorem (FDT). In Section \textcolor{purple}{blah}, we will also see how linear response theory, when applied to quantum many-body systems, provides us with \textcolor{blue}{things to measure to find out about properties of system etc. I don't know how to phrase this yet.}



This paper is structured as follows: 

Talk about magnetic susceptibility, which is how much the magnetization change as a function of the applied/external magnetic field. Of course, talk about the trivial example of the damped (driven) harmonic oscillator. s

A non-trivial example is how does one study, say, sound in a BEC or a Fermi gas? Typically the setup is such that one has the Hamiltonian governing the dynamics of some quantum gas and then turn on perturbation: like shaking the trapping potential or stirring the BEC with a laser or something like that. How do we predict the response of the gas under such perturbations, as a function of parameters of the perturbation and as a function of other state variables? 



\section{Classical Linear Response}

%
\subsection{Causality}
The response function tells us how a system responds \textit{as a consequence} of some perturbation, which means it respects causality, i.e., $\chi(t) = 0$ for $t < t_0 = 0$ when the perturbation is turned on. Consider $\chi(t)$ in Fourier basis:
\begin{align*}
\chi(t) = \int_{-\infty}^\infty  e^{-i\omega t } \tilde{\chi}(\omega) \,d\omega.
\end{align*}
For $t<0$, we integrate by closing the contour in the upper half-plane. Since the answer is zero, $\chi(\omega)$ must be analytic in the upper half-plane. In other words, 
\begin{align*}
\text{Causality} \implies \chi(\omega) \text{ analytic for } \Im(\chi) > 0.
\end{align*}
Let $\chi'(\omega) = \Re \chi(\omega)$ and $\chi''(\omega) = \Im \chi(\omega)$. $\chi'(\omega)$ is called the \textit{reactive part} of the response, and $\chi''(\omega)$ the \textit{dissipative part}. SInce $\chi(t)$ is invariance under time translations, one can show that $\chi'(\omega)$ and $\chi''(\omega)$ are even and odd functions, respectively.


If $\chi(\omega)$ decays faster than $1/\abs{\omega}$ as $\omega \to \infty$ in addition to being analytic in the upper half-plane, then $\chi'(\omega)$ and $\chi''(\omega)$ are related via the Kramer-Kronig relations. In particular, one can reconstruct the full complex response function knowing only its imaginary part:
\begin{align*}
\chi(\omega) = \int_{-\infty}^\infty \f{d\omega'}{\pi} \f{\Im \chi'(\omega')}{\omega' - \omega - i\epsilon}.
\end{align*}

As an example, let us consider the damped driven harmonic oscillator. Fourier-transforming the equation of motion and using  \eqref{eq:linear_response} give the response function in frequency domain:
\begin{align*}
\ddot{x} + \gamma \dot{x} + \omega_0^2 x = h(t) \implies 
\tilde{\chi}(\omega) = ( \omega_0^2 - \omega^2 - i \gamma \omega)^{-1}.
\end{align*}
The static response function is $\tilde{\chi}(\omega = 0) = 1/\omega_0^2$, giving $x = h/\omega_0^2$ as expected. For a drive with frequency $\omega$, the response function has poles $\omega_\star$ given by 
\begin{align*}
\omega_* = -i\gamma/2 \pm \sqrt{\omega_0^2  - \gamma^2/4}.
\end{align*}
When the oscillator is underdamped $(\gamma/2 < \omega_0)$, the poles have both real and imaginary parts and are in the lower half-plane. When the oscillator is overdamped $(\gamma/2 > \omega_0)$, the poles are on the negative imaginary axis. In both cases, $\tilde{\chi}(\omega)$ is analytic in the upper half-plane, consistent with causality. 

Suppose the drive is $h(t) = h_0 \Re(e^{-i\omega t})$ and that the system is in steady state. The average power dissipated by the system is the average power absorbed: 
\begin{align*}
\langle P \rangle = 2h_0^2 \omega \Im\chi(\omega) = 2h_0^2  \gamma \omega^2 [(\omega_0^2 - \omega^2)^2 + \gamma^2 \omega^2]^{-1},
\end{align*}
which has a Lorentzian line shape with FWHM $\gamma$ near resonance. Note that the first equality holds in general and depends only on the imaginary part of the response function. As we will see, there is a deeper reason for the dependence on $\Im \tilde{\chi}(\omega)$.

%
\subsection{Response functions and correlation functions}
Let us generalize a bit and apply the idea of linear response theory to a canonical ensemble that is perturbed out of equilibrium. Let $\mathcal{H}$ be the unperturbed Hamiltonian and consider some dynamical variable $A$, which is a function defined over the phase space for the system. The equilibrium average for $A$ is:
\begin{align*}
\langle A \rangle =  \mathcal{Z}^{-1}(\mathcal{H}) \int e^{-\be \mathcal{H}} A\, d\Omega,
\end{align*}
where $\mathcal{Z}(\mathcal{H}) = \int e^{-\be \mathcal{H}} \, d\Omega$, and $\int \cdot \,d\Omega$ is an integral over phase space. Next, suppose the system is initially perturbed by $\Delta \mathcal{H} = -f A$ where $f$ is some external field. We are interested in what happens to the non-equilibrium average $\overline{A}$ as time progresses. This quantity is given by 
\begin{align*}
\overline{A}(t) = \mathcal{Z}^{-1}(\mathcal{H} + \Delta \mathcal{H})  \int e^{-\be(\mathcal{H} + \Delta \mathcal{H})} A(t) \, d\Omega. 
\end{align*}
For $t>0$, $\Delta \mathcal{H}$ vanishes, so it is $\mathcal{H}$ that governs the time evolution of $A(t)$. Assuming that the perturbations are small, i.e. $A(t)$ does not deviate much from $\langle A \rangle$, we can find $\overline{A}(t)$ in terms of $A$ with corrections due to $\Delta \mathcal{H}$:
\begin{align*}
\overline{A}(t) 
%&= \f{\int e^{-\be \mathcal{H} (1- \be \Delta \mathcal{H} + \dots )A(t) \,d\Omega}}{\int e^{-\be \mathcal{H} (1 - \be \Delta \mathcal{H}) + \dots } \, d\Omega } \\ 
&\approx \mathcal{Z}^{-1}(\mathcal{H})  \int d\Omega \, e^{-\be \mathcal{H}} A(t)   \lb 1 - \be \Delta \mathcal{H}  + \langle  \be \Delta \mathcal{H}  \rangle  \rb \\ 
&= \langle A \rangle - \be \lb \langle \Delta \mathcal{H} A(t) \rangle - \langle A \rangle \langle \Delta \mathcal{H} \rangle  \rb.
\end{align*}
Insert the expression for $\Delta \mathcal{H}$ into the result above to get
\begin{align}
\label{eq:fluctuation}
\Delta \overline{A}(t) = \be f \langle \delta A(t) \delta A(0) \rangle + {O}(f^2),
\end{align}
where $\Delta \overline{A}(t) = \overline{A}(t) - \langle A \rangle$ is the (averaged, thus macroscopic) spontaneous fluctuation, and $\delta A(t) = A(t) - \langle A \rangle$ is the (microscopic) instantaneous fluctuation at time $t$. This is a remarkable result that we just showed: the spontaneous fluctuations is related to the time correlation function of instantaneous fluctuations. Since generally, $\langle \delta A \rangle = 0$ but $\langle A^2 \rangle \neq 0$, we see that $\Delta \overline{A}(t)$ decays in time. This is the essence of Onsanger's remarkable \textit{regression hypothesis} which was later realized to be a consequence of the fluctuation-dissipation theorem, which we will brief discuss in Section \textcolor{red}{blah}.

Next, we write the spontaneous fluctuation $\Delta \overline{A}(t)$ in terms of the linear response function $\chi(t-t')$:
\begin{align}
\label{eq:fluctuation_response_function}
\Delta \overline{A}(t) = \int_{-\infty}^\infty dt' \chi(t-t') f(t'). 
\end{align}
For convenience, consider $f(t) = f \Theta(-t)$ which tells us that the perturbation is turned off at $t=0$, then we quickly find
\begin{align*}
\Delta \overline{A}(t) = f \int_{t}^\infty \chi(t') \,dt'.
\end{align*}
By comparing this with \eqref{eq:fluctuation}, we find that
\begin{align*}
\chi(t) = -\be \f{d}{dt} \langle \delta A(0) \delta A(t) \rangle \Theta(t).
\end{align*}
We see that the response function for a non-equilibrium system is related to the correlations between spontaneous fluctuations at different times as they occur in the equilibrium system.

\section{Quantum Linear Response}

\subsection{Overview}
The treatment of linear response theory in quantum mechanics is similar to the classical case, with the exception that functions become operators, which do not commute in general. Let the dynamics of a generic quantum system be governed by a bare Hamiltonian $\mathcal{H}$ and let a small perturbation be $\mathcal{H}'(t)$ which turns on for $t\geq t_0$. In the interaction picture, the states evolve as $\ket{\psi(t)}_I = U(t,t_0) \ket{\psi(t_0)}_I$, where
\begin{align*}
U(t,t_0) 
&= \mathcal{T} \exp\lp -i \int_{t_0}^t  \mathcal{H}'(t')\,dt'\rp.
\end{align*}
Suppose we are interested in an observable $\mathcal{O}$. By expanding $U(t,t_0)$ to second order in $\mathcal{H}'$, we can approximate the expectation value of $\mathcal{O}$ at time $t$:
\begin{align*}
\langle \mathcal{O}(t) \rangle_{\phi} = \langle \mathcal{O}(t) \rangle_{\phi = 0} + i \int_{-\infty}^t dt' \langle [\mathcal{H'}(t'), \mathcal{O} (t)] \rangle_{\phi=0} + \dots
\end{align*}
Here, the $\phi=0$ subscript denotes evaluation using $\ket{\psi(t_0)}$.  This result is known as Kubo's formula. 

Suppose that the perturbation Hamiltonian has the form $\mathcal{H}'(t) = \sum_j \phi_j (t) \mathcal{O}_j(t)$, with an implicit sum over $j$ and let $\delta \langle \mathcal{O}_i \rangle = \langle \mathcal{O}_i \rangle_\phi - \langle  \mathcal{O}_i \rangle_{\phi = 0}$ denote the deviation of the observable from its equilibrium value, we have
\begin{align*}
\delta \langle \mathcal{O}_i \rangle 
&=  i \int_{-\infty}^t dt' \langle  [ \mathcal{O}_j(t'), \mathcal{O}_i(t)   ] \rangle \phi_j(t') \\
&= i \int_{-\infty}^\infty dt' \Theta(t-t') \langle  [ \mathcal{O}_j(t'), \mathcal{O}_i(t)   ] \rangle \phi_j(t').
\end{align*} 
Comparing this to \eqref{eq:linear_response} and \eqref{eq:fluctuation_response_function}, we can identify (similar to what we did in Section \textcolor{purple}{blah}) that the response function for the system is a two-point correlation function:
\begin{align}\label{eq:corr_func}
\chi_{ij}(t-t') = -i \Theta(t-t') \langle  [ \mathcal{O}_j(t'), \mathcal{O}_i(t)   ] \rangle,
\end{align}
which is a more specific version of the Kubo's formula. By further assuming that the system is a canonical ensemble and going to Fourier space, the response function takes a more familiar form:
\begin{align*}
\widetilde{\chi}_{ij}(\omega) = \sum_{m,n} e^{-\be E_m} \lb 
\f{(\mathcal{O}_j)_{mn}(\mathcal{O}_i)_{mn}}{\omega - \omega_{nm} + i\epsilon} 
- 
\f{(\mathcal{O}_i)_{mn}(\mathcal{O}_j)_{mn}}{\omega + \omega_{nm} + i\epsilon} 
\rb.
\end{align*}
Formal properties of $\widetilde{\chi}$ can be studied from this representation. While they are beyond the scope of this review, the reader may refer to \textcolor{blue}{blah blah} for more details.  


\subsection{Measurements, correlation functions, and FDT}
While the Kubo formula seems rather abstract, we may recall that correlation functions are experimental observables. Consider the transition probability $S(\omega)$ which follows Fermi's golden rule:
\begin{align*}
\widetilde{S}(\omega) = \sum_{nm} P_m \abs{ \bra{\psi_n} \mathcal{O} \ket{\psi_m}  }^2  \delta(\omega - \omega_{nm}) 
\end{align*}
where $P_m$ is some equilibrium distribution to account for finite temperature. By Fourier transforming, we can relate $\widetilde{S}(\omega)$ back to a correlation function like in \eqref{eq:corr_func}:
\begin{align*}
S(\tau) = (2\pi)^{-1}\langle \mathcal{O}(\tau) \mathcal{O}^\dagger(0) \rangle. 
\end{align*}
In general, one can \textit{define} $\widetilde{S}(\omega)$, called the \textit{dynamical structure factor} as:
\begin{align*}
\widetilde{S}(\omega) 
&= \f{1}{2\pi}\int_{-\infty}^\infty \langle \mathcal{O}(t) \mathcal{O}^\dagger (0) \rangle e^{i \omega t }\,dt\\
&= \abs{\mathcal{O}_0}^2 \delta(\omega) + \f{1}{2\pi} \int_{-\infty}^\infty \langle \delta \mathcal{O}(t) \delta \mathcal{O}^\dagger (0) \rangle e^{i\omega t }\,dt,
\end{align*}
where $ \mathcal{O} = \mathcal{O}_0 + \delta \mathcal{O}$, and $\langle \delta \mathcal{O} \rangle = 0$. For $ \omega \neq 0$, we see that $\widetilde{S}(\omega)$ captures the dynamics of the fluctuations of $\mathcal{O}$. Assuming $P_m \sim e^{-\be \omega_m}$, one can prove the fluctuation-dissipation theorem:
\begin{align*}
\Im \widetilde{\chi}(\omega) = \pi[\widetilde{S}(\omega) - \widetilde{S}(-\omega)] = \pi (1 - e^{-\be \omega}) S(\omega),
\end{align*}
which relates fluctuations, captured by $\widetilde{S}(\omega)$, to dissipation in the system, captured by the imaginary part of the response function, just like in the damped harmonic oscillator example. 

We notice that while at zero temperature, $\Im \widetilde{\chi}(\omega)$ and $S(\omega)$ for all $\omega >0$, the two functions differ significantly for $T \neq 0$ with $\widetilde{S}(\omega)$ having stronger $T$-dependence. As a result, the dynamical structure factor is a more fundamental quantity for many-body theory. 



\section{Applications}

In this section, we review some applications of linear response theory in ultracold atom experiments. 
 
\subsection{Density response of a Bose gas}

Consider the density operator for a Bose gas $n(\mathbf{r}) = \sum_i \delta(\mathbf{r} - \mathbf{r}_i)$. The $\mathbf{q}$-component of this operator in Fourier space is $\widetilde{n}_\mathbf{q} = \sum_i e^{-i \mathbf{q}\cdot \mathbf{r}}$. From the formalism in the Section \textcolor{purple}{blah}, the density response function in Fourier space is
\begin{align*}
\chi(\mathbf{q},\omega) = 
\end{align*}
\subsubsection{Ideal Bose gas}

\subsubsection{Weakly-interacting Bose gas}


\subsection{RF spectroscopy of degenerate Fermi mixtures}


\textcolor{purple}{Talk about how linear response theory is applied to calculating spectroscopic lineshapes, etc.}

\subsection{Lattice modulation spectroscopy}




\section{Conclusions}



\begin{acknowledgments}
	The author would like to thanks Professor Martin Zwierlein for an exciting semester of AMO II. 
\end{acknowledgments}


\bibliographystyle{apsrev4-1}
% \bibliography{HuanBui_AMO_refs} 


\end{document}
