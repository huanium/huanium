\documentclass[reprint,
%superscriptaddress,
%groupedaddress,
%unsortedaddress,
%runinaddress,
%frontmatterverbose, 
%preprint,
%preprintnumbers,
nofootinbib,
%nobibnotes,
%bibnotes,
amsmath,amssymb,
aps]{revtex4-1}
\usepackage{physics}
\usepackage{graphicx}
\usepackage{amsmath}
\usepackage{subfigure}
\usepackage{bm}
\usepackage{dcolumn}% Align table columns on decimal point
\usepackage{framed}
\usepackage{empheq}
\usepackage{amsfonts}
\usepackage{esint}
\usepackage[makeroom]{cancel}
\usepackage{dsfont}
\usepackage{centernot}
\usepackage{mathtools}
\usepackage{bigints}
\usepackage{empheq}
\usepackage{tensor}
\usepackage{xcolor}
\definecolor{colby}{rgb}{0.0, 0.0, 0.5}
\definecolor{MIT}{RGB}{163, 31, 52}
\usepackage[pdftex]{hyperref}
%\hypersetup{colorlinks,linkcolor={MIT},citecolor={MIT},urlcolor={MIT}}  
\hypersetup{colorlinks,linkcolor=blue,citecolor=blue,urlcolor=blue}  
%\usepackage[left=1in,right=1in,top=1in,bottom=1in]{geometry}
%\setcounter{MaxMatrixCols}{20}
% \usepackage{newpxtext,newpxmath}
%\newcommand*\widefbox[1]{\fbox{\hspace{2em}#1\hspace{2em}}}

\newcommand{\p}{\partial}
\newcommand{\R}{\mathbb{R}}
\newcommand{\C}{\mathbb{C}}
\newcommand{\lag}{\mathcal{L}}
\newcommand{\nn}{\nonumber}
\newcommand{\ham}{\mathcal{H}}
\newcommand{\M}{\mathcal{M}}
\newcommand{\I}{\mathcal{I}}
\newcommand{\K}{\mathcal{K}}
\newcommand{\F}{\mathcal{F}}
\newcommand{\w}{\omega}
\newcommand{\lam}{\lambda}
\newcommand{\al}{\alpha}
\newcommand{\be}{\beta}
\newcommand{\x}{\xi}
\newcommand{\G}{\mathcal{G}}
\newcommand{\f}[2]{\frac{#1}{#2}}
\newcommand{\ift}{\infty}
\newcommand{\lp}{\left(}
\newcommand{\rp}{\right)}
\newcommand{\lb}{\left[}
\newcommand{\rb}{\right]}
\newcommand{\lc}{\left\{}
\newcommand{\rc}{\right\}}
\newcommand{\V}{\mathbf{V}}
\newcommand{\U}{\mathcal{U}}
\newcommand{\Id}{\mathcal{I}}
\newcommand{\D}{\mathcal{D}}
\newcommand{\Z}{\mathcal{Z}}

\definecolor{gris245}{RGB}{245,245,245}
\definecolor{olive}{RGB}{50,140,50}
\definecolor{brun}{RGB}{175,100,80}

\newcommand{\diag}{\text{diag}}
\newcommand{\psirot}{\ket{\psi_\text{rot}(t)} }
\newcommand{\RWA}{\ham_\text{rot}^\text{RWA}}

% 3j symbol
\newcommand{\tj}[6]{ \begin{pmatrix}
		#1 & #2 & #3 \\
		#4 & #5 & #6 
\end{pmatrix}}



\begin{document}
	
	

\title{Linear response theory and applications in many-body quantum physics}
\author{Huan Q. Bui}
\email{huanbui@mit.edu}
\affiliation{
	MIT-Harvard Center for Ultracold Atoms, Research Laboratory of Electronics, and Department of Physics, Massachusetts Institute of Technology, Cambridge, Massachusetts 02139, USA}
\date{\today}


\begin{abstract}
	In this paper, we...
\end{abstract}

\maketitle


\section{Introduction}
\textcolor{purple}{Talk about linear response theory. Significant because from it we can calculate how complex systems, in particular system of interacting quantum gases, respond to external perturbations which may or may not be time-dependent. What else can we get out of linear response theory? What is linear response theory useful? Because its results are things we can measure in the lab: correlation functions. Will talk about the relationship between measurements and correlations. Will talk about how from linear response theory one obtains sum rules which are related to correlation functions -- which again are things one can measure in the lab... }

The object of study in linear response theory is the linear response function. In the simplest case, the response function is simply a proportionality constant $\chi$ relating the response of a system $x(t)$ due to a perturbation $h(t)$ at that instant $t$. For linear, time-invariant systems with memory, the response $x(t)$ gets contributions due to past perturbations $h(t')$ and is given by the convolution of the response function with the perturbation:
\begin{align}\label{eq:linear_response}
x(t) = \int_{-\infty}^{\infty} \chi(t-t') h(t') \,dt'.
\end{align}
In general, $\chi$ solves the equation $ L \chi(t-t') = \delta(t-t')$ where $L$ is the linear differential operator associated with the system under consideration. It is identically the Green's function associated with $L$ and characterizes the system's response to an external impulse. 

As Section \textcolor{purple}{blah} will show, linear response theory provides a means for extracting rich physics. For the case of the damped harmonic oscillator, we can fully characterize its resonance behavior and energy dissipation based solely on the (complex) response function. Linear response theory also finds applications in statistical mechanics: When it is in statistical mechanics, we can understand and derive the famous fluctuation-dissipation theorem. In Section \textcolor{purple}{blah}, we will also see how linear response theory, when applied to quantum many-body systems, provides us with \textcolor{blue}{things to measure to find out about properties of system etc. I don't know how to phrase this yet.}



This paper is structured as follows: 

\textcolor{purple}{say more about what happens in the quantum case. Talk about this thing called the Kubo formula and dissipation-fluctuation theorem. Talk about the Kramer-Kronig relation. Talk about this thing called the structure factor.}

Talk about magnetic susceptibility, which is how much the magnetization change as a function of the applied/external magnetic field. Of course, talk about the trivial example of the damped (driven) harmonic oscillator. s

A non-trivial example is how does one study, say, sound in a BEC or a Fermi gas? Typically the setup is such that one has the Hamiltonian governing the dynamics of some quantum gas and then turn on perturbation: like shaking the trapping potential or stirring the BEC with a laser or something like that. How do we predict the response of the gas under such perturbations, as a function of parameters of the perturbation and as a function of other state variables? 



\section{Classical Linear Response}

%
\subsection{Causality}
The response function tells us how a system responds \textit{as a consequence} of some perturbation, which means that it respects causality, i.e., $\chi(t) = 0$ for all $t < t_0 = 0$ when the perturbation is turned on. Consider the Fourier expansion of $\chi(t)$:
\begin{align*}
\chi(t) = \int_{-\infty}^\infty  e^{-i\omega t } \tilde{\chi}(\omega) \,d\omega.
\end{align*}
For $t<0$, we can perform the integral by closing the contour in the upper half-plane. Since the answer is zero, it must be that $\chi(\omega)$ is analytic in the upper half-plane. In other words, 
\begin{align*}
\text{Causality} \implies \chi(\omega) \text{ analytic for } \Im(\chi) > 0.
\end{align*}
Let $\chi'(\omega) = \Re \chi(\omega)$ and $\chi''(\omega) = \Im \chi(\omega)$. $\chi'(\omega)$ is called the \textit{reactive part} of the response, and $\chi''(\omega)$ the \textit{dissipative part}. From the time-invariance of $\chi(t)$, one can show that $\chi'(\omega)$ and $\chi''(\omega)$ are even and odd functions, respectively.


If $\chi(\omega)$ decays faster than $1/\abs{\omega}$ as $\omega \to \infty$ in addition to being analytic in the upper half-plane, then $\chi'(\omega)$ and $\chi''(\omega)$ are related via the Kramer-Kronig relations. In particular, one can reconstruct the full complex response function knowing only its imaginary part:
\begin{align*}
\chi(\omega) = \int_{-\infty}^\infty \f{d\omega'}{\pi} \f{\Im \chi'(\omega')}{\omega' - \omega - i\epsilon}.
\end{align*}

%
\subsection{Example: Damped driven harmonic oscillator}
Consider the damped driven harmonic oscillator. Fourier-transforming the equation of motion and using  \eqref{eq:linear_response} give the response function in frequency domain:
\begin{align*}
\ddot{x} + \gamma \dot{x} + \omega_0^2 x = h(t) \implies 
\tilde{\chi}(\omega) = ( \omega_0^2 - \omega^2 - i \gamma \omega)^{-1}.
\end{align*}
The static response function is $\tilde{\chi}(\omega = 0) = 1/\omega_0^2$, giving $x = h/\omega_0^2$ as expected. For a drive with frequency $\omega$, the response function has poles $\omega_\star$ given by 
\begin{align*}
\omega_* = -i\gamma/2 \pm \sqrt{\omega_0^2  - \gamma^2/4}.
\end{align*}
When the oscillator is underdamped $(\gamma/2 < \omega_0)$, the poles have both real and imaginary parts and are in the lower half-plane. When the oscillator is overdamped $(\gamma/2 > \omega_0)$, the poles are on the negative imaginary axis. In both cases, $\tilde{\chi}(\omega)$ is analytic in the upper half-plane, consistent with causality. 

Suppose the drive is $h(t) = h_0 \Re(e^{-i\omega t})$ and that the system is in steady state. The average power dissipated by the system is the average power absorbed: 
\begin{align*}
\langle P \rangle = 2h_0^2 \omega \Im\chi(\omega) = 2h_0^2  \gamma \omega^2 [(\omega_0^2 - \omega^2)^2 + \gamma^2 \omega^2]^{-1},
\end{align*}
which has a Lorentzian line shape with FWHM $\gamma$ near resonance. Note that the first equality holds in general and depends only on the imaginary part of the response function. As we will see, there is a deeper reason for the dependence on $\Im \tilde{\chi}(\omega)$.

\subsection{Fluctuation-Dissipation Theorem}
Let us apply the idea of linear response theory to a statistical ensemble that is slightly perturbed out of equilibrium. Suppose the unperturbed Hamiltonian is $\mathcal{H}$ and consider some dynamical variable $A$, which is a function defined over the phase space for the system. The equilibrium average for $A$ is:
\begin{align*}
\langle A \rangle =  \mathcal{Z}^{-1}(\mathcal{H}) \int e^{-\be \mathcal{H}} A\, d\Omega,
\end{align*}
where $\mathcal{Z}(\mathcal{H}) = \int e^{-\be \mathcal{H}} \, d\Omega$, and $\int \cdot \,d\Omega$ is an integral over phase space. Next, suppose the system is initially perturbed by $\Delta \mathcal{H} = -f A$ where $f$ is some external field. We are interested in seeing what happens when the perturbation is turned off. The initial ensemble average for $A$ is thus 
\begin{align*}
\overline{A}(0) = \mathcal{Z}^{-1}(\mathcal{H} + \Delta \mathcal{H})  \int e^{-\be (\mathcal{H} + \Delta \mathcal{H})} A \, d\Omega 
\end{align*}
For $t>0$, $\Delta \mathcal{H}$ vanishes, so $A(0)$ evolves into $A(t)$ by $\mathcal{H}$. However, we are still interested in how $\overline{A}(t)$, the ensemble average of $A(t)$ under the non-equilibrium distribution due to $\Delta \mathcal{H}$:
\begin{align*}
\overline{A}(t) = \mathcal{Z}^{-1}(\mathcal{H} + \Delta \mathcal{H})  \int e^{-\be(\mathcal{H} + \Delta \mathcal{H})} A(t) \, d\Omega. 
\end{align*}
Assuming that the perturbations are small, i.e. $A(t)$ does not deviate much from $\angle A \rangle$, we can find $\overline{A}(t)$ in terms of $A$ with corrections due to $\Delta \mathcal{H}$:
\begin{align*}
\overline{A}(t) 
%&= \f{\int e^{-\be \mathcal{H} (1- \be \Delta \mathcal{H} + \dots )A(t) \,d\Omega}}{\int e^{-\be \mathcal{H} (1 - \be \Delta \mathcal{H}) + \dots } \, d\Omega } \\ 
&\approx \mathcal{Z}^{-1}(\mathcal{H})  \int d\Omega \, e^{-\be \mathcal{H}} A(t)   \lb 1 - \be \Delta \mathcal{H}  + \langle  \be \Delta \mathcal{H}  \rangle  \rb \\ 
&= \langle A \rangle - \be \lb \langle \Delta \mathcal{H} A(t) \rangle - \langle A \rangle \langle \Delta \mathcal{H} \rangle  \rb.
\end{align*}
Finally, we insert the expression for $\mathcal{H}$ into the result above to find:
\begin{align*}
\Delta \overline{A}(t) = \be f \langle \delta A(t) \delta A(0) \rangle,
\end{align*}
where $\Delta \overline{A}(t) = \overline{A}(t) - \langle A \rangle$ and $\delta A(t) = A(t) - \langle A \rangle$ are the spontaneous and instantaneous fluctuation at time $t$, the former being an averaged, thus macroscopic quantity, while the latter microscopic. This is a remarkable result that we just showed: the spontaneous fluctuations is related to the time correlation function of instantaneous fluctuations. Since generally, $\langle \delta A \rangle = 0$ but $\langle A^2 \rangle \neq 0$, we see that $\Delta \overline{A}(t)$ decays in time. This is the essence of Onsanger's remarkable \textit{regression hypothesis} which was later realized to be a consequence of the fluctuation-dissipation theorem. 

If we take an extra step and express the fluctuations in the spontaneous response in terms of external perturbations, then we get the fluctuation dissipation theorem...

\section{Quantum Linear Response}





\subsection{The Kubo formula}

\subsection{Structure factor}

\subsection{Fluctuation-Dissipation Theorem}

\section{From linear response function to measurements}

\subsection{From linear response functions to correlation functions}

\subsection{From correlation functions to measurements}


\section{Applications}

\section{Conclusions}

\begin{acknowledgments}
	The author would like 
\end{acknowledgments}


\bibliographystyle{apsrev4-1}
% \bibliography{HuanBui_AMO_refs} 


\end{document}
