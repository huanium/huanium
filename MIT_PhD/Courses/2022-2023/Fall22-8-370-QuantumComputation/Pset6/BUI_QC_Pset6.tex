\documentclass{article}
\usepackage{physics}
\usepackage{graphicx}
\usepackage{caption}
\usepackage{amsmath}
\usepackage{bm}
\usepackage{framed}
\usepackage{authblk}
\usepackage{empheq}
\usepackage{amsfonts}
\usepackage{esint}
\usepackage[makeroom]{cancel}
\usepackage{dsfont}
\usepackage{centernot}
\usepackage{mathtools}
\usepackage{subcaption}
\usepackage{bigints}
\usepackage{amsthm}
\theoremstyle{definition}
\newtheorem{lemma}{Lemma}
\newtheorem{defn}{Definition}[section]
\newtheorem{prop}{Proposition}[section]
\newtheorem{rmk}{Remark}[section]
\newtheorem{thm}{Theorem}[section]
\newtheorem{exmp}{Example}[section]
\newtheorem{prob}{Problem}[section]
\newtheorem{sln}{Solution}[section]
\newtheorem*{prob*}{Problem}
\newtheorem{exer}{Exercise}[section]
\newtheorem*{exer*}{Exercise}
\newtheorem*{sln*}{Solution}
\usepackage{empheq}
\usepackage{tensor}
\usepackage{xcolor}
%\definecolor{colby}{rgb}{0.0, 0.0, 0.5}
\definecolor{MIT}{RGB}{163, 31, 52}
\usepackage[pdftex]{hyperref}
%\hypersetup{colorlinks,urlcolor=colby}
\hypersetup{colorlinks,linkcolor={MIT},citecolor={MIT},urlcolor={MIT}}  
\usepackage[left=1in,right=1in,top=1in,bottom=1in]{geometry}

\usepackage{newpxtext,newpxmath}
\newcommand*\widefbox[1]{\fbox{\hspace{2em}#1\hspace{2em}}}

\newcommand{\p}{\partial}
\newcommand{\R}{\mathbb{R}}
\newcommand{\C}{\mathbb{C}}
\newcommand{\lag}{\mathcal{L}}
\newcommand{\nn}{\nonumber}
\newcommand{\ham}{\mathcal{H}}
\newcommand{\M}{\mathcal{M}}
\newcommand{\I}{\mathcal{I}}
\newcommand{\K}{\mathcal{K}}
\newcommand{\F}{\mathcal{F}}
\newcommand{\w}{\omega}
\newcommand{\lam}{\lambda}
\newcommand{\al}{\alpha}
\newcommand{\be}{\beta}
\newcommand{\x}{\xi}

\newcommand{\G}{\mathcal{G}}

\newcommand{\f}[2]{\frac{#1}{#2}}

\newcommand{\ift}{\infty}

\newcommand{\lp}{\left(}
\newcommand{\rp}{\right)}

\newcommand{\lb}{\left[}
\newcommand{\rb}{\right]}

\newcommand{\lc}{\left\{}
\newcommand{\rc}{\right\}}


\newcommand{\V}{\mathbf{V}}
\newcommand{\U}{\mathcal{U}}
\newcommand{\Id}{\mathcal{I}}
\newcommand{\D}{\mathcal{D}}
\newcommand{\Z}{\mathcal{Z}}

%\setcounter{chapter}{-1}


\usepackage{enumitem}



\usepackage{listings}
\captionsetup[lstlisting]{margin=0cm,format=hang,font=small,format=plain,labelfont={bf,up},textfont={it}}
\renewcommand*{\lstlistingname}{Code \textcolor{violet}{\textsl{Mathematica}}}
\definecolor{gris245}{RGB}{245,245,245}
\definecolor{olive}{RGB}{50,140,50}
\definecolor{brun}{RGB}{175,100,80}

%\hypersetup{colorlinks,urlcolor=colby}
\lstset{
	tabsize=4,
	frame=single,
	language=mathematica,
	basicstyle=\scriptsize\ttfamily,
	keywordstyle=\color{black},
	backgroundcolor=\color{gris245},
	commentstyle=\color{gray},
	showstringspaces=false,
	emph={
		r1,
		r2,
		epsilon,epsilon_,
		Newton,Newton_
	},emphstyle={\color{olive}},
	emph={[2]
		L,
		CouleurCourbe,
		PotentielEffectif,
		IdCourbe,
		Courbe
	},emphstyle={[2]\color{blue}},
	emph={[3]r,r_,n,n_},emphstyle={[3]\color{magenta}}
}






\begin{document}
\begin{framed}
\noindent Name: \textbf{Huan Q. Bui}\\
Course: \textbf{8.370 - QC}\\
Problem set: \textbf{\#6}\\
Due: Wednesday, Nov 2, 2022\\
Collaborators/References: 
\end{framed}


\noindent \textbf{1. Pauli group, Clifford group} 

\begin{enumerate}[label=(\alph*)]
	\item The Clifford group is 
	\begin{align*}
		\mathcal{C} \equiv \{ \text{unitaries } U : U g U^\dagger \in \mathcal{P}\, \forall g \in \mathcal{P}     \}
	\end{align*}
where $\mathcal{P}$ is the Pauli group. Suppose $A,B\in \mathcal{C}$. Then for any $g\in \mathcal{P}$:
\begin{align*}
	AB g (AB)^\dagger =  A \underbrace{ (BgB^\dagger)}_{\in \mathcal{P}} A^\dagger= A g' A^\dagger \in \mathcal{P}.
\end{align*}
So $AB\in \mathcal{C}$. Moreover, if $A\in \mathcal{C}$, then for any $g\in \mathcal{P}$
\begin{align*}
	A^\dagger g A = (A g^\dagger A^\dagger)^\dagger = (  A g A^\dagger)^\dagger \in \mathcal{P}.
\end{align*}
So every element of $\mathcal{C}$ has an inverse. The identity element is simply the identity matrix. Associativity is inherited from associativity of matrix multiplication. So, $\mathcal{C}$ is a group. 
	
	\item The Pauli group for 1 qubit is given by 
	\begin{align*}
		\mathcal{P}_1 = \{ \pm I, \pm iI\,\pm X, \pm iX, \pm Y, \pm i Y, \pm Z, \pm iZ \}.
	\end{align*}
	The Pauli group for $n$ qubits is simply generated, via the tensor product, by the operators in $\mathcal{G}_1$, which are generated by the Pauli matrices $X,Y,Z$. $H$ is unitary. So, it remains to check that $H$ is in $\mathcal{C}_1$. Using the fact that $H^\dagger = H^{-1} = H$, we can simply check that $HXH, HYH, HZH \in \mathcal{P}_2$:
	\begin{align*}
		&HXH = Z \in \mathcal{P}_1 \\
		&HZH = X \in \mathcal{P}_1 \\
		&HYH = -Y \in \mathcal{P}_1
	\end{align*}
	And we're done!
	
	\item To check that the $CNOT$ gate is in $\mathcal{C}$, we check that it is in $\mathcal{C}_2$. $CNOT$ is unitary, so we check that $CNOT \, g \, CNOT^\dagger$ is in $\mathcal{P}_2$ for all generators $g$ of $\mathcal{P}_2$. The generators of $\mathcal{P}_2$ are once again $X,Y,Z$ tensored with the identity matrix either on the first or second qubit. This means there are six cases:
	 \begin{align*}
	 	&CNOT\,X \otimes \Id \,CNOT^\dagger = X\otimes X \in \mathcal{P}_2 \\
	 	&CNOT\,Y \otimes \Id \,CNOT^\dagger = Y\otimes X \in \mathcal{P}_2 \\
	 	&CNOT\,Z \otimes \Id \,CNOT^\dagger = Z\otimes \Id \in \mathcal{P}_2 \\
	 	&CNOT\,\Id \otimes X \,CNOT^\dagger = \Id \otimes X \in \mathcal{P}_2 \\
	 	&CNOT\,\Id \otimes Y \,CNOT^\dagger = Z \otimes Y \in \mathcal{P}_2 \\
	 	&CNOT\,\Id \otimes Z \,CNOT^\dagger = Z \otimes Z \in \mathcal{P}_2 \quad\quad \checkmark
	 \end{align*}
 	Here, the $CNOT$ gate in consideration is one where the first qubit is the control. However, since the other CNOT gate only differs on on which qubits it uses as control and target, we only need to check one of the two CNOTs.
	
	\item We want to check that $T\notin \mathcal{C}_1$. Consider $g = X\in \mathcal{P}_1$. 
	\begin{align*}
		TXT^\dagger = \begin{pmatrix}
			0 & e^{-i\pi4} \\ e^{i\pi/4} & 0
		\end{pmatrix} \notin \mathcal{P}_1.
	\end{align*}
	So $T$ is not in the Clifford group.
\end{enumerate}



\noindent \textbf{2. Gotta erase workbits!}\\




\noindent \textbf{3. Simon's algorithm}


\noindent \textbf{4. Partial transpose}

\begin{enumerate}[label=(\alph*)]
	\item Suppose $M$ is separable, i.e., 
	\begin{align*}
		M = \sum_i \lambda_i \ket{v_i} \bra{v_i} \otimes \ket{w_i}\bra{w_i}
	\end{align*}
	where $\lambda_i$'s are positive. Then the partial transpose of $M$ according to the definition in the problem is
	\begin{align*}
		pt(M) = \sum_i \lambda_i (\ket{v_i}\bra{v_i})^\top \otimes \ket{w_i}\bra{w_i}.
	\end{align*} 
	Since  $\Pi_i = \ket{v_i}\bra{v_i}$ are orthogonal projections, the matrices $(\ket{v_i}\bra{v_i})^\top$ are also orthogonal projections. We may very well consider the transposition as a unitary change of basis in the first qubit and write
	\begin{align*}
		pt(M) = \sum_i \lambda_i \ket{v_i'}\bra{v_i'} \otimes \ket{w_i}\bra{w_i}.
	\end{align*}
	It is clear that the spectrum of $pt(M)$ is exactly the same as that of $M$ in this case, so $pt(M)$ must also be positive. However, we could also be explicit: Let $x = \sum_{i,j} c_{ij} \ket{v_i'}\ket{w_j}$. Then
	\begin{align*}
		\bra{x}  \, pt(M)\, \ket{x} = \sum_i \lambda_i |c_{ii}|^2 \geq 0.
	\end{align*}
	And we're done. 
	
	\item The density matrix for $\ket{\psi} = (1/\sqrt{2})(\ket{00} + \ket{11})$ is 
	\begin{align*}
		\rho = \ket{\psi} \bra{\psi} = \f{1}{2}\begin{pmatrix}
			1 & 0 & 0 & 1 \\
			0 & 0 & 0 & 0 \\
			0 & 0 & 0 & 0 \\
			1 & 0 & 0 & 1
		\end{pmatrix}
	\end{align*}
	From here we find 
	\begin{align*}
		pt(\rho) =  \f{1}{2}\begin{pmatrix}
			1 & 0 & 0 & 0 \\
			0 & 0 & 1 & 0 \\
			0 & 1 & 0 & 0 \\
			0 & 0 & 0 & 1
		\end{pmatrix}.
	\end{align*}
The eigenvalues of this matrix are $-1/2,1/2,1/2,1/2$, so $pt(\rho)$ is not non-negative which implies that $\rho$ is not separable in view of Part (a).
\end{enumerate}


\noindent \textbf{5. Teleporting a qutrit directly}\\

\noindent Instead of embedding the qutrit in a set of qubits of higher dimensions, we can teleport qutrits directly. Let $\omega = e^{2\pi i /3}$, the cube root of unity. Define 
\begin{align*}
	P = \begin{pmatrix}
		1 & 0 & 0 \\ 0 & \omega & 0 \\ 0 & 0 & \omega^2  
	\end{pmatrix} \quad \quad \text{and} \quad\quad 
	T = \begin{pmatrix}
		0 & 1 & 0 \\ 0 & 0 & 1 \\ 1 & 0 & 0
	\end{pmatrix}.
\end{align*}
The analogue of the EPR pair is the state
\begin{align*}
	\ket{EPR_3} = \f{1}{\sqrt{3}}(\ket{00} + \ket{11} + \ket{22})
\end{align*}
and the analogue of the Pauli matrices are the nine matries $P^aT^b$, with $0 \leq a,b< 3$.  We teleport the qutrit as follows:\\

\noindent Alice has some qutrit in state $\ket{\psi} = \al \ket{0} + \be \ket{1} + \gamma \ket{2}$. She now measures her two qutrits (the qutrit in state $\ket{\psi}$ and her half of the $EPR_3$ pair) in the EPR-pair basis which consists of nine states in the form
\begin{align*}
	(\Id \otimes P^a T^b)(\ket{00} + \ket{11} + \ket{22}).
\end{align*}
Alice will then send her measurement result (one of the nine possible ones) to Bob, and Bob will apply one of nine unitaries to his qutrit to obtain $\psi$. In particular, if Alice sees a state associated with $P^a T^b$, then Bob applies $P^a T^b$ to his qutrit. Below we will show explicitly.\\

Here we show the nine-element basis for Alice's measurement, the resulting state on Bob's qutrit, and Bob's corrective unitary for each case:
\begin{align*}
	(1/\sqrt{3})(\Id \otimes P^0 T^0)(\ket{00} + \ket{11} + \ket{22}) &\to \ket{B} = (1/\sqrt{3})\Id(\al\ket{0} + \be\ket{1} + \gamma\ket{2}) \to P^0T^0 \\
	(1/\sqrt{3})(\Id \otimes P^1 T^0)(\ket{00} + \ket{11} + \ket{22}) &\to \ket{B} = (1/\sqrt{3}) P^2(\al\ket{0} + \be\ket{1} + \gamma\ket{2}) \to P^1 T^0\\
	(1/\sqrt{3})(\Id \otimes P^2 T^0)(\ket{00} + \ket{11} + \ket{22}) &\to \ket{B} = (1/\sqrt{3})P^1(\al\ket{0} + \be\ket{1} + \gamma\ket{2}) \to P^2 T^0\\
	(1/\sqrt{3})(\Id \otimes P^0 T^1)(\ket{00} + \ket{11} + \ket{22}) &\to \ket{B} = (1/\sqrt{3})T^2(\al\ket{0} + \be\ket{1} + \gamma\ket{2}) \to P^0 T^1\\
	(1/\sqrt{3})(\Id \otimes P^1 T^1)(\ket{00} + \ket{11} + \ket{22}) &\to \ket{B} = (1/\sqrt{3})P^2  T^2(\al\ket{0} + \be\ket{1} + \gamma\ket{2}) \to P^1T^1\\
	(1/\sqrt{3})(\Id \otimes P^2 T^1)(\ket{00} + \ket{11} + \ket{22}) &\to \ket{B} = (1/\sqrt{3}) P T^2(\al\ket{0} + \be\ket{1} + \gamma\ket{2}) \to P^2 T^1\\
	(1/\sqrt{3})(\Id \otimes P^0 T^2)(\ket{00} + \ket{11} + \ket{22}) &\to \ket{B} = (1/\sqrt{3}) T(\al\ket{0} + \be\ket{1} + \gamma\ket{2}) \to P^0 T^2\\
	(1/\sqrt{3})(\Id \otimes P^1 T^2)(\ket{00} + \ket{11} + \ket{22}) &\to \ket{B} = (1/\sqrt{3}) P^2 T (\al\ket{0} + \be\ket{1} + \gamma\ket{2}) \to P^1 T^2\\
	(1/\sqrt{3})(\Id \otimes P^2 T^2)(\ket{00} + \ket{11} + \ket{22}) &\to \ket{B} = (1/\sqrt{3}) P T(\al\ket{0} + \be\ket{1} + \gamma\ket{2}) \to P^2 T^2.
\end{align*}




\end{document}











