\documentclass{article}
\usepackage{physics}
\usepackage{graphicx}
\usepackage{caption}
\usepackage{amsmath}
\usepackage{bm}
\usepackage{framed}
\usepackage{authblk}
\usepackage{empheq}
\usepackage{amsfonts}
\usepackage{esint}
\usepackage[makeroom]{cancel}
\usepackage{dsfont}
\usepackage{centernot}
\usepackage{mathtools}
\usepackage{subcaption}
\usepackage{bigints}
\usepackage{amsthm}
\theoremstyle{definition}
\newtheorem{lemma}{Lemma}
\newtheorem{defn}{Definition}[section]
\newtheorem{prop}{Proposition}[section]
\newtheorem{rmk}{Remark}[section]
\newtheorem{thm}{Theorem}[section]
\newtheorem{exmp}{Example}[section]
\newtheorem{prob}{Problem}[section]
\newtheorem{sln}{Solution}[section]
\newtheorem*{prob*}{Problem}
\newtheorem{exer}{Exercise}[section]
\newtheorem*{exer*}{Exercise}
\newtheorem*{sln*}{Solution}
\usepackage{empheq}
\usepackage{tensor}
\usepackage{xcolor}
%\definecolor{colby}{rgb}{0.0, 0.0, 0.5}
\definecolor{MIT}{RGB}{163, 31, 52}
\usepackage[pdftex]{hyperref}
%\hypersetup{colorlinks,urlcolor=colby}
\hypersetup{colorlinks,linkcolor={MIT},citecolor={MIT},urlcolor={MIT}}  
\usepackage[left=1in,right=1in,top=1in,bottom=1in]{geometry}

\usepackage{newpxtext,newpxmath}
\newcommand*\widefbox[1]{\fbox{\hspace{2em}#1\hspace{2em}}}

\newcommand{\p}{\partial}
\newcommand{\R}{\mathbb{R}}
\newcommand{\C}{\mathbb{C}}
\newcommand{\lag}{\mathcal{L}}
\newcommand{\nn}{\nonumber}
\newcommand{\ham}{\mathcal{H}}
\newcommand{\M}{\mathcal{M}}
\newcommand{\I}{\mathcal{I}}
\newcommand{\K}{\mathcal{K}}
\newcommand{\F}{\mathcal{F}}
\newcommand{\w}{\omega}
\newcommand{\lam}{\lambda}
\newcommand{\al}{\alpha}
\newcommand{\be}{\beta}
\newcommand{\x}{\xi}

\newcommand{\G}{\mathcal{G}}

\newcommand{\f}[2]{\frac{#1}{#2}}

\newcommand{\ift}{\infty}

\newcommand{\lp}{\left(}
\newcommand{\rp}{\right)}

\newcommand{\lb}{\left[}
\newcommand{\rb}{\right]}

\newcommand{\lc}{\left\{}
\newcommand{\rc}{\right\}}


\newcommand{\V}{\mathbf{V}}
\newcommand{\U}{\mathcal{U}}
\newcommand{\Id}{\mathcal{I}}
\newcommand{\D}{\mathcal{D}}
\newcommand{\Z}{\mathcal{Z}}

%\setcounter{chapter}{-1}


\usepackage{enumitem}



\usepackage{listings}
\captionsetup[lstlisting]{margin=0cm,format=hang,font=small,format=plain,labelfont={bf,up},textfont={it}}
\renewcommand*{\lstlistingname}{Code \textcolor{violet}{\textsl{Mathematica}}}
\definecolor{gris245}{RGB}{245,245,245}
\definecolor{olive}{RGB}{50,140,50}
\definecolor{brun}{RGB}{175,100,80}

%\hypersetup{colorlinks,urlcolor=colby}
\lstset{
	tabsize=4,
	frame=single,
	language=mathematica,
	basicstyle=\scriptsize\ttfamily,
	keywordstyle=\color{black},
	backgroundcolor=\color{gris245},
	commentstyle=\color{gray},
	showstringspaces=false,
	emph={
		r1,
		r2,
		epsilon,epsilon_,
		Newton,Newton_
	},emphstyle={\color{olive}},
	emph={[2]
		L,
		CouleurCourbe,
		PotentielEffectif,
		IdCourbe,
		Courbe
	},emphstyle={[2]\color{blue}},
	emph={[3]r,r_,n,n_},emphstyle={[3]\color{magenta}}
}






\begin{document}
\begin{framed}
\noindent Name: \textbf{Huan Q. Bui}\\
Course: \textbf{8.370 - QC}\\
Problem set: \textbf{\#8}\\
Due: Wednesday, Nov 16, 2022\\
Collaborators/References: Piazza
\end{framed}



\noindent \textbf{1. }  A generator $g$ of the multiplicative group modulo $P$ is a number such that $g^{P -1} =
1 \mod P$, but $gk \neq 1 \mod P$ for any $1 < k < P -1$. As far as I know, we
	know of no classical algorithms, even probabilistic ones, for testing whether $g$ is
	a generator mod $P$.\\

\noindent While not explicit, the phrasing of the problem implies that $P$ is prime: since the order of a generator $g$ of $(\mathbb{Z}/P\mathbb{Z})^\times$ is the order of the group, $P-1$, we must have $\phi(P) = P-1$, which is true in general if $P$ is prime. Now, I'm not sure why we have to use the discrete log algorithm here, especially since we do not know of \textit{a} generator $g$ for the multiplicative group modulo $P$  to begin with. Instead, given some element $h\in (\mathbb{Z}/P\mathbb{Z})^\times$, we can use the \textbf{period-finding algorithm} to efficiently compute the order $r$ of $h$. Once done, we simply compare $r$ to the order of  $(\mathbb{Z}/P\mathbb{Z})^\times$, which is $\phi(P) = P-1$. If $r = P-1$ then $h$ is a generator of $(\mathbb{Z}/P\mathbb{Z})^\times$. Otherwise, $h$ is not a generator of  $(\mathbb{Z}/P\mathbb{Z})^\times$.\\

\noindent Okay but what if we absolutely have to use the discrete logarithm algorithm? \\

\noindent \textbf{2. The Principle of Deferred Measurement} \\

\noindent Suppose the state of the system after the first set of unitaries is 
\begin{align*}
	\ket{\Psi} = \ket{0}_1 \ket{\al_0}_2 \ket{\psi_0} + \ket{1}_1 \ket{\al_1}_2\ket{\psi_1}.
\end{align*}
Then after the measurement and possibly the unitary gate $U$ on the second qubit, the state of the system is 
\begin{align*}
	\ket{\Psi'} = \ket{j}_1 U^j \ket{\al_j}_2 \ket{\psi_j}
\end{align*}
where $j\in \{0,1\}$ is the measurement outcome. After the last set of unitaries $\mathcal{U}$, the state of the system is 
\begin{align*}
	\ket{\Psi''} = \ket{j}_1 \mathcal{U} U^j \ket{\al_j}_2 \ket{\psi_j}.
\end{align*}
If the measurement outcome is $j=0$ then we have
\begin{align*}
	\ket{\Psi''}_{j=0} =  \ket{0}_1 \mathcal{U} \ket{\al_0}_2 \ket{\psi_0}.
\end{align*}
Else if $j=1$:
\begin{align*}
	\ket{\Psi''}_{j=1} =  \ket{1}_1 \mathcal{U} U \ket{\al_1}_2 \ket{\psi_1}.
\end{align*}

In the second case, the state of the system after first state of unitaries is the same as before. So we look at the  system after the controlled-unitary $U$:
\begin{align*}
	\ket{\Phi'} =  \ket{0}_1 \ket{\al_0}_2 \ket{\psi_0} + \ket{1}_1 U \ket{\al_1}_2\ket{\psi_1}.
\end{align*}
Now we apply the second set of unitaries $\mathcal{U}$. By linearity we have
\begin{align*}
		\ket{\Phi''} = \ket{0}_1 \mathcal{U} \ket{\al_0}_2 \ket{\psi_0} + \ket{1}_1 \mathcal{U} U \ket{\al_1}_2\ket{\psi_1}
\end{align*}
Now we measure the first qubit. Let the measurement outcome be $j$, then the state of the system is 
\begin{align*}
	\ket{\Phi''}_{j} = \ket{0}_1 \mathcal{U} \ket{\al_0}_2 \ket{\psi_0} \delta_{j,0} + \ket{1}_1 \mathcal{U} U \ket{\al_1}_2\ket{\psi_1} \delta_{j,1}.
\end{align*}
In particular, if $j=0$ then 
\begin{align*}
		\ket{\Phi''}_{j=0} = \ket{0}_1 \mathcal{U} \ket{\al_0}_2 \ket{\psi_0}
\end{align*}
Else if $j=1$ then
\begin{align*}
	\ket{\Phi''}_{j=1} = \ket{1}_1 \mathcal{U} U \ket{\al_1}_2 \ket{\psi_1}
\end{align*}
which is exactly what we have before. 

\newpage

\noindent \textbf{3. Impatient runner of Grover's algorithm...}\\

\noindent Let $k$ be such that $K < k < 2K$. And let $S$ denotes the space of solutions. By definition, $|S| = M$. The state of the computer after $k$ Grover iterations is
\begin{align*}
	G^k \ket{\psi} = \cos\lp \f{2k+1}{2}\theta \rp \ket{\al} + \sin\lp \f{2k+1}{2}\theta \rp \ket{\be}.
\end{align*}
Here
\begin{align*}
	\ket{\al} = \f{1}{\sqrt{N-M}} \sum_{x \,\notin\, S} \ket{x} 
	\quad \text{and} \quad
	\ket{\be} = \f{1}{\sqrt{M}} \sum_{x \,\in\, S } \ket{x} 
	\quad \text{and} \quad \cos\f{\theta}{2} = \sqrt{\f{N-M}{N}}
\end{align*}
Now, the impatient experimenter checks whether this state is in a solution state by measuring in the $\ket{\al}, \ket{\be}$ basis. The probability that he finds the state to be in the solution space is 
\begin{align*}
	p = \sin^2 \lp \f{2k+1}{2} \theta \rp.
\end{align*}
If he finds that the state is in the solution space then he stops. Else, he repeats the process. Under the assumption that $M\ll N$ and the additional assumption that $K\theta \ll \pi/2$, we make small-angle approximation on $p$ to obtain 
\begin{align*}
	p \approx \lp \f{2k+1}{2}\theta \rp^2.
\end{align*}
Now $\theta = 2\acos\lp \sqrt{(N-M)/N} \rp$. Since $M\ll N$, we can expand this in terms of (small) $M/N$ to get
\begin{align*}
	\theta \sim \sqrt{\f{M}{N}} + \f{(M/N)^{3/2}}{6} + \mathcal{O}((M/N)^{5/2})
\end{align*}
With this, we have
\begin{align*}
	p \approx \lp \f{2k+1}{2} \rp^2 \f{M}{N}.
\end{align*}

Suppose the experimenter measures $j-1$ failures before seeing the first success at the $j$th measurement. Then the probability for this event is 
\begin{align*}
	\Pr(X=j) = (1-p)^{j-1}p
\end{align*}
The expected value for the random variable $X$, which is the expected number of trials before the first success, is thus
\begin{align*}
	E[X] = \f{1}{p}  = \f{N}{M} \lp \f{2}{2k+1} \rp ^2.
\end{align*}
Since $K < k < 2K$, the experimenter has to try $\boxed{\mathcal{O}(N/MK^2)}$ times. 


\end{document}











