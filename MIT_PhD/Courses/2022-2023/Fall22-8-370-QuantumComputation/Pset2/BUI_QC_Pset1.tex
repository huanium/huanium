\documentclass{article}
\usepackage{physics}
\usepackage{graphicx}
\usepackage{caption}
\usepackage{amsmath}
\usepackage{bm}
\usepackage{framed}
\usepackage{authblk}
\usepackage{empheq}
\usepackage{amsfonts}
\usepackage{esint}
\usepackage[makeroom]{cancel}
\usepackage{dsfont}
\usepackage{centernot}
\usepackage{mathtools}
\usepackage{subcaption}
\usepackage{bigints}
\usepackage{amsthm}
\theoremstyle{definition}
\newtheorem{lemma}{Lemma}
\newtheorem{defn}{Definition}[section]
\newtheorem{prop}{Proposition}[section]
\newtheorem{rmk}{Remark}[section]
\newtheorem{thm}{Theorem}[section]
\newtheorem{exmp}{Example}[section]
\newtheorem{prob}{Problem}[section]
\newtheorem{sln}{Solution}[section]
\newtheorem*{prob*}{Problem}
\newtheorem{exer}{Exercise}[section]
\newtheorem*{exer*}{Exercise}
\newtheorem*{sln*}{Solution}
\usepackage{empheq}
\usepackage{tensor}
\usepackage{xcolor}
%\definecolor{colby}{rgb}{0.0, 0.0, 0.5}
\definecolor{MIT}{RGB}{163, 31, 52}
\usepackage[pdftex]{hyperref}
%\hypersetup{colorlinks,urlcolor=colby}
\hypersetup{colorlinks,linkcolor={MIT},citecolor={MIT},urlcolor={MIT}}  
\usepackage[left=1in,right=1in,top=1in,bottom=1in]{geometry}

\usepackage{newpxtext,newpxmath}
\newcommand*\widefbox[1]{\fbox{\hspace{2em}#1\hspace{2em}}}

\newcommand{\p}{\partial}
\newcommand{\R}{\mathbb{R}}
\newcommand{\C}{\mathbb{C}}
\newcommand{\lag}{\mathcal{L}}
\newcommand{\nn}{\nonumber}
\newcommand{\ham}{\mathcal{H}}
\newcommand{\M}{\mathcal{M}}
\newcommand{\I}{\mathcal{I}}
\newcommand{\K}{\mathcal{K}}
\newcommand{\F}{\mathcal{F}}
\newcommand{\w}{\omega}
\newcommand{\lam}{\lambda}
\newcommand{\al}{\alpha}
\newcommand{\be}{\beta}
\newcommand{\x}{\xi}

\newcommand{\G}{\mathcal{G}}

\newcommand{\f}[2]{\frac{#1}{#2}}

\newcommand{\ift}{\infty}

\newcommand{\lp}{\left(}
\newcommand{\rp}{\right)}

\newcommand{\lb}{\left[}
\newcommand{\rb}{\right]}

\newcommand{\lc}{\left\{}
\newcommand{\rc}{\right\}}


\newcommand{\V}{\mathbf{V}}
\newcommand{\U}{\mathcal{U}}
\newcommand{\Id}{\mathcal{I}}
\newcommand{\D}{\mathcal{D}}
\newcommand{\Z}{\mathcal{Z}}

%\setcounter{chapter}{-1}


\usepackage{enumitem}



\usepackage{listings}
\captionsetup[lstlisting]{margin=0cm,format=hang,font=small,format=plain,labelfont={bf,up},textfont={it}}
\renewcommand*{\lstlistingname}{Code \textcolor{violet}{\textsl{Mathematica}}}
\definecolor{gris245}{RGB}{245,245,245}
\definecolor{olive}{RGB}{50,140,50}
\definecolor{brun}{RGB}{175,100,80}

%\hypersetup{colorlinks,urlcolor=colby}
\lstset{
	tabsize=4,
	frame=single,
	language=mathematica,
	basicstyle=\scriptsize\ttfamily,
	keywordstyle=\color{black},
	backgroundcolor=\color{gris245},
	commentstyle=\color{gray},
	showstringspaces=false,
	emph={
		r1,
		r2,
		epsilon,epsilon_,
		Newton,Newton_
	},emphstyle={\color{olive}},
	emph={[2]
		L,
		CouleurCourbe,
		PotentielEffectif,
		IdCourbe,
		Courbe
	},emphstyle={[2]\color{blue}},
	emph={[3]r,r_,n,n_},emphstyle={[3]\color{magenta}}
}






\begin{document}
\begin{framed}
\noindent Name: \textbf{Huan Q. Bui}\\
Course: \textbf{8.370 - QC}\\
Problem set: \textbf{\#2}\\
Due: Wednesday, Sep 28, 2022\\
Collaborators: 
\end{framed}

\noindent \textbf{1. Tensor products}\\

\noindent Starting with
\begin{align*}
	\ket{\psi} = \f{1}{\sqrt{2}}\lp \ket{01} - \ket{10} \rp
\end{align*}
we have
\begin{align*}
	&\ket{\psi_x} = \sigma_x \otimes \mathbb{I} \ket{\psi} = \f{1}{\sqrt{2}}\lp \ket{11} - \ket{00} \rp\\
	&\ket{\psi_y} = \sigma_y \otimes \mathbb{I} \ket{\psi} = \f{1}{\sqrt{2}}\lp i \ket{11} + i \ket{00}   \rp = \f{i}{\sqrt{2}}\lp \ket{11} + \ket{00} \rp\\
	&\ket{\psi_z} = \sigma_z\otimes \mathbb{I} \ket{\psi} = \f{1}{\sqrt{2}}\lp \ket{01} + \ket{10} \rp  
\end{align*}
Now we check that they are all orthogonal:
\begin{align*}
	&\braket{\psi_x} = \f{1}{2}\lp \braket{11} + \braket{00} - \bra{00}\ket{11} - \bra{11}\ket{00}  \rp  = 1 \\
	&\braket{\psi_y} = \frac{1}{2}\lp  \braket{11} + \braket{00}  +  \bra{00}\ket{11} + \bra{11}\ket{00}  \rp = 1\\
	&\braket{\psi_z} = \frac{1}{2}\lp \braket{01} + \braket{10} + \bra{01}\ket{10} + \bra{10}\ket{01}  \rp = 1 \\
	&\bra{\psi_x}\ket{\psi_y} =  \f{i}{2}\lp \braket{11} - \braket{00} + \bra{11}\ket{00} - \bra{00}\ket{11} \rp = 0\\
	& \bra{\psi_y} \ket{\psi_z} = \f{-i}{2}\lp \bra{11}\ket{01} + \bra{11}\ket{10} + \bra{00}\ket{01} +  \bra{00}\ket{10}\rp = 0 \\
	&\bra{\psi_z}\ket{\psi_x} = \f{1}{2}\lp \bra{01}\ket{11} + \bra{10}\ket{11} - \bra{01}\ket{00} - \bra{10}\ket{00} \rp = 0
\end{align*}



\noindent \textbf{2. Observable with repeated eigenvalues}

\begin{align*}
	M = \begin{pmatrix}
		2 & 2 & 2 \\
		2 & -1 & 1 \\
		2 & 1 & -1 
	\end{pmatrix}
\end{align*}
\begin{enumerate}[label=(\alph*)]
	\item Computing the eigenvalues associated with the with given eigenvalues is easy:
	\begin{align*}
		\vec{v}_1 = (2\,\,1\,\,1)^\top: &\quad\quad \lambda = 4\\
		\vec{v}_2 = (1\,\,-1\,\,-1)^\top: &\quad\quad \lambda = -2\\
		\vec{v}_3 = (0\,\,1\,\,-1)^\top: &\quad\quad \lambda = -2
	\end{align*}
	Before forming orthogonal projections, we need to make sure that the provided vectors form an orthogonal basis. By inspection, we need to first normalize the vectors to get
	\begin{align*}
		\ket{\psi_1} = \f{\vec{v}_1}{\norm{\vec{v}_1}} = \f{1}{\sqrt{6}}(2\,\,1\,\,1)^\top , 
		\quad \ket{\psi_2} = \f{\vec{v}_2}{\norm{\vec{v}_2}} = \f{1}{\sqrt{3}}  (1\,\,-1\,\,-1)^\top, 
		\quad 
		\ket{\psi_3} = \f{\vec{v}_3}{\norm{\vec{v}_3}} = \f{1}{\sqrt{2}} (0\,\,1\,\,-1)^\top.
	\end{align*}
	Now we check for orthonormality. We can do this by inspection so I won't write out the algebra. 
	\begin{align*}
		&\braket{\psi_1} = \braket{\psi_2} = \braket{\psi_3} = 1\\
		&\bra{\psi_1}\ket{\psi_2} = \bra{\psi_2}\ket{\psi_3} = \bra{\psi_3}\ket{\psi_1} = 0.
	\end{align*}
	With these conditions satisfied, the orthogonal projections are:
	\begin{align*}
		&\Pi_1 = \ketbra{\psi_1} = \f{1}{6}\begin{pmatrix}
			4 & 2 & 2 \\
			2 & 1 & 1\\
			2 & 1 & 1
		\end{pmatrix}\\
		&\Pi_2 = \ketbra{\psi_2} = \f{1}{3}\begin{pmatrix}
			1 & -1 & -1 \\
			-1 & 1 & 1\\
			-1 & 1 & 1
		\end{pmatrix}\\
		&\Pi_3 = \ketbra{\psi_3} = \f{1}{2}\begin{pmatrix}
			0 & 0 & 0 \\
			0 & 1 & -1 \\
			0 & -1 & 1
		\end{pmatrix}
	\end{align*}
	Sanity check:
	\begin{align*}
		\Pi_1 + \Pi_2 + \Pi_3 = \mathbb{I} \,\, \checkmark\\
		\Pi_1 \Pi_2 = \Pi_2 \Pi_3 = \Pi_3 \Pi_1 = \mathcal{O} \,\, \checkmark
	\end{align*}



	All the algebra is from above is verified in Matheatica. Mathematica calculations:
	\begin{lstlisting}
		(*eigv check*)
		In[53]:= M = {{2, 2, 2}, {2, -1, 1}, {2, 1, -1}};
		In[5]:= Eigenvalues[M]
		Out[5]= {4, -2, -2}
		
		(*vectors in the ONB*)
		p1 = {2, 1, 1}/Norm[{2, 1, 1}];
		p2 = {1, -1, -1}/Norm[{1, -1, -1}];
		p3 = {0, 1, -1}/Norm[{0, 1, -1}];
		
		(*check ONB*)
		In[14]:= Dot[p1, p2]
		Out[14]= 0
		In[15]:= Dot[p2, p3]
		Out[15]= 0
		In[16]:= Dot[p3, p1]
		Out[16]= 0
		In[21]:= Dot[p1, p1]
		Out[21]= 1
		In[22]:= Dot[p2, p2]
		Out[22]= 1
		In[23]:= Dot[p3, p3]
		Out[23]= 1
		
		(*Compute projectors*)
		In[46]:= M1 = KroneckerProduct[p1, p1]
		Out[46]= {{2/3, 1/3, 1/3}, {1/3, 1/6, 1/6}, {1/3, 1/6, 1/6}}
		In[47]:= M2 = KroneckerProduct[p2, p2]
		Out[47]= {{1/3, -(1/3), -(1/3)}, {-(1/3), 1/3, 1/3}, {-(1/3), 1/3, 1/3}}
		In[48]:= M3 = KroneckerProduct[p3, p3]
		Out[48]= {{0, 0, 0}, {0, 1/2, -(1/2)}, {0, -(1/2), 1/2}}
		
		(*Check resolution of identity:*)
		In[52]:= M1 + M2 + M3
		Out[52]= {{1, 0, 0}, {0, 1, 0}, {0, 0, 1}}
		In[205]:= M1 . M2
		Out[205]= {{0, 0, 0}, {0, 0, 0}, {0, 0, 0}}
		In[206]:= M2 . M3
		Out[206]= {{0, 0, 0}, {0, 0, 0}, {0, 0, 0}}
		In[207]:= M3 . M1
		Out[207]= {{0, 0, 0}, {0, 0, 0}, {0, 0, 0}}
	\end{lstlisting}
	
	
	\item We are given 
	\begin{align*}
		\ket{\psi} = \f{2}{3}\ket{0} + \f{2}{3}\ket{1} - \f{1}{3}\ket{2}.
	\end{align*}
	When this qutrit is measured, the possible outcomes are $4$ and $-2$, with probabilities:
	\begin{align*}
		\Pr(4) = \bra{\psi}\Pi_1 \ket{\psi} = \f{4}{9}\quad \text{and} \quad
		\Pr(-2) = \bra{\psi}\Pi_2 \ket{\psi} + \bra{\psi}\Pi_3 \ket{\psi} = \f{5}{9}.
	\end{align*}
	There are two ways to get the answer. By inspection, we can immediately see that the probability of measuring $4$ is $4/9$, since the coefficient for $\ket{0}$ is $2/3$. From there, we can conclude that the probability of measuring $-2$ is simply $1-4/9=5/9$. The other way to find these values is by directly doing the algebra. The Mathematica code below has the explicit calculations.
	
	\begin{lstlisting}
		In[58]:= \[Psi] = (2/3)*p1 + (2/3)*p2 - (1/3)*p3;
		
		(*Pr(4)*)
		In[71]:= Transpose[\[Psi]] . M1 . \[Psi] // FullSimplify
		Out[71]= 4/9
		
		(*Pr(-2)*)
		In[72]:= Transpose[\[Psi]] . M2 . \[Psi] +  Transpose[\[Psi]] . M3 . \[Psi] // FullSimplify
		Out[72]= 5/9
	\end{lstlisting} 	
\end{enumerate}

\noindent \textbf{3. Spin-1 particle}\\

\noindent We are given a spin-1 particle with three quantum states $\ket{1},\ket{0},\ket{-1}$. The observables corresponding to the spin along the three spatial directions are $J_x,J_y,J_z$:
\begin{align*}
	J_z = \begin{pmatrix}
		1 & 0 & 0 \\
		0 & 0 & 0 \\
		0 & 0 & -1
	\end{pmatrix}, \quad 
	J_x = \f{1}{\sqrt{2}}\begin{pmatrix}
		0 &1 &0 \\ 
		1 &0 &1 \\
		0 &1 &0 
	\end{pmatrix}, \quad 
	J_y = \f{1}{\sqrt{2}}\begin{pmatrix}
		0 &-i &0 \\
		i &0 &-i \\
		0 &i &0 
	\end{pmatrix}
\end{align*}


\begin{enumerate}[label=(\alph*)]
	\item We will show that $J_x,J_z$ cannot be measured simultaneously by showing that they do not commute:
	\begin{align*}
		[J_x,J_z] = J_xJ_z - J_zJ_x = \f{1}{\sqrt{2}}\begin{pmatrix}
			0 & -1 & 0 \\
			1 & 0 & -1 \\
			0 & 1 & 0
		\end{pmatrix} = -i J_y \neq \mathcal{O}.
	\end{align*}
	Mathematica code:
	\begin{lstlisting}
		In[83]:= Jz = {{1, 0, 0}, {0, 0, 0}, {0, 0, -1}};
		In[84]:= Jx = (1/Sqrt[2])*{{0, 1, 0}, {1, 0, 1}, {0, 1, 0}};
		In[85]:= Jy = (1/Sqrt[2])*{{0, -I, 0}, {I, 0, -I}, {0, I, 0}};
		In[87]:= Jx . Jz - Jz . Jx
		Out[87]= {{0, -(1/Sqrt[2]), 0}, {1/Sqrt[2], 0, -(1/Sqrt[2])}, {0, 1/Sqrt[2], 0}}
	\end{lstlisting}
	
	\item However, the observables $J_x^2, J_y^2,J_z^2$ all commute. We can do this by hand or use Mathematica again:
	\begin{lstlisting}
		(*[Jx^2,Jy^2]*)
		In[91]:= (Jx . Jx) . (Jy . Jy) - (Jy . Jy) . (Jx . Jx)
		Out[91]= {{0, 0, 0}, {0, 0, 0}, {0, 0, 0}}
		
		(*[Jy^2,Jz^2]*)	
		In[92]:= (Jy . Jy) . (Jz . Jz) - (Jz . Jz) . (Jy . Jy)
		Out[92]= {{0, 0, 0}, {0, 0, 0}, {0, 0, 0}}
		
		(*[Jz^2,Jx^2]*)
		In[93]:= (Jz . Jz) . (Jx . Jx) - (Jx . Jx) . (Jz . Jz)
		Out[93]= {{0, 0, 0}, {0, 0, 0}, {0, 0, 0}}
	\end{lstlisting}
	There are possibly multiple ways (including clever math tricks) to find the simultaneous eigenvectors for $J_x^2,J_y^2,J_z^2$. However, it turns out that we could also do this by inspection:
	\begin{align*}
		J_x^2 = \f{1}{2}\begin{pmatrix}
			1 & 0 & 1 \\
			0 & 2 & 0 \\
			1 & 0 & 1
		\end{pmatrix}, \quad 
		J_y^2 = \f{1}{2}\begin{pmatrix}
			1 & 0 &- 1 \\
			0 & 2 & 0 \\
			-1 & 0 & 1
		\end{pmatrix}, \quad
		J_z^2 = \begin{pmatrix}
			1 & 0 & 0 \\
			0 & 0 & 0 \\
			0 & 0 & 1
		\end{pmatrix}.
	\end{align*}
	By the form of the matrices, we can guess that the three normalized simultaneous eigenvectors are
	\begin{align*}
		\ket{+} = \f{1}{\sqrt{2}}(\ket{1} + \ket{-1}), \quad 
		\ket{-} = \f{1}{\sqrt{2}}(\ket{1} - \ket{-1}), \quad
		\ket{0} = \ket{0}.
	\end{align*}
	The corresponding eigenvalues can be found from the results below:
	\begin{align*}
		J_x^2 \ket{+} &= \ket{+} \\
		J_x^2 \ket{-} &= 0\\
		J_x^2 \ket{0} &= \ket{0}\\
		J_y^2 \ket{+} &= 0\\
		J_y^2 \ket{-} &= \ket{-}\\
		J_y^2 \ket{0} &= \ket{0}\\
		J_z^2 \ket{+} &= \ket{+}\\
		J_z^2 \ket{-} &= \ket{-}\\
		J_z^2 \ket{0} &= 0
	\end{align*}
	So, $J_i^2$ has spectrum $\{0,1\}$ for all $i=x,y,z$. Finally, we have
	\begin{align*}
		J^2 = J_x^2 + J_y^2 + J_z^2 = 2\mathbb{I}.
	\end{align*}
	
	
	
	
	While a lot of the calculations in this problem could be done by hand, it is faster and more accurate to do them in Mathematica:
	\begin{lstlisting}
		(*squaring*)
		Jx2 = Jx . Jx;
		Jy2 = Jy . Jy;
		Jz2 = Jz . Jz;
		
		In[117]:= Jx2
		Out[117]= {{1/2, 0, 1/2}, {0, 1, 0}, {1/2, 0, 1/2}}
		In[118]:= Jy2
		Out[118]= {{1/2, 0, -(1/2)}, {0, 1, 0}, {-(1/2), 0, 1/2}}
		In[119]:= Jz^2
		Out[119]= {{1, 0, 0}, {0, 0, 0}, {0, 0, 1}}
		
		
		(*eigenvalues calcs*)
		In[140]:= plus = (1/Sqrt[2]) {1, 0, 1};
		In[152]:= minus = (1/Sqrt[2])*{1, 0, -1};
		In[153]:= zero = {0, 1, 0};	
		In[145]:= Jx2 . plus
		Out[145]= {1/Sqrt[2], 0, 1/Sqrt[2]}
		In[154]:= Jx2 . minus
		Out[154]= {0, 0, 0}
		In[149]:= Jx2 . zero
		Out[149]= {0, 1, 0}
		In[150]:= Jy2 . plus
		Out[150]= {0, 0, 0}
		In[155]:= Jy2 . minus
		Out[155]= {1/Sqrt[2], 0, -(1/Sqrt[2])}
		In[156]:= Jy2 . zero
		Out[156]= {0, 1, 0}
		In[157]:= Jz2 . plus
		Out[157]= {1/Sqrt[2], 0, 1/Sqrt[2]}
		In[159]:= Jz2 . minus
		Out[159]= {1/Sqrt[2], 0, -(1/Sqrt[2])}
		In[160]:= Jz2 . zero
		Out[160]= {0, 0, 0}
		
		(*finally, total spin*)
		In[94]:= Jx . Jx + Jy . Jy + Jz . Jz
		Out[94]= {{2, 0, 0}, {0, 2, 0}, {0, 0, 2}}
	\end{lstlisting}

\end{enumerate}





\noindent \textbf{4. Deriving Spin-1 Observables}\\

\noindent In this problem we derive the matrix $J_x$ in the previous problem. Suppose we have two qubits $A$ and $B$. The observable giving the spin in the $x$-direction is 
\begin{align*}
	S_x = \f{1}{2}\lp \sigma_x^A\otimes \mathbb{I}^B + \mathbb{I}^A \otimes \sigma_x^B \rp.
\end{align*}
The 3-dimensional subspace of the 4-dimensional state space of two qubits which corresponds to the state space of a spin-1 particle is the subspace orthogonal to the state $(\ket{01}- \ket{10})/\sqrt{2}$. To avoid confusion, let us replace $0$ with $\uparrow$ and $1$ with $\downarrow$\\

Since we have two qubits, we can treat them as two spin-1/2 particles, each denoted by $\ket{s,m_s}$. In this notation, we have
\begin{align*}
	\ket{\uparrow\uparrow} &= \ket{1/2,+1/2} \otimes \ket{1/2,+1/2}\\
	\ket{\uparrow\downarrow} &= \ket{1/2,+1/2} \otimes \ket{1/2,-1/2}\\
	\ket{\downarrow\uparrow} &= \ket{1/2,-1/2}\otimes \ket{1/2,+1/2}\\
	\ket{\downarrow\downarrow} &= \ket{1/2,-1/2}\otimes \ket{1/2,-1/2}
\end{align*}
When the spins are added, we can express the total spin and its projection as $\ket{s,m}\in \mathcal{H}^{\otimes 2}$ where
\begin{align*}
	\ket{s,m} = \sum_{m_{s,1}=-s_1}^{s_1} \sum_{m_{s,2}=-s_2}^{s_2} C^{s,m}_{s_1,m_{s,1},s_2,m_{s,2}} \ket{s_1,m_{s,1}}\ket{s_2,m_{s,2}}
\end{align*}
where $C^{\dots}_{\dots}$'s are the Clebsch-Gordan coefficients. For this problem, the solution is rather simple. In the two-qubit Hilbert space, there is one state (the singlet) for which the total spin is zero ($s=0$), and this state is one given in the problem: $(\ket{\uparrow\downarrow}- \ket{\downarrow\uparrow})/\sqrt{2}$, and there are three states (triplet) which correspond to $s=1$ (total spin equal to 1). It turns out that these are
\begin{align*}
	\ket{s=1,m=+1} &= \ket{\uparrow\uparrow}\\
	\ket{s=1,m=\,\,\,\,0} &= \f{1}{\sqrt{2}}\lp \ket{\uparrow\downarrow} + \ket{\downarrow\uparrow} \rp\\
	\ket{s=1,m=-1} &= \ket{\downarrow\downarrow}
\end{align*} 
With this information, we can now construct a unitary matrix which transforms the standard basis $\{\ket{\uparrow\uparrow}, \ket{\uparrow\downarrow}, \ket{\downarrow\uparrow}, \ket{\downarrow\downarrow}\}$ into the new basis where the first elements has spin 0 and the subsequent three has spin 1: $\{ \ket{0,0}, \ket{1,1},\ket{1,0},\ket{1,-1} \}$. By inspection, this matrix is 
\begin{align*}
	U = \begin{pmatrix}
		 0 & 1 & 0& 0\\
		 1/\sqrt{2} & 0 & 1/\sqrt{2}& 0\\
		-1/\sqrt{2} & 0 & 1/\sqrt{2}& 0\\
		 0 & 0 & 0& 1
	\end{pmatrix}
\end{align*}
We expect that a similarity transformation on $S_x$ by $U$ will take the form of a $(2\times 2)$-block diagonal matrix of the form $\text{diag}(0,Jx)$. And indeed, using Mathematica, we find that
\begin{align*}
	U^\dagger S_x U = \f{1}{\sqrt{2}}\begin{pmatrix}
		0&0&0&0\\
		0&0&1&0\\
		0&1&0&1\\
		0&0&1&0
	\end{pmatrix} = \begin{pmatrix}
	0 & \\
	& J_x
\end{pmatrix}.
\end{align*}
With this, we have
\begin{align*}
		J_x = \f{1}{\sqrt{2}}\begin{pmatrix}
		0 &1 &0 \\ 
		1 &0 &1 \\
		0 &1 &0 
	\end{pmatrix}, 
\end{align*}
as desired. \\


Mathematica calculations:
\begin{lstlisting}
	In[220]:= 
	U = {{0, 1, 0, 0}, 1/Sqrt[2]*{1, 0, 1, 0}, 
		1/Sqrt[2]*{-1, 0, 1, 0}, {0, 0, 0, 1}};
	
	In[222]:= ConjugateTranspose[U] . SX . U
	Out[222]= {{0, 0, 0, 0}, {0, 0, 1/Sqrt[2], 0}, {0, 1/Sqrt[2], 0, 1/
			Sqrt[2]}, {0, 0, 1/Sqrt[2], 0}}
		
	(*displaying the result in matrix form gives Jx*)
\end{lstlisting}


\noindent \textbf{5. Generalized Measurements}\\


Here we derive an example of a non-von Neumann measurement. We're given one of the three states
\begin{align*}
	\ket{\psi_1} = \ket{0} , \quad \ket{\psi_2} = -\f{1}{2} \ket{0} + \f{\sqrt{3}}{2}\ket{1}, \quad \ket{\psi_3} = -\f{1}{2}\ket{0} - \f{\sqrt{3}}{2}\ket{1}
\end{align*}
with equal probabilities. 

\begin{enumerate}[label=(\alph*)]
	\item Suppose that we make a measurement of the state in the following arbitrary basis:
	\begin{align*}
		\{ \ket{A} = \cos\theta \ket{0} + \sin\theta \ket{1} , \ket{B} =  -\sin\theta\ket{0} + \cos\theta\ket{1}    \}
	\end{align*}
	Let us focus on when we find state $\ket{A}$ after the measurement. Suppose we guess that the input state is $\ket{\psi_1}$ whenever we see state $\ket{A}$, then the probability of success is given by
	\begin{align*}
		\Pr_{\ket{A}\implies \ket{\psi_1}} = \f{\abs{\bra{\psi_1} \ket{A}}^2}{\abs{\bra{\psi_1} \ket{A}}^2+\abs{\bra{\psi_2} \ket{A}}^2+\abs{\bra{\psi_3} \ket{A}}^2}
	\end{align*}
	which comes from the fact that there are three ways we could find state $\ket{A}$ after the measurement, each contributing some probability. Plugging in the numbers, we find that
	\begin{align*}
		\Pr_{\ket{A}\implies \ket{\psi_1}} = \f{2}{3}\cos^2\theta \leq \f{2}{3}, \quad \forall \theta \in [0,2\pi]
	\end{align*}
	Similarly, if we guess it is state $\ket{\psi_2}$ or $\ket{\psi_3}$ then the probability of success in each case is 
	\begin{align*}
		&\Pr_{\ket{A}\implies \ket{\psi_2}} = \f{1}{6}\lp\cos\theta - \sqrt{3}\sin\theta\rp^2 \leq \f{2}{3}, \quad \forall \theta \in [0,2\pi] \\
		&\Pr_{\ket{A}\implies \ket{\psi_3}} = \f{1}{6}\lp\cos\theta + \sqrt{3}\sin\theta\rp^2 \leq \f{2}{3}, \quad \forall \theta \in [0,2\pi] 
	\end{align*}
	From the last three inequalities, we see that the best success probability is 2/3. \\
	
	Since there's nothing special about whether we pick $\ket{A}$ or $\ket{B}$ as the ''indicator,'' we expect the same result to hold if we use $\ket{B}$ instead of $\ket{A}$ and can stop here. However, it doesn't hurt to be explicit. So, following the same notation as in the argument above, we have
	\begin{align*}
		&\Pr_{\ket{B}\implies \ket{\psi_1}} = \f{2}{3}\sin^2\theta \leq \f{2}{3}, \quad \forall \theta \in [0,2\pi]\\
		&\Pr_{\ket{B}\implies \ket{\psi_2}} = \f{1}{6}\lp\sqrt{3}\cos\theta + \sin\theta\rp^2 \leq \f{2}{3}, \quad \forall \theta \in [0,2\pi] \\
		&\Pr_{\ket{B}\implies \ket{\psi_3}} = \f{1}{6}\lp-\sqrt{3} \cos\theta + \sin\theta\rp^2 \leq \f{2}{3}, \quad \forall \theta \in [0,2\pi] 
	\end{align*}
	which is not surprisingly the same as before. \\
	
	
	Mathematica code:
	\begin{lstlisting}
		In[159]:= (*Problem 5a*)
		In[133]:= (*define input states*)		
		In[128]:= v0 = {1, 0};		
		In[129]:= v1 = {-1/2, Sqrt[3]/2};		
		In[130]:= v2 = {-1/2, -Sqrt[3]/2};
		
		In[134]:= (*define measurement basis*)		
		In[131]:= A = {Cos[\[Theta]], Sin[\[Theta]]};
		B = {-Sin[\[Theta]], Cos[\[Theta]]};		
		In[142]:= (*calculate inner products*)		
		In[143]:= A0 = Dot[A, v0]^2 // FullSimplify		
		Out[143]= Cos[\[Theta]]^2		
		In[144]:= B0 = Dot[B, v0]^2 // FullSimplify		
		Out[144]= Sin[\[Theta]]^2		
		In[145]:= A1 = Dot[A, v1]^2 // FullSimplify		
		Out[145]= 1/4 (Cos[\[Theta]] - Sqrt[3] Sin[\[Theta]])^2		
		In[146]:= B1 = Dot[B, v1]^2 // FullSimplify		
		Out[146]= 1/4 (Sqrt[3] Cos[\[Theta]] + Sin[\[Theta]])^2		
		In[147]:= A2 = Dot[A, v2]^2 // FullSimplify		
		Out[147]= 1/4 (Cos[\[Theta]] + Sqrt[3] Sin[\[Theta]])^2		
		In[148]:= B2 = Dot[B, v2]^2 // FullSimplify		
		Out[148]= 1/4 (-Sqrt[3] Cos[\[Theta]] + Sin[\[Theta]])^2
		
		In[155]:= (*if pick second basis vector as indicator*)
		(*calculate success probabilities*)		
		In[151]:= PrBv0 = B0/(B0 + B1 + B2) // FullSimplify		
		Out[151]= (2 Sin[\[Theta]]^2)/3		
		In[152]:= PrBv1 = B1/(B0 + B1 + B2) // FullSimplify		
		Out[152]= 1/6 (Sqrt[3] Cos[\[Theta]] + Sin[\[Theta]])^2		
		In[153]:= PrBv2 = B2/(B0 + B1 + B2) // FullSimplify		
		Out[153]= 1/6 (-Sqrt[3] Cos[\[Theta]] + Sin[\[Theta]])^2
		
		In[154]:= (*if pick first basis vector as indicator*)
		(*calculate success probabilities*)		
		In[156]:= PrAv0 = A0/(A0 + A1 + A2) // FullSimplify		
		Out[156]= (2 Cos[\[Theta]]^2)/3		
		In[157]:= PrAv1 = A1/(A0 + A1 + A2) // FullSimplify		
		Out[157]= 1/6 (Cos[\[Theta]] - Sqrt[3] Sin[\[Theta]])^2		
		In[158]:= PrAv2 = A2/(A0 + A1 + A2) // FullSimplify		
		Out[158]= 1/6 (Cos[\[Theta]] + Sqrt[3] Sin[\[Theta]])^2
		
		(*find Max*)
		In[161]:= MaxValue[PrAv0, \[Theta]]		
		Out[161]= 2/3		
		In[162]:= MaxValue[PrAv1, \[Theta]]		
		Out[162]= 2/3		
		In[163]:= MaxValue[PrAv2, \[Theta]]		
		Out[163]= 2/3		
		In[164]:= MaxValue[PrBv0, \[Theta]]		
		Out[164]= 2/3		
		In[165]:= MaxValue[PrBv1, \[Theta]]		
		Out[165]= 2/3		
		In[166]:= MaxValue[PrBv2, \[Theta]]		
		Out[166]= 2/3
	\end{lstlisting}



	
	\item Now we take the first qubit and tensor it with a second qubit in $\ket{0}$. Consider the following states:
	\begin{align*}
		\{ \ket{a}, \ket{b}, \ket{c}, \ket{d}  \} = \lc \ket{11}, -\f{\al}{2}\ket{00}+\f{\sqrt{3}\al}{2}\ket{10} + \be\ket{01}, \al\ket{00} + \be\ket{01}, -\f{\al}{2}\ket{00} - \f{\sqrt{3}\al}{2}\ket{10} + \be\ket{01} \rc.
	\end{align*}
	In order for these to form an orthonormal basis, $\al$ and $\be$ must satisfy the following conditions:
	\begin{align*}
		\abs{\al}^2 + \abs{\be}^2 &= 1\\
		-\f{\abs{\al}^2}{2} + \abs{\be}^2 &= 0 
	\end{align*}
	From the first two equations, we find that $\abs{\al}^2 = 2/3$ and $\abs{\be}^2 = 1/3$. Assuming $\al,\be\in \mathbb{R}$, we can let $\al = \sqrt{2/3}$ and $\be = \sqrt{1/3}$. 
	
	
	\item With probability 1/3 we are given $\ket{\psi_1}$, which we transform to $\ket{\psi_1}\ket{0}$. Measuring this state in the basis above, we find 
	\begin{align*}
		\Pr(\ket{a}) &= 0\\
		\Pr(\ket{b}) &= 1/6\\
		\Pr(\ket{c}) &= 2/3 \\
		\Pr(\ket{d}) &= 1/6 
	\end{align*}
	With probability 1/3 we are given $\ket{\psi_2}$, which we transform to $\ket{\psi_2}\ket{0}$. Measuring this state in the basis above, we find 
	\begin{align*}
		\Pr(\ket{a}) &= 0\\
		\Pr(\ket{b}) &= 2/3\\
		\Pr(\ket{c}) &= 1/6 \\
		\Pr(\ket{d}) &= 1/6
	\end{align*}
	With probability 1/3 we are given $\ket{\psi_3}$, which we transform to $\ket{\psi_3}\ket{0}$. Measuring this state in the basis above, we find 
	\begin{align*}
		\Pr(\ket{a}) &= 0\\
		\Pr(\ket{b}) &= 1/6\\
		\Pr(\ket{c}) &= 1/6 \\
		\Pr(\ket{d}) &= 2/3
	\end{align*}
	Now we make the following rules for guessing:
	\begin{itemize}
		\item If we measure and find $\ket{b}$ then guess $\ket{\psi_2}$
		
		\item If we measure and find $\ket{c}$ then guess $\ket{\psi_1}$ 
		
		\item If we measure and find $\ket{d}$ then guess $\ket{\psi_3}$
	\end{itemize}

	Since the cases are symmetric, the success probability is simply given by 
	\begin{align*}
		\Pr(\text{success}) = \f{2/3}{2/3+1/6+1/6} = \f{2}{3}
	\end{align*}

	Mathematica calculations:
	\begin{lstlisting}
		In[17]:= (*5b*)
		In[60]:= a1 = {0, 0, 0, 1};
		In[59]:= a3 = {\[Alpha], \[Beta], 0, 0};
		In[56]:= a2 = {-\[Alpha]/2, \[Beta], \[Alpha]*Sqrt[3]/2, 0};
		In[57]:= a4 = {-\[Alpha]/2, \[Beta], -\[Alpha]*Sqrt[3]/2, 0};
		In[64]:= Psi1 = {1, 0, 0, 0};
		In[62]:= Psi2 = {-1/2, 0, Sqrt[3]/2, 0};
		In[63]:= Psi3 = {-1/2, 0, -Sqrt[3]/2, 0};
		In[65]:= Dot[Psi2, a1]^2
		Out[65]= 0
		In[66]:= Dot[Psi2, a2]^2
		Out[66]= \[Alpha]^2		
		In[67]:= Dot[Psi2, a3]^2		
		Out[67]= \[Alpha]^2/4		
		In[68]:= Dot[Psi2, a4]^2		
		Out[68]= \[Alpha]^2/4		
		In[52]:= Dot[Psi3, a1]^2		
		Out[52]= 0		
		In[53]:= Dot[Psi3, a2]^2		
		Out[53]= \[Alpha]^2/4		
		In[54]:= Dot[Psi3, a3]^2		
		Out[54]= \[Alpha]^2/4		
		In[55]:= Dot[Psi3, a4]^2	
		Out[55]= \[Alpha]^2
	\end{lstlisting}


\end{enumerate}


\end{document}











