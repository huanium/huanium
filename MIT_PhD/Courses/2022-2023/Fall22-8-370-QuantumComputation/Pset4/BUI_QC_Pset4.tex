\documentclass{article}
\usepackage{physics}
\usepackage{graphicx}
\usepackage{caption}
\usepackage{amsmath}
\usepackage{bm}
\usepackage{framed}
\usepackage{authblk}
\usepackage{empheq}
\usepackage{amsfonts}
\usepackage{esint}
\usepackage[makeroom]{cancel}
\usepackage{dsfont}
\usepackage{centernot}
\usepackage{mathtools}
\usepackage{subcaption}
\usepackage{bigints}
\usepackage{amsthm}
\theoremstyle{definition}
\newtheorem{lemma}{Lemma}
\newtheorem{defn}{Definition}[section]
\newtheorem{prop}{Proposition}[section]
\newtheorem{rmk}{Remark}[section]
\newtheorem{thm}{Theorem}[section]
\newtheorem{exmp}{Example}[section]
\newtheorem{prob}{Problem}[section]
\newtheorem{sln}{Solution}[section]
\newtheorem*{prob*}{Problem}
\newtheorem{exer}{Exercise}[section]
\newtheorem*{exer*}{Exercise}
\newtheorem*{sln*}{Solution}
\usepackage{empheq}
\usepackage{tensor}
\usepackage{xcolor}
%\definecolor{colby}{rgb}{0.0, 0.0, 0.5}
\definecolor{MIT}{RGB}{163, 31, 52}
\usepackage[pdftex]{hyperref}
%\hypersetup{colorlinks,urlcolor=colby}
\hypersetup{colorlinks,linkcolor={MIT},citecolor={MIT},urlcolor={MIT}}  
\usepackage[left=1in,right=1in,top=1in,bottom=1in]{geometry}

\usepackage{newpxtext,newpxmath}
\newcommand*\widefbox[1]{\fbox{\hspace{2em}#1\hspace{2em}}}

\newcommand{\p}{\partial}
\newcommand{\R}{\mathbb{R}}
\newcommand{\C}{\mathbb{C}}
\newcommand{\lag}{\mathcal{L}}
\newcommand{\nn}{\nonumber}
\newcommand{\ham}{\mathcal{H}}
\newcommand{\M}{\mathcal{M}}
\newcommand{\I}{\mathcal{I}}
\newcommand{\K}{\mathcal{K}}
\newcommand{\F}{\mathcal{F}}
\newcommand{\w}{\omega}
\newcommand{\lam}{\lambda}
\newcommand{\al}{\alpha}
\newcommand{\be}{\beta}
\newcommand{\x}{\xi}

\newcommand{\G}{\mathcal{G}}

\newcommand{\f}[2]{\frac{#1}{#2}}

\newcommand{\ift}{\infty}

\newcommand{\lp}{\left(}
\newcommand{\rp}{\right)}

\newcommand{\lb}{\left[}
\newcommand{\rb}{\right]}

\newcommand{\lc}{\left\{}
\newcommand{\rc}{\right\}}


\newcommand{\V}{\mathbf{V}}
\newcommand{\U}{\mathcal{U}}
\newcommand{\Id}{\mathcal{I}}
\newcommand{\D}{\mathcal{D}}
\newcommand{\Z}{\mathcal{Z}}

%\setcounter{chapter}{-1}


\usepackage{enumitem}



\usepackage{listings}
\captionsetup[lstlisting]{margin=0cm,format=hang,font=small,format=plain,labelfont={bf,up},textfont={it}}
\renewcommand*{\lstlistingname}{Code \textcolor{violet}{\textsl{Mathematica}}}
\definecolor{gris245}{RGB}{245,245,245}
\definecolor{olive}{RGB}{50,140,50}
\definecolor{brun}{RGB}{175,100,80}

%\hypersetup{colorlinks,urlcolor=colby}
\lstset{
	tabsize=4,
	frame=single,
	language=mathematica,
	basicstyle=\scriptsize\ttfamily,
	keywordstyle=\color{black},
	backgroundcolor=\color{gris245},
	commentstyle=\color{gray},
	showstringspaces=false,
	emph={
		r1,
		r2,
		epsilon,epsilon_,
		Newton,Newton_
	},emphstyle={\color{olive}},
	emph={[2]
		L,
		CouleurCourbe,
		PotentielEffectif,
		IdCourbe,
		Courbe
	},emphstyle={[2]\color{blue}},
	emph={[3]r,r_,n,n_},emphstyle={[3]\color{magenta}}
}






\begin{document}
\begin{framed}
\noindent Name: \textbf{Huan Q. Bui}\\
Course: \textbf{8.370 - QC}\\
Problem set: \textbf{\#4}\\
Due: Wednesday, Oct 12, 2022\\
Collaborators/References: 
\end{framed}



\noindent \textbf{1. A two-qubit gate} 

\begin{enumerate}[label=(\alph*)]
	\item To write down the $4\times 4$ matrix for this two-qubit gate, we look at what it does to the basis states:
	\begin{align*}
		&\ket{00} \to \mathbb{I} \otimes \sigma_x \ket{00} = \ket{01}\\
		&\ket{01} \to \mathbb{I} \otimes \sigma_x \ket{01} = \ket{00}\\
		&\ket{10} \to \mathbb{I} \otimes \sigma_z \ket{10} = \ket{10}\\
		&\ket{11} \to \mathbb{I} \otimes \sigma_z \ket{11} = -\ket{11}
	\end{align*}
	So the matrix for this two-qubit gate is 
	\begin{align*}
		G = \begin{pmatrix}
			0 & 1 & 0 & 0\\
			1 & 0 & 0 & 0\\
			0 & 0 & 1 & 0\\
			0 & 0 & 0 & -1
		\end{pmatrix}
	\end{align*}


	\item The circuit can be built by putting together two parts. The first part does the anti-controlled-$\sigma_x$ where the action on qubit 2 occurs if qubit 1 has value 0. The second half of the circuit is just a controlled-$Z$, which we build using CNOTs and Hadamard gates\\
	
	For the first part, we simply invert qubit 1 using a $\sigma_x$ before applying a $CNOT$, then apply $\sigma_x$ on qubit 1. So, this looks like:
	\begin{align*}
		G_1 = (\sigma_x \otimes \Id) \cdot CNOT \cdot (\sigma_x \otimes \Id) = \begin{pmatrix}
			0&1&0&0\\
			1&0&0&0\\
			0&0&1&0\\
			0&0&0&1
		\end{pmatrix}
	\end{align*}
	The second part is also quite simple: we turn the $\sigma_x$ in the CNOT by applying a Hadamard gate on the second qubit before and after the CNOT. This gives
	\begin{align*}
		G_2 = (\Id \otimes H) \cdot CNOT (\Id \otimes H) = \begin{pmatrix}
			1&0&0&0\\
			0&1&0&0\\
			0&0&1&0\\
			0&0&0&-1
		\end{pmatrix}
	\end{align*}
	It is clear that $G = G_1 G_2 = G_2 G_1$, so we're done. The full circuit in algebraic form is 
	\begin{align*}
		G &= (\Id \otimes H) \cdot CNOT (\Id \otimes H) \cdot (\sigma_x \otimes \Id) \cdot CNOT \cdot (\sigma_x \otimes \Id) \\
		&= (\sigma_x \otimes \Id) \cdot CNOT \cdot (\sigma_x \otimes \Id) \cdot (\Id \otimes H) \cdot CNOT (\Id \otimes H)
	\end{align*}
	
\end{enumerate}



\noindent \textbf{2. Alice, Bob, and Charlie teleporting}\\

\noindent Alice and Bob share a Bell pair in $(\ket{00} + \ket{11})/\sqrt{2}$. Bob and Charlie share a Bell pair in a similar state. They want to perform some operations so that Alice and Charlie end up with a pair of qubits in this Bell state. The four-qubit state in the beginning of the procedure is 
\begin{align*}
	\ket{\Psi} &= \f{1}{\sqrt{2}} \lp \ket{0_A0_{B,1}} + \ket{1_A1_{B,1}} \rp \otimes \f{1}{\sqrt{2}}\lp  \ket{0_{B,1} 0_C} + \ket{1_{B,1}1_C} \rp \\
	&= \f{1}{2}\lp \ket{0_A0_{B,1}0_{B,2}0_{C}} + \ket{0_A0_{B,1}1_{B,2}1_C} + \ket{1_A1_{B,1}0_{B,2}0_C}+ \ket{1_A1_{B,1}1_{B,2}1_C} \rp
\end{align*}
The procedure is quantum teleportation. Bob makes a measurement in the Bell basis (since he's the only one to have two qubits). Here are the possible outcomes (I won't worry about normalization here to make things easier to see):
\begin{align*}
	&(\bra{0_{B,1}0_{B,2}} + \ket{1_{B,1}1_{B,2}}) \ket{\Psi} \to \ket{0_A 0_C} +  \ket{1_A 1_C} \\
	&(\bra{0_{B,1}0_{B,2}} - \ket{1_{B,1}1_{B,2}}) \ket{\Psi} \to \ket{0_A 0_C} -  \ket{1_A 1_C} \\
	&(\bra{0_{B,1}1_{B,2}} + \ket{1_{B,1}0_{B,2}}) \ket{\Psi} \to \ket{0_A 1_C} +  \ket{1_A 0_C}\\
	&(\bra{0_{B,1}0_{B,2}} - \ket{1_{B,1}1_{B,2}}) \ket{\Psi} \to \ket{0_A 1_C} -  \ket{1_A 0_C}
\end{align*}
After the measurement, Bob sends Charlie classical bits (2 of them) to let Charlie know which single-qubit gate to apply in order to make Alice and himself share the correct Bell state. The instructions are as follows:
\begin{align*}
	&\ket{0_A 0_C} +  \ket{1_A 1_C}  \to (\Id \otimes \Id)(\ket{0_A 0_C} +  \ket{1_A 1_C})\\
	&\ket{0_A 0_C} -  \ket{1_A 1_C} \to (\Id \otimes \sigma_z)(\ket{0_A 0_C} +  \ket{1_A 1_C})\\
	&\ket{0_A 1_C} +  \ket{1_A 0_C} \to (\Id \otimes \sigma_x)(\ket{0_A 0_C} +  \ket{1_A 1_C})\\
	&\ket{0_A 1_C} -  \ket{1_A 0_C} \to  (\Id \otimes \sigma_x\sigma_z)(\ket{0_A 0_C} +  \ket{1_A 1_C})
\end{align*}




\noindent \textbf{3. Alice, Bob, and Charlie superdense coding}\\

\noindent Alice has two classical bits $a_1,a_2$. Bob has two classical bits $b_1,b_2$. Alice and Charlie share a Bell pair in the state $\ket{0_A0_B} + \ket{1_A1_B}$. Now, just like in superdense coding, Alice will applying the different gates to her qubit depending on what the state of $(a_1,a_2)$ is. Let the rules be 
\begin{align*}
	&(0,0)_A \to (\Id \otimes \Id)( \ket{00} + \ket{11}) = \ket{00} + \ket{11} \\
	&(0,1)_A \to (\sigma_x \otimes \Id)( \ket{00} + \ket{11}) = \ket{10} + \ket{01} \\ 
	&(1,0)_A \to (\sigma_z \otimes \Id)( \ket{00} + \ket{11}) = \ket{00} - \ket{11} \\ 
	&(1,1)_A \to (\sigma_z \sigma_x \otimes \Id)( \ket{00} + \ket{11}) = \ket{01} - \ket{10}
\end{align*}
Now Alice sends her qubit to Bob, and Bob does exactly like what Alice did to her qubit:
\begin{align*}
	&(0,0)_B \to (\Id \otimes \Id)\ket{\Psi_A}  \\
	&(0,1)_B \to (\sigma_x \otimes \Id) \ket{\Psi_A} \\ 
	&(1,0)_B \to (\sigma_z \otimes \Id) \ket{\Psi_A} \\ 
	&(1,1)_B \to (\sigma_z \sigma_x \otimes \Id) \ket{\Psi_A}  
\end{align*}
Now Bob sends his qubit (Alice's ex-qubit) to Charlie.  Charlie now has two qubits and measures in the Bell basis. The decoding procedure is as follows:
\begin{itemize}
	\item If Charlie sees $\ket{00} + \ket{11}$ then he has received $(0,0)$ 
	
	\item If Charlie sees $\ket{10} + \ket{01}$ then he has received $(0,1)$
	
	\item If Charlie sees $\ket{00} - \ket{11}$ then he has received $(1,0)$
	
	\item If Charlies sees $\ket{01} - \ket{10}$ then he has received $(1,1)$
\end{itemize}

To check this, we make a table with Alice and Bob's classical bits and the resultant 2-qubit state at Charlie. 
\begin{align*}
	&(0,0)_A \oplus (0,0)_B \to \ket{\Psi_C} = \ket{00} + \ket{11} \to (0,0)_C = (0,0)_A \oplus (0,0)_B \quad\checkmark\\
	&(0,0)_A \oplus (0,1)_B \to \ket{\Psi_C} = \ket{10} + \ket{01} \to (0,1)_C = (0,0)_A \oplus (0,1)_B \quad\checkmark\\
	&(0,0)_A \oplus (1,0)_B \to \ket{\Psi_C} = \ket{00} - \ket{11} \to (1,0)_C = (0,0)_A \oplus (1,0)_B \quad\checkmark\\
	&(0,0)_A \oplus (1,1)_B \to \ket{\Psi_C} = \ket{01} - \ket{10} \to (1,1)_C = (0,0)_A \oplus (1,1)_B \quad\checkmark\\
	%
	&(0,1)_A \oplus (0,0)_B \to \ket{\Psi_C} = \ket{10} + \ket{01} \to (0,1)_C = (0,1)_A \oplus (0,0)_B \quad\checkmark\\
	&(0,1)_A \oplus (0,1)_B \to \ket{\Psi_C} = \ket{00} + \ket{11} \to (0,0)_C = (0,1)_A \oplus (0,1)_B \quad\checkmark\\
	&(0,1)_A \oplus (1,0)_B \to \ket{\Psi_C} = \ket{01} - \ket{10} \to (1,1)_C = (0,1)_A \oplus (1,0)_B \quad\checkmark\\
	&(0,1)_A \oplus (1,1)_B \to \ket{\Psi_C} = \ket{00} - \ket{11} \to (1,0)_C = (0,1)_A \oplus (1,1)_B \quad\checkmark\\
	%
	&(1,0)_A \oplus (0,0)_B \to \ket{\Psi_C} = \ket{00} - \ket{11} \to (1,0)_C = (1,0)_A \oplus (0,0)_B \quad\checkmark\\
	&(1,0)_A \oplus (0,1)_B \to \ket{\Psi_C} = \ket{01} - \ket{10} \to (1,1)_C = (1,0)_A \oplus (0,1)_B \quad\checkmark\\
	&(1,0)_A \oplus (1,0)_B \to \ket{\Psi_C} = \ket{00} + \ket{11} \to (0,0)_C = (1,0)_A \oplus (1,0)_B \quad\checkmark\\
	&(1,0)_A \oplus (1,1)_B \to \ket{\Psi_C} = \ket{10} + \ket{01} \to (0,1)_C = (1,0)_A \oplus (1,1)_B \quad\checkmark\\
	%
	&(1,1)_A \oplus (0,0)_B \to \ket{\Psi_C} = \ket{01} - \ket{10} \to (1,1)_C = (1,1)_A \oplus (0,0)_B \quad\checkmark\\
	&(1,1)_A \oplus (0,1)_B \to \ket{\Psi_C} = \ket{00} - \ket{11} \to (1,0)_C = (1,1)_A \oplus (0,1)_B \quad\checkmark\\
	&(1,1)_A \oplus (1,0)_B \to \ket{\Psi_C} = \ket{01} + \ket{10} \to (0,1)_C = (1,1)_A \oplus (1,0)_B \quad\checkmark\\
	&(1,1)_A \oplus (1,1)_B \to \ket{\Psi_C} = \ket{00} + \ket{11} \to (0,0)_C = (1,1)_A \oplus (1,1)_B \quad\checkmark
\end{align*}
Of course this is overkill since we're guaranteed to be correct by the "cyclical" nature of the one-qubit gates. It is still nice, however, to explicitly see that things are correct. \\




\noindent \textbf{4. Challenger problem, revisited}\\

\noindent Consider the challenger problem which appeared on HW \#1.  Now suppose we have a mapping $B$ that embeds a single qubit into a subspace of a three-qubit system that behaves as:
\begin{align*}
	&B\ket{0} = \f{\sqrt{2}}{\sqrt{3}} \ket{000} + \f{1}{\sqrt{6}} \ket{011} + \f{1}{\sqrt{6}} \ket{101}\\
	&B\ket{1} = \f{\sqrt{2}}{\sqrt{3}} \ket{111} + \f{1}{\sqrt{6}} \ket{100} + \f{1}{\sqrt{6}} \ket{010}
\end{align*}
We take the qubit given to us and apply $B$, then give the challenger the first two qubits of the result. Let us calculate the probability that we pass the challenger's test. \\

Is it easy to see that if we are given $\ket{0}$ (or $\ket{1}$), then the challenger measures the first two qubits and finds $\ket{00}$ (or $\ket{11}$) with probability $2/3$. \\

Now if we are given $\ket{+}$ then the three-qubit state after we apply $B$ is 
\begin{align*}
	B\ket{+} = \f{1}{\sqrt{2}}B\ket{0} + \f{1}{\sqrt{2}} B\ket{1} 
\end{align*}
If the challenger measures the first two qubits then he finds $\ket{++} = (1/2)(\ket{00} + \ket{01} + \ket{10} + \ket{11})$ with probability
\begin{align*}
	\norm{\bra{++}B\ket{+}}^2 
	&= \lp \f{1}{2}\f{1}{\sqrt{2}} \rp^2 \norm{\lp \sqrt{\f{2}{3}} + \sqrt{\f{1}{6}}  + \sqrt{\f{1}{6}}\rp \ket{0} + \lp \sqrt{\f{2}{3}} + \sqrt{\f{1}{6}}  + \sqrt{\f{1}{6}}\rp \ket{1}}^2 = \f{2}{3}.
\end{align*} 


If we are given $\ket{-}$ then the three-qubit state after we apply $B$ is 
\begin{align*}
	B\ket{-} = \f{1}{\sqrt{2}}B\ket{0} - \f{1}{\sqrt{2}} B\ket{1} 
\end{align*}
If the challenger measures the first two qubits then he finds $\ket{--} = (1/2)(\ket{00} - \ket{01} - \ket{10} + \ket{11})$ with probability
\begin{align*}
	\norm{\bra{--}B\ket{-}}^2 
	&= \lp \f{1}{2}\f{1}{\sqrt{2}} \rp^2 \norm{\lp \sqrt{\f{2}{3}} + \sqrt{\f{1}{6}}  + \sqrt{\f{1}{6}}\rp \ket{0} - \lp \sqrt{\f{2}{3}} + \sqrt{\f{1}{6}}  + \sqrt{\f{1}{6}}\rp \ket{1}}^2 = \f{2}{3}.
\end{align*} 
So the probability of success is
\begin{align*}
	\Pr(\text{success}) = \f{1}{4} \times \f{2}{3} +  \f{1}{4} \times \f{2}{3} +  \f{1}{4} \times \f{2}{3} +  \f{1}{4} \times \f{2}{3}  = \boxed{\f{2}{3}}
\end{align*}



\noindent \textbf{5. Finding $B$ in 4.}\\

\noindent To start this problem, we find calculate the matrix for the entire circuit. This can be done by hand or in Mathematica. I prefer the second method:
\begin{align*}
	M = CNOT_{21} \cdot CNOT_{31} \cdot CNOT_{13} \cdot CNOT_{12} = 
	\begin{pmatrix}
		1& & & && & & \\
		 & & & && &1& \\
		 & & & &&1& & \\
		 & & &1&& & & \\
		 & & & && & & \\
		 &1& & && & &1\\
		 & &1& && & & \\
		 & & & &1&& &
	\end{pmatrix}.
\end{align*}




Mathematica code:
\begin{lstlisting}
	In[60]:= (*Problem 5*)
	
	In[63]:= SWAP = {{1, 0, 0, 0}, {0, 0, 1, 0}, {0, 1, 0, 0}, {0, 0, 0, 
			1}};
	
	In[61]:= CNOT12 = Tensor[CNOT, Id];
	
	In[62]:= CNOT21 = 
	Tensor[{{1, 0, 0, 0}, {0, 0, 0, 1}, {0, 0, 1, 0}, {0, 1, 0, 0}}, Id];
	
	In[65]:= CNOT13 = 
	Tensor[SWAP, Id] . Tensor[Id, CNOT] . Tensor[SWAP, Id];
	
	In[66]:= CNOT31 = 
	Tensor[Id, SWAP] . 
	Tensor[{{1, 0, 0, 0}, {0, 0, 0, 1}, {0, 0, 1, 0}, {0, 1, 0, 0}}, 
	Id] . Tensor[Id, SWAP];
	
	In[68]:= circ = CNOT21 . CNOT31 . CNOT13 . CNOT12;
	
	In[100]:= circ
	
	Out[100]= {{1, 0, 0, 0, 0, 0, 0, 0}, {0, 0, 0, 0, 0, 0, 1, 0}, {0, 0, 
			0, 0, 0, 1, 0, 0}, {0, 0, 0, 1, 0, 0, 0, 0}, {0, 0, 0, 0, 0, 0, 0, 
			1}, {0, 1, 0, 0, 0, 0, 0, 0}, {0, 0, 1, 0, 0, 0, 0, 0}, {0, 0, 0, 0,
			1, 0, 0, 0}}
\end{lstlisting}

Let $\ket{\psi} = x\ket{00} + y\ket{01} + z\ket{10} + w\ket{11} =  (x\quad y \quad z \quad w)^\top$. Then we have that
\begin{align*}
	&M\ket{t=0}\ket{\psi} = (x\quad 0 \quad 0 \quad w \quad 0 \quad y \quad z \quad 0)^\top = x\ket{000} + w\ket{011} + y\ket{101} + z\ket{110} \\
	&M\ket{t=1}\ket{\psi} = (0\quad z \quad y \quad 0 \quad w \quad 0 \quad 0 \quad x)^\top  = x\ket{111} + w\ket{100} + y \ket{010} + z\ket{001}
\end{align*}
We want the first row to be $B\ket{0}$ and the second row to be $B\ket{1}$. The solution is to set
\begin{align*}
	x= \sqrt{\f{2}{3}}, \quad y = w = \sqrt{\f{1}{6}}, \quad z = 0
\end{align*}
So, 
\begin{align*}
	\boxed{\ket{\psi} =  \sqrt{\f{2}{3}} \ket{00} + \sqrt{\f{1}{6}} \ket{01} + \sqrt{\f{1}{6}}\ket{11}  }
\end{align*}
The circuit now gives the desired result by construction. Given $t=0$, the output is $B\ket{0}$. Given $t=1$, the output is $B\ket{1}$. Since the Hilbert space of the input is only 2-dimensional, it suffices to only check what the circuit does on the basis states for the input.\\



\noindent Mathematica code:
\begin{lstlisting}
	In[101]:= t0 = {1, 0};
	
	In[102]:= psi = {x, y, z, w};
	
	In[109]:= circ . Flatten[Tensor[t0, psi]]
	
	Out[109]= {x, 0, 0, w, 0, y, z, 0}
	
	In[105]:= t1 = {0, 1};
	
	In[110]:= circ . Flatten[Tensor[t1, psi]]
	
	Out[110]= {0, z, y, 0, w, 0, 0, x}
\end{lstlisting}



\end{document}











