\documentclass{article}
\usepackage{physics}
\usepackage{graphicx}
\usepackage{caption}
\usepackage{amsmath}
\usepackage{bm}
\usepackage{framed}
\usepackage{authblk}
\usepackage{empheq}
\usepackage{amsfonts}
\usepackage{esint}
\usepackage[makeroom]{cancel}
\usepackage{dsfont}
\usepackage{centernot}
\usepackage{mathtools}
\usepackage{subcaption}
\usepackage{bigints}
\usepackage{amsthm}
\theoremstyle{definition}
\newtheorem{lemma}{Lemma}
\newtheorem{defn}{Definition}[section]
\newtheorem{prop}{Proposition}[section]
\newtheorem{rmk}{Remark}[section]
\newtheorem{thm}{Theorem}[section]
\newtheorem{exmp}{Example}[section]
\newtheorem{prob}{Problem}[section]
\newtheorem{sln}{Solution}[section]
\newtheorem*{prob*}{Problem}
\newtheorem{exer}{Exercise}[section]
\newtheorem*{exer*}{Exercise}
\newtheorem*{sln*}{Solution}
\usepackage{empheq}
\usepackage{tensor}
\usepackage{xcolor}
%\definecolor{colby}{rgb}{0.0, 0.0, 0.5}
\definecolor{MIT}{RGB}{163, 31, 52}
\usepackage[pdftex]{hyperref}
%\hypersetup{colorlinks,urlcolor=colby}
\hypersetup{colorlinks,linkcolor={MIT},citecolor={MIT},urlcolor={MIT}}  
\usepackage[left=1in,right=1in,top=1in,bottom=1in]{geometry}

\usepackage{newpxtext,newpxmath}
\newcommand*\widefbox[1]{\fbox{\hspace{2em}#1\hspace{2em}}}

\newcommand{\p}{\partial}
\newcommand{\R}{\mathbb{R}}
\newcommand{\C}{\mathbb{C}}
\newcommand{\lag}{\mathcal{L}}
\newcommand{\nn}{\nonumber}
\newcommand{\ham}{\mathcal{H}}
\newcommand{\M}{\mathcal{M}}
\newcommand{\I}{\mathcal{I}}
\newcommand{\K}{\mathcal{K}}
\newcommand{\F}{\mathcal{F}}
\newcommand{\w}{\omega}
\newcommand{\lam}{\lambda}
\newcommand{\al}{\alpha}
\newcommand{\be}{\beta}
\newcommand{\x}{\xi}

\newcommand{\G}{\mathcal{G}}

\newcommand{\f}[2]{\frac{#1}{#2}}

\newcommand{\ift}{\infty}

\newcommand{\lp}{\left(}
\newcommand{\rp}{\right)}

\newcommand{\lb}{\left[}
\newcommand{\rb}{\right]}

\newcommand{\lc}{\left\{}
\newcommand{\rc}{\right\}}


\newcommand{\V}{\mathbf{V}}
\newcommand{\U}{\mathcal{U}}
\newcommand{\Id}{\mathcal{I}}
\newcommand{\D}{\mathcal{D}}
\newcommand{\Z}{\mathcal{Z}}

%\setcounter{chapter}{-1}


\usepackage{enumitem}



\usepackage{listings}
\captionsetup[lstlisting]{margin=0cm,format=hang,font=small,format=plain,labelfont={bf,up},textfont={it}}
\renewcommand*{\lstlistingname}{Code \textcolor{violet}{\textsl{Mathematica}}}
\definecolor{gris245}{RGB}{245,245,245}
\definecolor{olive}{RGB}{50,140,50}
\definecolor{brun}{RGB}{175,100,80}

%\hypersetup{colorlinks,urlcolor=colby}
\lstset{
	tabsize=4,
	frame=single,
	language=mathematica,
	basicstyle=\scriptsize\ttfamily,
	keywordstyle=\color{black},
	backgroundcolor=\color{gris245},
	commentstyle=\color{gray},
	showstringspaces=false,
	emph={
		r1,
		r2,
		epsilon,epsilon_,
		Newton,Newton_
	},emphstyle={\color{olive}},
	emph={[2]
		L,
		CouleurCourbe,
		PotentielEffectif,
		IdCourbe,
		Courbe
	},emphstyle={[2]\color{blue}},
	emph={[3]r,r_,n,n_},emphstyle={[3]\color{magenta}}
}






\begin{document}
\begin{framed}
\noindent Name: \textbf{Huan Q. Bui}\\
Course: \textbf{8.370 - QC}\\
Problem set: \textbf{\#1}\\
Due: Wednesday, Sep 21, 2022\\
Collaborators: None
\end{framed}


\noindent \textbf{1. Useful properties of unitary matrices }

\begin{enumerate}[label=(\alph*)]
	\item Consider a $d$-dimension quantum space with orthonormal bases $\{ \ket{1}, \ket{2}, \dots, \ket{d}\}$ and $\{\ket{v_1}, \ket{v_2}, \dots, \ket{v_d}\}$. We shall construct a unitary matrix $U$ for which $U\ket{j} = \ket{v_j}$. To this end, we use the standard basis $\{ \ket{e_1}, \ket{e_2},\dots, \ket{e_d}\}$ as an intermediate basis. The matrix that transforms $\ket{e_j}$ to $\ket{j}$ is simply one whose $j$th-column has the components of $\ket{j}$ in the standard basis:
	\begin{align*}
		\ket{j} = U_A \ket{e_j} \,\, \forall j=1,2,\dots,d \quad \text{ if } \quad U_A = \begin{pmatrix}
			\uparrow & \uparrow& \dots & \uparrow\\
			\ket{1}& \ket{2}& \dots & \ket{d}\\
			\downarrow& \downarrow& \dots&\downarrow 
		\end{pmatrix}.
	\end{align*}
	Similarly, 
	\begin{align*}
		\ket{v_j} = U_B \ket{e_j} \,\, \forall j=1,2,\dots,d \quad \text{ if } \quad U_B = \begin{pmatrix}
			\uparrow & \uparrow& \dots & \uparrow\\
			\ket{v_1}& \ket{v_2}& \dots & \ket{v_d}\\
			\downarrow& \downarrow& \dots&\downarrow 
		\end{pmatrix}.
	\end{align*}
	Since the provided bases are orthonormal, it is clear by definition of $U_A$ and $U_B$ that $U_A^\dagger U_A = U_B^\dagger U_B = \mathbb{I}$, so both $U_A$ and $U_B$ are unitary. Our desired matrix $U$ is then given by 
	\begin{align*}
		U = U_B U_A^\dagger =  \begin{pmatrix}
			\uparrow & \uparrow& \dots & \uparrow\\
			\ket{v_1}& \ket{v_2}& \dots & \ket{v_d}\\
			\downarrow& \downarrow& \dots&\downarrow 
		\end{pmatrix} 
	\begin{pmatrix}
		\leftarrow & \bra{1} &\rightarrow \\
		\leftarrow & \bra{2} &\rightarrow \\
		\vdots & \vdots &\vdots \\
		\leftarrow & \bra{d} &\rightarrow 
	\end{pmatrix},
	\end{align*}
	which is also unitary since $U^\dagger U = U_A U_B^\dagger U_B U_A^\dagger = U_AU_A^\dagger = \mathbb{I}$. It is clear that $U\ket{j} = \ket{v_j}$, but to see explicitly, suppose we apply $U$ to $\ket{1}$. The application of $U_A^\dagger$ returns the column vector $\ket{e_1} = (1\,\, 0 \,\,0 \,\, \dots)^\top$. The subsequent application of $U_B$ therefore returns its first column, which is $\ket{v_1}$, as desired. 

	
	
	\item Let an orthonormal basis $\{ \ket{v_1},\ket{v_2},\dots,\ket{v_d} \}$ be given. In the standard basis $\{\ket{e_k}\}$, we may write
	\begin{align*}
		\ket{v_i} = \sum_{k=1}^d (v_i)_k \ket{e_k},
	\end{align*}
	so that
	\begin{align*}
		\sum_{i=1}^d \ket{v_i} \bra{v_i} = \sum_{i=1}^d \lb \sum_{k=1}^d (v_i)_k \ket{e_k}  \rb \lb \sum_{l=1}^d (v_i)^*_l \bra{e_l} \rb. 
	\end{align*}
	Using the fact that $\ket{e_m}\bra{e_n} = 0$ if $m\neq n$ and $\ket{e_m}\bra{e_m} = \Pi_m$ we have
	\begin{align*}
		\sum_{i=1}^d \ket{v_i} \bra{v_i} = \sum_{i=1}^d \sum_{k=1}^d  |(v_i)_k|^2 \Pi_k = \sum_{k=1}^d \sum_{i=1}^d  |(v_i)_k|^2 \Pi_k = \sum_{k=1}^d\Pi_k = \mathbb{I},
	\end{align*}
	where we have used the fact that the given basis is orthonormal in the third equality and resolution of identity with standard projections in the last equality. 
	
\end{enumerate}



\noindent \textbf{2. Angle between quantum states and angle between associated points on the Bloch sphere} 

\begin{enumerate}[label=(\alph*)]
	\item The point $p_i = (x_i,y_i,z_i)$ on the Bloch sphere is associated with the quantum state of a qubit $\ket{v_i}$ where
	\begin{align*}
		\ket{v_i}\bra{v_i} - \ket{\bar{v}_i} \bra{\bar{v}_i} = x_i \sigma_x + y_i \sigma_y + z_i \sigma_z
	\end{align*}
	where $\ket{\bar{v}_i}$ is orthogonal to $\ket{v_i}$. Because the quantum system is 2-dimensional and $\ket{v_i}\perp \ket{\bar{v_i}}$, we have that $\{ \ket{v_i}, \ket{\bar{v}_i} \}$ is an orthonormal basis. This implies
	\begin{align*}
		\ket{v_i}\bra{v_i} + \ket{\bar{v}_i}\bra{\bar{v}_i} = \mathbb{I}.
	\end{align*}
	Combine this with the equation above, we find that
	\begin{align*}
		\ket{v_i} \bra{v_i} = \f{\mathbb{I} + x_i \sigma_x + y_i \sigma_y + z_i \sigma_z }{2} = \f{\mathbb{I} + \vec{p}_i\cdot \vec{\sigma}}{2}.
	\end{align*}



	\item Using the fact that
	\begin{align*}
		\abs{\bra{v_1}\ket{v_2}}^2 = \bra{v_1}\ket{v_2} \bra{v_2}\ket{v_1} = \Tr\lp \ket{v_1} \bra{v_1}   \ket{v_2} \bra{v_2}  \rp,
	\end{align*}
	which can be proved using the cyclic property of the trace, we find that
	\begin{align*}
			\abs{\bra{v_1}\ket{v_2}}^2 = \Tr\lp  \f{\mathbb{I} + \vec{p}_1\cdot \vec{\sigma}}{2} \f{\mathbb{I} + \vec{p}_2\cdot \vec{\sigma}}{2}  \rp = \f{1+\vec{p}_1 \cdot \vec{p}_2}{2}. \quad\quad \text{(using Mathematica)}
	\end{align*}
	Let $\theta$ denote the angle between $\ket{v_1}$ and $\ket{v_2}$ and $\theta'$ denote the angle between $\vec{p}_1$ and $\vec{p}_2$, then 
	\begin{align*}
		\theta = \arccos \abs{\bra{v_1}\ket{v_2}} = \arccos \lp \sqrt{ \f{1 + \cos\theta'}{2} } \rp = \arccos \lp \abs{\cos\f{\theta'}{2}} \rp \to \f{\theta'}{2}
	\end{align*}
	If we ignore a possible minus sign due to relative orientation, the angle $\theta$ between quantum states is \textbf{half} the angle between associated points on the Bloch sphere. This makes sense, as \textit{orthogonal} quantum states occupy opposite poles on the Bloch sphere. \\
	
	
	
	


Mathematica code:
\begin{lstlisting}
	In[7]:= Id = {{1, 0}, {0, 1}};
	
	In[8]:= \[Sigma]x = PauliMatrix[1];
	
	In[9]:= \[Sigma]y = PauliMatrix[2];
	
	In[10]:= \[Sigma]z = PauliMatrix[3];
	
	In[11]:= \[Sigma] = {\[Sigma]x, \[Sigma]y, \[Sigma]z};
	
	In[15]:= p1 = {x1, y1, z1};
	
	In[16]:= p2 = {x2, y2, z2};
	
	In[32]:= M = (Id + Dot[p1, \[Sigma]]) . (Id + Dot[p2, \[Sigma]])/4;
	
	In[29]:= Tr[M] // Simplify
	
	Out[29]= 1/2 (1 + x1 x2 + y1 y2 + z1 z2)
\end{lstlisting}
\end{enumerate}


\noindent \textbf{3. \textbf{von Neumann measurement}}\\

We have 
\begin{align*}
	\ket{\psi} = \f{1}{\sqrt{3}} \ket{0} + \f{1+i}{\sqrt{3}} \ket{1}.
\end{align*} 
Suppose we make a von Neumann measurement in the basis 
\begin{align*}
	\lc \f{1}{\sqrt{2}} (\ket{0} + i\ket{1}), \f{1}{\sqrt{2}} (\ket{0} - i\ket{1})     \rc
\end{align*}
then
\begin{align*}
	&\Pr\lp  \f{1}{\sqrt{2}} (\ket{0} + i\ket{1}) \rp = \abs{\lp \f{1}{\sqrt{2}} (\bra{0} - i\bra{1}) \rp \lp \f{1}{\sqrt{3}} \ket{0} + \f{1+i}{\sqrt{3}} \ket{1} \rp }^2 = \abs{\f{1}{\sqrt{6}} - i \f{i+1}{\sqrt{3}}}^2 = \abs{\f{2-i}{\sqrt{6}}}^2 = \boxed{\f{5}{6}}\\
	&\Pr\lp \f{1}{\sqrt{2}} (\ket{0} - i\ket{1})  \rp = \abs{\lp \f{1}{\sqrt{2}} (\bra{0} + i\bra{1}) \rp\lp \f{1}{\sqrt{3}} \ket{0} + \f{1+i}{\sqrt{3}} \ket{1} \rp}^2 = \abs{\f{1}{\sqrt{6}} + i \f{1+i}{\sqrt{3}}}^2 = \abs{\f{i}{\sqrt{6}}}^2 = \boxed{\f{1}{6}} 
\end{align*}




\noindent \textbf{4. Qutrit}

\begin{enumerate}[label=(\alph*)]
	\item The quickest way to do this problem is writing the down matrix $U$ that transforms $\{\ket{0},\ket{1},\ket{2}\}$ to $\{\ket{a},\ket{b}, \ket{c}\}$ and check that it is unitary. Since unitary matrices are invertible and preserve orthonormality, we can conclude that $\{\ket{a},\ket{b}, \ket{c}\}$ is an orthonormal basis. From the problem statement, we immediately that 
	\begin{align*}
		U = \begin{pmatrix}
			1/2 & 1/2 & 1/\sqrt{2} \\
			1/\sqrt{2} & -1\sqrt{2} & 0 \\
			1/2 & 1/2 & -1/\sqrt{2}
		\end{pmatrix}
	\end{align*}
	Since $U^\dagger U = \mathbb{I}$ (checked in Mathematica), $U$ is indeed unitary and we're done. 
	
	
	
	Mathematica code:
	\begin{lstlisting}
		In[26]:= U = {{1/2, 1/2, 1/Sqrt[2]}, {1/Sqrt[2], -1/Sqrt[2], 0}, {1/2,
				1/2, -1/Sqrt[2]}};
		
		In[25]:= ConjugateTranspose[U] . U
		
		Out[25]= {{1, 0, 0}, {0, 1, 0}, {0, 0, 1}}
	\end{lstlisting}
	
	Alternatively, we could also verify that $\bra{a}\ket{b}= \bra{b}\ket{c}= \bra{c}\ket{a}=0$ and $\braket{a} = \braket{b} = \braket{c} = 1$:
	\begin{align*}
		&\bra{a}\ket{b} = \f{1}{2}\f{1}{2} - \f{1}{\sqrt{2}} \f{1}{\sqrt{2}} + \f{1}{2}\f{1}{2} = 0 \,\,\checkmark\\
		&\bra{b}\ket{c} = \f{1}{2}\f{1}{\sqrt{2}} - \f{1}{\sqrt{2}} 0 - \f{1}{2}\f{1}{\sqrt{2}} = 0 \,\,\checkmark \\
		&\bra{c}\ket{a} = \f{1}{\sqrt{2}}\f{1}{2} + 0 \f{1}{\sqrt{2}} - \f{1}{\sqrt{2}} \f{1}{2} = 0 \,\,\checkmark\\
		&\braket{a} = \f{1}{2}\f{1}{2} + \f{1}{\sqrt{2}} \f{1}{\sqrt{2}} + \f{1}{2}\f{1}{2} = 1 \,\,\checkmark\\
		&\braket{b} = \f{1}{2}\f{1}{2} + \f{-1}{\sqrt{2}} \f{-1}{\sqrt{2}} + \f{1}{2}\f{1}{2} = 1 \,\,\checkmark\\
		&\braket{c} = \f{1}{\sqrt{2}} \f{1}{\sqrt{2}} + \f{-1}{\sqrt{2}} \f{-1}{\sqrt{2}} = 0\,\,\checkmark 
	\end{align*}
	
	
	
	\item Given 
	\begin{align*}
		\ket{\psi} = \f{1}{\sqrt{3}} (\ket{0} + \ket{1} - \ket{2}),
	\end{align*}
	we find 
	\begin{align*}
		&\Pr(\ket{a}) = \abs{\bra{a}\ket{\psi}}^2 = \abs{ \f{1}{2}\f{1}{\sqrt{3}} + \f{1}{\sqrt{2}}\f{1}{\sqrt{3}} + \f{1}{2}\f{-1}{\sqrt{3}}}^2 = \boxed{\f{1}{6}} \\
		&\Pr(\ket{b}) = \abs{\bra{b}\ket{\psi}}^2 = \abs{ \f{1}{2}\f{1}{\sqrt{3}} + \f{-1}{\sqrt{2}}\f{1}{\sqrt{3}} + \f{1}{2}\f{-1}{\sqrt{3}}}^2 = \boxed{\f{1}{6}}  \\
		&\Pr(\ket{c}) = \abs{\bra{c}\ket{\psi}}^2 = \abs{\f{1}{\sqrt{2}}\f{1}{\sqrt{3}} +\f{-1}{\sqrt{2}}\f{-1}{\sqrt{3}}  }^2 = \boxed{\f{2}{3}} 
	\end{align*}
\end{enumerate}

\noindent \textbf{5. Perfect polarizing filter } 

\begin{enumerate}[label=(\alph*)]
	\item Suppose the incoming photons have polarization state $\ket{\psi} = \cos\theta \ket{H} + e^{i\phi}\sin\theta \ket{V}$ (where $H$ stands for horizontal and $V$ stands for vertical). The first polarizing filter is horizontal, so $\cos^2\theta$ of the photons make it through and become $\ket{H}$. The second filter is at 45$^\circ$ relative to the first polarizing filter (and thus $\ket{H}$). So, half the photons make it through and their state becomes $\ket{D}$ where $D$ stands for diagonal. A similar scenario applies for the last polarizing filter. So, \textbf{25\%} of photons coming out of the first polarizing filter make it out of the final one. 
	
	\item Intuition tells us that we get the most light through if the angle of the polarizing filter relative to the polarization state is as small as possible, so we want to put the 30$^\circ$-rotated polarizing filter after the horizontal polarizing filter, followed by the 60$^\circ$-rotated one. Let's check. 
	
	For the case described above, $\cos^2 30^\circ = 3/4$ of the photons coming through the horizontal filter make it through the 30$^\circ$-rotated filter. Next 60$^\circ$-rotated filter is at $60^\circ - 30^\circ = 30^\circ$ relative to the polarization state of photons leaving the 30$^\circ$-rotated filter, so we once again lose $1/4$ of the incoming photons. Finally, the vertical polarizing filter is at $30^\circ$ relative to the incoming photons, so we lose yet another $1/4$. So, 
	\begin{align*}
		\f{3}{4} \times \f{3}{4} \times \frac{3}{4} = \f{27}{64}
	\end{align*}
	of the photons coming out of the first polarizing filter make it out of the final one. Assuming the incoming photons are randomly polarized, we end up with $\textbf{27/128}$ of the initial light. \\
	
	
	On the other hand, if we put the $60^\circ$-rotated filter first, then only $\cos^2 60^\circ = 1/4$ of the photons leaving the horizontally rotated filter make it through. The subsequent $30^\circ$-rotated filter is $-30^\circ$ relative to the polarization state leaving the $60^\circ$-rotated filter, so $3/4$ of the incoming photon make it through. Finally, the vertical filter is at $60^\circ$ relative to the polarization state leaving the $30^\circ$ filter, so $1/4$ of the incoming photons make it through. We see that only 
	\begin{align*}
		\f{1}{4}\times \f{3}{4} \times \f{1}{4} = \f{3}{64}
	\end{align*}
	of the photons coming out of the first polarizing filter make it out of the final one, which is less than what we have in the first case.   
\end{enumerate}

\noindent \textbf{6. Copying a qubit?}

\begin{enumerate}[label=(\alph*)]
	\item After measuring the challenger's qubit in the $\{ \ket{0}, \ket{1} \}$ basis and and making two copies of the resulting state, we are guaranteed to hand the challenger either two qubits in state $\ket{0}$ or two qubits in state $\ket{1}$. 
	
	With probability $1/4$ the challenger gives us a qubit in state $\ket{0}$. After our measurement, we return two qubits in state $\ket{0}$ to the challenger with probability 1. The challenger measures both qubits in the $\{\ket{0}, \ket{1}\}$ basis and finds both to be in $\ket{0}$ with probability 1. We succeed. In this scenario we always succeed.
		
	A similar argument applies if the challenger gives us a qubit in $\ket{1}$. We also always succeed
	
	With probability $1/4$ the challenger gives us a qubit in $\ket{+}$. We measure this qubit and obtain $\ket{0}$ with probability $1/2$ and $\ket{1}$ with probability $1/2$. No matter which two qubits we give the challenger ($\ket{0}\ket{0}$ or $\ket{1}\ket{1}$), he measures and finds them both in $\ket{+}$ with probability $1/4$. In this scenario, we succeed $ 1/4 $ of the time.  
	
	A similar argument applies if the challenger gives us a qubit in $\ket{-}$. We also succeed with probability $1/4$ in this scenario. 
	
	In summary, we pass the test with probability
	\begin{align*}
		\Pr(\text{pass}) = \f{1}{4}\times 1 + \f{1}{4}\times 1 + \f{1}{4}\times \f{1}{4} + \f{1}{4}\times \f{1}{4} = \boxed{\f{5}{8}}
	\end{align*}
	
	\item Suppose that we now measure the provided qubits in the basis 
	\begin{align*}
		\lc \ket{\theta_+}, \ket{\theta_-} \rc \equiv \lc \cos\theta \ket{0} + \sin\theta \ket{1}, -\sin\theta\ket{0} + \cos\theta \ket{1} \rc.
	\end{align*}
	
	With probability $1/4$ the challenger gives us a qubit in $\ket{0}$. With probability $\cos^2\theta$ we find it in $\ket{\theta_+}$ and $\sin^2\theta$ in $\ket{\theta_-}$. If we give two $\ket{\theta_+}$ to the challenger, he finds them both in $\ket{0}$ with probability $\cos^4\theta$. If we give two $\ket{\theta_-}$, he finds them both in $\ket{1}$ with probability $\sin^4\theta$. So, in this scenario we succeed with probability $\cos^6\theta+\sin^6\theta$. 
	
	By the same analysis, we will also succeed the probability $\cos^6\theta+\sin^6\theta$ if the challenger give us $\ket{1}$. 
	
	With probability $1/4$ the challenger gives us a qubit in $\ket{+}$. With probability $(\cos\theta+\sin\theta)^2/2$ we find it in $\ket{\theta_+}$ and $(\cos\theta-\sin\theta)^2/2$ in $\ket{\theta_-}$. If we give two $\ket{\theta_+}$ to the challenger, he finds them both in $\ket{0}$ with probability $(\cos\theta+\sin\theta)^4/4$. If we give two $\ket{\theta_-}$, he finds them both in $\ket{1}$ with probability $(\cos\theta-\sin\theta)^4/4$. So, in this scenario we succeed with probability $(\cos\theta+\sin\theta)^6/8 + (\cos\theta-\sin\theta)^6/8$.\\
	
	By the same analysis, we will also succeed the probability $(\cos\theta+\sin\theta)^6/8 + (\cos\theta-\sin\theta)^6/8$ if the challenger give us $\ket{-}$. 
	
	In summary, we pass the test with probability:
	\begin{align*}
		\Pr(\text{pass}) =& \f{1}{4}\times (\cos^6\theta+\sin^6\theta) + \f{1}{4}\times (\cos^6\theta+\sin^6\theta) \\
		&\quad + \f{1}{4}\times \lb \f{(\cos\theta+\sin\theta)^6}{8} + \f{(\cos\theta-\sin\theta)^6}{8}\rb  + \f{1}{4}\times \lb \f{(\cos\theta+\sin\theta)^6}{8} + \f{(\cos\theta-\sin\theta)^6}{8}\rb\\
		=& \,\boxed{\f{5}{8}}
	\end{align*}
	
	\item From this analysis in Part (b), the choice of $\theta$ does not matter. 
\end{enumerate}



\end{document}








