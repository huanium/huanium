\documentclass{article}
\usepackage{physics}
\usepackage{graphicx}
\usepackage{caption}
\usepackage{amsmath}
\usepackage{bm}
\usepackage{framed}
\usepackage{authblk}
\usepackage{empheq}
\usepackage{amsfonts}
\usepackage{esint}
\usepackage[makeroom]{cancel}
\usepackage{dsfont}
\usepackage{centernot}
\usepackage{mathtools}
\usepackage{subcaption}
\usepackage{bigints}
\usepackage{amsthm}
\theoremstyle{definition}
\newtheorem{lemma}{Lemma}
\newtheorem{defn}{Definition}[section]
\newtheorem{prop}{Proposition}[section]
\newtheorem{rmk}{Remark}[section]
\newtheorem{thm}{Theorem}[section]
\newtheorem{exmp}{Example}[section]
\newtheorem{prob}{Problem}[section]
\newtheorem{sln}{Solution}[section]
\newtheorem*{prob*}{Problem}
\newtheorem{exer}{Exercise}[section]
\newtheorem*{exer*}{Exercise}
\newtheorem*{sln*}{Solution}
\usepackage{empheq}
\usepackage{tensor}
\usepackage{xcolor}
%\definecolor{colby}{rgb}{0.0, 0.0, 0.5}
\definecolor{MIT}{RGB}{163, 31, 52}
\usepackage[pdftex]{hyperref}
%\hypersetup{colorlinks,urlcolor=colby}
\hypersetup{colorlinks,linkcolor={MIT},citecolor={MIT},urlcolor={MIT}}  
\usepackage[left=1in,right=1in,top=1in,bottom=1in]{geometry}

\usepackage{newpxtext,newpxmath}
\newcommand*\widefbox[1]{\fbox{\hspace{2em}#1\hspace{2em}}}

\newcommand{\p}{\partial}
\newcommand{\R}{\mathbb{R}}
\newcommand{\C}{\mathbb{C}}
\newcommand{\lag}{\mathcal{L}}
\newcommand{\nn}{\nonumber}
\newcommand{\ham}{\mathcal{H}}
\newcommand{\M}{\mathcal{M}}
\newcommand{\I}{\mathcal{I}}
\newcommand{\K}{\mathcal{K}}
\newcommand{\F}{\mathcal{F}}
\newcommand{\w}{\omega}
\newcommand{\lam}{\lambda}
\newcommand{\al}{\alpha}
\newcommand{\be}{\beta}
\newcommand{\x}{\xi}

\newcommand{\G}{\mathcal{G}}

\newcommand{\f}[2]{\frac{#1}{#2}}

\newcommand{\ift}{\infty}

\newcommand{\lp}{\left(}
\newcommand{\rp}{\right)}

\newcommand{\lb}{\left[}
\newcommand{\rb}{\right]}

\newcommand{\lc}{\left\{}
\newcommand{\rc}{\right\}}


\newcommand{\V}{\mathbf{V}}
\newcommand{\U}{\mathcal{U}}
\newcommand{\Id}{\mathcal{I}}
\newcommand{\D}{\mathcal{D}}
\newcommand{\Z}{\mathcal{Z}}

%\setcounter{chapter}{-1}


\usepackage{enumitem}



\usepackage{listings}
\captionsetup[lstlisting]{margin=0cm,format=hang,font=small,format=plain,labelfont={bf,up},textfont={it}}
\renewcommand*{\lstlistingname}{Code \textcolor{violet}{\textsl{Mathematica}}}
\definecolor{gris245}{RGB}{245,245,245}
\definecolor{olive}{RGB}{50,140,50}
\definecolor{brun}{RGB}{175,100,80}

%\hypersetup{colorlinks,urlcolor=colby}
\lstset{
	tabsize=4,
	frame=single,
	language=mathematica,
	basicstyle=\scriptsize\ttfamily,
	keywordstyle=\color{black},
	backgroundcolor=\color{gris245},
	commentstyle=\color{gray},
	showstringspaces=false,
	emph={
		r1,
		r2,
		epsilon,epsilon_,
		Newton,Newton_
	},emphstyle={\color{olive}},
	emph={[2]
		L,
		CouleurCourbe,
		PotentielEffectif,
		IdCourbe,
		Courbe
	},emphstyle={[2]\color{blue}},
	emph={[3]r,r_,n,n_},emphstyle={[3]\color{magenta}}
}






\begin{document}
\begin{framed}
\noindent Name: \textbf{Huan Q. Bui}\\
Course: \textbf{8.370 - QC}\\
Problem set: \textbf{\#9}\\
Due: Wednesday, Dec 7, 2022\\
Collaborators/References: Piazza
\end{framed}



\noindent \textbf{1. The SWAP test} \\


\noindent The SWAP test tests whether two pure quantum states $\ket{\phi}$ and $\ket{\psi}$ are the same. Before the measurement in the first qubit is made, the circuit does the following (ignoring normalization):
\begin{align*}
	\ket{+} \ket{\psi} \ket{\phi}  &\to \ket{0}\ket{\psi}\ket{\phi} + \ket{1}\ket{\phi}\ket{\psi}\\
	&\to  \ket{+}\ket{\psi}\ket{\phi}+ \ket{-}\ket{\phi}\ket{\psi}
\end{align*}

\begin{enumerate}[label=(\alph*)]
	\item If $\ket{\phi} = \ket{\psi}$, then the state of the circuit before the measurement is 
	\begin{align*}
		(\ket{+} + \ket{-}) \ket{\psi}\ket{\psi} = \ket{0} \ket{\psi} \ket{\psi}.
	\end{align*}
	So the probability that we observe $\ket{0}$ in the first wire is $\boxed{1}$.
	
	\item The state of the circuit before the measurement is 
	\begin{align*}
		\f{1}{2}\ket{0} \lp \ket{\psi}\ket{\phi} + \ket{\phi}\ket{\psi} \rp + \f{1}{2} \ket{1}\lp \ket{\psi}\ket{\phi} - \ket{\phi}\ket{\psi} \rp.
	\end{align*}
	The probability that we observe $\ket{0}$ in the first wire is 
	\begin{align*}
		\f{1}{4}\lp \bra{\psi}\bra{\phi} + \bra{\phi}\bra{\psi} \rp \lp \ket{\psi}\ket{\phi} + \ket{\phi}\ket{\psi} \rp
		= \f{1}{4}\lp 1 + 1 \rp = \boxed{\f{1}{2}} 
	\end{align*}
	
	\item Suppose we apply the SWAP test with the inputs being two identical density matrices:
	\begin{align*}
		\rho_1 = \rho_2= p \ket{0}\bra{0} + (1-p) \ket{1}\bra{1}.
	\end{align*}
	We can do this problem probabilistically. The initial states of the circuit and associated probabilities are:
	\begin{align*}
		\Pr(\ket{+}\ket{0}\ket{0}) &= p^2 \\
		\Pr(\ket{+}\ket{0}\ket{1}) &= p(1-p) \\
		\Pr(\ket{+}\ket{1}\ket{0}) &= (1-p)p \\  
		\Pr(\ket{+}\ket{1}\ket{1}) &= (1-p)(1-p)
	\end{align*}
	From the previous parts, the probability that we observe $\ket{0}$ on the top wire is 
	\begin{align*}
		p^2 + (1-p)(1-p) + \f{1}{2}\lb p(1-p) + (1-p)p \rb =\boxed{ 1 -p + p^2}.
	\end{align*}
	
\end{enumerate}


\noindent \textbf{2. $kl$-qubit code} \\

\noindent The generalization of the 9-qubit code to a $kl$-qubit code has the codewords:
\begin{align*}
	\ket{0}_L = \f{1}{2^{l/2}} ( |\underbrace{000\dots 0}_{k}\rangle + |\underbrace{111\dots 1}_{k}\rangle )^{\otimes l} \\
	\ket{1}_L = \f{1}{2^{l/2}} ( |\underbrace{000\dots 0}_{k}\rangle - |\underbrace{111\dots 1}_{k}\rangle )^{\otimes l}
\end{align*}
Here $k,l$ are odd numbers. In the known case where $k=l=3$, we know that the code can correct 1 bit error and 1 phase error. To correct the bit error we need to measure 2 syndrome bits. To correct the phase error we also need to  measure 2 syndrome bits. 





\noindent \textbf{3.}

\begin{enumerate}[label=(\alph*)]
	\item 
	
	\item 
	
\end{enumerate}


\noindent \textbf{4. }

\begin{enumerate}[label=(\alph*)]
	\item 
	
	\item 
	
	\item 
	
	\item 
\end{enumerate}




\end{document}











