\documentclass{article}
\usepackage{physics}
\usepackage{graphicx}
\usepackage{caption}
\usepackage{amsmath}
\usepackage{bm}
\usepackage{framed}
\usepackage{authblk}
\usepackage{empheq}
\usepackage{amsfonts}
\usepackage{esint}
\usepackage[makeroom]{cancel}
\usepackage{dsfont}
\usepackage{centernot}
\usepackage{mathtools}
\usepackage{subcaption}
\usepackage{bigints}
\usepackage{amsthm}
\theoremstyle{definition}
\newtheorem{lemma}{Lemma}
\newtheorem{defn}{Definition}[section]
\newtheorem{prop}{Proposition}[section]
\newtheorem{rmk}{Remark}[section]
\newtheorem{thm}{Theorem}[section]
\newtheorem{exmp}{Example}[section]
\newtheorem{prob}{Problem}[section]
\newtheorem{sln}{Solution}[section]
\newtheorem*{prob*}{Problem}
\newtheorem{exer}{Exercise}[section]
\newtheorem*{exer*}{Exercise}
\newtheorem*{sln*}{Solution}
\usepackage{empheq}
\usepackage{tensor}
\usepackage{xcolor}
%\definecolor{colby}{rgb}{0.0, 0.0, 0.5}
\definecolor{MIT}{RGB}{163, 31, 52}
\usepackage[pdftex]{hyperref}
%\hypersetup{colorlinks,urlcolor=colby}
\hypersetup{colorlinks,linkcolor={MIT},citecolor={MIT},urlcolor={MIT}}  
\usepackage[left=1in,right=1in,top=1in,bottom=1in]{geometry}

\usepackage{newpxtext,newpxmath}
\newcommand*\widefbox[1]{\fbox{\hspace{2em}#1\hspace{2em}}}

\newcommand{\p}{\partial}
\newcommand{\R}{\mathbb{R}}
\newcommand{\C}{\mathbb{C}}
\newcommand{\lag}{\mathcal{L}}
\newcommand{\nn}{\nonumber}
\newcommand{\ham}{\mathcal{H}}
\newcommand{\M}{\mathcal{M}}
\newcommand{\I}{\mathcal{I}}
\newcommand{\K}{\mathcal{K}}
\newcommand{\F}{\mathcal{F}}
\newcommand{\w}{\omega}
\newcommand{\lam}{\lambda}
\newcommand{\al}{\alpha}
\newcommand{\be}{\beta}
\newcommand{\x}{\xi}

\newcommand{\G}{\mathcal{G}}

\newcommand{\f}[2]{\frac{#1}{#2}}

\newcommand{\ift}{\infty}

\newcommand{\lp}{\left(}
\newcommand{\rp}{\right)}

\newcommand{\lb}{\left[}
\newcommand{\rb}{\right]}

\newcommand{\lc}{\left\{}
\newcommand{\rc}{\right\}}


\newcommand{\V}{\mathbf{V}}
\newcommand{\U}{\mathcal{U}}
\newcommand{\Id}{\mathcal{I}}
\newcommand{\D}{\mathcal{D}}
\newcommand{\Z}{\mathcal{Z}}

%\setcounter{chapter}{-1}


\usepackage{enumitem}



\usepackage{listings}
\captionsetup[lstlisting]{margin=0cm,format=hang,font=small,format=plain,labelfont={bf,up},textfont={it}}
\renewcommand*{\lstlistingname}{Code \textcolor{violet}{\textsl{Mathematica}}}
\definecolor{gris245}{RGB}{245,245,245}
\definecolor{olive}{RGB}{50,140,50}
\definecolor{brun}{RGB}{175,100,80}

%\hypersetup{colorlinks,urlcolor=colby}
\lstset{
	tabsize=4,
	frame=single,
	language=mathematica,
	basicstyle=\scriptsize\ttfamily,
	keywordstyle=\color{black},
	backgroundcolor=\color{gris245},
	commentstyle=\color{gray},
	showstringspaces=false,
	emph={
		r1,
		r2,
		epsilon,epsilon_,
		Newton,Newton_
	},emphstyle={\color{olive}},
	emph={[2]
		L,
		CouleurCourbe,
		PotentielEffectif,
		IdCourbe,
		Courbe
	},emphstyle={[2]\color{blue}},
	emph={[3]r,r_,n,n_},emphstyle={[3]\color{magenta}}
}






\begin{document}
\begin{framed}
\noindent Name: \textbf{Huan Q. Bui}\\
Course: \textbf{8.370 - QC}\\
Problem set: \textbf{\#5}\\
Due: Wednesday, Oct 26, 2022\\
Collaborators/References: 
\end{framed}




\noindent \textbf{1. Teleportation} \\

\noindent Suppose the qubit Alice wants to teleport has state $\ket{\psi}_A = \al\ket{0}+ \be\ket{1}$. If Alice and Bob start with $(1/\sqrt{2})(\ket{01}_{AB} - \ket{10}_{AB})$ then Bob's states for each of Alice's measurement outcome and Bob's required unitary transformations are
\begin{align*}
	&\f{1}{\sqrt{2}}(\ket{00} + \ket{11}) \to \ket{\psi}_B = +\al\ket{1} - \be\ket{0} \implies \sigma_y\\
	&\f{1}{\sqrt{2}}(\ket{01} + \ket{10}) \to \ket{\psi}_B = -\al\ket{0} + \be\ket{1} \implies \sigma_z \\
	&\f{1}{\sqrt{2}}(\ket{00} - \ket{11}) \to \ket{\psi}_B = +\al\ket{1} + \be\ket{0} \implies \sigma_x\\
	&\f{1}{\sqrt{2}}(\ket{01} - \ket{10}) \to \ket{\psi}_B = +\al\ket{0} + \be\ket{1} \implies \text{id}\\
\end{align*}
Of course, another way to do this problem is to notice that the new EPR pair which Alice and Bob share is the original EPR pair transformed by $\sigma_y$. Therefore, Bob needs to apply a $\sigma_y$ to his original unitary transformations to obtain the new unitary transformations. \\



\noindent \textbf{2. Deutsch-Jozsa}\\ 


In the Deutsch-Jozsa algorithm, the probability for a state $\ket{k}$ to be measured is 
\begin{align*}
	\abs{\f{1}{2^n} \sum^{2^n-1}_{j=0}  (-1)^{f(j)}  (-1)^{j\cdot k} }^2
\end{align*}
In the algorithm, we are interested in the probability of measuring $k=0$, which is 
\begin{align*}
	\Pr = \abs{\f{1}{2^n} \sum^{2^n-1}_{j=0}  (-1)^{f(j)}}^2.
\end{align*}
For $f(j)$ constant, this quantity is $1$. For $f(j)$ balanced, this quantity is $0$. \\

In the case that $f(j) = 0$ for $r$ times and $f(j)=1$ for $s$ times, the probability that the algorithm tells us that $f$ is constant is the probability of measuring $\ket{0}$, which is:
\begin{align*}
	\Pr = \abs{\f{1}{2^n} (r-s)   }^2 = \boxed{\f{(r-s)^2}{2^{2n}}}
\end{align*}

\newpage

\noindent \textbf{3. Two qubits}

\begin{enumerate}[label=(\alph*)]
	\item It suffices to check that the states are mutually orthogonal:
	\begin{align*}
		&\bra{00}\ket{01} = \braket{0}\bra{0}\ket{1} = 0\\
		&\bra{00}\ket{1+} = \bra{0}\ket{1}\bra{0}\ket{+} = 0 \\
		&\bra{00}\ket{1-} = \bra{0}\ket{1}\bra{0}\ket{-} = 0\\
		&\bra{01}\ket{1+} = \bra{0}\ket{1}\bra{1}\ket{+} = 0 \\
		&\bra{01}\ket{1-} = \bra{1}\ket{1}\bra{1}\ket{-} = 0\\
		&\bra{1+}\ket{1-} = \bra{1}\ket{1}\bra{+}\ket{-} =0. 
	\end{align*}
	Since the states also span the 4-dimensional Hilbert space for states of a 2-qubit system, the states provided form an orthonormal basis. 
	
	\item Since each of the given states are product states, Alice and Bob can make sequential measurements to find what state their qubits are in originally. Consider the following procedure: Alice makes a measurement in the $\{\ket{0},\ket{1}\}$ basis. If she sees $\ket{0}$, she calls Bob and tells him to measure his qubit in the $\{\ket{0}, \ket{1}\}$ basis. Else if Alice sees $\ket{1}$, then she calls and tells Bob to measure his qubit in the $\{ \ket{+}, \ket{-}  \}$ basis. They will be able to unambiguously identify the state they had in the beginning.
	
	\item Alice can always tell with certainty the state of her qubit by measuring in the $\{  \ket{0},\ket{1} \}$ basis. However, Bob cannot, simply because $\ket{\pm}$ is a superposition of $\ket{0}$ and $\ket{1}$ and vice versa. There is no measurement basis with which Bob (alone) can use to unambiguously identify the state of his qubit.  One can also answer this question using density matrices, which I think is less convenient, as one has to assign probabilities to the possibilities given in the problem, etc.
	
	\item Yes. The strategy is as follows: Bob applies a CNOT on the two qubits that he has. The control qubit is that of his half of the EPR pair, and the target qubit is his qubit. Alice measures her qubit in the $\{\ket{0}, \ket{1}\}$ basis. If she sees that she has a $\ket{0}$ then she measures her half of the EPR pair in the $\{\ket{0}, \ket{1}\}$ basis. Else, she measures her half of the EPR pair in the $\{ \ket{+}, \ket{-} \}$ basis. Meanwhile, Bob measures his qubits in the $\{ \ket{0}, \ket{1}   \}$ basis and his half of the EPR pair in the $\{ \ket{+}, \ket{-}  \}$ basis. To see how this works, we first consider the starting state. The four possible initial states of the problem are (here I am writing everything without normalization... I'm also dropping the kets to clarify everything):
	\begin{align*}
		\ket{a}	&= \ket{0_A0_B} (\ket{0_A0_B} + \ket{1_A 1_B}) =  (00)_A (00)_B + (01)_A (01)_B \\
		\ket{b} &= \ket{0_A1_B} (\ket{0_A0_B} + \ket{1_A 1_B}) =  (00)_A (10)_B + (01)_A (11)_B \\
		\ket{c} &= \ket{1_A+_B} (\ket{0_A0_B} + \ket{1_A 1_B}) =  (10)_A (+0)_B + (11)_A (+1)_B  \\
		\ket{d} &= \ket{1_A-_B} (\ket{0_A0_B} + \ket{1_A 1_B}) =  (10)_A (-0)_B + (11)_A (-1)_B 
	\end{align*}
	Here, the Alice's second qubit comes from her half of the EPR pair. Bob's second qubit comes from his half of the EPR pair. Now, Bob performs a CNOT where the control qubit is from his half of the EPR pair. After this application, the possible states become:
	\begin{align*}
		\ket{a'} &= \ket{0_A0_B} (\ket{0_A0_B} + \ket{1_A 1_B}) =  (00)_A (00)_B + (01)_A (11)_B \\
		\ket{b'} &= \ket{0_A1_B} (\ket{0_A0_B} + \ket{1_A 1_B}) =  (00)_A (10)_B + (01)_A (01)_B \\
		\ket{c'} &= \ket{1_A+_B} (\ket{0_A0_B} + \ket{1_A 1_B}) =  (10)_A (+0)_B + (11)_A (+1)_B  \\
		\ket{d'} &= \ket{1_A-_B} (\ket{0_A0_B} + \ket{1_A 1_B}) =  (10)_A (-0)_B - (11)_A (-1)_B 
	\end{align*}
	The table below shows the Bob's resulting states after Alice makes her measurement:\\
	
	\begin{center}
	\begin{tabular}{| c | c | c | c  | c | }
		\hline
		 			& $\ket{a'}$ &  $\ket{b'}$ &  $\ket{c'}$ &  $\ket{d'}$ \\
		 			\hline
		 $\bra{00}_A$	& $\ket{00}_B$  & $\ket{10}_B$  & $\varnothing$& $\varnothing$\\
		 $\bra{01}_A$	& $\ket{11}_B$  & $\ket{01}_B$  & $\varnothing$& $\varnothing$\\
		 $\bra{1+}_A$	& $\varnothing$ & $\varnothing$ & $\ket{++}_B$ & $\ket{--}_B$\\
		 $\bra{1-}_A$	& $\varnothing$ & $\varnothing$ & $\ket{+-}_B$ & $\ket{-+}_B$\\
		 \hline
	\end{tabular}
\end{center}

Notice that if Bob measures his first qubit (which is the qubit he was given originally) in the $\{ \ket{0}, \ket{1}  \}$ basis and his second qubit (which is his half of the EPR pair) in the $\{ \ket{+}, \ket{-} \}$ basis, then after communicating with Alice after the measurement, they can work out which of the four possibilities $(\ket{a}, \ket{b}, \ket{c}, \ket{d})$ they were given originally, based on the this table. \\

For example, if Bob measures $\ket{0}$ on his first qubit and Alice measures $\ket{01}$, then they know they were given $\ket{0}\ket{1}$, which becomes $\ket{b}$ after we concatenate it with the EPR pair. 


	
	
\end{enumerate}


\noindent \textbf{4. GHZ}\\

Alice, Bob, and Charlie hold the GHZ state:
\begin{align*}
	\f{1}{\sqrt{2}} \lp \ket{000}_{ABC} + \ket{111}_{ABC} \rp
\end{align*}


\begin{enumerate}[label=(\alph*)]
	
	\item In order to arrange so that Alice and Charlie share the Bell pair $(1/\sqrt{2})(\ket{00} + \ket{11})$, Alice, Bob, and Charlie must somehow ``factor'' Bob's qubit out of the GHZ state. This requires a controlled-unitary gate or a classical controlled-operation whose control bit is Alice's or Charlie's qubit and target bit is Bob's. This is not possible if the trio are not allowed to communicate. One can also answer this part using density matrices. 
	
	\item For Alice and Charlie to have the Bell pair state in the problem, Bob first applies a Hadamard on his qubit, so that the system becomes
	\begin{align*}
		\f{1}{\sqrt{2}}\lp \ket{0+0} + \ket{1-1} \rp = \f{1}{2}\lp \ket{000} + \ket{010} + \ket{101} - \ket{111}  \rp.
	\end{align*}
	Now Bob measures his qubit in the $\{ \ket{0}, \ket{1}  \}$ basis. The system after this measurement is in one of the two states:
	\begin{align*}
		\f{1}{\sqrt{2}}\lp \ket{00} + \ket{11} \rp_{AC}\ket{0}_B \quad\quad\quad \f{1}{\sqrt{2}}\lp \ket{00} - \ket{11} \rp_{AC} \ket{1}_B
	\end{align*}
	with equal probability. If Bob measures $\ket{0}$, he does nothing, Alice and Charlie already share the correct EPR state. If Bob measures $\ket{1}$, then he lets either Charlie or Alice knows so that one of them applies a $\sigma_z$ on their qubit, so that $(\ket{00} - \ket{11})_{AC} \to (\ket{00} + \ket{11})_{AC}$. 
\end{enumerate}


\end{document}











