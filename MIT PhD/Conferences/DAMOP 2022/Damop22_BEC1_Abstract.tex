% See the REVTeX 4 README file for restrictions and more information.
%
% TeX'ing this file requires that you have AMS-LaTeX 2.0 installed.
% It also requires running BibTeX. The commands are as follows:
%
%  1)  latex apssamp.tex
%  2)  bibtex apssamp
%  3)  latex apssamp.tex
%  4)  latex apssamp.tex
%
%\documentclass[twocolumn,showpacs]{revtex4}
\documentclass[preprint,aps]{revtex4}

% Some other (several out of many) possibilities
%\documentclass[preprint,eqsecnum,aps]{revtex4}
%\documentclass[eqsecnum,aps,draft]{revtex4}
%\documentclass[prb]{revtex4}% Physical Review B

\usepackage{times}%
%\usepackage{graphicx}%
%\usepackage{dcolumn}
\usepackage{amsmath}
%\usepackage{textcomp}
\usepackage{xcolor}
%\special{papersize=8.5 in,11 in}






\begin{document}
\bibliographystyle{apsrev}
%\preprint{HEP/123-qed}

\title{Hydrodynamic properties of the Unitary Fermi Gas}


\author{Eric Wolf}%
\email{eawolf@mit.edu}%
\author{Huan Q. Bui}%
\email{huanbui@mit.edu}%
\author{Parth B. Patel}
\author{Zhenjie Yan}
\author{Biswaroop Mukherjee}
\author{Martin W. Zwierlein}


\affiliation{MIT-Harvard Center for Ultracold Atoms, Research Laboratory of Electronics, and Department of Physics, Massachusetts Institute of Technology, Cambridge, Massachusetts 02139, USA}
%Abstract log number: 
%

\date{\today}% It is always \today, today, but you may specify any date with \date.

\begin{abstract}
The unitary, contact-interacting Fermi gas is challenging to treat numerically or analytically, but is also relevant to a wide variety of physical systems thanks to its scale invariance. We prepare a spin-balanced, homogeneous gas of fermionic $^6$Li, trapped within a box potential formed by blue-detuned light. We observe the response of the gas to local density and temperature perturbations in both the normal and superfluid phases and extract the associated diffusivities. These diffusivities are at a Heisenberg limit $\sim \frac{\hbar}{m}$ and contrast with the predictions of Fermi liquid theory, informing new models.

\texttt{Opening sentence feels a bit hmmm? Let's try...}

\textcolor{blue}{Experimental studies on the strongly interacting unitary Fermi gas not only reveal properties challenging to obtain analytically or numerically, but are also relevant to a wide variety of physical systems thanks to its scale invariance. In this work, we prepare a spin-balanced, homogeneous gas of fermionic $^6$Li in the unitary limit, trapped within a box potential formed by blue-detuned light. We observe the response of the gas to local density and temperature perturbations in both the normal and superfluid phases and extract the associated diffusivities. These diffusivities attain a Heisenberg limit $\sim \hbar/m$ and contrast with the predictions of Fermi liquid theory, informing new models for strongly interacting fermionic matter.   }

This work was supported by the National Science Foundation (Center for Ultracold Atoms Awards No. PHY-1734011 and No. PHY-1506019), Air Force Office of Scientific Research (FA9550-16-1-0324 and MURI Quantum Phases of Matter FA9550-14-1-0035), Office of Naval Research (N00014-17-1-2257) and the David and Lucile Packard Foundation.

\end{abstract}

\pacs{32.80.Rm 32.30.Bv 32.60.+i 42.50.Hz}% PACS, the Physics and Astronomy Classification Scheme.

%\maketitle must follow title, authors, abstract and \pacs
\maketitle


\end{document}
