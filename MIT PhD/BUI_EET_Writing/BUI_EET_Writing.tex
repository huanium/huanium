\documentclass[12pt]{article}
\usepackage[left=1in, right=1in,top=1.25in,bottom=1in]{geometry}

%\usepackage{newpxtext,newpxmath}
%\usepackage{amsfonts}
\usepackage{tgtermes}
\usepackage{amsmath}
\usepackage{color}
\usepackage{bold-extra}
\usepackage{setspace}
\usepackage{framed}
%\pagenumbering{gobble}
\setstretch{1.5}
%\onehalfspacing


%\fontsize{12pt}{15pt}\selectfont

\usepackage{fancyhdr}
\pagestyle{fancy}
\lhead{Huan Q. \underline{Bui}}
\rhead{MIT\# 915434286}
%\cfoot{center of the footer!}
%\renewcommand{\headrulewidth}{0.4pt}
%\renewcommand{\footrulewidth}{0.4pt}

\begin{document}
%\noindent Huan Quang Bui \hfill Application for MIT Physics
\begin{center}
	\textbf{EET Writing Task: Prompt}
\end{center}

\begin{framed}
	\noindent \textbf{Topic B:}  In 2020, the COVID-19 pandemic transformed college teaching and learning from in-person to largely online. The platform Zoom alone jumped from some 10 million users in December 2019 to over 300 million by April 2020. But as Stanford University Communications Professor Jeremy Bailenson writes, ``something about being on video conference all day seems particularly exhausting,'' and the term ``Zoom fatigue'' was coined and spread quickly (1). (In addition, over 94\% of 325 undergraduates surveyed reported moderate to extreme difficulty with online learning (Peper 47).) However, this exhaustion has not been fully explained, although many theories have been offered.\\
	
	
	
	\noindent Based on your experience with virtual learning (or videoconferencing), what do you think the main causes of Zoom fatigue are—and its major effects? And although it may be unlikely, if MIT should suddenly have to go back to completely online learning this year, how could the school, instructors, or the designers of videoconferencing software such as Zoom prevent or moderate this fatigue?\\
	
	
	
	\noindent To help you get started, consider factors such as the ones below as well as those you have experienced. Then ask yourself how these problems might be reduced and write a focused, unified, and coherent response that could serve as an opinion piece in the MIT student newspaper The Tech. Include your experience of the last year at your undergraduate institution, place of work, or general environment (e.g., at home in your country) as evidence for your claims.
	
	\begin{itemize}
		\item excessive close-up gazing by the eye
		\item increased cognitive load
		\item increased self-evaluation (from seeing oneself on video)
		\item constraints on physical mobility
	\end{itemize}
	
	\noindent Do not do any outside research in writing your essay.
\end{framed}


\newpage


\begin{center}
	\textbf{EET Writing Task: Response}
\end{center}


We have all been there: snoozing our alarm until it is moments before the maliciously early Zoom call begins and leaving just enough seconds to freshen up and put on a half-decent shirt. Sweatpants on and snacks strategically placed outside the field of view of the webcam, we are ready as ever, in the comfort of our home, to take on yet another Zoom call. But are we really? For even with our home advantage, why do we feel so exhausted from being on video conference calls?\\


Those of us who have used Zoom for school or work over the course of the pandemic must have experienced some level of Zoom fatigue. It is thus important to find out what causes it and ways to mitigate it so that we could all be more productive. While there has not been a complete explanation for this effect, personally I have noticed two things which have generated physical and mental stress on myself and eventually made me feel  drained after long video conferences. \\


One, I become overly self-aware and eventually fail to maintain a high level of attentiveness. For me, this is a corollary of two effects: excessive cognitive load and constant self-evaluation. When I am tasked with processing a large amount of information in a short time such as making sure my finicky audio, video, and shared slides are functioning properly while maintaining the natural flow of my speech, my body tends to tense up, and my brain often goes into a hyper-focus mode. While achievable in short bursts, I simply cannot sustain such a high level of alertness over many hours. In addition, I find myself constantly checking how my video appears on the call window: Am I out of frame? Is there something in the background that I don't want someone in the call to see? Am I acting in a way that makes me look like I am not paying attention to what the speaker is saying? These ``background processes'' accumulate and eventually take up a significant portion of my mental bandwidth. \\








Two, I find myself constantly getting distracted and working hard to regain focus. A consequence of having to be highly focused on video call is that any small disruption, which my mind is actively trying to reject, can take me out of that fragile mental state. I was once in an Zoom interview as part of my graduate school application when my roommate knocked on my door despite an ``Interview in Progress'' sign. While my roommate eventually stopped, I not only was surprised and lost my train of thought, but was also left wondering for the rest of the interview whether he would come back and knock again. It was a mental struggle to regain my focus after that event. \\








While there are inevitable effects such as extensive screen time and constrained physical mobility due to being on a long video conference, I believe there are solutions which instructors, designers of videoconferencing software, and the users could implement to moderate mental Zoom fatigue. The simplest is to have small and frequent breaks over long calls. For every 50-minute session, conference organizers may allow a few minutes for participants to stand up, stretch, and take their eyes of their screens. This should be more than enough to reduce the physical strain of having to sit at one spot and allow participants to freshen up mentally. Software developers may also streamline the design of their videoconference application so that it is easier and less distracting to use. Moreover, the user could learn to cope with or reduce Zoom fatigue themselves by paying attention to what works for them and what doesn't. Personally, I find turning off all notifications from my devices allows me to focus better. A good internet connection is also helpful since it improves the call quality and reduces the probability of freezes or crashes, allowing me to take my mind of potential technical issues.  
















	
	
	
	
	
\end{document}