\documentclass[12pt]{article}
\usepackage[left=1in, right=1in,top=1.25in,bottom=1in]{geometry}

%\usepackage{newpxtext,newpxmath}
%\usepackage{amsfonts}
\usepackage{tgtermes}
\usepackage{amsmath}
\usepackage{color}
\usepackage{bold-extra}
\usepackage{setspace}
\usepackage{framed}
%\pagenumbering{gobble}
\setstretch{1.1}
%\onehalfspacing


%\fontsize{12pt}{15pt}\selectfont

\usepackage{fancyhdr}
\pagestyle{fancy}
\lhead{Huan Q. Bui}
\rhead{MIT\# 915434286}
%\cfoot{center of the footer!}
%\renewcommand{\headrulewidth}{0.4pt}
%\renewcommand{\footrulewidth}{0.4pt}

\begin{document}
%\noindent Huan Quang Bui \hfill Application for MIT Physics
\begin{center}
	\textbf{EET Writing Task: Prompt}
\end{center}

\begin{framed}
	\noindent \textbf{Topic B:}  In 2020, the COVID-19 pandemic transformed college teaching and learning from in-person to largely online. The platform Zoom alone jumped from some 10 million users in December 2019 to over 300 million by April 2020. But as Stanford University Communications Professor Jeremy Bailenson writes, ``something about being on video conference all day seems particularly exhausting,'' and the term ``Zoom fatigue'' was coined and spread quickly (1). (In addition, over 94\% of 325 undergraduates surveyed reported moderate to extreme difficulty with online learning (Peper 47).) However, this exhaustion has not been fully explained, although many theories have been offered.\\
	
	
	
	\noindent Based on your experience with virtual learning (or videoconferencing), what do you think the main causes of Zoom fatigue are—and its major effects? And although it may be unlikely, if MIT should suddenly have to go back to completely online learning this year, how could the school, instructors, or the designers of videoconferencing software such as Zoom prevent or moderate this fatigue?\\
	
	
	
	\noindent To help you get started, consider factors such as the ones below as well as those you have experienced. Then ask yourself how these problems might be reduced and write a focused, unified, and coherent response that could serve as an opinion piece in the MIT student newspaper The Tech. Include your experience of the last year at your undergraduate institution, place of work, or general environment (e.g., at home in your country) as evidence for your claims.
	
	\begin{itemize}
		\item excessive close-up gazing by the eye
		\item increased cognitive load
		\item increased self-evaluation (from seeing oneself on video)
		\item constraints on physical mobility
	\end{itemize}
	
	\noindent Do not do any outside research in writing your essay.
\end{framed}


\newpage


\begin{center}
	\textbf{EET Writing Task: Response}
\end{center}












	
	
	
	
	
\end{document}