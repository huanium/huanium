\documentclass{article}
\usepackage{physics}
\usepackage{graphicx}
\usepackage{caption}
\usepackage{amsmath}
\usepackage{bm}
\usepackage{framed}
\usepackage{authblk}
\usepackage{empheq}
\usepackage{amsfonts}
\usepackage{esint}
\usepackage[makeroom]{cancel}
\usepackage{dsfont}
\usepackage{centernot}
\usepackage{mathtools}
\usepackage{subcaption}
\usepackage{bigints}
\usepackage{amsthm}
\theoremstyle{definition}
\newtheorem{lemma}{Lemma}
\newtheorem{defn}{Definition}[section]
\newtheorem{prop}{Proposition}[section]
\newtheorem{rmk}{Remark}[section]
\newtheorem{thm}{Theorem}[section]
\newtheorem{exmp}{Example}[section]
\newtheorem{prob}{Problem}[section]
\newtheorem{sln}{Solution}[section]
\newtheorem*{prob*}{Problem}
\newtheorem{exer}{Exercise}[section]
\newtheorem*{exer*}{Exercise}
\newtheorem*{sln*}{Solution}
\usepackage{empheq}
\usepackage{tensor}
\usepackage{xcolor}
%\definecolor{colby}{rgb}{0.0, 0.0, 0.5}
\definecolor{MIT}{RGB}{163, 31, 52}
\usepackage[pdftex]{hyperref}
%\hypersetup{colorlinks,urlcolor=colby}
\hypersetup{colorlinks,linkcolor={MIT},citecolor={MIT},urlcolor={MIT}}  
\usepackage[left=1in,right=1in,top=1in,bottom=1in]{geometry}
\setcounter{MaxMatrixCols}{20}
\usepackage{newpxtext,newpxmath}
\newcommand*\widefbox[1]{\fbox{\hspace{2em}#1\hspace{2em}}}

\newcommand{\p}{\partial}
\newcommand{\R}{\mathbb{R}}
\newcommand{\C}{\mathbb{C}}
\newcommand{\lag}{\mathcal{L}}
\newcommand{\nn}{\nonumber}
\newcommand{\ham}{\mathcal{H}}
\newcommand{\M}{\mathcal{M}}
\newcommand{\I}{\mathcal{I}}
\newcommand{\K}{\mathcal{K}}
\newcommand{\F}{\mathcal{F}}
\newcommand{\w}{\omega}
\newcommand{\lam}{\lambda}
\newcommand{\al}{\alpha}
\newcommand{\be}{\beta}
\newcommand{\x}{\xi}

\newcommand{\G}{\mathcal{G}}

\newcommand{\f}[2]{\frac{#1}{#2}}

\newcommand{\ift}{\infty}

\newcommand{\lp}{\left(}
\newcommand{\rp}{\right)}

\newcommand{\lb}{\left[}
\newcommand{\rb}{\right]}

\newcommand{\lc}{\left\{}
\newcommand{\rc}{\right\}}


\newcommand{\V}{\mathbf{V}}
\newcommand{\U}{\mathcal{U}}
\newcommand{\Id}{\mathcal{I}}
\newcommand{\D}{\mathcal{D}}
\newcommand{\Z}{\mathcal{Z}}

%\setcounter{chapter}{-1}


\usepackage{enumitem}



\usepackage{listings}
\captionsetup[lstlisting]{margin=0cm,format=hang,font=small,format=plain,labelfont={bf,up},textfont={it}}
\renewcommand*{\lstlistingname}{Code \textcolor{violet}{\textsl{Mathematica}}}
\definecolor{gris245}{RGB}{245,245,245}
\definecolor{olive}{RGB}{50,140,50}
\definecolor{brun}{RGB}{175,100,80}

%\hypersetup{colorlinks,urlcolor=colby}
\lstset{
	tabsize=4,
	frame=single,
	language=mathematica,
	basicstyle=\scriptsize\ttfamily,
	keywordstyle=\color{black},
	backgroundcolor=\color{gris245},
	commentstyle=\color{gray},
	showstringspaces=false,
	emph={
		r1,
		r2,
		epsilon,epsilon_,
		Newton,Newton_
	},emphstyle={\color{olive}},
	emph={[2]
		L,
		CouleurCourbe,
		PotentielEffectif,
		IdCourbe,
		Courbe
	},emphstyle={[2]\color{blue}},
	emph={[3]r,r_,n,n_},emphstyle={[3]\color{magenta}}
}

\newcommand{\diag}{\text{diag}}
\newcommand{\psirot}{\ket{\psi_\text{rot}(t)} }
\newcommand{\RWA}{\ham_\text{rot}^\text{RWA}}


\begin{document}
\begin{framed}
\noindent Name: \textbf{Huan Q. Bui}\\
Course: \textbf{8.421 - AMO I}\\
Problem set: \textbf{\#5}\\
Due: Friday, March 11, 2022.
\end{framed}



\noindent \textbf{1. Magnetic field of a magnetic dipole}


\begin{enumerate}[label=(\alph*)]
	\item From the identity
	\begin{align*}
	\p_i \p_j \lp \f{1}{r} \rp = -\p_i \lp \f{\hat{\bm{r}}_j}{r^2} \rp = -\p_i \lp \f{x_j}{r^3} \rp = \f{3 \hat{\bm{r}}_i \hat{\bm{r}}_j - \delta_{ij}}{r^3} - \f{4\pi}{3} \delta_{ij} \delta^3(\bm{r}),
	\end{align*}
	we simply contract to get
	\begin{align*}
	\nabla^2 \lp \f{1}{r} \rp 
	&= \sum_{i=1}^3 \p_i \p_i \lp \f{1}{r} \rp \\
	&=  \sum_{i=1}^3 \f{3 \hat{\bm{r}}_i \hat{\bm{r}}_i - \delta_{ii}}{r^3} - \f{4\pi}{3} \delta_{ii} \delta^3(\bm{r}) \\
	&= \sum_{i=1}^3 \f{3 \hat{\bm{r}}_i \hat{\bm{r}}_i - 1}{r^3} - \f{4\pi}{3} \delta^3(\bm{r}) \\
	&= \f{3(x^2+y^2+z^2)/r^2-3}{r^3} - 4\pi \delta^3(\bm{r})\\
	&= \f{3r^2/r^2-3}{r^3} - 4\pi \delta^3(\bm{r})\\
	&= -4\pi \delta^3(\bm{r})
	\end{align*}
	as desired. Note that here $\delta_{ii}=1$ is a matrix element since we are not using Einstein summation convention here. We will use it in the next part of the problem, however.  
	
	
	\item Let the vector potential for a magnetic dipole be given 
	\begin{align*}
	\bm{A}^\text{dip}(\bm{r}) = \f{\bm{m} \times \hat{\bm{r}}}{r^2}.
	\end{align*}
	For ease of computation, we may rewrite this using the Levi-Civita symbol and Einstein summation convention:
	\begin{align*}
	\bm{A}^\text{dip}_i(\bm{r}) = \f{1}{r^2}\epsilon_{ijk} \bm{m}_j \hat{\bm{r}}_k 
	\end{align*}
	The magnetic field of a magnetic dipole is thus given by taking the curl of $\bm{A}_\text{dip}$, by definition:
	\begin{align*}
	\bm{B}^\text{dip}_a(\bm{r}) 
	&= [\bm{\nabla}\times \bm{A}^\text{dip}(\bm{r})]_a \\
	&= \epsilon_{abc} \p_b \bm{A}^\text{dip}_c(\bm{r})\\
	&= \epsilon_{abc} \p_b \lp \f{1}{r^2}\epsilon_{cjk} \bm{m}_j \hat{\bm{r}}_k  \rp\\
	&= \epsilon_{abc}\epsilon_{cjk} \bm{m}_j \p_b \lp \f{\hat{\bm{r}}_k}{r^2}  \rp.
	\end{align*}
	Using the identity given in the problem statement and the fact that
	\begin{align*}
	\epsilon_{abc}\epsilon_{cjk} = \epsilon_{cab}\epsilon_{cjk} = \delta_{aj}\delta_{bk} - \delta_{ak}\delta_{bj}
	\end{align*}
	we have
	\begin{align*}
	\bm{B}^\text{dip}_a(\bm{r}) &= -(\delta_{aj}\delta_{bk} - \delta_{ak}\delta_{bj})\, \bm{m}_j \p_b \p_k  \lp \f{1}{r}  \rp\\
	&= -(\delta_{aj}\delta_{bk} - \delta_{ak}\delta_{bj})\, \bm{m}_j \lp \f{3 \hat{\bm{r}}_b \hat{\bm{r}}_k - \delta_{bk}}{r^3} - \f{4\pi}{3} \delta_{bk} \delta^3(\bm{r})  \rp\\
	&= -\delta_{aj}\delta_{bk} \bm{m}_j \lp \f{3 \hat{\bm{r}}_b \hat{\bm{r}}_k - \delta_{bk}}{r^3} - \f{4\pi}{3} \delta_{bk} \delta^3(\bm{r})  \rp + \delta_{ak}\delta_{bj}\bm{m}_j \lp \f{3 \hat{\bm{r}}_b \hat{\bm{r}}_k - \delta_{bk}}{r^3} - \f{4\pi}{3} \delta_{bk} \delta^3(\bm{r})  \rp\\
	&= -\bm{m}_a \lp \cancel{\f{3 \hat{\bm{r}}_b \hat{\bm{r}}_b - \delta_{bb}}{r^3}} - \f{4\pi}{3} \delta_{bb} \delta^3(\bm{r})  \rp + \bm{m}_b \lp \f{3 \hat{\bm{r}}_b \hat{\bm{r}}_a - \delta_{ba}}{r^3} - \f{4\pi}{3} \delta_{ba} \delta^3(\bm{r})  \rp \\
	&= \bm{m}_a 4\pi \delta^3(\bm{r})    +   \f{3 (\bm{m}_b\hat{\bm{r}}_b) \hat{\bm{r}}_a - \bm{m}_a}{r^3} - \f{4\pi}{3} \bm{m}_a \delta^3(\bm{r})   \\
	&=   \f{3 (\bm{m}\cdot \hat{\bm{r}}) \hat{\bm{r}}_a - \bm{m}_a}{r^3} + \f{8\pi}{3} \bm{m}_a \delta^3(\bm{r}),
	\end{align*}
	where we have used the contraction identity $\delta_{ii}=3$.  Putting back into vector form, we find 
	\begin{align*}
	\bm{B}^\text{dip}(\bm{r}) =  \f{3 (\bm{m}\cdot \hat{\bm{r}}) \hat{\bm{r}} - \bm{m}}{r^3} + \f{8\pi}{3} \bm{m} \delta^3(\bm{r})
	\end{align*}
	as desired. 
	
	
	
	\item It remains to prove the provided identity:
	\begin{align*}
	\p_i \p_j \lp \f{1}{r} \rp = -\p_i \lp \f{\hat{\bm{r}}_j}{r^2} \rp = -\p_i \lp \f{x_j}{r^3} \rp = \f{3 \hat{\bm{r}}_i \hat{\bm{r}}_j - \delta_{ij}}{r^3} - \f{4\pi}{3} \delta_{ij} \delta^3(\bm{r}),
	\end{align*}
	
	The first two equalities are straightforward to show, but I will show the proofs here anyway as the technique carries over to proving the third equality (which is the one we really care about):
	\begin{align*}
	\p_i \p_j \lp \f{1}{r} \rp 
	&= \p_i \p_j \f{1}{\lp \sum_{a=1}^3 x_a^2 \rp^{1/2}} \\
	&= -\p_i\lb \f{1}{2\lp \sum_{a=1}^3 x_a^2 \rp^{3/2}} \p_j \lp \sum_{a=1}^3  x_a\rp\rb\\
	&= -\p_i\lb \f{1}{2r^3} \sum_{a=1}^3 2x_a \delta_{ja}\rb\\
	&= -\p_i\lp \f{x_j}{r^3} \rp\\
	&= -\p_i\lp \f{\hat{\bm{r}}_j}{r^2} \rp.
	\end{align*}
	where we have used $x_j = r \hat{\bm{r}}_j$. Now we focus on the last equality. We will consider two cases. For $r\neq 0$, we may prove the identity above but ignoring the $\delta$-function piece: 
	\begin{align*}
	\p_i \p_j \lp \f{1}{r} \rp = -\p_i \lp \f{\hat{\bm{r}}_j}{r^2} \rp = -\p_i \lp \f{x_j}{r^3} \rp = \f{3 \hat{\bm{r}}_i \hat{\bm{r}}_j - \delta_{ij}}{r^3}.
	\end{align*}
	To prove this, we simply calculate away:
	\begin{align*}
	\p_i \p_j \lp \f{1}{r} \rp &= -\p_i \lp \f{\hat{\bm{r}}_j}{r^2} \rp \\
	&= -\p_i \lp \f{x_j}{r^3} \rp \\
	&= \f{-\p_i x_j}{r^3} - x_j \p_i \f{1}{r^3}\\
	&= -\f{\delta_{ij}}{r^3} - x_j \p_i \f{1}{\lp \sum_{a=1}^3 x_a^2 \rp^{3/2}}\\
	&= -\f{\delta_{ij}}{r^3} + \f{3}{2} \f{x_j}{r^5} \p_i \lp \sum_{a=1}^3 x_a^2 \rp\\
	&= -\f{\delta_{ij}}{r^3} + \f{3}{2} \f{x_j}{r^5} \lp \sum_{a=1}^3 2 x_a \delta_{ia} \rp\\
	&= -\f{\delta_{ij}}{r^3} + \f{3 x_i x_j}{r^5}\\
	&= \f{3 \hat{\bm{r}}_i \hat{\bm{r}}_j - \delta_{ij}}{r^3}.
	\end{align*}
	And we're done. 
	
	
	Now consider the case where $r$ can be $0$. We will to calculate $\p_i \p_j (1/r)$ via integration rather than taking derivatives. To this end, we make use of Gauss-Ostrogradsky theorem for volume integral of a gradient field:
	\begin{align*}
	\int_V \grad \psi \,dV = \int_{\p V} \psi \bm{n} \,da.
	\end{align*}
	In index notation, this is 
	\begin{align*}
	\int_V \p_i \psi \,dV = \int_{\p V} \psi \bm{n}_i \,da.
	\end{align*}
	Let $\psi = \hat{\bm{r}}_j / r^2$ and the volume $V$ to be that of a sphere centered at the origin with radius $\epsilon$. We have that
	\begin{align*}
	I_{ij} &= \int_V  \p_i \lp \f{\hat{\bm{r}}_j}{r^2}\rp \,dV \\
	&= \int_{\p V} \f{\hat{\bm{r}}_j \hat{\bm{r}}_i }{r^2} \,da \\
	&= \int_{\p V, r=\epsilon} \f{\hat{\bm{r}}_j  \hat{\bm{r}}_i }{r^2}r^2\sin\theta\,dr d\theta d\phi\\
	&= \int_{\p V, r=\epsilon} \f{x_i x_j}{r^2}\sin\theta\,dr d\theta d\phi.
	\end{align*}
	At this point one may argue that due to spherical symmetry, only the diagonal terms $i=j$ are nonzero and are equal to a third of the trace. It then suffices to find the trace using the usual form of the Gauss-Ostrogradsky theorem. Here, I will present an explicit calculation of the (tensor) elements by expressing $x_i$ in spherical coordinates. 
	\begin{align*}
	&x = \epsilon\sin\theta\cos\phi \\
	&y = \epsilon\sin\theta\cos\phi\\
	&z = \epsilon\cos\theta.
	\end{align*}
	Using Mathematica (this is really not necessary since we can tell which integrand is odd/even... but for completeness I will just show everything explicitly):
	\begin{align*}
	&I_{xx} = I_{yy} = I_{zz} = \f{4\pi}{3} \\
	&I_{xy} = I_{yx} = I_{xz} = I_{zx} = I_{yz} = I_{zy} = 0.
	\end{align*}
	With this, we're done. Putting everything together, we find that
	\begin{align*}
	I_{ij} = \f{4\pi}{3}\delta_{ij}.
	\end{align*}
	Using the same technique\footnote{This integral diverges and therefore requires \textit{regularization} in the sense that integration over the angular variables is carried out first, giving zero, rendering the radial integration unnecessary. The Mathematica code in the box has only integration over $\theta,\phi$.} (or in view of Gauss-Ostrogradsky theorem), we can also show that 
	\begin{align*}
	\int_V \f{3 \hat{\bm{r}}_i \hat{\bm{r}}_j - \delta_{ij}}{r^3}\,dV = 0.
	\end{align*}
	Therefore the condition that 
	\begin{align*}
	-\int_V \p_i \p_j \f{1}{r} = \int_V  \p_i \lp \f{\hat{\bm{r}}_j}{r^2}\rp\,dV = \f{4\pi}{3}\delta_{ij} \quad \text{if } {0 \in V} 
	\end{align*}
	is only satisfied if $\p_i \p_j (1/r)$ has a Dirac $\delta$-function piece in addition to the usual ``dipole'' piece which doesn't contribute to the integral over $V \ni 0$:
	\begin{align*}
	\p_i \p_j \lp \f{1}{r} \rp = -\p_i \lp \f{\hat{\bm{r}}_j}{r^2} \rp = -\p_i \lp \f{x_j}{r^3} \rp = \f{3 \hat{\bm{r}}_i \hat{\bm{r}}_j - \delta_{ij}}{r^3} - \f{4\pi}{3} \delta_{ij} \delta^3(\bm{r}).
	\end{align*}
	We have thus proved the identity. \\
	
	
	
	Mathematica code:
	\begin{lstlisting}
	(*Ixx*)
	Integrate[
	r*Cos[p]*Sin[t]*r*Cos[p]*Sin[t]*r^2/r^4*Sin[t], {t, 0, Pi}, {p, 0, 
	2 Pi}]
	
	Out[19]= (4 \[Pi])/3
	
	(*Iyy*)
	Integrate[
	r*Cos[p]*Sin[t]*r*Cos[p]*Sin[t]*r^2*Sin[t]/r^4, {t, 0, Pi}, {p, 0, 
	2 Pi}]
	
	Out[17]= (4 \[Pi])/3
	
	(*Izz*)
	Integrate[
	r^2 Cos[t]^2*r^2*Sin[t]/r^4, {t, 0, Pi}, {p, 0, 2 Pi}]
	
	Out[23]= (4 \[Pi])/3
	
	(*Ixy*)
	Integrate[
	r*Cos[p]*Sin[t]*r*Sin[p]*Sin[t]*r^2/r^4*Sin[t], {t, 0, Pi}, {p, 0, 
	2 Pi}]
	
	Out[29]= 0
	
	(*Iyz*)
	Integrate[
	r*Cos[t]*r*Sin[p]*Sin[t]*r^2/r^4*Sin[t], {t, 0, Pi}, {p, 0, 2 Pi}]
	
	Out[30]= 0
	
	(*Ixz*)
	Integrate[
	r*Cos[t]*r*Sin[p]*Cos[t]*r^2/r^4*Sin[t], {t, 0, Pi}, {p, 0, 2 Pi}]
	
	Out[31]= 0
	
	
	
	
	
	
	(*Dipole stuff... regularization is needed, i.e. I won't do \
	the integral over r. Suffices to integrate over angles*)
	
	(*xx*)
	Integrate[(3*Cos[p]*Sin[t]*Cos[p]*Sin[t] - 1)/r*Sin[t], {t, 
	0, Pi}, {p, 0, 2 Pi}]
	
	Out[36]= 0
	
	(*yy*)
	Integrate[(3*Sin[p]*Sin[t]*Sin[p]*Sin[t] - 1)/r*Sin[t], {t, 
	0, Pi}, {p, 0, 2 Pi}]
	
	Out[39]= 0
	
	(*zz*)
	Integrate[(3*Cos[t] Cos[t] - 1)/r*Sin[t], {t, 0, Pi}, {p, 0, 
	2 Pi}]
	
	Out[40]= 0
	
	(*xy*)
	Integrate[(3*Cos[p]*Sin[t]*Sin[p]*Sin[t])/r*Sin[t], {t, 0, 
	Pi}, {p, 0, 2 Pi}]
	
	Out[44]= 0
	
	(*yz*)
	Integrate[(3*Cos[t]*Sin[t]*Sin[p])/r*Sin[t], {t, 0, Pi}, {p, 
	0, 2 Pi}]
	
	Out[45]= 0
	
	(*xz*)
	Integrate[(3*Cos[t]*Cos[p]*Sin[t])/r*Sin[t], {t, 0, Pi}, {p, 
	0, 2 Pi}]
	
	Out[47]= 0
	
	\end{lstlisting}
	
	
	
\end{enumerate}



\noindent \textbf{2. Atoms in magnetic fields: the Breit-Rabi formula}



\begin{enumerate}[label=(\alph*)]
	\item 
	
	\item 
	
	\item 
	
	\item 
	
	\item 
	
	\item 
\end{enumerate}


\noindent \textbf{3. Atomic $g$ factors}

$\,$


\noindent Before starting this problem, let us write down the formula for $g_F$:
\begin{align*}
g_F &= \f{g_J}{2}\f{F(F+1) + J(J+1)-I(I+1)}{F(F+1)} \\
&= \f{1}{2}\f{F(F+1) + J(J+1)-I(I+1)}{F(F+1)}\lp 1 + \f{J(J+1)+S(S+1)-L(L+1)}{2J(J+1)} \rp
\end{align*} 
where we have taken $g_E = 2$ and neglected $g_I \ll g_J$. Inserting this into Mathematica gives us a nice routine to find the Land\'{e} $g$-factor for given $(F,I,J,L,S=1/2)$. For this problem, we're looking at sodium with $I=3/2$. $S=1/2$ as usual. 


\begin{enumerate}[label=(\alph*)]
	\item
	\begin{align*}
	&^2P_{1/2}, F=1 \quad& g_F = -1/6\\
	&^2P_{1/2}, F=2 \quad& g_F = +1/6
	\end{align*}
	
	\item 
	\begin{align*}
	&^2P_{3/2}, F=0 \quad& g_F = \text{n/a}\\
	&^2P_{3/2}, F=1 \quad& g_F = +2/3\\
	&^2P_{3/2}, F=2 \quad& g_F = +2/3\\
	&^2P_{3/2}, F=3 \quad& g_F = +2/3
	\end{align*}
	
	\item 
	\begin{align*}
	&^2S_{1/2}, F=1 \quad& g_F = -1/2\\
	&^2S_{1/2}, F=2 \quad& g_F = +1/2
	\end{align*}
	
	
	
\end{enumerate}




For the stretched states, we simply have $F= I+J = I+ L +S$. The Land\'{e} $g$-factor by definition is given by the projection of $J$ on $F$ multiplied by $g_J$, which is by definition (plus the condition that $J$ is maximal) is just the ratio $(J+2S)/J$. As a result, we have that the Land\'{e} $g$-factor $g_F$ for the stretched states is given by 
\begin{align*}
g_F = \f{J}{F}g_J = \f{J}{F} \f{L+2S}{J} = \f{L+2S}{F}.
\end{align*}
With this, we can find that for the stretched state $^2P_{3/2}, F=3$, 
\begin{align*}
g_F= \f{1+2(1/2)}{3} = \f{2}{3} \quad\quad \checkmark
\end{align*}
Similarly, we may find for the state $^2S_{1/2}, F=2$:
\begin{align*}
g_F = \f{0+2(1/2)}{2} =\f{1}{2} \quad\quad\checkmark
\end{align*}




















\noindent Mathematica code:
\begin{lstlisting}
In[48]:= (*g-factors*)
In[50]:= gJ[J_,S_,L_]:=1+(J*(J+1)+S(S+1)-L(L+1))/(2*J*(J+1));
In[51]:= gF[F_,I_,J_,L_,S_]:=gJ[J,S,L]/2*(F*(F+1)+J(J+1)-I*(I+1))/(F*(F+1))
(*F=1,I=3/2,J=1/2,L=1,S=1/2*)
In[53]:= gF[1,3/2,1/2,1,1/2]
Out[53]= -(1/6)
(*F=2,I=3/2,J=1/2,L=1,S=1/2*)
In[54]:= gF[2,3/2,1/2,1,1/2]
Out[54]= 1/6
In[55]:= (*F=0,I=3/2,J=3/2,L=1,S=1/2*)
In[63]:= (*Indeterminate*)
In[57]:= (*F=1,I=3/2,J=3/2,L=1,S=1/2*)
In[58]:= gF[1,3/2,3/2,1,1/2]
Out[58]= 2/3
In[59]:= (*F=2,I=3/2,J=3/2,L=1,S=1/2*)
In[60]:= gF[2,3/2,3/2,1,1/2]
Out[60]= 2/3
In[61]:= (*F=3,I=3/2,J=3/2,L=1,S=1/2*)
In[62]:= gF[3,3/2,3/2,1,1/2]
Out[62]= 2/3
In[64]:= (*F=1,I=3/2,J=1/2,L=0,S=1/2*)
In[65]:= gF[1,3/2,1/2,0,1/2]
Out[65]= -(1/2)
In[66]:= (*F=2,I=3/2,J=1/2,L=0,S=1/2*)
In[67]:= gF[2,3/2,1/2,0,1/2]
Out[67]= 1/2
\end{lstlisting}


\end{document}








