\documentclass{article}
\usepackage{physics}
\usepackage{graphicx}
\usepackage{caption}
\usepackage{amsmath}
\usepackage{bm}
\usepackage{framed}
\usepackage{authblk}
\usepackage{empheq}
\usepackage{amsfonts}
\usepackage{esint}
\usepackage[makeroom]{cancel}
\usepackage{dsfont}
\usepackage{centernot}
\usepackage{mathtools}
\usepackage{subcaption}
\usepackage{bigints}
\usepackage{amsthm}
\theoremstyle{definition}
\newtheorem{lemma}{Lemma}
\newtheorem{defn}{Definition}[section]
\newtheorem{prop}{Proposition}[section]
\newtheorem{rmk}{Remark}[section]
\newtheorem{thm}{Theorem}[section]
\newtheorem{exmp}{Example}[section]
\newtheorem{prob}{Problem}[section]
\newtheorem{sln}{Solution}[section]
\newtheorem*{prob*}{Problem}
\newtheorem{exer}{Exercise}[section]
\newtheorem*{exer*}{Exercise}
\newtheorem*{sln*}{Solution}
\usepackage{empheq}
\usepackage{tensor}
\usepackage{xcolor}
%\definecolor{colby}{rgb}{0.0, 0.0, 0.5}
\definecolor{MIT}{RGB}{163, 31, 52}
\usepackage[pdftex]{hyperref}
%\hypersetup{colorlinks,urlcolor=colby}
\hypersetup{colorlinks,linkcolor={MIT},citecolor={MIT},urlcolor={MIT}}  
\usepackage[left=1in,right=1in,top=1in,bottom=1in]{geometry}
\setcounter{MaxMatrixCols}{20}
\usepackage{newpxtext,newpxmath}
\newcommand*\widefbox[1]{\fbox{\hspace{2em}#1\hspace{2em}}}

\newcommand{\p}{\partial}
\newcommand{\R}{\mathbb{R}}
\newcommand{\C}{\mathbb{C}}
\newcommand{\lag}{\mathcal{L}}
\newcommand{\nn}{\nonumber}
\newcommand{\ham}{\mathcal{H}}
\newcommand{\M}{\mathcal{M}}
\newcommand{\I}{\mathcal{I}}
\newcommand{\K}{\mathcal{K}}
\newcommand{\F}{\mathcal{F}}
\newcommand{\w}{\omega}
\newcommand{\lam}{\lambda}
\newcommand{\al}{\alpha}
\newcommand{\be}{\beta}
\newcommand{\x}{\xi}

\newcommand{\G}{\mathcal{G}}

\newcommand{\f}[2]{\frac{#1}{#2}}

\newcommand{\ift}{\infty}

\newcommand{\lp}{\left(}
\newcommand{\rp}{\right)}

\newcommand{\lb}{\left[}
\newcommand{\rb}{\right]}

\newcommand{\lc}{\left\{}
\newcommand{\rc}{\right\}}


\newcommand{\V}{\mathbf{V}}
\newcommand{\U}{\mathcal{U}}
\newcommand{\Id}{\mathcal{I}}
\newcommand{\D}{\mathcal{D}}
\newcommand{\Z}{\mathcal{Z}}

%\setcounter{chapter}{-1}


\usepackage{enumitem}



\usepackage{listings}
\captionsetup[lstlisting]{margin=0cm,format=hang,font=small,format=plain,labelfont={bf,up},textfont={it}}
\renewcommand*{\lstlistingname}{Code \textcolor{violet}{\textsl{Mathematica}}}
\definecolor{gris245}{RGB}{245,245,245}
\definecolor{olive}{RGB}{50,140,50}
\definecolor{brun}{RGB}{175,100,80}

%\hypersetup{colorlinks,urlcolor=colby}
\lstset{
	tabsize=4,
	frame=single,
	language=mathematica,
	basicstyle=\scriptsize\ttfamily,
	keywordstyle=\color{black},
	backgroundcolor=\color{gris245},
	commentstyle=\color{gray},
	showstringspaces=false,
	emph={
		r1,
		r2,
		epsilon,epsilon_,
		Newton,Newton_
	},emphstyle={\color{olive}},
	emph={[2]
		L,
		CouleurCourbe,
		PotentielEffectif,
		IdCourbe,
		Courbe
	},emphstyle={[2]\color{blue}},
	emph={[3]r,r_,n,n_},emphstyle={[3]\color{magenta}}
}

\newcommand{\diag}{\text{diag}}
\newcommand{\psirot}{\ket{\psi_\text{rot}(t)} }
\newcommand{\RWA}{\ham_\text{rot}^\text{RWA}}

% 3j symbol
\newcommand{\tj}[6]{ \begin{pmatrix}
		#1 & #2 & #3 \\
		#4 & #5 & #6 
\end{pmatrix}}


\begin{document}
\begin{framed}
\noindent Name: \textbf{Huan Q. Bui}\\
Course: \textbf{8.421 - AMO I}\\
Problem set: \textbf{\#8}\\
Due: Friday, April 8, 2022.
\end{framed}
	

\noindent \textbf{1. Optical Traps and Scattering.} What are are the proper power and wavelength needed to trap an ultracold atomic gas? Consider an alkali atom with resonance frequency $\omega_0$ on the principal $nS \to nP$ transition. A sample of atoms in the ground state $nS$ are exposed to monochromatic radiation of intensity $I$ and frequency $\omega_L < \omega_0$.  Using the fact that essentially all of the oscillator strength out of the ground state comes from the $nS \to nP$ transition, we have
\begin{align*}
\alpha(\omega_L) \approx \f{2e^2}{\hbar} \abs{\bra{nP} z \ket{nS}}^2 \f{\omega_0}{\omega_0^2 - \omega_L^2} \implies \abs{\bra{nP} z \ket{nS}}^2 = \f{\hbar \al(\omega_L)}{2e^2} \f{\omega_0^2 - \omega_L^2}{\omega_0}.
\end{align*} 

\begin{enumerate}[label=(\alph*)]
	\item AC Stark shift: 
	
	\begin{enumerate}[label=(\roman*)]
		\item From lecture, the AC Stark shift $U_i$ from time-dependent perturbation theory is given by 
		\begin{align*}
		U_i = -\f{1}{4}\al(\omega_L) \mathcal{E}^2 = -\f{2I\al(\omega_L)}{4c\epsilon_0} = -\f{I\al(\omega_L)}{2c\epsilon_0}.
		\end{align*}
		
		
		\item Now, we want to use the rotating wave approximation to obtain the AC Stark shift. This can be done by first writing down the true (symmetrized) Hamiltonian:
		\begin{align*}
		\ham = \f{\hbar}{2}\begin{pmatrix}
		-\omega_0 & \omega_R e^{i\omega_L t} \\ \omega_R^* e^{-i\omega_L t} & \omega_0
		\end{pmatrix}.
		\end{align*}
		By going into the rotating frame, plus making the rotating wave approximation, we find that
		\begin{align*}
		\ham_\text{rot}^\text{RWA} = \f{\hbar}{2}\begin{pmatrix}
		-\delta & \omega_R \\ \omega_R & \delta
		\end{pmatrix}
		\end{align*}
		where $\delta = \omega_0 - \omega_L$. The energy shifts can be obtained from the eigenvalues:
		\begin{align*}
		\Delta E = \pm  \f{\hbar}{2}\sqrt{\omega_R^2 + \delta^2} = \f{\hbar}{2}\sqrt{\omega_R^2 + (\omega_0 - \omega_L)^2}.
		\end{align*}
		In the limit where the Rabi frequency is much less than the detuning, we simply have that
		\begin{align*}
		\abs{U_{ii}} = \f{\hbar}{2}\abs{\omega_0 - \omega_L}. 
		\end{align*}
		In particular, the energy of the lower state gets shifted down while the energy of the higher state gets shifted up (since we're red-detuning). 
		
		
		\item From the previous two parts, we find that
		\begin{align*}
		\f{U_i}{U_{ii}} = \f{I\al(\omega_L)}{2c\epsilon_0} \f{2}{\hbar (\omega_0 - \omega_L)} = \f{I}{c\epsilon_0 }\f{2e^2}{\hbar^2} \abs{\bra{nP} z \ket{nS}}^2 \f{\omega_0}{(\omega_0 - \omega_L)^2(\omega_0 + \omega_L)}.
		\end{align*}
		When $\omega_L \approx 0$, we have
		\begin{align*}
		\f{U_i}{U_{ii}} \approx \f{I}{c\epsilon_0 }\f{2e^2}{\hbar^2} \abs{\bra{nP} z \ket{nS}}^2 \f{1}{\omega_0^2}\lp 1 + \f{\omega_L}{\omega_0} + \dots \rp.
		\end{align*}
		When $\omega_L \approx \omega_0$, we may write $\omega_L + \omega_0 = 2\omega_0$, so that
		\begin{align*}
		\f{U_i}{U_{ii}} \approx  \f{I}{c\epsilon_0 }\f{2e^2}{\hbar^2} \abs{\bra{nP} z \ket{nS}}^2 \f{1}{2(\omega_0 - \omega_L)^2} 
		\end{align*}
	\end{enumerate}
	We see that if the intensity has spatial structure, with the appropriate detuning, the AC Stark shift can have energy minima where the atoms can be trapped. 
	
	\item Using time-dependent perturbation theory, we have that 
	\begin{align*}
	blah
	\end{align*}
	In the RWA picture, we know that
	\begin{align*}
	P_{e,ii}(t) = \f{\omega_R^2}{\omega^2_R + \delta^2} \sin^2 \lp \f{\sqrt{\omega_R^2 + (\omega_0 - \omega_L)^2} t}{2} \rp.
	\end{align*}
	Under time-averaging this is 
	\begin{align*}
	P_{e,ii} = \f{\omega_R^2}{2(\omega^2_R + \delta^2)} 
	\end{align*}
	
	
	\item 
	
	\begin{enumerate}[label=(\roman*)]
		\item 
		
		\item 
		
		\item 
	\end{enumerate}
	
	\item 
	
	\begin{enumerate}[label=(\roman*)]
		\item 
		
		\item 
		

	\end{enumerate}
\end{enumerate}


\noindent \textbf{2. Magic Wavelength Optical Trap.}

\begin{enumerate}[label=(\alph*)]
	\item 
	
	\begin{enumerate}[label=(\roman*)]
		\item 
		
		\item 
		
		\item 
	\end{enumerate}
	
	\item 
	
	\begin{enumerate}[label=(\roman*)]
		\item 
		
		\item 
		
		
	\end{enumerate}
\end{enumerate}


\noindent \textbf{3. Species-Dependent and Spin-Dependent AC Stark shift}	
	
\begin{enumerate}[label=(\alph*)]
	\item 
	
	\begin{enumerate}[label=(\roman*)]
		\item 
		
		\item 
		

	\end{enumerate}
	
	\item 
	
	\begin{enumerate}[label=(\roman*)]
		\item 
		
		\item 
		
		
	\end{enumerate}
\end{enumerate}
	
	
\end{document}








