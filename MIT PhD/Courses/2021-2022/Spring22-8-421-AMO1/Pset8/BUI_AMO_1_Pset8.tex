\documentclass{article}
\usepackage{physics}
\usepackage{graphicx}
\usepackage{caption}
\usepackage{amsmath}
\usepackage{bm}
\usepackage{framed}
\usepackage{authblk}
\usepackage{empheq}
\usepackage{amsfonts}
\usepackage{esint}
\usepackage[makeroom]{cancel}
\usepackage{dsfont}
\usepackage{centernot}
\usepackage{mathtools}
\usepackage{subcaption}
\usepackage{bigints}
\usepackage{amsthm}
\theoremstyle{definition}
\newtheorem{lemma}{Lemma}
\newtheorem{defn}{Definition}[section]
\newtheorem{prop}{Proposition}[section]
\newtheorem{rmk}{Remark}[section]
\newtheorem{thm}{Theorem}[section]
\newtheorem{exmp}{Example}[section]
\newtheorem{prob}{Problem}[section]
\newtheorem{sln}{Solution}[section]
\newtheorem*{prob*}{Problem}
\newtheorem{exer}{Exercise}[section]
\newtheorem*{exer*}{Exercise}
\newtheorem*{sln*}{Solution}
\usepackage{empheq}
\usepackage{tensor}
\usepackage{xcolor}
%\definecolor{colby}{rgb}{0.0, 0.0, 0.5}
\definecolor{MIT}{RGB}{163, 31, 52}
\usepackage[pdftex]{hyperref}
%\hypersetup{colorlinks,urlcolor=colby}
\hypersetup{colorlinks,linkcolor={MIT},citecolor={MIT},urlcolor={MIT}}  
\usepackage[left=1in,right=1in,top=1in,bottom=1in]{geometry}
\setcounter{MaxMatrixCols}{20}
\usepackage{newpxtext,newpxmath}
\newcommand*\widefbox[1]{\fbox{\hspace{2em}#1\hspace{2em}}}

\newcommand{\p}{\partial}
\newcommand{\R}{\mathbb{R}}
\newcommand{\C}{\mathbb{C}}
\newcommand{\lag}{\mathcal{L}}
\newcommand{\nn}{\nonumber}
\newcommand{\ham}{\mathcal{H}}
\newcommand{\M}{\mathcal{M}}
\newcommand{\I}{\mathcal{I}}
\newcommand{\K}{\mathcal{K}}
\newcommand{\F}{\mathcal{F}}
\newcommand{\w}{\omega}
\newcommand{\lam}{\lambda}
\newcommand{\al}{\alpha}
\newcommand{\be}{\beta}
\newcommand{\x}{\xi}

\newcommand{\G}{\mathcal{G}}

\newcommand{\f}[2]{\frac{#1}{#2}}

\newcommand{\ift}{\infty}

\newcommand{\lp}{\left(}
\newcommand{\rp}{\right)}

\newcommand{\lb}{\left[}
\newcommand{\rb}{\right]}

\newcommand{\lc}{\left\{}
\newcommand{\rc}{\right\}}


\newcommand{\V}{\mathbf{V}}
\newcommand{\U}{\mathcal{U}}
\newcommand{\Id}{\mathcal{I}}
\newcommand{\D}{\mathcal{D}}
\newcommand{\Z}{\mathcal{Z}}

%\setcounter{chapter}{-1}


\usepackage{enumitem}



\usepackage{listings}
\captionsetup[lstlisting]{margin=0cm,format=hang,font=small,format=plain,labelfont={bf,up},textfont={it}}
\renewcommand*{\lstlistingname}{Code \textcolor{violet}{\textsl{Mathematica}}}
\definecolor{gris245}{RGB}{245,245,245}
\definecolor{olive}{RGB}{50,140,50}
\definecolor{brun}{RGB}{175,100,80}

%\hypersetup{colorlinks,urlcolor=colby}
\lstset{
	tabsize=4,
	frame=single,
	language=mathematica,
	basicstyle=\scriptsize\ttfamily,
	keywordstyle=\color{black},
	backgroundcolor=\color{gris245},
	commentstyle=\color{gray},
	showstringspaces=false,
	emph={
		r1,
		r2,
		epsilon,epsilon_,
		Newton,Newton_
	},emphstyle={\color{olive}},
	emph={[2]
		L,
		CouleurCourbe,
		PotentielEffectif,
		IdCourbe,
		Courbe
	},emphstyle={[2]\color{blue}},
	emph={[3]r,r_,n,n_},emphstyle={[3]\color{magenta}}
}

\newcommand{\diag}{\text{diag}}
\newcommand{\psirot}{\ket{\psi_\text{rot}(t)} }
\newcommand{\RWA}{\ham_\text{rot}^\text{RWA}}

% 3j symbol
\newcommand{\tj}[6]{ \begin{pmatrix}
		#1 & #2 & #3 \\
		#4 & #5 & #6 
\end{pmatrix}}


\begin{document}
\begin{framed}
\noindent Name: \textbf{Huan Q. Bui}\\
Course: \textbf{8.421 - AMO I}\\
Problem set: \textbf{\#8}\\
Due: Friday, April 8, 2022.
\end{framed}
	

\noindent \textbf{1. Optical Traps and Scattering.} What are are the proper power and wavelength needed to trap an ultracold atomic gas? Consider an alkali atom with resonance frequency $\omega_0$ on the principal $nS \to nP$ transition. A sample of atoms in the ground state $nS$ are exposed to monochromatic radiation of intensity $I$ and frequency $\omega_L < \omega_0$.  Using the fact that essentially all of the oscillator strength out of the ground state comes from the $nS \to nP$ transition, we have
\begin{align*}
\alpha(\omega_L) \approx \f{2e^2}{\hbar} \abs{\bra{nP} z \ket{nS}}^2 \f{\omega_0}{\omega_0^2 - \omega_L^2} \implies \abs{\bra{nP} z \ket{nS}}^2 = \f{\hbar \al(\omega_L)}{2e^2} \f{\omega_0^2 - \omega_L^2}{\omega_0}.
\end{align*} 

\begin{enumerate}[label=(\alph*)]
	\item AC Stark shift: 
	
	\begin{enumerate}[label=(\roman*)]
		\item From lecture, the AC Stark shift $U_i$ from time-dependent perturbation theory is given by 
		\begin{align*}
		U_i = -\f{1}{4}\al(\omega_L) \mathcal{E}^2 = -\f{2I\al(\omega_L)}{4c\epsilon_0} = -\f{I\al(\omega_L)}{2c\epsilon_0}.
		\end{align*}
		
		
		\item Now, we want to use the rotating wave approximation to obtain the AC Stark shift. This can be done by first writing down the true (symmetrized) Hamiltonian:
		\begin{align*}
		\ham = \f{\hbar}{2}\begin{pmatrix}
		-\omega_0 & \omega_R e^{i\omega_L t} \\ \omega_R^* e^{-i\omega_L t} & \omega_0
		\end{pmatrix}.
		\end{align*}
		By going into the rotating frame, plus making the rotating wave approximation, we find that
		\begin{align*}
		\ham_\text{rot}^\text{RWA} = \f{\hbar}{2}\begin{pmatrix}
		-\delta & \omega_R \\ \omega_R & \delta
		\end{pmatrix}
		\end{align*}
		where $\delta = \omega_0 - \omega_L$. The energy shifts can be obtained from the eigenvalues:
		\begin{align*}
		\Delta E = \pm  \f{\hbar}{2}\sqrt{\omega_R^2 + \delta^2} =\pm \f{\hbar}{2}\sqrt{\omega_R^2 + (\omega_0 - \omega_L)^2} \approx \pm \f{\hbar(\omega_0 - \omega_L)}{2} \pm \f{\hbar \omega_R^2}{4(\omega_0 - \omega_L)}
		\end{align*}
		where we have used the fact that $\omega_R \ll \abs{\omega_0 - \omega_L}$. From here, we find that the shift is 
		\begin{align*}
		U_{ii} = -\f{\hbar \omega_R^2}{4(\omega_0 - \omega_L)} 
		\end{align*}
		In particular, the energy of the lower state gets shifted down while the energy of the higher state gets shifted up (since we're red-detuning). 
		
		
		\item From the previous two parts, we find that
		\begin{align*}
		\f{U_i}{U_{ii}} = \f{I\al(\omega_L)}{2c\epsilon_0} \f{4(\omega_0 - \omega_L)}{\hbar \omega_R^2}  = \f{I}{c\epsilon_0 \omega_R^2 }\f{4e^2}{\hbar^2} \abs{\bra{nP} z \ket{nS}}^2 \f{\omega_0}{\omega_0 + \omega_L}.
		\end{align*}
		To simplify this, we must write the Rabi frequency in terms of the intensity:
		\begin{align*}
			\omega_R = \f{e\mathcal{E}|\bra{nS} z \ket{nP}|}{\hbar} \implies \omega_R^2 = \f{e^2|\bra{nS} z \ket{nP}|^2}{\hbar^2} \f{2I}{c\epsilon_0}.
		\end{align*}
		With this, we have
		\begin{align*}
			\boxed{\f{U_{i}}{U_{ii}} = \f{2\omega_0}{\omega_0 + \omega_L}}
		\end{align*}
		When $\omega_L \approx 0$, we have
		\begin{align*}
		\f{U_i}{U_{ii}} \approx 2.
		\end{align*}
		When $\omega_L \approx \omega_0$, we may write $\omega_L + \omega_0 = 2\omega_0$, so that
		\begin{align*}
		\f{U_i}{U_{ii}} \approx 1.
		\end{align*}
	\end{enumerate}
	We see that if the intensity has spatial structure, with the appropriate detuning, the AC Stark shift can have energy minima where the atoms can be trapped. 
	
	\item From time-dependent perturbation theory, the amplitude of the excited state is
	\begin{align*}
	c_2(t) =  \f{e\mathcal{E}}{2\hbar} \bra{nS} z \ket{nP} \lb \f{e^{i(\omega_0+\omega_L)t}-1}{\omega_0-\omega_L} + \f{e^{i(\omega_0-\omega_L)t}-1}{\omega_0+\omega_L} \rb.
	\end{align*}
	Ignoring the $-1$ terms which are associated with transients, we have
	\begin{align*}
		P_e(t) = \f{e^2\mathcal{E}^2}{4\hbar^2}|\bra{nS}z \ket{nP}|^2 \f{2[\omega_0^2+\omega_L^2+(\omega_0^2-\omega_L^2)\cos(2\omega_L t)]}{(\omega_0^2-\omega_L^2)^2}.
	\end{align*}
	After time-averaging, this quantity is 
	\begin{align*}
		P_{e,i} = \boxed{\f{\omega_R^2}{2} \f{\omega_0^2+\omega_L^2}{(\omega_0^2-\omega_L^2)^2}}
	\end{align*}


	In the RWA picture, we know that
	\begin{align*}
	P_{e,ii}(t) = \f{\omega_R^2}{\omega^2_R + \delta^2} \sin^2 \lp \f{\sqrt{\omega_R^2 + (\omega_0 - \omega_L)^2} t}{2} \rp.
	\end{align*}
	After time-averaging this is 
	\begin{align*}
	P_{e,ii} = \f{\omega_R^2}{2(\omega^2_R + \delta^2)} \approx  \boxed{\f{\omega_R^2}{2 (\omega_0 - \omega_L)^2}}
	\end{align*}
	where we have used the approximation that the Rabi frequency is much less than the detuning.
	
	\item Calculate the photon scattering rate:
	
	\begin{enumerate}[label=(\roman*)]
		\item Starting with
		\begin{align*}
			P = \f{ck^4|\bm{d}|^2}{3} = \f{\omega_L^4}{3c^3}\abs{\bm{d}}^2,
		\end{align*}
		if we say $d = \al(\omega_L)\mathcal{E}$ then we have
		\begin{align*}
			R_\text{sc} &= \f{P}{\hbar \omega_L}= \f{\omega_L^3}{3\hbar c^3}\abs{\al(\omega_L)}^2 \mathcal{E}^2= \f{\omega_L^3}{3\hbar c^3}\abs{\al(\omega_L)}^2 \f{8\pi I}{c}
		\end{align*}
		where we have converted the intensity into CGS units. Now we recall from perturbation theory that
		\begin{align*}
			\al(\omega_L) = \f{2e^2}{\hbar} |\bra{nS} z \ket{nP}|^2 \f{\omega_0}{\omega_0^2 - \omega_L^2} = \f{e^2}{\hbar} |\bra{nS} z \ket{nP}|^2 \lp \f{1}{\omega_0 - \omega_L} + \f{1}{\omega_0 + \omega_L} \rp.
		\end{align*}
		From here we find that
		\begin{align*}
			|\al(\omega_L)|^2 = \f{e^4}{\hbar^2} |\bra{nS} z \ket{nP}|^4 \lp \f{1}{\omega_0 - \omega_L} + \f{1}{\omega_0 + \omega_L} \rp^2.
		\end{align*}
		Putting everything together, we find 
		\begin{align*}
			R_\text{sc,i} &= \f{\omega_L^3}{3\hbar c^3}\f{8\pi I}{c}\f{e^4}{\hbar^2} |\bra{nS} z \ket{nP}|^4 \lp \f{1}{\omega_0 - \omega_L} + \f{1}{\omega_0 + \omega_L} \rp^2\\
			&= \f{8\pi I\omega_L^3 e^4}{3 c^4 \hbar^3} |\bra{nS} z \ket{nP}|^4 \lp \f{1}{\omega_0 - \omega_L} + \f{1}{\omega_0 + \omega_L} \rp^2.
		\end{align*}
		Under RWA, we simply drop the counter-rotating term to find 
		\begin{align*}
			R_\text{sc,ii}= \f{8\pi I\omega_L^3 e^4}{3 c^4 \hbar^3} |\bra{nS} z \ket{nP}|^4 \lp \f{1}{\omega_0 - \omega_L} \rp^2.
		\end{align*}
		
		\item Let us write $R_\text{cs,i}$ and $R_\text{sc,ii}$ in terms of the spontaneous emission rate $\Gamma = 4e^2 \omega_0^3 |\bra{nS}z\ket{nP}|^2/\hbar c^3$: 
		\begin{align*}
			R_\text{sc,i} &= \f{8\pi I\omega_L^3 e^4}{3 c^4 \hbar^3} |\bra{nS} z \ket{nP}|^4 \lp \f{1}{\omega_0 - \omega_L} + \f{1}{\omega_0 + \omega_L} \rp^2 \\
			&= \f{8\pi I\omega_L^3 }{3 c^4 \hbar^3} \f{9\Gamma^2 \hbar^2 c^6}{16\omega_0^6}  \lp \f{1}{\omega_0 - \omega_L} + \f{1}{\omega_0 + \omega_L} \rp^2\\
			&= \f{3\pi c^2}{2\hbar \omega_0^3}\lp \f{\omega_L}{\omega_0} \rp^3 \lp \f{\Gamma}{\omega_0 - \omega_L} + \f{\Gamma}{\omega_0 + \omega_L} \rp^2 I,
		\end{align*}
	which is a well-known result given in many textbooks. Making the RWA, we find
	\begin{align*}
		R_\text{sc,ii} &= \f{3\pi c^2}{2\hbar \omega_0^3}\lp \f{\omega_L}{\omega_0} \rp^3 \lp \f{\Gamma}{\omega_0 - \omega_L}  \rp^2 I.
	\end{align*}
	From Part (b), we have that
	\begin{align*}
		P_{e,ii} &= \f{\omega_R^2}{2(\omega_0 - \omega_L)^2},
	\end{align*}
	which gives
	\begin{align*}
		R_\text{sc,ii} &=  \f{3\pi c^2}{2\hbar \omega_0^3}\lp \f{\omega_L}{\omega_0} \rp^3 \lp \f{\Gamma}{\omega_0 - \omega_L}  \rp^2 I \\
		&= \f{3\pi c^2}{2\hbar \omega_0^3}\lp \f{\omega_L}{\omega_0} \rp^3 \f{I\Gamma^2 }{\omega_R^2} 2P_{e,ii}\\
		&= \f{3\pi c^2}{\hbar \omega_0^3}\lp \f{\omega_L}{\omega_0} \rp^3 \Gamma^2 P_{e,ii} \f{\hbar^2}{e^2}\f{c}{8\pi} \f{4e^2\omega_0^3}{3\hbar c^3 \Gamma}\\
		&= \boxed{\f{ 1}{2}\lp \f{\omega_L}{\omega_0} \rp^3 \Gamma P_{e,ii}}
	\end{align*}
	
	
	
	
		\item (Optional) We describe scattering as spontaneous emission from a virtual energy level. The energy diagram for this process is:\\
		
		
		The lifetime of this virtual state is:
		\begin{align*}
			blah
		\end{align*}
	\end{enumerate}
	
	\item 
	
	\begin{enumerate}[label=(\roman*)]
		\item  The $D_{1,2}$ lines of Na have $\omega_0 \approx 2\pi \cdot 508 $ THz. Now the infrared laser at 985 nm has $\omega_L = 2\pi \cdot 304$ THz, which corresponds to a detuning of $2\pi\cdot 204$ THz, which is much larger than the detuning of the yellow laser (a few GHz). From the RWA expression, we see that the trap depth goes like $I(0)/\delta$. Assuming that the waist radius is the same for both lasers, we see that to get the same trap depth, a far detuned laser requires more power and vice versa. However, the scattering rate goes like $I(0)/\delta^2$. The most ideal optical trap requires sufficient trap depth plus low scattering rate. From the two factors ($U,R_\text{sc}$) and their dependence on $I$ and $\delta$ we conclude that infrared laser is more suitable than the yellow laser. 
		
		\item Here we want to calculate the required power and scattering rate for each of the two types of lasers. Using results from time-dependent perturbation theory, we have two equations:
		\begin{align*}
			&U_i = \f{I(0) \al(\omega_L)}{2c\epsilon_0} =   \f{3\pi c^2}{2\omega_0^3} \lp \f{\Gamma}{\omega_0 - \omega_L} + \f{\Gamma}{\omega_0 + \omega_L}\rp I(0) = k_B \cdot 10 \, \mu\text{K} \\
			& R_\text{sc,i} = \f{3\pi c^2}{2\hbar \omega_0^3}\lp \f{\omega_L}{\omega_0} \rp^3 \lp \f{\Gamma}{\omega_0 - \omega_L} + \f{\Gamma}{\omega_0 + \omega_L} \rp^2 I(0).
		\end{align*}
	where we have converted to CGS units in the first line. To solve for $P,R_{\text{sc,ii}}$, we will use $I(0)=2P/\pi w^2$ and Mathematica. For each laser, we will have:
	\begin{align*}
		\text{Yellow laser: } \,\,& 
		P = 0.102 \,\text{mW}; \quad 
		R_{\text{sc,ii}} = 7637 \,\text{s}^{-1}\\
		\text{Infrared laser:}\,\,&  
		P = 9.871 \, \text{mW}; \quad 
		R_{\text{sc,ii}} = 17.1102 \,\text{s}^{-1}
	\end{align*}

	\end{enumerate}
\end{enumerate}


\noindent \textbf{2. Magic Wavelength Optical Trap.} Here we have a system with lower state $\ket{S}$, upper state $\ket{P}$ with bare energy separation $\hbar \omega_{PS}$. The dipole moment is $d_{PS}$. 

\begin{enumerate}[label=(\alph*)]
	\item 
	
	\begin{enumerate}[label=(\roman*)]
		\item The AC Stark shift for the state $S$ is given by 
		\begin{align*}
			\Delta E_S = -\f{1}{4}\al(\omega_L) \mathcal{E}^2 = -\f{d_{PS}^2\mathcal{E}^2}{4\hbar}\lp \f{1}{\omega_{PS} - \omega_L} + \f{1}{\omega_{PS} + \omega_L} \rp
		\end{align*}
		
		\item 
		
		\item 
	\end{enumerate}
	
	\item 
	
	\begin{enumerate}[label=(\roman*)]
		\item 
		
		\item 
		
		
	\end{enumerate}
\end{enumerate}


\noindent \textbf{3. Species-Dependent and Spin-Dependent AC Stark shift}	
	
\begin{enumerate}[label=(\alph*)]
	\item 
	
	\begin{enumerate}[label=(\roman*)]
		\item 
		
		\item 
		

	\end{enumerate}
	
	\item 
	
	\begin{enumerate}[label=(\roman*)]
		\item 
		
		\item 
		
		
	\end{enumerate}
\end{enumerate}
	
	
\end{document}








