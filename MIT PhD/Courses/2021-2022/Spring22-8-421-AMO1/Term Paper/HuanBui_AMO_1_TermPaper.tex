\documentclass[reprint,
%superscriptaddress,
%groupedaddress,
%unsortedaddress,
%runinaddress,
%frontmatterverbose, 
%preprint,
%preprintnumbers,
nofootinbib,
%nobibnotes,
%bibnotes,
amsmath,amssymb,
aps]{revtex4-1}
\usepackage{physics}
\usepackage{graphicx}
\usepackage{caption}
\usepackage{amsmath}
\usepackage{bm}
\usepackage{dcolumn}% Align table columns on decimal point
\usepackage{framed}
\usepackage{empheq}
\usepackage{amsfonts}
\usepackage{esint}
\usepackage[makeroom]{cancel}
\usepackage{dsfont}
\usepackage{centernot}
\usepackage{mathtools}
\usepackage{subcaption}
\usepackage{bigints}
\usepackage{empheq}
\usepackage{tensor}
\usepackage{xcolor}
%\definecolor{colby}{rgb}{0.0, 0.0, 0.5}
\definecolor{MIT}{RGB}{163, 31, 52}
\usepackage[pdftex]{hyperref}
%\hypersetup{colorlinks,urlcolor=colby}
\hypersetup{colorlinks,linkcolor={MIT},citecolor={MIT},urlcolor={MIT}}  
\usepackage[left=1in,right=1in,top=1in,bottom=1in]{geometry}
\setcounter{MaxMatrixCols}{20}
% \usepackage{newpxtext,newpxmath}
\newcommand*\widefbox[1]{\fbox{\hspace{2em}#1\hspace{2em}}}

\newcommand{\p}{\partial}
\newcommand{\R}{\mathbb{R}}
\newcommand{\C}{\mathbb{C}}
\newcommand{\lag}{\mathcal{L}}
\newcommand{\nn}{\nonumber}
\newcommand{\ham}{\mathcal{H}}
\newcommand{\M}{\mathcal{M}}
\newcommand{\I}{\mathcal{I}}
\newcommand{\K}{\mathcal{K}}
\newcommand{\F}{\mathcal{F}}
\newcommand{\w}{\omega}
\newcommand{\lam}{\lambda}
\newcommand{\al}{\alpha}
\newcommand{\be}{\beta}
\newcommand{\x}{\xi}
\newcommand{\G}{\mathcal{G}}
\newcommand{\f}[2]{\frac{#1}{#2}}
\newcommand{\ift}{\infty}
\newcommand{\lp}{\left(}
\newcommand{\rp}{\right)}
\newcommand{\lb}{\left[}
\newcommand{\rb}{\right]}
\newcommand{\lc}{\left\{}
\newcommand{\rc}{\right\}}
\newcommand{\V}{\mathbf{V}}
\newcommand{\U}{\mathcal{U}}
\newcommand{\Id}{\mathcal{I}}
\newcommand{\D}{\mathcal{D}}
\newcommand{\Z}{\mathcal{Z}}

\definecolor{gris245}{RGB}{245,245,245}
\definecolor{olive}{RGB}{50,140,50}
\definecolor{brun}{RGB}{175,100,80}

\newcommand{\diag}{\text{diag}}
\newcommand{\psirot}{\ket{\psi_\text{rot}(t)} }
\newcommand{\RWA}{\ham_\text{rot}^\text{RWA}}

% 3j symbol
\newcommand{\tj}[6]{ \begin{pmatrix}
		#1 & #2 & #3 \\
		#4 & #5 & #6 
\end{pmatrix}}



\begin{document}
	
	
	
\preprint{APS/123-QED}

\title{Observation of the Gravitational Aharonov-Bohm Effect}
\thanks{A footnote to the article title}%
\author{Huan Q. Bui}
\email{huanbui@mit.edu}
\affiliation{
	Department of Physics, Massachusetts Institute of Technology\\}
\date{\today}


\begin{abstract}
	An article usually includes an abstract, a concise summary of the work
	covered at length in the main body of the article. 
%	\begin{description}
%		\item[Usage]
%		Secondary publications and information retrieval purposes.
%		\item[Structure]
%		You may use the \texttt{description} environment to structure your abstract;
%		use the optional argument of the \verb+\item+ command to give the category of each item. 
%	\end{description}
\end{abstract}

%\keywords{Suggested keywords}%Use showkeys class option if keyword
%display desired
\maketitle

%\tableofcontents



\section{Introduction}



Section \ref{sect:1}... outlines some theory. While these topics well-known and are standard subjects of many quantum mechanics textbooks, the author feels compelled to present a short summary to have the essentials at our fingertips. 


Section \ref{sect:2}... presents the experimental observation of the gravitational Aharonov-Bohm effect. The theory and results are addressed. A proposal is reviewed and a recently published work is described. However, the main focus is the experimental technique: atom interferometry. 








\section{Berry phase}\label{sect:1}
Consider $\mathcal{H}(\bm{R}(t))$, a time-dependent Hamiltonian parameterized by a family of variables $\bm{R}(t)$. Let $\ket{\psi(0)} = \ket{n(\bm{R}(0))}$ where $\ket{n(\bm{R}(0))}$ is the $n^\text{th}$ eigenstate of $\mathcal{H}(\bm{R}(0))$. By the adiabatic theorem, $\ket{\psi(t)}$ is $\ket{n(\bm{R}(t))}$, the $n^\text{th}$ instantaneous eigenstate of $\mathcal{H}(t)$, up to a phase factor, i.e.,
\begin{align*}
\ket{\psi(t)} = e^{ -\f{i}{\hbar }\int_0^t E_n(\bm{R} (t'))\,dt' } \exp(i\gamma_n(t)) \ket{n(\bm{R}(t))},
\end{align*}
where $\gamma_n(t)$ is called the \textit{Berry phase}. Since $\ket{\psi(t)}$ solves the Schr\"{o}dinger equation $\mathcal{H}(\bm{R}(t)) \ket{\psi(t)} = i\hbar (d/dt) \ket{{\psi}(t)}$, we have 
\begin{align*}
\dot{\gamma}_n(t) = i \bra{n(\bm{R}(t))} \nabla_{\bm{R}}  \ket{n(\bm{R}(t)) } \cdot \dot{\bm{R}}(t).
\end{align*}
In particular, at some final time $t_f$,
\begin{align}\label{eq:def}
\gamma_n(t_f) =   \int^{\bm{R}_f}_{\bm{R}_i} i \bra{n(\bm{R})} \nabla_{\bm{R}}  \ket{n(\bm{R}) } \cdot d{\bm{R}},
\end{align} 
which depends only on the path in parameter space over which the evolution takes place. Define the \textit{Berry connection}, 
\begin{align*}
\bm{A}_n(\bm{R}) = i \bra{n(\bm{R})} \nabla_{\bm{R}}  \ket{n(\bm{R}) }
\end{align*} 
and consider gauge transformation in parameter phase of instantaneous eigenstates $\ket{n(\bm{R})} \to \ket{\widetilde{n}(\bm{R})} = e^{-i\beta(\bm{R})}\ket{n(\bm{R})}$. The Berry connection transforms like the electromagnetic  vector potential:
\begin{align*}
\bm{A}_n(\bm{R}) \to  \widetilde{\bm{A}_n}(\bm{R}) = \bm{A}_n(\bm{R}) + \nabla_{\bm{R}} \beta (\bm{R}). 
\end{align*}
and therefore is also known as the Berry potential. Meanwhile the Berry phase transforms as
\begin{align*}
\widetilde{\gamma_n} (\bm{R}) = \int_{\bm{R}_i}^{\bm{R}_f} \widetilde{A_n}(\bm{R})\cdot d\bm{R} = \gamma_n(\bm{R}_f)  + \beta(\bm{R}_f) - \beta({\bm{R}_i})
\end{align*}
which is gauge-invariant exactly when the Hamiltonian evolution is cyclical in parameter space, i.e., $\bm{R}(t_f) = \bm{R}(0)$. A remarkable consequence of cyclic evolutions is that the Berry phase is well-defined and is measurable by means of interferometry. 

The Berry phase is topological in the sense that it depends on the topology of the parameter space containing the path $C$ along which the system evolves. Consider a closed path $C$ in a parameter space $\mathfrak{R}$. If $\mathfrak{R}$ is one-dimensional, the Berry phase vanishes. In the case that $\mathfrak{R}$ is three-dimensional, Stokes' theorem states that 
\begin{align*}
\gamma_n(C) &=  \oint_C \bm{A}_n(\bm{R}) \cdot d{\bm{R}} \\
&= \iint_{S} \lb  \nabla_{\bm{R}} \times \bm{A}_n(\bm{R}) \rb  \cdot d\vec{S} \equiv \iint_S \bm{D}_n \cdot d\vec{S}
\end{align*}
where $S$ is the surface with boundary $C$ and $\bm{D}_n \equiv \nabla_{\bm{R}} \times \bm{A}(\bm{R}) $ is the \textit{Berry curvature}. We immediately see that if we think of the Berry connection as the electromagnetic vector potential, then the Berry curvature plays the role of the associated magnetic field, which is gauge-invariant. 



\subsection{Example: Spin-1/2 in a magnetic field}

The Hamiltonian for a spin-1/2 in a magnetic field has the form
\begin{align*}
\mathcal{H}(\bm{B}) = \bm{B}\cdot \bm{\sigma} = r \begin{pmatrix}
\cos\theta & \sin\theta e^{-i\phi} \\ \sin\theta e^{i\phi} & -\cos\theta
\end{pmatrix}.
\end{align*}
The eigenvalues are $\pm r$, with associated eigenvectors
\begin{align*}
\ket{+} = \begin{pmatrix}
\cos(\theta/2) \\ e^{i\phi}\sin(\theta/2) 
\end{pmatrix}, \quad \ket{-} = \begin{pmatrix}
\cos(\theta/2) \\ - e^{i\phi}\sin(\theta/2) 
\end{pmatrix} 
\end{align*}
Since the adiabatic theorem requires that the relevant instantaneous eigenstates are non-degenerate, we require that $r  \neq 0$. The components of the Berry connection for $\ket{+}$ are readily calculated:
\begin{align*}
&A_r = i\bra{+}\p_r \ket{+} = 0\\
&A_\theta = i\bra{+}\p_\theta\ket{+} = 0\\
&A_\phi = i\bra{+}\p_\phi \ket{+} = \f{\cos\theta - 1}{2}. 
\end{align*}
Here, $\bm{A}(\bm{B})$ is actually not defined on the negative $z$-axis. Consider a closed, piece-wise smooth path $C$ enclosing a surface $S$ such that no point of $S$ lies on the negative $z$-axis. The Berry phase is 
\begin{align*}
\gamma[C] = \oint_C \bm{A}(\bm{B}) \cdot d\bm{B} = \iint_S \nabla \times \bm{A}(\bm{B})  \, d\bm{S} = -\f{\Omega}{2} 
\end{align*}
where $\Omega$ is nothing but the solid angle enclosed by $S$. We note that if we had chosen the $z$-axis to lie in the opposite direction, then the solid angle would have been $|\Omega'| = 4\pi - |\Omega|$.  While this appears problematic,  $\exp(i\gamma[C])$ is the same in both cases, and therefore the Berry phase is still well-defined. 




\subsection{Aharonov-Bohm Effect}


The Aharonov-Bohm effect is often discussed in the context of the path integral formulation of quantum mechanics where one compares the wavefunctions passing along two (distinct) paths in a vector potential associated with some magnetic field $\bm{B}$.  Here, the author presents M. V. Berry's interpretation of the Aharonov-Bohm effect as a Berry phase change \cite{berry1984quantal}. This presentation is not only a highly illustrative application of \eqref{eq:def}, but also avoids issues with single-valuedness of wavefunctions that arise in \cite{aharonov1959significance} and \cite{ehrenberg1949refractive}. 


To start, consider particles of mass $m$ and charge $q$ in a magnetic field $\bm{B}$ generated by a thin long solenoid. For positions $\bm{R}$ outside the solenoid and enclosing it by a closed path $C$, the magnetic field is zero but the circulation of $\bm{A}$ along $C$ is the total magnetic flux:
\begin{align*}
\oint_C \bm{A}(\bm{R}) \cdot d\bm{R} = \Phi_B.
\end{align*}
Let the particles be confined to a box at $\bm{R}$. The particle Hamiltonian depends on position ${\bm{r}}$ and conjugate momentum $\bm{p}$ as $\mathcal{H} = \mathcal{H}(\bm{p}, \bm{r} - \bm{R})$ in the case when $\bm{A} = 0$. Let the wavefunctions be $\psi_n(\bm{r} - \bm{R})$ with eigenvalues $E_n$.  When $\vec{A} \neq 0$, the Hamiltonian satisfies
\begin{align*}
\mathcal{H}(\bm{p}-q\bm{A}(\bm{R}), \bm{r}- \bm{R}) \ket{n(\bm{R})} = E_n \ket{n(\bm{R})}
\end{align*}
since the vector potential does not affect the energies. The solutions for this Hamiltonian,
\begin{align*}
\bra{\bm{r}} \ket{n(\bm{R})} = \exp\lb \f{iq}{\hbar} \int_{\bm{R}}^{\bm{r}} d\bm{r}'\cdot \bm{A}(\bm{r}') \rb \psi_n(\bm{r} - \bm{R}),
\end{align*}
can be obtained by considering the gauge freedom of $\bm{A}$ and the fact that $\bm{B}=0$ for all $\bm{R}$. With this, we can calculate the total phase change after transporting the box around $C$. Starting with
\begin{align*}
&\bra{n(\bm{R})} \nabla_{\bm{R}} \ket{n(\bm{R})}\\
&= \int d^3\bm{r}\psi^*_n(\bm{r} - \bm{R}) \lb \f{-iq}{\hbar}\psi_n(\bm{r} - \bm{R}) + \nabla_{\bm{R}} \psi_n(\bm{r} - \bm{R}) \rb\\
&= -\f{iq\bm{A}(\bm{R})}{\hbar},
\end{align*}
we find 
\begin{align*}
\gamma_n(C) = \f{q}{\hbar}\oint_C \bm{A}(\bm{R}) \cdot d\bm{R} = \f{q\Phi_B}{\hbar}.
\end{align*}
Note that that $\psi_n(C)$ is independent of both $n$ and $C$, so long as $C$ encloses the solenoid once.



\section{Observation of a Gravitational Aharonov-Bohm effect} \label{sect:2}






\subsection{Experimental Techniques}









\subsubsection{Atom interferometry}


\subsubsection{Mach-Zehnder atom interferometer}

\subsubsection{Ramsey-Bord\'{e} atom interferemeter}



\subsubsection{Raman (look at Steven Chu paper)}

\subsubsection{Bragg diffraction}

Bragg diffraction is used as a tool for large-momentum transfer beam splitters in atom interferometry. \textcolor{blue}{Say something about how the higher momentum transfer the better...}  


\textcolor{purple}{What is the idea of Bragg diffraction?...}




The following treatment of Bragg diffraction follows from \cite{estey2016precision}. Let $\omega_0$ be the transition frequency, $\ket{g}$  the ground state, and $\ket{e}$  the excited state and $\Omega \equiv \vec{d}_\text{ge}\cdot \vec{E}_0/\hbar$ be the Rabi frequency, where $\vec{d}_\text{ge}$ is the dipole moment matrix element of the atom. Consider the interaction between the atom and an electric field of the form $\vec{E} = \vec{E}_0(e^{ikz - i \omega_Lt} + e^{-ikz+i\omega_Lt}) /2 $.  In the near-resonance limit where $\Delta \equiv \omega_L - \omega_0 \ll \omega_0$, we may make the rotating wave approximation\footnote{The procedure for which is standard: Go to the frame rotating at $\omega_L$, eliminate the counter-rotating term $e^{\pm i(\omega_L+\omega_0)t}$ in the transformed interaction Hamiltonian, then transform back to the stationary frame.} to obtain 
\begin{align*}
\mathcal{H} = \underbrace{\f{\vec{p}\,^2}{2m} + \hbar \omega_0 \ketbra{e}}_{\equiv\mathcal{H}_0} - \lp \f{\hbar\Omega}{2}e^{ikz - i\omega_Lt} \ket{e}\bra{g} + h.c.\rp.
\end{align*}
For generalized electric fields, $\vec{E} = \sum_j \vec{E}_j \cos(k_j z - (\omega_L - \delta_j)t)$, a generalized rotating wave approximation gives
\begin{align*}
\mathcal{H} \approx \mathcal{H}_0 -  \lp \sum_j \f{\hbar\Omega_j}{2} e^{ik_j z - i(\omega_L - \delta_j) t} \ket{e}\bra{g} + h.c. \rp
\end{align*}
where $|\delta_j|\ll \omega_L$ are small detunings from the ``main" frequency $\omega_L$ and $\Omega_j \equiv \vec{d}_\text{ge}\cdot \vec{E}_j / \hbar$. Going back to the rotating frame, the Hamiltonian is 
\begin{align*}
\mathcal{H}^\text{rot} = &\f{\vec{p}\,^2}{2m} - \hbar \Delta \ketbra{e} \\
&- \lp \sum_j \f{\hbar \Omega_j}{2} e^{ik_jz + i \delta_i t} \ket{e}\bra{g} + h.c. \rp
\end{align*}
In Bragg diffraction, the electric field is a nearly-standing wave. After the rotating wave approximation, 
\begin{align*}
\vec{E} \to \f{\vec{E}_0}{2}u(z,t) = \f{\vec{E}_0}{2} \lb  e^{-ikz + i\delta t} + e^{ikz-i\delta t}\rb
\end{align*}
where $k$ is the laser wavevector, $2\delta$ is the detuning between the counter-propagating beams. With this, 
\begin{align*}
\mathcal{H}^\text{rot} = \f{\vec{p}\,^2}{2m} -\hbar \Delta  \ketbra{e} -\lp \f{\hbar\Omega u(z,t)}{2}\ket{e}\bra{g} + h.c.\rp.
\end{align*}
The solutions to this Hamiltonian have the form 
\begin{align*}
\ket{\Psi} = e(z,t)\ket{e} + g(z,t)\ket{g}.
\end{align*} 
Plugging this ansatz into the Schr\"{o}dinger equation with $\mathcal{H}^\text{rot}$ we find 
\begin{align*}
i\hbar \dot{e}(z,t) &= \f{\vec{p}\,^2}{2m} e(z,t) - \hbar \Delta e - \f{\hbar\Omega}{2}ug(z,t)\\
i\hbar \dot{g}(z,t) &= \f{\vec{p}\,^2}{2m} g(z,t) - \f{\hbar\Omega^*}{2}u^*e(z,t).
\end{align*}








\subsection{Literature Review}


\subsubsection{A proposal}


\subsubsection{Observation of GAB effect}





\begin{acknowledgments}
	I thank Trader Joe's. \cite{berry1984quantal}
\end{acknowledgments}



\bibliography{term_paper_references}% Produces the bibliography via BibTeX.











\end{document}
