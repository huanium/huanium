\documentclass{article}
\usepackage{physics}
\usepackage{graphicx}
\usepackage{caption}
\usepackage{amsmath}
\usepackage{bm}
\usepackage{framed}
\usepackage{authblk}
\usepackage{empheq}
\usepackage{amsfonts}
\usepackage{esint}
\usepackage[makeroom]{cancel}
\usepackage{dsfont}
\usepackage{centernot}
\usepackage{mathtools}
\usepackage{subcaption}
\usepackage{bigints}
\usepackage{amsthm}
\theoremstyle{definition}
\newtheorem{lemma}{Lemma}
\newtheorem{defn}{Definition}[section]
\newtheorem{prop}{Proposition}[section]
\newtheorem{rmk}{Remark}[section]
\newtheorem{thm}{Theorem}[section]
\newtheorem{exmp}{Example}[section]
\newtheorem{prob}{Problem}[section]
\newtheorem{sln}{Solution}[section]
\newtheorem*{prob*}{Problem}
\newtheorem{exer}{Exercise}[section]
\newtheorem*{exer*}{Exercise}
\newtheorem*{sln*}{Solution}
\usepackage{empheq}
\usepackage{tensor}
\usepackage{xcolor}
%\definecolor{colby}{rgb}{0.0, 0.0, 0.5}
\definecolor{MIT}{RGB}{163, 31, 52}
\usepackage[pdftex]{hyperref}
%\hypersetup{colorlinks,urlcolor=colby}
\hypersetup{colorlinks,linkcolor={MIT},citecolor={MIT},urlcolor={MIT}}  
\usepackage[left=1in,right=1in,top=1in,bottom=1in]{geometry}
\setcounter{MaxMatrixCols}{20}
\usepackage{newpxtext,newpxmath}
\newcommand*\widefbox[1]{\fbox{\hspace{2em}#1\hspace{2em}}}

\newcommand{\p}{\partial}
\newcommand{\R}{\mathbb{R}}
\newcommand{\C}{\mathbb{C}}
\newcommand{\lag}{\mathcal{L}}
\newcommand{\nn}{\nonumber}
\newcommand{\ham}{\mathcal{H}}
\newcommand{\M}{\mathcal{M}}
\newcommand{\I}{\mathcal{I}}
\newcommand{\K}{\mathcal{K}}
\newcommand{\F}{\mathcal{F}}
\newcommand{\w}{\omega}
\newcommand{\lam}{\lambda}
\newcommand{\al}{\alpha}
\newcommand{\be}{\beta}
\newcommand{\x}{\xi}

\newcommand{\G}{\mathcal{G}}

\newcommand{\f}[2]{\frac{#1}{#2}}

\newcommand{\ift}{\infty}

\newcommand{\lp}{\left(}
\newcommand{\rp}{\right)}

\newcommand{\lb}{\left[}
\newcommand{\rb}{\right]}

\newcommand{\lc}{\left\{}
\newcommand{\rc}{\right\}}


\newcommand{\V}{\mathbf{V}}
\newcommand{\U}{\mathcal{U}}
\newcommand{\Id}{\mathcal{I}}
\newcommand{\D}{\mathcal{D}}
\newcommand{\Z}{\mathcal{Z}}

%\setcounter{chapter}{-1}


\usepackage{enumitem}



\usepackage{listings}
\captionsetup[lstlisting]{margin=0cm,format=hang,font=small,format=plain,labelfont={bf,up},textfont={it}}
\renewcommand*{\lstlistingname}{Code \textcolor{violet}{\textsl{Mathematica}}}
\definecolor{gris245}{RGB}{245,245,245}
\definecolor{olive}{RGB}{50,140,50}
\definecolor{brun}{RGB}{175,100,80}

%\hypersetup{colorlinks,urlcolor=colby}
\lstset{
	tabsize=4,
	frame=single,
	language=mathematica,
	basicstyle=\scriptsize\ttfamily,
	keywordstyle=\color{black},
	backgroundcolor=\color{gris245},
	commentstyle=\color{gray},
	showstringspaces=false,
	emph={
		r1,
		r2,
		epsilon,epsilon_,
		Newton,Newton_
	},emphstyle={\color{olive}},
	emph={[2]
		L,
		CouleurCourbe,
		PotentielEffectif,
		IdCourbe,
		Courbe
	},emphstyle={[2]\color{blue}},
	emph={[3]r,r_,n,n_},emphstyle={[3]\color{magenta}}
}

\newcommand{\diag}{\text{diag}}
\newcommand{\psirot}{\ket{\psi_\text{rot}(t)} }
\newcommand{\RWA}{\ham_\text{rot}^\text{RWA}}

% 3j symbol
\newcommand{\tj}[6]{ \begin{pmatrix}
		#1 & #2 & #3 \\
		#4 & #5 & #6 
\end{pmatrix}}



\begin{document}
	\begin{titlepage}\centering
		\clearpage
		\title{\textsc{$\,$\\$\,$\\$\,$\\$\,$\\$\,$\\ \bf{A Quick Guide to \\ Geometric Phase: Theory and Experimental Observation}}}
		\author{\bigskip Huan Q. Bui}
		 \affil{DEPARTMENT OF PHYSICS \\ MASSACHUSETTS INSTITUTE OF TECHNOLOGY}
		\date{\today}
		\maketitle
		\thispagestyle{empty}
	\end{titlepage}




\newpage
\tableofcontents
\newpage





\section{Berry phase}
Consider the Hamiltonian described in the adiabatic theorem.  Let $\ket{\psi(0)} = \ket{n(\bm{R}(0))}$ where $\ket{n(\bm{R}(0))}$ is the $n^\text{th}$ eigenstate of $\mathcal{H}(\bm{R}(0))$. In view of the adiabatic theorem, $\ket{\psi(t)}$ must be $\ket{n(\bm{R}(t))}$, the $n^\text{th}$ instantaneous eigenstate of $\mathcal{H}(t)$, up to a phase factor:
\begin{align*}
\ket{\psi(t)} = \exp\lp -\f{i}{\hbar }\int_0^t E_n(\bm{R} (t'))\,dt'   \rp \exp(i\gamma_n(t)) \ket{n(\bm{R}(t))},
\end{align*}
where $\gamma_n(t)$ is a function of time and is called the \textbf{Berry phase} or \textbf{geometric phase}. Since $\ket{\psi(t)}$ solves the Schr\"{o}dinger equation $\mathcal{H}(\bm{R}(t)) \ket{\psi(t)} = i\hbar (d/dt) \ket{{\psi}(t)}$, we have by the chain rule in calculus:
\begin{align*}
\dot{\gamma}_n(t) = i \bra{n(\bm{R}(t))} \nabla_{\bm{R}}  \ket{n(\bm{R}(t)) } \cdot \dot{\bm{R}}(t).
\end{align*}
In particular, we find that at some final time $t_f$, the Berry phase is given by 
\begin{align*}
\gamma_n(t_f) = \int_{0}^{t_f} \dot{\gamma}_n(t')\,dt' =   \int^{\bm{R}_f}_{\bm{R}_i} i \bra{n(\bm{R})} \nabla_{\bm{R}}  \ket{n(\bm{R}) } \cdot d{\bm{R}},
\end{align*} 
which depends only on the path in parameter space over which the evolution takes place. Define the \textbf{Berry connection}, 
\begin{align*}
\bm{A}_n(\bm{R}) = i \bra{n(\bm{R})} \nabla_{\bm{R}}  \ket{n(\bm{R}) }
\end{align*} 
and consider gauge transformation in parameter phase of instantaneous eigenstates. The Berry connection is also known as the Berry potential since it transforms like the electromagnetic  vector potential:
\begin{align*}
\ket{n(\bm{R})} \to \ket{\widetilde{n}(\bm{R})} = e^{-i\beta(\bm{R})}\ket{n(\bm{R})}  \implies \bm{A}_n(\bm{R}) \to  \widetilde{\bm{A}_n}(\bm{R}) = \bm{A}_n(\bm{R}) + \nabla_{\bm{R}} \beta (\bm{R}). 
\end{align*}
Meanwhile the Berry phase transforms as
\begin{align*}
\widetilde{\gamma_n} (\bm{R}) = \int_{\bm{R}_i}^{\bm{R}_f} \widetilde{A_n}(\bm{R})\cdot d\bm{R} = \gamma_n(\bm{R}_f)  + \beta(\bm{R}_f) - \beta({\bm{R}_i})
\end{align*}
which is gauge-invariant exactly when the Hamiltonian evolution is cyclical in parameter space, i.e., $\bm{R}(t_f) = \bm{R}(0)$. A remarkable consequence of cyclic evolutions is that the Berry phase is well-defined and is a measurable quantity. To see this, consider the overlap integral between the initial and final state of the system $\bra{\psi(0)}\ket{\psi(t_f)} \sim e^{i\gamma_n(t_f)}$. We see that this relative phase can be observed via interferometry. \\

The Berry phase is topological in the sense that it depends on the topology of the parameter space containing the path $C$ along which the system evolves. Consider a closed path $C$ in a parameter space $\mathfrak{R}$. In the case that $\mathfrak{R}$ is one-dimensional, the Berry phase vanishes. In the case that $\mathfrak{R}$ is three-dimensional, we can invoke Stokes' theorem to obtain 
\begin{align*}
\gamma_n(C) =  \oint_C \bm{A}_n(\bm{R}) \cdot d{\bm{R}} = \iint_{S} \lb  \nabla_{\bm{R}} \times \bm{A}_n(\bm{R}) \rb  \cdot d\vec{S} \equiv \iint_S \bm{D}_n \cdot d\vec{S}
\end{align*}
where $S$ is the surface with boundary $C$ and $\bm{D}_n \equiv \nabla_{\bm{R}} \times \bm{A}(\bm{R}) $ is the \textbf{Berry curvature}. We immediately see that if we think of te Berry connection as the electromagnetic vector potential, then the Berry curvature plays the role of the associated magnetic field, which is gauge-invariant. 






\subsection{Example: Spin-1/2 in a magnetic field}

The Hamiltonian for a spin-1/2 in a magnetic field has the form
\begin{align*}
\mathcal{H}(\vec{B}) = \vec{B}\cdot \vec{\sigma} = r \begin{pmatrix}
\cos\theta & \sin\theta e^{-i\phi} \\ \sin\theta e^{i\phi} & -\cos\theta
\end{pmatrix}.
\end{align*}
The eigenvalues and associated eigenvectors are
\begin{align*}
+r: \quad \ket{+} = \begin{pmatrix}
\cos(\theta/2) \\ e^{i\phi}\sin(\theta/2) 
\end{pmatrix} \quad \text{ and } \quad -r: \quad \ket{-} = \begin{pmatrix}
\cos(\theta/2) \\ - e^{i\phi}\sin(\theta/2) 
\end{pmatrix} 
\end{align*}
Since the adiabatic theorem requires that the relevant instantaneous eigenstates are non-degenerate, we require that $r  \neq 0$. The components of the Berry connection for $\ket{+}$ are readily calculated:
\begin{align*}
A_r = i\bra{+}\p_r \ket{+} = 0 \quad\quad A_\theta = i\bra{+}\p_\theta\ket{+} = 0\quad\quad 
A_\phi = i\bra{+}\p_\phi \ket{+} = \f{\cos\theta - 1}{2}. 
\end{align*}
Here, $\vec{A}(\vec{B})$ is actually not defined on the negative $z$-axis. Consider a closed, piece-wise smooth path $C$ enclosing a surface $S$ such that no point of $S$ lies on the negative $z$-axis. The Berry phase is 
\begin{align*}
\gamma[C] = \oint_C \vec{A}(\vec{B}) \cdot d\vec{B} = \iint_S \nabla_{\{r,\theta,\phi\}} \times \vec{A}(\vec{B})  \, d\vec{S} = -\f{\Omega}{2} 
\end{align*}
where $\Omega$ is nothing but the solid angle enclosed by $S$. We note that if we had chosen the $z$-axis to lie in the opposite direction, then the solid angle would have been $|\Omega'| = 4\pi - |\Omega|$.  While this appears problematic,  $\exp(i\gamma[C])$ is the same in both cases, and therefore the Berry phase is still well-defined. 




\section{Aharonov-Bohm Effect}

Consider a charged particle in a magnetic field. 



\section{Observation of a Gravitational Aharonov-Bohm effect}




\subsection{Experimental Techniques}


\subsubsection{Atom interferometry: The Mach-Zehnder Interferometer for Atoms}

\subsubsection{Bragg diffraction}
















\end{document}
