\documentclass{article}
\usepackage{physics}
\usepackage{graphicx}
\usepackage{caption}
\usepackage{amsmath}
\usepackage{bm}
\usepackage{framed}
\usepackage{authblk}
\usepackage{empheq}
\usepackage{amsfonts}
\usepackage{esint}
\usepackage[makeroom]{cancel}
\usepackage{dsfont}
\usepackage{centernot}
\usepackage{mathtools}
\usepackage{subcaption}
\usepackage{bigints}
\usepackage{amsthm}
\theoremstyle{definition}
\newtheorem{lemma}{Lemma}
\newtheorem{defn}{Definition}[section]
\newtheorem{prop}{Proposition}[section]
\newtheorem{rmk}{Remark}[section]
\newtheorem{thm}{Theorem}[section]
\newtheorem{exmp}{Example}[section]
\newtheorem{prob}{Problem}[section]
\newtheorem{sln}{Solution}[section]
\newtheorem*{prob*}{Problem}
\newtheorem{exer}{Exercise}[section]
\newtheorem*{exer*}{Exercise}
\newtheorem*{sln*}{Solution}
\usepackage{empheq}
\usepackage{tensor}
\usepackage{xcolor}
%\definecolor{colby}{rgb}{0.0, 0.0, 0.5}
\definecolor{MIT}{RGB}{163, 31, 52}
\usepackage[pdftex]{hyperref}
%\hypersetup{colorlinks,urlcolor=colby}
\hypersetup{colorlinks,linkcolor={MIT},citecolor={MIT},urlcolor={MIT}}  
\usepackage[left=1in,right=1in,top=1in,bottom=1in]{geometry}

\usepackage{newpxtext,newpxmath}
\newcommand*\widefbox[1]{\fbox{\hspace{2em}#1\hspace{2em}}}

\newcommand{\p}{\partial}
\newcommand{\R}{\mathbb{R}}
\newcommand{\C}{\mathbb{C}}
\newcommand{\lag}{\mathcal{L}}
\newcommand{\nn}{\nonumber}
\newcommand{\ham}{\mathcal{H}}
\newcommand{\M}{\mathcal{M}}
\newcommand{\I}{\mathcal{I}}
\newcommand{\K}{\mathcal{K}}
\newcommand{\F}{\mathcal{F}}
\newcommand{\w}{\omega}
\newcommand{\lam}{\lambda}
\newcommand{\al}{\alpha}
\newcommand{\be}{\beta}
\newcommand{\x}{\xi}

\newcommand{\G}{\mathcal{G}}

\newcommand{\f}[2]{\frac{#1}{#2}}

\newcommand{\ift}{\infty}

\newcommand{\lp}{\left(}
\newcommand{\rp}{\right)}

\newcommand{\lb}{\left[}
\newcommand{\rb}{\right]}

\newcommand{\lc}{\left\{}
\newcommand{\rc}{\right\}}


\newcommand{\V}{\mathbf{V}}
\newcommand{\U}{\mathcal{U}}
\newcommand{\Id}{\mathcal{I}}
\newcommand{\D}{\mathcal{D}}
\newcommand{\Z}{\mathcal{Z}}

%\setcounter{chapter}{-1}


\usepackage{enumitem}



\usepackage{listings}
\captionsetup[lstlisting]{margin=0cm,format=hang,font=small,format=plain,labelfont={bf,up},textfont={it}}
\renewcommand*{\lstlistingname}{Code \textcolor{violet}{\textsl{Mathematica}}}
\definecolor{gris245}{RGB}{245,245,245}
\definecolor{olive}{RGB}{50,140,50}
\definecolor{brun}{RGB}{175,100,80}

%\hypersetup{colorlinks,urlcolor=colby}
\lstset{
	tabsize=4,
	frame=single,
	language=mathematica,
	basicstyle=\scriptsize\ttfamily,
	keywordstyle=\color{black},
	backgroundcolor=\color{gris245},
	commentstyle=\color{gray},
	showstringspaces=false,
	emph={
		r1,
		r2,
		epsilon,epsilon_,
		Newton,Newton_
	},emphstyle={\color{olive}},
	emph={[2]
		L,
		CouleurCourbe,
		PotentielEffectif,
		IdCourbe,
		Courbe
	},emphstyle={[2]\color{blue}},
	emph={[3]r,r_,n,n_},emphstyle={[3]\color{magenta}}
}

\newcommand{\diag}{\text{diag}}
\newcommand{\psirot}{\ket{\psi_\text{rot}(t)} }
\newcommand{\RWA}{\ham_\text{rot}^\text{RWA}}


\begin{document}
\begin{framed}
\noindent Name: \textbf{Huan Q. Bui}\\
Course: \textbf{8.421 - AMO I}\\
Problem set: \textbf{\#3}\\
Due: Friday, Feb 25, 2022.
\end{framed}
	
	
\noindent \textbf{1. Determination of the fine structure constant, $\al$}


\begin{enumerate}[label=(\alph*)]
	\item Our goal is to write $\al$ in terms of $f_\infty = cR_\infty$ and $f_e = m_ec^2/h$. This can be done using the definitions. From $\al =e^2/4\pi \epsilon_0 \hbar c$ we find 
	\begin{align*}
	\al^2 = \f{e^4}{(4\pi \epsilon_0)^2 \hbar^2 c^2} = \f{1}{c^2} f_\infty \f{4\pi \hbar}{m_e} = f_\infty \f{2h}{m_ec^2} = \f{2f_\infty}{f_e}
	\end{align*} 
	where we have used the definition:
	\begin{align*}
	f_\infty = c R_\infty = c\f{m_ee^4}{8 \epsilon_0^2 h^3 c }  =\f{m_ee^4}{8 \epsilon_0^2 h^3 } = \f{m_ee^4}{(4\pi\epsilon_0 \hbar )^2 4\pi \hbar }.
	\end{align*}
	So, 
	\begin{align*}
	\boxed{\al = \sqrt{\f{2f_\infty}{f_e} }}
	\end{align*}
	From here, we see that 
	\begin{align*}
	\al = \sqrt{\f{2cR_\infty}{m_e c^2/h}}
	\end{align*}
	which depends only on the experimental values $R_\infty$ and $h/m_e$. The speed of light $c$ is \textit{defined}. 
	
	
	
	
	\item The de Broglie wavelength for a neutron beam of velocity $v$ is 
	\begin{align*}
	\lambda = \f{h}{m_n v}. 
	\end{align*}
	Therefore we see that by measuring $v$ and $\lambda$ we can measure the ratio $h/m_n$. 
	
	\item The momentum that an atom attains when absorbing a photon of frequency $\nu$ can be derived from conservation of momentum: 
	\begin{align*}
	mv_R = \f{h\nu}{c} \implies \f{h}{m} = \f{cv_R}{\nu}.
	\end{align*}
	
	\item Let's say that an atom absorbs a photon and gets a velocity $v_R$. Assuming that we are working in the non-relativistic limit, the atom sees oppositely Doppler-shifted $\nu_2$ and $\nu_1$:
	\begin{align*}
	&\nu_2' = \nu_2 \lp 1 + \f{v_R}{c} \rp\\
	&\nu_1' = \nu_1 \lp 1 - \f{v_R}{c} \rp
	\end{align*}
	Since we want $\nu_2' = \nu_1'$ which is on resonance with the atoms, we find that
	\begin{align*}
	v_R = c\lp \f{\nu_1- \nu_2}{\nu_1 + \nu_2}\rp.
	\end{align*}
	In the limit where $v_R \ll c$, we may take $\nu = \nu_1 \approx \nu_2$, so that $v_R = h\nu / mc$ and 
	\begin{align*}
	v_R = \f{h\nu}{mc} = \f{c}{2\nu}\Delta \nu \implies \Delta \nu = \f{2h\nu^2}{mc^2} \implies \Delta \omega = \f{2\hbar k^2}{m} \implies \boxed{\f{\hbar}{m} = \f{\Delta \omega}{2k^2}}
	\end{align*}
	where $\nu = \omega/2\pi$ and $k = \omega/c$.
	
	
	
	
	\item By conservation of energy, we must have, for the $m \to m' + \gamma$ transition,
	\begin{align*}
	mc^2 = m'c^2 + pc \implies p = \f{h}{\lambda} = c \Delta m  \implies \boxed{\f{h}{\Delta m} = c\lambda}
	\end{align*}
	where $p$ is the momentum of the photon and $m'<m$. 
\end{enumerate}





\noindent \textbf{2. Ground state energy of the helium atom.}


\begin{enumerate}[label=(\alph*)]
	\item Suppose that both electrons are in the 1s state and that the spin wavefunction is anti-symmetric, then the two-electron spatial wavefunction is the product of the single-electron (normalized) wavefunctions $\psi_{100}(\vec{r})$:
	\begin{align*}
	\Psi(\vec{r}_1, \vec{r}_2) 
	&= \psi_{100}(\vec{r}_1)\psi_{100}(\vec{2})\\
	&= \lb \f{1}{\sqrt{\pi}} \lp \f{Z_\text{He}}{a_0} \rp^{3/2}  e^{-Z_\text{He} r_1/a_0}\rb 
	\lb \f{1}{\sqrt{\pi}} \lp \f{Z_\text{He}}{a_0} \rp^{3/2}  e^{-Z_\text{He} r_2/a_0} \rb\\
	&= \f{1}{\pi} \lp \f{Z_\text{He}}{a_0} \rp^3 e^{-Z_\text{He}(r_1 + r_2)/a_0}.
	\end{align*}
	With this, we may calculate the interaction energy $\langle e^2/r_{12}\rangle$ as follows: 
	\begin{align*}
	\bigg\langle \f{e^2}{r_{12}}\bigg\rangle 
	&= \f{1}{\pi^2}\lp \f{2}{a_0} \rp^6 \int d\phi_1 d\theta_1 dr_1    d\phi_2 d\theta_2 dr_2 \, r_1^2\sin\theta_1  r_2^2\sin\theta_2 \, \f{1}{\abs{\vec{r}_1 - \vec{r}_2}}\exp\lb \f{-4 (r_1 + r_2)}{a_0} \rb.
	\end{align*}
	This is rather complicated, so we make the following simplification. Since we only care about the difference $\vec{r}_1 - \vec{r}_2$, we may take $\vec{r}_1$ to be the $\hat z$-axis for the $\vec{r}_2$ integration. As we integrate through $\vec{r}_1$, this $\hat{z}$-axis will change, but the variables $\theta_1,\phi_1$ will simply integrate out. The point is that with this logic we may take
	\begin{align*}
	\abs{\vec{r}_1 - \vec{r}_2} = \sqrt{r_1^2 + r_2^2 - 2r_1r_2\cos(\theta_2)} 
	\end{align*}
	rather than a complicated formula for the distance between two points in spherical coordinates (which involves all angles $\theta_1,\theta_2,\phi_1,\phi_2$ -- I won't show the full formula here, but it can be derived by going back to Cartesian coordinates). In any case, the integral can be rewritten and evaluated in Mathematica (or by hand if the problem-solver is feeling adventurous):
	\begin{align*}
	\bigg\langle \f{e^2}{r_{12}}\bigg\rangle 
	= \f{1}{\pi^2}\lp \f{2}{a_0} \rp^6  \int_{\mathbb{R}^3} d\phi_1 d\theta_1 dr_1 r_1^2\sin\theta_1  \int_{\mathbb{R}^3}   \, 
	\f{d\phi_2 d\theta_2 dr_2 r_2^2\sin\theta_2}{\sqrt{r_1^2 + r_2^2 - 2r_1r_2\cos(\theta_2)}}\exp\lb \f{-4 (r_1 + r_2)}{a_0} \rb
	= \boxed{\f{5e^2}{4a_0}}
	\end{align*}  
	
	What is this energy in eV? Working in SI units, the energy is given by 
	\begin{align*}
	\bigg\langle \f{e^2}{r_{12}}\bigg\rangle = \f{1}{4\pi \epsilon_0} \lp \f{5e^2}{4a_0} \rp \approx 34.014 \text{ eV}. 
	\end{align*}
	The ground state energy is therefore 
	\begin{align*}
	E_g = -108.848 \text{ eV} + 34.014 \text{ eV} = \boxed{-74.831 \text{ eV}}
	\end{align*}
	which is much closer to the measured value of $-79.006 \text{ eV}$.
	
	
	
	
	
	
	
	
	
	Mathematica code:
	\begin{lstlisting}
	(*define \psi_total^2*)
	In[1]:= f = (1/Sqrt[Pi])*(2/a0)^(3/2)*
	Exp[-2 r1/a0]*(1/Sqrt[Pi])*(2/a0)^(3/2)*Exp[-2 r2/a0]
	
	Out[1]= (8 E^(-((2 r1)/a0) - (2 r2)/a0))/(a0^3 \[Pi])
	
	(*Evaluate integral*)
	In[2]:= e^2*
	Integrate[
	f^2*1/Sqrt[r1^2 + r2^2 - 2 r1*r2*Cos[t2]]*r2^2*Sin[t2]*r1^2*
	Sin[t1], {r1, 0, Infinity}, {r2, 0, Infinity}, {t1, 0, Pi}, {t2, 0,
	Pi}, {p1, 0, 2*Pi}, {p2, 0, 2*Pi}]
	
	Out[2]= ConditionalExpression[(5 e^2)/(4 a0), Re[a0] > 0]
	\end{lstlisting}
	
	
	
	
	
	\item Let us now use variational method to obtain the ground state energy for helium, assuming only the interaction energy $\langle e^2/r_{12}\rangle$ is present. To this end, let us take the trial wavefunction:
	\begin{align*}
	\psi(\vec{r}_1, \vec{r}_2) = \phi(\vec{r}_1)\phi{\vec{r}_2} = \lb \f{1}{\sqrt{\pi} } \lp \f{Z'}{a_0}\rp^{3/2} e^{-Z'r_1/a_0} \rb \lb \f{1}{\sqrt{\pi} } \lp \f{Z'}{a_0}\rp^{3/2} e^{-Z'r_2/a_0} \rb = \f{1}{\pi}\lp \f{Z'}{a_0} \rp^3 e^{-Z'(r_1 + r_2)/a_0}.
	\end{align*}
	The original Hamiltonian in SI units is 
	\begin{align*}
	\ham = -\f{\hbar^2}{2m}\lp \nabla^2_1 + \nabla_2^2  \rp - \f{e^2}{4\pi \epsilon_0}\lp \f{2}{r_1} + \f{2}{r_2}\rp  +  \f{e^2}{4\pi \epsilon_0}\lp \f{1}{\abs{\vec{r}_1 - \vec{r}_2}}\rp 
	\end{align*}
	Now since we are introducing the variational parameter $Z'$, we may rewrite the Hamiltonian to include $Z'$ as 
	\begin{align*}
	\ham = \underbrace{-\f{\hbar^2}{2m}\lp \nabla^2_1 + \nabla_2^2  \rp - \f{e^2}{4\pi \epsilon_0}\lp \f{Z'}{r_1} + \f{Z'}{r_2}\rp}_{\ham_0}  +  \f{e^2}{4\pi \epsilon_0}\lp \f{Z'-2}{r_1} + \f{Z'-2}{r_2} + \f{1}{\abs{\vec{r}_1 - \vec{r}_2}}\rp.
	\end{align*}
	which has been intentionally put in the form where the trial wavefunction is an eigenstate of $\ham_0$, which we may refer to as the ``unperturbed'' Hamiltonian. With this, we may evaluate the expectation value of the energy, in Gaussian units\footnote{The conversion to Gaussian units is fairly easy: we simply rescale the Hamiltonian so that energy is counted in units of $e^2/a_0$ -- I won't show the details here.}:
	\begin{align*}
	\langle \ham \rangle 
	&= \langle \ham_0 \rangle + e^2 \bigg\langle \f{Z'-2}{r_1} + \f{Z'-2}{r_2} + \f{1}{\abs{\vec{r}_1 - \vec{r}_2}} \bigg\rangle \\
	&= 2Z'^2 \lp \f{-e^2}{2a_0} \rp + e^2 \lb \f{2Z'(Z'-2)}{a_0} + \f{5Z'}{8a_0}  \rb
	\end{align*} 
	where we have evaluated the terms $\langle 1/r_1\rangle = \langle 1/r_2\rangle$ in Mathematica. The interaction term is also evaluated in Mathematica in a similar way as in the preceeding part. 
	
	
	\textbf{Interpretation:} The free parameter $Z'$ can be thought of as an effective nuclear charge. By having $Z'\neq 2$ we are saying that the electrons influence each other not just through the Coulomb interaction but also through the \textit{shielding} the nucleus: One electron always sees the nucleus plus a negative cloud due to the other electron, hence it sees an effective nuclear charge of less than 2. 
	
	
	Mathematica code:
	\begin{lstlisting}
	(*evaluate integral for <1/r1> = <1/r2> terms*)
	
	In[25]:= e^2*
	Integrate[
	g^2*(Z - 2)/r1*r2^2*Sin[t2]*r1^2*Sin[t1], {r1, 0, Infinity}, {r2, 0,
	Infinity}, {t1, 0, Pi}, {t2, 0, Pi}, {p1, 0, 2*Pi}, {p2, 0, 2*Pi},
	Assumptions -> {Z/a0 > 0}]
	
	Out[25]= (e^2 (-2 + Z) Z)/a0
	\end{lstlisting}
	
	
	\item Using calculus, we find the improved ground state energy:
	\begin{align*}
	E_g = \min_{Z'} \lp -Z'^2 +2Z'(Z'-2) + \f{5 Z'}{8} \rp \f{e^2}{a_0} = -\f{729}{256}\f{e^2}{a_0} \approx \boxed{-77.489 \text{ eV}}
	\end{align*}
	which is attained at $Z' = 27/16 = 1.6875 < 2$.\\
	
	Mathematica code:
	\begin{lstlisting}
	In[27]:= h[Z_] := -Z^2 + (5 Z)/8 + 2 Z (Z - 2);
	
	In[29]:= Solve[D[h[z], z] == 0, z]
	
	Out[29]= {{z -> 27/16}}
	
	In[30]:= h[27/16]
	
	Out[30]= -(729/256)
	
	In[32]:= N[27/16]
	
	Out[32]= 1.6875
	\end{lstlisting}
	
	
%	\item \textbf{(Just for Fun) } What if we introduce correlations into the trial wavefunctions? Consider the following trial (unnormalized) wavefunction:
%	\begin{align*}
%	\psi = e^{-c_1(r_1+r_2)}(1+c_2r_{12} + c_3(r_1 - r_2)^2).
%	\end{align*}
%	Using the Hamiltonian:
%	\begin{align*}
%	\ham = -\f{\hbar^2}{2m}\lp \nabla^2_1 + \nabla_2^2  \rp - \f{e^2}{4\pi \epsilon_0}\lp \f{2}{r_1} + \f{2}{r_2}\rp  +  \f{e^2}{4\pi \epsilon_0}\lp \f{1}{\abs{\vec{r}_1 - \vec{r}_2}}\rp 
%	\end{align*}
%	Going to Gaussian units, we find 
%	\begin{align*}
%	\ham^G = -\f{e^2}{2a_0} \lp \nabla^2_1 + \nabla_2^2  \rp - \f{e^2}{a_0}\lp \f{2}{r_1} + \f{2}{r_2} + \f{1}{\abs{\vec{r}_1 - \vec{r}_2}} \rp.
%	\end{align*}
%	Now we calculate $\langle \ham \rangle$:
%	\begin{align*}
%	blah
%	\end{align*}
\end{enumerate}





\noindent \textbf{3. Energy shifts in hydrogen due to the size of the proton.}


\begin{enumerate}[label=(\alph*)]
	\item The potential $\phi(r)$ produced by a uniformly charged sphere with
	\begin{align*}
	\rho = \begin{cases}
	\rho_0 &\quad (r<a) \\ 
	0 &\quad (r>a)
	\end{cases}
	\end{align*}
	is the solution to the following boundary-value problem:
	\begin{align*}
	\begin{cases}
	\nabla^2 \phi = \rho/\epsilon_0 \\
	\phi(\infty) = 0
	\end{cases}
	\end{align*}
	Obviously we may solve this problem in two regions: inside and outside the sphere. However, it is probably easiest to do this by looking at the electric field $\vec{E}(\vec{r})$. By Gauss's law, we have
	\begin{align*}
	\vec{E}(r>a) = \f{Q}{4\pi \epsilon_0 r^2} \hat{r} = \f{\rho}{3 \epsilon_0 } \f{a^3}{r^2} \hat{r}
	\end{align*}
	and 
	\begin{align*}
	\vec{E}(r<a) = \f{Q_\text{encl}}{4\pi \epsilon_0 r^2} \hat{r} = \f{\rho}{ 3\epsilon_0} \f{r^3}{r^2} \hat{r} = \f{\rho}{3\epsilon_0} r \hat{r}.
	\end{align*}
	From here we have calculate the electric potential:
	\begin{align*}
	\phi(r>a) = -\int_\infty^r \f{\rho}{3 \epsilon_0 } \f{a^3}{r^2} \,dr = \f{\rho}{3\epsilon_0} \f{a^3}{r}  
	\end{align*}
	\begin{align*}
	\phi(r<a) = -\int_\infty^a \f{\rho}{3 \epsilon_0 } \f{a^3}{r^2} \,dr -\int_a^r \f{\rho}{3\epsilon_0} r\,dr = \f{\rho}{3\epsilon_0} \lp a^2 - \f{1}{2}r^2 + \f{1}{2}a^2 \rp = \f{\rho a^2}{6\epsilon_0} \lp 3 - \f{r^2}{a^2} \rp. 
	\end{align*}
	For convenience, let us rewrite the potentials in terms of the total charge on the sphere $Q = 4\pi \rho a^3/3$:
	\begin{align*}
	&\phi(r>a) = \f{kQ}{r} \\
	&\phi(r<a) = \f{kQ}{2a}\lp 3 - \f{r^2}{a^2} \rp  
	\end{align*}
	
	
	\item 
	We are interested in the Hamiltonian
	\begin{align*}
	\ham = -\f{\hbar^2}{2m} \nabla^2 + V_\text{finite}(r)
	\end{align*}
	where $V_\text{finite}(r)$ represents the total potential energy of the electron after including the effects due to the finite size of the proton. To find the energy shift due to this effect from energy due to $V(r) \sim e^2/r$, we must first find $\ham_\text{pert}$, the pertubation Hamiltonian. To this end, we may write
	\begin{align*}
	\ham = \underbrace{-\f{\hbar^2}{2m} \nabla^2 - \f{e^2}{4\pi \epsilon_0 r}}_{\ham_0} + \underbrace{\f{e^2}{4\pi \epsilon_0 r} + V_\text{finite}(r)}_{\ham_\text{pert}}.
	\end{align*}
	
	Consider the $\psi_{100}$ wavefunction for hydrogen. To first-order, the energy shift (in Gaussian units) is given by 
	\begin{align*}
	\Delta E(n=1,l=0) &= \bra{\psi_{100}} \ham_\text{pert} \ket{\psi_{100}} \\ 
	&= \int \psi_{100}^*(\vec{r}) \lb \f{e^2}{r} + e\phi(r)  \rb  \psi_{100}(\vec{r})  r^2\sin\theta \,dr d\theta d\phi  \\
	&= \cancel{\int_\text{out} \psi_{100}^*(\vec{r}) \lb \f{e^2}{r} - \f{e^2}{r}  \rb \psi_{100}(\vec{r}) r^2\sin\theta \,dr d\theta d\phi} \\
	&\quad\quad\quad\quad + \int_\text{in} \psi_{100}^*(\vec{r}) \lb \f{e^2}{r} - \f{e^2}{2r_p}\lp 3 - \f{r^2}{r_p^2} \rp  \rb \psi_{100}(\vec{r})  r^2\sin\theta \,dr d\theta d\phi \\
	&= \f{1}{\pi a_0^3}\int_0^{r_p} \int_0^{\pi}\int_0^{2\pi} e^{-2r/a_0} \lb \f{e^2}{r} - \f{e^2}{2r_p}\lp 3 - \f{r^2}{r_p^2} \rp  \rb  r^2 \sin\theta\,dr d\theta d\phi \\
	&= \f{e^2}{2a_0 r_p^3}\lb 3a_0^3 - 3a_0 r_p^2 + 2r_p^3 - 3a_0 e^{-2r_p/a_0}(a_0 + r_p)^2 \rb \\
	&\approx \boxed{\f{2e^2}{5a_0} \lp \f{r_p}{a_0} \rp^2} - \f{e^2}{3a_0}\lp \f{r_p}{a_0} \rp^3 + \mathcal{O}\lp \lp \f{r_p}{a_0}\rp ^4\rp 
	\end{align*}
	where we have used the fact that the radius of the proton $r_p$, which is on the order of femtometers, is much smaller than the Bohr radius, which is the order of an Angstrom.  We observe that the finite-size nucleus shifts the 1S  state energy \textbf{up}. 
	
	
	The numerical value for this first order correction is
	\begin{align*}
	\Delta E(n=1,l=0) = \f{4}{5} \times 13.6 \text{ eV } \times  \lp \f{0.9 \times 10^{-15} \text{ m}}{5.29 \times 10^{-11} \text{ m}} \rp^2 \approx \boxed{3.148 \times 10^{-9} \text{ eV}} 
	\end{align*}
	Converting this to frequency we find $\omega (n=1,l=0) \approx 2\pi \cdot 761 \text{ kHz}$. \\
	
	
	Mathematica code:
	\begin{lstlisting}
	(*define psi_100*)
	In[4]:= psi100 = (1/Sqrt[Pi])*(1/a0)^(3/2)*Exp[-r/a0];
	
	(*Evaluate integral*)
	In[5]:= Integrate[
	psi100*psi100*(e^2/r - (e^2/(2*rp))*(3 - r^2/rp^2))*r^2*Sin[t], {r, 
	0, rp}, {t, 0, Pi}, {p, 0, 2 Pi}] // FullSimplify
	
	Out[5]= (e^2 (3 a0^3 - 3 a0 rp^2 + 2 rp^3 - 
	3 a0 E^(-((2 rp)/a0)) (a0 + rp)^2))/(2 a0 rp^3)
	
	(*get corrections term by term by expanding the exponential*)
	In[29]:= FullSimplify[Series[Exp[-x], {x, 0, 6}]] /. {x -> 2*rp/a0}
	
	Out[29]= SeriesData[2 a0^(-1) rp, 0, {1, -1, 
	Rational[1, 2], 
	Rational[-1, 6], 
	Rational[1, 24], 
	Rational[-1, 120], 
	Rational[1, 720]}, 0, 7, 1]
	
	In[32]:= 1/(2 a0 rp^3)
	e^2 (3 a0^3 - 3 a0 rp^2 + 2 rp^3 - 
	3 a0 (1 - (2 rp)/a0 + 1/2 ((2 rp)/a0)^2 - 1/6 ((2 rp)/a0)^3 + 
	1/24 ((2 rp)/a0)^4 - 1/120 ((2 rp)/a0)^5 + 
	1/720 ((2 rp)/a0)^6) (a0 + rp)^2) // Expand
	
	Out[32]= (2 e^2 rp^2)/(5 a0^3) - (e^2 rp^3)/(3 a0^4) + (2 e^2 rp^4)/(
	15 a0^5) - (2 e^2 rp^5)/(15 a0^6)
	\end{lstlisting}
	
	
	
	\item For the 2S state, we may repeat the calculation above but for $n=2$. The wavefunction is 
	\begin{align*}
	\psi_{200}(\vec{r}) = \f{1}{4\sqrt{2 \pi} a_0^{3/2}} \lb 2 - \f{r}{a_0} \rb e^{-r/2a_0}.
	\end{align*}
	Using the same technique as above we find that
	\begin{align*}
	\Delta E(n=2,l=0) 
	&= \bra{\psi_{200}} \ham_\text{pert} \ket{\psi_{200}} \\
	&= \int_0^{r_p} \int_0^{\pi}\int_0^{2\pi} \psi^*_{200}(\vec{r}) \lb \f{e^2}{r} - \f{e^2}{2r_p}\lp 3 - \f{r^2}{r_p^2} \rp  \rb \psi_{200}(\vec{r}) r^2 \sin\theta\,drd\theta d\phi \\
	&= \frac{e^2 \left(336 a_0^7-24 a_0^5 r_p^2+4 a_0^4 r_p^3-6
		a_0^3 e^{-r_p/a_0} \left(56 a_0^4+56 a_0^3
		r_p+24 a_0^2 r_p^2+6 a_0
		r_p^3+r_p^4\right)\right)}{16 a_0^5 r_p^3}\\
	&\approx \boxed{\f{e^2}{20a_0} \lp \f{r_p}{a_0} \rp^2} - \f{e^2}{24 a_0}\lp \f{r_p}{a_0} \rp^3 + \mathcal{O}\lp \lp \f{r_p}{a_0}\rp ^4\rp 
	\end{align*}
	
	
	
	Now let us look at the full energy level diagram for the 1S-2S transition. Without the finite-size nucleus effect, the spacing between the 1S and 2S can be calculated using the Rydberg formula:
	\begin{align*}
	E_{12} = \f{hc}{\lambda_{12}} = hc R_H \lp \f{1}{1^2} - \f{1}{2^2} \rp = \f{3}{4} \times  13.6 \text{ eV}.
	\end{align*}
	We may convert $E_{12}$ to $\omega_{12} = 2\pi \cdot 2.466 \times 10^{15} \text{ Hz}$. With the finite-size nucleus, the 1S state energy is up-shifted more than 2S. Therefore, spacing between 1S and 2S is reduced. This reduction is 
	\begin{align*}
	\Delta = \Delta E(n=1,l=0) - \Delta E(n=2,l=0) = \lp \f{2}{5} - \f{1}{20} \rp \f{e^2}{a_0} \lp \f{r_p^2}{a_0} \rp^2 = \f{7e^2}{20a_0} \lp \f{r_p}{a_0} \rp^2. 
	\end{align*}
	We first note that because the 1S-2S two-photon transition in atomic hydrogen has a natural linewidth of only 1.3 Hz (this turns out to be well-known, see \cite{udem1997phase}), while the energy shift in consideration is on the order of $10^2$ kHz. As a result, it is possible to resolve $r_p$. In order to resolve $r_p$ with 0.010 fm accuracy, we must first find a relationship between the variance (or standard deviation) in the frequency and that in the measured $r_p$. Let us write the equation above as 
	\begin{align*}
	\Delta = hf_{12} = C r_p^2 \implies f_{12} = \f{C}{h} r_p^2
	\end{align*}
	where $f_{12}$ indicates the 1S-2S transition frequency including the non-zero proton radius effect and $C$ contains all other constant factors. Let $\sigma_{r}^2$ denote the variance in the measured $r_p$ and $\sigma^2_f$ be the variance in the frequency. Assuming normal distributions, we have from statistics that 
	\begin{align*}
	\sigma_f = \f{C}{h} \sqrt{2} \sigma_r^2
	\end{align*}
	Substituting all constant factors and $\sigma_r = 0.010$ fm, we find that $\boxed{\sigma_f \sim 10^2 \text{ Hz}}$. Comparing this to the center frequency of $f_{12} \sim 10^{15}$ Hz, we will require a frequency accuracy of approximately $\boxed{1 \text{ part in } 10^{13}}$ to get the desired standard deviation of $0.010$ fm in $r_p$. We note that since the 1S-2S transition is a two-photon process, the actual frequency source is typically multiplied, so the uncertainty could be defined as that at the source or that which the atoms see. In any case, the difference is typically well within one order of magnitude.  \\
	
	
	Mathematica code:
	\begin{lstlisting}
	(*define psi_200*)
	In[40]:= psi200 = (1/4)*(1/Sqrt[2*Pi])*(1/a0)^(3/2)*
	Exp[-r/(2*a0)]*(2 - r/a0);
	
	(*find energy shift*)
	In[56]:= Integrate[
	psi200^2*(e^2/r - (e^2/(2*rp))*(3 - r^2/rp^2))*r^2*Sin[t], {r, 0, 
	rp}, {t, 0, Pi}, {p, 0, 2 Pi}]
	
	Out[56]= (e^2 (336 a0^7 - 24 a0^5 rp^2 + 4 a0^4 rp^3 - 
	6 a0^3 E^(-(rp/
	a0)) (56 a0^4 + 56 a0^3 rp + 24 a0^2 rp^2 + 6 a0 rp^3 + 
	rp^4)))/(16 a0^5 rp^3)
	
	(*expand the exponential*)
	In[45]:= FullSimplify[Series[Exp[-x], {x, 0, 6}]] /. {x -> rp/a0}
	
	During evaluation of In[45]:= SeriesData::sdatv: First argument rp/a0 is not a valid variable.
	
	Out[45]= SeriesData[a0^(-1) rp, 0, {1, -1, 
	Rational[1, 2], 
	Rational[-1, 6], 
	Rational[1, 24], 
	Rational[-1, 120], 
	Rational[1, 720]}, 0, 7, 1]
	
	(*plug into full expression*)
	In[55]:= 1/(16 a0^5 rp^3)
	e^2 (336 a0^7 - 24 a0^5 rp^2 + 4 a0^4 rp^3 - 
	6 a0^3 (1 - rp/a0 + 1/2 (rp/a0)^2 - 1/6 (rp/a0)^3 + 
	1/24 (rp/a0)^4 - 1/120 (rp/a0)^5 + 1/720 (rp/a0)^6) (56 a0^4 + 
	56 a0^3 rp + 24 a0^2 rp^2 + 6 a0 rp^3 + rp^4)) // Expand
	
	Out[55]= (e^2 rp^2)/(20 a0^3) - (e^2 rp^3)/(24 a0^4) + (7 e^2 rp^4)/(
	480 a0^5) - (3 e^2 rp^5)/(320 a0^6) - (e^2 rp^7)/(1920 a0^8)
	\end{lstlisting}
\end{enumerate}


\begin{thebibliography}{100}
	\bibitem{udem1997phase} Udem, Th and Huber, A and Gross, B and Reichert, J and Prevedelli, M and Weitz, M and H{\"a}nsch, Th W, "Phase-Coherent Measurement of the Hydrogen 1 S- 2 S Transition Frequency with an Optical Frequency Interval Divider Chain," \emph{Physical review letters}, vol. 79, No. 14, pp. 2646, 1997.
	
	
\end{thebibliography}

\end{document}








