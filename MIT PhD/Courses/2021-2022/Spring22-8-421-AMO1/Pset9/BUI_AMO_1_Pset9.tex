\documentclass{article}
\usepackage{physics}
\usepackage{graphicx}
\usepackage{caption}
\usepackage{amsmath}
\usepackage{bm}
\usepackage{framed}
\usepackage{authblk}
\usepackage{empheq}
\usepackage{amsfonts}
\usepackage{esint}
\usepackage[makeroom]{cancel}
\usepackage{dsfont}
\usepackage{centernot}
\usepackage{mathtools}
\usepackage{subcaption}
\usepackage{bigints}
\usepackage{amsthm}
\theoremstyle{definition}
\newtheorem{lemma}{Lemma}
\newtheorem{defn}{Definition}[section]
\newtheorem{prop}{Proposition}[section]
\newtheorem{rmk}{Remark}[section]
\newtheorem{thm}{Theorem}[section]
\newtheorem{exmp}{Example}[section]
\newtheorem{prob}{Problem}[section]
\newtheorem{sln}{Solution}[section]
\newtheorem*{prob*}{Problem}
\newtheorem{exer}{Exercise}[section]
\newtheorem*{exer*}{Exercise}
\newtheorem*{sln*}{Solution}
\usepackage{empheq}
\usepackage{tensor}
\usepackage{xcolor}
%\definecolor{colby}{rgb}{0.0, 0.0, 0.5}
\definecolor{MIT}{RGB}{163, 31, 52}
\usepackage[pdftex]{hyperref}
%\hypersetup{colorlinks,urlcolor=colby}
\hypersetup{colorlinks,linkcolor={MIT},citecolor={MIT},urlcolor={MIT}}  
\usepackage[left=1in,right=1in,top=1in,bottom=1in]{geometry}
\setcounter{MaxMatrixCols}{20}
\usepackage{newpxtext,newpxmath}
\newcommand*\widefbox[1]{\fbox{\hspace{2em}#1\hspace{2em}}}

\newcommand{\p}{\partial}
\newcommand{\R}{\mathbb{R}}
\newcommand{\C}{\mathbb{C}}
\newcommand{\lag}{\mathcal{L}}
\newcommand{\nn}{\nonumber}
\newcommand{\ham}{\mathcal{H}}
\newcommand{\M}{\mathcal{M}}
\newcommand{\I}{\mathcal{I}}
\newcommand{\K}{\mathcal{K}}
\newcommand{\F}{\mathcal{F}}
\newcommand{\w}{\omega}
\newcommand{\lam}{\lambda}
\newcommand{\al}{\alpha}
\newcommand{\be}{\beta}
\newcommand{\x}{\xi}

\newcommand{\G}{\mathcal{G}}

\newcommand{\f}[2]{\frac{#1}{#2}}

\newcommand{\ift}{\infty}

\newcommand{\lp}{\left(}
\newcommand{\rp}{\right)}

\newcommand{\lb}{\left[}
\newcommand{\rb}{\right]}

\newcommand{\lc}{\left\{}
\newcommand{\rc}{\right\}}


\newcommand{\V}{\mathbf{V}}
\newcommand{\U}{\mathcal{U}}
\newcommand{\Id}{\mathcal{I}}
\newcommand{\D}{\mathcal{D}}
\newcommand{\Z}{\mathcal{Z}}

%\setcounter{chapter}{-1}


\usepackage{enumitem}



\usepackage{listings}
\captionsetup[lstlisting]{margin=0cm,format=hang,font=small,format=plain,labelfont={bf,up},textfont={it}}
\renewcommand*{\lstlistingname}{Code \textcolor{violet}{\textsl{Mathematica}}}
\definecolor{gris245}{RGB}{245,245,245}
\definecolor{olive}{RGB}{50,140,50}
\definecolor{brun}{RGB}{175,100,80}

%\hypersetup{colorlinks,urlcolor=colby}
\lstset{
	tabsize=4,
	frame=single,
	language=mathematica,
	basicstyle=\scriptsize\ttfamily,
	keywordstyle=\color{black},
	backgroundcolor=\color{gris245},
	commentstyle=\color{gray},
	showstringspaces=false,
	emph={
		r1,
		r2,
		epsilon,epsilon_,
		Newton,Newton_
	},emphstyle={\color{olive}},
	emph={[2]
		L,
		CouleurCourbe,
		PotentielEffectif,
		IdCourbe,
		Courbe
	},emphstyle={[2]\color{blue}},
	emph={[3]r,r_,n,n_},emphstyle={[3]\color{magenta}}
}

\newcommand{\diag}{\text{diag}}
\newcommand{\psirot}{\ket{\psi_\text{rot}(t)} }
\newcommand{\RWA}{\ham_\text{rot}^\text{RWA}}

% 3j symbol
\newcommand{\tj}[6]{ \begin{pmatrix}
		#1 & #2 & #3 \\
		#4 & #5 & #6 
\end{pmatrix}}


\begin{document}
\begin{framed}
\noindent Name: \textbf{Huan Q. Bui}\\
Course: \textbf{8.421 - AMO I}\\
Problem set: \textbf{\#9}\\
Due: Friday, April 15, 2022.
\end{framed}



\textbf{1. Transition Lifetimes and Blackbody Radiation. }

\begin{enumerate}[label=(\alph*)]
	\item 
	
	\begin{enumerate}[label=(\roman*)]
		\item The rate of \textit{absorption} is given by the product of Einstein's $B$ coefficient and the average number of photons per mode $\langle n \rangle_{\omega_0}$ where $\omega_0$ is angular frequency associated with the (dominant) transition with $\lambda_0 = 590$ nm. 
		\begin{align*}
		W = B\langle n \rangle = 1/60 \text{ s}^{-1}.
		\end{align*} 
		From lecture, we know that the Einstein's $B$ coefficient can be written in terms of the (known) Einstein's $A = \Gamma_0 = 1/\tau$ coefficient:
		\begin{align*}
		B = \f{\pi^2 c^3}{\hbar \omega_0^3}A = \f{\pi^2 c^3}{\hbar \omega_0^3}\f{1}{\tau}
		\end{align*}
		With this, we can plug in the numbers to find 
		\begin{align*}
		\langle n \rangle = \f{W}{B} = \f{W \hbar \omega_0^3 \tau}{\pi^2 c^3} \approx 1.0324 \times 10^{-15}.
		\end{align*}
		
		
		
		\item For blackbody radiation, we have
		\begin{align*}
		\langle n \rangle = \f{1}{e^{\hbar \omega_0 / k_BT} - 1} \implies T\approx 707.2 \text{ K}.
		\end{align*}
		We see that in order for the absorption rate to reach 1 photon per minute, the blackbody temperature has to be $\sim 700$ K, which is way above room temperature (of course the higher the temperature, the higher the absorption rate, and vice versa). As a result, there is no need to shield the vacuum system from room temperature radiation or col the vacuum system to cryogenic temperatures (as expected).
	\end{enumerate}
	
	\item Here we estimate the lifetime of hydrogen in the $F=1$ hyperfine level of the $1S$ state. 
	
	\begin{enumerate}[label=(\roman*)]
		\item $F=1 \to F=0$ is a magnetic dipole transition. 
		
		\item To estimate the lifetime of the $F=1$ state, we may assume that the (magnetic dipole) transition matrix element is $\mu_B$. From here, we work entirely in the CGS unit system to find 
		\begin{align*}
		\Gamma_0 = \f{4\omega_0^3 \mu_B^2}{3\hbar c^3} \approx 2.91 \times 10^{-15} \text{ s}^{-1}
		\end{align*}
		where the numerical values for the fundamental constants can be found on Wikipedia. The lifetime is 
		\begin{align*}
		\tau = \f{1}{\Gamma_0} \approx 1.1 \times 10^7 \text{ years}.
		\end{align*}
		
		
	\end{enumerate}
	
	\item Now we look at a hydrogen BEC in the $F=1$ state. 
	
	\begin{enumerate}[label=(\roman*)]
		\item To find the average number of photons per mode from blackbody radiation at the 21 cm line at $300$ K and $4$ K, we simply calculate
		\begin{align*}
		\langle n \rangle = \f{1}{e^{\hbar \omega/k_B T} - 1}
		\end{align*}
		at the corresponding temperatures and angular frequency. The answers are
		\begin{align*}
		&T = 300 \text{ K}, \quad\quad \langle n \rangle \approx 4375 \\
		&T = 4 \text{ K}, \quad\quad \langle n \rangle \approx 57.84.
		\end{align*}
		
		\item Similar to what we did before (but reversed), we can find the transition rates at $T=300$ K and $T = 4$ K. We will also need the lifetime $\tau \approx 1.1 \times 10^7 $ years for this calculation.
		\begin{align*}
		&T = 300 \text{ K}, \quad\quad W = 1.484\times 10^{11} \text{ s}^{-1}\\
		&T = 4 \text{ K}, \quad\quad W = 1.962 \times 10^9 \text{ s}^{-1}.
		\end{align*}
		\item It is clear that we should be concerned about blackbody radiation from the environment limiting our experiment with hydrogen in the $F=1$ state if we need a trapping times on the order of seconds/minutes. 

	\end{enumerate}
	
	\item The lifetime is much longer for the case of a magnetic dipole transition in hydrogen where there are more photons per mode mainly because the lifetime $\tau$ scales as $1/\omega_0^3$. The ratio between the sodium wavelength of 590 nm versus the 21 cm of hydrogen is $\sim 10^{-6}$. This gives a reduction factor of $10^{-18}$. On top of this, there is also another factor of $\al^2$ reduction when replacing the electric with magnetic dipole matrix element. 

\end{enumerate}


\textbf{2. Saturation Intensity.}

\begin{enumerate}[label=(\alph*)]
	\item 
	
	\item 
	
\end{enumerate}


\textbf{3. Saturation of Atomic Transitions.}

\begin{enumerate}[label=(\alph*)]
	\item 
	
	\item 
	
	\item 
	
	\item 
	
	\item 
	
	\item 
	
	\item 
\end{enumerate}

	
	
\end{document}








