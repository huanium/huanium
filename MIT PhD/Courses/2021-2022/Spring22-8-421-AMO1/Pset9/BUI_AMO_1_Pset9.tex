\documentclass{article}
\usepackage{physics}
\usepackage{graphicx}
\usepackage{caption}
\usepackage{amsmath}
\usepackage{bm}
\usepackage{framed}
\usepackage{authblk}
\usepackage{empheq}
\usepackage{amsfonts}
\usepackage{esint}
\usepackage[makeroom]{cancel}
\usepackage{dsfont}
\usepackage{centernot}
\usepackage{mathtools}
\usepackage{subcaption}
\usepackage{bigints}
\usepackage{amsthm}
\theoremstyle{definition}
\newtheorem{lemma}{Lemma}
\newtheorem{defn}{Definition}[section]
\newtheorem{prop}{Proposition}[section]
\newtheorem{rmk}{Remark}[section]
\newtheorem{thm}{Theorem}[section]
\newtheorem{exmp}{Example}[section]
\newtheorem{prob}{Problem}[section]
\newtheorem{sln}{Solution}[section]
\newtheorem*{prob*}{Problem}
\newtheorem{exer}{Exercise}[section]
\newtheorem*{exer*}{Exercise}
\newtheorem*{sln*}{Solution}
\usepackage{empheq}
\usepackage{tensor}
\usepackage{xcolor}
%\definecolor{colby}{rgb}{0.0, 0.0, 0.5}
\definecolor{MIT}{RGB}{163, 31, 52}
\usepackage[pdftex]{hyperref}
%\hypersetup{colorlinks,urlcolor=colby}
\hypersetup{colorlinks,linkcolor={MIT},citecolor={MIT},urlcolor={MIT}}  
\usepackage[left=1in,right=1in,top=1in,bottom=1in]{geometry}
\setcounter{MaxMatrixCols}{20}
\usepackage{newpxtext,newpxmath}
\newcommand*\widefbox[1]{\fbox{\hspace{2em}#1\hspace{2em}}}

\newcommand{\p}{\partial}
\newcommand{\R}{\mathbb{R}}
\newcommand{\C}{\mathbb{C}}
\newcommand{\lag}{\mathcal{L}}
\newcommand{\nn}{\nonumber}
\newcommand{\ham}{\mathcal{H}}
\newcommand{\M}{\mathcal{M}}
\newcommand{\I}{\mathcal{I}}
\newcommand{\K}{\mathcal{K}}
\newcommand{\F}{\mathcal{F}}
\newcommand{\w}{\omega}
\newcommand{\lam}{\lambda}
\newcommand{\al}{\alpha}
\newcommand{\be}{\beta}
\newcommand{\x}{\xi}

\newcommand{\G}{\mathcal{G}}

\newcommand{\f}[2]{\frac{#1}{#2}}

\newcommand{\ift}{\infty}

\newcommand{\lp}{\left(}
\newcommand{\rp}{\right)}

\newcommand{\lb}{\left[}
\newcommand{\rb}{\right]}

\newcommand{\lc}{\left\{}
\newcommand{\rc}{\right\}}


\newcommand{\V}{\mathbf{V}}
\newcommand{\U}{\mathcal{U}}
\newcommand{\Id}{\mathcal{I}}
\newcommand{\D}{\mathcal{D}}
\newcommand{\Z}{\mathcal{Z}}

%\setcounter{chapter}{-1}


\usepackage{enumitem}



\usepackage{listings}
\captionsetup[lstlisting]{margin=0cm,format=hang,font=small,format=plain,labelfont={bf,up},textfont={it}}
\renewcommand*{\lstlistingname}{Code \textcolor{violet}{\textsl{Mathematica}}}
\definecolor{gris245}{RGB}{245,245,245}
\definecolor{olive}{RGB}{50,140,50}
\definecolor{brun}{RGB}{175,100,80}

%\hypersetup{colorlinks,urlcolor=colby}
\lstset{
	tabsize=4,
	frame=single,
	language=mathematica,
	basicstyle=\scriptsize\ttfamily,
	keywordstyle=\color{black},
	backgroundcolor=\color{gris245},
	commentstyle=\color{gray},
	showstringspaces=false,
	emph={
		r1,
		r2,
		epsilon,epsilon_,
		Newton,Newton_
	},emphstyle={\color{olive}},
	emph={[2]
		L,
		CouleurCourbe,
		PotentielEffectif,
		IdCourbe,
		Courbe
	},emphstyle={[2]\color{blue}},
	emph={[3]r,r_,n,n_},emphstyle={[3]\color{magenta}}
}

\newcommand{\diag}{\text{diag}}
\newcommand{\psirot}{\ket{\psi_\text{rot}(t)} }
\newcommand{\RWA}{\ham_\text{rot}^\text{RWA}}

% 3j symbol
\newcommand{\tj}[6]{ \begin{pmatrix}
		#1 & #2 & #3 \\
		#4 & #5 & #6 
\end{pmatrix}}


\begin{document}
\begin{framed}
\noindent Name: \textbf{Huan Q. Bui}\\
Course: \textbf{8.421 - AMO I}\\
Problem set: \textbf{\#9}\\
Due: Friday, April 15, 2022.
\end{framed}



\textbf{1. Transition Lifetimes and Blackbody Radiation. }

\begin{enumerate}[label=(\alph*)]
	\item 
	
	\begin{enumerate}[label=(\roman*)]
		\item The rate of \textit{absorption} is given by the product of Einstein's $B$ coefficient and the average number of photons per mode $\langle n \rangle_{\omega_0}$ where $\omega_0$ is angular frequency associated with the (dominant) transition with $\lambda_0 = 590$ nm. 
		\begin{align*}
		W = B\langle n \rangle = 1/60 \text{ s}^{-1}.
		\end{align*} 
		From lecture, we know that the Einstein's $B$ coefficient can be written in terms of the (known) Einstein's $A = \Gamma_0 = 1/\tau$ coefficient:
		\begin{align*}
		B = \f{\pi^2 c^3}{\hbar \omega_0^3}A = \f{\pi^2 c^3}{\hbar \omega_0^3}\f{1}{\tau}
		\end{align*}
		With this, we can plug in the numbers to find 
		\begin{align*}
		\langle n \rangle = \f{W}{B} = \f{W \hbar \omega_0^3 \tau}{\pi^2 c^3} \approx 1.0324 \times 10^{-15}.
		\end{align*}
		
		
		
		\item For blackbody radiation, we have
		\begin{align*}
		\langle n \rangle = \f{1}{e^{\hbar \omega_0 / k_BT} - 1} \implies T\approx 707.2 \text{ K}.
		\end{align*}
		We see that in order for the absorption rate to reach 1 photon per minute, the blackbody temperature has to be $\sim 700$ K, which is way above room temperature (of course the higher the temperature, the higher the absorption rate, and vice versa). As a result, there is no need to shield the vacuum system from room temperature radiation or col the vacuum system to cryogenic temperatures (as expected).
	\end{enumerate}
	
	\item Here we estimate the lifetime of hydrogen in the $F=1$ hyperfine level of the $1S$ state. 
	
	\begin{enumerate}[label=(\roman*)]
		\item $F=1 \to F=0$ is a magnetic dipole transition. 
		
		\item To estimate the lifetime of the $F=1$ state, we may assume that the (magnetic dipole) transition matrix element is $\mu_B$. From here, we work entirely in the CGS unit system to find 
		\begin{align*}
		\Gamma_0 = \f{4\omega_0^3 \mu_B^2}{3\hbar c^3} \approx 2.91 \times 10^{-15} \text{ s}^{-1}
		\end{align*}
		where the numerical values for the fundamental constants can be found on Wikipedia. The lifetime is 
		\begin{align*}
		\tau = \f{1}{\Gamma_0} \approx 1.1 \times 10^7 \text{ years}.
		\end{align*}
		
		
	\end{enumerate}
	
	\item Now we look at a hydrogen BEC in the $F=1$ state. 
	
	\begin{enumerate}[label=(\roman*)]
		\item To find the average number of photons per mode from blackbody radiation at the 21 cm line at $300$ K and $4$ K, we simply calculate
		\begin{align*}
		\langle n \rangle = \f{1}{e^{\hbar \omega/k_B T} - 1}
		\end{align*}
		at the corresponding temperatures and angular frequency. The answers are
		\begin{align*}
		&T = 300 \text{ K}, \quad\quad \langle n \rangle \approx 4375 \\
		&T = 4 \text{ K}, \quad\quad \langle n \rangle \approx 57.84.
		\end{align*}
		
		\item Similar to what we did before (but reversed), we can find the transition rates at $T=300$ K and $T = 4$ K. We will also need the lifetime $\tau \approx 1.1 \times 10^7 $ years for this calculation.
		\begin{align*}
		&T = 300 \text{ K}, \quad\quad W = 1.484\times 10^{11} \text{ s}^{-1}\\
		&T = 4 \text{ K}, \quad\quad W = 1.962 \times 10^9 \text{ s}^{-1}.
		\end{align*}
		\item It is clear that we should be concerned about blackbody radiation from the environment limiting our experiment with hydrogen in the $F=1$ state if we need a trapping times on the order of seconds/minutes. 

	\end{enumerate}
	
	\item The lifetime is much longer for the case of a magnetic dipole transition in hydrogen where there are more photons per mode mainly because the lifetime $\tau$ scales as $1/\omega_0^3$. The ratio between the sodium wavelength of 590 nm versus the 21 cm of hydrogen is $\sim 10^{-6}$. This gives a reduction factor of $10^{-18}$. On top of this, there is also another factor of $\al^2$ reduction when replacing the electric with magnetic dipole matrix element. 

\end{enumerate}


\textbf{2. Saturation Intensity.}

\begin{enumerate}[label=(\alph*)]
	\item We first manipulate the Einstein $A$ coefficient so that it is written in terms of the oscillator strength $f$, the fine structure constant $\al$ and the transition frequency $\omega$. The oscillator strength is given by 
	\begin{align*}
	f_{21} = \f{2m\omega_{21}}{3\hbar} \f{1}{2J_1 + 1} \sum_{m_1,m_2} |\langle J_1 m_1 | \vec{r}| J_2 m_2 \rangle|^2 = \f{2m\omega_{21}}{3\hbar} \f{S_{12}}{2J_1+1}.
	\end{align*}
	where $S_{12}$ is the line strength, while 
	\begin{align*}
	A_{12} = \f{4\omega^3 e^2 }{3\hbar c^3} \sum_{m_2} |\langle 1 m_1| \vec{r} | 2 m_2 \rangle|^2 = \f{4\omega^3 e^2 }{3\hbar c^3} \f{S_{12}}{2J_1+1}.
	\end{align*}
	From here we have that
	\begin{align*}
	A_{12} = \f{4e^2 \omega_{21}^3}{3\hbar c^3} \f{3\hbar f_{21}}{2m\omega_{21}} = \f{2e^2 \omega_{21}^2f_{21}}{mc^3} =\f{2 \hbar \omega_{21}^2f_{21}}{mc^2} \f{e^2}{\hbar c} = \f{2\al \hbar \omega_{21}^2 f_{21}}{mc^2}.
	\end{align*}
	Assuming $f_{21} = 1$ we may use this formula to estimate the lifetime of sodium. Plugging in the numerical values for the constants above (in CGS units), we find 
	\begin{align*}
	A \approx 1.917 \times 10^8 \text{ s}^{-1} \implies \tau = \f{1}{A} \approx 5.2 \text{ ns}.
	\end{align*}
	
	
	More precisely, if we call $f$ the \textit{absorption} oscillator strength, then we actually have
	\begin{align*}
	A = \f{2\al \hbar \omega^2}{mc^2} \f{2J_1+1}{2J_2+1} f.
	\end{align*}
	Assuming that we're working the D lines of sodium, we will take $J_1 = 1/2$ and $J_2 = 1/2$ and $3/2$. From Steck's, we find that $f = 0.64$ for the $3^2S_{1/2} \to 3^2P_{3/2}$ transition and $f=0.32$ for the $3^2 S_{1/2} \to 3^2 P_{1/2}$ transition. Plugging in the numbers we find that
	\begin{align*}
	\tau_{1/2 \to 3/2} \approx \tau_{1/2\to 3/2} \approx 16 \text{ ns}, 
	\end{align*} 
	as expected.
	
	\item The saturation intensity for the principal transition in sodium, with $\sigma_0 = 3\lambda^2/2\pi$, is:
	\begin{align*}
	I_{\text{sat}} = \f{\hbar \omega A}{2\sigma_0}  \approx 38.8 \text{ mW/cm}^2.
	\end{align*}
	where we are using $A$ from the previous part and ignoring fine and hyperfine structure. 
	
\end{enumerate}


\textbf{3. Saturation of Atomic Transitions.}

\begin{enumerate}[label=(\alph*)]
	\item Here we consider a two-state atom with $R_{ge} = R_{eg} = R$ the stimulated absorption/emission rate and $A = \Gamma$ the spontaneous emission rate. Define the saturation parameter $s$ as $s = 2R/\Gamma$. In equilibrium, we have
	\begin{align*}
	\Gamma N_b + R N_b   - R N_a  = 0 \implies \f{N_b}{N_a} = \f{R}{R+\Gamma} = \f{s\Gamma/2}{s\Gamma/2 + \Gamma} =  \f{s}{s+2}.
	\end{align*}
	
	\item The equilibrium spontaneous emission rate per atom $A N_b$ can be expressed in terms of $\Gamma$ and $s$ and the total density $N = N_a + N_b$. From part (a), we have that
	\begin{align*}
	\f{N_b}{N} = \f{N_b}{N_b + N_a} = \f{N_b/N_a}{N_b/N_a + 1} = \f{s}{s + (s+2)} = \f{1}{2} \f{s}{s+1}.
	\end{align*}
	So, we find 
	\begin{align*}
	\Gamma N_b = \f{N}{2} \f{\Gamma s}{s+1}.
	\end{align*}
	Recalling the optical Bloch equations, we know that
	\begin{align*}
	\f{N_b}{N} = \f{\Omega^2/\Gamma^2}{1 +2\Omega^2/\Gamma^2}. 
	\end{align*}
	Thus we may identify $s$ with $2\Omega^2/\Gamma^2$. The scattering cross section has the form 
	\begin{align*}
	\sigma = \f{\sigma_0}{1 + I/I_\text{sat}} = \f{\sigma_0}{1 + 2\Omega^2/\Gamma^2}.
	\end{align*}
	where we have ignored the (small) detuning term and used the definition of $I_\text{sat}$. We can immediately see that $\sigma(s)$ bleaches out as $\sigma(s=0)/(1+s)$, as desired. 
	
	\item For $s=1$, the energy density $\langle w\rangle_\text{SAT}$ per unit frequency is given by 
	\begin{align*}
	\langle w \rangle =   \f{R}{B}, 
	\end{align*}
	where $B$ is the Einstein $B$ coefficient. With $s=1$, we find that
	\begin{align*}
	s = 1 = \f{2R}{\Gamma} \implies R = \f{\Gamma}{2} = B \langle w \rangle \implies \langle w \rangle = \f{\Gamma}{2B}. 
	\end{align*}
	$\langle w \rangle$ is independent of the atomic dipole matrix element since both the $A=\Gamma$ and $B$ Einstein coefficients are proportional to the matrix element (modulus) squared. 
	
	
	
	
	
	\item With 
	\begin{align*}
	\f{\Gamma}{B} = \f{8\pi \hbar }{\lambda^3} = \f{\hbar \omega^3}{\pi^2 c^3},
	\end{align*}
	we find 
	\begin{align*}
	\langle w \rangle_\text{SAT} =  \f{\hbar \omega^3}{2\pi^2 c^3}.
	\end{align*} 
	The mean occupation number $n$ per photon mode for $s=1$ is given by 
	\begin{align*}
	\langle n \rangle = \f{B\langle w \rangle_\text{SAT}}{\Gamma} = \f{B}{\Gamma} \f{\Gamma}{2B} = \f{1}{2}.   
	\end{align*}
	
	
	
	\item We have laser light of intensity $I_0$ and Lorentzian lineshape centered at the atomic transition frequency $\omega_0$ with FWHM $\Gamma'\gg \Gamma$. The energy density of this beam per frequency interval at $\omega_0$ is 
	\begin{align*}
	S(\omega_0) = \f{I(\omega_0)}{c} = I_0 \f{1}{\pi c} \f{\Gamma'/2}{(\omega_0 - \omega_0)^2 + (\Gamma'/2)^2} = \f{2I_0}{\pi c\Gamma'}.
	\end{align*}
	We now want $I_s$ such that
	\begin{align*}
	s = 1 = \f{2R}{\Gamma} = \f{2B\langle w \rangle}{\Gamma} = 2S(\omega_0) \f{B}{\Gamma} = \f{4I_s}{\pi c\Gamma'} \f{\pi^2 c^3}{\hbar\omega^3} = \f{4\pi c^2}{\Gamma'\hbar \omega^3}I_s \implies I_s = \f{\Gamma' \hbar \omega^3}{4\pi c^2}.
	\end{align*}
	
	\item The stimulated broadband absorption rate is given by 
	\begin{align*}
	R = B\langle w\rangle= B \f{2I_0}{\pi c \Gamma'} = \f{2I_0}{\pi c \Gamma'} \f{\Gamma \pi^2 c^3}{\hbar \omega^3} = \f{2\pi \Gamma c^2 I_0}{\hbar \omega^3 \Gamma'} = \f{\omega_R^2}{\Gamma'},  
	\end{align*}
	as desired. Here, we have used the result seen on the last problem set:
	\begin{align*}
	\omega_R^2 = \f{2\pi \Gamma c^2 I_0}{\hbar \omega^3}.
	\end{align*}
	For $s=1$, we have $I_0 = I_s$, so
	\begin{align*}
	\omega_R^2 = \f{2\pi \Gamma c^2 }{\hbar \omega^3} \f{\Gamma' \hbar \omega^3}{4\pi c^2} = \f{\Gamma \Gamma'}{2}.
	\end{align*}
	
	\item If we set $\Gamma' = \Gamma$, then we get exactly the saturation intensity of a monochromatic laser beam and the Rabi frequency at saturation. This is not surprising since when setting $\Gamma'=\Gamma$ we are no longer in the broadband regime. 
\end{enumerate}

	
	
\end{document}








