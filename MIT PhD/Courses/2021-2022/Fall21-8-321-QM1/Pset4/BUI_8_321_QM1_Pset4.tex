\documentclass{article}
\usepackage{physics}
\usepackage{graphicx}
\usepackage{caption}
\usepackage{amsmath}
\usepackage{bm}
\usepackage{framed}
\usepackage{authblk}
\usepackage{empheq}
\usepackage{amsfonts}
\usepackage{esint}
\usepackage[makeroom]{cancel}
\usepackage{dsfont}
\usepackage{centernot}
\usepackage{mathtools}
\usepackage{bigints}
\usepackage{amsthm}
\theoremstyle{definition}
\newtheorem{lemma}{Lemma}
\newtheorem{defn}{Definition}[section]
\newtheorem{prop}{Proposition}[section]
\newtheorem{rmk}{Remark}[section]
\newtheorem{thm}{Theorem}[section]
\newtheorem{exmp}{Example}[section]
\newtheorem{prob}{Problem}[section]
\newtheorem{sln}{Solution}[section]
\newtheorem*{prob*}{Problem}
\newtheorem{exer}{Exercise}[section]
\newtheorem*{exer*}{Exercise}
\newtheorem*{sln*}{Solution}
\usepackage{empheq}
\usepackage{tensor}
\usepackage{xcolor}
%\definecolor{colby}{rgb}{0.0, 0.0, 0.5}
\definecolor{MIT}{RGB}{163, 31, 52}
\usepackage[pdftex]{hyperref}
%\hypersetup{colorlinks,urlcolor=colby}
\hypersetup{colorlinks,linkcolor={MIT},citecolor={MIT},urlcolor={MIT}}  
\usepackage[left=1in,right=1in,top=1in,bottom=1in]{geometry}

\usepackage{newpxtext,newpxmath}
\newcommand*\widefbox[1]{\fbox{\hspace{2em}#1\hspace{2em}}}

\newcommand{\p}{\partial}
\newcommand{\R}{\mathbb{R}}
\newcommand{\C}{\mathbb{C}}
\newcommand{\lag}{\mathcal{L}}
\newcommand{\nn}{\nonumber}
\newcommand{\ham}{\mathcal{H}}
\newcommand{\M}{\mathcal{M}}
\newcommand{\I}{\mathcal{I}}
\newcommand{\K}{\mathcal{K}}
\newcommand{\F}{\mathcal{F}}
\newcommand{\w}{\omega}
\newcommand{\lam}{\lambda}
\newcommand{\al}{\alpha}
\newcommand{\be}{\beta}
\newcommand{\x}{\xi}

\newcommand{\G}{\mathcal{G}}

\newcommand{\f}[2]{\frac{#1}{#2}}

\newcommand{\ift}{\infty}

\newcommand{\lp}{\left(}
\newcommand{\rp}{\right)}

\newcommand{\lb}{\left[}
\newcommand{\rb}{\right]}

\newcommand{\lc}{\left\{}
\newcommand{\rc}{\right\}}


\newcommand{\V}{\mathbf{V}}
\newcommand{\U}{\mathcal{U}}
\newcommand{\Id}{\mathcal{I}}
\newcommand{\D}{\mathcal{D}}
\newcommand{\Z}{\mathcal{Z}}

%\setcounter{chapter}{-1}


\usepackage{enumitem}



\usepackage{subfig}
\usepackage{listings}
\captionsetup[lstlisting]{margin=0cm,format=hang,font=small,format=plain,labelfont={bf,up},textfont={it}}
\renewcommand*{\lstlistingname}{Code \textcolor{violet}{\textsl{Mathematica}}}
\definecolor{gris245}{RGB}{245,245,245}
\definecolor{olive}{RGB}{50,140,50}
\definecolor{brun}{RGB}{175,100,80}

%\hypersetup{colorlinks,urlcolor=colby}
\lstset{
	tabsize=4,
	frame=single,
	language=mathematica,
	basicstyle=\scriptsize\ttfamily,
	keywordstyle=\color{black},
	backgroundcolor=\color{gris245},
	commentstyle=\color{gray},
	showstringspaces=false,
	emph={
		r1,
		r2,
		epsilon,epsilon_,
		Newton,Newton_
	},emphstyle={\color{olive}},
	emph={[2]
		L,
		CouleurCourbe,
		PotentielEffectif,
		IdCourbe,
		Courbe
	},emphstyle={[2]\color{blue}},
	emph={[3]r,r_,n,n_},emphstyle={[3]\color{magenta}}
}






\begin{document}
\begin{framed}
\noindent Name: \textbf{Huan Q. Bui}\\
Course: \textbf{8.321 - Quantum Theory I}\\
Problem set: \textbf{\#4}
\end{framed}
	




\noindent \textbf{1. Particle in a box.} It is well known that the solution to the SE
\begin{align*}
-\f{\hbar^2}{2m}\f{\p^2}{\p x^2}\psi(x) + V(x)\psi(x) = E\psi(x)
\end{align*}
where 
\begin{align*}
V(x) = \begin{cases}
0, \quad 0 < x < a\\
\infty, \quad \text{else}
\end{cases}
\end{align*}
is normalized standing waves:
\begin{align*}
\psi_n(x) = \sqrt{\f{2}{a}}\sin(k_n x) = \sqrt{\f{2}{a}}\sin\lp \f{\pi n x}{a} \rp
\end{align*}
where $k_n$ are the wavenumbers and $E_n = n^2\pi^2 \hbar^2 / 2ma^2$. One can solve this quickly by noting that $\psi = 0$ outside of $[0,a]$, and the solution inside $[0,a]$ must be a sinusoid. Because $\psi(0) = 0$, the only satisfactory solution is $\psi(x) \sim \sin(k_n x)$. To satisfy the boundary conditions $\psi(0) = \psi(a) = 0$, the wavenumbers $k_n = n\pi/a$. Plugging this back into the SE, we find that $E_n = n^2\pi^2\hbar^2/2ma^2$. Finally, we integrate $\int_0^a \sin^2(k_n x)\,dx$ to find the normalization factor $N = \sqrt{2/a}$, so that $\psi_n(x) = \sqrt{2/a}\sin(k_n x)$. \\

We now wish to calculate the uncertainty product for the ground state and first excited states. For the ground state, the wavefunction is $\psi_1 = \sqrt{2/a}\sin(\pi x/a)$. 
\begin{align*}
&\langle x  \rangle_1 = \int_0^a x \lb \sqrt{\f{2}{a}}\sin\lp \f{\pi x }{a} \rp\rb^2\,dx = \f{a}{2} \quad \text{(as expected by symmetry)}\\
&\langle x^2  \rangle_1 = \int_0^a x^2 \lb \sqrt{\f{2}{a}}\sin\lp \f{\pi x }{a} \rp\rb^2\,dx = a^2\lp \f{1}{3} - \f{1}{2\pi^2}\rp.
\end{align*}
So,
\begin{align*}
\langle (\Delta x)^2 \rangle_1 = \langle x^2 \rangle_1 - \langle x \rangle^2_1 =  \f{a^2}{12} - \f{a^2}{2\pi^2}.
\end{align*}
Next, we find the moments of the momentum by using $\hat p  = -i\hbar \p_x$:
\begin{align*}
&\langle p  \rangle_1 = \int_0^a \sqrt{\f{2}{a}}\sin\lp \f{\pi x }{a} \rp
\lb -i\hbar \p_x \rb\sqrt{\f{2}{a}}\sin\lp \f{\pi x }{a} \rp\,dx = 0 \quad \text{(as expected by symmetry)}\\
&\langle p^2  \rangle_1 = \int_0^a \sqrt{\f{2}{a}}\sin\lp \f{\pi x }{a} \rp
\lb -i\hbar \rb^2 \p^2_x \sqrt{\f{2}{a}}\sin\lp \f{\pi x }{a} \rp\,dx = \f{\hbar^2 \pi^2}{a}.
\end{align*}
So,
\begin{align*}
\langle (\Delta p)^2 \rangle_1 = \langle p^2 \rangle_1 - \langle p \rangle^2_1 =  \f{\hbar^2 \pi^2}{2}.
\end{align*}
With these, we find 
\begin{align*}
\boxed{\langle (\Delta x)^2 \rangle_1\langle (\Delta p)^2 \rangle_1 = \f{\pi^2-6}{12} \hbar^2 > \f{\hbar^2}{4}}
\end{align*}

Next, we do the same for the first excited state, whose wavefunction is $\psi_2(x)= \sqrt{2/a}\sin(2\pi x/a)$.
\begin{align*}
&\langle x  \rangle_2 = \int_0^a x \lb \sqrt{\f{2}{a}}\sin\lp \f{2\pi x }{a} \rp\rb^2\,dx = \f{a}{2} \quad \text{(as expected by symmetry)}\\
&\langle x^2  \rangle_2 = \int_0^a x^2 \lb \sqrt{\f{2}{a}}\sin\lp \f{2\pi x }{a} \rp\rb^2\,dx = a^2\lp \f{1}{3} - \f{1}{8\pi^2}\rp.
\end{align*}
So,
\begin{align*}
\langle (\Delta x)^2 \rangle_2 = \langle x^2 \rangle_2 - \langle x \rangle^2_2 =  \f{a^2}{12} - \f{a^2}{8\pi^2}.
\end{align*}
Next, we find the moments of the momentum by using $\hat p  = -i\hbar \p_x$:
\begin{align*}
&\langle p  \rangle_2 = \int_0^a \sqrt{\f{2}{a}}\sin\lp \f{2\pi x }{a} \rp
\lb -i\hbar \p_x \rb\sqrt{\f{2}{a}}\sin\lp \f{2\pi x }{a} \rp\,dx = 0 \quad \text{(as expected by symmetry)}\\
&\langle p^2  \rangle_2 = \int_0^a \sqrt{\f{2}{a}}\sin\lp \f{2\pi x }{a} \rp
\lb -i\hbar \rb^2 \p^2_x \sqrt{\f{2}{a}}\sin\lp \f{2\pi x }{a} \rp\,dx = \f{4\hbar^2 \pi^2}{a}.
\end{align*}
So,
\begin{align*}
\langle (\Delta p)^2 \rangle_2 = \langle p^2 \rangle_2 - \langle p \rangle^2_2 =  \f{4\hbar^2 \pi^2}{2}.
\end{align*}
With these, we find 
\begin{align*}
\boxed{\langle (\Delta x)^2 \rangle_2 \langle (\Delta p)^2 \rangle_2 = \f{2\pi^2-3}{6} \hbar^2 > \f{\hbar^2}{4}}
\end{align*}




Mathematica code:
\begin{lstlisting}
(* GROUND STATE *)

In[36]:= x11 = 
Integrate[x*(Sqrt[2/a]*Sin[(Pi/a)*x])^2, {x, 0, a}] // FullSimplify

Out[36]= a/2

In[37]:= x12 = 
Integrate[x^2*(Sqrt[2/a]*Sin[(Pi/a)*x])^2, {x, 0, a}] // Expand

Out[37]= a^2/3 - a^2/(2 \[Pi]^2)

In[40]:= x12 - x11^2

Out[40]= a^2/12 - a^2/(2 \[Pi]^2)

In[41]:= p11 = 
Integrate[(Sqrt[2/a]*Sin[(Pi/a)*x])*(I*h)*
D[(Sqrt[2/a]*Sin[(Pi/a)*x]), x], {x, 0, a}] // FullSimplify

Out[41]= 0

In[42]:= p12 = 
Integrate[(Sqrt[2/a]*Sin[(Pi/a)*x])*(I*h)^2*
D[D[(Sqrt[2/a]*Sin[(Pi/a)*x]), x], x], {x, 0, a}] // FullSimplify

Out[42]= (h^2 \[Pi]^2)/a^2

In[44]:= (x12 - x11^2)*(p12 - p11^2) // FullSimplify

Out[44]= 1/12 h^2 (-6 + \[Pi]^2)



(* FIRST EXCITED STATE *)
In[46]:= x21 = 
Integrate[x*(Sqrt[2/a]*Sin[(2*Pi/a)*x])^2, {x, 0, a}] // FullSimplify

Out[46]= a/2

In[47]:= x22 = 
Integrate[x^2*(Sqrt[2/a]*Sin[(2*Pi/a)*x])^2, {x, 0, a}] // Expand

Out[47]= a^2/3 - a^2/(8 \[Pi]^2)

In[48]:= x22 - x21^2

Out[48]= a^2/12 - a^2/(8 \[Pi]^2)

In[49]:= p21 = 
Integrate[(Sqrt[2/a]*Sin[(2*Pi/a)*x])*(I*h)*
D[(Sqrt[2/a]*Sin[(2*Pi/a)*x]), x], {x, 0, a}] // FullSimplify

Out[49]= 0

In[50]:= p22 = 
Integrate[(Sqrt[2/a]*Sin[(2*Pi/a)*x])*(I*h)^2*
D[D[(Sqrt[2/a]*Sin[(2*Pi/a)*x]), x], x], {x, 0, a}] // FullSimplify

Out[50]= (4 h^2 \[Pi]^2)/a^2

In[52]:= (x22 - x21^2)*(p22 - p21^2) // FullSimplify

Out[52]= 1/6 h^2 (-3 + 2 \[Pi]^2)
\end{lstlisting}









\noindent \textbf{2. Balancing an Ice Pick.} Let us model the ice pick of length $L$ and mass $m$ as an inverted pendulum initially in the classical equilibrium position $\theta(0) = 0, \dot\theta(0) = 0$ (the angle $\theta$ is measured from the vertical). Quantum mechanically, we have
\begin{align*}
\Delta x \Delta p \geq \f{\hbar}{2} \implies L \Delta \theta(0) mL \Delta\dot\theta(0) \geq \f{\hbar^2}{2}.   
\end{align*}
Let us assume for simplicity that the Heisenberg uncertainty is saturated, so that 
\begin{align*}
\Delta \theta(0) \Delta \dot \theta(0) = \f{\hbar}{2mL^2}.
\end{align*}
To proceed, let us assume that the uncertainty in the initial conditions is also on the same order as the values of the initial conditions themselves, i.e., $\Delta \theta(0) \sim \theta(0)$ and $\Delta \dot \theta(0) \sim \dot\theta(0)$ so that
\begin{align*}
\theta(0)\dot\theta(0) = \f{\hbar^2}{2mL^2}.
\end{align*}
Now, by Newton's second law of motion and the small angle approximation, we have a simple equation of motion
\begin{align*}
\ddot\theta = \f{g}{L}\theta \implies \theta(t) = A\exp(\omega t) + B\exp(-\omega t)
\end{align*}
where $\omega = \sqrt{g/L}$. We're interested in the exponentiall growing solution, so we'll set $B=0$. Plugging in the initial conditions, we find that $\theta(t) = \theta(0)\exp(t\sqrt{g/L})$ and therefore we may set $\dot \theta(0) = \theta(0)\sqrt{g/L}$. With the Heisenberg uncertainty conditions, we find that
\begin{align*}
\theta(0) \sim \sqrt{\f{\sqrt{L/g}\hbar^2}{2mL^2}} \quad\quad\quad \dot\theta(0) = \sqrt{\f{\hbar^2}{2mL^2 \sqrt{L/g}}}.
\end{align*}
With these the solution is 
\begin{align*}
\theta(t) = \sqrt{\f{\sqrt{L/g}\hbar^2}{2mL^2}} \exp \lp \sqrt{\f{g}{L}}t \rp.
\end{align*}
Inverting this and solve for $t$ we find 
\begin{align*}
t(\theta) = \sqrt{\f{L}{g}}\ln \lc \theta \sqrt{\f{2mL^2}{\hbar^2 \sqrt{L/g}}} \rc.
\end{align*}
We will now assume the dimension and weight of the ice pick to be $L = 0.1$m, $m = 0.1$kg. Assume also that the angle at which we define the ice pick as ``falling'' to be $\theta \sim 1$. Putting these numbers into the formula above we find that the length of time that this ice pick can be balanced on a point is
\begin{align*}
t \approx 3.91\text{s} \approx 4 \text{s}
\end{align*}
So, the answer is \textbf{``in a few seconds.''}\\


\noindent \textbf{3. }

\begin{enumerate}[label=(\alph*)]
	\item 
	\begin{align*}
	\sum_{a'}\abs{\bra{a}x\ket{a'}}^2 (E_{a'} - E_a) 
	&= \sum_{a'}\bra{a}x \ket{a'}\bra{a'}x \ket{a} (E_{a'} - E_a) \\
	&= 
	\end{align*}
	
	\item 
	
	
	\item 
\end{enumerate}

\noindent \textbf{4. }


	
\end{document}








