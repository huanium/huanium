\documentclass{article}
\usepackage{physics}
\usepackage{graphicx}
\usepackage{caption}
\usepackage{amsmath}
\usepackage{bm}
\usepackage{framed}
\usepackage{authblk}
\usepackage{empheq}
\usepackage{amsfonts}
\usepackage{esint}
\usepackage[makeroom]{cancel}
\usepackage{dsfont}
\usepackage{centernot}
\usepackage{mathtools}
\usepackage{bigints}
\usepackage{amsthm}
\theoremstyle{definition}
\newtheorem{lemma}{Lemma}
\newtheorem{defn}{Definition}[section]
\newtheorem{prop}{Proposition}[section]
\newtheorem{rmk}{Remark}[section]
\newtheorem{thm}{Theorem}[section]
\newtheorem{exmp}{Example}[section]
\newtheorem{prob}{Problem}[section]
\newtheorem{sln}{Solution}[section]
\newtheorem*{prob*}{Problem}
\newtheorem{exer}{Exercise}[section]
\newtheorem*{exer*}{Exercise}
\newtheorem*{sln*}{Solution}
\usepackage{empheq}
\usepackage{tensor}
\usepackage{xcolor}
%\definecolor{colby}{rgb}{0.0, 0.0, 0.5}
\definecolor{MIT}{RGB}{163, 31, 52}
\usepackage[pdftex]{hyperref}
%\hypersetup{colorlinks,urlcolor=colby}
\hypersetup{colorlinks,linkcolor={MIT},citecolor={MIT},urlcolor={MIT}}  
\usepackage[left=1in,right=1in,top=1in,bottom=1in]{geometry}

\usepackage{newpxtext,newpxmath}
\newcommand*\widefbox[1]{\fbox{\hspace{2em}#1\hspace{2em}}}

\newcommand{\p}{\partial}
\newcommand{\R}{\mathbb{R}}
\newcommand{\C}{\mathbb{C}}
\newcommand{\lag}{\mathcal{L}}
\newcommand{\nn}{\nonumber}
\newcommand{\ham}{\mathcal{H}}
\newcommand{\M}{\mathcal{M}}
\newcommand{\I}{\mathcal{I}}
\newcommand{\K}{\mathcal{K}}
\newcommand{\F}{\mathcal{F}}
\newcommand{\w}{\omega}
\newcommand{\lam}{\lambda}
\newcommand{\al}{\alpha}
\newcommand{\be}{\beta}
\newcommand{\x}{\xi}

\newcommand{\G}{\mathcal{G}}

\newcommand{\f}[2]{\frac{#1}{#2}}

\newcommand{\ift}{\infty}

\newcommand{\lp}{\left(}
\newcommand{\rp}{\right)}

\newcommand{\lb}{\left[}
\newcommand{\rb}{\right]}

\newcommand{\lc}{\left\{}
\newcommand{\rc}{\right\}}


\newcommand{\V}{\mathbf{V}}
\newcommand{\U}{\mathcal{U}}
\newcommand{\Id}{\mathcal{I}}
\newcommand{\D}{\mathcal{D}}
\newcommand{\Z}{\mathcal{Z}}

%\setcounter{chapter}{-1}


\usepackage{enumitem}



\usepackage{subfig}
\usepackage{listings}
\captionsetup[lstlisting]{margin=0cm,format=hang,font=small,format=plain,labelfont={bf,up},textfont={it}}
\renewcommand*{\lstlistingname}{Code \textcolor{violet}{\textsl{Mathematica}}}
\definecolor{gris245}{RGB}{245,245,245}
\definecolor{olive}{RGB}{50,140,50}
\definecolor{brun}{RGB}{175,100,80}

%\hypersetup{colorlinks,urlcolor=colby}
\lstset{
	tabsize=4,
	frame=single,
	language=mathematica,
	basicstyle=\scriptsize\ttfamily,
	keywordstyle=\color{black},
	backgroundcolor=\color{gris245},
	commentstyle=\color{gray},
	showstringspaces=false,
	emph={
		r1,
		r2,
		epsilon,epsilon_,
		Newton,Newton_
	},emphstyle={\color{olive}},
	emph={[2]
		L,
		CouleurCourbe,
		PotentielEffectif,
		IdCourbe,
		Courbe
	},emphstyle={[2]\color{blue}},
	emph={[3]r,r_,n,n_},emphstyle={[3]\color{magenta}}
}






\begin{document}
\begin{framed}
\noindent Name: \textbf{Huan Q. Bui}\\
Course: \textbf{8.321 - Quantum Theory I}\\
Problem set: \textbf{\#3}
\end{framed}
	




\noindent \textbf{1. }
\begin{equation*}
A = \begin{pmatrix}
1 & 0 & 1\\
0 & 0 & 0\\
1 & 0 & 1
\end{pmatrix} 
\quad\quad \quad 
B = \begin{pmatrix}
2 & 1 & 1 \\
1 & 0 & -1\\
1 & -1 & 2
\end{pmatrix}.
\end{equation*}

\begin{enumerate}[label=(\alph*)]
	\item To show that $AB$ commute, we simply compute their commutator:
	\begin{align*}
	[A,B] = \begin{pmatrix}
	1 & 0 & 1\\
	0 & 0 & 0\\
	1 & 0 & 1
	\end{pmatrix} 
	\begin{pmatrix}
	2 & 1 & 1 \\
	1 & 0 & -1\\
	1 & -1 & 2
	\end{pmatrix}
	- 
	\begin{pmatrix}
	2 & 1 & 1 \\
	1 & 0 & -1\\
	1 & -1 & 2
	\end{pmatrix}
	\begin{pmatrix}
	1 & 0 & 1\\
	0 & 0 & 0\\
	1 & 0 & 1
	\end{pmatrix} 
	= 
	\begin{pmatrix}
	3 & 0 & 3\\
	0 & 0 & 0\\
	3 & 0 & 3
	\end{pmatrix}
	-
	\begin{pmatrix}
	3 & 0 & 3\\
	0 & 0 & 0\\
	3 & 0 & 3
	\end{pmatrix} = 
	\begin{pmatrix}
	0 & 0 & 0\\
	0 & 0 & 0\\
	0 & 0 & 0
	\end{pmatrix}.   
	\end{align*}
	So, $A$ and $B$ commute.
	
	
	\item Notice that $\rank(A) = 1$. So $A$ must have eigenvalue of zero with multiplicity of two. The other eigenvalue is $2$ by inspection, where the corresponding eigenvector is $(1,0,1)^\top$. The other two $0$-eigenvectors must span the subspace orthogonal to $(1,0,1)^\top$. We may choose $(0,1,0)^\top$ and $(-1,0,1)^\top$.\\
	
	To find the eigenvalues of $B$ we may use the traditional method of characteristic polynomials. 
	\begin{align*}
	0 = \det(B-\lambda \mathbb{I}) = -6-\lambda +4 \lambda^2 - \lambda^3 \implies 0 = (\lambda-3)(\lambda-2)(\lambda+1).
	\end{align*}
	The corresponding eigenvectors are
	\begin{align*}
	&\begin{pmatrix}
	2 & 1 & 1 \\
	1 & 0 & -1\\
	1 & -1 & 2
	\end{pmatrix}\vec{x}_1 = 3\vec{x}_1 \implies \vec{x}_1 = \begin{pmatrix}
	1 \\ 0 \\ 1
	\end{pmatrix}\\
	&\begin{pmatrix}
	2 & 1 & 1 \\
	1 & 0 & -1\\
	1 & -1 & 2
	\end{pmatrix}\vec{x}_2 = 2\vec{x}_2 \implies \vec{x}_2 = \begin{pmatrix}
	-1 \\ -1 \\ 1
	\end{pmatrix}\\
	&\begin{pmatrix}
	2 & 1 & 1 \\
	1 & 0 & -1\\
	1 & -1 & 2
	\end{pmatrix}\vec{x}_3 = -1\vec{x}_3 \implies \vec{x}_3 = \begin{pmatrix}
	-1 \\ 2 \\ 1
	\end{pmatrix}
	\end{align*}
	
	
	\item It is clear that $(1,0,1)^\top$ is a simultaneous eigenvector of $A$ and $B$. Also notice that the eigenvectors $\vec{x}_2$ and $\vec{x}_3$ of $B$ are orthogonal to each other and to $(1,0,1)^\top$. This means $\vec{x}_2$ and $\vec{x}_3$ span the subspace associated with the eigenvalue zero for $A$. Thus, $\vec{x}_2$, $\vec{x}_3$ are eigenvectors of $A$ and it suffices to normalize $\vec{x}_1, \vec{x}_2, \vec{x}_3$ to form a unitary matrix:
	\begin{align*}
	\boxed{U = \begin{pmatrix}
	1/\sqrt{2}  & -1/\sqrt{3} & -1/\sqrt{6}  \\
	0 & -1/\sqrt{3} & 2/\sqrt{6}   \\
	1 /\sqrt{2} & 1/\sqrt{3} & 1/\sqrt{6} 	
	\end{pmatrix}}
	\end{align*} 
	Simultaneous diagonalization of $A$ and $B$:
	\begin{align*}
	&U^\dagger A U = \begin{pmatrix}
	2 & 0 & 0 \\
	0&0&0\\
	0&0&0
	\end{pmatrix}\\
	&U^\dagger B U = \begin{pmatrix}
	3 & 0 & 0 \\
	0 & 2 & 0 \\ 
	0 & 0 & -1
	\end{pmatrix}
	\end{align*}
	as desired.
\end{enumerate}



\noindent \textbf{2. } $N$ spin-$1/2$ particles in 
\begin{align*}
\ham = \ham_2^{(1)} \otimes \ham_2^{(2)} \otimes \dots \otimes \ham_2^{(n)}.
\end{align*}
where each $\ham_2^{(i)}$ is two-dimensional.


\begin{enumerate}[label=(\alph*)]
	\item The dimension of $\ham$ is $2^n$.
	
	\item $S_z = S_z^{(1)} + S_z^{(2)} + \dots + S_z^{(n)}$. There are ${n\choose i}$ product (eigen)states with $i$ particles in $\ket{\uparrow}$ and $(n-i)$ particles in $\ket{\downarrow}$. For the product state with $i$ particles in $\ket{\uparrow}$, the corresponding eigenvalue is 
	\begin{align*}
	\lambda = \f{\hbar}{2}i - \f{\hbar}{2}(n-i) = \f{\hbar}{2}\lp 2i-n \rp,\quad\quad i = 0,1,2,\dots,n
	\end{align*}
	So, the spectrum of $S_z$ is 
	\begin{align*}
	\sigma(S_z) = \lc \f{n\hbar}{2}, \f{(n-2)\hbar}{2},\dots, \f{-(n-2)\hbar}{2}, \f{-n\hbar}{2}   \rc
	\end{align*}
	There are $n+1$ distinct eigenvalues. The multiplicity of each $\lambda_i$ is ${n\choose{i}}$ where $\lambda_i$ is the eigenvalue associated with the product state with $i$ spins in $\ket{\uparrow}$. \\
	
	As a sanity check, the sum of the multiplicities must be $2^n$. This is the case here due to a well-known combinatorial relation:
	\begin{align*}
	\sum_{i=0}^n {n\choose{i}} = (1+1)^n = 2^n.
	\end{align*}
	
	\item $I = \mathbf{S}^{(1)}\cdot \mathbf{S}^{(2)}+ \mathbf{S}^{(2)}\cdot \mathbf{S}^{(3)} + \dots + \mathbf{S}^{(N-1)}\cdot \mathbf{S}^{(N)}+ \mathbf{S}^{(N)}\cdot \mathbf{S}^{(1)}$. We claim that $[I,S_z] = 0$ and shall prove this by induction.\\
	
	
	 
	
	
	\item 
\end{enumerate}





\noindent \textbf{3. }




	
\end{document}








