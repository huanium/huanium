\documentclass{article}
\usepackage{physics}
\usepackage{graphicx}
\usepackage{caption}
\usepackage{amsmath}
\usepackage{bm}
\usepackage{framed}
\usepackage{authblk}
\usepackage{empheq}
\usepackage{amsfonts}
\usepackage{esint}
\usepackage[makeroom]{cancel}
\usepackage{dsfont}
\usepackage{centernot}
\usepackage{mathtools}
\usepackage{bigints}
\usepackage{amsthm}
\theoremstyle{definition}
\newtheorem{lemma}{Lemma}
\newtheorem{defn}{Definition}[section]
\newtheorem{prop}{Proposition}[section]
\newtheorem{rmk}{Remark}[section]
\newtheorem{thm}{Theorem}[section]
\newtheorem{exmp}{Example}[section]
\newtheorem{prob}{Problem}[section]
\newtheorem{sln}{Solution}[section]
\newtheorem*{prob*}{Problem}
\newtheorem{exer}{Exercise}[section]
\newtheorem*{exer*}{Exercise}
\newtheorem*{sln*}{Solution}
\usepackage{empheq}
\usepackage{tensor}
\usepackage{xcolor}
%\definecolor{colby}{rgb}{0.0, 0.0, 0.5}
\definecolor{MIT}{RGB}{163, 31, 52}
\usepackage[pdftex]{hyperref}
%\hypersetup{colorlinks,urlcolor=colby}
\hypersetup{colorlinks,linkcolor={MIT},citecolor={MIT},urlcolor={MIT}}  
\usepackage[left=1in,right=1in,top=1in,bottom=1in]{geometry}

\usepackage{newpxtext,newpxmath}
\newcommand*\widefbox[1]{\fbox{\hspace{2em}#1\hspace{2em}}}

\newcommand{\p}{\partial}
\newcommand{\R}{\mathbb{R}}
\newcommand{\C}{\mathbb{C}}
\newcommand{\lag}{\mathcal{L}}
\newcommand{\nn}{\nonumber}
\newcommand{\ham}{\mathcal{H}}
\newcommand{\M}{\mathcal{M}}
\newcommand{\I}{\mathcal{I}}
\newcommand{\K}{\mathcal{K}}
\newcommand{\F}{\mathcal{F}}
\newcommand{\w}{\omega}
\newcommand{\lam}{\lambda}
\newcommand{\al}{\alpha}
\newcommand{\be}{\beta}
\newcommand{\x}{\xi}

\newcommand{\G}{\mathcal{G}}

\newcommand{\f}[2]{\frac{#1}{#2}}

\newcommand{\ift}{\infty}

\newcommand{\lp}{\left(}
\newcommand{\rp}{\right)}

\newcommand{\lb}{\left[}
\newcommand{\rb}{\right]}

\newcommand{\lc}{\left\{}
\newcommand{\rc}{\right\}}


\newcommand{\V}{\mathbf{V}}
\newcommand{\U}{\mathcal{U}}
\newcommand{\Id}{\mathcal{I}}
\newcommand{\D}{\mathcal{D}}
\newcommand{\Z}{\mathcal{Z}}

%\setcounter{chapter}{-1}


\usepackage{enumitem}



\usepackage{subfig}
\usepackage{listings}
\captionsetup[lstlisting]{margin=0cm,format=hang,font=small,format=plain,labelfont={bf,up},textfont={it}}
\renewcommand*{\lstlistingname}{Code \textcolor{violet}{\textsl{Mathematica}}}
\definecolor{gris245}{RGB}{245,245,245}
\definecolor{olive}{RGB}{50,140,50}
\definecolor{brun}{RGB}{175,100,80}

%\hypersetup{colorlinks,urlcolor=colby}
\lstset{
	tabsize=4,
	frame=single,
	language=mathematica,
	basicstyle=\scriptsize\ttfamily,
	keywordstyle=\color{black},
	backgroundcolor=\color{gris245},
	commentstyle=\color{gray},
	showstringspaces=false,
	emph={
		r1,
		r2,
		epsilon,epsilon_,
		Newton,Newton_
	},emphstyle={\color{olive}},
	emph={[2]
		L,
		CouleurCourbe,
		PotentielEffectif,
		IdCourbe,
		Courbe
	},emphstyle={[2]\color{blue}},
	emph={[3]r,r_,n,n_},emphstyle={[3]\color{magenta}}
}






\begin{document}
\begin{framed}
\noindent Name: \textbf{Huan Q. Bui}\\
Course: \textbf{8.321 - Quantum Theory I}\\
Problem set: \textbf{\#5}
\end{framed}
	


\noindent \textbf{1. Coherent states}

\begin{enumerate}[label=(\alph*)]
	\item 
	\begin{align*}
	\ket{\phi} = e^{\phi a^\dagger }\ket{0} = \sum_{n=0}^\infty \f{\phi^n (a^\dagger)^n}{n!} \ket{0} 
	= \sum^\infty_{n=0} \f{\phi^n}{n!} \sqrt{n!} \ket{n} 
	= \sum^\infty_{n=0} \f{\phi^n}{\sqrt{n!}}\ket{n}.
	\end{align*}
	
	
	\item 
	\begin{align*}
	a\ket{\phi} = \sum^\infty_{n=0} \f{\phi^n}{\sqrt{n!}}a \ket{n} = \sum^\infty_{n=0} \f{\phi^n}{\sqrt{n!}}\sqrt{n} \ket{n-1} = \phi \sum^\infty_{n-1=0} \f{\phi^{n-1}}{\sqrt{(n-1)!}}\ket{n-1} = \phi \ket{\phi}. 
	\end{align*}
	
	\item 
	\begin{align*}
	\bra{\phi}\ket{\phi'} = \sum^\infty_{m=0} \f{(\phi^*)^m}{\sqrt{m!}} \sum^\infty_{n=0} \f{\phi'^n}{\sqrt{n!}}\bra{m}\ket{n} = \sum^\infty_{n=0} \f{(\phi^*\phi')^n}{n!} = e^{\phi^* \phi'}.
	\end{align*}
	
	\item 
	\begin{align*}
	\bra{\phi} : A(a^\dagger, a) : \ket{\phi'}
	=  \sum^{\infty}_{m=0}\sum^\infty_{n=0} C(m,n) \bra{\phi} (a^\dagger)^m a^n \ket{\phi'} 
	=  \sum^{\infty}_{m=0}\sum^\infty_{n=0} C(m,n) (\phi^*)^m \phi'^n \bra{\phi}\ket{\phi'} 
	=  e^{\phi^* \phi'} A(\phi^*, \phi')
	\end{align*}
	
	
	
	\item 
	\begin{align*}
	\f{1}{2\pi i}\int d\phi^* d\phi e^{-\phi^*\phi}\ketbra{\phi} = 
	\f{1}{2\pi i} \sum^\infty_{n,m} \f{\ket{m}\bra{n}}{\sqrt{m!n!}} 
	\int d\phi^* d\phi  (\phi^*)^n \phi^m e^{-\phi^*\phi}
	\end{align*}
	In polar coordinates, $\phi =re^{i\theta}$, and $\int d\phi^* d\phi = 2i\int r\,drd\theta$. With this, 
	\begin{align*}
	\f{1}{2\pi i}\sum^\infty_{n,m} \f{\ket{m}\bra{n}}{\sqrt{m!n!}} \int d\phi^* d\phi e^{-\phi^*\phi}\ketbra{\phi} 
	&= \f{2i}{2\pi i}\sum^\infty_{n,m} \f{\ket{m}\bra{n}}{\sqrt{m!n!}} \int^{2\pi}_0 \,d\theta e^{i(m-n)\theta} \int_0^\infty \,dr r^{m+n+1} e^{-r^2}\\
	&= \f{2i}{2\pi i}\sum^\infty_{n,m} \f{\ket{m}\bra{n}}{\sqrt{m!n!}} 2\pi \delta_{mn} \f{1}{2}\Gamma\lp  \f{2+m+n}{2}\rp\\
	&= \f{2i}{2i} \sum^\infty_{n=0} \f{\ket{n}\bra{n}}{n!} \Gamma(n+1)\\
	&= \f{2i}{2i} \sum^\infty_{n=0} \f{\ket{n}\bra{n}}{n!} n!\\
	&= \mathbb{I}.
	\end{align*}
\end{enumerate}



\noindent \textbf{2. Squeezed states}

\begin{enumerate}[label=(\alph*)]
	\item When $\be = 0$ we have
	\begin{align*}
	\braket{\al,0,\gamma} 
	&= e^{\al^* \al} \bra{0} \lp e^{\gamma(a^\dagger)^2}\rp^\dagger e^{\gamma(a^\dagger)^2} \ket{0} \\
	&= e^{\al^* \al}\bra{0} e^{\gamma^* a^2 } e^{\gamma (a^\dagger)^2} \ket{0}
	\end{align*} 
	Let's calculate $e^{\gamma (a^\dagger)^2}\ket{0}$:
	\begin{align*}
	e^{\gamma (a^\dagger)^2}\ket{0} 
	&= \sum^\infty_{n=0}\f{\gamma^n (a^\dagger)^n (a^\dagger)^n}{n!}\ket{0} \\
	&= \sum^\infty_{n=0}\f{\gamma^n}{\sqrt{n!}} (a^\dagger)^n \ket{n} \\
	&= \sum^\infty_{n=0}\f{\gamma^n}{\sqrt{n!}} \sqrt{\f{(2n)!}{n!}} \ket{2n} \\
	& \sum^\infty_{n=0} \f{\gamma^n}{n!} \sqrt{(2n)!} \ket{2n}.
	\end{align*}
	With this, 
	\begin{align*}
	\braket{\al,0,\be} = e^{\al^*\al}\sum^\infty_{n,m} \f{\gamma^n\gamma^m}{n!m!}\sqrt{(2n)!(2m)!}\delta_{mn} = e^{\al^*\al}\sum^\infty_{n=0} \f{\gamma^{2n}}{(n!)^2} (2n)!
	\end{align*}
	In order for this norm to converge, the series satisfies the ratio test:
	\begin{align*}
	1 > e^{\abs{\al}^2}\lim_{n\to \infty} \f{\gamma^{2(n+1)} (2(n+1))!/((n+1)!)^2}{\gamma^{2n} (2n)!/(n!)^2} = \lim_{n\to \infty}  e^{\abs{\al}^2}\gamma^2 \f{(2n+1)(2n+2)}{(n+1)(n+1)} = 4 e^{\abs{\al}^2}\gamma^2 \implies  \boxed{e^{\abs{\al}^2}\gamma^2 < 1/4}
	\end{align*}
	\textcolor{blue}{Extend this result for $\be \neq 0$?}
	
	\item We claim that 
	\begin{align*}
	\boxed{\ket{x'} = \lp \f{m\omega}{\pi \hbar} \rp^{1/4} \exp\lp -\f{m\omega}{2\hbar} x'^2\rp\exp\lp \sqrt{\f{2m\omega}{\hbar}} x' {a}^\dagger - \f{1}{2}({a}^\dagger)^2 \rp \ket{0}}
	\end{align*}
	from which we read off the coefficients:
	\begin{align*}
	\gamma = -\f{1}{2}, \quad\quad \be = \sqrt{\f{2m\omega}{\hbar}} x',\quad\quad \al = -\f{m\omega}{2\hbar} x'^2 + \f{1}{4}\ln\lp \f{m\omega}{\pi \hbar} \rp.
	\end{align*}
	Now we prove that the boxed equation is true. To this end, we check that the normalization is correct and that the the equation $\hat{x}\ket{x'} = x'\ket{x'}$ is satisfied. 
	\begin{align*}
	\hat{x}\ket{x'} &= \sqrt{\f{\hbar}{2m\omega}} ( a + {a}^\dagger) \ket{x'} \\
	&=  \lp \f{m\omega}{\pi \hbar} \rp^{1/4} \exp\lp -\f{m\omega}{2\hbar} x'^2\rp\sqrt{\f{\hbar}{2m\omega}} ( a + {a}^\dagger)
	\exp\lp \sqrt{\f{2m\omega}{\hbar}} x' {a}^\dagger - \f{1}{2}({a}^\dagger)^2 \rp \ket{0}\\
	&=  \lp \f{m\omega}{\pi \hbar} \rp^{1/4} \exp\lp -\f{m\omega}{2\hbar} x'^2\rp\sqrt{\f{\hbar}{2m\omega}} ( a + {a}^\dagger)
	 \exp\lp - \f{1}{2}({a}^\dagger)^2 \rp \exp\lp \sqrt{\f{2m\omega}{\hbar}} x' {a}^\dagger\rp\ket{0}
	\end{align*}
	since things commute. This is rather complicated to deal with. However, we may insert the identity operator $I$ defined by 
	\begin{align*}
	I = 
	\exp(-\f{1}{2}(a^\dagger)^2) 
	\exp\lp \sqrt{\f{2m\omega}{\hbar}} x' {a}^\dagger  \rp 
	\exp\lp -\sqrt{\f{2m\omega}{\hbar}} x' {a}^\dagger  \rp 
	\exp\lp \f{1}{2}(a^\dagger)^2 \rp
	\end{align*}
	to the left and observe that
	\begin{align*}
	\exp\lp \f{1}{2}(a^\dagger)^2 \rp (a+a^\dagger)\exp\lp -\f{1}{2}(a^\dagger)^2 \rp 
	&= \exp\lp \f{1}{2}(a^\dagger)^2 \rp a\exp\lp -\f{1}{2}(a^\dagger)^2 \rp + a^\dagger \\
	&= a + \f{1}{2}[a^\dagger a^\dagger,a] + a^\dagger\\
	&= a + \f{1}{2}(a^\dagger[a^\dagger,a] + [a^\dagger,a]a^\dagger) + a^\dagger\\
	&= a - a^\dagger + a^\dagger \\
	&= a,
	\end{align*}
	where we have used the identity for $e^A B e^{-A}$ from Pset 1 and the fact that $a^\dagger$ commute with itself. Next, we find (using the same identity) 
	\begin{align*}
	\exp\lp -\sqrt{\f{2m\omega}{\hbar}} x' {a}^\dagger  \rp  a \exp\lp \sqrt{\f{2m\omega}{\hbar}} x' {a}^\dagger  \rp
	&= a -\sqrt{\f{2m\omega}{\hbar}}x'[a^\dagger,a]\\
	&= a + \sqrt{\f{2m\omega}{\hbar}}  x'.
	\end{align*}
	Since $a\ket{0} = 0$, we have
	\begin{align*}
	\hat{x}\ket{x'} &= \lp \f{m\omega}{\pi \hbar} \rp^{1/4} \exp\lp -\f{m\omega}{2\hbar} x'^2\rp \cancel{\sqrt{\f{\hbar}{2m\omega}}} \exp(-\f{1}{2}(a^\dagger)^2) 
	\exp\lp \sqrt{\f{2m\omega}{\hbar}} x' {a}^\dagger  \rp \cancel{\sqrt{\f{2m\omega}{\hbar}}}  x' \ket{0}\\
	&= x' \lc  \lp \f{m\omega}{\pi \hbar} \rp^{1/4} \exp\lp -\f{m\omega}{2\hbar} x'^2\rp\exp\lp \sqrt{\f{2m\omega}{\hbar}} x' {a}^\dagger - \f{1}{2}({a}^\dagger)^2 \rp \ket{0}\rc\\
	&= x'\ket{x'} \quad\quad\checkmark
	\end{align*}
	The normalization is obtained by finding $\bra{0}\ket{x'}$. Suppose that it is $N$, then 
	\begin{align*}
	\bra{0}\ket{x'} = N\bra{0}\exp\lp \sqrt{\f{2m\omega}{\hbar}} x a^\dagger - \f{1}{2}(a^\dagger)^2 \rp \ket{0} = N\braket{0} = N \implies  N = \psi_0(x) =  \lp \f{m\omega}{\pi \hbar} \rp^{1/4} \exp\lp -\f{m\omega}{2\hbar} x'^2\rp.
	\end{align*}
	With this we're done. \\
	
	To see if $\braket{x'}$ is bounded or not, we may look at $e^{\abs{\al}^2} \gamma^2$ for this case. Notice that $e^{\abs{\al}^2} \geq 1$ for all $\al$, and so the norm is finite only if $\gamma^2 < 1/4$. However, in this case we have $\gamma = -1/2 \implies \gamma^2 = 1/4$. We therefore conclude that $\braket{x'}$ is infinite, as expected. 
	
	
\end{enumerate}

\noindent \textbf{3. Low-lying states}


\begin{enumerate}[label=(\alph*)]
	\item Ground and first excited energy for particle in the potential:
	\begin{align*}
	V(x) = \f{1}{4}x^4
	\end{align*}
	
	\item Ground and first excited energy for particle in the potential:
	\begin{align*}
	V(x) = -\f{1}{2}x^2 + \f{1}{24}x^4
	\end{align*}
	
	
	\item Ground state energy for particle in the potential:
	\begin{align*}
	W(x,y) = \f{1}{2}x^2 + \f{1}{2}y^2 - \sqrt{2}\abs{x-y}
	\end{align*}
	
	
	\item Ground state energy for particle in the potential:
	\begin{align*}
	V(x,y) = \f{1}{4}x^4 + \f{1}{6}y^6 + 2xy
	\end{align*}
\end{enumerate}

	
\end{document}








