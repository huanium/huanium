\documentclass{article}
\usepackage{physics}
\usepackage{graphicx}
\usepackage{caption}
\usepackage{amsmath}
\usepackage{bm}
\usepackage{framed}
\usepackage{authblk}
\usepackage{empheq}
\usepackage{amsfonts}
\usepackage{esint}
\usepackage[makeroom]{cancel}
\usepackage{dsfont}
\usepackage{centernot}
\usepackage{mathtools}
\usepackage{bigints}
\usepackage{amsthm}
\theoremstyle{definition}
\newtheorem{lemma}{Lemma}
\newtheorem{defn}{Definition}[section]
\newtheorem{prop}{Proposition}[section]
\newtheorem{rmk}{Remark}[section]
\newtheorem{thm}{Theorem}[section]
\newtheorem{exmp}{Example}[section]
\newtheorem{prob}{Problem}[section]
\newtheorem{sln}{Solution}[section]
\newtheorem*{prob*}{Problem}
\newtheorem{exer}{Exercise}[section]
\newtheorem*{exer*}{Exercise}
\newtheorem*{sln*}{Solution}
\usepackage{empheq}
\usepackage{tensor}
\usepackage{xcolor}
%\definecolor{colby}{rgb}{0.0, 0.0, 0.5}
\definecolor{MIT}{RGB}{163, 31, 52}
\usepackage[pdftex]{hyperref}
%\hypersetup{colorlinks,urlcolor=colby}
\hypersetup{colorlinks,linkcolor={MIT},citecolor={MIT},urlcolor={MIT}}  
\usepackage[left=1in,right=1in,top=1in,bottom=1in]{geometry}

\usepackage{newpxtext,newpxmath}
\newcommand*\widefbox[1]{\fbox{\hspace{2em}#1\hspace{2em}}}

\newcommand{\p}{\partial}
\newcommand{\R}{\mathbb{R}}
\newcommand{\C}{\mathbb{C}}
\newcommand{\lag}{\mathcal{L}}
\newcommand{\nn}{\nonumber}
\newcommand{\ham}{\mathcal{H}}
\newcommand{\M}{\mathcal{M}}
\newcommand{\I}{\mathcal{I}}
\newcommand{\K}{\mathcal{K}}
\newcommand{\F}{\mathcal{F}}
\newcommand{\w}{\omega}
\newcommand{\lam}{\lambda}
\newcommand{\al}{\alpha}
\newcommand{\be}{\beta}
\newcommand{\x}{\xi}

\newcommand{\G}{\mathcal{G}}

\newcommand{\f}[2]{\frac{#1}{#2}}

\newcommand{\ift}{\infty}

\newcommand{\lp}{\left(}
\newcommand{\rp}{\right)}

\newcommand{\lb}{\left[}
\newcommand{\rb}{\right]}

\newcommand{\lc}{\left\{}
\newcommand{\rc}{\right\}}


\newcommand{\V}{\mathbf{V}}
\newcommand{\U}{\mathcal{U}}
\newcommand{\Id}{\mathcal{I}}
\newcommand{\D}{\mathcal{D}}
\newcommand{\Z}{\mathcal{Z}}

%\setcounter{chapter}{-1}


\usepackage{enumitem}



\usepackage{subfig}
\usepackage{listings}
\captionsetup[lstlisting]{margin=0cm,format=hang,font=small,format=plain,labelfont={bf,up},textfont={it}}
\renewcommand*{\lstlistingname}{Code \textcolor{violet}{\textsl{Mathematica}}}
\definecolor{gris245}{RGB}{245,245,245}
\definecolor{olive}{RGB}{50,140,50}
\definecolor{brun}{RGB}{175,100,80}

%\hypersetup{colorlinks,urlcolor=colby}
\lstset{
	tabsize=4,
	frame=single,
	language=mathematica,
	basicstyle=\scriptsize\ttfamily,
	keywordstyle=\color{black},
	backgroundcolor=\color{gris245},
	commentstyle=\color{gray},
	showstringspaces=false,
	emph={
		r1,
		r2,
		epsilon,epsilon_,
		Newton,Newton_
	},emphstyle={\color{olive}},
	emph={[2]
		L,
		CouleurCourbe,
		PotentielEffectif,
		IdCourbe,
		Courbe
	},emphstyle={[2]\color{blue}},
	emph={[3]r,r_,n,n_},emphstyle={[3]\color{magenta}}
}






\begin{document}
\begin{framed}
\noindent Name: \textbf{Huan Q. Bui}\\
Course: \textbf{8.321 - Quantum Theory I}\\
Problem set: \textbf{\#2}
\end{framed}
	


\noindent \textbf{1.} Let $A$ be a skew-Hermitian operator, i.e., $A^\dagger = -A$.
\begin{enumerate}[label=(\alph*)]
	\item Let $\lambda$ and $\ket{\lambda}$ be an eigenvalue and eigenvector of $A$, respectively. Then we have
	\begin{align*}
	A\ket{\lambda} = \lambda\ket{\lambda} \implies \lambda\braket{\lambda} = \bra{\lambda}A\ket{\lambda} = -\bra{\lambda}A\ket{\lambda}  = \bra{\lambda}A^*\ket{\lambda} = \lambda^* \braket{\lambda} \implies -\lambda = \lambda^*.
	\end{align*} 
	Since $\lambda \in \mathbb{C}$, the only solution is $\lambda = 0$. Thus, the only real eigenvalue of $A$ (up to multiplicity/degeneracy) is $0$.
	
	  
	
	\item  Let $A,B$ be Hermitian operators. Then
	\begin{align*}
	[A,B] = AB - BA = A^\dagger B^\dagger - B^\dagger A^\dagger = (BA - AB)^\dagger = -(AB-BA)^\dagger = -[A,B]^\dagger.
	\end{align*}
	Thus $[A,B]$ is skew-Hermitian. 
\end{enumerate}



\noindent \textbf{2.} Let $H,K$ be Hermitian operators with non-negative eigenvalues and assume that that the trace defined throughout this problem. Since $H,K$ are Hermitian operators we may assume that there exist complete orthonormal (eigen)bases $\{ \ket{h_i}\}$ and $\ket{k_i}$ for $H,K$ respectively with $H\ket{h_i} = h_i \ket{h_i}$ and $K\ket{k_i} = k_i \ket{k_i}$, and $h_i,k_i \geq 0$ for all $i$. Then we can spectral-decompose $H,K$ in their product as follows
\begin{align*}
HK = \sum_n h_n \ketbra{h_n}\sum_m k_m \ketbra{k_m} = \sum_{n,m} h_n k_m \ket{h_n}\bra{h_n} \ket{k_m}\bra{k_m}.
\end{align*}
Since $\tr(A) = \sum_i \bra{\phi_i} A \ket{\phi_i}$ for any orthonormal basis $\{ \phi_i\}$, we have
\begin{align*}
\tr(HK) &= \sum_j \bra{h_j} \lb  \sum_{n,m} h_n k_m \ket{h_n}\bra{h_n} \ket{k_m}\bra{k_m} \rb    \ket{h_j} \\
&= \sum_{n,m} h_n k_m \bra{h_n} \ket{k_m} \bra{k_m}\ket{h_n}, \quad \text{by orthonormality}\\
&= \sum_{n,m} h_n k_m \abs{\bra{h_n} \ket{k_m}}^2.
\end{align*}
Since $h_i,k_i \geq 0$ for all $i$, and the modulus square is always nonnegative, we see that $\tr(HK) \geq 0$, as desired. \\

Suppose $\tr(HK) = 0$, then by nonnegativity we must have $h_n k_m \abs{\bra{h_n}\ket{k_m}}^2 = 0$ for all $n,m$, or equivalently $h_nk_m \bra{h_n}\ket{k_m} = 0$ for all $n,m$. In view of the first equation for $HK$, we see that $HK= 0$.  \\













\noindent \textbf{3.} Let a Hermitian operator $H$ be given with positive spectrum and a complete orthonormal basis. 

\begin{enumerate}[label=(\alph*)]
	\item We want to prove that for any two vectors $\ket{\al}, \ket{\be}$ 
	\begin{align*}
	\abs{\bra{\al} H \ket{\be}}^2 \leq \bra{\al}H \ket{\al} \bra{\be} H \ket{\be}.
	\end{align*}
	
	There are two ways to go about this proof, in which both approaches are actually the same and only differ by appearance. I will present the notationally ``light'' version first. This goes as follows: Since $H$ is Hermitian with positive spectrum, we may find a complete orthonormal basis in which $H$ is diagonal. The transformation between $H$ and its diagonalization $D$ is given by a unitary operator $U$ as $H  = U^\dagger D U$. Since $D$ is diagonal with positive entries, we can define its square root $\sqrt{D}$. From here, we can also define the square root of $H$, denoted $\sqrt{H}$ by $U^\dagger \sqrt{D} U$. We can check:
	\begin{align*}
	\sqrt{H} \sqrt{H} = U^\dagger \sqrt{D} U U^\dagger \sqrt{D} U = U^\dagger \sqrt{D}\sqrt{D} U = U^\dagger D U = H.
	\end{align*}
	It is easy to show that $\sqrt{H}$ is also Hermitian: 
	\begin{align*}
	\sqrt{H}^\dagger = \lp U^\dagger \sqrt{D} U \rp^\dagger  = U^\dagger \sqrt{D}^\dagger U = U^\dagger \sqrt{D} U = \sqrt{H}, 
	\end{align*}
	where we have used the fact that $\sqrt{D}$ is strictly diagonal and positive, thus Hermitian. The rest of the proof is now a simple application of the Cauchy-Schwarz inequality for inner products:
	\begin{align*}
	\abs{\bra{\al} H \ket{\be}}^2 &= \abs{ \bra{\al} \sqrt{H}\sqrt{H} \ket{\be} }^2 = \abs{ \bra{\al} \sqrt{H}^\dagger \sqrt{H} \ket{\be} }^2 =  \abs{ \bra{\al \sqrt{H}^\dagger}\ket{ \sqrt{H}\be} }^2 \\
	&\leq \braket{\sqrt{H}\al} \braket{\sqrt{H} \be}\\
	&= \bra{\al} \sqrt{H}^\dagger\sqrt{H}\ket{\al} \bra{\be}\sqrt{H}^\dagger\sqrt{H}\ket{ \be}= \bra{\al} {H}\ket{\al} \bra{\be}{H}\ket{ \be}
	\end{align*}
	as desired.\\
	
	
	
	The more notationally heavy approach is to consider a complete orthonormal eigenbasis for $H$, which we may call $\{\ket{\lambda_i} \}$ where $\{\lambda_i\}$ are the eigenvalues of $H$. Under this basis, we have
	\begin{align*}
	\ket{\al} = \sum_i a_i \ket{\lambda_i} \quad\quad\quad \ket{\be} = \sum_i b_i \ket{\lambda_i}
	\end{align*} 
	and so 
	\begin{align*}
	\abs{\bra{\al} H \ket{\be}}^2  = \abs{ \sum_i a_i^* \bra{\lambda_i}\lambda_j b_j \ket{\lambda_j} }^2 = \abs{ \sum_{i} a_i^* \lambda_i b_i }^2 = \abs{ \sum_i \lp a_i \sqrt{\lambda_i}\rp^\dagger \lp b_i\sqrt{\lambda_i}\rp }^2.
	\end{align*}
	Note that $\sqrt{\lambda_i}\in \mathbb{R}^+$, which is possible because $\lambda_i > 0$. Now, call 
	\begin{align*}
	\ket{\al'} = \sum_i a_i \sqrt{\lambda_i} \ket{\lambda_i} \quad\quad\quad \ket{\be'} = \sum_i b_i \sqrt{\lambda_i} \ket{\lambda_i}.
	\end{align*}
	It is clear that 
	\begin{align*}
	\abs{\bra{\al} H \ket{\be}}^2  = \abs{ \bra{\al'}\ket{\be'}}^2.
	\end{align*}
	On the other hand, we have
	\begin{align*}
	&\bra{\al}H \ket{\al} = \sum_{i,j} a_i^* a_j \lambda_j \bra{\lambda_i}\ket{\lambda_j} = \sum_i \abs{a_i}^2 \lambda_i = \braket{\al'}\\
	&\bra{\be}H \ket{\be} = \sum_{i,j} b_i^* b_j \lambda_j \bra{\lambda_i}\ket{\lambda_j} = \sum_i \abs{b_i}^2 \lambda_i = \braket{\be'}.
	\end{align*}
	Applying the Cauchy-Schwarz inequality,
	\begin{align*}
	\abs{\bra{\al} H \ket{\be}}^2 = \abs{ \bra{\al'}\ket{\be'}}^2 \leq \braket{\al'} \braket{\be'} = \bra{\al}H \ket{\al} \bra{\be}H \ket{\be}
	\end{align*}
	we successfully proved the desired result. 
	
	\item The trace of $H$ is simply the sum of its eigenvalues, so $\tr(H) > 0$. To show explicitly, we use the orthonormal basis introduced in Part (a). Since $\lambda_i > 0$ for all $i$, we have
	\begin{align*}
	\tr(H) = \sum_i \bra{\lambda_i} H \ket{\lambda_i} = \sum_i \lambda_i \braket{\lambda_i} = \sum_i \lambda_i > 0.
	\end{align*}
\end{enumerate}



\noindent \textbf{4.} Let a unitary operator $U$ be given which satisfies the eigenvalue equation $U\ket{\lambda} = \lambda\ket{\lambda}$.

\begin{enumerate}[label = (\alph*)]
	\item Since $\braket{\lambda} \neq 0$ (because $\ket{\lambda}$ is an eigenvector), we have
	\begin{align*}
	\braket{\lambda} = \bra{\lambda} U^\dagger U  \ket{\lambda } = \abs{\lambda}^2 \braket{\lambda} \implies  \abs{\lambda}^2 = 1.
	\end{align*}
	Since $\lambda\in \mathbb{C}$, it must be of the form $\lambda = e^{i\theta}$	where $\theta\in \mathbb{R}$. 
	
	
	\item Let distinct  eigenvectors $\ket{\mu}$ and $\ket{\lambda}$ be given with corresponding (distinct) eigenvalues $e^{i\theta_\mu}$ and $e^{i\theta_\lambda}$. We have
	\begin{align*}
	\bra{\mu}\ket{\lambda} = \bra{\mu} U^\dagger U \ket{\lambda} =  e^{-i\theta_\mu}e^{i\theta_\lambda}\bra{\mu}\ket{\lambda}.
	\end{align*}
	Since the eigenvalues are not the same, we have that $e^{-i\theta_\mu}e^{i\theta_\lambda} \neq 1$ (i.e., that the complex conjugate of one is not the complex conjugate of the other). Thus, equality holds only if $\bra{\mu}\ket{\lambda} = 0$. 
\end{enumerate}



\noindent \textbf{5.}



\noindent \textbf{6.}


\noindent \textbf{7.}
	
\end{document}








