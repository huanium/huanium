\documentclass{article}
\usepackage{physics}
\usepackage{graphicx}
\usepackage{caption}
\usepackage{amsmath}
\usepackage{bm}
\usepackage{framed}
\usepackage{authblk}
\usepackage{empheq}
\usepackage{amsfonts}
\usepackage{esint}
\usepackage[makeroom]{cancel}
\usepackage{dsfont}
\usepackage{centernot}
\usepackage{mathtools}
\usepackage{bigints}
\usepackage{amsthm}
\theoremstyle{definition}
\newtheorem{defn}{Definition}[section]
\newtheorem{prop}{Proposition}[section]
\newtheorem{rmk}{Remark}[section]
\newtheorem{thm}{Theorem}[section]
\newtheorem{exmp}{Example}[section]
\newtheorem{prob}{Problem}[section]
\newtheorem{sln}{Solution}[section]
\newtheorem*{prob*}{Problem}
\newtheorem{exer}{Exercise}[section]
\newtheorem*{exer*}{Exercise}
\newtheorem*{sln*}{Solution}
\usepackage{empheq}
\usepackage{tensor}
\usepackage{xcolor}
%\definecolor{colby}{rgb}{0.0, 0.0, 0.5}
\definecolor{MIT}{RGB}{163, 31, 52}
\usepackage[pdftex]{hyperref}
%\hypersetup{colorlinks,urlcolor=colby}
\hypersetup{colorlinks,linkcolor={MIT},citecolor={MIT},urlcolor={MIT}}  
\usepackage[left=1.25in,right=1.25in,top=1.25in,bottom=1.25in]{geometry}

\usepackage{newpxtext,newpxmath}
\newcommand*\widefbox[1]{\fbox{\hspace{2em}#1\hspace{2em}}}

\newcommand{\p}{\partial}
\newcommand{\R}{\mathbb{R}}
\newcommand{\C}{\mathbb{C}}
\newcommand{\lag}{\mathcal{L}}
\newcommand{\nn}{\nonumber}
\newcommand{\ham}{\mathcal{H}}
\newcommand{\M}{\mathcal{M}}
\newcommand{\I}{\mathcal{I}}
\newcommand{\K}{\mathcal{K}}
\newcommand{\F}{\mathcal{F}}
\newcommand{\w}{\omega}
\newcommand{\lam}{\lambda}
\newcommand{\al}{\alpha}
\newcommand{\be}{\beta}
\newcommand{\x}{\xi}
\def\dbar{{\mkern3mu\mathchar'26\mkern-12mu   d}}


\newcommand{\G}{\mathcal{G}}

\newcommand{\f}[2]{\frac{#1}{#2}}

\newcommand{\ift}{\infty}

\newcommand{\lp}{\left(}
\newcommand{\rp}{\right)}

\newcommand{\lb}{\left[}
\newcommand{\rb}{\right]}

\newcommand{\lc}{\left\{}
\newcommand{\rc}{\right\}}


\newcommand{\V}{\mathbf{V}}
\newcommand{\U}{\mathcal{U}}
\newcommand{\Id}{\mathcal{I}}
\newcommand{\D}{\mathcal{D}}
\newcommand{\Z}{\mathcal{Z}}

%\setcounter{chapter}{-1}



\usepackage{enumitem}


\usepackage{subfig}
\usepackage{listings}
\captionsetup[lstlisting]{margin=0cm,format=hang,font=small,format=plain,labelfont={bf,up},textfont={it}}
\renewcommand*{\lstlistingname}{Code \textcolor{violet}{\textsl{Mathematica}}}
\definecolor{gris245}{RGB}{245,245,245}
\definecolor{olive}{RGB}{50,140,50}
\definecolor{brun}{RGB}{175,100,80}

%\hypersetup{colorlinks,urlcolor=colby}
\lstset{
	tabsize=4,
	frame=single,
	language=mathematica,
	basicstyle=\scriptsize\ttfamily,
	keywordstyle=\color{black},
	backgroundcolor=\color{gris245},
	commentstyle=\color{gray},
	showstringspaces=false,
	emph={
		r1,
		r2,
		epsilon,epsilon_,
		Newton,Newton_
	},emphstyle={\color{olive}},
	emph={[2]
		L,
		CouleurCourbe,
		PotentielEffectif,
		IdCourbe,
		Courbe
	},emphstyle={[2]\color{blue}},
	emph={[3]r,r_,n,n_},emphstyle={[3]\color{magenta}}
}






\begin{document}
		\begin{framed}
			\noindent Name: \textbf{Huan Q. Bui}\\
			Course: \textbf{8.398 - Selected Topics Grad Physics}
		\end{framed}
	
	
\noindent \textbf{1. What progress have you made towards achieving your goals for this semester?}\\

I have made very good progress in the lab (BEC1 experiment). We successfully revived the machine and was able to create a BEC. Once that was done, we were able to work on bringing back the other components of the lab to what they were before. We recently got a Lithium MOT along with the Na BEC, which is already 70\% of the way there. I was made good progress on designing and building a new laser setup which will hopefully improve the stability and rep-rate of the experiment. \\

Outside of lab, I have made friends with a few people from my cohort through pset sessions and been getting to know more people in my lab, which is very nice. \\



\noindent \textbf{2. What challenges have you faced in pursuing your goals?}\\


I'm beginning to see why taking three classes was not a very good idea. It is increasingly difficult to manage doing well in class and making good progress on the experimental work. I did extremely well in the first month. However, midterms have been rough on me. While I didn't do terribly I think I could have done much better if I had managed my time and learning habits better. I admit that I had a mini-burnout which set me off by a few days which affected my midterm performance. While the situation wasn't ideal, I think that the midterms were a necessary wake-up call for me, sort of, to manage my time and ``work/life balance'' better. I'm learning to be more efficient and more on top of things coming into the last month of the semester. Hopefully I will pass the classes (i.e., passing the quals) and more importantly learn some more cool physics. \\



\noindent \textbf{3. What have you learned thus far that you didn't know before coming to grad school?}\\
	
	
I knew that grad school will be a lot of work. I knew that I would probably work extremely hard but would also fail from time to time along the way. However, actually experiencing failures and setbacks in general was still rough. I had quite an easy time in undergrad where things were more ``chill'' and I was learning things at my own pace, being at the top of my class etc. Things have been different at MIT. I'm learning  how to deal with failure and cherishing my success. I'm also learning first hand that ``grinding'' probably won't be good for me in the long run. \\

In the beginning of the semester I was the one who asked about the ``grinding'' versus 9-5 mentality. While I'm still the ``grind'' type I'm learning the benefit of taking breaks and trying to work more efficiently in order to maintain a good balance of productivity and my own well-being. \\

I also realize that grad school could be quite lonely. I'm usually by myself, chipping away at my psets, or working my new laser design, or reading something for me research. I'm still navigating the social part of life at MIT. 

\end{document}














