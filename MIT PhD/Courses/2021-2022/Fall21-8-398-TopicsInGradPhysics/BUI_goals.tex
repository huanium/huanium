\documentclass{article}
\usepackage{physics}
\usepackage{graphicx}
\usepackage{caption}
\usepackage{amsmath}
\usepackage{bm}
\usepackage{framed}
\usepackage{authblk}
\usepackage{empheq}
\usepackage{amsfonts}
\usepackage{esint}
\usepackage[makeroom]{cancel}
\usepackage{dsfont}
\usepackage{centernot}
\usepackage{mathtools}
\usepackage{bigints}
\usepackage{amsthm}
\theoremstyle{definition}
\newtheorem{defn}{Definition}[section]
\newtheorem{prop}{Proposition}[section]
\newtheorem{rmk}{Remark}[section]
\newtheorem{thm}{Theorem}[section]
\newtheorem{exmp}{Example}[section]
\newtheorem{prob}{Problem}[section]
\newtheorem{sln}{Solution}[section]
\newtheorem*{prob*}{Problem}
\newtheorem{exer}{Exercise}[section]
\newtheorem*{exer*}{Exercise}
\newtheorem*{sln*}{Solution}
\usepackage{empheq}
\usepackage{tensor}
\usepackage{xcolor}
%\definecolor{colby}{rgb}{0.0, 0.0, 0.5}
\definecolor{MIT}{RGB}{163, 31, 52}
\usepackage[pdftex]{hyperref}
%\hypersetup{colorlinks,urlcolor=colby}
\hypersetup{colorlinks,linkcolor={MIT},citecolor={MIT},urlcolor={MIT}}  
\usepackage[left=1.25in,right=1.25in,top=1.25in,bottom=1.25in]{geometry}

\usepackage{newpxtext,newpxmath}
\newcommand*\widefbox[1]{\fbox{\hspace{2em}#1\hspace{2em}}}

\newcommand{\p}{\partial}
\newcommand{\R}{\mathbb{R}}
\newcommand{\C}{\mathbb{C}}
\newcommand{\lag}{\mathcal{L}}
\newcommand{\nn}{\nonumber}
\newcommand{\ham}{\mathcal{H}}
\newcommand{\M}{\mathcal{M}}
\newcommand{\I}{\mathcal{I}}
\newcommand{\K}{\mathcal{K}}
\newcommand{\F}{\mathcal{F}}
\newcommand{\w}{\omega}
\newcommand{\lam}{\lambda}
\newcommand{\al}{\alpha}
\newcommand{\be}{\beta}
\newcommand{\x}{\xi}
\def\dbar{{\mkern3mu\mathchar'26\mkern-12mu   d}}


\newcommand{\G}{\mathcal{G}}

\newcommand{\f}[2]{\frac{#1}{#2}}

\newcommand{\ift}{\infty}

\newcommand{\lp}{\left(}
\newcommand{\rp}{\right)}

\newcommand{\lb}{\left[}
\newcommand{\rb}{\right]}

\newcommand{\lc}{\left\{}
\newcommand{\rc}{\right\}}


\newcommand{\V}{\mathbf{V}}
\newcommand{\U}{\mathcal{U}}
\newcommand{\Id}{\mathcal{I}}
\newcommand{\D}{\mathcal{D}}
\newcommand{\Z}{\mathcal{Z}}

%\setcounter{chapter}{-1}



\usepackage{enumitem}


\usepackage{subfig}
\usepackage{listings}
\captionsetup[lstlisting]{margin=0cm,format=hang,font=small,format=plain,labelfont={bf,up},textfont={it}}
\renewcommand*{\lstlistingname}{Code \textcolor{violet}{\textsl{Mathematica}}}
\definecolor{gris245}{RGB}{245,245,245}
\definecolor{olive}{RGB}{50,140,50}
\definecolor{brun}{RGB}{175,100,80}

%\hypersetup{colorlinks,urlcolor=colby}
\lstset{
	tabsize=4,
	frame=single,
	language=mathematica,
	basicstyle=\scriptsize\ttfamily,
	keywordstyle=\color{black},
	backgroundcolor=\color{gris245},
	commentstyle=\color{gray},
	showstringspaces=false,
	emph={
		r1,
		r2,
		epsilon,epsilon_,
		Newton,Newton_
	},emphstyle={\color{olive}},
	emph={[2]
		L,
		CouleurCourbe,
		PotentielEffectif,
		IdCourbe,
		Courbe
	},emphstyle={[2]\color{blue}},
	emph={[3]r,r_,n,n_},emphstyle={[3]\color{magenta}}
}






\begin{document}
		\begin{framed}
			\noindent Name: \textbf{Huan Q. Bui}\\
			Course: \textbf{8.398 - Selected Topics Grad Physics}
		\end{framed}
	
	
	
\noindent \textbf{1. What strategies do you plan to use to help identify a research supervisor and/or research field this semester?}\\


\noindent \textbf{Answer: } I work in AOM physics, and at the moment I have already chosen a research advisor. I started working in an AMO lab in undergrad and really liked it, so I decided to continue further studies in this field at MIT. As for my advisor, I became interested in his research projects during my graduate school search. Having worked in his lab since June, I can say with certainty that I do not regret coming to MIT and choosing him as my advisor. \\

\noindent I technically have done one rotation, but it was between two experiments in the same lab. I'm currently very happy with the experiment I am in now: while we're only in the debugging stage and not working on new science, I find myself learning much more here than in the previous lab which was fully functioning. I like the various challenges of bringing the lab back to producing new data. They teach me how everything works (and doesn't work) and also give me a sense of pride whenever I get something to work again. It's really awesome.  \\

\noindent So I'd say that I don't really have a strategy for finding a research field/advisor. I like AMO, and I did some research on my advisor and decided to join his group. I suppose following what you want to do/what you find interesting counts as a strategy. I'd say that picking something to do and see if they are interesting is a reasonable strategy.\\  

\noindent \textbf{2. What other goals (physics or otherwise) do you have this semester?}\\


\noindent \textbf{Answer:} Physics-wise I want to restore and upgrade the experimental setup I am currently debugging and hopefully begin taking data by the end of December. There is already a very interesting research project lined up for the machine once it is back to running, so I really look forward to that.  \\

\noindent Otherwise, I hope to manage my time better. I believe there exists an optimal time-management strategy in which I could accomplish what I want to do in the lab, pass all my classes (3 of them), and exercise/sleep regularly. My goal is to successfully manage all three. I would also love to dedicate some time to make friends around MIT, but I think that could be a next-semester thing where I have fewer academic obligations.\\


\noindent \textbf{3. What do you anticipate being potential roadblocks to accomplishing your goals?}\\


\noindent \textbf{Answer:} Primarily time-management. I find myself working \textit{all the time} -- which is a lot of fun, but could be tiring sometimes. I would love some time outside of lab work and classes to learn about my research field from the books and papers which I have collected but have not had the chance to read. \\

\noindent I remember someone on the grad student panel suggest that I should never take three classes. I can totally see why it is not the optimal strategy in the long run, but I think it is worth the extra effort to fill in my knowledge gaps as quickly as possible before digging deeper in my research.  
\end{document}














