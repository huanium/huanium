\documentclass{article}
\usepackage{physics}
\usepackage{graphicx}
\usepackage{caption}
\usepackage{amsmath}
\usepackage{bm}
\usepackage{framed}
\usepackage{authblk}
\usepackage{empheq}
\usepackage{amsfonts}
\usepackage{esint}
\usepackage[makeroom]{cancel}
\usepackage{dsfont}
\usepackage{centernot}
\usepackage{mathtools}
\usepackage{bigints}
\usepackage{amsthm}
\theoremstyle{definition}
\newtheorem{defn}{Definition}[section]
\newtheorem{prop}{Proposition}[section]
\newtheorem{rmk}{Remark}[section]
\newtheorem{thm}{Theorem}[section]
\newtheorem{exmp}{Example}[section]
\newtheorem{prob}{Problem}[section]
\newtheorem{sln}{Solution}[section]
\newtheorem*{prob*}{Problem}
\newtheorem{exer}{Exercise}[section]
\newtheorem*{exer*}{Exercise}
\newtheorem*{sln*}{Solution}
\usepackage{empheq}
\usepackage{tensor}
\usepackage{xcolor}
%\definecolor{colby}{rgb}{0.0, 0.0, 0.5}
\definecolor{MIT}{RGB}{163, 31, 52}
\usepackage[pdftex]{hyperref}
%\hypersetup{colorlinks,urlcolor=colby}
\hypersetup{colorlinks,linkcolor={MIT},citecolor={MIT},urlcolor={MIT}}  
\usepackage[left=1in,right=1in,top=1in,bottom=1in]{geometry}

\usepackage{newpxtext,newpxmath}
\newcommand*\widefbox[1]{\fbox{\hspace{2em}#1\hspace{2em}}}

\newcommand{\p}{\partial}
\newcommand{\R}{\mathbb{R}}
\newcommand{\C}{\mathbb{C}}
\newcommand{\lag}{\mathcal{L}}
\newcommand{\nn}{\nonumber}
\newcommand{\ham}{\mathcal{H}}
\newcommand{\M}{\mathcal{M}}
\newcommand{\I}{\mathcal{I}}
\newcommand{\K}{\mathcal{K}}
\newcommand{\F}{\mathcal{F}}
\newcommand{\w}{\omega}
\newcommand{\lam}{\lambda}
\newcommand{\al}{\alpha}
\newcommand{\be}{\beta}
\newcommand{\x}{\xi}

\newcommand{\G}{\mathcal{G}}

\newcommand{\f}[2]{\frac{#1}{#2}}

\newcommand{\ift}{\infty}

\newcommand{\lp}{\left(}
\newcommand{\rp}{\right)}

\newcommand{\lb}{\left[}
\newcommand{\rb}{\right]}

\newcommand{\lc}{\left\{}
\newcommand{\rc}{\right\}}


\newcommand{\V}{\mathbf{V}}
\newcommand{\U}{\mathcal{U}}
\newcommand{\Id}{\mathcal{I}}
\newcommand{\D}{\mathcal{D}}
\newcommand{\Z}{\mathcal{Z}}

%\setcounter{chapter}{-1}


\usepackage{enumitem}



\usepackage{subfig}
\usepackage{listings}
\captionsetup[lstlisting]{margin=0cm,format=hang,font=small,format=plain,labelfont={bf,up},textfont={it}}
\renewcommand*{\lstlistingname}{Code \textcolor{violet}{\textsl{Mathematica}}}
\definecolor{gris245}{RGB}{245,245,245}
\definecolor{olive}{RGB}{50,140,50}
\definecolor{brun}{RGB}{175,100,80}

%\hypersetup{colorlinks,urlcolor=colby}
\lstset{
	tabsize=4,
	frame=single,
	language=mathematica,
	basicstyle=\scriptsize\ttfamily,
	keywordstyle=\color{black},
	backgroundcolor=\color{gris245},
	commentstyle=\color{gray},
	showstringspaces=false,
	emph={
		r1,
		r2,
		epsilon,epsilon_,
		Newton,Newton_
	},emphstyle={\color{olive}},
	emph={[2]
		L,
		CouleurCourbe,
		PotentielEffectif,
		IdCourbe,
		Courbe
	},emphstyle={[2]\color{blue}},
	emph={[3]r,r_,n,n_},emphstyle={[3]\color{magenta}}
}






\begin{document}
\begin{framed}
	\noindent Name: \textbf{Huan Q. Bui}\\
	Course: \textbf{8.309 - Classical Mechanics III}\\
	Problem set: \textbf{\#1}
\end{framed}
	
	
	
\noindent \textbf{1.} \textbf{Two Particles in a Gravitational Field}


\begin{enumerate}[label=(\alph*)]
	\item In the center of mass (COM) frame, the Lagrangian is given by 
	\begin{align*}
	\lag 
	&= T - U\\
	&= T - U_\text{attraction} - U_g\\
	&= \lb \f{1}{2} m_1 \dot{\vec{r}}_1^2  + \f{1}{2}m_2\dot{\vec{r}}_2^2\rb - \lb - \f{Gm_1m_2}{\abs{\vec{r}}} - g(m_1x_1 + m_2x_2) \rb \\ 
	&= \f{1}{2}(m_1+m_2) \dot{\vec{R}}_{\text{COM}}^2 + \f{1}{2} \f{m_1m_2}{m_1+m_2} \dot{\vec{r}}^2 + \f{Gm_1m_2}{r} + g(m_1+m_2) X_\text{COM},
	\end{align*}
	where $X_\text{COM}$ is the $x$-component of $\vec{R}_\text{COM}$. To obtain this Lagrangian, we have solved for $\vec{r}_1$ and $\vec{r}_2$ in terms of $\vec{R}_\text{COM}$ and $\vec{r}$ from the following definitions:
	\begin{equation*}
	\begin{cases}
	\vec{R}_\text{COM} = (m_1 \vec{r}_1 + m_2 \vec{r}_2)/(m_1+m_2)  \\
	\vec{r} = \vec{r}_2 - \vec{r}_1
	\end{cases} 
	\implies \begin{cases}
	\vec{r}_1 = \vec{R}_\text{COM} - m_2 \vec{r}/(m_1+m_2) \\
	\vec{r}_2 = \vec{R}_\text{COM} + m_1 \vec{r}/(m_1+m_2) \\
	X_\text{COM} = (m_1 x_1 + m_1 x_2)/(m_1 + m_2)
	\end{cases}
	\end{equation*}
	Calling the total mass $m_1 + m_2 = M$ and reduced mass $m_1m_2/(m_1+m_2) = \mu$, we have
	\begin{equation*}
	\boxed{\lag = 
	\underbrace{ \f{1}{2}M \dot{\vec{R}}_{\text{COM}}^2 + gMX_\text{COM} }_{\text{COM}}  + \underbrace{ \f{1}{2} \mu \dot{\vec{r}}^2 + \f{Gm_1m_2}{r}  }_{\text{relative}}}
	\end{equation*}
	The Lagrangian splits into two parts which describe the center-of-mass and relative dynamics, respectively. This makes sense physically because both bodies are essentially in ``free fall'' with each other. The center of mass of the system is decoupled from the relative motion, i.e. we can go to a frame in which the center of mass is stationary, and the only dynamics left is the relative motion of the masses.  
	
	
	\item Going to the center of mass frame, we have the following Lagrangian
	\begin{equation*}
	\lag_r = \f{1}{2} \mu \dot{\vec{r}}^2 + \f{Gm_1m_2}{r}.
	\end{equation*}
	To avoid taking derivatives of the basis vectors in spherical coordinates, we may write $\dot{\vec{r}} = \dot{x}^2 + \dot{y}^2 + \dot{z}^2$ where $(x,y,z) = r(\sin\theta\cos\phi, \sin\theta\sin\phi, \cos\theta)$ and let Mathematica compute the more familiar derivatives. The result is 
	\begin{equation*}
	\boxed{\lag_r = \f{1}{2}\mu \lb  \dot{r}^2 + r^2\lp\dot{\theta}^2 + \dot{\phi}^2\sin^2\theta \rp  \rb + \f{Gm_1m_2}{r}}
	\end{equation*}
	The corresponding Euler-Lagrange equations are
	\begin{align*}
	\f{d}{dt}\lp \f{\p \lag_r}{\p \dot{r}}\rp = \f{\p \lag_r}{\p r}  
	&\implies 
	\mu \ddot{r} = -\f{G m_1 m_2}{r^2} + \mu r\lp\dot{\theta}^2 + \dot{\phi}^2\sin^2\theta\rp\\
	\f{d}{d\theta}\lp \f{\p \lag_r}{\p \dot{\theta}}\rp = \f{\p \lag_r}{\p \theta}  
	&\implies 
	2\dot{r}\dot{\theta} + r\ddot{\theta} = r\dot{\phi}^2\cos\theta \sin\theta \\
	\f{d}{d\phi}\lp \f{\p \lag_r}{\p \dot{\phi}}\rp = \f{\p \lag_r}{\p \phi} 
	&\implies 
	\mu r \sin\theta \lb 2\lp \dot{r}\sin\theta  + r \dot{\theta} \cos\theta \rp \dot{\phi} + r \ddot{\phi} \sin\theta \rb =0
	\end{align*}
	
	\item The Hamiltonian corresponding to this Lagrangian is obtained via the Legendre transform. To do this, we first find the canonical momenta $p_r, p_\theta, p_\phi$ in Mathematica using $p_i = \p \lag_r/ \p \dot{q}_i$. 
	\begin{align*}
	p_r = \mu \dot{r} \quad\quad p_\theta = \mu r^2 \dot{\theta} \quad\quad p_\phi = \mu r^2\sin^2\theta \dot{\phi}
	\end{align*}
	The Hamiltonian is 
	\begin{align*}
	\ham_r &= \lp p_r \dot{r} + p_\theta \dot{\theta} + p_\phi \dot{\phi} \rp - \lag_r \\
	&= \boxed{\f{1}{2}\mu \lb \dot{r}^2 + r^2 \lp \dot{\theta}^2+ \dot{\phi}^2\sin^2\theta  \rp \rb - \f{Gm_1m_2}{r}}
	\end{align*}
	Alternatively, we can get this Hamiltonian (which is the total energy) by recognizing that the kinetic part of the Lagrangian is quadratic and the potential is not velocity dependent. \\
	
	To find the Hamiltonian equations of motion, we first express $\ham_r$ in terms of the canonical momenta:
	\begin{equation*}
	\ham_r = \f{p_r^2}{2\mu} + \f{p_\theta^2}{2\mu r^2} + \f{p_\phi^2}{2\mu r^2\sin^2\theta} - \f{G m_1m_2}{r}
	\end{equation*}
	With this, we find
	\begin{align*}
	\dot{r} = \f{\p \ham_r }{\p p_r} = \f{p_r}{\mu}; 
	\quad\quad 
	\dot{\theta} = \f{\p \ham}{\p p_\theta} = \f{p_\theta}{\mu r^2}; 
	\quad\quad 
	\dot{\phi} = \f{p_\phi}{\mu r^2 \sin^2\theta}
	\end{align*}
	and 
	\begin{align*}
	\dot{p}_r = -\f{\p \ham_r}{\p r} = \f{p_\theta^2}{\mu r^3} + \f{p_\phi^2}{\mu r^3\sin^2\theta} - \f{Gm_1m_2}{r^2}; 
	\quad\quad 
	\dot{p}_\theta = -\f{\p\ham_r}{\p \theta}  = \f{p_\phi^2\cos\theta}{\mu r^2\sin^3\theta}; 
	\quad\quad \dot{p}_\phi = - \f{\p \ham}{\p \phi} = 0.
	\end{align*}

	
	
	
	
	
	
	
	
	
	
	
	
	\item 
	Mathematica code: 
	\begin{lstlisting}
	(* Problem 1 *)
	
	(* KE, PE, and Lagrangian *)
	
	In[1]:= KE = (\[Mu]/2)*(D[x[t], t]^2 + D[y[t], t]^2 + D[z[t], t]^2);
	
	In[2]:= PE = -G*m1*m2/r[t];
	
	In[3]:= L = KE - PE
	
	Out[3]= (G m1 m2)/r[t] + 
	1/2 \[Mu] (Derivative[1][x][t]^2 + Derivative[1][y][t]^2 + 
	Derivative[1][z][t]^2)
	
	In[4]:= L = 
	L /. {x[t] -> r[t]*Sin[\[Theta][t]]*Cos[\[Phi][t]], 
	y[t] -> r[t]*Sin[\[Theta][t]]*Sin[\[Phi][t]], 
	z[t] -> r[t]*Cos[\[Theta][t]],
	x'[t] -> D[r[t]*Sin[\[Theta][t]]*Cos[\[Phi][t]], t], 
	y'[t] -> D[r[t]*Sin[\[Theta][t]]*Sin[\[Phi][t]], t], 
	z'[t] -> D[r[t]*Cos[\[Theta][t]], t]} // FullSimplify
	
	Out[4]= 1/2 ((2 G m1 m2)/
	r[t] + \[Mu] Derivative[1][r][t]^2 + \[Mu] r[
	t]^2 (Derivative[1][\[Theta]][t]^2 + 
	Sin[\[Theta][t]]^2 Derivative[1][\[Phi]][t]^2))
	
	In[5]:= (* The 'r' equation *)
	
	In[6]:= D[D[L, r'[t]], t] // FullSimplify
	
	Out[6]= \[Mu] (r^\[Prime]\[Prime])[t]
	
	In[7]:= D[L, r[t]] // FullSimplify
	
	Out[7]= -((G m1 m2)/
	r[t]^2) + \[Mu] r[
	t] (Derivative[1][\[Theta]][t]^2 + 
	Sin[\[Theta][t]]^2 Derivative[1][\[Phi]][t]^2)
	
	In[8]:= (* The 'Theta' equation *)
	
	In[9]:= D[D[L, \[Theta]'[t]], t] // FullSimplify
	
	Out[9]= \[Mu] r[
	t] (2 Derivative[1][r][t] Derivative[1][\[Theta]][t] + 
	r[t] (\[Theta]^\[Prime]\[Prime])[t])
	
	In[10]:= D[L, \[Theta][t]] // FullSimplify
	
	Out[10]= \[Mu] Cos[\[Theta][t]] r[t]^2 Sin[\[Theta][t]] Derivative[
	1][\[Phi]][t]^2
	
	(* The 'Phi' equation *)
	
	In[11]:= D[D[L, \[Phi]'[t]], t] // FullSimplify
	
	Out[11]= \[Mu] r[
	t] Sin[\[Theta][
	t]] (2 (Sin[\[Theta][t]] Derivative[1][r][t] + 
	Cos[\[Theta][t]] r[t] Derivative[1][\[Theta]][t]) Derivative[
	1][\[Phi]][t] + 
	r[t] Sin[\[Theta][t]] (\[Phi]^\[Prime]\[Prime])[t])
	
	In[12]:= D[L, \[Phi][t]] // FullSimplify
	
	Out[12]= 0
	
	(* Canonical momenta *)
	
	In[13]:= pr[t] = D[L, r'[t]]
	
	Out[13]= \[Mu] Derivative[1][r][t]
	
	In[14]:= p\[Theta][t] = D[L, \[Theta]'[t]]
	
	Out[14]= \[Mu] r[t]^2 Derivative[1][\[Theta]][t]
	
	In[15]:= p\[Phi][t] = D[L, \[Phi]'[t]]
	
	Out[15]= \[Mu] r[t]^2 Sin[\[Theta][t]]^2 Derivative[1][\[Phi]][t]
	
	(* Lagrangian to Hamiltonian *)
	
	In[17]:= H = (pr[t]*r'[t] + p\[Theta][t]*\[Theta]'[t] + 
	p\[Phi][t]*\[Phi]'[t]) - L // Expand
	
	Out[17]= -((G m1 m2)/r[t]) + 1/2 \[Mu] Derivative[1][r][t]^2 + 
	1/2 \[Mu] r[t]^2 Derivative[1][\[Theta]][t]^2 + 
	1/2 \[Mu] r[t]^2 Sin[\[Theta][t]]^2 Derivative[1][\[Phi]][t]^2
	
	In[63]:= (* Velocities: new instances of P to put in Hamiltonian *)
	
	In[18]:= velocities = 
	Solve[{Pr[t] == D[L, r'[t]], P\[Theta][t] == D[L, \[Theta]'[t]], 
	P\[Phi][t] == D[L, \[Phi]'[t]]}, {r'[t], \[Theta]'[t], \[Phi]'[
	t]}][[1]]
	
	Out[18]= {Derivative[1][r][t] -> Pr[t]/\[Mu], 
	Derivative[1][\[Theta]][t] -> P\[Theta][t]/(\[Mu] r[t]^2), 
	Derivative[1][\[Phi]][t] -> (
	Csc[\[Theta][t]]^2 P\[Phi][t])/(\[Mu] r[t]^2)}
	
	(* Write Hamiltonian in terms of momenta: *)
	
	In[19]:= H = H /. velocities // Expand
	
	Out[19]= Pr[t]^2/(2 \[Mu]) + P\[Theta][t]^2/(2 \[Mu] r[t]^2) + (
	Csc[\[Theta][t]]^2 P\[Phi][t]^2)/(2 \[Mu] r[t]^2) - (G m1 m2)/r[t]
	
	(*Hamiltonian EOMs*)
	
	In[20]:= D[r[t], t] == D[H, Pr[t]]
	
	Out[20]= Derivative[1][r][t] == Pr[t]/\[Mu]
	
	In[21]:= D[\[Theta][t], t] == D[H, P\[Theta][t]]
	
	Out[21]= Derivative[1][\[Theta]][t] == P\[Theta][t]/(\[Mu] r[t]^2)
	
	In[22]:= D[\[Phi][t], t] == D[H, P\[Phi][t]]
	
	Out[22]= Derivative[1][\[Phi]][t] == (
	Csc[\[Theta][t]]^2 P\[Phi][t])/(\[Mu] r[t]^2)
	
	In[26]:= D[Pr[t], t] == -D[H, r[t]] // Expand
	
	Out[26]= Derivative[1][Pr][t] == 
	P\[Theta][t]^2/(\[Mu] r[t]^3) + (
	Csc[\[Theta][t]]^2 P\[Phi][t]^2)/(\[Mu] r[t]^3) - (G m1 m2)/r[t]^2
	
	In[27]:= D[P\[Theta][t], t] == -D[H, \[Theta][t]] // Expand
	
	Out[27]= Derivative[1][P\[Theta]][t] == (
	Cot[\[Theta][t]] Csc[\[Theta][t]]^2 P\[Phi][t]^2)/(\[Mu] r[t]^2)
	
	In[83]:= D[P\[Phi][t], t] == -D[H, \[Phi][t]] // Expand
	
	Out[83]= Derivative[1][P\[Phi]][t] == 0
	\end{lstlisting}
	
\end{enumerate}







\newpage
\noindent \textbf{2.} \textbf{Double Pendulum in a Plane with Gravity}

\begin{enumerate}[label=(\alph*)]
	\item In rectangular coordinates:
	\begin{align*}
	\lag 
	&= T - U \\
	&= \f{1}{2}m_1(\dot{x}_1^2 + \dot{y}_1^2) + \f{1}{2}m_2(\dot{x}_2^2 + \dot{y}_2^2) - m_1gy_1 - m-2 g y_2.
	\end{align*}
	With 
	\begin{align*}
	x_1 = l_1 \sin\theta_1; &\quad\quad x_2 = l_2 \sin\theta_2 + l_1 \sin\theta_1 \\
	y_1 = -l_1\cos\theta_1; &\quad\quad y_2 = -l_2\cos\theta_2 - l_1\cos\theta_1
	\end{align*}
	we have
	\begin{align*}
	\boxed{\lag = \f{1}{2}(m_1+m_2) l_1^2\dot{\theta}_1^2  + \f{1}{2}l_1^2m_2\dot{\theta}_2^2 + l_1l_2m_2\cos(\theta_1 - \theta_2) \dot{\theta_1}\dot{\theta_2} + gl_1(m_1+m_2)\cos\theta_1 + gl_2m_2\cos\theta_2}
	\end{align*}
	The equations of motion are:
	\begin{equation*}
	\f{d}{dt}\lp \f{\p \lag}{\p \dot{\theta}_1} \rp = \f{\p \lag}{\p \theta_1} 
	\implies 
	l_2m_2\sin(\theta_1 - \theta_2) \dot{\theta}_2^2 + (m_1+m_2)(g\sin\theta_1 + l_1 \ddot{\theta}_1) + l_2m_2\cos(\theta_1 - \theta_2)\ddot{\theta}_2 = 0
	\end{equation*}
	\begin{equation*}
	\f{d}{dt}\lp \f{\p \lag}{\p \dot{\theta}_2} \rp = \f{\p \lag}{\p \theta_1}
	\implies 
	g\sin\theta_2 - l_1\sin(\theta_1 - \theta_2)\dot{\theta}_1^2 + l_1\cos(\theta_1-\theta_2) \ddot{\theta}_1 + l_2 \ddot{\theta}_2 = 0.
	\end{equation*}
	
	
	\item Now take $m_1 = m_2 = m$. Following a similar procedure as before, we first find the canonical momenta using $p_i = \p \lag / \p \dot{q}_i$, then find the Hamiltonian by Legendre-transforming the Lagrangian. Equivalently, we can simply take the total energy, as the kinetic part of the Lagrangian is quadratic and the potential is velocity independent. 
	\begin{equation*}
	\boxed{\ham = l_1^2 m \dot\theta_1^2 +\f{1}{2}l_2^2 m \dot\theta_2^2 + l_1l_2m\cos(\theta_1-\theta_2)\dot\theta_1\dot\theta_2 - 2gl_1m\cos\theta_1 - g l_2 m \cos\theta_2 }
	\end{equation*}
	To write $\ham$ in terms of the canonical momenta, we need to find how they are related to the velocities. Setting $p_{\theta_i} = \p \lag / \p \dot{\theta}_i$ we find 
	\begin{equation*}
	\dot{\theta}_1 = \f{-l_2 p_{\theta_1} + l_1\cos(\theta_1-\theta_2)p_{\theta_2}}{l_1^2 l_2 m [\cos^2(\theta_1 - \theta_2) - 2]}
	\quad\quad 
	\dot{\theta}_2 = \f{l_2\cos(\theta_1-\theta_2)p_{\theta_1} - 2l_1 p_{\theta_2}}{l_1l_2^2 m [\cos^2(\theta_1 - \theta_2)-2]}
	\end{equation*}
	In terms of $p_{\theta_1}$ and $p_{\theta_2}$, the Hamiltonian is 
	\begin{equation*}
	\boxed{\ham = 
	-\frac{l_1^2 \left(g{l_2}^2 m^2 [\cos (2 (\theta_1-\theta_2))-3] 
	(2l_1 \cos \theta_1+l_2 \cos \theta_2 +2 2p_{\theta_2}^2\right)
	-2l_1l_2p_{\theta_1}p_{\theta_2} \cos (\theta_1-\theta_2)+ l_2^2 p_{\theta_1}^2}
	{l_1^2l_2^2 m [\cos (2 (\theta_1-\theta_2))-3]}}
	\end{equation*}
	The remain equations of motion are the ``$\dot{p}$'' equations:
	\begin{align*}
	\dot{p}_{\theta_1} = 
	\frac{-2 g l_1^3 l_2^2 m^2 \sin \theta_1[\cos
		(2 (\theta_1-\theta_2))-3]^2
		+2 \sin (2
		(\theta_1-\theta_2)) \left(2 l_1^2
		p_{\theta_2}^2+l_2^2 p_{\theta_1}^2\right)
		}
	{l_1^2 l_2^2 m [\cos (2
		(\theta_1-\theta_2))-3]^2} \\
	+ \frac{-2
		l_1l_2 p_{\theta_1}p_{\theta_2} \sin
		(\theta_1-\theta_2) (\cos (2 (\theta_1-\theta_2))+5)}{l_1^2 l_2^2 m [\cos (2
		(\theta_1-\theta_2))-3]^2}
	\end{align*}
	and 
	\begin{align*}
	\dot{p}_{\theta_2} 
	= 
	&g l_2 m \sin\theta_2 \\
	&+\frac{2 \sin (\theta_1-\theta_2) \left(-4
		l_1^2 p_{\theta_2}^2 \cos (\theta_1-\theta_2)+l_1 l_2 p_{\theta_1}p_{\theta_2} [\cos (2 (\theta_1-\theta_2))+5]-2 l_2^2 p_{\theta_1}^2 \cos (\theta_1-\theta_2)\right)}{l_1^2 l_2^2 m [\cos (2 (\theta_1-\theta_2))-3]^2}
	\end{align*}
	
	\item Mathematica code:
	\begin{lstlisting}
	(* Problem 2 *)
	
	(* KE, PE, and Lagrangian *)
	
	In[1]:= KE = (1/2)*m1*(D[x1[t], t]^2 + D[y1[t], t]^2) + (1/2)*
	m2*(D[x2[t], t]^2 + D[y2[t], t]^2);
	
	In[2]:= PE = m1*g*y1[t] + m2*g*y2[t];
	
	In[3]:= L = KE - PE;
	
	In[4]:= L = L /. {
	x1[t] -> l1*Sin[\[Theta]1[t]],
	y1[t] -> -l1*Cos[\[Theta]1[t]],
	x2[t] -> l2*Sin[\[Theta]2[t]] + l1*Sin[\[Theta]1[t]],
	y2[t] -> -l2*Cos[\[Theta]2[t]] - l1*Cos[\[Theta]1[t]],
	x1'[t] -> D[l1*Sin[\[Theta]1[t]], t],
	y1'[t] -> D[-l1*Cos[\[Theta]1[t]], t],
	x2'[t] -> D[l2*Sin[\[Theta]2[t]] + l1*Sin[\[Theta]1[t]], t],
	y2'[t] -> D[-l2*Cos[\[Theta]2[t]] - l1*Cos[\[Theta]1[t]], t]} // 
	FullSimplify
	
	Out[4]= 1/2 (2 g (l1 (m1 + m2) Cos[\[Theta]1[t]] + 
	l2 m2 Cos[\[Theta]2[t]]) + 
	l1^2 (m1 + m2) Derivative[1][\[Theta]1][t]^2 + 
	2 l1 l2 m2 Cos[\[Theta]1[t] - \[Theta]2[t]] Derivative[
	1][\[Theta]1][t] Derivative[1][\[Theta]2][t] + 
	l2^2 m2 Derivative[1][\[Theta]2][t]^2)
	
	(* Lagrangian EOMs *)
	
	In[5]:= D[D[L, \[Theta]1'[t]], t] == D[L, \[Theta]1[t]] // FullSimplify
	
	Out[5]= l1 (l2 m2 Sin[\[Theta]1[t] - \[Theta]2[t]] Derivative[
	1][\[Theta]2][
	t]^2 + (m1 + m2) (g Sin[\[Theta]1[t]] + 
	l1 (\[Theta]1^\[Prime]\[Prime])[t]) + 
	l2 m2 Cos[\[Theta]1[t] - \[Theta]2[
	t]] (\[Theta]2^\[Prime]\[Prime])[t]) == 0
	
	In[6]:= D[D[L, \[Theta]2'[t]], t] == D[L, \[Theta]2[t]] // FullSimplify
	
	Out[6]= l2 m2 (g Sin[\[Theta]2[t]] - 
	l1 Sin[\[Theta]1[t] - \[Theta]2[t]] Derivative[1][\[Theta]1][
	t]^2 + l1 Cos[\[Theta]1[t] - \[Theta]2[
	t]] (\[Theta]1^\[Prime]\[Prime])[t] + 
	l2 (\[Theta]2^\[Prime]\[Prime])[t]) == 0
	
	(* Take m1 = m2 = m *)
	
	In[7]:= D[D[L, \[Theta]1'[t]], t] == D[L, \[Theta]1[t]] /. {m1 -> m, 
	m2 -> m} // FullSimplify
	
	Out[7]= l1 m (2 g Sin[\[Theta]1[t]] + 
	l2 Sin[\[Theta]1[t] - \[Theta]2[t]] Derivative[1][\[Theta]2][
	t]^2 + 2 l1 (\[Theta]1^\[Prime]\[Prime])[t] + 
	l2 Cos[\[Theta]1[t] - \[Theta]2[t]] (\[Theta]2^\[Prime]\[Prime])[
	t]) == 0
	
	In[8]:= D[D[L, \[Theta]2'[t]], t] == D[L, \[Theta]2[t]] /. {m1 -> m, 
	m2 -> m} // FullSimplify
	
	Out[8]= l2 m (g Sin[\[Theta]2[t]] - 
	l1 Sin[\[Theta]1[t] - \[Theta]2[t]] Derivative[1][\[Theta]1][
	t]^2 + l1 Cos[\[Theta]1[t] - \[Theta]2[
	t]] (\[Theta]1^\[Prime]\[Prime])[t] + 
	l2 (\[Theta]2^\[Prime]\[Prime])[t]) == 0
	
	(*Hamiltonian*)
	
	In[9]:= H = (D[L, \[Theta]1'[t]]*D[\[Theta]1[t], t] + 
	D[L, \[Theta]2'[t]]*D[\[Theta]2[t], t]) - L /. {m1 -> m, 
	m2 -> m} // Expand
	
	Out[9]= -2 g l1 m Cos[\[Theta]1[t]] - g l2 m Cos[\[Theta]2[t]] + 
	l1^2 m Derivative[1][\[Theta]1][t]^2 + 
	l1 l2 m Cos[\[Theta]1[t] - \[Theta]2[t]] Derivative[1][\[Theta]1][
	t] Derivative[1][\[Theta]2][t] + 
	1/2 l2^2 m Derivative[1][\[Theta]2][t]^2
	
	(*solve for velocities in terms of momenta to write H in terms of \
	momenta*)
	
	In[10]:= D[L, \[Theta]1'[t]] /. {m1 -> m, m2 -> m} // FullSimplify
	
	Out[10]= l1 m (2 l1 Derivative[1][\[Theta]1][t] + 
	l2 Cos[\[Theta]1[t] - \[Theta]2[t]] Derivative[1][\[Theta]2][t])
	
	In[11]:= velocities = 
	Solve[{P\[Theta]1[t] == D[L, \[Theta]1'[t]], 
	P\[Theta]2[t] == D[L, \[Theta]2'[t]]}, {\[Theta]1'[
	t], \[Theta]2'[t]}][[1]] /. {m1 -> m, m2 -> m} // FullSimplify
	
	Out[11]= {Derivative[1][\[Theta]1][t] -> (-l2 P\[Theta]1[t] + 
	l1 Cos[\[Theta]1[t] - \[Theta]2[t]] P\[Theta]2[t])/(
	l1^2 l2 m (-2 + Cos[\[Theta]1[t] - \[Theta]2[t]]^2)), 
	Derivative[1][\[Theta]2][t] -> (
	l2 Cos[\[Theta]1[t] - \[Theta]2[t]] P\[Theta]1[t] - 
	2 l1 P\[Theta]2[t])/(
	l1 l2^2 m (-2 + Cos[\[Theta]1[t] - \[Theta]2[t]]^2))}
	
	In[12]:= H = H /. velocities // FullSimplify
	
	Out[12]= -((l2^2 P\[Theta]1[t]^2 - 
	2 l1 l2 Cos[\[Theta]1[t] - \[Theta]2[t]] P\[Theta]1[
	t] P\[Theta]2[t] + 
	l1^2 (g l2^2 m^2 (-3 + 
	Cos[2 (\[Theta]1[t] - \[Theta]2[t])]) (2 l1 Cos[\[Theta]1[
	t]] + l2 Cos[\[Theta]2[t]]) + 
	2 P\[Theta]2[t]^2))/(l1^2 l2^2 m (-3 + 
	Cos[2 (\[Theta]1[t] - \[Theta]2[t])])))
	
	(* Hamiltonian EOMs *)
	
	In[16]:= Solve[P\[Theta]1'[t] == -D[H, \[Theta]1[t]], 
	P\[Theta]1'[t]][[1]] // FullSimplify
	
	Out[16]= {Derivative[1][P\[Theta]1][
	t] -> (-2 g l1^3 l2^2 m^2 (-3 + 
	Cos[2 (\[Theta]1[t] - \[Theta]2[t])])^2 Sin[\[Theta]1[t]] - 
	2 l1 l2 (5 + Cos[2 (\[Theta]1[t] - \[Theta]2[t])]) P\[Theta]1[
	t] P\[Theta]2[t] Sin[\[Theta]1[t] - \[Theta]2[t]] + 
	2 (l2^2 P\[Theta]1[t]^2 + 2 l1^2 P\[Theta]2[t]^2) Sin[
	2 (\[Theta]1[t] - \[Theta]2[t])])/(l1^2 l2^2 m (-3 + 
	Cos[2 (\[Theta]1[t] - \[Theta]2[t])])^2)}
	
	In[17]:= Solve[P\[Theta]2'[t] == -D[H, \[Theta]2[t]], 
	P\[Theta]2'[t]][[1]] // FullSimplify
	
	Out[17]= {Derivative[1][P\[Theta]2][
	t] -> (2 (-2 l2^2 Cos[\[Theta]1[t] - \[Theta]2[t]] P\[Theta]1[
	t]^2 + l1 l2 (5 + 
	Cos[2 (\[Theta]1[t] - \[Theta]2[t])]) P\[Theta]1[
	t] P\[Theta]2[t] - 
	4 l1^2 Cos[\[Theta]1[t] - \[Theta]2[t]] P\[Theta]2[
	t]^2) Sin[\[Theta]1[t] - \[Theta]2[t]])/(l1^2 l2^2 m (-3 + 
	Cos[2 (\[Theta]1[t] - \[Theta]2[t])])^2) - 
	g l2 m Sin[\[Theta]2[t]]}
	
	In[66]:= \[Theta]1'[t] == D[H, P\[Theta]1[t]] // FullSimplify
	
	Out[66]= l1 Derivative[1][\[Theta]1][t] == (
	2 l2 P\[Theta]1[t] - 
	2 l1 Cos[\[Theta]1[t] - \[Theta]2[t]] P\[Theta]2[t])/(
	3 l1 l2 m - l1 l2 m Cos[2 \[Theta]1[t] - 2 \[Theta]2[t]])
	
	In[67]:= \[Theta]2'[t] == D[H, P\[Theta]2[t]] // FullSimplify
	
	Out[67]= Derivative[1][\[Theta]2][t] == (
	2 (l2 Cos[\[Theta]1[t] - \[Theta]2[t]] P\[Theta]1[t] - 
	2 l1 P\[Theta]2[t]))/(
	l1 l2^2 m (-3 + Cos[2 (\[Theta]1[t] - \[Theta]2[t])]))
	\end{lstlisting}
\end{enumerate}




\newpage
\noindent \textbf{3.} \textbf{Point Mass on a Hoop:} Goldstein Ch.2 Problem \#18. 

By the geometry of the problem, we the system may be described by one (spherical) coordinate $\theta$ defined as usual. $r$ is fixed at $r=a$ and $\phi(t) = \omega t$, where $\omega$ is fixed.
\begin{equation*}
(x,y,z) = (a\sin\theta \cos\omega t, a\sin\theta\sin\omega t, a\cos\theta).
\end{equation*}
With this, the Lagrangian is 
\begin{align*}
\lag 
&= \f{1}{2}m \lb \dot{x}^2 + \dot{y}^2 + \dot{z}^2\rb - mgz\\
&= \f{1}{2} am \lb -2g\cos\theta + a\omega^2\sin^2\theta  + a\dot{\theta}^2 \rb.
\end{align*}
There is only one Lagrangian equation of motion:
\begin{equation*}
\f{d}{dt}\lp \f{\p \lag}{\p \dot\theta} \rp = \f{\p \lag}{\p \theta} \implies a \ddot{\theta} =  \lp g + a\omega^2 \cos\theta \rp\sin\theta.
\end{equation*}
It is clear from the functional form of $\lag$ that a constant of motion is $r$ since $r=a$ fixed. Moreover, since $\lag$ is time-independent, energy is conserved, therefore is also a constant of motion. In this case, the energy function is identically the Hamiltonian because the Lagrangian is quadratic in the kinetic part and velocity-independent in the potential part. Thus, we have
\begin{equation*}
\text{const} = \ham = \f{\p \lag}{\p \dot\theta}\dot\theta - \lag = \frac{1}{2} a m \left(a \dot\theta^2-a \omega^2 \sin^2\theta+2 g \cos\theta\right).
\end{equation*}


We now want to find the critical value $\omega_0$ described in the problem statement. Since the particle is stationary, $\dot{\theta} = 0$. The particle is at a stationary point exactly when it is at a local minimum of the effective potential. From the Hamiltonian, we can read off the effective potential as 
\begin{equation*}
V(\theta) = \f{1}{2}am\lp -a\omega^2\sin^2\theta + 2g\cos\theta \rp.
\end{equation*}
We find that 
\begin{align*}
0 = \f{\p V(\theta)}{\p \theta} = \f{1}{2}am\lp -a\omega^2\sin 2\theta -2g\sin\theta \rp = am\sin\theta \lp -a\omega^2\cos\theta - g  \rp 
\end{align*}
so the stationary points are $\theta = 0, \pi$ or $\theta = \arccos(-\omega_0^2/\omega^2)$ where $\boxed{\omega_0 \equiv \sqrt{g/a}}$. Of these three points, $\theta=0$ is always unstable since it for $\theta\approx 0$ $V(\theta)$ looks like $\cos\theta$, so the particle will move away from $\theta=0$. \\

When $\omega_0 \geq  \omega$, the only equilibrium is $\theta=\pi$ since $\p V(\theta)/\p \theta \leq 0$ for all $\theta \in[0,\pi]$. When $\omega_0 \leq \omega$, the equilibrium point becomes $\theta = \theta_0 = \arccos(-\omega_0^2/\omega^2)$ because $\p V/\p \theta \geq 0$ for $\theta \geq \theta_0$.\\


Mathematica code:
\begin{lstlisting}
(*Problem 3*)
In[20]:= KE = (m/2)*(D[x[t], t]^2 + D[y[t], t]^2 + D[z[t], t]^2);
In[21]:= PE = m*g*z[t];
In[22]:= L = KE - PE
Out[22]= -g m z[t] + 
1/2 m (Derivative[1][x][t]^2 + Derivative[1][y][t]^2 + 
Derivative[1][z][t]^2)
(*Lagrangian*)
In[23]:= L = L /. {x[t] -> a*Sin[\[Theta][t]]*Cos[\[Omega]*t], 
y[t] -> a*Sin[\[Theta][t]]*Sin[\[Omega]*t], 
z[t] -> a*Cos[\[Theta][t]],
x'[t] -> D[a*Sin[\[Theta][t]]*Cos[\[Omega]*t], t], 
y'[t] -> D[a*Sin[\[Theta][t]]*Sin[\[Omega]*t], t], 
z'[t] -> D[a*Cos[\[Theta][t]], t]} // FullSimplify
Out[23]= 1/2 a m (-2 g Cos[\[Theta][t]] + a \[Omega]^2 Sin[\[Theta][t]]^2 + 
a Derivative[1][\[Theta]][t]^2)
In[19]:= (*Hamiltonian*)
In[25]:= H = D[L, \[Theta]'[t]]*\[Theta]'[t] - L // FullSimplify
Out[25]= 1/2 a m (2 g Cos[\[Theta][t]] - a \[Omega]^2 Sin[\[Theta][t]]^2 + 
a Derivative[1][\[Theta]][t]^2)
In[26]:= 1/2 a m (2 g Cos[\[Theta][t]] - a \[Omega]^2 Sin[\[Theta][t]]^2 + 
a Derivative[1][\[Theta]][t]^2) // TeXForm
Out[26]//TeXForm=
\frac{1}{2} a m \left(a \theta '(t)^2-a \omega ^2 \sin ^2(\theta
(t))+2 g \cos (\theta (t))\right)
(*Lagrangian EOM*)
In[88]:= FullSimplify[D[D[L, \[Theta]'[t]], t] == D[L, \[Theta][t]]]
Out[88]= a^2 m (\[Theta]^\[Prime]\[Prime])[t] == 
a m (g + a \[Omega]^2 Cos[\[Theta][t]]) Sin[\[Theta][t]]
\end{lstlisting}







\newpage
\noindent \textbf{4.} \textbf{Spring System on a Plane}

\begin{enumerate}[label = (\alph*)]
	\item The Lagrangian in Cartesian coordinates is 
	\begin{equation*}
	\boxed{
	\lag = \f{1}{2}m_1(\dot{x}_1^2 + \dot{y}_1^2)+ \f{1}{2}m_2(\dot{x}_2^2 + \dot{y}_2^2) - \f{1}{2}k \lp \sqrt{(x_1-x_2)^2+(y_1-y_2)^2} -b \rp^2
	}
	\end{equation*}
	
	
	
	
	\item New coordinates are  $(x_\text{COM},y_{\text{COM}},r,\theta)$, where $(x_\text{COM}, y_\text{COM})$ describes the position of the center of mass of the system, while $(r,\theta)$ together describe the relative position vector of the two masses. These new coordinates are defined by 
	\begin{align*}
	x_\text{COM} = \f{m_1 x_1 + m_2 x_2}{m_1+m_2}; \quad y_\text{COM} = \f{m_1 y_1 + m_2 y_2}{m_1+m_2}; \quad r= \sqrt{(x_1-x_2)^2+(y_1-y_2)^2}; \quad \theta = \arctan\f{y_2-y_1}{x_2-x_1}
	\end{align*}
	
	
	The Lagrangian in these coordinates is:
	\begin{align*}
	\lag = -\f{k}{2} (b-r)^2+\frac{m_1m_2 \left(\dot r^2+r^2 \dot\theta^2\right)}{2(m_1+m_2)}
	+\frac{1}{2}(m_1+m_2) \left(\dot{x}_\text{COM}^2+\dot{y}_\text{COM}^2\right)
	\end{align*}
	
	The equations of motion are:
	\begin{align*}
	\f{d}{dt}\f{\p \lag}{\p \dot{x_\text{COM}}} = \f{\p \lag}{\p x_\text{COM}}
	\implies \ddot{x}_\text{COM} = 0
	\end{align*}
	\begin{align*}
	\f{d}{dt}\f{\p \lag}{\p \dot{y_\text{COM}}} = \f{\p \lag}{\p y_\text{COM}} \implies \ddot{y}_\text{COM} = 0
	\end{align*}
	\begin{align*}
	\f{d}{dt}\f{\p \lag}{\p \dot{r}} = \f{\p \lag}{\p r}
	\implies \ddot{r} = \f{k(m_1+m_2)(b-r)}{m_1m_2} + r\dot{\theta}^2
	\end{align*}
	\begin{align*}
	\f{d}{dt}\f{\p \lag}{\p \dot{\theta}} = \f{\p \lag}{\p \theta}\implies \ddot{\theta}  = -\f{2\dot{r}\dot{\theta}}{r}
	\end{align*}

	
	\item The conserved generalized momenta are: ...... Show that there is a solution that rotates but does not oscillate, and discuss what happens to this solution when rotates faster...
	
	The cyclic coordinates are $x_\text{COM}$, $y_\text{COM}$, and $\theta$, since $\lag$ does not explicitly depend on them. The conserved generalized momenta are thus
	\begin{equation*}
	p_{x_\text{COM}} = \f{\p \lag}{\p \dot{x_\text{COM}}} = (m_1+m_2)\dot{x}_\text{COM}; \quad\quad p_{y_\text{COM}} = \f{\p \lag}{\p \dot{y_\text{COM}}} = (m_1+m_2)\dot{y}_\text{COM}; \quad\quad
	p_\theta = \f{\p \lag}{\p \dot{\theta}} = \f{m_1m_2}{m_1+m_2} r^2\dot{\theta}.
	\end{equation*}
 
	
	
	
	\item Mathematica code:
	\begin{lstlisting}
	(*Problem 4*)
	
	In[50]:= KE = (1/2)*(m1 + m2)*(D[xCOM[t], t]^2 + 
	D[yCOM[t], t]^2) + (1/2)*(m1*
	m2)*(D[r[t], t]^2 + r[t]^2*D[\[Theta][t], t]^2)/(m1 + m2);
	
	In[51]:= PE = (k/2)*(r[t] - b)^2;
	
	In[52]:= L = KE - PE;
	
	(*Lagrangian*)
	
	In[53]:= L // FullSimplify
	
	Out[53]= 1/2 (-k (b - r[t])^2 + (m1 + m2) (Derivative[1][xCOM][t]^2 + 
	Derivative[1][yCOM][t]^2) + (
	m1 m2 (Derivative[1][r][t]^2 + 
	r[t]^2 Derivative[1][\[Theta]][t]^2))/(m1 + m2))
	
	(*Lagrangian EOMs*)
	
	(*xCOM equation*)
	
	In[54]:= Solve[FullSimplify[D[D[L, xCOM'[t]], t] == D[L, xCOM[t]]], 
	xCOM''[t]][[1]] // Expand
	
	Out[54]= {(xCOM^\[Prime]\[Prime])[t] -> 0}
	
	(*yCOM equation*)
	
	In[55]:= Solve[FullSimplify[D[D[L, yCOM'[t]], t] == D[L, yCOM[t]]], 
	yCOM''[t]][[1]] // Expand
	
	Out[55]= {(yCOM^\[Prime]\[Prime])[t] -> 0}
	
	(*r equation*)
	
	In[65]:= Solve[FullSimplify[D[D[L, r'[t]], t] == D[L, r[t]]], 
	r''[t]][[1]] // FullSimplify
	
	Out[65]= {(r^\[Prime]\[Prime])[t] -> (k (m1 + m2) (b - r[t]))/(
	m1 m2) + r[t] Derivative[1][\[Theta]][t]^2}
	
	In[15]:= (*theta equation*)
	
	In[57]:= Solve[
	FullSimplify[
	D[D[L, \[Theta]'[t]], t] == D[L, \[Theta][t]]], \[Theta]''[t]][[
	1]] // Expand
	
	Out[57]= {(\[Theta]^\[Prime]\[Prime])[t] -> -((
	2 Derivative[1][r][t] Derivative[1][\[Theta]][t])/r[t])}
	
	(*Cyclic coordinates*)
	
	In[58]:= D[L, xCOM[t]]
	
	Out[58]= 0
	
	In[59]:= D[L, yCOM[t]]
	
	Out[59]= 0
	
	In[60]:= D[L, \[Theta][t]]
	
	Out[60]= 0
	
	(*Conserved generalized momenta*)
	
	In[61]:= D[L, xCOM'[t]] // Simplify
	
	Out[61]= (m1 + m2) Derivative[1][xCOM][t]
	
	In[62]:= D[L, yCOM'[t]] // Simplify
	
	Out[62]= (m1 + m2) Derivative[1][yCOM][t]
	
	In[63]:= D[L, \[Theta]'[t]] // Simplify
	
	Out[63]= (m1 m2 r[t]^2 Derivative[1][\[Theta]][t])/(m1 + m2)
	\end{lstlisting}
\end{enumerate}











\newpage
\noindent \textbf{6.} \textbf{Routhian Mechanics}\\


\noindent We start with the definition of a particular Routhian:
\begin{equation*}
R(q_1,\dots,q_n, p_1,\dots,p_s, \dot{q}_{s+1}, \dots, \dot{q}_{n},t) = \sum^s_{k=1} p_k \dot{q}_k  - \lag(q_1,\dots,q_n, \dot{q}_1,\dots \dot{q}_n,t).
\end{equation*}
For $i = 1,\dots,n$ we have
\begin{equation*}
\boxed{\f{\p R}{\p q_i}} = -\f{\p \lag}{\p q_i} = -\f{d}{dt} \lp \f{\p \lag}{\p \dot{q}_i} \rp = -\f{d}{dt}p_i = \boxed{-\dot{p}_i}
\end{equation*}
For $i=1,\dots,s$, we have
\begin{equation*}
\boxed{\f{\p R}{\p p_i} = \dot{q}_i}
\end{equation*}
For $i = s+1,\dots,n$, we have 
\begin{equation*}
\boxed{\f{\p R}{\p \dot{q}_i}} = -\f{\p \lag}{\p \dot{q}_i} = \boxed{-p_i} \implies \boxed{\f{d}{dt}\f{\p R}{\p \dot{q}_i}} = -\f{d}{dt}\lp \f{\p \lag}{\p \dot{q}_i} \rp = -\f{d}{dt} p_i = \boxed{\f{\p R}{\p q_i}}
\end{equation*}
Finally, 
\begin{equation*}
\f{\p R}{\p t} = -\f{\p \lag}{\p t}.
\end{equation*}

\newpage
\noindent \textbf{7.} \textbf{Extra Problem: Equivalent Lagrangians}\\


\noindent We claim that 
\begin{equation*}
\lag' = \lag + \f{dF(q,t)}{dt}
\end{equation*}
where $F$ is a differential function, satisfies the Euler-Lagrange equations.


\begin{proof}
	On the one hand, 
	\begin{equation*}
	\f{d}{dt} \f{\p \lag'}{\p \dot{q}}  =
	\f{d}{dt} \f{\p \lag}{\p \dot{q}}  + \f{d}{dt} \f{\p }{\p \dot{q}} \f{d F}{d t} =  \f{d}{dt} \f{\p \lag}{\p \dot{q}}  
	+ \f{d}{dt}\f{\p F}{\p q} 
	= \f{d}{dt} \f{\p \lag}{\p \dot{q}} 
	+ \f{\p}{\p q}\lp \f{\p F}{\p q} \rp \dot{q}
	+ \f{\p^2 F}{\p t \p q},
	\end{equation*}
	where we have used the fact that $F$ does not depend on $\dot{q}$ for the second equality (i.e., $d F/d t$ generates a factor of $\dot{q}$, which gets taken away by $\p/\p \dot{q}$). \\
	
	On the other hand,
	\begin{equation*}
	\f{\p \lag'}{\p q}  = \f{\p \lag }{\p q} +  \f{\p}{\p q} \f{dF}{dt} = \f{\p \lag }{\p q} +  \f{\p}{\p q} \lp \f{\p F}{\p q}\rp\dot{q} + \f{\p^2 F}{\p q\p t}.
	\end{equation*}
	Since $\p^2 F/\p q\p t = \p^2 F/\p t \p q$ and that $\lag$ satisfies the Euler-Lagrange equations, we see that 
	\begin{equation*}
	\f{d}{dt}\f{\p \lag'}{\p \dot{q}} = \f{\p \lag'}{\p q}
	\end{equation*}
	as desired.
\end{proof} 

	
	
	
	
	
	
	
	
	
	
	
	
	
	
	
\end{document}



