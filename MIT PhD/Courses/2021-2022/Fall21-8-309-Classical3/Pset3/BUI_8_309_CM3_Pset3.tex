\documentclass{article}
\usepackage{physics}
\usepackage{graphicx}
\usepackage{caption}
\usepackage{amsmath}
\usepackage{bm}
\usepackage{framed}
\usepackage{authblk}
\usepackage{empheq}
\usepackage{amsfonts}
\usepackage{esint}
\usepackage[makeroom]{cancel}
\usepackage{dsfont}
\usepackage{centernot}
\usepackage{mathtools}
\usepackage{bigints}
\usepackage{amsthm}
\theoremstyle{definition}
\newtheorem{defn}{Definition}[section]
\newtheorem{prop}{Proposition}[section]
\newtheorem{rmk}{Remark}[section]
\newtheorem{thm}{Theorem}[section]
\newtheorem{exmp}{Example}[section]
\newtheorem{prob}{Problem}[section]
\newtheorem{sln}{Solution}[section]
\newtheorem*{prob*}{Problem}
\newtheorem{exer}{Exercise}[section]
\newtheorem*{exer*}{Exercise}
\newtheorem*{sln*}{Solution}
\usepackage{empheq}
\usepackage{tensor}
\usepackage{xcolor}
%\definecolor{colby}{rgb}{0.0, 0.0, 0.5}
\definecolor{MIT}{RGB}{163, 31, 52}
\usepackage[pdftex]{hyperref}
%\hypersetup{colorlinks,urlcolor=colby}
\hypersetup{colorlinks,linkcolor={MIT},citecolor={MIT},urlcolor={MIT}}  
\usepackage[left=1in,right=1in,top=1in,bottom=1in]{geometry}

\usepackage{newpxtext,newpxmath}
\newcommand*\widefbox[1]{\fbox{\hspace{2em}#1\hspace{2em}}}

\newcommand{\p}{\partial}
\newcommand{\R}{\mathbb{R}}
\newcommand{\C}{\mathbb{C}}
\newcommand{\lag}{\mathcal{L}}
\newcommand{\nn}{\nonumber}
\newcommand{\ham}{\mathcal{H}}
\newcommand{\M}{\mathcal{M}}
\newcommand{\I}{\mathcal{I}}
\newcommand{\K}{\mathcal{K}}
\newcommand{\F}{\mathcal{F}}
\newcommand{\w}{\omega}
\newcommand{\lam}{\lambda}
\newcommand{\al}{\alpha}
\newcommand{\be}{\beta}
\newcommand{\x}{\xi}

\newcommand{\G}{\mathcal{G}}

\newcommand{\f}[2]{\frac{#1}{#2}}

\newcommand{\ift}{\infty}

\newcommand{\lp}{\left(}
\newcommand{\rp}{\right)}

\newcommand{\lb}{\left[}
\newcommand{\rb}{\right]}

\newcommand{\lc}{\left\{}
\newcommand{\rc}{\right\}}


\newcommand{\V}{\mathbf{V}}
\newcommand{\U}{\mathcal{U}}
\newcommand{\Id}{\mathcal{I}}
\newcommand{\D}{\mathcal{D}}
\newcommand{\Z}{\mathcal{Z}}

%\setcounter{chapter}{-1}


\usepackage{enumitem}



\usepackage{subfig}
\usepackage{listings}
\captionsetup[lstlisting]{margin=0cm,format=hang,font=small,format=plain,labelfont={bf,up},textfont={it}}
\renewcommand*{\lstlistingname}{Code \textcolor{violet}{\textsl{Mathematica}}}
\definecolor{gris245}{RGB}{245,245,245}
\definecolor{olive}{RGB}{50,140,50}
\definecolor{brun}{RGB}{175,100,80}

%\hypersetup{colorlinks,urlcolor=colby}
\lstset{
	tabsize=4,
	frame=single,
	language=mathematica,
	basicstyle=\scriptsize\ttfamily,
	keywordstyle=\color{black},
	backgroundcolor=\color{gris245},
	commentstyle=\color{gray},
	showstringspaces=false,
	emph={
		r1,
		r2,
		epsilon,epsilon_,
		Newton,Newton_
	},emphstyle={\color{olive}},
	emph={[2]
		L,
		CouleurCourbe,
		PotentielEffectif,
		IdCourbe,
		Courbe
	},emphstyle={[2]\color{blue}},
	emph={[3]r,r_,n,n_},emphstyle={[3]\color{magenta}}
}






\begin{document}
\begin{framed}
	\noindent Name: \textbf{Huan Q. Bui}\\
	Course: \textbf{8.309 - Classical Mechanics III}\\
	Problem set: \textbf{\#3}
\end{framed}
	
	


\noindent \textbf{1. Rotation Angle in the Euler Theorem}


\begin{enumerate}[label=(\alph*)]
	\item Suppose $\vec{\xi}_1 = \vec{\xi}_2^* = (p+iq \quad m+in)^\top$ is normalized ($p,q,m,n\in \mathbb{R}$), then the combinations
	\begin{align*}
	\boxed{\vec{\xi}_a = \f{1}{\sqrt{2}}\lb \vec{\xi}_1 + \vec{\xi}_2  \rb \quad\quad\quad 
	\vec{\xi}_b = \f{i}{\sqrt{2}}\lb \vec{\xi}_1 - \vec{\xi}_2 \rb}
	\end{align*}
	are what we want. First,  $\xi_a \perp \xi_b$ by inspection and both are orthogonal to $\xi_3$ because they belong to the subspace spanned by $\vec{\xi}_1$ and $\vec{\xi}_2$ which is orthogonal to $\vec{\xi}_3$. Second, it is clear that both $\vec{\xi}_1$ and $\vec{\xi}_2$ are normalized. Finally, $\vec{\xi}_a$ and $\vec{\xi}_b$ are real because $\vec{\xi}_1 = \vec{\xi}_2^*$. We see that the components of $\vec{\xi}_a$ only involve the real part of $\vec{\xi}_1$ and $\vec{\xi}_2$, whereas the components of $\vec{\xi_b}$ only involve the imaginary parts of $\vec{\xi}_1$ and $\vec{\xi}_2$. 
	
	
	\item The $(1\quad 0 \quad 0)^\top$ vector in the $\{\vec{\xi}_1, \vec{\xi}_2, \vec{\xi}_3\}$ basis appears as $(1/\sqrt{2}) (1\quad -i\quad 0)$ in the $\{\vec{\xi}_a, \vec{\xi}_b, \vec{\xi}_3\}$ basis. Likewise, the $(0\quad 1 \quad 0)^\top$ vector in the $\{\vec{\xi}_1, \vec{\xi}_2, \vec{\xi}_3\}$ basis appears as $(1/\sqrt{2}) (1\quad i\quad 0)$ in the $\{\vec{\xi}_a, \vec{\xi}_b, \vec{\xi}_3\}$ basis. Therefore, the matrix which connects the components in the former basis to the latter is 
	\begin{align*}
	\boxed{W = \f{1}{\sqrt{2}}\begin{pmatrix}
	1 & 1 & 0 \\ -i & i & 0 \\ 0 & 0 & \sqrt{2}
	\end{pmatrix}}
	\end{align*}
	By inspection, we see that all columns of $W$ are mutually orthogonal with unit norm (with respect to the inner product for complex-valued vectors). So, $W$ is unitary. Alternatively, we can explicitly calculate that $W^\dagger = W^{-1}$. 
	
	
	
	\item With $\vec{u} = W \vec{s} = W X^\dagger \vec{r}$, we find
	\begin{align*}
	\vec{u}' = WX^\dagger \vec{r}' = W X^\dagger U \vec{r} = WX^\dagger U X W^\dagger \vec{u} \eqcolon \widetilde{U}\vec{u}.
	\end{align*}
	So we have
	\begin{align*}
	\widetilde{U} = \f{1}{2} \begin{pmatrix}
	1 & 1 & 0 \\ -i & i & 0 \\ 0 & 0 & \sqrt{2}
	\end{pmatrix}    
	\begin{pmatrix}
	e^{i\Phi} && \\ &e^{-i\Phi}& \\ && 1
	\end{pmatrix}
	\begin{pmatrix}
	1 & i & 0\\
	1 & -i & 0 \\
	0 & 0 & \sqrt{2}
	\end{pmatrix}
	= \begin{pmatrix*}
	\cos\Phi & -\sin\Phi & \\ \sin\Phi & \cos\Phi & \\ 0 & 0& 1
	\end{pmatrix*}.
	\end{align*}
	The matrix $\widetilde{U}$ has the form of a standard rotation matrix with rotation angle $\Phi$, as desired. Note that defining the sign of $\vec{\xi}_b$ in Part (b) will impact whether the only $(-)$ sign in the matrix $\widetilde{U}$ goes on the $(12)$ or $(21)$ element. The answers in both cases are, of course, equivalent. 
	
	
	
	\item Mathematica code:
	\begin{lstlisting}
	(*Problem 1*)
	
	In[7]:= W = (1/Sqrt[2])*{{1, 1, 0}, {-I, I, 0}, {0, 0, Sqrt[2]}};
	
	In[8]:= ConjugateTranspose[W] - Inverse[W]
	
	Out[8]= {{0, 0, 0}, {0, 0, 0}, {0, 0, 0}}
	
	In[10]:= ExpToTrig[
	W . {{E^(I*\[Phi]), 0, 0}, {0, E^(-I*\[Phi]), 0}, {0, 0, 1}} . ConjugateTranspose[W]]
	
	Out[10]= {{Cos[\[Phi]], -Sin[\[Phi]], 0}, {Sin[\[Phi]], Cos[\[Phi]], 0}, {0, 0, 1}}
	\end{lstlisting}
\end{enumerate} 



\newpage







\noindent \textbf{2. Foucault Pendulum and the Coriolis Effect}

\begin{enumerate}[label=(\alph*)]
	\item The Lagrangian is 
	\begin{align*}
	\lag = \f{m}{2}\lb \vec{v} - \vec{\omega} \times (\vec{r} + R_e \hat{z})  \rb^2 -V
	\end{align*}
	where $\vec{v}$ is velocity in the rotating frame. According the definition of the spherical coordinates in this problem, we have
	\begin{align*}
	\vec{r} = \begin{pmatrix}
	x & y & z
	\end{pmatrix}^\top
	= 
	\begin{pmatrix}
	l\sin\theta\sin\phi & 
	l\sin\theta\cos\phi &
	-l\cos\theta
	\end{pmatrix}^\top
	\end{align*}
	since $\theta$ is now defined as the angle from the $-\hat{z}$. In these coordinates, the potential energy is $V = -mgl\cos\theta$. The relevant vectors in this problem apart from $\vec{r}$ are
	\begin{align*}
	\vec{v} = \f{d}{dt}\begin{pmatrix}
	x\\y\\z
	\end{pmatrix} = 
	l\begin{pmatrix}
	\dot\theta \cos \theta \sin\phi+\sin\theta\dot\phi\cos\phi \\
	\dot\theta \cos\theta \cos\phi-\sin
	\theta \dot\phi \sin\phi\\
	-\dot\theta \sin \theta
	\end{pmatrix}
	\quad \quad \text{and} \quad\quad
	\vec{\omega} = \begin{pmatrix}
	0 \\ \omega \cos\lambda \\ \omega\sin\lambda
	\end{pmatrix}
	\end{align*}
	We can now plug everything into Mathematica to obtain the full Lagrangian. Under the approximation that $\omega R_e \to 0$ and $\omega^n \to 0$ for $n > 1$, we find the approximate Lagrangian to be \textcolor{red}{This Lagrangian is wrong. Check Mathematica file for correct version. I initially defined the $z$-component of $\vec{v}$ with the wrong sign which led to a wrong Lagrangian.}
	\begin{align*}
	\lag = \frac{1}{2} l m \left(2 g \cos\theta+l \left(-2 \omega  \cos\lambda  \dot\theta
	\cos 2 \theta \sin \phi +\dot\theta^2+\dot\phi \left(\sin^2\theta\left(\dot\phi-2 \omega  \sin \lambda \right)-\omega  \cos \lambda  \sin 2\theta  \cos \phi\right)\right)\right).
	\end{align*}
	The equations of motion are:
	\begin{align*}
	&\f{d}{dt} \f{\p \lag}{\p \dot{\theta}} = \f{\p \lag}{\p \theta} \implies 
	\boxed{\ddot\theta = \f{-\sin\theta }{l}\lb g+ l\cos\theta(2\omega\sin\lambda - \dot\phi)\dot\phi  \rb}\\
	&\f{d}{dt} \f{\p \lag}{\p \dot\phi} = \f{\p \lag}{\p \phi} \implies \boxed{\ddot\phi = 2\cot\theta \dot\theta \lp \omega\sin\lambda - \dot\phi \rp }
	\end{align*}
	
	
	\item Consider the small angle approximation where $l$ is large, and $\theta$ is small. Then $\sin\theta \to \theta, \cos\theta \to 1$. Moreover, we will assume that locally the pendulum undergoes simple harmonic oscillation and therefore $\ddot\theta \theta \sim \theta^2 \to 0$. With these, the equations of motion become
	\begin{align*}
	\boxed{\ddot\theta = \f{-\theta  }{l}\lb g+ l\dot\phi (2\omega\sin\lambda - \dot\phi) \rb} 
	\quad \quad 
	\boxed{\ddot\phi = \f{2\dot\theta }{\theta} \lb \omega\sin\lambda - \dot\phi \rb }
	\end{align*}
	We notice that the first equation is a quadratic equation in $\dot\phi$. Solving it gives
	\begin{align*}
	\dot\phi_\pm = \omega \sin\lambda \pm \sqrt{\f{g}{l}}.
	\end{align*}
	Taking $l$ to be arbitrarily large, we find the desired expression
	\begin{align*}
	\boxed{\dot\phi = \omega\sin\lambda}
	\end{align*}
	
	
	\item Mathematica code:
	\begin{lstlisting}
	In[1]:= (*Problem 2*)
	
	In[1]:= SphericalCoords = {x[t], y[t], 
	z[t]} -> {L*Sin[\[Theta][t]]*Sin[\[Phi][t]], 
	L*Sin[\[Theta][t]]*Cos[\[Phi][t]], -L*Cos[\[Theta][t]]};
	
	In[3]:= SphericalDerivs = {D[x[t], t], D[y[t], t], 
	D[z[t], t]} -> {D[L*Sin[\[Theta][t]]*Sin[\[Phi][t]], t], 
	D[L*Sin[\[Theta][t]]*Cos[\[Phi][t]], t], D[L*Cos[\[Theta][t]], t]};
	
	In[4]:= D[{x[t], y[t], z[t]}, t] /. SphericalDerivs // FullSimplify
	
	Out[4]= {L (Cos[\[Theta][t]] Sin[\[Phi][t]] Derivative[1][\[Theta]][
	t] + Cos[\[Phi][t]] Sin[\[Theta][t]] Derivative[1][\[Phi]][t]), 
	L (Cos[\[Theta][t]] Cos[\[Phi][t]] Derivative[1][\[Theta]][t] - 
	Sin[\[Theta][t]] Sin[\[Phi][t]] Derivative[1][\[Phi]][
	t]), -L Sin[\[Theta][t]] Derivative[1][\[Theta]][t]}
	
	In[5]:= (*Unit vectors in Cartesian*)
	
	In[5]:= zHat = {0, 0, 1};
	
	In[6]:= xHat = {1, 0, 0};
	
	In[7]:= yHat = {0, 1, 0};
	
	In[9]:= (*Relevant vectors*)
	
	In[8]:= rVec = {L*Sin[\[Theta][t]]*Sin[\[Phi][t]], 
	L*Sin[\[Theta][t]]*Cos[\[Phi][t]], -L*Cos[\[Theta][t]]};
	
	In[9]:= \[Omega]Vec = {0, \[Omega]*Cos[\[Lambda]], \[Omega]*
	Sin[\[Lambda]]};
	
	In[10]:= vVec = {D[x[t], t], D[y[t], t], D[z[t], t]} /. 
	SphericalDerivs;
	
	In[11]:= wrRz = Cross[\[Omega]Vec, (rVec + RE*zHat)];
	
	In[14]:= (*Potential*)
	
	In[12]:= Pot = -m*g*L*Cos[\[Theta][t]] /. SphericalCoords;
	
	In[16]:= (*Lagrangian*)
	
	In[13]:= Lag = (m/2)*(Dot[vVec, vVec] + 2*Dot[vVec, wrRz] + 
	Dot[wrRz, wrRz]) - Pot // FullSimplify
	
	Out[13]= 1/2 m (2 g L Cos[\[Theta][
	t]] + \[Omega]^2 ((Cos[\[Lambda]] (RE - L Cos[\[Theta][t]]) - 
	L Cos[\[Phi][t]] Sin[\[Lambda]] Sin[\[Theta][t]])^2 + 
	L^2 Sin[\[Theta][t]]^2 Sin[\[Phi][t]]^2) + 
	2 L \[Omega] Cos[\[Lambda]] (RE Cos[\[Theta][t]] - 
	L Cos[2 \[Theta][t]]) Sin[\[Phi][t]] Derivative[1][\[Theta]][
	t] + L^2 Derivative[1][\[Theta]][t]^2 - 
	2 L \[Omega] Sin[\[Theta][
	t]] (Cos[\[Lambda]] (-RE + L Cos[\[Theta][t]]) Cos[\[Phi][t]] + 
	L Sin[\[Lambda]] Sin[\[Theta][t]]) Derivative[1][\[Phi]][t] + 
	L^2 Sin[\[Theta][t]]^2 Derivative[1][\[Phi]][t]^2)
	
	In[14]:= Approx = {\[Omega]*RE -> 0, 
	RE*\[Omega] -> 0, \[Omega]^2 -> 0};
	
	In[15]:= Lag = Lag /. Approx // FullSimplify
	
	Out[15]= 1/2 L m (2 g Cos[\[Theta][t]] + 
	2 \[Omega] Cos[\[Lambda]] (RE Cos[\[Theta][t]] - 
	L Cos[2 \[Theta][t]]) Sin[\[Phi][t]] Derivative[1][\[Theta]][
	t] + L Derivative[1][\[Theta]][t]^2 + 
	Sin[\[Theta][t]] Derivative[1][\[Phi]][
	t] (2 \[Omega] Cos[\[Lambda]] (RE - 
	L Cos[\[Theta][t]]) Cos[\[Phi][t]] + 
	L Sin[\[Theta][t]] (-2 \[Omega] Sin[\[Lambda]] + 
	Derivative[1][\[Phi]][t])))
	
	In[20]:= (*EOMs*)
	
	In[21]:= (*\[Theta] equation*)
	
	In[16]:= Eqn\[Theta] = 
	FullSimplify[D[D[Lag, \[Theta]'[t]], t] == D[Lag, \[Theta][t]]]
	
	Out[16]= L m (Sin[\[Theta][t]] (g + 
	L Cos[\[Theta][t]] (2 \[Omega] Sin[\[Lambda]] - 
	Derivative[1][\[Phi]][t]) Derivative[1][\[Phi]][t]) + 
	L (\[Theta]^\[Prime]\[Prime])[t]) == 0
	
	In[23]:= (*\[Phi] equation*)
	
	In[17]:= Eqn\[Phi] = 
	FullSimplify[D[D[Lag, \[Phi]'[t]], t] == D[Lag, \[Phi][t]]]
	
	Out[17]= L m (Sin[2 \[Theta][t]] Derivative[1][\[Theta]][
	t] (\[Omega] Sin[\[Lambda]] - Derivative[1][\[Phi]][t]) - 
	Sin[\[Theta][t]]^2 (\[Phi]^\[Prime]\[Prime])[t]) == 0
	
	In[18]:= Solve[Eqn\[Theta], \[Theta]''[t]] // FullSimplify
	
	Out[18]= {{(\[Theta]^\[Prime]\[Prime])[t] -> -((
	Sin[\[Theta][t]] (g + 
	L Cos[\[Theta][t]] (2 \[Omega] Sin[\[Lambda]] - 
	Derivative[1][\[Phi]][t]) Derivative[1][\[Phi]][t]))/L)}}
	
	In[19]:= Solve[Eqn\[Phi], \[Phi]''[t]] // FullSimplify
	
	Out[19]= {{(\[Phi]^\[Prime]\[Prime])[t] -> 
	2 Cot[\[Theta][t]] Derivative[1][\[Theta]][
	t] (\[Omega] Sin[\[Lambda]] - Derivative[1][\[Phi]][t])}}
	
	In[20]:= SmallAngle = {Sin[\[Theta][t]] -> \[Theta][t], 
	Cos[\[Theta][t]] -> 1, 
	Sin[2 \[Theta][t]] -> 2*\[Theta][t], \[Theta][t]*\[Theta]''[t] -> 
	0};
	
	In[21]:= Eqn\[Theta] = Eqn\[Theta] /. SmallAngle /. Approx
	
	Out[21]= L m (\[Theta][
	t] (g + L (2 \[Omega] Sin[\[Lambda]] - 
	Derivative[1][\[Phi]][t]) Derivative[1][\[Phi]][t]) + 
	L (\[Theta]^\[Prime]\[Prime])[t]) == 0
	
	In[22]:= Eqn\[Phi] = Eqn\[Phi] /. SmallAngle /. Approx
	
	Out[22]= L m (2 \[Theta][t] Derivative[1][\[Theta]][
	t] (\[Omega] Sin[\[Lambda]] - 
	Derivative[1][\[Phi]][t]) - \[Theta][
	t]^2 (\[Phi]^\[Prime]\[Prime])[t]) == 0
	
	In[23]:= Solve[Eqn\[Theta], \[Theta]''[t]] // FullSimplify
	
	Out[23]= {{(\[Theta]^\[Prime]\[Prime])[
	t] -> -((\[Theta][
	t] (g + L (2 \[Omega] Sin[\[Lambda]] - 
	Derivative[1][\[Phi]][t]) Derivative[1][\[Phi]][t]))/L)}}
	
	In[25]:= Solve[Eqn\[Phi], \[Phi]''[t]] // FullSimplify
	
	Out[25]= {{(\[Phi]^\[Prime]\[Prime])[t] -> (
	2 Derivative[1][\[Theta]][
	t] (\[Omega] Sin[\[Lambda]] - 
	Derivative[1][\[Phi]][t]))/\[Theta][t]}}
	
	In[28]:= (*Equation for \[Phi]'[t]. When L -> \[Infinity] then we get \
	desired result*)
	
	In[26]:= FullSimplify[
	Solve[Eqn\[Theta], \[Phi]'[t]] /. SmallAngle /. Approx // 
	Expand, {L > 0, \[Theta][t] > 0, g > 0}]
	
	Out[26]= {{Derivative[1][\[Phi]][
	t] -> -(g/Sqrt[g L]) + \[Omega] Sin[\[Lambda]]}, {Derivative[
	1][\[Phi]][t] -> Sqrt[g/L] + \[Omega] Sin[\[Lambda]]}}
	\end{lstlisting}
\end{enumerate}





\noindent \textbf{3. Angular Velocity with Euler Angles}



\begin{enumerate}[label=(\alph*)]
	\item Since $\vec{\omega}_\phi$ is parallel to the space $z$-axis, it has the form $\vec{\omega}_\phi = (0\quad 0 \quad \dot\phi)^\top$ in the space-basis. To go from the space basis to the body basis we simply apply the full orthogonal transformation $A = BCD$. The result is simply the third column of $A$, mutiplied by $\dot\phi$:
	\begin{align*}
	\begin{pmatrix}
	(\omega_\phi)_{x'} \\
	(\omega_\phi)_{y'} \\
	(\omega_\phi)_{z'} 
	\end{pmatrix}
	= \begin{pmatrix}
	\dot\phi \sin\theta \sin\psi \\ \dot\phi \sin\theta \cos\psi  \\ \dot\phi \cos\theta
	\end{pmatrix}.
	\end{align*}
	Next, since $\vec{\omega}_\theta$ is in line with the $\xi'$ axis (which is the space $x$-axis after the orthogonal transformations $D$ and $C$), we only need to apply the final orthogonal transformation $B$ to $CD\vec{\omega}_\theta = (\dot\theta \quad 0 \quad 0)^\top$. This give the first column of $B$, multiplied by $\dot\theta$:
	\begin{align*}
	\begin{pmatrix}
	(\omega_\theta)_{x'} \\
	(\omega_\theta)_{y'} \\
	(\omega_\theta)_{z'} 
	\end{pmatrix}
	= \begin{pmatrix}
	\dot\theta\cos\psi \\ -\dot\theta\sin\psi \\ 0
	\end{pmatrix}.
	\end{align*}
	Finally, $\vec{\omega}_\psi$ lies along the $z'$-axis, so no transformation is necessary. 
	\begin{align*}
	\begin{pmatrix}
	(\omega_\psi)_{x'} \\
	(\omega_\psi)_{y'} \\
	(\omega_\psi)_{z'} 
	\end{pmatrix} = \begin{pmatrix}
	0 \\ 0 \\ \dot\psi
	\end{pmatrix}
	\end{align*}
	Adding everything together, we find 
	\begin{align*}
	\omega_{x'} &= \dot\phi \sin\theta\sin\psi + \dot\theta \cos\psi  \\
	\omega_{y'} &= \dot\phi \sin\theta \cos\psi - \dot\theta \sin\psi \\
	\omega_{z'} &= \dot\phi \cos\theta + \dot\psi
	\end{align*}
	
	
	\item From the previous part, we have $\vec{\omega'} = (\omega_{x'} \quad \omega_{y'} \quad \omega_{z'})$. To transform these into $\vec{\omega} = (\omega_{x} \quad \omega_{y} \quad \omega_{z})$ we simply apply $A^{-1}$, since $\vec{\omega'} = A \vec{\omega}$. To reduce our chance of making a mistake, we may write $A^{-1} = (BCD)^{-1} = D^{-1}C^{-1} B^{-1}$ where
	\begin{align*}
	D = \begin{pmatrix}
	\cos\phi & \sin\phi & 0\\
	-\sin\phi & \cos\phi & 0\\
	0 &0 & 1
	\end{pmatrix} \quad\quad 
	C= \begin{pmatrix}
	1 & 0 & 0 \\
	0 & \cos\theta&\sin\theta \\
	0 & -\sin\theta  & \cos\theta
	\end{pmatrix}
	\quad\quad
	B = \begin{pmatrix}
	\cos\psi & \sin\psi & 0 \\
	-\sin\psi & \cos\psi & 0 \\
	0 & 0 & 1
	\end{pmatrix}
	\end{align*} 
	Plugging everything we find that 
	\begin{align*}
	\vec{\omega} = \begin{pmatrix}
	\omega_x \\ \omega_y \\ \omega_z 
	\end{pmatrix}
	= 
	D^{-1}C^{-1} B^{-1}
	\begin{pmatrix}
	\omega_{x'} \\
	\omega_{y'} \\
	\omega_{z'} 
	\end{pmatrix} = 
	\begin{pmatrix}
	\dot\theta \cos\phi + \dot\psi \sin\theta\sin\phi  \\ 
	\dot\theta \sin\phi - \dot\psi \sin\theta\cos\phi  \\ 
	\dot\psi \cos\theta  + \dot\phi
	\end{pmatrix},
	\end{align*}
	as desired.
	
	\item Working in the body axes, the moment of inertia tensor is diagonal $\hat I  = \text{diag}(I_{x'}, I_{y'} , I_{z'})$. The kinetic energy is then simply 
	\begin{align*}
	T &= \f{1}{2} \vec{\omega'}^\top \hat{I} \vec{\omega}\\
	&= \f{1}{2} \lb I_{x'} \omega_{x'}^2 + I_{y'}\omega_{y'}^2 + I_{z'}\omega_{z'}^2  \rb\\
	&= \f{1}{2}\lb I_{x'}(\cos\psi \dot\theta + \sin\theta \sin\psi \dot\phi)^2  + 
	I_{y'} (\sin\psi \dot\theta - \cos\psi \sin\theta \dot\phi)^2 + 
	I_{z'}(\cos\theta \dot\phi + \dot\psi)^2\rb.
	\end{align*}
	Using the generalied coordinate $\psi$, the Euler-Lagrange equation in the problem becomes 
	\begin{align*}
	I_{z'} \lb -\sin\theta \dot\theta \dot\phi+ \cos\theta \ddot\phi + \ddot\psi  \rb - (I_{x'} - I_{y'})
	\lb -\cos\psi \sin\psi \dot\theta^2+\cos 2\psi \sin\theta \dot\theta \dot \phi +\cos\psi \sin^2\theta\sin\psi \dot \phi^2 \rb 
	= Q_{z'}
	\end{align*}
	where $q_j = q_{z'} = \psi$, and $Q_j = Q_{z'}$. Now notice that 
	\begin{align*}
	\dot\omega_{z'} = -\sin\theta \dot\theta \dot\phi+ \cos\theta \ddot\phi + \ddot\psi\quad 
	\text{\&} 
	\quad
	\omega_{x'} \omega_{y'} = -\cos\psi \sin\psi \dot\theta^2+\cos 2\psi \sin\theta \dot\theta \dot \phi +\cos\psi \sin^2\theta\sin\psi \dot \phi^2.
	\end{align*}
	Thus, we find 
	\begin{align*}
	\boxed{I_{z'} \omega_{z'} - \omega_{x'} \omega_{y'} (I_{x'} - I_{y'}) = Q_{z'}}
	\end{align*}
	which is what we would expect from deriving the equations of motion from Newtonian mechanics. 
	
	
	\item Mathematica code:
	\begin{lstlisting}
	(*Problem 3*)
	
	In[2]:= d = {{Cos[\[Phi]], Sin[\[Phi]], 0}, {-Sin[\[Phi]], 
	Cos[\[Phi]], 0}, {0, 0, 1}};
	
	In[3]:= c = {{1, 0, 0}, {0, Cos[\[Theta]], 
	Sin[\[Theta]]}, {0, -Sin[\[Theta]], Cos[\[Theta]]}};
	
	In[5]:= b = {{Cos[\[Psi]], Sin[\[Psi]], 0}, {-Sin[\[Psi]], 
	Cos[\[Psi]], 0}, {0, 0, 1}};
	
	In[12]:= \[Omega]Prime = {\[Phi]'*Sin[\[Theta]]*
	Sin[\[Psi]] + \[Theta]'*Cos[\[Psi]], \[Phi]'*Sin[\[Theta]]*
	Cos[\[Psi]] - \[Theta]'*Sin[\[Psi]], \[Phi]'*
	Cos[\[Theta]] + \[Psi]'};
	
	In[13]:= \[Omega] = 
	Inverse[d] . Inverse[c] . Inverse[b] . \[Omega]Prime // FullSimplify
	
	Out[13]= {Cos[\[Phi]] Derivative[1][\[Theta]] + 
	Sin[\[Theta]] Sin[\[Phi]] Derivative[1][\[Psi]], 
	Sin[\[Phi]] Derivative[1][\[Theta]] - 
	Cos[\[Phi]] Sin[\[Theta]] Derivative[1][\[Psi]], 
	Derivative[1][\[Phi]] + Cos[\[Theta]] Derivative[1][\[Psi]]}
	
	In[31]:= \[Omega]1 = \[Phi]'[t]*Sin[\[Theta][t]]*
	Sin[\[Psi][t]] + \[Theta]'[t]*Cos[\[Psi][t]];
	
	In[40]:= \[Omega]2 = \[Phi]'[t]*Sin[\[Theta][t]]*
	Cos[\[Psi][t]] - \[Theta]'[t]*Sin[\[Psi][t]];
	
	In[41]:= \[Omega]3 = \[Phi]'[t]*Cos[\[Theta][t]] + \[Psi]'[t];
	
	In[51]:= T = (1/2)*(I1*\[Omega]1^2 + I2*\[Omega]2^2 + 
	I3*\[Omega]3^2) // Simplify
	
	Out[51]= 1/2 (I2 (Sin[\[Psi][t]] Derivative[1][\[Theta]][t] - 
	Cos[\[Psi][t]] Sin[\[Theta][t]] Derivative[1][\[Phi]][t])^2 + 
	I1 (Cos[\[Psi][t]] Derivative[1][\[Theta]][t] + 
	Sin[\[Theta][t]] Sin[\[Psi][t]] Derivative[1][\[Phi]][t])^2 + 
	I3 (Cos[\[Theta][t]] Derivative[1][\[Phi]][t] + 
	Derivative[1][\[Psi]][t])^2)
	
	In[43]:= (*LHS of EOM using \[Psi]*)
	
	In[52]:= D[D[T, \[Psi]'[t]], t] - D[T, \[Psi][t]] // FullSimplify
	
	Out[52]= (I1 - I2) Cos[\[Psi][t]] Sin[\[Psi][t]] Derivative[
	1][\[Theta]][
	t]^2 - (I3 + (I1 - I2) Cos[2 \[Psi][t]]) Sin[\[Theta][
	t]] Derivative[1][\[Theta]][t] Derivative[1][\[Phi]][
	t] + (-I1 + I2) Cos[\[Psi][t]] Sin[\[Theta][t]]^2 Sin[\[Psi][
	t]] Derivative[1][\[Phi]][t]^2 + 
	I3 (Cos[\[Theta][t]] (\[Phi]^\[Prime]\[Prime])[
	t] + (\[Psi]^\[Prime]\[Prime])[t])
	
	(*Sanity check*)
	
	In[45]:= \[Omega]1*\[Omega]2 // FullSimplify
	
	Out[45]= (-Sin[\[Psi][t]] Derivative[1][\[Theta]][t] + 
	Cos[\[Psi][t]] Sin[\[Theta][t]] Derivative[1][\[Phi]][
	t]) (Cos[\[Psi][t]] Derivative[1][\[Theta]][t] + 
	Sin[\[Theta][t]] Sin[\[Psi][t]] Derivative[1][\[Phi]][t])
	
	In[47]:= D[\[Omega]3, t] // FullSimplify
	
	Out[47]= -Sin[\[Theta][t]] Derivative[1][\[Theta]][t] Derivative[
	1][\[Phi]][t] + 
	Cos[\[Theta][t]] (\[Phi]^\[Prime]\[Prime])[
	t] + (\[Psi]^\[Prime]\[Prime])[t]
	
	In[56]:= (*Simplifications to find \[Omega]3*)
	
	In[60]:= D[\[Omega]3, 
	t] - (- Sin[\[Theta][t]] Derivative[1][\[Theta]][t] Derivative[
	1][\[Phi]][
	t] + (Cos[\[Theta][t]] (\[Phi]^\[Prime]\[Prime])[
	t] + (\[Psi]^\[Prime]\[Prime])[t])) // FullSimplify
	
	Out[60]= 0
	
	(*Simplifications to find \[Omega]1*\[Omega]2*)
	
	In[66]:= \[Omega]1*\[Omega]2 - (-Cos[\[Psi][t]] Sin[\[Psi][
	t]] Derivative[1][\[Theta]][t]^2 + 
	Cos[2 \[Psi][t]]*
	Sin[\[Theta][t]] Derivative[1][\[Theta]][t] Derivative[
	1][\[Phi]][t] + 
	Cos[\[Psi][t]] Sin[\[Theta][t]]^2 Sin[\[Psi][t]] Derivative[
	1][\[Phi]][t]^2) // FullSimplify
	
	Out[66]= 0
	\end{lstlisting}
\end{enumerate}






\noindent \textbf{4. Point Mass on a Disk}


\begin{enumerate}[label=(\alph*)]
	\item By inspection, we can pick out the principal axes about the disk's center as the usual $x-,y-,z-$axes where the origin is at the disk's center and $z$ points out of the page. By symmetry, we only need to calculate the moment of inertia for rotations about $z$ and rotations about $x$. 
	\begin{align*}
	I_{zz,\text{disk}} = \int_A \f{M}{\pi R^2} (r^2 - \cancel{z^2})\,dA = \f{M}{\pi R^2}\int_{r=0}^R \int_{0}^{2\pi} r^3 \,d\theta\,dr = \f{1}{2}MR^2.
	\end{align*}
	\begin{align*}
	I_{xx,\text{disk}} = I_{yy,\text{disk}} = \int_A \f{M}{\pi R^2}(r^2 - xx)\,dA = \f{M}{\pi R^2}\int_{r=0}^R \int_{0}^{2\pi} r^3(1-\sin^2\theta)\,d\theta \,dr  = \f{1}{4}MR^2.
	\end{align*}
	So, the moment of inertia tensor of the disk about its center is 
	\begin{align*}
	\boxed{\hat I_{\text{disk, disk center}} = \text{diag}\lb \f{1}{4}MR^2, \f{1}{4}MR^2, \f{1}{2}MR^2 \rb}
	\end{align*}
	Now we wish to compute the moment of inertia of the disk about point $A$. To this end, we use the parallel axis theorem. 
	\begin{align*}
	\hat I_{ab}^A = M(\delta_{ab}\mathbf{R}^2 - \mathbf{R}_a \mathbf{R}_b) + \hat I_{ab}^{\text{center}}
	\end{align*}
	where $\mathbf{R} = (0,R,0)^\top$ denotes the translation vector from $Q$ to the center of the disk. It is clear that there is no change to $\hat I$ whenever $a\neq b$. The only possible changes are along in $xx,yy,zz$:
	\begin{align*}
	&\hat I_{xx}^A = MR^2 + I_{xx}^{\text{center}} = \f{5}{4}MR^2\\
	&\hat I_{yy}^A = M(R^2-R^2) + I_{yy}^{\text{center}} = \f{1}{4}MR^2\\
	&\hat I_{zz}^A = MR^2 + I_{zz}^{\text{center}} = \f{3}{2}MR^2.
	\end{align*}
	Finally, we calculate the moment of inertia tensor of the mass $m$, which is attached on the disk, about the point $A$. The distance from the mass to point $A$ is $r = R\sqrt{2}$. Based on the geometry of the problem we also know that $y=x=R$ and $z=0$. We shall proceed:
	\begin{align*}
	\hat I_m^{A} = \f{3}{8}M\begin{pmatrix}
	r^2-x^2 & -xy & -xz\\
	-xy & r^2 - y^2& -yz\\
	-xz &  -yz & r^2 - z^2
	\end{pmatrix} = \f{3}{8}M\begin{pmatrix}
	2R^2-R^2 & -R^2 & 0\\
	-R^2 & 2R^2 - R^2& 0\\
	0 &  0 & 2R^2
	\end{pmatrix} = \f{3}{8}MR^2\begin{pmatrix}
	1 & -1 & 0\\
	-1 & 1& 0\\
	0 &  0 & 2
	\end{pmatrix}.
	\end{align*}
	With this, the combined moment of inertia of the system about point $A$ is 
	\begin{align*}
	\hat{I}^A_\text{system} = MR^2
	\begin{pmatrix}
	3/8+5/4& -3/8& 0 \\
	-3/8 & 3/8+1/4& 0 \\
	0 & 0 &  6/8+ 3/2
	\end{pmatrix} = 
	\boxed{\f{MR^2}{8}
	\begin{pmatrix}
	13& -3& 0 \\
	-3 & 5& 0 \\
	0 & 0 &  18
	\end{pmatrix}}
	\end{align*}
	
	
	\item To find the principal axes and principal moments of inertia, we simply have to diagonalize $\hat I^A_\text{system}$ found in the previous part. We may use Mathematica to do this. The principal moment of inertia tensor is  
	\begin{align*}
	\boxed{\hat I_\text{system}^\text{principal} = MR^2 \text{diag}\lb \f{9}{4}, \f{7}{4}, \f{1}{2} \rb}
	\end{align*}
	where the corresponding normalized principal axes are
	\begin{align*}
	\boxed{\begin{pmatrix}
	0 \\ 0 \\ 1
	\end{pmatrix}, \quad \f{1}{\sqrt{10}}\begin{pmatrix}
	-3 \\ 1 \\ 0
	\end{pmatrix}, \quad 
	\f{1}{\sqrt{10}}\begin{pmatrix}
	1 \\ 3 \\ 0
	\end{pmatrix}}
	\end{align*}
	To do all this by hand one will need to write out the characteristic equation for $\hat{I}^A_\text{system}$ and solve for the eigenvalues, which are the roots of that equation. Then, for each eigenvalue $\lambda$, one will find the kernel of the matrix $\hat{I}^A_\text{system} - \lambda \mathbb{I}$ and then find an orthonormal basis for each subspace. In principle, the various subspaces corresponding to distinct eigenvalues will be orthogonal, so concatenating the basis vectors will gives us an orthonormal basis aka a set of principal axes. 
	
	\item In the coordinate system centered at $A$, the angular velocity vector of the disk plus mass system is simply $\vec{\omega}_A = (0,\omega,0)^\top$. The angular momentum about point $A$ is then 
	\begin{align*}
	\vec{L}^A = \hat I^A \vec{\omega}= \f{\omega MR^2}{8}\begin{pmatrix}
	-3 \\ 5 \\ 0
	\end{pmatrix}
	\end{align*}
	Here, $\vec{L}^A$ is the angular momentum in the body frame which is attached to the system. To calculate the angular momentum of the system in the lab frame, we must transform out of the body frame via a time-dependent rotation matrix about $y$. Let the lab frame be $(x_l, y_l, z_l)$. The transformation matrix 
	\begin{align*}
	\begin{pmatrix}
	x_l \\ y_l \\ z_l 
	\end{pmatrix}
	= 
	\begin{pmatrix}
	\cos\omega t & 0 & \sin\omega t \\
	0&1&0 \\
	-\sin\omega t& 0 & \cos\omega t
	\end{pmatrix}
	\begin{pmatrix}
	x\\y\\z
	\end{pmatrix}.
	\end{align*}
	With this, the angular momentum vector in the lab frame is given by 
	\begin{align*}
	\vec{L}_\text{lab} = 
	\f{\omega MR^2}{8}\begin{pmatrix}
	\cos\omega t & 0 & \sin\omega t \\
	0&1&0 \\
	-\sin\omega t& 0 & \cos\omega t
	\end{pmatrix} 
	\begin{pmatrix}
	-3 \\ 5 \\ 0
	\end{pmatrix} = \boxed{\f{\omega MR^2}{8}
	\begin{pmatrix}
	-3\cos\omega t \\
	5 \\ 
	3\sin\omega t
	\end{pmatrix}}
	\end{align*}
	
	\item Mathematica code:
	\begin{lstlisting}
	In[68]:= (*Problem 4*)
	
	In[69]:= Integrate[(M/(Pi*R^2))*r^3, {r, 0, R}, {\[Theta], 0, 2*Pi}]
	
	Out[69]= (M R^2)/2
	
	In[70]:= Integrate[(M/(Pi*R^2))*r^3 (1 - Sin[\[Theta]]^2), {r, 0, 
	R}, {\[Theta], 0, 2*Pi}]
	
	Out[70]= (M R^2)/4
	
	In[95]:= Ihat = 
	M*R^2*{{3/8 + 5/4, -3/8, 0}, {-3/8, 3/8 + 1/4, 0}, {0, 0, 
	2*3/8 + 3/2}} // FullSimplify
	
	Out[95]= {{(13 M R^2)/8, -((3 M R^2)/8), 0}, {-((3 M R^2)/8), (
	5 M R^2)/8, 0}, {0, 0, (9 M R^2)/4}}
	
	In[96]:= Eigensystem[Ihat] // Simplify
	
	Out[96]= {{(9 M R^2)/4, (7 M R^2)/4, (M R^2)/
	2}, {{0, 0, 1}, {-3, 1, 0}, {1, 3, 0}}}
	\end{lstlisting}
\end{enumerate}






\noindent \textbf{5. A Rolling Cone (Goldstein Ch.5 \#17)}


\begin{enumerate}[label=(\alph*)]
	\item The volume of the cone is well known: $V = \pi R^2 h/3$. To find the center of mass of the cone, we use the following definition
	\begin{align*}
	\vec{r}_\text{CM} = \f{1}{\int \rho \, dV} \int \vec{r} \,dm = \f{1}{V}\int \vec{r} \,dV.  
	\end{align*}
	By symmetry, the center of mass must lie on the axis which goes through the tip and the center of the base. So, choosing our coordinates such that the origin is at the tip and the $z$-axis goes through the center of the base, we find that it suffices to find where the center of mass lies on the $z$-axis. To do this, we simply compute:
	\begin{align*}
	{z}_\text{CM} =\f{1}{V} \int_{0}^h z\underbrace{\lp \pi z^2 \f{R^2}{h^2} \rp \,dz}_{dV} = \f{3h}{4}.
	\end{align*}
	So, the center of mass is on axis which goes through the tip and the center of the base of the cone, at a distance of $3h/4$ from the tip. \\
	
	Now pick a new set of axes so that the $y'$-axis goes through the tip and the center of mass as required by the problem. The moment of inertia tensor is obtained using the following formula
	\begin{align*}
	I_{ab} = \int \rho(\vec{r}) \lb \vec{r}^2 \delta_{ab} - \vec{r}_a \vec{r}_b \rb \,dV.
	\end{align*}
	where $\vec{r}_a$ denotes the components of the position vector $\vec{r}$. By symmetry, we may use cylindrical coordinates where
	\begin{align*}
	(x',y',z') = (r\cos\phi, y', r\sin\phi).
	\end{align*}
	To make things look a bit more compact, let the mass of the cone be $M = \rho V$. The various components of $\hat{I}$ are
	\begin{align*}
	&I_{x'x'} = \f{M}{V} \int_{0}^h \int_{0}^{Ry'/h} \int_{0}^{2\pi} (r^2 + y'^2 - r^2 \cos^2\phi  )r\,d\phi\,dr\,dy' = \f{3}{20}M(4h^2 + R^2)\\
	&I_{y'y'} = \f{M}{V} \int_{0}^h \int_{0}^{Ry'/h} \int_{0}^{2\pi} (r^2 + y'^2 - y'^2  )r\,d\phi\,dr\,dy' = \f{3}{10}MR^2\\
	&I_{z'z'} = \f{M}{V} \int_{0}^h \int_{0}^{Ry'/h} \int_{0}^{2\pi} (r^2 + y'^2 - r^2 \sin^2\phi )r\,d\phi\,dr\,dy' = \f{3}{20}M(4h^2 + R^2)\\
	&I_{x'y'} = \f{M}{V} \int_{0}^h \int_{0}^{Ry'/h} \int_{0}^{2\pi} ( - r y' \cos\phi  )r\,d\phi\,dr\,dy' = 0 \\
	&I_{x'z'} = \f{M}{V} \int_{0}^h \int_{0}^{Ry'/h} \int_{0}^{2\pi} ( - r^2\cos\phi\sin\phi  )r\,d\phi\,dr\,dy' = 0 \\
	&I_{y'z'} = \f{M}{V} \int_{0}^h \int_{0}^{Ry'/h} \int_{0}^{2\pi} ( - ry'\sin\phi  )r\,d\phi\,dr\,dy' = 0.
	\end{align*}
	So, 
	\begin{align*}
	\boxed{\hat I = \text{diag}\lb \f{3}{20}M(4h^2 + R^2), \f{3}{10}MR^2,  \f{3}{20}M(4h^2 + R^2)\rb}
	\end{align*}
	
	\item Now let us move the axes to the center of mass. The displacement vector is thus $\vec{k} = (0,3h/4,0)$, and we do not introduce any off-diagonal terms to the new moment of inertia tensor. By the parallel axes theorem, the new moment of inertia tensor is given by
	\begin{align*}
	\hat I^{\text{CM}}_{ab} &= \hat{I}_{ab} - M\lp \delta_{ab} \vec{k}^2 - \vec{k}_a \vec{k}_b \rp \\
	&= \text{diag}\lb \f{3}{20}M(4h^2 + R^2) - \f{9Mh^2}{16}, \f{3}{10}MR^2,  \f{3}{20}M(4h^2 + R^2) -\f{9Mh^2}{16}\rb \\ 
	&= \boxed{\text{diag}\lb \f{3}{80}M(h^2 + 4R^2), \f{3}{10}MR^2,  \f{3}{80}M(h^2 + 4R^2)\rb}
	\end{align*}
	
	
	
	\item The new set of axes $(x,y,z)$ is simply the old set of axes $(x',y',z')$ rotated clockwise by the angle $\al$ about the $x$ axis. Therefore, 
	\begin{align*}
	\begin{pmatrix}
	x\\y\\z
	\end{pmatrix}
	= 
	\begin{pmatrix}
	x \\ y'\cos\al + z'\sin\al  \\ -y'\sin\al + z'\cos\al 
	\end{pmatrix}
	= \begin{pmatrix}
	1 & 0 & 0\\
	0 & \cos\al& \sin\al \\
	0 & -\sin\al & \cos\al
	\end{pmatrix}
	\begin{pmatrix}
	x'\\y'\\z'
	\end{pmatrix} \eqqcolon R_x \begin{pmatrix}
	x'\\y'\\z'
	\end{pmatrix}
	\end{align*}
	The new moment of inertia is thus
	\begin{align*}
	\boxed{\hat I_{\text{unprimed}} = R_x \hat I R_x^\top = 
	\f{3M}{80}\begin{pmatrix}
	h^2+4R^2&0 &0 \\
	0& 2R^2(3+\cos 2\al) + h^2 \sin^2\al& (h^2-4R^2)\cos\al \sin\al \\
	0&(h^2-4R^2)\cos\al \sin\al& (h^2+4R^2)\cos^2\al + 8R^2\sin^2\al
	\end{pmatrix}}
	\end{align*}
	
	
	
	\item  Let us use Part (b) to find the kinetic energy of the rolling cone, then use Part (c) to confirm our finding. In view of Part (b), we may split the kinetic energy $T$ into the translational part of the center of mass and rotation part about the center of mass.  
	\begin{align*}
	T = T_{\text{trans}}^\text{CM} + T_{\text{rot}}^\text{CM}.
	\end{align*}
	The center of mass moves in a circular path at constant speed. Let $\omega_z$ denote the angular velocity of the center of mass (which points in the $z$ direction). Then the translational kinetic energy is given by 
	\begin{align*}
	T_{\text{trans}}^\text{CM}  = \f{1}{2} M \lp \f{3h}{4}\cos\al \rp^2 \omega_z^2.
	\end{align*}
	Here, the speed at which the center of mass travels is 
	\begin{align*}
	v = \f{3h}{4}\cos\al \omega_z.
	\end{align*}
	As the cone rolls without slipping, it is instantaneously rotating about the axis which coincides with the line of contact with angular velocity $\vec{\omega}$, which points in the $y$ direction in view of Part (c). In the $(x',y',z')$ body axes, this angular velocity can be written as
	\begin{align*}
	(0, \omega \cos\al, \omega \sin\al) 
	\end{align*}
	where we have let $\omega$ denote the magnitude of $\vec{\omega}$. From here, we have two equivalent expressions for the speed of the center of mass
	\begin{align*}
	v = \f{3h}{4}\cos\al \omega_z = \f{3h}{4}\sin\al \omega.
	\end{align*}
	which implies that
	\begin{align*}
	\omega = \omega_z \cot\al.
	\end{align*}
	We can now write down the kinetic energy of the rolling cone. With $R/h = \tan \al$, we have
	\begin{align*}
	T 
	&= T_{\text{trans}}^\text{CM} + \f{1}{2}\lb I_{z'z'}\omega_{z'}^2 + I_{y'y'} \omega_{y'}^2 \rb \\
	&= \f{1}{2} M \lp \f{3h}{4}\cos\al \rp^2 \omega_z^2 + \f{1}{2}\lb \f{3MR^2}{10}\lp \omega_z \cos\al\cot\al  \rp^2 + \f{3M}{80}\lp h^2 + 4R^2 \rp\lp  \omega_z \sin\al \cot\al \rp^2 \rb \\
	&= \boxed{\f{3h^2 M \omega_z^2}{80}\lp 7 + 5\cos 2\al \rp} \\
	&=  \boxed{\f{3h^2 M \omega_z^2}{40}\lp 1+5\cos^2\al \rp}
	\end{align*}
	\textcolor{blue}{I have left my answer in different trigonometric forms so it is easier for the grader.} 
	
	
	As a sanity check, let us use the result of Part (c) to get the same answer. Instead of decomposing $\vec{\omega}$ into components in the $(x',y',z')$ basis, we can directly use $\hat{I}_\text{unprimed}$. In the $(x,y,z)$ basis, $\vec{\omega}$ is simply $(0,\omega,0)^\top = (0,\omega_z \cot\al ,0)^\top$. 
	\begin{align*}
	T 
	&= T_{\text{trans}}^\text{CM} + \f{1}{2}\vec{\omega}^\top \hat{I}_\text{unprimed} \vec{\omega}\\
	&= T_{\text{trans}}^\text{CM} + \f{1}{2} \hat{I}_{\text{unprimed},yy}\omega_z^2 \cot^2\al \\
	&= \f{1}{2} M \lp \f{3h}{4}\cos\al \rp^2 \omega_z^2 
	+ \f{1}{2} \f{3M}{80} \lb 2R^2(3+\cos 2\al) + h^2 \sin^2\al \rb \omega_z^2 \cot^2\al \\
	&= \f{3h^2 M \omega_z^2}{80}\lp 7 + 5\cos 2\al \rp,
	\end{align*}
	as expected (where we have once again used $R/h=\tan \al$). So, all is good.
	
	\item Mathematica code:
	\begin{lstlisting}
	In[69]:= (*Problem 5*)
	
	In[70]:= (*Volume of a cone*)
	
	In[21]:= Vol = (1/3)*h*Pi*R^2;
	
	In[71]:= (*COM position on z*)
	
	In[72]:= (1/M) (M/Vol)*Integrate[Pi*z*z^2*(R/h)^2, {z, 0, h}]
	
	Out[72]= (3 h)/4
	
	In[73]:= (*Moment of inertia tensor elements*)
	
	In[37]:= Ixx = (M/Vol)*
	Integrate[(r^2 + y^2 - r^2*Cos[\[Theta]]^2)*r, {y, 0, h}, {r, 0, 
	y*R/h}, {\[Theta], 0, 2*Pi}] // FullSimplify
	
	Out[37]= 3/20 M (4 h^2 + R^2)
	
	In[39]:= Izz = (M/Vol)*
	Integrate[(r^2 + y^2 - r^2*Sin[\[Theta]]^2)*r, {\[Theta], 0, 
	2*Pi}, {y, 0, h}, {r, 0, y*R/h}] // FullSimplify
	
	Out[39]= 3/20 M (4 h^2 + R^2)
	
	In[27]:= Iyy = (M/Vol)*
	Integrate[(r^2 + y^2 - y^2)*r, {\[Theta], 0, 2*Pi}, {y, 0, h}, {r, 
	0, y*R/h}] // FullSimplify
	
	Out[27]= (3 M R^2)/10
	
	In[28]:= Ixy = (M/Vol)*
	Integrate[(-r*Cos[\[Theta]]*y)*r, {\[Theta], 0, 2*Pi}, {y, 0, 
	h}, {r, 0, y*R/h}] // FullSimplify
	
	Out[28]= 0
	
	In[29]:= Ixz = (M/Vol)*
	Integrate[(-r^2*Cos[\[Theta]]*Sin[\[Theta]])*r, {\[Theta], 0, 
	2*Pi}, {y, 0, h}, {r, 0, y*R/h}] // FullSimplify
	
	Out[29]= 0
	
	In[30]:= Iyz = (M/Vol)*
	Integrate[(-y*r*Sin[\[Theta]])*r, {\[Theta], 0, 2*Pi}, {y, 0, 
	h}, {r, 0, y*R/h}] // FullSimplify
	
	Out[30]= 0
	
	In[42]:= Ixx - M*9*h^2/16 // Simplify
	
	Out[42]= 3/80 M (h^2 + 4 R^2)
	
	In[74]:= (*Rotation matrix in X*)
	
	In[75]:= Rx = {{1, 0, 0}, {0, Cos[\[Alpha]], 
	Sin[\[Alpha]]}, {0, -Sin[\[Alpha]], Cos[\[Alpha]]}};
	
	In[79]:= (*Moment of Inertia Tensor*)
	
	In[77]:= Ihat = {{(3 M/80)*(h^2 + 4 R^2), 0, 0}, {0, (3/10) M*R^2, 
	0}, {0, 0, (3 M/80)*(h^2 + 4 R^2)}};
	
	In[80]:= (*Transformed Moment of Inertia Tensor*)
	
	In[47]:= Rx . Ihat . Transpose[Rx] // FullSimplify
	
	Out[47]= {{3/80 M (h^2 + 4 R^2), 0, 0}, {0, 
	3/80 M (2 R^2 (3 + Cos[2 \[Alpha]]) + h^2 Sin[\[Alpha]]^2), 
	3/80 M (h^2 - 4 R^2) Cos[\[Alpha]] Sin[\[Alpha]]}, {0, 
	3/80 M (h^2 - 4 R^2) Cos[\[Alpha]] Sin[\[Alpha]], 
	3/80 M ((h^2 + 4 R^2) Cos[\[Alpha]]^2 + 8 R^2 Sin[\[Alpha]]^2)}}
	
	In[64]:= (*Kinetic Energy*)
	
	In[81]:= T = (1/
	2)*((3 M/80)*(h^2 + 
	4*R^2)*(\[Omega]z*Sin[\[Alpha]]*Cot[\[Alpha]])^2 + (3 M/10)*
	R^2*(\[Omega]z*Cos[\[Alpha]]*Cot[\[Alpha]])^2) + (1/2)*
	M*(3*h*Cos[\[Alpha]]/4)^ 2*\[Omega]z^2 /. {R -> 
	h*Tan[\[Alpha]]} // FullSimplify
	
	Out[81]= 3/80 h^2 M \[Omega]z^2 (7 + 5 Cos[2 \[Alpha]])
	
	In[82]:= (*Sanity check: using transformed moment of inertia tensor*)
	
	In[84]:= TT = (1/2)*
	M*(3*h*Cos[\[Alpha]]/4)^ 
	2*\[Omega]z^2 + (1/2)*(\[Omega]z*Cot[\[Alpha]])^2*(3/
	80 M (2 R^2 (3 + Cos[2 \[Alpha]]) + 
	h^2 Sin[\[Alpha]]^2)) /. {R -> h*Tan[\[Alpha]]} // 
	FullSimplify
	
	Out[84]= 3/80 h^2 M \[Omega]z^2 (7 + 5 Cos[2 \[Alpha]])
	\end{lstlisting}
\end{enumerate}


	
\end{document}



