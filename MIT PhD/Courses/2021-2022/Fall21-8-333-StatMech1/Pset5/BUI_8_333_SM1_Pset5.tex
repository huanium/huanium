\documentclass{article}
\usepackage{physics}
\usepackage{graphicx}
\usepackage{caption}
\usepackage{amsmath}
\usepackage{bm}
\usepackage{framed}
\usepackage{authblk}
\usepackage{empheq}
\usepackage{amsfonts}
\usepackage{esint}
\usepackage[makeroom]{cancel}
\usepackage{dsfont}
\usepackage{centernot}
\usepackage{mathtools}
\usepackage{bigints}
\usepackage{amsthm}
\theoremstyle{definition}
\newtheorem{defn}{Definition}[section]
\newtheorem{prop}{Proposition}[section]
\newtheorem{rmk}{Remark}[section]
\newtheorem{thm}{Theorem}[section]
\newtheorem{exmp}{Example}[section]
\newtheorem{prob}{Problem}[section]
\newtheorem{sln}{Solution}[section]
\newtheorem*{prob*}{Problem}
\newtheorem{exer}{Exercise}[section]
\newtheorem*{exer*}{Exercise}
\newtheorem*{sln*}{Solution}
\usepackage{empheq}
\usepackage{tensor}
\usepackage{xcolor}
%\definecolor{colby}{rgb}{0.0, 0.0, 0.5}
\definecolor{MIT}{RGB}{163, 31, 52}
\usepackage[pdftex]{hyperref}
%\hypersetup{colorlinks,urlcolor=colby}
\hypersetup{colorlinks,linkcolor={MIT},citecolor={MIT},urlcolor={MIT}}  
\usepackage[left=1in,right=1in,top=1in,bottom=1in]{geometry}

\usepackage{newpxtext,newpxmath}
\newcommand*\widefbox[1]{\fbox{\hspace{2em}#1\hspace{2em}}}

\newcommand{\p}{\partial}
\newcommand{\R}{\mathbb{R}}
\newcommand{\C}{\mathbb{C}}
\newcommand{\lag}{\mathcal{L}}
\newcommand{\nn}{\nonumber}
\newcommand{\ham}{\mathcal{H}}
\newcommand{\M}{\mathcal{M}}
\newcommand{\I}{\mathcal{I}}
\newcommand{\K}{\mathcal{K}}
\newcommand{\F}{\mathcal{F}}
\newcommand{\w}{\omega}
\newcommand{\lam}{\lambda}
\newcommand{\al}{\alpha}
\newcommand{\be}{\beta}
\newcommand{\x}{\xi}
\def\dbar{{\mkern3mu\mathchar'26\mkern-12mu   d}}


\newcommand{\G}{\mathcal{G}}

\newcommand{\f}[2]{\frac{#1}{#2}}

\newcommand{\ift}{\infty}

\newcommand{\lp}{\left(}
\newcommand{\rp}{\right)}

\newcommand{\lb}{\left[}
\newcommand{\rb}{\right]}

\newcommand{\lc}{\left\{}
\newcommand{\rc}{\right\}}


\newcommand{\V}{\mathbf{V}}
\newcommand{\U}{\mathcal{U}}
\newcommand{\Id}{\mathcal{I}}
\newcommand{\D}{\mathcal{D}}
\newcommand{\Z}{\mathcal{Z}}

%\setcounter{chapter}{-1}



\usepackage{enumitem}


\usepackage{subfig}
\usepackage{listings}
\captionsetup[lstlisting]{margin=0cm,format=hang,font=small,format=plain,labelfont={bf,up},textfont={it}}
\renewcommand*{\lstlistingname}{Code \textcolor{violet}{\textsl{Mathematica}}}
\definecolor{gris245}{RGB}{245,245,245}
\definecolor{olive}{RGB}{50,140,50}
\definecolor{brun}{RGB}{175,100,80}

%\hypersetup{colorlinks,urlcolor=colby}
\lstset{
	tabsize=4,
	frame=single,
	language=mathematica,
	basicstyle=\scriptsize\ttfamily,
	keywordstyle=\color{black},
	backgroundcolor=\color{gris245},
	commentstyle=\color{gray},
	showstringspaces=false,
	emph={
		r1,
		r2,
		epsilon,epsilon_,
		Newton,Newton_
	},emphstyle={\color{olive}},
	emph={[2]
		L,
		CouleurCourbe,
		PotentielEffectif,
		IdCourbe,
		Courbe
	},emphstyle={[2]\color{blue}},
	emph={[3]r,r_,n,n_},emphstyle={[3]\color{magenta}}
}






\begin{document}
		\begin{framed}
			\noindent Name: \textbf{Huan Q. Bui}\\
			Course: \textbf{8.333 - Statistical Mechanics I}\\
			Problem set: \textbf{\#5}
		\end{framed}
	



\noindent \textbf{1. Repulsive shell potential.} The potential is 
\begin{align*}
\mathcal{V}(r) = \begin{cases}
0, \quad 0 < r < a,\\
u, \quad a < r < b,\\
0, \quad b < r < \infty
\end{cases}
\end{align*}

\begin{enumerate}[label=(\alph*)]
	\item 
	\begin{align*}
	B_2(T) = -\f{b_2}{2} = -\f{1}{2}\int_{\mathbb{R}^3} d^3 \vec{r} \lp e^{-\be \mathcal{V}(r)} - 1 \rp = -2\pi \int_a^b \,dr r^2 \lp e^{-\be u} - 1 \rp = \boxed{-\f{2\pi}{3} \lp e^{-\be u } - 1 \rp (b^3 - a^3)}
	\end{align*}
	\item The equation of state is 
	\begin{align*}
	\f{P}{k_B T} = \f{N}{V}\lp 1 + B_2(T) \f{N}{V} \rp \implies PV \lp 1 + B_2(T)\f{N}{V} \rp^{-1} = N k_B T
	\end{align*}
	In the low density limit where $N/V \equiv n \ll 1$, we have
	\begin{align*}
	\lp 1 + B_2(T) \f{N}{V}  \rp^{-1} = \lp 1 + B_2(T) \f{N}{V}\rp^{-1} \approx 1 - B_2(T) \f{N}{V}
	\end{align*}
	with which we set the LHS to be equal to $P(V-N\Omega)$ and find that 
	\begin{align*}
	\boxed{\Omega = -\f{2\pi}{3} (b^3 - a^3) \lp e^{-\be u} - 1  \rp }
	\end{align*}
	
	
	\item 
	\begin{align*}
	\Omega_\text{hard shell} = \lim_{u\to \infty} \Omega = \boxed{\f{2\pi}{3} (b^3 - a^3) }
	\end{align*}
	I suppose the dependence on $a$ makes sense. When $a=0$, we get the result for hard-sphere potential. When $a\neq 0$ like this case, the excluded volume is smaller since the potential acts on a more limited range. \textbf{\textcolor{blue}{Not sure how to interpret what happens inside the shell... I guess once a particles makes it inside the shell, it gets trapped there since there is no repulsion.}}
\end{enumerate}

\noindent \textbf{2. Surfactant condensation.}

\begin{enumerate}[label=(\alph*)]
	\item The Hamiltonian is 
	\begin{align*}
	\ham = \sum^N_{i=1} \f{p_i^2}{2m} + \f{1}{2}\sum_{i,j} \mathcal{V}(\vec{q}_i - \vec{q}_j)
	\end{align*}
	from which we may calculate the partition function (\textcolor{blue}{inspired by Lecture Notes \# 17, Eq. V.57}):
	\begin{align*}
	\mathcal{Z}(N,T,A) 
	&= \f{1}{N! h^{2N}}\int \prod^N_{i=1}  d^2 \vec{q}_i\, d^2 \vec{p}_i\, \exp\lb -\be \sum^N_{i=1} \f{p_i^2}{2m} - \be \sum^N_{i<j} \mathcal{V}(\vec{q}_i - \vec{q}_j) \rb
	\end{align*}
	where the factor of $1/2$ in the potential goes away when the sum is changed from taking all $i,j$ to only $i<j$. The Gaussian integral over momentum is well-known, but it's in 2D so we'll redo it (in polar coordinates) just to be safe:
	\begin{align*}
	\f{1}{h^{2N}} \lp \int e^{-p^2/2m } d^2\vec{p}  \rp^N = \f{1}{h^{2N}} \lp 2\pi \int_0^\infty p e^{-\be p^2/2m } dp  \rp^N = \f{(2\pi m)^N}{h^{2N} \be^N} = \f{1}{\lambda^{2N}}
	\end{align*}
	where $\lambda = h/\sqrt{2\pi m k_B T}$ is the thermal wavelength. With this, we find
	\begin{align*}
	\boxed{\mathcal{Z}(N,T,A) = \f{1}{N! \lambda^{2N}} \int \prod^N_{i=1} d^2 \vec{q}_i \exp\lb -\be \sum_{i<j} \mathcal{V}(\vec{q}_i - \vec{q}_j)  \rb}  
	\end{align*}
	
	\item Now we know that
	\begin{align*}
	\mathcal{V}(\vec{r}) = 
	\begin{cases}
	\infty, &\quad \abs{\vec{r}} < a \\
	\text{s.t.} \int_a^\infty 2\pi r \mathcal{V}(r) \,dr = -u_0 , &\quad  \abs{\vec{r}} > a
	\end{cases}
	\end{align*}
	
	Following the end of page 112 of Lecture Notes \#17, we have that the first molecule can explore an area of $A$, while the next molecule can only explore $A - \pi a^2$, and the second $A - 2\pi a^2$, and so on. The total non-excluded area is thus a product of these:
	\begin{align*}
	\mathcal{A} 
	&= A(A-\pi a^2)(A-2\pi a^2) \dots (A - (N-1)\pi a^2) \\
	&=  \textcolor{red}{A} \textcolor{blue}{(A-\pi a^2)}\textcolor{purple}{(A-2\pi a^2)} \dots \textcolor{purple}{(A - (N-3)\pi a^2)} \textcolor{blue}{(A - (N-2)\pi a^2)}\textcolor{red}{(A - (N-1)\pi a^2)}\\ 
	&\approx \boxed{\lp A - \f{N \pi a^2}{2}\rp^N}
	\end{align*}
	where we match the terms of the same color and ignore $a^4$ terms to make the final approximation. 
	
	
	\item With this total excluded area $\mathcal{A}$ we can rewrite the partition function as 
	\begin{align*}
	\mathcal{Z}(N,T,A) = \f{1}{N! \lambda^{2N}} \mathcal{A} \exp(-\be \overline{\mathcal{U}})
	\end{align*}
	where $\overline{\mathcal{U}}$ represents the total (average) potential (\textcolor{blue}{I think the problem meant ``attractive'' rather than ``potential''}) energy, obtained by assuming a uniform density $n=N/V$ as
	\begin{align*}
	\overline{\mathcal{U}} &= \f{1}{2} \sum_{i,j} \mathcal{V}(\vec{q}_i - \vec{q}_j) \\
	&= \f{1}{2} \int d^2 \vec{q_1} d^2 \vec{q}_2 n(\vec{q}_1) n(\vec{q}_2) \mathcal{V}(\vec{q}_1 - \vec{q}_2) \\
	&\approx \f{ A n^2}{2} \int_a^\infty  2\pi r   \mathcal{V}(\vec{r})\,dr \\
	&= \f{- u_0 AN^2}{2 A^2} = \boxed{\f{- u_0 N^2}{2A}}
	\end{align*} 
	With this, the partition function is 
	\begin{align*}
	\boxed{\mathcal{Z}(N,T,A) = \f{1}{N! \lambda^{2N}} \lp A - \f{N \pi a^2}{2}\rp^N \exp\lp \f{u_0 N^2}{2A k_BT} \rp}
	\end{align*}
	
	
	\item The free energy resulting from this partition function is 
	\begin{align*}
	F = -k_B T \ln \mathcal{Z} = - N k_BT \ln \lp A - \f{N a^2 \pi}{2} \rp + N k_B T \ln \f{N}{e} + 2N k_B T \ln \lambda - \f{u_0 N^2}{2A}.
	\end{align*}
	The thermodynamic relation between the free energy and surface tension is 
	\begin{align*}
	d F = -S d T + \sigma d A + \mu dN 
	\end{align*}
	where $\sigma$ is the contribution to surface tension due to surfactants. From here we obtain the expression for $\sigma$ via
	\begin{align*}
	\sigma = \f{\p F}{\p A}\bigg\vert_{T,N} = {\f{-N k_BT}{A - N a^2 \pi/2} + \f{u_0 N^2}{2A^2}}
	\end{align*}
	in terms of (uniform) density $n = N/A$ we have
	\begin{align*}
	\boxed{\sigma(n,T) = \f{-n k_B T}{1 - n a^2 \pi/2} + \f{n^2 u_0}{2} }
	\end{align*}
	
	
	
	\item The critical temperature can be found via the stability condition $\p \sigma / \p A = 0$ and $\p^2 \sigma/ \p A^2 = 0$ (Lecture Notes \#17, Page 111)
	\begin{align*}
	\f{\p \sigma}{\p A}\bigg\vert_{A,T \text{ crit.}} = 0 \implies \f{N k_B T}{ \lp A - N a^2 \pi/2 \rp^2} - \f{N^2 u_0}{A^3} = 0
	\end{align*}
	and 
	\begin{align*}
	\f{\p^2 \sigma}{\p A^2}\bigg\vert_{A,T \text{ crit.}}  = 0 \implies \f{-2 N k_BT}{ \lp A -  N a^2 \pi/a \rp^3} + \f{3 N^2 u_0}{A^4} = 0 
	\end{align*}
	These conditions are simultaneously met if 
	\begin{align*}
	\boxed{A_c = \f{3N a^2 \pi}{2} \quad\quad T_c = \f{8 u_0}{27 k_B a^2 \pi} }
	\end{align*}
	Mathematica code:
	\begin{lstlisting}
	In[12]:= eqn1 = D[-M*kB*T/(A - M*a^2*Pi/2) + u0*M^2/(2*A^2), A]
	
	Out[12]= (kB M T)/(A - 1/2 a^2 M Pi)^2 - (M^2 u0)/A^3
	
	In[13]:= eqn2 = D[D[-M*kB*T/(A - M*a^2*Pi/2) + u0*M^2/(2*A^2), A], A]
	
	Out[13]= -((2 kB M T)/(A - 1/2 a^2 M Pi)^3) + (3 M^2 u0)/A^4
	
	In[14]:= Solve[{eqn1 == 0, eqn2 == 0}, {A, T}] // FullSimplify
	
	Out[14]= {{A -> 3/2 a^2 M Pi, T -> (8 u0)/(27 a^2 kB Pi)}}
	\end{lstlisting}
	
	
	
	\item To find heat capacities, we must first find $E$, the energy due surfactants
	\begin{align*}
	E = -\f{\p \ln \mathcal{Z}}{\p \be} = N k_B T - \f{N^2 u_0}{2A} = Q - \text{ work } = Q - \sigma A 
	\end{align*}
	from here we may compute the heat capacities using $\delta Q = d E - \sigma d A$. For $C_A$, $A$ is constant, so
	\begin{align*}
	\boxed{C_A = \f{\delta Q}{\delta T}\bigg\vert_{A}  = \f{\p E}{\p T}\bigg\vert_{A} - 0  = N k_B}
	\end{align*}
	The expression for $C_\sigma$ is 
	\begin{align*}
	\boxed{C_\sigma = \f{\delta Q}{ \delta T}\bigg\vert_{\sigma} = \f{\p E}{\p T}\bigg\vert_\sigma - \sigma \f{\p A}{\p T}\bigg\vert_{\sigma}}
	\end{align*}
	Mathematica code for finding $E$:
	\begin{lstlisting}
	In[23]:= Z = (1/(Factorial[M]*\[Lambda]^(2*M)))*(A - M*Pi^2)^N*
	Exp[\[Beta]*u0*M^2/(2*A)] /. {\[Lambda] -> 
	h*Sqrt[\[Beta]]/Sqrt[2*Pi*m]};
	
	In[24]:= -D[Log[Z], \[Beta]] // FullSimplify
	
	Out[24]= -((M^2 u0)/(2 A)) + M/\[Beta]
	\end{lstlisting}
\end{enumerate}


\noindent \textbf{3. Critical point behavior.} The partition function is 
\begin{align*}
\mathcal{Z}(T,N,V) = \mathcal{Z}_\text{ideal gas}(T,N,V) \exp\lp \f{\be b N^2}{2V} - \f{\be c N^3}{6V^2} \rp
\end{align*}
where $\mathcal{Z}_\text{ideal gas} = V^N/(N! \lambda^{3N})$ is the partition function of a classical gas where $\lambda$ is the thermal wavelength. 

\begin{enumerate}[label=(\alph*)]
	\item We may compute the pressure from the free energy as follows. First we calculate the free energy then use it to find the pressure. We'll do this in one swoop using Mathematica:
	\begin{align*}
	F = -k_BT \ln \mathcal{Z} \implies \boxed{P} = -\f{\p F}{\p V}\bigg\vert_{T,N} = {\f{N^2(2cN - 3bV)}{6V^3} + \f{N}{V\be}} = \f{2cn^3 - 3bn^2}{6} + \f{n}{\be} = \boxed{nk_BT + \f{cn^3}{3} - \f{b n^2}{2}}
	\end{align*}
	
	
	\item The critical temperature can be located by setting $blah$
	\item 
	\item 
	\item 
	\item
\end{enumerate}

\noindent \textbf{4. Quantum-Classical correspondence.}

\begin{enumerate}[label=(\alph*)]
	\item 
	\item 
\end{enumerate}

\noindent \textbf{5. Vibrational and rotational heat capacities at high temperatures.}

\begin{enumerate}[label=(\alph*)]
	\item 
	\item 
	\item 
	\item 
	\item 
	\item 
\end{enumerate}

\noindent \textbf{6. van Leeuwen's Theorem.}


\noindent \textbf{7. (Optional) Zero-point energy.}

\begin{enumerate}[label=(\alph*)]
	\item 
	\item 
	\item 
\end{enumerate}

\noindent \textbf{8. (Optional) Quantum mechanical entropy.}

\begin{enumerate}[label=(\alph*)]
	\item 
	\item 
	\item 
\end{enumerate}

\noindent \textbf{9. (Optional) Electron spin. }

\begin{enumerate}[label=(\alph*)]
	\item 
	\item 
	\item 
\end{enumerate}




\end{document}














