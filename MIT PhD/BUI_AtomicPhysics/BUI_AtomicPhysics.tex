\documentclass{book}
\usepackage{physics}
\usepackage{graphicx}
\usepackage{caption}
\usepackage{amsmath}
\usepackage{bm}
\usepackage{framed}
\usepackage{authblk}
\usepackage{empheq}
\usepackage{amsfonts}
\usepackage{esint}
\usepackage[makeroom]{cancel}
\usepackage{dsfont}
\usepackage{centernot}
\usepackage{mathtools}
\usepackage{bigints}
\usepackage{amsthm}
\theoremstyle{definition}
\newtheorem{defn}{Definition}[section]
\newtheorem{prop}{Proposition}[section]
\newtheorem{rmk}{Remark}[section]
\newtheorem{thm}{Theorem}[section]
\newtheorem{exmp}{Example}[section]
\newtheorem{prob}{Problem}[section]
\newtheorem{sln}{Solution}[section]
\newtheorem*{prob*}{Problem}
\newtheorem{exer}{Exercise}[section]
\newtheorem*{exer*}{Exercise}
\newtheorem*{sln*}{Solution}
\usepackage{empheq}
\usepackage{tensor}
\usepackage{xcolor}
%\definecolor{colby}{rgb}{0.0, 0.0, 0.5}
\definecolor{MIT}{RGB}{163, 31, 52}
\usepackage[pdftex]{hyperref}
%\hypersetup{colorlinks,urlcolor=colby}
\hypersetup{colorlinks,linkcolor={MIT},citecolor={MIT},urlcolor={MIT}}  



\newcommand*\widefbox[1]{\fbox{\hspace{2em}#1\hspace{2em}}}

\newcommand{\p}{\partial}
\newcommand{\R}{\mathbb{R}}
\newcommand{\C}{\mathbb{C}}
\newcommand{\lag}{\mathcal{L}}
\newcommand{\nn}{\nonumber}
\newcommand{\ham}{\mathcal{H}}
\newcommand{\M}{\mathcal{M}}
\newcommand{\I}{\mathcal{I}}
\newcommand{\K}{\mathcal{K}}
\newcommand{\F}{\mathcal{F}}
\newcommand{\w}{\omega}
\newcommand{\lam}{\lambda}
\newcommand{\al}{\alpha}
\newcommand{\be}{\beta}
\newcommand{\x}{\xi}

\newcommand{\G}{\mathcal{G}}

\newcommand{\f}[2]{\frac{#1}{#2}}

\newcommand{\ift}{\infty}

\newcommand{\lp}{\left(}
\newcommand{\rp}{\right)}

\newcommand{\lb}{\left[}
\newcommand{\rb}{\right]}

\newcommand{\lc}{\left\{}
\newcommand{\rc}{\right\}}


\newcommand{\V}{\mathbf{V}}
\newcommand{\U}{\mathcal{U}}
\newcommand{\Id}{\mathcal{I}}
\newcommand{\D}{\mathcal{D}}
\newcommand{\Z}{\mathcal{Z}}

%\setcounter{chapter}{-1}






\usepackage{subfig}
\usepackage{listings}
\captionsetup[lstlisting]{margin=0cm,format=hang,font=small,format=plain,labelfont={bf,up},textfont={it}}
\renewcommand*{\lstlistingname}{Code \textcolor{violet}{\textsl{Mathematica}}}
\definecolor{gris245}{RGB}{245,245,245}
\definecolor{olive}{RGB}{50,140,50}
\definecolor{brun}{RGB}{175,100,80}

%\hypersetup{colorlinks,urlcolor=colby}
\lstset{
	tabsize=4,
	frame=single,
	language=mathematica,
	basicstyle=\scriptsize\ttfamily,
	keywordstyle=\color{black},
	backgroundcolor=\color{gris245},
	commentstyle=\color{gray},
	showstringspaces=false,
	emph={
		r1,
		r2,
		epsilon,epsilon_,
		Newton,Newton_
	},emphstyle={\color{olive}},
	emph={[2]
		L,
		CouleurCourbe,
		PotentielEffectif,
		IdCourbe,
		Courbe
	},emphstyle={[2]\color{blue}},
	emph={[3]r,r_,n,n_},emphstyle={[3]\color{magenta}}
}


\begin{document}
\begin{titlepage}\centering
 \clearpage
 \title{{\textsc{\textbf{ATOMIC PHYSICS}}}\\ \smallskip - A Quick Guide - \\}
 \author{\bigskip Huan Q. Bui}
  \affil{B.A., COLBY COLLEGE (2021) $\,$\\  
  	MASSACHUSETTS INSTITUTE OF TECHNOLOGY}
 \date{June 15, 2021}
 \maketitle
 \thispagestyle{empty}
\end{titlepage}

\subsection*{Preface}
\addcontentsline{toc}{subsection}{Preface}


$\,$\\


\noindent Greetings, \\


While I have spent most of my undergraduate years in Professor Charles Conover's lab at Colby College working on cold atom experiments, I never had \textit{formal} training in atomic, molecular, and optical physics. The closest to formal training I have for AMO physics is a standard quantum mechanics course I took in the fall of my junior year. Most of the intuition I have for atomic physics, I learned from my discussions with Professor Conover or read from books and articles here and there. This article is my attempt at \textit{formally} teaching myself atomic physics.   \\

This article is basically my version of an ``atomic physics dictionary,'' which should keep growing as I go along in my education and research at MIT. As a result of this, there is no good way for me to organize the topics in here but by alphabetical order (hence ``dictionary''). I don't know how well I'll be able to curate this article, but we'll see.\\


\noindent In any case, good luck and, most importantly, enjoy!


\newpage





\chapter*{A}
\chapter*{B}


\section*{Bloch's Theorem and Bloch States}


Consider a periodic potential $V(\mathbf{r})$ associated with a lattice whose \textcolor{blue}{primitive lattice translation vectors} are given by 
\begin{equation*}
\mathbf{T} = n_1 \mathbf{a}_1 +  n_2 \mathbf{a}_2  +  n_3 \mathbf{a}_3,
\end{equation*}
where $n_i$ are integers and $\mathbf{a_i}$ are the three noncoplanar vectors ($\mathbf{T}$ is basically vectors which translates from one vertex in the lattice to another arbitrary one). Since $V$ is periodic, we have
\begin{equation*}
V(\mathbf{T} + \mathbf{r}) = V(\mathbf{r}). 
\end{equation*}
In Fourier components, 
\begin{equation*}
V(\mathbf{r}) = \sum_\mathbf{G} V_\mathbf{G} e^{i\mathbf{G}\cdot \mathbf{r}}
\end{equation*}
where $\mathbf{G}$ are a set of vectors and $V_\mathbf{G}$ are Fourier coefficients. By the periodicity of $V$, we have
\begin{equation*}
e^{i\mathbf{G}\cdot \mathbf{T}} = 1 \implies \mathbf{G}\cdot \mathbf{T} = 2\rho \pi, \quad \rho \in \mathbb{Z}.
\end{equation*}
The only way to define $\mathbf{G}$ such that the above equation makes sense is:
\begin{equation*}
\mathbf{G} = m_1 \mathbf{A}_1 + m_2 \mathbf{A}_2 + m_3 \mathbf{A}_3
\end{equation*}
where $m_j$ are integers and $\mathbf{A}_j$ are three noncoplanar vectors defined by 
\begin{equation*}
\mathbf{a}_j \cdot \mathbf{A}_l = 2\pi \delta_{jl}
\end{equation*}
This shows the existence of an $r$-lattice implies that of a $k$-lattice, and we call $\mathbf{G}$ the \textcolor{blue}{reciprocal lattice}. \\


What set of functions describes the motion of electrons in such a potential? Since we want to reflect the translation symmetry of the lattice, we may impose the \textit{Born-von Karman periodic boundary condition} on the plane wave 
\begin{equation*}
\phi(\mathbf{r}) = e^{i(\mathbf{k}\cdot \mathbf{r} - \omega t)}
\end{equation*}
to get
\begin{equation*}
\phi(\mathbf{r} + N_j \mathbf{a_j}) = \phi(\mathbf{r})
\end{equation*}
where $j=1,2,3$ and $N = N_1N_2N_3$ is the number of primitive unit cells in the crystal; $N_j$ is the number of unit cells in the $j$th direction. From here, we have that 
\begin{equation*}
e^{iN_j \mathbf{k}\cdot \mathbf{a}_j} = 1.
\end{equation*}
Following a similar argument as before, we find that the only allowed $\mathbf{k}$ vectors are of the form
\begin{equation*}
\mathbf{k} = \sum^3_{j=1}\f{m_j}{N_j}\mathbf{A}_j
\end{equation*}








Now, consider a Schr\"{o}dinger equation with potential $V(\mathbf{r})$:
\begin{equation*}
\widehat{H}\psi = \lb -\f{\hbar^2 \nabla^2}{2m} + V(\mathbf{r}) \rb \psi = E\psi.
\end{equation*}
In Fourier components, we again have
\begin{equation*}
V(\mathbf{r}) = \sum_\mathbf{G} V_\mathbf{G} e^{i\mathbf{G}\cdot \mathbf{r}}.
\end{equation*}
Let us set the background potential to zero, i.e., $V_0 \equiv 0$. Next, let us write the solution $\phi(\mathbf{r})$ as a combination of plane waves obeying the Born-von Karman PBC:
\begin{equation*}
\psi(\mathbf{r}) = \sum_\mathbf{k}C_\mathbf{k} e^{i\mathbf{k}\cdot \mathbf{r}},
\end{equation*}
so that $\phi(\mathbf{r})$ also satisfies the Born-von Karman PBC. Plugging this into the SE, we find 
\begin{equation*}
\sum_\mathbf{k} \f{\hbar^2k^2}{2m} C_\mathbf{k} e^{i\mathbf{k}\cdot \mathbf{r}} + \underbrace{\lb \sum_\mathbf{G} V_\mathbf{G} e^{i\mathbf{G}\cdot \mathbf{r}} \rb \lb \sum_\mathbf{k} C_\mathbf{k} e^{i\mathbf{k}\cdot \mathbf{r}} \rb}_{V(\mathbf{r}\psi)} = E\sum_\mathbf{k} C_\mathbf{k} e^{i\mathbf{k}\cdot \mathbf{r}}
\end{equation*}
where we can re-write:
\begin{equation*}
V(\mathbf{r})\psi = \sum_{\mathbf{G},\mathbf{k}} V_\mathbf{G}C_\mathbf{k} e^{i(\mathbf{G}+ \mathbf{k})\cdot \mathbf{r}}  = \sum_{\mathbf{G},\mathbf{k}} V_\mathbf{G}C_{\mathbf{k}-\mathbf{G}} e^{i\mathbf{k}\cdot \mathbf{r}}.
\end{equation*}
With this, we can factor out $e^{i\mathbf{k}\cdot \mathbf{r}}$ in each term of the SE and use the fact that the plane waves form an orthogonal basis, we find 
\begin{equation*}
\lp \f{\hbar^2k^2}{2m} - E \rp C_\mathbf{k} + \sum_\mathbf{G} V_\mathbf{G} C_{\mathbf{k}-\mathbf{G}} = 0.
\end{equation*}

Let us write $\mathbf{k} = \mathbf{q} - \mathbf{G}'$ and let $\mathbf{G}'' = \mathbf{G}' + \mathbf{G}$, where $\mathbf{q}$ lies in the \textcolor{blue}{first Brillouin zone}. With this change of variables, we have the result
\begin{equation*}
\lp \f{\hbar^2 (\mathbf{q} - \mathbf{G}')^2}{2m} - E \rp C_{\mathbf{q} - \mathbf{G}'} + \sum_{\mathbf{G}''}V_{\mathbf{G}'' - \mathbf{G}'}C_{\mathbf{q} - \mathbf{G}''} = 0.
\end{equation*}


Now, we're ready for the statement of the \textbf{Bloch's Theorem}. The result above involves coefficients $C_{\mathbf{k}}$ in which $\mathbf{k} = \mathbf{q} - \mathbf{G}$, where $\mathbf{G}$ are general reciprocal lattice vectors. This means that if we fix $\mathbf{q}$, then the only $C_\mathbf{k}$ that feature are of the form $C_{\mathbf{q} - \mathbf{G}}$. In other words, for each $\mathbf{q}$, there is a wavefunction $\psi_\mathbf{q}(r)$ that takes the form
\begin{equation*}
\psi_\mathbf{q}(\mathbf{r}) = \sum_\mathbf{G} C_{\mathbf{q} - \mathbf{G}} e^{i(\mathbf{q} - \mathbf{G})\cdot \mathbf{r}},
\end{equation*}
where we have substituted $\mathbf{k} = \mathbf{q} - \mathbf{G}$. Factoring out $e^{i\mathbf{q}\cdot \mathbf{r}}$, we find 
\begin{equation*}
\boxed{\psi_\mathbf{q}(\mathbf{r})  = e^{i\mathbf{q}\cdot \mathbf{r}} \sum_\mathbf{G} C_{\mathbf{q} - \mathbf{G}} e^{-i \mathbf{G}\cdot \mathbf{r}} \equiv e^{i\mathbf{q}\cdot\mathbf{r}}u_{j,\mathbf{q}}}
\end{equation*}
So, the solution is a plane wave with wave vector within the first Brillouin zone TIMES a function with the periodicity of the lattice. Functions of this form are known as \textbf{Bloch functions} or \textbf{Bloch states}. They serve as a suitable basis for the wave functions or states of electrons in crystalline solids. 



\textbf{Bloch's Theorem} is as follows: \textit{The eigenstates $\psi$ of a one-electron Hamiltonian defined above for all Bravais lattice translation vectors $\mathbf{T}$ can be chosen to be a plane wave times a function with the periodicity of the Bravais lattice.} We note two things:
\begin{itemize}
	\item This is true for any particle propagating in a lattice
	\item The theorem makes no assumption about the \textit{strength/depth} of the potential. 
\end{itemize}









\hrule 


\noindent \textbf{Notes:} The terminologies in \textcolor{blue}{blue} can be found in \cite{kittel1996introduction} or Wikipedia. The concepts are simple enough, so I won't include their definitions here. 











%%%%%%%%%%%%%%%%%%%%%%%%%%%%%%%%%%%%%%%
\chapter*{C}
%%%%%%%%%%%%%%%%%%%%%%%%%%%%%%%%%%%%%%%
\chapter*{D}
%%%%%%%%%%%%%%%%%%%%%%%%%%%%%%%%%%%%%%%
\chapter*{E}
%%%%%%%%%%%%%%%%%%%%%%%%%%%%%%%%%%%%%%%
\chapter*{F}


\section*{Feshbach Resonance}

%%%%%%%%%%%%%%%%%%%%%%%%%%%%%%%%%%%%%%%
\chapter*{G}
%%%%%%%%%%%%%%%%%%%%%%%%%%%%%%%%%%%%%%%
\chapter*{H}
%%%%%%%%%%%%%%%%%%%%%%%%%%%%%%%%%%%%%%%
\chapter*{I}
%%%%%%%%%%%%%%%%%%%%%%%%%%%%%%%%%%%%%%%
\chapter*{K}
%%%%%%%%%%%%%%%%%%%%%%%%%%%%%%%%%%%%%%%
\chapter*{L}
%%%%%%%%%%%%%%%%%%%%%%%%%%%%%%%%%%%%%%%
\chapter*{M}
%%%%%%%%%%%%%%%%%%%%%%%%%%%%%%%%%%%%%%%
\chapter*{N}
%%%%%%%%%%%%%%%%%%%%%%%%%%%%%%%%%%%%%%%
\chapter*{O}
%%%%%%%%%%%%%%%%%%%%%%%%%%%%%%%%%%%%%%%
\chapter*{P}
%%%%%%%%%%%%%%%%%%%%%%%%%%%%%%%%%%%%%%%
\chapter*{Q}

\section*{Quantum Harmonic Oscillator}



%%%%%%%%%%%%%%%%%%%%%%%%%%%%%%%%%%%%%%%
\chapter*{R}


\section*{Raman side-band cooling}


\section*{Recoil temperature}
%%%%%%%%%%%%%%%%%%%%%%%%%%%%%%%%%%%%%%%
\chapter*{S}
%%%%%%%%%%%%%%%%%%%%%%%%%%%%%%%%%%%%%%%
\chapter*{T}
%%%%%%%%%%%%%%%%%%%%%%%%%%%%%%%%%%%%%%%
\chapter*{U}
%%%%%%%%%%%%%%%%%%%%%%%%%%%%%%%%%%%%%%%
\chapter*{V}
%%%%%%%%%%%%%%%%%%%%%%%%%%%%%%%%%%%%%%%
\chapter*{W}
%%%%%%%%%%%%%%%%%%%%%%%%%%%%%%%%%%%%%%%
\chapter*{X}
%%%%%%%%%%%%%%%%%%%%%%%%%%%%%%%%%%%%%%%
\chapter*{Y}
%%%%%%%%%%%%%%%%%%%%%%%%%%%%%%%%%%%%%%%
\chapter*{Z}
%%%%%%%%%%%%%%%%%%%%%%%%%%%%%%%%%%%%%%%



\newpage

\bibliography{Bui_AMO} 
\bibliographystyle{ieeetr}


\end{document}