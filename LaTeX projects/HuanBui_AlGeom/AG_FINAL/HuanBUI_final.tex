\documentclass[12pt]{article}
\usepackage[margin=1in]{geometry}
%\usepackage{physics}
\usepackage{graphicx}
\usepackage{newpxtext,newpxmath}
%\usepackage{caption}
\usepackage{amsmath}
\usepackage[shortlabels]{enumitem}
\usepackage{amssymb}
%\usepackage{bm}
\usepackage{framed}
%\usepackage{authblk}
\usepackage{hyperref}
\usepackage{xcolor}
\hypersetup{
	colorlinks,
	linkcolor={black!50!black},
	citecolor={blue!50!black},
	urlcolor={blue!80!black}
}
\usepackage{MnSymbol,wasysym}
%\usepackage{empheq}
\usepackage{amsfonts}
%\usepackage{esint}
%\usepackage[makeroom]{cancel}
%\usepackage{dsfont}
%\usepackage{centernot}
%\usepackage{mathtools}
%\usepackage{bigints}
%\usepackage{amsthm}
%\theoremstyle{definition}
%\newtheorem{defn}{Definition}[section]
%\newtheorem{prop}{Proposition}[section]
%\newtheorem{rmk}{Remark}[section]
%\newtheorem{thm}{Theorem}[section]
%\newtheorem{exmp}{Example}[section]
%\newtheorem{prob}{Problem}[section]
%\newtheorem{sln}{Solution}[section]
%\newtheorem*{prob*}{Problem}
%\newtheorem{exer}{Exercise}[section]
%\newtheorem*{exer*}{Exercise}
%\newtheorem*{sln*}{Solution}
%\usepackage{empheq}
%\usepackage{hyperref}
%\usepackage{tensor}
%\usepackage{xcolor}
%\hypersetup{
%	colorlinks,
%	linkcolor={black!50!black},
%	citecolor={blue!50!black},
%	urlcolor={blue!80!black}
%}





%\newcommand*\widefbox[1]{\fbox{\hspace{2em}#1\hspace{2em}}}

\newcommand{\A}{\mathbb{A}}
\newcommand{\p}{\partial}
\newcommand{\R}{\mathbb{R}}
\newcommand{\C}{\mathbb{C}}
\newcommand{\lag}{\mathcal{L}}
\newcommand{\nn}{\nonumber}
\newcommand{\ham}{\mathcal{H}}
\newcommand{\M}{\mathcal{M}}
\newcommand{\I}{\mathcal{I}}
\newcommand{\K}{\mathcal{K}}
\newcommand{\F}{\mathcal{F}}
\newcommand{\w}{\omega}
\newcommand{\lam}{\lambda}
\newcommand{\al}{\alpha}
\newcommand{\be}{\beta}
\newcommand{\x}{\xi}

\newcommand{\G}{\mathcal{G}}

\newcommand{\f}[2]{\frac{#1}{#2}}

\newcommand{\ift}{\infty}

\newcommand{\lp}{\left(}
\newcommand{\rp}{\right)}

\newcommand{\lb}{\left[}
\newcommand{\rb}{\right]}

\newcommand{\lc}{\left\{}
\newcommand{\rc}{\right\}}


\newcommand{\V}{\mathbf{V}}
\newcommand{\U}{\mathcal{U}}
\newcommand{\Id}{\mathcal{I}}
\newcommand{\D}{\mathcal{D}}
\newcommand{\Z}{\mathcal{Z}}

%\setcounter{chapter}{-1}


%\makeatletter
%\renewcommand{\@chapapp}{Part}
%\renewcommand\thechapter{$\bf{\ket{\arabic{chapter}}}$}
%\renewcommand\thesection{$\bf{\ket{\arabic{section}}}$}
%\renewcommand\thesubsection{$\bf{\ket{\arabic{subsection}}}$}
%\renewcommand\thesubsubsection{$\bf{\ket{\arabic{subsubsection}}}$}
%\makeatother



\usepackage{subfig}
\usepackage{listings}
\captionsetup[lstlisting]{margin=0cm,format=hang,font=small,format=plain,labelfont={bf,up},textfont={it}}
\renewcommand*{\lstlistingname}{Code \textcolor{violet}{\textsl{Mathematica}}}
\definecolor{gris245}{RGB}{245,245,245}
\definecolor{olive}{RGB}{50,140,50}
\definecolor{brun}{RGB}{175,100,80}
\lstset{
	tabsize=4,
	frame=single,
	language=mathematica,
	basicstyle=\scriptsize\ttfamily,
	keywordstyle=\color{black},
	backgroundcolor=\color{gris245},
	commentstyle=\color{gray},
	showstringspaces=false,
	emph={
		r1,
		r2,
		epsilon,epsilon_,
		Newton,Newton_
	},emphstyle={\color{olive}},
	emph={[2]
		L,
		CouleurCourbe,
		PotentielEffectif,
		IdCourbe,
		Courbe
	},emphstyle={[2]\color{blue}},
	emph={[3]r,r_,n,n_},emphstyle={[3]\color{magenta}}
}






\begin{document}

\begin{center}
	\LARGE{\textbf{FINAL EXAM}}
	\\
	\large{Huan Q. Bui}
	
	\noindent \hrulefill\\
	\small{MA434: Algebraic Geometry}\\
	\small{May, 2020}\\\vspace{-6pt}
	\hrulefill
\end{center}

 

$\,$\\
$\,$\\


\begin{center}
\begin{tabular}{|c|c|c|}
	\hline
	Problem & Earned & Total\\
	\hline
	1&&10\\
	\hline
	4&&20\\
	\hline
	5&&20\\
	\hline
	6&&10\\
	\hline
	11&&30\\
	\hline
	12&&30\\
	\hline
	\textbf{Total}&$\,\quad\quad$/100&120\\
	\hline
\end{tabular}
\end{center}



\noindent \textbf{References:} For this exam I referred extensively to the Moodle reading guides, videos, and problems, and Reid's \textit{Undergraduate Algebraic Geometry}.

\newpage



%\section*{References}
%
%For this exam, I only referenced Gallian's \textit{Contemporary Abstract Algebra, 8th edition} and my MA434 notes. I used Mathematica to perform some of the calculations. 
%
%
%
%\newpage




\section*{Problem 1 \small{(10 pts)}}
Show that the affine variety in $\A^2$ defined by $xy=1$ is not isomorphic to $\A^1$. \\

\noindent \textit{\underline{Solution}:} Let $V=V(xy-1)\subset \A^2$ denote the affine variety defined by $xy=1$. We know that $V$ is isomorphic to $\A^2$ if and only if the coordinate rings $k[V]$ and $k[\A^2] = k[t]$ are also isomorphic. It suffices to show $k[V] = k[x,y]/\langle xy-1\rangle$ is \textbf{not} isomorphic to $k[t]$. To this end, consider any ring homomorphism $\Phi: k[x,y] \to k[t]$ with $\ker\Phi = \langle xy-1\rangle$. This is possible because every ideal $I$ of a ring $R$ is the kernel of some ring homomorphism of $R$.\\

Let $\Phi(x) = \alpha(t) \in k[t]$ and $\Phi(y) = \beta(t)\in k[t]$, then 
\begin{align*}
0 = \Phi(xy-1) = \Phi(x)\Phi(y)-1 = \alpha(t)\beta(t) - 1. 
\end{align*}
This means 
\begin{align*}
\alpha(t)\beta(t) = 1.
\end{align*}
Since $\alpha(t)$ and $\beta(t)$ must be polynomials in $t$, the equation holds only if $\alpha(t)$ and $\beta(t)$ are elements of $k$, which are just constants. Now, since $\Phi$ is a ring homomorphism, for any $f\in k[V]$, $\Phi(f)$ must also be a constant (in $k$). So, $\Phi$ is not surjective, i.e., $\Phi(k[x,y])$ is not isomorphic to $k[V]$. \\

The first isomorphism theorem for rings says $k[x,y]/\langle xy-1\rangle$ is isomorphic to $\Phi(k[x,y])$. But we just showed $\Phi(k[x,y])$ is not isomorphic to $k[t]$, so $k[V]$ is not isomorphic to $k[t]$. This implies $V$ is not isomorphic to $\A^1$.   \hfill$\square$


\newpage



















%\section*{Problem 2 \small{(20 pts)}}
%
%Given two affine varieties $V$ and $W$ and a polynomial map $f: V\to W$, we get a ring homomorphism $f^* : k[W] \to k[V]$. When is $f^*$ injective?\\
%
%\noindent \textit{\underline{Solution}:} 




\newpage
















\section*{Problem 4 \small{(20 pts)}}
Suppose that $f$ is a rational function on $\mathbb{P}^1$.
\begin{enumerate}[(a)]
	\item Show that if $f$ is regular at every point of $\mathbb{P}^1$ then it is constant. (Hint: consider two affine pieces $\A_{(0)}^1$ and $\A^1_{(1)}$.)
	
	
	\item Show that there are no non-constant morphisms $\mathbb{P}^1 \to \A^m$.
\end{enumerate}



\noindent \textit{\underline{Solution}:}

\begin{enumerate}[(a)]
	\item Let $f \in k(\mathbb{P}^1)$ be given such that $f$ is regular at every point in $\mathbb{P}^1$. From the last exam/the beginning of chapter 5, we know that $\mathbb{P}^1$ can be thought of as two copies of $\A^1$ glued together. Call $x_0,y_1$ the coordinates of the two $\A^1$, respectively. The ``glueing'' action is given by the isomorphism $\A^1_{(0)} - \{x_0 = 0\} \to \A^1_{(1)}-\{y_1=0\}$:
	\begin{align*}
	x_0 \mapsto y_1 = \f{1}{x_0}
	\end{align*}
	Explicitly, $\mathbb{P}^1 = \A^1_{(0)} \cup \A^1_{(1)}$ where
	\begin{align*}
	\A^1_{(0)} = \A^1 - (x_0 = 0), \quad \A^1_{(1)} = \A^1 - (y_1 =0).
	\end{align*} 
	Applying theorem 4.8 (II) (which says $\mbox{dom}(f) = V \iff f \in k[V]$) to the affine piece $\A^1_{(0)}$, we get $f = p(x_0) \in k[x_0]$. Applying theorem 4.8 (II) to the affine piece $\A^1_{(1)}$ and applying the ``change of variables'' $x_0 = 1/y_1$ we get $f = p(1/y_1) \in k[y_1]$. Now, the only way $p(1/y_1)$ can be a polynomial is that $p$ is a constant. So, $f$ is constant. 
	
	
	
	\item  From the previous item we should able to deduce that there are no non-constant morphisms $\mathbb{P}^1 \to \A^m$. A morphism $f$ on $\mathbb{P}^1$ must have $\mathbb{P}^1 \subset \mbox{dom}(f) $. This means $f$ is regular at every point in $\mathbb{P}^1$. $f$ is also rational map (because it is a morphism). So, we conclude $f$ must be constant, i.e., there are no non-constant morphisms $\mathbb{P}^1 \to \A^m$. 	
\end{enumerate}
\hfill $\square$


















\newpage





\section*{Problem 5 \small{(20 pts)}} 
Below are three formulas that possibly define rational maps $f: \mathbb{P}^2 \dashrightarrow \mathbb{P}^2$. Decided whether the formulas do define rational maps. If they do, determine $\mbox{dom}(f)$ and decide whether $f$ is birational. 
\begin{enumerate}[(i)]
	\item $f([x:y:z]) = [1/x:1/y:1/z]$
	\item $f[(x:y:z)] = [x:y:1]$
	\item $f([x:y:z]) = [(x^3+y^3)/z^3 : y^2/z^2 : 1]$.
\end{enumerate}

Rationals maps must be ratio(s) of homogeneous polynomials of the same degree. On first glance we see that (ii) does not define a rational map because there is no way to write its output into ratios of homogeneous polynomials of the same degree: 
\begin{align*}
[x:y:1] = [1:y/x:1/x] = [x/y:1:1/y].
\end{align*}
On the other hand, the outputs in (i) and (iii) can be written in the desired forms:
\begin{align*}
\left[\frac{1}{x}:\f{1}{y}:\f{1}{z}\right] = \left[1:\f{x}{y}:\f{x}{z}\right] = \left[\f{y}{x}:1:\f{y}{z}\right] = \left[\f{z}{x}:\f{z}{y}:1\right] = \dots
\end{align*}
and
\begin{align*}
\left[ \f{x^3 + y^3}{z^3} : \f{y^2}{z^2} :  1 \right] = \left[ \f{x^3 + y^3}{zy^2} : 1 : \f{z^2}{y^2} \right] = \left[ 1 : \f{y^2z}{x^3+y^3} : \f{z^3}{ x^3 + y^3} \right] = \dots
\end{align*}
Now we want to find $\mbox{dom}(f)$ for (i) and (iii). By definition, 
\begin{align*}
\mbox{dom}(f) = \{ P\in \mathbb{P}^2 \vert f \mbox{ is regular at } P \}.
\end{align*}

\noindent \textbf{Find the domain:}
\begin{itemize}
\item For (i), clearly $f$ is regular at all points with $x,y,z \neq 0$. Without loss of generality, suppose $x=0$ and $y,z\neq 0$ then we write the output as $[1:x/y:x/z] = [1:0:0] $. So $f$ is also regular there. Similarly, we can see $f$ is also regular at $[x:y:z]$ where only $z=0$ and only $y=0$. However, when two of $x,y,z$ are zero, $f([x,y,z])$ is no longer defined. So, for (i),
\begin{align*}
\boxed{\mbox{dom}(f_{(i)}) = \mathbb{P}^2 - \{ [1,0,0],[0,1,0],[0,0,1] \}}
\end{align*} 

\item For (iii), we are interested in cases where $z = 0$,  $y=0$, and $x^3 + y^3 = 0$. By writing the output of $f$ in different forms above, we see that $f$ is still regular at $[x:y:z]$ where only \textbf{one} of the possibilities $z = 0$,  $y=0$, or $x^3 + y^3 = 0$ occurs, or if only $z=y=0$, $x=y=x^3+y^3=0$ occurs. However, since we have the factor $[(x^3+y^3)/z]^{\pm 1}$ in all of the three representations of the output of $f$, we see that $f$ fails to be regular when $z=0$ and $x^3 + y^3 = 0$. So, for (iii),
\begin{align*}
\boxed{\mbox{dom}(f_{(iii)}) = \mathbb{P}^2 - \{[-1:1:0]\}}
\end{align*} 
\end{itemize}


\noindent \textbf{Birational?} Next, $f: \mathbb{P}^2 \dashrightarrow \mathbb{P}^2$ is \textit{birational} if there exists a rational (inverse) map $g: \mathbb{P}^2 \dashrightarrow \mathbb{P}$ such that $f\circ g = \mbox{id}_{\mathbb{P}^2}$ and $g\circ f = \mbox{id}_{\mathbb{P}^2}$. 
\begin{itemize}
\item For (i), we consider the rational function $g : \mathbb{P}^2 \dashrightarrow \mathbb{P}^2$ defined by $g([u:v:w]) = [1/u:1/v:1/w]$. So, $g$ is just $f$. For $[u:v:w] \in \mbox{dom}(g) = \mbox{dom}(f)$, we have
\begin{align*}
f\circ g([u:v:w]) = f([1/u:1/v:1/w]) = [u:v:w].
\end{align*}
for all $[u:v:w]$ in $\mbox{dom}(g) = \mbox{dom}(f)$. Similarly, $g\circ f$ is also the identity function on $\mbox{dom}(f) = \mbox{dom}(g)$. Finally, since $f$ and $g$ are really the same function, it remains to show $f$ is dominant. By definition, $f$ is dominant if $f(\mbox{dom}(f))$ is dense in $\mathbb{P}^2$. This is equivalent to saying $f(\mbox{dom}(f)) \cap \mathcal{O} \neq \emptyset$ for any nonempty open set $\mathcal{O} \subset \mathbb{P}^2$. It is clear that the output of $f$ is not only all tuples $[1/x:1/y:1/z]$ with $x,y,z\neq 0$ but also $[1:0:0], [0:1:0], [0:0:1]$: 
\begin{align*}
f([1:1:0]) = [0:0:1], \quad f([1:0:1]) = [0:1:0], \quad 
f([0:1:1]) = [1:0:0].
\end{align*}  
So, $f(\mbox{dom}(f)) = \mathbb{P}^2$. It follows that $f$ is dominant. With the dominant rational inverse $g$ (which is just $f$ itself), we conclude that $f$ is birational. 


$(\dagger)$ Alternatively, we can see that the induced $k$-algebra homomorphism $f^*: k(\mathbb{P}^2) \to k(\mathbb{P}^2)$ given by $g \mapsto g\circ f$ is an isomorphism. This (I believe) is easy to see because for any $g\in k[\mathbb{P}^2]$, if $g([1/x:1/y:1/z]) = 0$ then $g =0$ necessarily (because the factors $1/x, 1/y,1/z$ are in some sense ``independent''), which shows $f^*$ is injective. Further, any element of $k(\mathbb{P}^2)$ (which has the form of a ratio of two homogeneous polynomials of the same degree) can be put into the form $g\circ f$ where $g\in (\mathbb{P}^2)$. So $f^*$ is an isomorphism. This combined with the fact that $f$ is dominant is equivalent to $f$ being birational.  


\item For (iii), we claim that $f$ is not birational. This is because the induced $k$-algebra homomorphism $f^* : k(\mathbb{P}) \to k(\mathbb{P}^2)$ is \textbf{not} onto (hence not an isomorphism).  Consider the element $x/y \in k(\mathbb{P}^2)$. There is no $g\in k(\mathbb{P}^2)$ such that $g\circ f[x:y:z] = x/y$ because $x$ always appears as $x^3$ in the output of $f$. We conclude $f$ is not birational.
\end{itemize}
\hfill $\square$












\newpage


\section*{Problem 6 \small{(10 pts)}}
Prove statements (i), (ii), (iii), (iv) from Example I from section 5.7 of \textit{Undergraduate Algebraic Geometry}\\

\noindent \textit{\underline{Solution}:} Define $f : \mathbb{P}^1 \to \mathbb{P}^m$ by 
\begin{align*}
[U:V] \mapsto [U^m : U^{m-1}V : \dots : V^m]
\end{align*}
\begin{enumerate}[(i)]
	\item $f$ is a rational map: We notice that while $U^i V^j$ are it rational functions (since they are not given by a ratio of homogeneous polynomials of the same degree), we can re-write the definition of $f$ as
	\begin{align*}
	[U:V] \stackrel{f}{\mapsto} \left[ \f{U^m}{V^m} : \f{U^{m-1}}{V^{m-1}}: \dots : 1 \right].
	\end{align*}
	Now, each component $f_i$ is a rational function, so we have a rational map.
	
	
	\item $f$ is a morphism: $f$ is a morphism if we can show $\mathbb{P}^1 \subset \mbox{dom}(f)$, i.e., $f$ is regular at every point of $\mathbb{P}^1$. If $V\neq 0$ then there's nothing to prove because of the formula we just wrote down. If $U\neq 0$ then we can just rewrite the definition of $f$ as
	\begin{align*}
	[U:V] \stackrel{f}{\mapsto} \left[1 : \f{V}{U} : \dots  : \f{V^m}{U^m}\right]
	\end{align*} 
	which shows that $f$ is also regular at these points. When $U,V\neq 0$, there's nothing to worry about. So, $f$ is indeed regular at every point in $\mathbb{P}$, i.e., $f : \mathbb{P}^1 \to \mathbb{P}^m$ is a morphism.
	
	
	\item The image of $f$ is the set of points $[X_0:\dots : X_m] \in \mathbb{P}^m$ such that 
	\begin{align*}
	[X_0 : X_1] = [X_1 : X_2] = \dots = [X_{m-1} : X_m]
	\end{align*} 
	that is 
	\begin{align*}
	X_0 X_2 = X_1^2;\quad X_0X_3 = X_1 X_2; \quad X_0 X_4 = X_1X_3; \quad \mbox{etc}.
	\end{align*}
	We notice that for every input $[U:V]$, the output looks like
	\begin{align*}
	[X_0 : X_1 : \dots : X_m] = [U^m : U^{m_1}V : \dots V^m]
	\end{align*}
	So, we have that 
	\begin{align*}
	&[X_0 : X_1] = [U^m : U^{m-1}V] = [U:V]\\
	&[X_1 : X_2] = [U^{m-1}V : U^{m-2}V^2] = [U:V]
	\end{align*}
	and so forth. So, we end up with 
	\begin{align*}
	[X_0 : X_1] = [X_1 : X_2] = \dots = [X_{m-1} : X_m]
	\end{align*}
	From here it is not hard to generalize:
	\begin{align*}
	[X_0 : X_1] = [X_{n-1} : X_n]
	\end{align*}
	and so we have a chain of equalities $X_0 X_n = X_1 X_{n-1}$ for different values of $n$. This means any $2\times 2$ matrix of the form
	\begin{align*}
	\begin{bmatrix}
	X_0 & X_{n-1}\\ X_1 & X_n
	\end{bmatrix}
	\end{align*}
	has vanishing determinant. This leads to the condition
	\begin{align*}
	\mbox{rank}\begin{bmatrix}
	X_0 & X_1 & X_2 & \dots & X_{m-1}\\
	X_1 & X_2 & X_3 & \dots & X_{m}
	\end{bmatrix} \leq 1.
	\end{align*}
	This condition coincides exactly with the all-vanishing determinant condition above: If the matrix rank is zero, the matrix is the zero matrix, in which case there is nothing interesting (in fact this case won't happen because at least one $X_i$ has to be nonzero -- $[X_0 : \dots : X_m] \in \mathbb{P}^m$). If the matrix has rank one, then one row is a constant multiple of the other. After writing, say, the first row as some multiple of the second row, we see that any $2\times 2$ minor has the form $a(X_n X_m - X_m X_n)$, which vanishes identically. When the matrix has rank 2, the both rows are linearly independent, and we no longer have the vanishing $2\times 2$ minor condition.
	
	
	\item There is an inverse morphism $g: C\to \mathbb{P}^1$. The inverse morphism takes a point of $C$ into the common ratio:
	\begin{align*}
	[X_0 : \dots : X_m] \stackrel{g}{\mapsto} [X_0 : X_1]
	\end{align*}
	where $[X_0 : X_1]$ is ``common'' in the sense of the previous item. We want to check that this is actually a morphism, i.e., it is a rational map that is regular at every point in $C$. Clearly, we can write
	\begin{align*}
	[X_0 : \dots : X_m] \stackrel{g}{\mapsto} \left[1 : \f{X_1}{X_0}\right] \mbox{ or } \left[\f{X_{m-1}}{X_m} : 1\right]
	\end{align*}
	depending on whether $X_1 =0$ or $X_0 = 0$ (or both). In any case, we see that $g$ is a rational function (as given by ratios of homogeneous polynomials of the same degree) that is regular at every point on $C$. \hfill $\square$ 
	
	
	
	
\end{enumerate}


\newpage







\section*{Problem 11 \small{(30 pts)}}
Given an invertible matrix $A = \begin{bmatrix}
a&b\\c&d
\end{bmatrix}$ with complex coefficients, define a function $f_A : \mathbb{P}^1_{\mathbb{C}} \to \mathbb{P}^1_{\mathbb{C}}$ by
\begin{align*}
f_A ([u:v]) = [au + bv : cu + dv].
\end{align*}
\begin{enumerate}[(a)]
	\item Show that $f_A$ is a morphism. 

	\item How does $f_{AB}$ relate to $f_A$ and $f_B$? 
	
	\item Show that $f_A$ has an inverse morphism, so that $f_A$ defines an automorphism of $\mathbb{P}^1_{\mathbb{C}}$. 
	
	 
	\item If we identify $\mathbb{C}$ with the standard $\A^1 \subset \mathbb{P}^1$ defined by $v\neq 0$, show that the restriction of $f_A$ to $\mathbb{C}$ is a rational function, and find its formula.  
\end{enumerate}


\noindent \textit{\underline{Solution}:}

\begin{enumerate}[(a)]
	\item $f_A$ is a morphism if $\mathbb{P}^1_{\mathbb{C}} \subset \mbox{dom}(f)$, i.e., $f$ is regular at every point in $\mathbb{P}^1_{\mathbb{C}}$, i.e., $au+bv$ and $cu+dv$ are never simultaneously zero for any $u,v$. Now, we don't have the possibility $u=v=0$ because $[u:v] \in \mathbb{P}^1_{\mathbb{C}}$. So, $au+bv = 0 = cu+dv$ for some pair $u,v$ if and only if $\det(A) = 0$. But this never happens because $A$ is invertible. So, $f$ is regular at every point $[u:v] \in \mathbb{P}^1_{\mathbb{C}}$, i.e., $f$ is a morphism. 
	
	
	\item We claim that $f_{AB} = f_A \circ f_B$. Let an invertible matrix $B$ be given, 
	\begin{align*}
	B= \begin{bmatrix}
	a' & b' \\ c' & d'
	\end{bmatrix} \implies AB = \begin{bmatrix}
	a&b\\c&d
	\end{bmatrix}\begin{bmatrix}
	a'&b'\\c'&d'
	\end{bmatrix} =  \begin{bmatrix}
	aa' + bc' & ab' + bd' \\ ca' + dc' & cb' + dd'
	\end{bmatrix}.
	\end{align*}
	Then 
	\begin{align*}
	f_{AB}([u:v]) &= [(aa'+bc')u + (ab'+bd')v : (ca'+dc')u + (cb'+dd')v]\\
	&= [a(a'u + b'v) + b(c'u+d'v) :  c(a'u + b'v) + d(c'u+d'v)]\\
	&= f_A[a'u + b'v : c'u+d'v]\\
	&= f_A\circ f_B([u:v]).
	\end{align*}
	
	\item To show that $f_A$ has an inverse morphism, it suffices to construct one. Consider $f_{A^{-1}}$ defined by $A^{-1}$, the matrix inverse of $A$:
	\begin{align*}
	A^{-1} = 
	\f{1}{ad-bc}\begin{bmatrix}
	d & -b \\ -c & a
	\end{bmatrix}.
	\end{align*}
	We know that $\det(A) \neq 0$, so $A^{-1}$ exists. Also, since the scaling factor $1/\det(A) \neq 0$ appears at every entry of $A^{-1}$, we can ignore it in the definition of $f_{A^{-1}}$:
	\begin{align*}
	f_{A^{-1}}([u:v]) &= \left[\f{d}{\det(A)}u + \f{-b}{\det(A)}v : \f{-c}{\det(A)}u + \f{a}{\det(A)}v \right] \\
	&= [du-bv : -cu+av].
	\end{align*}
	Next we check that $f_{A}$ and $f{A^{-1}}$ are inverses. By the previous item, we know that $f_{A^{-1}A} = f_{AA^{-1}} = f_{I}$ where $I$ is the $2\times 2$ identity matrix. Since 
	\begin{align*}
	f_I([u:v]) = [u + 0v : 0u+v] = [u:v],
	\end{align*} 
	$f_{A}$ and $f_{A^{-1}}$ are inverses. So, $f_{A}$ is an isomorphism from (the entire) $\mathbb{P}^1_{\mathbb{C}}$ to itself. This makes $f_A$ an automorphism. 
	
	
	
	\item We want to look at $f_A : \C \to \C$ where $\mathbb{C}$ is identified with the standard $\A^1 \subset \mathbb{P}^1$ defined by $v\neq 0$ (here the restriction is at both ends). We want to show that $f_A$ in this case is a rational function and find its formula. Now, when $v\neq 0$, we can write the input $[u:v]$ as $[u/v:1] = [t:1]$ where $t\in \mathbb{C}$. With this, 
	\begin{align*}
	f_A([t:1]) = [at+b : ct+d] = \left[\f{at+b}{ct+d} : 1\right].
	\end{align*} 	 
	Restricting both ends to $\C$, we can identify a rational function $f: \mathbb{C} \to \mathbb{C}$ defined by
	\begin{align*}
	f(t) = \f{at+b}{ct+d}.
	\end{align*}
	This is a function from $\C$ to $\C$ (or equivalently from $\A^1 \to \A^1$). Also, because it is a ratio of polynomials, it is a rational function. 
\end{enumerate}
\hfill $\square$












\newpage







\section*{Problem 12 \small{(30 pts)}}
Show that any automorphism of $\mathbb{P}^1_\C$ is of the form $f_A$ as in the previous problem. \\


\noindent \textit{\underline{Solution}:} Let an automorphism $f$ on $\mathbb{P}^1_\C$ be given. It is an automorphism so it is an isomorphism - a morphism with an inverse morphism. This means 
\begin{align*}
f([u:v]) = [f_1(u,v) : f_2(u,v)]
\end{align*}
where $f_1,f_2$ are necessarily ratios of homogeneous polynomials of the same degree. We look at two cases:  either $f$ maps the point at infinity to the point at infinity, i.e., $f([1:0]) = [1:0]$, or to some point not at infinity - without loss of generality assume this point is $f([1:0]) = [\epsilon:1]$ where $\epsilon \in \C$. 


\begin{itemize}
\item If $f$ maps the point at infinity to the point at infinity, i.e., $f([1:0]) = [1:0]$, then because $f$ is an isomorphism, it must map any ``regular'' point to a ``regular point.'' This means we can make the restriction (at both ends, with $v\neq 0, f_2(u,v)\neq 0$) so that $f\vert_\C : \C \to \C$, with $t\mapsto f\vert_\C(t)$. With this we have
\begin{align*}
f([t:1]) = [f\vert_\C(t) : 1],
\end{align*}
where $f\vert_\C$ must be defined for all $t\in \C$, is bijective in $\C$, and must be a ratio of homogeneous polynomials of the same degree. For such $f\vert_\C$ to be defined for all $t\in \C$, $f$ is necessarily a polynomial (a non-constant denominator always has roots - not good). If this polynomial has degree 0 or greater than 1 then it fails to be bijective. So, $f\vert_\C$ is a polynomial of degree 1. With this, we write, for $a, b \in \C$, $a \neq 0$:
\begin{align*}
f([t:1]) = [a t + b : 1].
\end{align*}
We see that when we write the input as $[u:v] \in \mathbb{P}^1_\C$ where $[u:v] = [1:0]$ or $[u:v] = [u/v:1] = [t:1]$, we can write the output of this $f$ as
\begin{align*}
f([u:v]) = [au+ bv : cv], \quad c\neq 0
\end{align*}
which captures $[1:0]\mapsto [1:0]$ as well. We notice that the output can never have the form $[0:0]$. This corresponds exactly to 
\begin{align*}
\det\begin{bmatrix}
a & b \\ 0 & c
\end{bmatrix} \neq 0.
\end{align*}




\item If $f$ maps the point at infinity to a ``regular'' point $[\epsilon:1]$, then we can send this point back to the point at infinity using a known automorphism $g$. The composition $g\circ f$ is now an automorphism that sends $[1:0]$ to $[1:0]$. By the previous item, we know the form $g\circ f$ takes. To find the form of $f$, we want to find the form of $g$. To do this, we look at
\begin{align*}
g\circ f([1:0]) = g([\epsilon : 1]) = [1:0].
\end{align*} 
Take 
\begin{align*}
g([u:v]) = [v : u - \epsilon v].
\end{align*}
The matrix associated with $g$ is 
\begin{align*}
G = \begin{bmatrix}
0 & 1 \\ 1 & -\epsilon
\end{bmatrix}.
\end{align*}
We see that $\det(G) = -1 \neq 0$, so by the previous problem we know $g$ is indeed an automorphism on $\mathbb{P}^1_\C$. Now, the form of $g\circ f$, by the previous item, is 
\begin{align*}
g\circ f([u:v]) = [cu + dv : ev] = [f_2(u/v) : f_1(u/v) - \epsilon f_2(u/v)].
\end{align*}
where we are taking $v\neq 0$. Call $u/v= t$, then because $g\circ f$ only maps the point at infinity to the point at infinity, we know that $f_2(u/v)$ must be a polynomial of degree one in $u/v$ (by our previous argument). This means $f_1(u/v)$ is of degree one as well. After homogenizing, we have
\begin{align*}
f([u:v]) = [au+bv : cu + dv].
\end{align*}
Finally, we want conditions on $a,b,c,d$ such that $f$ is actually an automorphism. $f$ fails to be an automorphism exactly when the matrix $F = \begin{bmatrix}
a&b\\c&d
\end{bmatrix}$ is not invertible. So $f$ is an automorphism exactly when $\det(F) \neq 0$.
\end{itemize}

In either case, we have shown that any automorphism of $\mathbb{P}^1_\C$ is of the form $f_A$ as in the previous problem. \hfill $\square$   




 




	
\end{document}