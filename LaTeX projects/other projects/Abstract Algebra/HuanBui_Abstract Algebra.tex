\documentclass{book}
\usepackage{physics}
\usepackage{amsmath}
\usepackage{authblk}
\usepackage{amsfonts}
\usepackage{esint}
\usepackage{bbold}
\usepackage{mathtools}
\usepackage{dsfont}
\usepackage{amsthm}
\usepackage{bbm}
\usepackage{amssymb}
\theoremstyle{definition}
\newtheorem{defn}{Definition}[section]
\newtheorem{prop}{Properties}[section]
\newtheorem{rmk}{Remark}[section]
\newtheorem{exmp}{Example}[section]
\newtheorem{prob}{Problem}[section]
\newtheorem{sln}{Solution}[section]
\newtheorem{thm}{Theorem}[section]
\newtheorem*{prob*}{Problem}
\newtheorem*{sln*}{Solution}
\usepackage{empheq}
\usepackage{tensor}



\usepackage{hyperref}
\usepackage{xcolor}
\hypersetup{
	colorlinks,
	linkcolor={black!50!black},
	citecolor={blue!50!black},
	urlcolor={blue!80!black}
}
%\usepackage{mathabx}

\newcommand{\R}{\mathbb{R}}

\newcommand{\f}[2]{\frac{#1}{#2}}

\newcommand{\F}{\mathcal{F}}
\newcommand{\p}{\partial}

\newcommand{\G}{\mathcal{G}}
\newcommand{\C}{\mathbb{C}}
\newcommand{\Uni}{\mathcal{U}}

\newcommand{\V}{\mathbf{V}}
\newcommand{\W}{\mathbf{W}}
\newcommand{\Z}{\mathbf{Z}}
\newcommand{\Y}{\mathbf{Y}}
\newcommand{\U}{\mathbf{U}}
\newcommand{\X}{\mathbf{X}}

\newcommand*{\operp}{\perp\mkern-20.7mu\bigcirc}

\newcommand{\A}{\mathcal{A}}
\newcommand{\B}{\mathcal{B}}

\newcommand{\xpan}{\text{span}}

\newcommand{\lag}{\mathcal{L}}

\newcommand{\J}{\mathbf{J}}

\newcommand{\M}{\mathcal{M}}

\newcommand{\K}{\mathcal{K}}

\newcommand{\N}{\mathcal{N}}

\newcommand{\E}{\mathcal{E}}

\newcommand{\ima}{\text{Im}}
\newcommand{\lin}{\overset{\text{linear}}{\longrightarrow}}
\newcommand{\T}{\mathcal{T}}
\newcommand{\poly}{\mathbb{P}}
\newcommand{\s}{\mathcal{S}}

\newcommand{\jor}{\mathcal{J}}
\newcommand{\FF}{\mathfrak{F}}
\newcommand{\LL}{\mathfrak{L}}
\newcommand{\lat}{\mathfrak{Lat}}

\newcommand{\gives}{\rotatebox[origin=c]{180}{$\Rsh$}	}

\newcommand{\la}{\langle}
\newcommand{\ra}{\rangle}

\newcommand{\lp}{\left(}
\newcommand{\rp}{\right)}

\newcommand{\lb}{\left[}
\newcommand{\rb}{\right]}


\newcommand{\id}{\mathcal{I}}

\newcommand{\bigzero}{\mbox{\normalfont\Large\bfseries 0}}
\newcommand{\rvline}{\hspace*{-\arraycolsep}\vline\hspace*{-\arraycolsep}}



\usepackage{subfig}
\usepackage{listings}
\captionsetup[lstlisting]{margin=0cm,format=hang,font=small,format=plain,labelfont={bf,up},textfont={it}}
\renewcommand*{\lstlistingname}{Code \textcolor{violet}{\textsl{Mathematica}}}
\definecolor{gris245}{RGB}{245,245,245}
\definecolor{olive}{RGB}{50,140,50}
\definecolor{brun}{RGB}{175,100,80}
\lstset{
	tabsize=4,
	frame=single,
	language=mathematica,
	basicstyle=\scriptsize\ttfamily,
	keywordstyle=\color{black},
	backgroundcolor=\color{gris245},
	commentstyle=\color{gray},
	showstringspaces=false,
	emph={
		r1,
		r2,
		epsilon,epsilon_,
		Newton,Newton_
	},emphstyle={\color{olive}},
	emph={[2]
		L,
		CouleurCourbe,
		PotentielEffectif,
		IdCourbe,
		Courbe
	},emphstyle={[2]\color{blue}},
	emph={[3]r,r_,n,n_},emphstyle={[3]\color{magenta}}
}


\begin{document}
	\begin{titlepage}\centering
		\clearpage
		\title{\textsc{\bf{ABSTRACT ALGEBRA}}\\\smallskip - A Quick Guide -\\}
		\author{\bigskip Huan Q. Bui}
		 \affil{Colby College\\$\,$\\ PHYSICS \& MATHEMATICS\\ Statistics \\$\,$\\Class of 2021\\}
		\date{\today}
		\maketitle
		\thispagestyle{empty}
	\end{titlepage}

\newpage

\section*{Preface}
\addcontentsline{toc}{subsection}{Preface}

Greetings,\\

\textit{Abstract Algebra: A Quick Guide to} is compiled based on my MA333: Abstract Algebra notes with professor Tamar Friedmann. This guide is almost entirely based on \textit{Contemporary Abstract Algebra, Fourth edition} by Gallian and my class notes with professor Friedmann. \\

Enjoy!


\newpage
\tableofcontents
\newpage

\chapter{GROUPS}


\section{Definition}
A group is always defined \textit{under} some binary operation. What is a binary operation? Let a set $G$ be given. A binary operation on $G$ is a function that assigns to each ordered pair of elements of $G$ an element of $G$. \\

A group, then is defined as follows. Let a (nonempty) set $G$ be given with a binary operation that assigns to each ordered pair $(a,b)$ where $a,b\in G$ an element $ab \in G$. $G$ is a group under this operation if the following properties are satisfied:
\begin{enumerate}
	\item \textit{Associativity:} The operation is associative, i.e, $a(bc) = (ab)c$.
	\item \textit{Identity:} There exists an element $e$, called the identity, in $G$ such that $ae = ea = a \forall a \in G$. 
	\item \textit{Inverses:} $\forall a \in G, \exists b \in G$ s.t. $ab = ba  =e$. We call $b$ the inverse of $a$, denoted $a^{-1}$. 
\end{enumerate}

We note that the binary operation associated with each group is not necessarily commutative. A commutative group is called \textit{Abelian}. A non-commutative group is called \textit{non-Abelian}. 



\section{Elementary Properties}

\subsection{Uniqueness of Identity}

In a group $G$, there is only one identity element. The proof of this is quite simple. Let a group $G$ be given. Suppose $ae = a$ and $ae' = e'a = a$ for all $a \in G$. Then we have $ae = ea = a = ae' = ea'$. If $a = e$ then we immediately have $ee' = e = e'e = e'$. So $e = e'$. Thus, the identity is unique.

\subsection{Cancellation}
In a group $G$, the right and left cancellation laws hold, i.e, $ab = ac \implies b = c$, and $ba = ca \implies b = c$. The proof of this is also quite simple. We simply multiply both sides of each equation from the appropriate direction with $a^{-1}$. By associativity, the $a$ vanishes from both sides, leaving $b=c$. 

\subsection{Uniqueness of Inverses}

For each element $a$ in a group $G$, there is a unique element $b$ in $G$ such that $ab  = ba = e$. We once again prove by supposing there are two distinct inverses of $a$, say $b$ and $b'$. Then we have $ab = ab' = e$. By cancellation, we have $b = b'$. 

\subsection{Socks-Shoes Property}

\begin{align}
(ab)^{-1} = b^{-1}a^{-1}.
\end{align}












\newpage





\chapter{FINITE GROUPS \& SUBGROUPS}

















\end{document}
