\documentclass{article}
\usepackage{physics}
\usepackage{graphicx}
\usepackage{caption}
\usepackage{amsmath}
\usepackage{authblk}
\usepackage{amsfonts}
\usepackage{esint}
\usepackage{mathtools}
\usepackage{amsthm}
\theoremstyle{definition}
\newtheorem{defn}{Definition}[section]
\newtheorem{prop}{Proposition}[section]
\newtheorem{rmk}{Remark}[section]
\newtheorem{exmp}{Example}[section]
\usepackage{empheq}
\usepackage{hyperref}
\usepackage{tensor}
\usepackage{xcolor}
\hypersetup{
	colorlinks,
	linkcolor={black!50!black},
	citecolor={blue!50!black},
	urlcolor={blue!80!black}
}

\begin{document}
\begin{titlepage}\centering
 \clearpage
 \title{\textsc{\bf{PARTIAL\\ DIFFERENTIAL EQUATIONS}}\\\smallskip A Quick Guide\\}
 \author{\bigskip Huan Q. Bui}
 \affil{Colby College\\Physics \& Statistics\\Class of 2021\\}
 \date{\today}
 \maketitle
 \thispagestyle{empty}
\end{titlepage}

\subsection*{Preface}
\addcontentsline{toc}{subsection}{Preface}

Greetings,\\

\textit{Partial Differential Equations: A Quick Guide} is based on my lecture notes from MA411: Topics in Differential Equations - Partial Differential Equations with professor Evan Randles at Colby. The contents are somewhat based on Farlow's \textit{Partial Differential Equations for Scientists and Engineers}.\\	

Enjoy!

\newpage
\tableofcontents
\newpage

\section{Overview and Classification}

%\date{Feb 6, 2019}


\subsection{What in the world is a PDE?}
We shall begin with what PDEs are. 
\begin{defn}
	A partial differential equation (PDE) is an equation relating a function of several variables $\psi(t,\vec{x})$ to its partial derivatives: $\partial_{x_1}\psi$, $\partial^2_{x_1x_2}\psi$, etc.\\
	
	A note on notation:
	\begin{align*}
	\frac{\partial^2 \psi}{\partial x_1\,\partial x_2} \equiv \partial^2_{x_1x_2}\psi \equiv \partial_{x_1}\partial_{x_2}\psi.
	\end{align*}
\end{defn}

\subsection{Some notable examples}
Let us look at a couple of famous PDEs:
\begin{exmp}
	\textbf{Laplace Equation:}
	\begin{align*}
	\Delta \psi = \nabla^2\psi = \frac{\partial^2 \psi}{\partial x^2} + \frac{\partial^2 \psi}{\partial y^2} + \frac{\partial^2 \psi}{\partial z^2} = 0.
	\end{align*}
\end{exmp}

\begin{exmp}
	\textbf{Poisson's Equation:}
	\begin{align*}
	\Delta \psi = \nabla^2 \psi = F(x,y,z)
	\end{align*}
\end{exmp}

We take note of the \textbf{Laplacian} or the \textbf{Laplacian operator}:
\begin{align*}
\boxed{\Delta \psi \equiv \nabla^2 \psi = \frac{\partial^2 \psi}{\partial x^2} + \frac{\partial^2 \psi}{\partial y^2} + \frac{\partial^2 \psi}{\partial z^2}}
\end{align*}
The Laplacian operator takes a function $\psi$ linearly to another function $\nabla^2 \psi$. The Laplacian is one of the most important objects in mathematics, as it touches probability theory, potential theory, partial differential equations, mathematical physics, harmonic analysis, number theory, etc.\\

Another note on notation: the symbols $\Delta$ and $\nabla^2$ will be used interchangeably in this text. The $\nabla^2$ represents the divergence of the gradient.\\

Let us look at some more examples to see the ubiquity of the Laplacian in PDEs:
\begin{exmp}
	\textbf{The heat equation:}
	\begin{align*}
	\frac{\partial \psi}{\partial t} = \nabla^2 \psi.
	\end{align*}
	The heat equation describes heat transfer over time. But there is also a connection between the heat equation and probability theory. In particular, the Gaussian function:
	\begin{align*}
	\frac{1}{\sqrt{4\pi t}}e^{-\frac{x^2}{4t}}
	\end{align*}
	solves the heat equation.
\end{exmp}

\begin{exmp}
	\textbf{The wave equation:}
	\begin{align*}
	\frac{\partial^2 \psi}{\partial t^2} = \nabla^2 \psi.
	\end{align*}
	The wave equation describes physical vibrations. The second $t$-derivative in the equation is strongly correlated to Newton's second law of motion.
\end{exmp}

\begin{exmp}
	\textbf{The Schr\"{o}dinger equation:}
	\begin{align*}
	i\hbar \frac{\partial \psi}{\partial t} = -\frac{\hbar^2}{2m}\nabla^2\psi + V(t,\vec{x})\psi.
	\end{align*}
	One can hardly talk about PDEs without mentioning the Schr\"{o}dinger equation. There is a strong resemblance between the Schr\"{o}dinger equation and the wave equation. Of course, this is no coincidence, as the Schr\"{o}dinger equation is postulated based on a description of a harmonic oscillator.  
\end{exmp}

Our next example does not include the Laplacian operator. 

\begin{exmp}
	\textbf{The telegraphic equation:}
	\begin{align*}
	\frac{\partial^2 \psi}{\partial t^2} = \frac{\partial^2 \psi}{\partial x^2} + \alpha \frac{\partial \psi}{\partial t} + \beta \psi.
	\end{align*}
	The telegraphic equation describes the transfer of information. 
\end{exmp}

\subsection{Vocabulary}
\begin{itemize}
	\item The function $\psi$ appearing in a given PDE is called the ``dependent variable.''
	\item The variables $t,x_1,x_2,\dots$ are called ``independent variables.''
\end{itemize}

\subsection{Our goals}
Our goal is, given a PDE, to find a sufficiently differentiable function which satisfies it that is subject to \textbf{boundary} and \textbf{initial} conditions. 

\subsection{Our plan}
Here are the key concepts we will explore in this text:
\begin{itemize}
	\item Modeling: Formulate same physical problem in terms of PDEs.
	\item Learn how to solve (some) PDEs, subjection to initial conditions and boundary conditions. This means we will be looking at ideas like:
	\begin{itemize}
		\item Separation of variables, in order to reduce a PDE into a system of ODEs.
		\item Integral transforms, in order to reduce the number of independent variables.
		\item Change of coordinates, in order to change a complicated PDE into another one which is easier to solve.
		\item Eigenfunction expansion, which generally goes under the Stum-Louiville theory.
		\item Numerical methods, as most PDEs cannot be solved analytically. 
	\end{itemize}
\end{itemize}

\subsection{Classification}
\begin{itemize}
	\item The order of a PDE is the highest order of partial derivatives appearing (non-trivially) in the PDE.
	\begin{exmp}
		\begin{align*}
		\frac{\partial \psi}{\partial t} = \nabla^2\psi
		\end{align*}
		is a second-order PDE.
	\end{exmp}
	\begin{exmp}
		\begin{align*}
		\frac{\partial \psi}{\partial t} = \partial^4_x\psi
		\end{align*}
		- the biharmonic heat equation, is a fourth-order PDE.
	\end{exmp}
	\item Linearity: A PDE is linear if the function $\psi$ and its derivatives appear in a linear way.
	\begin{exmp}
		All second-order linear PDEs in 2 variables are of the form:
		\begin{align*}
		\boxed{A\frac{\partial^2 \psi}{\partial x^2} + B\frac{\partial^2 \psi}{\partial x\,\partial y} + C\frac{\partial^2 \psi}{\partial y^2} + D\frac{\partial \psi}{\partial x} + E\frac{\partial \psi}{\partial y} + F\psi = G}
		\end{align*}
	\end{exmp}
\end{itemize}

\end{document}
