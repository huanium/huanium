\documentclass{article}
\usepackage{physics}
\usepackage{graphicx}
\usepackage{caption}
\usepackage{amsmath}
\usepackage{authblk}
\usepackage{amsfonts}
\usepackage{esint}
\usepackage{mathtools}
\usepackage{amsthm}
\theoremstyle{definition}
\newtheorem{defn}{Definition}[section]
\newtheorem{prop}{Proposition}[section]
\newtheorem{rmk}{Remark}[section]
\newtheorem{thm}{Theorem}[section]
\newtheorem{exmp}{Example}[section]
\newtheorem{prob}{Problem}[section]
\newtheorem{sln}{Solution}[section]
\newtheorem*{prob*}{Problem}
\newtheorem{exer}{Exercise}[section]
\newtheorem*{exer*}{Exercise}
\newtheorem*{sln*}{Solution}
\usepackage{empheq}
\usepackage{hyperref}
\usepackage{tensor}
\usepackage{xcolor}
\hypersetup{
	colorlinks,
	linkcolor={black!50!black},
	citecolor={blue!50!black},
	urlcolor={blue!80!black}
}
\newcommand{\p}{\partial}
\newcommand{\R}{\mathbb{R}}
\newcommand{\C}{\mathbb{C}}
\newcommand{\lag}{\mathcal{L}}
\newcommand{\I}{\mathcal{I}}
\newcommand{\K}{\mathcal{K}}
\newcommand{\F}{\mathcal{F}}
\newcommand{\w}{\omega}
\newcommand{\lam}{\lambda}
\newcommand{\al}{\alpha}
\newcommand{\be}{\beta}
\newcommand{\x}{\xi}

\newcommand{\f}[2]{\frac{#1}{#2}}

\newcommand{\ift}{\infty}

\newcommand{\lp}{\left(}
\newcommand{\rp}{\right)}

\newcommand{\lb}{\left[}
\newcommand{\rb}{\right]}

\newcommand{\lc}{\left\{}
\newcommand{\rc}{\right\}}



\usepackage{subfig}
\usepackage{listings}
\captionsetup[lstlisting]{margin=0cm,format=hang,font=small,format=plain,labelfont={bf,up},textfont={it}}
\renewcommand*{\lstlistingname}{Code \textcolor{violet}{\textsl{Mathematica}}}
\definecolor{gris245}{RGB}{245,245,245}
\definecolor{olive}{RGB}{50,140,50}
\definecolor{brun}{RGB}{175,100,80}
\lstset{
	tabsize=4,
	frame=single,
	language=mathematica,
	basicstyle=\scriptsize\ttfamily,
	keywordstyle=\color{black},
	backgroundcolor=\color{gris245},
	commentstyle=\color{gray},
	showstringspaces=false,
	emph={
		r1,
		r2,
		epsilon,epsilon_,
		Newton,Newton_
	},emphstyle={\color{olive}},
	emph={[2]
		L,
		CouleurCourbe,
		PotentielEffectif,
		IdCourbe,
		Courbe
	},emphstyle={[2]\color{blue}},
	emph={[3]r,r_,n,n_},emphstyle={[3]\color{magenta}}
}


%\pagestyle{fancy}
%\fancyhf{}
%\rhead{Project Description - MA411: PDE - April 22, 2019}
%\lhead{Group: HUAN QUANG BUI}
%\rfoot{\thepage}


%\newcommand{\R}{\mathbb{R}}
%\newcommand{\lag}{\mathcal{L}}
%\newcommand{\F}{\mathcal{F}}


\begin{document}
	
	\begin{titlepage}\centering
		\clearpage
		\title{\textsc{\bf{Partial Differential Equations \\ \& \\ The Euler-Lagrange Equations}}}
		\author{\bigskip Huan Q. Bui}
%		\affil{Colby College\\$\,$\\ PHYSICS \& MATHEMATICS\\ Statistics \\$\,$\\Class of 2021\\}
		\date{\today}
		\maketitle
		\thispagestyle{empty}
	\end{titlepage}


	\tableofcontents
	
	
	
\section{Motivation}


\section{Introduction}

\section{Minimizer}

\section{Calculus of Variations}

\section{The Euler-Lagrange Equations}

\section{The Shortest Distance}

\section{The Brachistochrone Problem}

\section{PDE's and the Euler-Lagrange Equations}

\subsection{PDE's as Minimization Problems}

\subsection{Euler-Lagrange Equations as PDE's}

\subsection{Principle of Least Action}

\subsubsection{Physics}
	
	
	
\noindent \textbf{Topic:} PDEs and Euler-Lagrange Equations.\\

\noindent \textbf{Plan:} I will first discuss two motivating examples of optimization problems that use calculus of variations: showing that a straight line gives the shortest distance between two points (in Euclidean geometry), and the famous Brachistochrone problem. These examples will lead up to the idea of a common \textbf{functional} of the form
\begin{align*}
J[\phi_i] = \int^b_a \lag(t,\phi_i,\dot{\phi}_i)\,dt
\end{align*}
and how the \textbf{Euler-Lagrange equations}
\begin{align*}
\f{\p \lag }{\p \phi_i} = \f{\p}{\p t} \lp\f{\p \lag}{\p \dot{\phi}_i}\rp,
\end{align*}
where $\lag$ is called the \textbf{Lagrangian}, arise when finding the function $\bar{\phi}_i$ that minimizes $J[\phi_i]$. I will then show we can:
\begin{enumerate}
	\item Solve PDE's by formulating them as minimization problems. For example, the solution to the Dirichlet problem 
	\begin{align*}
	(\ast) \begin{cases}
	\nabla^2u = 0, \hspace{0.5cm}  \text{$x\in \Omega$}\\
	u = 0, \hspace{0.5cm} \text{$x \in \p \Omega$}
	\end{cases}
	\end{align*}
	minimizes 
	\begin{align*}
	J[\phi] = \f{1}{2}\int_\Omega \abs{\nabla \phi}^2\,dx
	\end{align*}
	where
	\begin{align*}
	\lag = \f{1}{2}\abs{\nabla \phi}^2.
	\end{align*}
	\item Recognize the Euler-Lagrange equations as PDE's whose solutions minimize $J[\phi_i]$. I will discuss \textit{Hamilton's Principle of Least Action} and how ``equations on motion'' including Newton's second law of motion, the wave equation, etc. can be derived from this principle. 
\end{enumerate}



% references

\end{document}