\documentclass[11pt]{article}
\usepackage{amsmath}
\usepackage{physics}
\usepackage{amssymb}
\usepackage{graphicx}
\usepackage{hyperref}
\usepackage{amsfonts}
\usepackage{cancel}
\usepackage{xcolor}
\hypersetup{
	colorlinks,
	linkcolor={black!50!black},
	citecolor={blue!50!black},
	urlcolor={blue!80!black}
}
\newcommand{\f}[2]{\frac{#1}{#2}}
\usepackage{newpxtext,newpxmath}
\usepackage[left=1.25in,right=1.25in,top=1.25in,bottom=1.25in]{geometry}
\usepackage{framed}
\usepackage{enumerate}
\usepackage{eqnarray}
\usepackage{caption}
\usepackage{subcaption}

%\newcommand{\fig}[1]{figure #1}
%\newcommand{\explain}{appendix?}
%\newcommand{\rat}{\mathbb{Q}}
%
%\newcommand{\mathbb{R}}{\mathbb{R}}
%\newcommand{\nat}{\mathbb{N}}
%\newcommand{\inte}{\mathbb{Z}}
%\newcommand{\M}{{\cal{M}}}
%\newcommand{\sss}{{\cal{S}}}
%\newcommand{\rrr}{{\cal{R}}}
%\newcommand{\uu}{2pt}
%\newcommand{\vv}{\vec{v}}
%\newcommand{\comp}{\mathbb{C}}
%\newcommand{\field}{\mathbb{F}}
%\newcommand{\f}[1]{ \hspace{.1in} (#1) }
%\newcommand{\set}[2]{\mbox{$\left\{ \left. #1 \hspace{3pt}
%\right| #2 \hspace{3pt} \right\}$}}
%\newcommand{\integral}[2]{\int_{#1}^{#2}}
%\newcommand{\ba}{\hookrightarrow}
%\newcommand{\ep}{\varepsilon}
%\newcommand{\limit}{\operatornamewithlimits{limit}}
%\newcommand{\ddd}{.1in}
%\newcommand{\ccc}{2in}
%\newcommand{\aaa}{1.5in}
%\newcommand{\B}{{\cal B}}
%\newcommand{\C}{{\cal C}}
%\newcommand{\D}{{\cal D}}
%\newcommand{\FF}{{\cal F}}
\usepackage{amssymb}% http://ctan.org/pkg/amssymb
\usepackage{pifont}% http://ctan.org/pkg/pifont
\newcommand{\cmark}{\ding{51}}%
\newcommand{\xmark}{\ding{55}}%
\newcommand{\p}{\partial}%
%\usepackage{MnSymbol,wasysym}

\begin{document}
	\begin{framed}
\begin{center}
{\large \bf MA355: Combinatorics Midterm 2 (Prof. Friedmann)}\\
{ Huan Q. Bui}\\
Due: Mar 19 2021
\end{center}
\end{framed}







\noindent \textbf{1.} \textbf{Book and Shelves}
\begin{enumerate}[(a)]
	\item There are $n+r+t+2$ slots for the $n+r+t$ books and $2$ dividers (to separate the books into 3 shelves). We pick 2 of them for the dividers then subsequently pick $n$ slots for the combinatorics books and $r$ for the algebra books. The total number of ways is
	\begin{equation*}
	\boxed{{ n+r+t+2\choose{2}} { n+r+t\choose{n} }{ r+t\choose{r} }}
	\end{equation*}
	The shelves are distinct, so we don't divide the answer by $3!$.
	
	\item If the $k$ books are identical then we know from Problem 128 that the number of ways to distribute them to $n$ distinct shelves so that each shelf gets at least 3 books is 
	\begin{equation*}
	{ n+k-1-3n \choose{n-1}} = { k-2n-1 \choose{n-1}}
	\end{equation*}
	But the $k$ books here are distinct, so we multiply this number by $k!$. The answer is 
	\begin{equation*}
	\boxed{k! { k-2n-1 \choose{n-1}}}
	\end{equation*}
	
\end{enumerate}




\newpage

\noindent \textbf{2. Games of Bridge} 
\noindent Let $B_n$ denote the number of ways to divide $4n$ people into sets of four for games of bridge. Suppose that 4 more people want to join. To re-assign the quartets, we first pick 4 players out of the new total of $4(n+1)$ to form one game, then have $B_n$ ways to form the remaining $n$ games. However, this is not enough: since the number of teams increases by 1, we also have to update the symmetry factor by multiplying by $n!/(n+1)!$. This gives
\begin{equation*}
\boxed{B_{n+1} = \f{n!}{(n+1)!}{4n+4 \choose 4} B_n}
\end{equation*} 
where $B_0 = 1$, so that $B_1 = 1$. We can also start at $B_1=1$ -- it doesn't matter. \\


\textcolor{purple}{Alternatively, since we have 
\begin{equation*}
B_n = \f{1}{n!}{4n \choose 4}{{4n-4}\choose 4}\dots  {{4}\choose 4} \quad \text{and} \quad
B_{n+1} = \f{1}{(n+1)!}{4n+4 \choose 4}{{4n}\choose 4}{{4n-4}\choose 4}\dots  {{4}\choose 4},
\end{equation*}
we find
\begin{equation*}
\boxed{B_{n+1} = \f{n!}{(n+1)!}{4n+4 \choose 4} B_n}
\end{equation*}}




\newpage




\noindent \textbf{3. Forest, Trees, and Roots}

\begin{enumerate}[(a)]
	\item $F$ has $n$ vertices and $k$ connected components, i.e., $k$ trees. From Problem 108, we know that for each tree the number of vertices is one bigger than the number of edges. Since there are $k$ trees and the numbers of vertices and edges over all trees sum to those of $F$, there must be $k$ more vertices than edges in $F$. So, $F$ has $\boxed{n-k}$ edges.
	
	
	\item 
	\begin{enumerate}[1.]
		\item To turn a rooted tree into a directed rooted tree we do the following. Call a vertex $r$ the root vertex and look at the collection $R_1$ of vertices connected to $r$. We assign the direction $r\to r_{1i}$ to each $r_{1i}\in R_1$. Next, we look at each $r_{1i}\in R_1$, which is connected (with direction) to $r$ and possibly (without direction) to some  collection $R_{2i}$ of vertices. Assign the direction $r_{1i}\to r_{2ij}$ for each $r_{2ij}\in R_{2i}$. For each $r_{2ij}\in R_{2i}$, we repeat the process: assign the direction $r_{2ij}\to r_{3ijk}\in R_{3ij}$, and so on until we have assigned directions to all edges and end up with a directed rooted tree. This proves \textit{existence}. 
		
		\indent $\quad$ Uniqueness really follows from the preceding paragraph. But we can also get it the following way. In a directed rooted tree, all directions point away from the root. So, if at least one edge changes direction, we no longer have a directed rooted tree. 
		
		\item Given a tree with $n$ vertices, there are $n$ ways to designate a root. From part 1., each rooted tree gotten by choosing a root gives exactly one directed rooted tree. So, the number of directed rooted trees with the same set of edges and vertices is the $\boxed{\text{number of vertices of the tree.}}$
	\end{enumerate}

	\item Since $F$ contains $F'$ as a directed graph, and that $F$ and $F'$ both have $n$ vertices, the number of edges in $F$ must be larger than that in $F'$. In view of Part 1., the number of components in $F'$ must be smaller than the number of components in $F$. 
	
	
	\item 
	\begin{enumerate}[1.]
		\item Since each $F_i$ has $i$ components and $n$ vertices, $F_{i}$ has one more edge compared to $F_{i-1}$. So, to go from $F_{k}$ to $F_{k-1}$ we must add one directed edge to $F_k$. To do this, we first pick a vertex $v$ ($n$ choices). Then, we need to connect some root $r$ that is not the root of the tree containing $v$ ($k-1$ choices) to the vertex $v$ (We have to connect $v$ to a root because otherwise $F_{k-1}$ fails to be a directed rooted forest). The direction of this edge must be $r\to v$ to ensure that $F_{k-1}$ is a directed rooted forest, which means there's only one choice for the direction. So, there are $n(k-1)$ ways to pick $F_{k-1}$. Repeating this argument, going from $F_{k-1}$ to $F_{k-2}$ and so on to $F_1$, we find that the number of ways to choose the sequence $F_1, F_2, \dots, F_k$ is 
		\begin{equation*}
		N^*(F_k) = n(k-1)n(k-2)n(k-3)\dots n(1) = \boxed{n^{k-1}(k-1)!}
		\end{equation*}
		
		\item $N(F_k)$, the number of directed rooted trees containing $F_k$, is just the number of possible $F_1$'s (because a directed rooted forests with one component is a directed rooted tree). From the previous part, we know that if the order in which we add the $k-1$ edges matters, then there are $n^{k-1}(k-1)!$ ways to get to go from $F_k$ to a directed rooted tree. Here, we only care about the number of possible trees rather than the number of ways to get those trees, so we must divide the answer from the previous part by $(k-1)!$, giving
		\begin{equation*}
		N(F_k) = \f{N^*(F_k)}{(k-1)!} = \boxed{n^{k-1}}
		\end{equation*}
		\textcolor{purple}{Alternatively, we can think about this in reverse order: As argued, there are $N(F_k)$ ways to start the refining sequence $F_1, F_2, \dots, F_k$. Now $F_1$ contains $(k-1)$ extra edges compared to $F_k$, so going from $F_1$ to $F_2$ requires deleting one of those $(k-1)$ edges, then $F_2$ to $F_3$ requires deleting one of the remaining $(k-2)$ edges, and so on, until there is only one edge to delete going from $F_{k-1}$ to $F_k$. This gives 
		\begin{equation*}
		N^*(F_k) = N(F_k)(k-1)!
		\end{equation*}
		Rearranging gives $N(F_k)$.}
		
		\item From the preceding part, $N(F_n)$ is the number of directed rooted trees containing just the $n$ roots. So, $N(F_n)$ is the number of directed rooted trees with vertex set $\{1,2,\dots,n\}$. From Part (b) and the fact that for each $n$-vertex tree there are $n$ choices for the root, we find the number of trees with vertex set $\{1,2,\dots,n\}$ to be $$\frac{N(F_n)}{n} = \frac{n^{n-1}}{n} = \boxed{n^{n-2}}$$
		
	\end{enumerate}
\end{enumerate}
 





\newpage
\noindent \textbf{4. Combinatorial sandwiches}


The spreads used on the two sides are either the same ($k$ possibilities) or different $k\choose 2$ possibilities. So, the number of distinct sandwich varieties is 
\begin{equation*}
V = {k\choose 2} + k  = \f{k(k-1)}{2} + k = \f{k(k+1)}{2}.
\end{equation*}
Now, we have $n$ identical sandwiches (objects) to be sent to $V$ distinct varieties (recipients) so that each variety appears at least once. Problem 128 tells us that there are
\begin{equation*}
{{V+n-1-V}\choose{V-1}} = {{n-1}\choose{V-1}} = \boxed{{{n-1}\choose{k(k+1)/2-1}}}
\end{equation*}
ways to do this. 

\end{document}