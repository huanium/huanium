\documentclass[11pt]{article}
\usepackage{amsmath}
\usepackage{physics}
\usepackage{amssymb}
\usepackage{graphicx}
\usepackage{hyperref}
\usepackage{amsfonts}
\usepackage{cancel}
\usepackage{amsthm}
\usepackage{xcolor}
\hypersetup{
	colorlinks,
	linkcolor={black!50!black},
	citecolor={blue!50!black},
	urlcolor={blue!80!black}
}
\newcommand{\f}[2]{\frac{#1}{#2}}
\usepackage{newpxtext,newpxmath}
\usepackage[left=1.25in,right=1.25in,top=1.25in,bottom=1.25in]{geometry}
\usepackage{framed}
\usepackage{enumerate}
\usepackage{eqnarray}
\usepackage{caption}
\usepackage{subcaption}

%\newcommand{\fig}[1]{figure #1}
%\newcommand{\explain}{appendix?}
%\newcommand{\rat}{\mathbb{Q}}
%
%\newcommand{\mathbb{R}}{\mathbb{R}}
%\newcommand{\nat}{\mathbb{N}}
%\newcommand{\inte}{\mathbb{Z}}
%\newcommand{\M}{{\cal{M}}}
%\newcommand{\sss}{{\cal{S}}}
%\newcommand{\rrr}{{\cal{R}}}
%\newcommand{\uu}{2pt}
%\newcommand{\vv}{\vec{v}}
%\newcommand{\comp}{\mathbb{C}}
%\newcommand{\field}{\mathbb{F}}
%\newcommand{\f}[1]{ \hspace{.1in} (#1) }
%\newcommand{\set}[2]{\mbox{$\left\{ \left. #1 \hspace{3pt}
%\right| #2 \hspace{3pt} \right\}$}}
%\newcommand{\integral}[2]{\int_{#1}^{#2}}
%\newcommand{\ba}{\hookrightarrow}
%\newcommand{\ep}{\varepsilon}
%\newcommand{\limit}{\operatornamewithlimits{limit}}
%\newcommand{\ddd}{.1in}
%\newcommand{\ccc}{2in}
%\newcommand{\aaa}{1.5in}
%\newcommand{\B}{{\cal B}}
%\newcommand{\C}{{\cal C}}
%\newcommand{\D}{{\cal D}}
%\newcommand{\FF}{{\cal F}}
\usepackage{amssymb}% http://ctan.org/pkg/amssymb
\usepackage{pifont}% http://ctan.org/pkg/pifont
\newcommand{\cmark}{\ding{51}}%
\newcommand{\xmark}{\ding{55}}%
\newcommand{\p}{\partial}%
%\usepackage{MnSymbol,wasysym}

\begin{document}
	\begin{framed}
\begin{center}
{\large \bf MA355: Combinatorics Final (Prof. Friedmann)}\\
{ Huan Q. Bui}\\
\today
\end{center}
\end{framed}







\noindent \textbf{1.} \textbf{9 and [9]}
\begin{enumerate}[(a)]
	\item There are $\boxed{2}$ partitions of 9 with all their parts of size 2 or 3: 
	\begin{align*}
	9 &= 3 + 2 + 2 + 2 \\
	&= 3+3+3.
	\end{align*}
	
	
	\item 
\end{enumerate}



\newpage



\noindent \textbf{2.} \textbf{$\mathbf{F_i}$bonacci}


\begin{enumerate}[(a)]
	\item \underline{Claim}: The number of subsets $S$ of $[n]$ such that $S$ contains no two consecutive integers is ${F_{n+2}}$.
	
	
	\begin{proof}
		
	\end{proof} 
	
	
	
	
	
	
	\item \underline{Claim}: The number of compositions of $n$ into parts of size greater than $1$ is $F_{n-2}$.
	
	
	\begin{proof}
		
	\end{proof}

\end{enumerate}



\newpage



\noindent \textbf{3.} \textbf{$\mathbf{S(t,i)}$rling.}

\begin{enumerate}[(a)]
	\item \underline{Claim}:
	\begin{equation*}
	\boxed{S(k,k-2) = \sum^k_{i=3} (i-2) { {i-1} \choose  2  }}
	\end{equation*}
	
	\begin{proof}
		From Problem 134, we know that 
		\begin{equation*}
		S(k,n) = S(k-1, n-1) + nS(k-1,n).
		\end{equation*}
		With $n=k-2$, we have
		\begin{align*}
		S(k,k-2) &= S(k-1, k-3) + (k-2)S(k-1,k-2)\\
		&= S(k-1, k-3) + (k-2)S(k-1,(k-1)-1)\\
		&= S(k-1, k-3) + (k-2){{k-1} \choose 2}.
		\end{align*}
		Let $S_k$ denote $S(k,k-2)$, then we have a recurrence relation 
		\begin{equation*}
		S_k = S_{k-1} + (k-2){{k-1} \choose 2}, \quad k \geq 2.
		\end{equation*}
		From here, we find a formula for $S_k$:
		\begin{equation*}
		S(k,k-2) = S_k = \underbrace{S_2}_{= S(2,0) = 0} + \sum^{k}_{i=3} (i-2){{i-1} \choose 2} = \sum^{k}_{i=3} (i-2){{i-1} \choose 2}.
		\end{equation*}
		as desired. 
	\end{proof}
	
	
	\item \underline{Claim}:
	\begin{equation*}
	\boxed{S(k,2) = 2^{k-1}-1, \, k\geq 2 \qquad \text{and} \qquad S(k,3) = ,\, k\geq 3}
	\end{equation*}
	
	
	\begin{proof}
		From Problem 134, we know that 
		\begin{equation*}
		S(k,n) = S(k-1, n-1) + nS(k-1,n).
		\end{equation*}
		With $n=2$, we have
		\begin{align*}
		S(k,2) &= S(k-1, 1) + nS(k-1,2)\\
		&= 1 + 2S(k-1,2).
		\end{align*}
		Let $S_k$ denote $S(k,2)$ then we have a first-order linear recurrence:
		\begin{equation*}
		S_k = 1 + 2S_{k-1}.
		\end{equation*}
		With $S_2 = S(2,2) = 1$, the formula for $S(k,2)$, due to Problem 98, is
		\begin{equation*}
		S(k,2) = S_k = 2^{k-2} S_2 + 1\times \left(\f{2^{k-2}-1}{2-1}\right) = 2^{k-2} +  2^{k-2}-1 = 2^{k-1}-1, \quad k \geq 2,
		\end{equation*}
		as claimed\footnote{Here, recurrence begins at $S_2$, so the exponent in the formula only goes up to $k-2$.}. Now, we will use this result and Problem 134 to find $S(k,3)$:
		\begin{align*}
		S(k,3) &= S(k-1,2) + 3S(k-1,3)\\
		&= (2^{k-1}-1) + 3S(k-1,3).
		\end{align*}
		Let $T_k$ denote $S(k,3)$, then we have the recurrence relation
		\begin{equation*}
		T_k = (2^{k-1}-1) + 3T_{k-1}.
		\end{equation*}
		
		%a(n) = (1 + 3^(n-1) - 2^n)/2
	\end{proof}
	
	
	
	\item \underline{Claim}: 
	\begin{equation*}
	\boxed{S(k,n) = \sum^{k}_{i=1} S(k-i,n-1) {{k-1}\choose{i-1}}}
	\end{equation*}
	
	\begin{proof}
		
	\end{proof}
\end{enumerate}


\newpage



\noindent \textbf{4.} \textbf{Latti$\mathbf{C_e}$ paths.} We break the journey from $(0,0) \to (20,30)$ into $(0,0) \to (8,15)$ followed by $(8,15)\to (20,30)$. We can do this because the lattice walker can't move backwards (i.e., to the left or down). The number of paths $P_1$ from $(0,0)$ to $(8,15)$ is given by
\begin{equation*}
P_1 = {{8+15}\choose 8} = {23\choose 8}.
\end{equation*}
Now we want to go from $(8,15)$ to $(20,30)$ but avoid $(14,23)$. Since a path from $(8,15)$ to $(20,30)$ either goes through $(14,23)$ or not, the number of paths from $(8,15)$ to $(20,30)$ is combination of paths through $(14,23)$ and not through $(14,23)$. There are:
\begin{equation*}
{ {(20 - 8) + (30-15)} \choose  {(30-15)} } = { {27} \choose  15 }
\end{equation*} 
paths from $(8,15)$ to $(20,30)$, while there are 
\begin{equation*}
{ {(14 - 8) + (23-15)} \choose  {(23-15)} }{ {(20 - 14) + (30-23)} \choose  {(30-23)} } = {14 \choose 8 }{ {13} \choose  7 }
\end{equation*}
paths from $(8,15)$ to $(20,30)$ that go through $(14,23)$. So, the number of paths from $(8,15)$ to $(20,30)$ that don't go through $(14,23)$ is 
\begin{equation*}
P_2 = { {27} \choose  15 } - {14 \choose 8 }{ {13} \choose  7 }.
\end{equation*}
With this, we combine the two parts of the journey to find that there are 
\begin{equation*}
P = P_1 P_2 = {{8+15}\choose 8} = \boxed{{23\choose 8} \times \left\{   { {27} \choose  15 } - {14 \choose 8 }{ {13} \choose  7 }\right\} }
\end{equation*}
paths from $(0,0)$ to $(20,30)$ that go through $(8,15)$ but not $(14,23)$. 




\newpage



\end{document}