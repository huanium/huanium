\documentclass[11pt]{article}
\usepackage{amsmath}
\usepackage{physics}
\usepackage{amssymb}
\usepackage{graphicx}
\usepackage{hyperref}
\usepackage{amsfonts}
\usepackage{cancel}
\usepackage{xcolor}
\hypersetup{
	colorlinks,
	linkcolor={black!50!black},
	citecolor={blue!50!black},
	urlcolor={blue!80!black}
}
\newcommand{\f}[2]{\frac{#1}{#2}}
\usepackage{newpxtext,newpxmath}
\usepackage[left=1.25in,right=1.25in,top=1.25in,bottom=1.25in]{geometry}
\usepackage{framed}
\usepackage{enumerate}
\usepackage{eqnarray}
\usepackage{caption}
\usepackage{subcaption}

%\newcommand{\fig}[1]{figure #1}
%\newcommand{\explain}{appendix?}
%\newcommand{\rat}{\mathbb{Q}}
%
%\newcommand{\mathbb{R}}{\mathbb{R}}
%\newcommand{\nat}{\mathbb{N}}
%\newcommand{\inte}{\mathbb{Z}}
%\newcommand{\M}{{\cal{M}}}
%\newcommand{\sss}{{\cal{S}}}
%\newcommand{\rrr}{{\cal{R}}}
%\newcommand{\uu}{2pt}
%\newcommand{\vv}{\vec{v}}
%\newcommand{\comp}{\mathbb{C}}
%\newcommand{\field}{\mathbb{F}}
%\newcommand{\f}[1]{ \hspace{.1in} (#1) }
%\newcommand{\set}[2]{\mbox{$\left\{ \left. #1 \hspace{3pt}
%\right| #2 \hspace{3pt} \right\}$}}
%\newcommand{\integral}[2]{\int_{#1}^{#2}}
%\newcommand{\ba}{\hookrightarrow}
%\newcommand{\ep}{\varepsilon}
%\newcommand{\limit}{\operatornamewithlimits{limit}}
%\newcommand{\ddd}{.1in}
%\newcommand{\ccc}{2in}
%\newcommand{\aaa}{1.5in}
%\newcommand{\B}{{\cal B}}
%\newcommand{\C}{{\cal C}}
%\newcommand{\D}{{\cal D}}
%\newcommand{\FF}{{\cal F}}
\usepackage{amssymb}% http://ctan.org/pkg/amssymb
\usepackage{pifont}% http://ctan.org/pkg/pifont
\newcommand{\cmark}{\ding{51}}%
\newcommand{\xmark}{\ding{55}}%
\newcommand{\p}{\partial}%
%\usepackage{MnSymbol,wasysym}

\begin{document}
	\begin{framed}
\begin{center}
{\large \bf MA355: Combinatorics Midterm 1 (Prof. Friedmann)}\\
{ Huan Q. Bui}\\
Due: Mar 19 2021
\end{center}
\end{framed}

\noindent \textbf{1.} (25 pts) 
\begin{enumerate}[(a)]
	\item An instructor has 13 students in her combinatorics class and wants to divide them
	into four groups with at least three students in each group. How many ways are
	there to do that? What if the instructor also picks a spokesperson for each group?
	
	\textbf{Answer:} The setup requires 3 groups of three and 1 group of four students. The assignment procedure can be described as follows. Choose 3 out of 13 students for the first group, 3 out of the remaining 9 students for the second, 3 out of the remaining 6 students for the third group, and the rest go to the fourth group. Finally, divide the total by $3!$ since each combination of the three groups of three has $3!$ permutations. The answer is:
	\begin{equation*}
	\boxed{\Omega_{13} = \f{1}{3!}\binom{13}{3}\binom{10}{3}\binom{7}{3} = 200200}
	\end{equation*}
	
	There are $i$ ways to pick a spokesperson for a group with $i$ members, so
	\begin{equation*}
	\boxed{\Omega_\text{13 + pick spokesperson for each group} = \Omega_{13} \times 4\times 3 \times 3 \times 3 = \Omega_{13} \times 4\times 3^3 = 21621600}
	\end{equation*} 
	
	
	\item Five weeks after the beginning of the semester and right after the first midterm,
	the instructor divides the students into a new set of four groups, again with at least
	three students in each group. The instructor would like the new groups to be such
	that no two students who previously were in the same group are again in the same
	group. How many ways are there to do that?

	
	
	\textbf{Answer:} Assume some group configuration was set for the first five weeks of the semester. Call the groups Groups 1 to 4, where Group 4 has four students. In order to rearrange the groups as required, Group 4 must be completely separated into the four new groups. Thus, let each of these four students start a new group. This leaves us with 9 students. The procedure to send these students to new groups is as follows:
	\begin{itemize}
		\item Separate Group 3 into three of the four new groups ($4\times 3!$ ways).
		
		\item Send one student of Group 2 and one student from Group 1 into the new group that doesn't have anyone from Group 3 ($3\times 3$ ways). 
		
		\item Send the remaining two students of Group 2 into the three new groups that don't have anyone from Groups 1 and 2 ($3\times 2!$ ways).
		
		\item There is now one new group with two students (the rest have three each). Send one student from the remaining two of Group 1 into this new group ($2$ ways). 
		
		\item The pigeon hole principle guarantees that there are still two of the new groups that don't have anyone from Group 1. This means there are 2 ways to send the last student of Group 1. 
	\end{itemize}
	We note that the order in which we did things (subsequently filling the new groups with students from Group 3, then 2 and 1) should not matter since the order of membership does not matter. Therefore, there is no extra symmetry factor\footnote{A more explicit way to do this part is to consider 3 group choices for the first step, then 2 group choices for the second step, and divide the final answer by $3!$, which gives us the same answer. The reason is that the group labels are arbitrary and membership doesn't care about the order of admission.}, and the grand total is: 
	\begin{equation*}
	4\times 3! \times 3\times 3 \times 3\times 2! \times 2 \times 2 = \boxed{5184}
	\end{equation*}	
	
	
	\item Answer all the above questions again, but for 12 students instead of 13.
	
	\textbf{Answer:} Following a similar argument (starting with 12 students and picking 3 students for each subsequent group) and noticing that we now have four groups of three, we must replace the $1/3!$ factor by $1/4!$ (because, say, the last group of three could have appeared as the first, unlike in Part (a) where the last group has four people) and of course change the total to 12 people: 
	\begin{equation*}
	\boxed{\Omega_{12} = \f{1}{4!}\binom{12}{3}\binom{9}{3}\binom{6}{3} = 15400}
	\end{equation*}
	and 
	\begin{equation*}
	\boxed{\Omega_\text{12 and pick spokesperson for each group} = \Omega_{12} \times 3\times 3 \times 3 \times 3 = \Omega_{12}\times 3^4 = 1 247 400}
	\end{equation*} 
	
		
	Now, for the second part. Assume some group configuration was set for the first five weeks of the semester. The procedure to send these students to new groups is as follows:
	\begin{itemize}
		\item Choose a group (of three students) and select one student from the remaining 9 to start four new groups ($4\times9$ ways).\footnote{The 9 ways here can also be thought of as a combination of picking one of the three remaining groups and picking one of three students from this group -- hence the reason for dividing by $4!$ later.}
		\item Call the group selected in the previous step Group 1 and the group from which the student was selected Group 2. Send the remaining two students in Group 2 into two of the three available new groups --  which are those with students from Group 1 ($3\times 2!$ ways). 
		\item Call the two remaining original groups Groups 3 and 4 ($2$ ways to do this). Pick two students from Group 3 and send them to the two new groups where each currently has only one student ($2\times 3\times 2!$ ways).		
		\item Pick two students from Group 3 and send them to same two new groups from the previous item ($3\times 2!$ ways).
		\item Send the remaining two students into the remaining two new groups ($2!$ ways).
	\end{itemize}
	In this problem, we took a more careful approach where we have methodically considered the number of ways each subsequent original group can be picked. As a result, we have to divide everything by $1/4!$ to account for the fact that membership doesn't care about which group gets separated first: 
	\begin{equation*}
	\f{1}{4!}\left( 4\times 9\times 3\times 2! \times 2\times  3\times 2! \times 3\times 2!\times 2! \right) = \boxed{1296}
	\end{equation*}
	\textcolor{blue}{Alternatively, to avoid symmetry factors altogether we can line the groups up, call them Groups 1 to 4, and do the following: Fix Group 1 and take one student from Group 2 to start four new groups (3 ways). Send the remaining two students from Group $2$ to the three allowed groups ($3\times 2!$ ways). Next, we have to fill the 1-person new groups (otherwise if we just sent three people into these four new groups we might some old members in the same group). So, we send two students from Group 3 into the two 1-person groups ($3\times 2!$ ways), and the send the remaining student from Group 3 elsewhere (2 ways). Finally, we send Group 4 into the remaining three available slots (3! ways). Note that we haven't missed anything since the new group that doesn't have anyone from Group 1 has one student from each of Group 2, 3, and 4. So, the total is: 
	\begin{equation*}
	3\times 3\times 2! \times 3\times 2! \times 2 \times 3! = 1296.
	\end{equation*} }
\end{enumerate} 





\newpage


\noindent \textbf{2.} (25 pts)
\begin{enumerate}[(a)]
	\item Let $(a_i, b_i, c_i)$ for $1 \leq i \leq 9$ be nine vectors in $\mathbb{R}^3$ with integer coordinates. Show that two of these vectors have a sum whose coordinates are all even integers.
	
	\textbf{Answer:} Let us assign to each vector $v_i = (a_i,b_i,c_i)$ a corresponding \textit{parity vector} defined by 
	\begin{equation*}
	C(v_i) = C((a_i,b_i,c_i)) = (a_i \text{ mod } 2, b_i \text{ mod } 2, c_i \text{ mod } 2).
	\end{equation*} 
	Because $(m+n) \text{ mod } 2 \equiv m \text{ mod }2 + n \text{ mod }2$ for all $m,n\in \mathbb{Z}$, it is easy to see that $v_i + v_j$ has only even integer coordinates (i.e., $C(v_i+v_j) \equiv (0,0,0)$) exactly when $C(v_i) = C(v_j)$.
	
	Since each of $a_i,b_i,c_i$ can only be even or odd, $C(v_i)$ can only be one of $2^3 = 8$ possible values for every $v_i$. Thus, by virtue of the pigeon hole principle, given 9 $v_i$'s, there must be at least two vectors $v_m,v_n$ ($m\neq n$) for which $C(v_m) = C(v_n)$. From the preceding paragraph, $v_n + v_m$ has only even integer coordinates, and Part (a) is proved. 
	
	
	\item Show that this result is best possible, i.e. the conclusion may fail if we have only eight vectors. 
	
	\textit{Answer:} Consider the following eight vectors $(a_i,b_i,c_i)^\top$ concatenated into a matrix:
	\begin{equation*}
	\begin{pmatrix}
	a_1 & a_2 & \dots & a_8\\
	b_1 & b_2 & \dots & b_8\\
	c_1 & c_2 & \dots & c_8\\
	\end{pmatrix}
	=
	\begin{pmatrix}
	0&0&0&0&1&1&1&1\\
	0&0&1&1&0&0&1&1\\
	0&1&0&1&0&1&0&1
	\end{pmatrix}
	\end{equation*}
	The conclusion from Part (a) fails here: The sum of any two vectors contains at least one instance of the entry $0+1=1$, which is an odd integer.
\end{enumerate}



\newpage



\noindent \textbf{3.} (25 pts) Chess is a game played on an 8 $\times$ 8 board with alternating light and dark squares. The rows of the board are called ``ranks'' and are numbered 1 through 8. There are two players, white and black, and each player has a set of 16 pieces: 8 pawns, 2 rooks, 2 bishops, 2 knights, 1 queen, and 1 king. Count the number of possible starting positions for this game given the following constraints:

\begin{enumerate}[(a)]
	\item All the white pieces are placed in the first and second ranks, and all the black pieces
	are placed in the seventh and eighth ranks.

	
	\item The black starting position is a mirror image of the white starting position.

	\item All the white pawns are placed in the second rank.
	
	\item The white king is placed on any square between the two white rooks (not necessarily
	adjacent to either of them).
	
	\item The white bishops are placed on opposite-color squares.

\end{enumerate}

\noindent \textbf{Answer:} By (a), we don't have to worry about possible symmetry factors due to the placement of the white and dark pieces relative to the orientation of the chessboard\footnote{Unless the orientation of the board is not defined (i.e., the chessboard has only the alternating squares). In this case, there is a factor of $2$: the ``first rank'' can begin with black-white alternating squares or white-black.}. By (b), it suffices to just arrange the white pieces. By (c), we don't have to worry about arranging the pawns. By (d) and (e), we don't to worry about arranging the 2 white knights and 1 white queen either (we can easily arrange them at the end). This leaves us with 5 pieces: 1 white king, 2 white rooks, and 2 white bishops (all in the first rank). Let the first rank be represented by the following eight slots, numbered 1 to 8. Without missing/gaining a symmetry factor\footnote{Again, this is under the assumption that the ranks are marked on the board, which fix the black-white pattern in the first rank. Otherwise we will need the factor of 2 in the previous footnote.}, assume a black-white pattern for the first rank: 
\begin{equation*}
\underline{\texttt{1b}}\,
\underline{\texttt{2w}}\,
\underline{\texttt{3b}}\,
\underline{\texttt{4w}}\,
\underline{\texttt{5b}}\,
\underline{\texttt{6w}}\,
\underline{\texttt{7b}}\,
\underline{\texttt{8w}}
\end{equation*}
By (d), the white king can only be in slots (2) through (7). By a two-fold symmetry, we will only consider the following three positions for the white king: (2), (3), and (4). 
\begin{itemize}
	\item The white king is on (2). In this case, one white rook must be on (1) to keep (d) true.
	\begin{equation*}
	\underline{\texttt{\textcolor{red}{R}}}\,
	\underline{\texttt{\textcolor{red}{K}}}\,
	\underline{\texttt{3b}}\,
	\underline{\texttt{4w}}\,
	\underline{\texttt{5b}}\,
	\underline{\texttt{6w}}\,
	\underline{\texttt{7b}}\,
	\underline{\texttt{8w}}
	\end{equation*}
	The other white rook has 6 options. Once this rook is placed, the color ratio is two black to three white slots (or vice versa -- it doesn't matter), which gives $2\times 3 = 6$ arrangements for the white bishops. Without placing the white queen and white knights, we have $6\times 6 =  \boxed{36}$ configurations. 
	
	
	\item The white king is on (3). There are two sub-cases:
	\begin{itemize}
		\item If one white rook is on (1):
		\begin{equation*}
		\underline{\texttt{\textcolor{red}{R}}}\,
		\underline{\texttt{2w}}\,
		\underline{\texttt{\textcolor{red}{K}}}\,
		\underline{\texttt{4w}}\,
		\underline{\texttt{5b}}\,
		\underline{\texttt{6w}}\,
		\underline{\texttt{7b}}\,
		\underline{\texttt{8w}}
		\end{equation*}
		then there are two sub-sub-cases for the other white rook:
		\begin{itemize}
			\item If it is on (4), (6), or (8), then the color ratio is two black to three white slots, which gives $2\times 3 = 6$ arrangements for the white bishops.
			
			\item If it is on (5) or (7), then one white bishop must go into the remaining black slot, and the other white bishop has 4 (white) options. 
		\end{itemize}
		In summary, there are $3\times6+2\times4= 26$ arrangements in this first sub-case.
		
		\item Or if one white rook is on (2):
		\begin{equation*}
		\underline{\texttt{1b}}\,
		\underline{\texttt{\textcolor{red}{R}}}\,
		\underline{\texttt{\textcolor{red}{K}}}\,
		\underline{\texttt{4w}}\,
		\underline{\texttt{5b}}\,
		\underline{\texttt{6w}}\,
		\underline{\texttt{7b}}\,
		\underline{\texttt{8w}}
		\end{equation*}
		then other white rook can only be on (4), (5), (6), (7), or (8). In any case, the color ratio is always two white to three black or vice versa, which gives $2\times 3 = 6$ ways to place the white bishops. So, there are $5\times 6 = 30$ possible arrangements in this second sub-case.
	\end{itemize}
	Therefore, without placing the queen and knights, there are $26+30 = \boxed{56}$ configurations if the white king is on (3).
	
	
	\item The white king is on (4).  There are three sub-cases:
	\begin{itemize}
		\item If one white rook is on (1):
		\begin{equation*}
		\underline{\texttt{\textcolor{red}{R}}}\,
		\underline{\texttt{2w}}\,
		\underline{\texttt{3b}}\,
		\underline{\texttt{\textcolor{red}{K}}}\,
		\underline{\texttt{5b}}\,
		\underline{\texttt{6w}}\,
		\underline{\texttt{7b}}\,
		\underline{\texttt{8w}}
		\end{equation*}
		
		
		\item Or if one white rook is on (2):
		\begin{equation*}
		\underline{\texttt{1b}}\,
		\underline{\texttt{\textcolor{red}{R}}}\,
		\underline{\texttt{3b}}\,
		\underline{\texttt{\textcolor{red}{K}}}\,
		\underline{\texttt{5b}}\,
		\underline{\texttt{6w}}\,
		\underline{\texttt{7b}}\,
		\underline{\texttt{8w}}
		\end{equation*}
		
		
		\item Or if one white rook is on (3):
		\begin{equation*}
		\underline{\texttt{1b}}\,
		\underline{\texttt{2w}}\,
		\underline{\texttt{\textcolor{red}{R}}}\,
		\underline{\texttt{\textcolor{red}{K}}}\,
		\underline{\texttt{5b}}\,
		\underline{\texttt{6w}}\,
		\underline{\texttt{7b}}\,
		\underline{\texttt{8w}}
		\end{equation*}
	\end{itemize}
	The other white rook must be on either (5), (6), (7), or (8). In the first and third sub-cases, the color ratio after placing the second white rook is two black to three white slots or vice versa, and as a result there are $2\times 3 = 6$ ways to place the two white bishops. In the second sub-case, if the second white rook is on (5) or (7), then as before, there are 6 ways to place the bishops; else, there are only four ways to place the bishops. In total, when the white king is on (4), there are $(4\times 6) + (2\times6 + 2\times4) + (4\times 6) = \boxed{68}$ possible configurations (without placing the queen and knights).
\end{itemize}


In the remaining 3 slots in the first rank, there are 3 ways to place the queen and two white knights. So, the grand total\footnote{assuming the orientation of the board is fixed; otherwise there's an extra multiplicative factor of 2} is:
\begin{equation*}
2 \times 3\times(36+56+68) = \boxed{960}
\end{equation*}
where the extra factor of 2 is due to the fact that we considered only half the possible slots for the white king (the other half is exactly the same, by symmetry).

\newpage




\noindent \textbf{4.} (25 pts) 
\begin{enumerate}[(a)]
	\item A circle of 17 friends has the property that no matter how we choose two from
	these 17 friends, those two people correspond with each other on one of three given
	subjects. Prove that there are three friends among the circle of these 17 friends
	such that any two of the three of them correspond with each other on the same
	subject.

	
	\textbf{Answer:} We pick out one friend $X$ from the group. The remaining 16 can be partitioned into Group 1, Group 2, and Group 3, where Group $i$ contains those who correspond with $X$ on Subject \#$i$. Let us consider Group 1:
	\begin{itemize}
		\item If there is one pair in this group who also correspond with each other on \#1, then together with $X$ they form a group of three who all correspond on \#1.
		
		\item Otherwise, all members of Group 1 correspond with each other only on \#2 and \#3. There are two sub-cases:
		\begin{itemize}
			\item There is a group of three who correspond on \#2 or \#3 (and we're done)
 			\item Or, there is no such group of three. Now, according to the 2-color version of the Ramsey problems we did in class (specifically 65 and 66), this holds only if Group 1 has at most 5 friends.\footnote{Reminder: If Group 1 has 6 friends who correspond only on Subjects \#2 and \#3, then we're guaranteed a group of at least three who correspond on either Subject \#2 or \#3, by Problems 65 and 66.} 
		\end{itemize}
	\end{itemize}
	As a result of symmetry, we see that Group 2 and Group 3 must also have at most 5 friends to avoid a group of three who correspond on the same subject. Thus, the total size of the friend circle must be at most 
	\begin{equation*}
	\abs{X} + \max\abs{\text{Group 1} } + \max\abs{\text{Group 2} } + \max\abs{\text{Group 3} } = 1+5+5+5=16
	\end{equation*}
	in order to avoid such a group of three. Thus, with 17 friends, this is not possible.  
	
	
	
	\textit{Remark:} We recognize that this problem is a 3-color version of the (2-color) Ramsey problems in our textbook. Here, the setup of the problem translates to ``given a 3-color $K_{17}$.'' Accordingly, our goal translates to showing that there always exists a monochromatic $K_3$. This translation is useful for Part (b).
	
	\item Let
	\begin{equation*}
	n_k = k! \left( 1 + \f{1}{1!}  + \f{1}{2!} + \dots \f{1}{k!} \right) + 1.
	\end{equation*}
	We color all edges of $K_{n_k}$ with one of $k$ colors. Prove that there will be a triangle with monochromatic edges (i.e., edges of a single color).
	
	\textbf{Answer:} We prove this result by induction on $k$. For $k=1$, $n_k = n_1 = 3$, and the statement holds (trivially) because we have a monochromatic ($k=1$) triangle ($n=3$). Just for assurance, we can consider $k=2$, which gives $n_k = n_2 = 6$. The statement is true by virtue of Problems 65 and 66 in our textbook.  With these, we now assume that the statement holds up to $k$ where $k\geq 1$. We will show that it still holds for $k+1$. 
	
	To this end, we follow our argument in Part (a). After picking out one vertex $X$ in $K_{n_{k+1}}$, we partition the remaining $(n_{k+1}-1)$ vertices into $(k+1)$ groups, where again Group $i$ contains all vertices connected to $X$ by an edge of Color \#$i$. Consider Group 1:
	\begin{itemize}
		\item If there are two vertices in Group 1 that are connected by an edge of Color \#1, then together with $X$ they form a $K_3$ of Color \#1.
		
		\item Otherwise, all vertices in Group 1 are connected only by edges of $k$ other colors. There are two sub-cases:
		\begin{itemize}
			\item There is a monochromatic $K_3$ (and we're done)
			\item Or, there is no monochromatic $K_3$. According to our $k$-color inductive hypothesis, this holds only if Group 1 has at most $(n_k-1)$ vertices.\footnote{Otherwise, there is a monochromatic $K_3$.}
		\end{itemize}
	\end{itemize}
	By symmetry, we see that to avoid a monochromatic $K_3$ in $K_{n_{k+1}}$, the total number of vertices must be at most:
	\begin{align*}
	\underbrace{1}_{|X|}+ \underbrace{(k+1)}_{\text{\# of groups}} \times \underbrace{(n_k-1)}_{\text{max size per group}} 
	&= 1 + (k+1) \times  k! \left(1 + \f{1}{1!} + \f{1}{2!} + \dots + \f{1}{k!} \right)\\
	&= 1 + (k+1)!\left(1 + \f{1}{1!} + \f{1}{2!} + \dots + \f{1}{k!} \right).
	\end{align*}
	As a result, when there is an extra vertex, i.e., when the number of vertices is 
	\begin{equation*}
	2 + (k+1)!\left(1 + \f{1}{1!} + \f{1}{2!} + \dots + \f{1}{k!} \right) = 1 + (k+1)!\left(1 + \f{1}{1!} + \f{1}{2!} + \dots + \f{1}{k!} + \f{1}{(k+1)!} \right) = n_{k+1},
	\end{equation*}
	where we have put $1 = (k+1)!/(k+1)!$, a monochromatic $K_3$ is inevitable. Thus, we are guaranteed a monochromatic $K_3$ in $K_{n_{k+1}}$, and we are done with the proof. 

	
\end{enumerate}







\end{document}