\documentclass{article}
\usepackage{physics}
\usepackage{amsmath}
\usepackage{authblk}
\usepackage{amsfonts}
\usepackage{esint}
\usepackage{mathtools}
\usepackage{amsthm}
\theoremstyle{definition}
\newtheorem{defn}{Definition}[section]
\newtheorem{rmk}{Remark}[section]
\newtheorem{exmp}{Example}[section]
\usepackage{empheq}
\usepackage{tensor}

\begin{document}
	\begin{titlepage}\centering
		\clearpage
		\title{\textsc{\bf{QUANTUM INFORMATION\\
			and\\
		QUANTUM COMPUTATION THEORY}}\\\smallskip A Quick Guide\\}
		\author{\bigskip Huan Q. Bui}
		 \affil{Colby College\\$\,$\\ PHYSICS \& MATHEMATICS\\ Statistics \\$\,$\\Class of 2021\\}
		\date{\today}
		\maketitle
		\thispagestyle{empty}
	\end{titlepage}

\newpage

\subsection*{Preface}
\addcontentsline{toc}{subsection}{Preface}

Greetings,\\

\textit{Quantum Information Theory, A Quick Guide to} is compiled based on my own exploration of \textit{Quantum Computation and Quantum Information} by Nielsen and Chuang and John Preskill's Caltech Physics 219/Computer Science 219 lecture notes. There will also be additional elements and structures from \textit{The Theory of Quantum Information} by John Watrous of University of Waterloo and other resources that can be found online. \textit{Quantum Information Theory, A Quick Guide to} requires some understanding of linear algebra, matrix analysis, probability theory, and of course quantum mechanics. There will be review chapters on these topics, but some familiarity is expected.  \\

The development of this text comes in different layers. The first layer, that I'm building now in February 2019 and perhaps a better part of my junior year (Fall 2019 to Spring 2020), will on the key concepts and the grand scheme of things. This will be somewhat a survey of what quantum information and quantum computation are and introductory topics in these subjects. Ideas and content will be derived from Nielsen and Chuang's text and Preskill's notes. The later waves of development will focus more on the (numerous) theoretical aspects of quantum information as explored in Watrous' book. \\

Enjoy!


\newpage
\tableofcontents
\newpage

\section{Mathematical Preliminaries}
\subsection{Linear Algebra \& Matrix Analysis}
\subsection{Analysis \& Probability Theory}

\section{Quantum Mechanical Preliminaries}

\section{Introduction to Quantum Computation and Information}
\subsection{Quantum bits}
\subsection{Quantum computation}
\subsection{Quantum algorithms}
\subsection{Experimental Quantum Information Processing}
\subsection{Quantum Information}

\newpage

\end{document}
