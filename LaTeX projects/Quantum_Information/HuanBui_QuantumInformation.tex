\documentclass{article}
\usepackage{physics}
\usepackage{amsmath}
\usepackage{authblk}
\usepackage{amsfonts}
\usepackage{esint}
\usepackage{mathtools}
\usepackage{amsthm}
\theoremstyle{definition}
\newtheorem{defn}{Definition}[section]
\newtheorem{rmk}{Remark}[section]
\newtheorem{exmp}{Example}[section]
\usepackage{empheq}
\usepackage{tensor}

\begin{document}
	\begin{titlepage}\centering
		\clearpage
		\title{\textsc{\bf{QUANTUM INFORMATION THEORY}}\\\smallskip A Quick Guide\\}
		\author{\bigskip Huan Bui}
		\affil{Colby College\\Physics \& Statistics\\Class of 2021\\}
		\date{\today}
		\maketitle
		\thispagestyle{empty}
	\end{titlepage}

\newpage

\subsection*{Preface}
\addcontentsline{toc}{subsection}{Preface}

Greetings,\\

\textit{Quantum Information Theory, A Quick Guide to} is compiled based on my own exploration of John Watrous'
\textit{The Theory of Quantum Information}. This text requires some understanding of linear algebra, matrix analysis, and quantum mechanics. \\

Enjoy!


\newpage
\tableofcontents
\newpage

\section{Review of Linear Algebra}
\section{Basic concepts}

\newpage

\end{document}
