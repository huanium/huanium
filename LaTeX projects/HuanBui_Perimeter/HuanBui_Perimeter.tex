\documentclass{book}
\usepackage{physics}
\usepackage{graphicx}
\usepackage{caption}
\usepackage{amsmath}
\usepackage{bm}
\usepackage{framed}
\usepackage{authblk}
\usepackage{empheq}
\usepackage{amsfonts}
\usepackage{esint}
\usepackage[makeroom]{cancel}
\usepackage{dsfont}
\usepackage{centernot}
\usepackage{mathtools}
\usepackage{bigints}
\usepackage{amsthm}
\theoremstyle{definition}
\newtheorem{defn}{Definition}[section]
\newtheorem{prop}{Proposition}[section]
\newtheorem{rmk}{Remark}[section]
\newtheorem{thm}{Theorem}[section]
\newtheorem{exmp}{Example}[section]
\newtheorem{prob}{Problem}[section]
\newtheorem{sln}{Solution}[section]
\newtheorem*{prob*}{Problem}
\newtheorem{exer}{Exercise}[section]
\newtheorem*{exer*}{Exercise}
\newtheorem*{sln*}{Solution}
\usepackage{empheq}
\usepackage{hyperref}
\usepackage{tensor}
\usepackage{xcolor}
\hypersetup{
	colorlinks,
	linkcolor={black!50!black},
	citecolor={blue!50!black},
	urlcolor={blue!80!black}
}


\newcommand*\widefbox[1]{\fbox{\hspace{2em}#1\hspace{2em}}}

\newcommand{\p}{\partial}
\newcommand{\R}{\mathbb{R}}
\newcommand{\C}{\mathbb{C}}
\newcommand{\lag}{\mathcal{L}}
\newcommand{\nn}{\nonumber}
\newcommand{\ham}{\mathcal{H}}
\newcommand{\M}{\mathcal{M}}
\newcommand{\I}{\mathcal{I}}
\newcommand{\K}{\mathcal{K}}
\newcommand{\F}{\mathcal{F}}
\newcommand{\w}{\omega}
\newcommand{\lam}{\lambda}
\newcommand{\al}{\alpha}
\newcommand{\be}{\beta}
\newcommand{\x}{\xi}

\newcommand{\G}{\mathcal{G}}

\newcommand{\f}[2]{\frac{#1}{#2}}

\newcommand{\ift}{\infty}

\newcommand{\lp}{\left(}
\newcommand{\rp}{\right)}

\newcommand{\lb}{\left[}
\newcommand{\rb}{\right]}

\newcommand{\lc}{\left\{}
\newcommand{\rc}{\right\}}


\newcommand{\V}{\mathbf{V}}
\newcommand{\U}{\mathcal{U}}
\newcommand{\Id}{\mathcal{I}}
\newcommand{\D}{\mathcal{D}}
\newcommand{\Z}{\mathcal{Z}}

%\setcounter{chapter}{-1}


\makeatletter
\renewcommand{\@chapapp}{Part}
%\renewcommand\thechapter{$\bf{\ket{\arabic{chapter}}}$}
%\renewcommand\thesection{$\bf{\ket{\arabic{section}}}$}
%\renewcommand\thesubsection{$\bf{\ket{\arabic{subsection}}}$}
%\renewcommand\thesubsubsection{$\bf{\ket{\arabic{subsubsection}}}$}
\makeatother



\usepackage{subfig}
\usepackage{listings}
\captionsetup[lstlisting]{margin=0cm,format=hang,font=small,format=plain,labelfont={bf,up},textfont={it}}
\renewcommand*{\lstlistingname}{Code \textcolor{violet}{\textsl{Mathematica}}}
\definecolor{gris245}{RGB}{245,245,245}
\definecolor{olive}{RGB}{50,140,50}
\definecolor{brun}{RGB}{175,100,80}
\lstset{
	tabsize=4,
	frame=single,
	language=mathematica,
	basicstyle=\scriptsize\ttfamily,
	keywordstyle=\color{black},
	backgroundcolor=\color{gris245},
	commentstyle=\color{gray},
	showstringspaces=false,
	emph={
		r1,
		r2,
		epsilon,epsilon_,
		Newton,Newton_
	},emphstyle={\color{olive}},
	emph={[2]
		L,
		CouleurCourbe,
		PotentielEffectif,
		IdCourbe,
		Courbe
	},emphstyle={[2]\color{blue}},
	emph={[3]r,r_,n,n_},emphstyle={[3]\color{magenta}}
}


\begin{document}
\begin{titlepage}\centering
 \clearpage
 \title{{\textsc{\textbf{Some Topics in Theoretical Physics}}}\\ \smallskip - A Quick Guide - \\}
 \author{\bigskip Huan Q. Bui}
  \affil{Colby College\\$\,$\\ PHYSICS \& MATHEMATICS\\ Statistics \\$\,$\\Class of 2021\\}
 \date{\today}
 \maketitle
 \thispagestyle{empty}
\end{titlepage}

\subsection*{Preface}
\addcontentsline{toc}{subsection}{Preface}

Greetings,\\

This guide is my notes from Perimeter Institute of Theoretical Physics Summer School, 2020. Topics include quantum information, thermodynamics, numerical methods, condensed matter physics, path integrals, and symmetries.\\

Enjoy!  



\newpage
\tableofcontents
\newpage





\chapter{Quantum Information \& Thermodynamics} 

The aim of this course is to understand the thermodynamics of quantum systems and in the process to learn some fundamental tools in Quantum Information. We will focus on the topics of foundations of quantum statistical mechanics, resource theories, entanglement, fluctuation theorems, and quantum machines. 

\newpage

\section{Foundations of Quantum Statistical Mechanics - Entanglement}
\newpage

\section{Resource Theories and Quantum Information}


\newpage
\section{Quantum Thermal Operations}


\newpage
\section{Fluctuation Theorems and Quantum Information}


\newpage
\section{Quantum Thermal Machines}






\newpage
\chapter{Numerical Methods \& Condensed Matter Physics}

This course has two main goals: (1) to introduce some key models from condensed matter physics; and (2) to introduce some numerical approaches to studying these (and other) models.  As a precursor to these objectives, we will carefully understand many-body states and operators from the perspective of condensed matter theory.  (However, I will cover only spin models.  We will not discuss or use second quantization.)\\



Once this background is established, we will study the method of exact diagonalization and write simple python programs to find ground states, correlation functions, energy gaps, and other properties of the transverse-field Ising model and XXZ model.  We will also discuss the computational limitations of exact diagonalization.  Finally, I will introduce the concept of matrix product states, and we will see how these can be used with algorithms such as the density matrix renormalization group (DMRG) to study ground state properties for much larger systems than can be studied with exact diagonalization.




\newpage

\section{Lecture 1}
\subsection{Introduction to many-particle states and operators}
\newpage
\subsection{Introduction to Ising and XXZ models}
\newpage
\subsection{Programming basics}
\newpage
\subsection{Finding expectation values}
\newpage

\section{Lecture 2: Exact diagonalization part 1}
\subsection{Representing models}
\newpage
\subsection{Finding eigenstates}
\newpage
\subsection{Energy gaps}
\newpage
\subsection{Phase transitions}
\newpage




\section{Lecture 3: Exact diagonalization part 2}
\subsection{Limitations of the method}
\newpage
\subsection{Using symmetries}
\newpage
\subsection{Dynamics}
\newpage




\section{Lecture 4: Matrix product states part 1}
\subsection{Entanglement and the singular value decomposition}
\newpage
\subsection{What is a matrix product state and why is it useful}
\newpage


\section{Lecture 5: Matrix product states part 2}
\subsection{Algorithms for finding ground states using matrix product states: iTEBD }

\newpage


\subsection{Algorithms for finding ground states using matrix product states: DMRG }

\newpage











\chapter{Path Integrals}

The goal of this course is to introduce the path integral formulation of quantum mechanics and a few of its applications. We will begin by motivating the path integral formulation and explaining its connections to other formulations of quantum mechanics and its relation to classical mechanics. We will then explore some applications of path integrals.


\newpage

\section{Introduction to path integrals and the semi-classical limit}

\newpage
\section{Propagator in real and imaginary time}


\newpage

\section{Perturbation theory}


\newpage

\section{Non-perturbative physics and quantum tunneling}


\newpage


\section{Topology and path integrals}


\newpage



\chapter{Symmetries}



The aim of this course is to  explore some of the many ways in which symmetries play a role in physics. We’ll start with an overview of the concept of symmetries and their description in the language of  group theory. We will then discuss continuous symmetries and infinitesimal symmetries, their fundamental role in Noether's theorem, and their formalization in terms of Lie groups and Lie algebras. In the last part of the course we will focus on symmetries in quantum theory and introduce representations of (Lie) groups and Lie algebras.


\newpage

\section{Lecture 1: Overview}
\subsection{Definition of symmetry}
\newpage
\subsection{Elements of group theory}
\newpage
\subsection{Examples }
\newpage



\section{Lecture 2}
\subsection{Continuous and discrete symmetries}
\newpage
\subsection{Infinitesimal symmetries}
\newpage
\subsection{Noether's theorem}


\newpage

\section{Lecture 3: Lie groups and Lie algebras}
\newpage

\section{Lecture 4: Symmetries in quantum mechanics}
\newpage
\section{Representation theory}











\end{document}
