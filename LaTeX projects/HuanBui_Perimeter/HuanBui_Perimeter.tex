\documentclass{book}
\usepackage{physics}
\usepackage{graphicx}
\usepackage{caption}
\usepackage{amsmath}
\usepackage{bm}
\usepackage{framed}
\usepackage{authblk}
\usepackage{empheq}
\usepackage{amsfonts}
\usepackage{esint}
\usepackage[makeroom]{cancel}
\usepackage{dsfont}
\usepackage{centernot}
\usepackage{mathtools}
\usepackage{bigints}
\usepackage{amsthm}
\theoremstyle{definition}
\newtheorem{defn}{Definition}[section]
\newtheorem{prop}{Proposition}[section]
\newtheorem{rmk}{Remark}[section]
\newtheorem{thm}{Theorem}[section]
\newtheorem{exmp}{Example}[section]
\newtheorem{prob}{Problem}[section]
\newtheorem{sln}{Solution}[section]
\newtheorem*{prob*}{Problem}
\newtheorem{exer}{Exercise}[section]
\newtheorem*{exer*}{Exercise}
\newtheorem*{sln*}{Solution}
\usepackage{empheq}
\usepackage{hyperref}
\usepackage{tensor}
\usepackage{xcolor}
\hypersetup{
	colorlinks,
	linkcolor={black!50!black},
	citecolor={blue!50!black},
	urlcolor={blue!80!black}
}


\newcommand*\widefbox[1]{\fbox{\hspace{2em}#1\hspace{2em}}}

\newcommand{\p}{\partial}
\newcommand{\R}{\mathbb{R}}
\newcommand{\C}{\mathbb{C}}
\newcommand{\lag}{\mathcal{L}}
\newcommand{\nn}{\nonumber}
\newcommand{\ham}{\mathcal{H}}
\newcommand{\M}{\mathcal{M}}
\newcommand{\I}{\mathcal{I}}
\newcommand{\K}{\mathcal{K}}
\newcommand{\F}{\mathcal{F}}
\newcommand{\w}{\omega}
\newcommand{\lam}{\lambda}
\newcommand{\al}{\alpha}
\newcommand{\be}{\beta}
\newcommand{\x}{\xi}

\newcommand{\G}{\mathcal{G}}

\newcommand{\f}[2]{\frac{#1}{#2}}

\newcommand{\ift}{\infty}

\newcommand{\lp}{\left(}
\newcommand{\rp}{\right)}

\newcommand{\lb}{\left[}
\newcommand{\rb}{\right]}

\newcommand{\lc}{\left\{}
\newcommand{\rc}{\right\}}


\newcommand{\V}{\mathbf{V}}
\newcommand{\U}{\mathcal{U}}
\newcommand{\Id}{\mathcal{I}}
\newcommand{\D}{\mathcal{D}}
\newcommand{\Z}{\mathcal{Z}}

%\setcounter{chapter}{-1}


\makeatletter
\renewcommand{\@chapapp}{Part}
%\renewcommand\thechapter{$\bf{\ket{\arabic{chapter}}}$}
%\renewcommand\thesection{$\bf{\ket{\arabic{section}}}$}
%\renewcommand\thesubsection{$\bf{\ket{\arabic{subsection}}}$}
%\renewcommand\thesubsubsection{$\bf{\ket{\arabic{subsubsection}}}$}
\makeatother



\usepackage{subfig}
\usepackage{listings}
\captionsetup[lstlisting]{margin=0cm,format=hang,font=small,format=plain,labelfont={bf,up},textfont={it}}
\renewcommand*{\lstlistingname}{Code \textcolor{violet}{\textsl{Mathematica}}}
\definecolor{gris245}{RGB}{245,245,245}
\definecolor{olive}{RGB}{50,140,50}
\definecolor{brun}{RGB}{175,100,80}
\lstset{
	tabsize=4,
	frame=single,
	language=mathematica,
	basicstyle=\scriptsize\ttfamily,
	keywordstyle=\color{black},
	backgroundcolor=\color{gris245},
	commentstyle=\color{gray},
	showstringspaces=false,
	emph={
		r1,
		r2,
		epsilon,epsilon_,
		Newton,Newton_
	},emphstyle={\color{olive}},
	emph={[2]
		L,
		CouleurCourbe,
		PotentielEffectif,
		IdCourbe,
		Courbe
	},emphstyle={[2]\color{blue}},
	emph={[3]r,r_,n,n_},emphstyle={[3]\color{magenta}}
}


\begin{document}
\begin{titlepage}\centering
 \clearpage
 \title{{\textsc{\textbf{PHYSICS at PERIMETER:\\  Topics in Theoretical Physics \\ 
 				\&\\ Quantum Simulation Research}}}\\ \smallskip - A Quick Guide - \\}
 \author{\bigskip Huan Q. Bui}
  \affil{Colby College\\$\,$\\ PHYSICS \& MATHEMATICS\\ Statistics \\$\,$\\Class of 2021\\}
 \date{\today}
 \maketitle
 \thispagestyle{empty}
\end{titlepage}

\subsection*{Preface}
\addcontentsline{toc}{subsection}{Preface}

Greetings,\\

This guide is my notes from Perimeter Institute of Theoretical Physics Summer School, 2020. Topics include quantum information, thermodynamics, numerical methods, condensed matter physics, path integrals, and symmetries.\\

Enjoy!  



\newpage
\tableofcontents
\newpage





\chapter{Quantum Information \& Thermodynamics} 

The aim of this course is to understand the thermodynamics of quantum systems and in the process to learn some fundamental tools in Quantum Information. We will focus on the topics of foundations of quantum statistical mechanics, resource theories, entanglement, fluctuation theorems, and quantum machines. 

\newpage

\section{Foundations of Quantum Statistical Mechanics - Entanglement}
\newpage

\section{Resource Theories and Quantum Information}


\newpage
\section{Quantum Thermal Operations}


\newpage
\section{Fluctuation Theorems and Quantum Information}


\newpage
\section{Quantum Thermal Machines}






\newpage
\chapter{Numerical Methods \& Condensed Matter Physics}

This course has two main goals: (1) to introduce some key models from condensed matter physics; and (2) to introduce some numerical approaches to studying these (and other) models.  As a precursor to these objectives, we will carefully understand many-body states and operators from the perspective of condensed matter theory.  (However, I will cover only spin models.  We will not discuss or use second quantization.)\\



Once this background is established, we will study the method of exact diagonalization and write simple python programs to find ground states, correlation functions, energy gaps, and other properties of the transverse-field Ising model and XXZ model.  We will also discuss the computational limitations of exact diagonalization.  Finally, I will introduce the concept of matrix product states, and we will see how these can be used with algorithms such as the density matrix renormalization group (DMRG) to study ground state properties for much larger systems than can be studied with exact diagonalization.




\newpage

\section{Lecture 1}
\subsection{Introduction to many-particle states and operators}
\newpage
\subsection{Introduction to Ising and XXZ models}
\newpage
\subsection{Programming basics}
\newpage
\subsection{Finding expectation values}
\newpage

\section{Lecture 2: Exact diagonalization part 1}
\subsection{Representing models}
\newpage
\subsection{Finding eigenstates}
\newpage
\subsection{Energy gaps}
\newpage
\subsection{Phase transitions}
\newpage




\section{Lecture 3: Exact diagonalization part 2}
\subsection{Limitations of the method}
\newpage
\subsection{Using symmetries}
\newpage
\subsection{Dynamics}
\newpage




\section{Lecture 4: Matrix product states part 1}
\subsection{Entanglement and the singular value decomposition}
\newpage
\subsection{What is a matrix product state and why is it useful}
\newpage


\section{Lecture 5: Matrix product states part 2}
\subsection{Algorithms for finding ground states using matrix product states: iTEBD }

\newpage


\subsection{Algorithms for finding ground states using matrix product states: DMRG }

\newpage











\chapter{Path Integrals}

The goal of this course is to introduce the path integral formulation of quantum mechanics and a few of its applications. We will begin by motivating the path integral formulation and explaining its connections to other formulations of quantum mechanics and its relation to classical mechanics. We will then explore some applications of path integrals.


\newpage

\section{Introduction to path integrals and the semi-classical limit}

\newpage
\section{Propagator in real and imaginary time}


\newpage

\section{Perturbation theory}


\newpage

\section{Non-perturbative physics and quantum tunneling}


\newpage


\section{Topology and path integrals}


\newpage



\chapter{Symmetries}



The aim of this course is to  explore some of the many ways in which symmetries play a role in physics. We’ll start with an overview of the concept of symmetries and their description in the language of  group theory. We will then discuss continuous symmetries and infinitesimal symmetries, their fundamental role in Noether's theorem, and their formalization in terms of Lie groups and Lie algebras. In the last part of the course we will focus on symmetries in quantum theory and introduce representations of (Lie) groups and Lie algebras.


\newpage

\section{Lecture 1: Overview}
\subsection{Definition of symmetry}
\newpage
\subsection{Elements of group theory}
\newpage
\subsection{Examples }
\newpage



\section{Lecture 2}
\subsection{Continuous and discrete symmetries}
\newpage
\subsection{Infinitesimal symmetries}
\newpage
\subsection{Noether's theorem}


\newpage

\section{Lecture 3: Lie groups and Lie algebras}
\newpage

\section{Lecture 4: Symmetries in quantum mechanics}
\newpage
\section{Representation theory}



\newpage



\chapter{Research Project: Quantum Simulation}

\textbf{Quantum many-body physics on quantum hardware}\\


\underline{Project description:} Recently, there have been significant advances in several quantum computing platforms, including superconducting qubits and trapped ions. At this point, nontrivial quantum operations can be implemented
on tens of qubits, and the frontier continues to expand. This project will explore how these emerging
technologies can assist the realization and understanding of complex many-body quantum systems,
regarding both static aspects such as ground state properties and dynamical aspects such as quantum chaos
and thermalization. What new physics can we learn from near-term quantum computers? The ideal
outcome is an interesting and realistic proposal that can be carried out in one of the quantum platforms (see
for example the interplay between theoretical proposals and experiment).



\newpage


\section{Review: The density operator}


We often use the language of state vectors in quantum mechanics. An alternative language is that of \textit{density matrices/density operators}. We will use this language extensively in quantum information/quantum computation. First, we will introduce the formulation. Second, we look at some properties of the density operator. Finally, we look at an application where the density operator really shines -- as a tool for describing \textit{individual subsystems} of a composite quantum system.  


\subsection{Ensembles of Quantum States}

The density operator language provides a convenient means for describing quantum systems whose state is not completely known. Suppose a quantum system is in one of the states $\ket{\psi_i}$ where $i$ is an index, with respective probabilities $p_i$. We call $\{p_i, \ket{\psi_i}\}$ an \textit{ensemble of pure states}. \\

\begin{defn}[Density Operator]
	\begin{align}
	\boxed{\rho \equiv \sum_i p_i \ket{\psi_i}\bra{\psi_i}}
	\end{align}
\end{defn}



Suppose we let a unitary $\U$ act on a closed system. If the system was initially in the state $\ket{\psi_i}$ with probability $p_i$ then we get $\U \ket{\psi_i}$ with probability $p_i$. The density operator evolves as
\begin{align}
\rho = \sum_i p_i \ket{\psi_i}\bra{\psi_i} \stackrel{\U}{\rightarrow} \sum_i p_i \U \ket{\psi_i}\bra{\psi_i}\U^\dagger = \U \rho \U^\dagger.
\end{align}


Suppose we have a measurement gate $\M_m$. If the intial state is $\ket{\psi_i}$ then the probability of getting result $m$ is 
\begin{align}
p(m\vert i) = \bra{\psi_i} \M_m^\dagger \M_m \ket{\psi_i} = \tr\lp \M_m^\dagger \M_m \braket{\psi_i} \rp
\end{align}
where we have used the identity:
\begin{align}
\tr(A\ket{\psi}\bra{\psi}) = \sum_i \bra{i}A \ket{\psi}\bra{\psi}\ket{i} = \bra{\psi}A\ket{\psi}.
\end{align}
With this, the total probability of measuring $m$ is 
\begin{align}
p(m) = \sum_i p(m|i)p_i = \sum_i p_i \tr\lp \M_m^\dagger \M_m \ket{\psi_i}\bra{\psi_i} \rp = \tr\lp \M_m^\dagger \M_m \rho \rp.
\end{align}

The state after the measurement is thus
\begin{align}
\ket{\psi_i^m} = \f{\M_m \ket{\psi_i}}{\sqrt{\bra{\psi_i}\M_m^\dagger \M_m \ket{\psi_i}}}.
\end{align}
The corresponding density operator for this state is thus
\begin{align}
\rho_m = \sum_i p(i|m)\ket{\psi_i^m}\bra{\psi_i^m} = \sum_i p(i|m)\f{\M_m \ket{\psi_i}\bra{\psi_i}\M_m^\dagger}{\bra{\psi_i}\M_m^\dagger \M_m \ket{\psi_i}}.
\end{align}
From probability theory we have $p(i|m) = p(m|i)p_i/p(m)$, so:
\begin{align}
\rho_m = \sum_i p_i \f{\M_m \ket{\psi_i}\bra{\psi_i}\M_m^\dagger}{\tr\lp \M^\dagger_m \M_m \rho \rp} = \f{\M_m \rho \M_m^\dagger}{\tr\lp \M_m^\dagger \M_m \rho \rp}.
\end{align}




Finally, suppose we have a quantum system in state $\rho_i$ with probability $p_i$. The system might be described by the density matrix
\begin{align}
\rho = \sum_i p_i \rho_i.
\end{align}
Here's why:
\begin{align}
\rho = \sum_{i,j}p_i p_{ij}\ket{\psi_{ij}}\bra{\psi_{ij}} = \sum_i p_i \rho_i
\end{align}
where we have used the definition
\begin{align}
\rho_i = \sum_{j}p_{ij}\ket{\psi_{ij}}\bra{\psi_{ij}}.
\end{align}

We call $\rho$ the \textit{mixture} of the states $\rho_i$ with probabilities $p_i$. This concept of mixture comes up up repeatedly in the analysis of problems like quantum noise, where the effect of the noise is to introduce ignorance into our knowledge of the quantum state. A simple example is provided by the measurement scenario. Imagine that for some reason out record of the result $m$ of the measurement was lost. We would have a quantum system in the state $\rho_m$ with probability $p(m)$, but would no longer know the actual value of $m$. The state of such a quantum system would therefore by described by the density operator
\begin{align}
\rho = \sum_m p(m)\rho_m = \sum_m \tr\lp \M_m^\dagger \M_m \rho \rp\f{\M_m \rho \M_m^\dagger}{\tr\lp \M_m^\dagger \M_m \rho \rp} = \sum_m \M_m \rho \M_m^\dagger.
\end{align}



























\newpage



\subsection{General Properties of the Density Operator}



\begin{thm}[Characterization of density operators] 
	An operator $\rho$ is the density operator to some ensemble $\{p_i, \ket{\psi_i} \}$ if and only if it statisfies the conditions
	\begin{itemize}
		\item \textbf{Trace condition:} $\tr(\rho) = 1$.
		\item \textbf{Positivity:} $\rho$ is a positive operator, i.e., $\bra{\varphi}\rho\ket{\varphi} \geq 0 \forall \varphi$.
	\end{itemize}
\end{thm}


With this definition we can reformulate the postulates of quantum mechanics in the language of density operators as 
\begin{itemize}
	\item \textbf{Postulate 1:}
	
	\item \textbf{Postulate 2:}
	
	
	\item \textbf{Postulate 3:}
	
	
	\item \textbf{Postulate 4:}
\end{itemize}


\newpage


\subsection{The Reduced Density Operator}


\newpage




















\end{document}
