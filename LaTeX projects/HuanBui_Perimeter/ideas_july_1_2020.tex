\documentclass{article}
\usepackage{geometry}
\geometry{
	letterpaper,
	left=20mm,
	right=20mm,
	top=10mm,
	bottom=20mm
}
\usepackage{physics}
\usepackage{graphicx}
\usepackage{caption}
\usepackage{amsmath}
\usepackage{bm}
\usepackage{framed}
\usepackage{authblk}
\usepackage{empheq}
\usepackage{amsfonts}
\usepackage{esint}
\usepackage[makeroom]{cancel}
\usepackage{dsfont}
\usepackage{centernot}
\usepackage{mathtools}
\usepackage{bigints}
\usepackage{amsthm}
\theoremstyle{definition}
\newtheorem{defn}{Definition}[section]
\newtheorem{prop}{Proposition}[section]
\newtheorem{rmk}{Remark}[section]
\newtheorem{thm}{Theorem}[section]
\newtheorem{exmp}{Example}[section]
\newtheorem{prob}{Problem}[section]
\newtheorem{sln}{Solution}[section]
\newtheorem*{prob*}{Problem}
\newtheorem{exer}{Exercise}[section]
\newtheorem*{exer*}{Exercise}
\newtheorem*{sln*}{Solution}
\usepackage{empheq}
\usepackage{hyperref}
\usepackage{tensor}
\usepackage{xcolor}
\hypersetup{
	colorlinks,
	linkcolor={black!50!black},
	citecolor={blue!50!black},
	urlcolor={blue!80!black}
}


\newcommand*\widefbox[1]{\fbox{\hspace{2em}#1\hspace{2em}}}

\newcommand{\p}{\partial}
\newcommand{\R}{\mathbb{R}}
\newcommand{\C}{\mathbb{C}}
\newcommand{\lag}{\mathcal{L}}
\newcommand{\nn}{\nonumber}
\newcommand{\had}{\mathcal{H}}
\newcommand{\M}{\mathcal{M}}
\newcommand{\I}{\mathcal{I}}
\newcommand{\K}{\mathcal{K}}
\newcommand{\F}{\mathcal{F}}
\newcommand{\w}{\omega}
\newcommand{\lam}{\lambda}
\newcommand{\al}{\alpha}
\newcommand{\be}{\beta}
\newcommand{\x}{\xi}
\newcommand{\ep}{\epsilon}

\newcommand{\G}{\mathcal{G}}

\newcommand{\f}[2]{\frac{#1}{#2}}
\newcommand{\td}[1]{\tilde{#1}}


\newcommand{\ift}{\infty}

\newcommand{\lp}{\left(}
\newcommand{\rp}{\right)}

\newcommand{\lb}{\left[}
\newcommand{\rb}{\right]}

\newcommand{\lc}{\left\{}
\newcommand{\rc}{\right\}}

\newcommand{\la}{\langle}
\newcommand{\ra}{\rangle}

\newcommand{\fig}[2]{
	\begin{figure}[!htb]
		\centering
		\includegraphics[scale=#1]{#2}
	\end{figure}}


\newcommand{\V}{\mathbf{V}}
\newcommand{\U}{\mathcal{U}}
\newcommand{\Id}{\mathcal{I}}
\newcommand{\D}{\mathcal{D}}
\newcommand{\Z}{\mathcal{Z}}

%\setcounter{chapter}{-1}


\usepackage{subfig}
\usepackage{listings}
\captionsetup[lstlisting]{margin=0cm,format=hang,font=small,format=plain,labelfont={bf,up},textfont={it}}
\renewcommand*{\lstlistingname}{Code \textcolor{violet}{\textsl{Mathematica}}}
\definecolor{gris245}{RGB}{245,245,245}
\definecolor{olive}{RGB}{50,140,50}
\definecolor{brun}{RGB}{175,100,80}
\lstset{
	tabsize=4,
	frame=single,
	language=mathematica,
	basicstyle=\scriptsize\ttfamily,
	keywordstyle=\color{black},
	backgroundcolor=\color{gris245},
	commentstyle=\color{gray},
	showstringspaces=false,
	emph={
		r1,
		r2,
		epsilon,epsilon_,
		Newton,Newton_
	},emphstyle={\color{olive}},
	emph={[2]
		L,
		CouleurCourbe,
		PotentielEffectif,
		IdCourbe,
		Courbe
	},emphstyle={[2]\color{blue}},
	emph={[3]r,r_,n,n_},emphstyle={[3]\color{magenta}}
}


\begin{document}
	\begin{center}
		\huge{Protocol Idea \& Questions}
	\end{center}	
	\begin{center}
		Huan Bui \\ \today
	\end{center}


\section{Protocol Idea }

After our meeting, I spent some more time thinking about the idea of using measurements and local unitaries to ``shrink'' and turn the ansatz (of large system size) to the ground state (a smaller system). Now, I want to run it through you again to see if it makes sense do this. \\

 
\noindent Suppose we want to prepare a $k$-qubit TFIM critical ground state. What I'm thinking is perhaps we can start with a ring-like cluster state of $k\times2^n$ qubits in for some $n$. The goal is to achieve the $k$-qubit critical state after $\sim n$ measurement layers, where $n \sim \log(k)$. \\


\noindent Here's the protocol I'm thinking of:
\begin{itemize}
	\item To start, we measure every other qubit on the ring in the $X$-basis, leaving the unmeasured $k\times2^{(n-1)}$ qubits in the highly entangled GHZ state (up to some bit flips). The effective system size is halved.
	
	\item Apply a set of non-uniform local unitaries $\{\mathcal{U}_i\}_{i=1}^p$ across the unmeasured qubits. The set of these unitaries will be parameterized by $p>2$ parameters (where 2 is what a single layer of QAOA has).
	
	\item Compute the cost function to find a new parameters for the next set of unitaries for the next layer of measurements (which shrinks the effective system size down further).(*)
	

	\item Repeat until convergence. 
\end{itemize}


  
\noindent \textbf{(*):} There's one detail regarding computing the cost function, though, because now we have not only the full system size ( $L_{\text{full}}=k\times 2^n$ qubits) but also an effective system size  which approaches $k$ as we go through the layers of measurements ($L_{\text{eff}} \downarrow k$). \\
 
\noindent Unlike in QAOA where $L_{\text{full}} = L_{\text{eff}}$ (since the number of effective qubits doesn't change), here $L_{\text{eff}} \downarrow k$ following each layer of measurements. As a result, we don't have a trial $\ket{\psi_{\text{trial}}}$ in the same Hilbert space as $\ket{\psi_{\text{target}}}$  to compute the cost function
\begin{equation*}
\bra{\psi_{\text{target}}} \had \ket{\psi_{\text{trial}}}
\end{equation*}
as in QAOA. To circumvent this issue, I think we can embed $\ket{\psi_{\text{target}}}$ into a larger Hilbert space by adjoining it with ancilla qubits $\bigotimes \ket{+}$ so that $\ket{\psi_{\text{target}}}\ket{\bigotimes+}$ and $\ket{\psi_{\text{trial}}}$ are in the same Hilbert space. With this, we can compute the cost function:
\begin{equation*}
\bra{\psi_{\text{target}}} \bra{\bigotimes +} (\had \otimes \Id) \ket{\psi_{\text{trial}}}.
\end{equation*}
\noindent I believe that by minimizing this new quantity at each layer, we will still be on the right track to minimizing the original cost function  $\bra{\psi_{\text{target}}} \had \ket{\psi_{\text{trial}}}$. The intuition here being that the ancilla qubits and the identity operator don't affect the optimization.\\


\noindent If this protocol works and if $n$ scales as $\log(k)$ then we have a protocol whose depth scales as $\log(k)$. Not only that, the number of qubits required also doesn't scale too fast ($\sim k^{\al}$ where $1 < \al < 2$).


\section{Questions} 

I want to give this idea a test, but I want to run it through you before perhaps trying to implement it. 

\begin{itemize}
	\item Am I explaining the protocol clearly enough? Does the protocol make sense?  
	\item Is there a limitation or potential problem here that you can foresee? 
	\item Do you see a potential issue with having too many qubits in the beginning (the full system size in this protocol scales as $\sim k^\al$ where $1 < \al < 2$)? If that's the case, we can also choose not to measure half the remaining qubits at each layer.
\end{itemize}

	
	
	
	
	
	
	
	
	
	
	
\end{document}