\documentclass{article}
\usepackage{physics}
\usepackage{graphicx}
\usepackage{caption}
\usepackage{amsmath}
\usepackage{authblk}
\usepackage{amsfonts}
\usepackage{esint}
\usepackage{mathtools}
\usepackage{amsthm}
\theoremstyle{definition}
\newtheorem{defn}{Definition}[section]
\newtheorem{prop}{Proposition}[section]
\newtheorem{rmk}{Remark}[section]
\newtheorem{thm}{Theorem}[section]
\newtheorem{exmp}{Example}[section]
\newtheorem{prob}{Problem}[section]
\newtheorem{sln}{Solution}[section]
\newtheorem*{prob*}{Problem}
\newtheorem{exer}{Exercise}[section]
\newtheorem*{exer*}{Exercise}
\newtheorem*{sln*}{Solution}
\usepackage{empheq}
\usepackage{hyperref}
\usepackage{tensor}
\usepackage{xcolor}
\hypersetup{
	colorlinks,
	linkcolor={black!50!black},
	citecolor={blue!50!black},
	urlcolor={blue!80!black}
}
\newcommand{\p}{\partial}
\newcommand{\R}{\mathbb{R}}
\newcommand{\C}{\mathbb{C}}
\newcommand{\lag}{\mathcal{L}}
\newcommand{\I}{\mathcal{I}}
\newcommand{\K}{\mathcal{K}}
\newcommand{\F}{\mathcal{F}}
\newcommand{\w}{\omega}
\newcommand{\lam}{\lambda}
\newcommand{\al}{\alpha}
\newcommand{\be}{\beta}
\newcommand{\x}{\xi}

\newcommand{\f}[2]{\frac{#1}{#2}}

\newcommand{\ift}{\infty}

\newcommand{\lp}{\left(}
\newcommand{\rp}{\right)}

\newcommand{\lb}{\left[}
\newcommand{\rb}{\right]}

\newcommand{\lc}{\left\{}
\newcommand{\rc}{\right\}}


\newcommand{\V}{\mathbf{V}}
\newcommand{\U}{\mathcal{U}}
\newcommand{\Id}{\mathcal{I}}






\usepackage{subfig}
\usepackage{listings}
\captionsetup[lstlisting]{margin=0cm,format=hang,font=small,format=plain,labelfont={bf,up},textfont={it}}
\renewcommand*{\lstlistingname}{Code \textcolor{violet}{\textsl{Mathematica}}}
\definecolor{gris245}{RGB}{245,245,245}
\definecolor{olive}{RGB}{50,140,50}
\definecolor{brun}{RGB}{175,100,80}
\lstset{
	tabsize=4,
	frame=single,
	language=mathematica,
	basicstyle=\scriptsize\ttfamily,
	keywordstyle=\color{black},
	backgroundcolor=\color{gris245},
	commentstyle=\color{gray},
	showstringspaces=false,
	emph={
		r1,
		r2,
		epsilon,epsilon_,
		Newton,Newton_
	},emphstyle={\color{olive}},
	emph={[2]
		L,
		CouleurCourbe,
		PotentielEffectif,
		IdCourbe,
		Courbe
	},emphstyle={[2]\color{blue}},
	emph={[3]r,r_,n,n_},emphstyle={[3]\color{magenta}}
}


\begin{document}
\begin{titlepage}\centering
 \clearpage
 \title{\textsc{\bf{QUANTUM MECHANICS}}\\\smallskip A Quick Guide\\}
 \author{\bigskip Huan Q. Bui}
  \affil{Colby College\\$\,$\\ PHYSICS \& MATHEMATICS\\ Statistics \\$\,$\\Class of 2021\\}
 \date{\today}
 \maketitle
 \thispagestyle{empty}
\end{titlepage}

\subsection*{Preface}
\addcontentsline{toc}{subsection}{Preface}

Greetings,\\

\textbf{Quantum Mechanics, A Quick Guide to...} is my reading notes from Shankar's \textit{Principles of Quantum Mechanics, Second Edition}. Additional material will come from my class notes and my comments/interpretations/solutions.\\

A strong background in linear algebra will be very helpful. I will try to cover some of the mathematical background, but a lot of familiarity will be assumed. \\

Enjoy!

\newpage
\tableofcontents
\newpage

\section{Mathematical Introduction}

\subsection{Linear Vector Spaces}

We should familiar with defining characteristics of linear vector spaces at this point. Here are some important definitions/theorems again:

\begin{defn}
	A linear vector space $\textbf{V}$ is a collection of objects called \textit{vectors} for which there exists
	
	\begin{enumerate}
		\item A definite rule for summing, and
		\item A definite rule for scaling, with the following features:
		
		
		\begin{itemize}
			\item Closed under addition: for $x,y \in \V$, $x+y \in \V$.
			\item Closed under scalar multiplication: $x\in \V$, then $ax \in \V$ for some scalar $a$.
			\item Scalar multiplication is distributive. 
			\item Scalar multiplication is associative.
			\item Addition is commutative.
			\item Addition is associative.
			\item There exists a (unique) null element in $\V$.
			\item There exists a (unique) additive inverse. 
		\end{itemize}
	\end{enumerate}
\end{defn}


Vector spaces are defined over some field. The field can be real numbers, complex numbers, or it can also be finite. As for good practice, we will begin to label vectors with Dirac bra-ket notation. So, for instance, $\ket{v} \in \V$ denotes vector $v \in \V$. Basic manipulations of these vectors are intuitive:
\begin{enumerate}
	\item $\ket{0}$ is unique, and is the null element.
	\item $0\ket{V} = \ket{0}$.
	\item $\ket{-V} = -\ket{V}$.
	\item $\ket{-V}$ is a unique additive inverse of $\ket{V}$.
\end{enumerate} 

The reasons for choosing to use the Dirac notation will become clear later on. Another important basic concept is \textit{linear (in)dependence}. Of course, there are a number of equivalent statement for linear independence. We shall just give one here:

\begin{defn}
	A set of vectors is said to be linearly independent if the only linear relation 
	\begin{align}
	\sum^n_{i=1}a_i\ket{i} = \ket{0}
	\end{align}
	is the trivial one where the components $a_i = 0$ for any $i$. 
\end{defn}



The next two basic concepts are \textit{dimension} and \textit{basis}. 

\begin{defn}
	A vector space $\V$ has dimension $n$ if it can accommodate a maximum of $n$ linearly independent vectors. We denote this $n$-dimensional vector space as $\V^n$.
\end{defn}

We can show that 

\begin{thm}
	Any vector $\ket{v} \in \V^n$ can be written (uniquely) as a linear combination of any $n$ linearly independent vectors.  
\end{thm}


\begin{defn}
	A set of $n$ linearly independent vectors in a $n$-dimensional space is called a \textit{basis}. So if $\ket{1},\dots,\ket{n}$ form a basis for $\V^n$, then any $\ket{v}\in \V$ can be written uniquely as
	\begin{align}
	\ket{v} = \sum^n_{i=1}a_i\ket{i}.
	\end{align}
\end{defn}

It is nice to remember the following:
\begin{align}
\boxed{\text{Linear Independence} = \text{Basis} + \text{Span}}
\end{align}

When a collection of vectors span a vector space $\V$, it just means that any $\ket{v} \in \V$ can be written as a linear combination of (some of) these vectors. 


The algebra of linear combinations is quite intuitive. If $\ket{v} = \sum_i a_i\ket{i}$ and $\ket{w} = \sum_i b_i\ket{i}$ then 

\begin{enumerate}
	\item $\ket{v + w} = \sum_i (a_i + b_i)\ket{i}$.
	\item $c\ket{v} = c\sum_i a_i\ket{i} = \sum_i ca_i\ket{i}$.
\end{enumerate}



A linear algebra text will of course provide a much better coverage of these topics. 















\subsection{Inner Product Spaces}


A generalization of the familiar dot product is the \textit{inner product} or the \textit{scalar product}. An inner product between two vectors $\ket{v}$ and $\ket{w}$ is denoted $\braket{v|w}$. An inner product has to satisfy the following properties:

\begin{enumerate}
	\item Conjugate symmetry (or skew-symmetry):$\braket{v}{w} = \braket{w}{v}^*$.
	\item Positive semi-definiteness: $\braket{v}{v} \geq 0$.
	\item Linearity in ket: $\braket{v}{aw + bz} = a\braket{v}{w} + b\braket{v}{z}$.
	\item Conjugate-linearity in bra: $\braket{av + bz}{w} = \bar{a}\braket{v}{w} + \bar{b}\braket{z}{w}$.
\end{enumerate}




\begin{defn}
	An inner product space is a vector space with an inner product. 
\end{defn}


\begin{defn}
	$\innerproduct{v}{w} = 0 \iff \ket{v} \perp \ket{w}$. 
\end{defn}


\begin{defn}
	The \textit{norm} (or length) of $\ket{v}$ is defined as 
	\begin{align}
	\norm{v} = \sqrt{\braket{v}}.
	\end{align}
	Unit vectors have unit norm. Unit vectors are said to be \textit{normalized}.  
\end{defn}




\begin{defn}
	A set of basis vectors all of unit norm, which are pairwise orthogonal will be called an \textit{orthonormal basis} or ONB. 
\end{defn}


Let $\ket{v} = \sum_i a_i\ket{i}$ and $\ket{w} = \sum_i b_i \ket{j}$, then 
\begin{align}
\braket{v}{w} = \sum_i a_i^*b_i \braket{i}{j}.
\end{align}



\begin{thm}
	\textbf{Gram-Schmidt:} Given a linearly independent basis, we can form linear combinations of the basis vectors to obtain an orthonormal basis. 
\end{thm}

Suppose that the Gram-Schmidt process gives us an ONB then we have
\begin{align}
\braket{i}{j} = \delta_{ij}.
\end{align}
As a result,
\begin{align}
\braket{v}{w} = \sum_i v_i^*w_i.
\end{align}
Alternatively, we can think this as doing the standard inner products of vectors whose entries are the components of the vectors $\ket{v}$, $\ket{w}$ in the basis:
\begin{align}
\ket{v} \to \begin{bmatrix}
v_1\\v_2\\\vdots\\v_n
\end{bmatrix}\hspace{0.5cm}
\ket{w} \to \begin{bmatrix}
w_1\\w_2\\\vdots\\w_n
\end{bmatrix} \implies \braket{v}{w} = \begin{bmatrix}
v_1^* & v_2^* & \dots & v_n^*
\end{bmatrix}\begin{bmatrix}
w_1\\ w_2 \\\vdots \\ w_n
\end{bmatrix}.
\end{align}
We can also easily see that 
\begin{align}
\braket{v}{v} = \sum_i \abs{v_i}^2 \geq 0.
\end{align}

\subsection{Dual Spaces and Dirac Notation}
Here we deal with some technical details involving the \textit{ket} (the column vectors) and the \textit{bra} (the row vectors). Column vectors are concrete manifestations of an abstract vector $\ket{v}$ in a basis, and we can work backward to go from the column vectors to the kets. We can do a similar thing with the bra vectors - since there's nothing special about writing the entries is a column versus in a row. However, we will do the following. We know that associated with every ket $\ket{v}$ is a column vector. So let its adjoint, which is a row vector, be associated with the bra, called $\bra{v}$. Now, we have two vector spaces, the space of kets and the dual space of bras. There is a basis of vectors $\ket{i}$ for expanding kets and a similar basis $\bra{i}$ for expanding bras. 

\subsubsection{Expansion of Vectors in an ONB}
It is extremely useful for us to be able to express a vector in an ONB. Suppose we have a vector $\ket{v}$ in an ONB $\ket{i}$. Then, let $\ket{v}$ be written as
\begin{align}
\ket{v} = \sum_i v_i \ket{i}.
\end{align}
To find the components $v_i$, we take the inner product of $\ket{v}$ with $\ket{j}$:
\begin{align}
\braket{j}{v} = \sum_i v_i \braket{j}{i} = \sum_i v_i\delta_{ij} = v_j.
\end{align}
With this, we can rewrite the vector $\ket{v}$ in the basis $\ket{i}$ as
\begin{align}
\ket{v} = \sum_i \ket{i}\braket{i}{v}.
\end{align}



\subsubsection{Adjoint Operations}
Here is a few details regarding taking the adjoints of vectors. Suppose that
\begin{align}
\ket{v} = \sum_i v_i\ket{i} = \sum_i \ket{i}\braket{i}{v}.
\end{align}
Then,
\begin{align}
\bra{v} = \sum_i\ket{i}v_i^*.
\end{align}
Now, because $v_i = \braket{i}{v}$, we have $v_i^* = \braket{v}{i}$. Thus, 
\begin{align}
\bra{v} = \sum_i \braket{v}{i}\bra{i}.
\end{align}
In plain words, the rule for taking the adjoint is the following. To take the adjoint of an equation involving bras and kets and coefficients, reverse the order of all factors, exchanging bras and kets and complex conjugating all coefficients. 


\subsubsection{Gram-Schmidt process}
Again, the Gram-Schmidt process lets us convert a linearly independent basis into an orthonormal one. For a two-dimensional case, procedure is the following:
\begin{enumerate}
	\item Rescale the first by its own length, so it becomes a unit vector. This is the first (orthonormal) unit vector.
	\item Subtract from the second vector its projection along the first, leaving behind only the part perpendicular to the first. (Such a part will remain since by assumption the vectors are nonparallel).
	\item Rescale the left over piece by its own length. We now have the second basis vector: it s orthogonal to the first and of unit length.
\end{enumerate}

In general, let $\ket{I}, \ket{II}, \dots$ be a linearly independent basis. The first vector of the orthonormal basis will be
\begin{align}
\ket{1} = \f{\ket{I}}{\norm{\ket{I}}}.
\end{align}
For the second vector in the basis, consider
\begin{align}
\ket{2'} = \ket{II} - \ket{1}\braket{1}{II}.
\end{align}
We can see that $\ket{2'}$ is orthogonal to $\ket{1}$:
\begin{align}
\braket{1}{2'} = \braket{1}{II} - \braket{1}\braket{1}{II} = 0.
\end{align}
So dividing $\ket{2'}$ by its norm gives us, $\ket{2}$, the second element in the ONB. To find the third element in the ONB, we have to first make sure it is orthogonal to both $\ket{I}$ and $\ket{II}$, so let us consider
\begin{align}
\ket{3'}= \ket{III} - \ket{1}\braket{1}{III} - \ket{2}\braket{2}{III}.
\end{align}
Once again we have $\ket{3'}$ orthogonal to both $\ket{1}$ and $\ket{2}$. Normalizing $\ket{3'}$ gives us $\ket{3}$, the third element in the ONB. We can now see how this process continues to the last element. 



\subsubsection{Schwarz and Triangle Inequality}
Just two small yet very important details:
\begin{thm}
	Schwarz Inequality:
	\begin{align}
	\abs{\braket{v}{w}} \leq \norm{v}\norm{w}
	\end{align}
\end{thm}


\begin{thm}
	Triangle Inequality:
	\begin{align}
	\norm{v+w} \leq \norm{v} + \norm{w}.
	\end{align}
\end{thm}



\subsection{Subspaces, Sum and Direct Sum of Subspaces}
I'm not too happy with the definitions given by Shankar's book. He also uses the notation for direct sum to indicate vector space addition, which is very confusing. Any linear algebra textbook would provide better definitions. For equivalent statements about directness of vector space sums, check out my \href{https://huanqbui.com/LaTeX\%20projects/Matrix_Analysis/HuanBui_MatrixAnalysis.pdf}{Matrix Analysis} notes. 




\subsection{Linear Operators}
Again, a rigorous definition of an operator can be found in almost any linear algebra textbook. But here,we can simply think of an operator as just some linear transformation from a vector space to itself. Say, if $\Omega$ is some operator that sends $\ket{v}$ to $\ket{v'}$, we write
\begin{align}
\Omega\ket{v} = \ket{v'}.
\end{align}
By definition, $\ket{v}$ and $\ket{v'}$ are contained in the same vector space. Now, we note that $\Omega$ can also act on bras:
\begin{align}
\bra{v}\Omega = \bra{v'}.
\end{align}
But of course the order of writing things is different, and once again, $\bra{v}$ and $\bra{v'}$ are contained in the same (dual) space. 


Next, because $\Omega$ is linear, we have the following familiar rules:
\begin{align}
\Omega \alpha \ket{v_i} &= \alpha\Omega \ket{v_i}.\\
\Omega \{ \alpha \ket{v_i} + \beta\ket{v_j} \} &= \alpha\Omega\ket{v_i} + \beta\Omega \ket{v_j}.\\
\bra{v_i}\alpha\Omega &= \bra{v_i}\Omega \alpha\\
\{\bra{v_i}\alpha + \bra{v_j}\beta \}\Omega &= \alpha\bra{v_i}\Omega + \beta \bra{v_j}\Omega.
\end{align}


One of the nice features of linear operators is that the action of an operator is completely determined by what it does to the basis vectors. Suppose 
\begin{align}
\ket{v} = \sum_i v_i \ket{i}
\end{align}
and 
\begin{align}
\Omega\ket{i} = \ket{i'},
\end{align}
then
\begin{align}
\Omega \ket{v} = \sum_i \Omega v_i \ket{i}= \sum_iv_i \Omega\ket{i} = \sum_iv_i\ket{i'}.
\end{align}



The next point of interest is \textit{products} of operators. As we might have seen, operators don't always commute. A product of operators applied to a vector just means operators are applied in sequence. The \textit{commutator} of two operators $\Omega, \Lambda$ is defined as
\begin{align}
\Omega\Lambda - \Lambda\Omega \equiv \lb \Omega,\Lambda \rb.
\end{align}
In general, $\lb \Omega,\Lambda\rb$ is not zero. Suppose three operators $\Omega, \Lambda, \Theta$ are involved, then we have two useful relations:
\begin{align}
&\lb \Omega, \Lambda\Theta \rb = \Lambda\lb \Omega, \Theta \rb + \lb \Omega, \Lambda \rb \Theta\\
&\lb \Lambda\Omega, \Theta \rb = \Lambda\lb \Omega, \Theta \rb + \lb \Lambda, \Theta \rb \Omega.
\end{align}
We notice that the form resembles the chain rule in calculus. 





\subsection{Matrix Elements of Linear Operators}
One thing we will hear very often in quantum mechanics is the idea of matrix elements. The idea, it turns out, is very simple. Suppose we have a basis $\ket{i}$, and an operator $\Omega$ such that
\begin{align}
\Omega\ket{i} = \ket{i'}.
\end{align}
Then, for
\begin{align}
\ket{v} = \sum_i v_i \ket{i}, 
\end{align}
we have
\begin{align}
\Omega\ket{v} = \Omega \sum_i v_i \ket{i} = \sum_i v_i \Omega\ket{i} = \sum_i v_i \ket{i'}.
\end{align}
Because we know $\Omega$ and $\ket{i}$, $\ket{i'}$ is also known, as in its components in the basis $\ket{j}$ (un-primed) are known:
\begin{align}
\braket{j}{i'} = \bra{j}\Omega\ket{i} \equiv \Omega_{ji},
\end{align}
where the $n^2$ numbers $\Omega_{ji}$ are the matrix elements of $\Omega$ in this basis. Now, if
\begin{align}
\Omega\ket{v} = \ket{v'}
\end{align}
then the components of the transformed ket $\ket{v'}$ can be expressed in terms of the components of $\ket{v}$ and the matrix elements $\Omega_{ji}$:
\begin{align}
v_i' = \braket{i}{v'} = \bra{i}\Omega\ket{v} = \bra{i}\Omega \sum_j v_j \ket{j} = \sum_j v_j \bra{i}\Omega\ket{j} = \sum_j \Omega_{ij}v_j.
\end{align}
We can see the above equation in matrix form as well:
\begin{align}
\begin{bmatrix}
v_1'  \\ \vdots \\ v_n'
\end{bmatrix}
=
\begin{bmatrix}
\bra{1}\Omega\ket{1} &  \dots & \bra{1}\Omega\ket{n} \\
\vdots & \ddots & \vdots\\	
\bra{n}\Omega\ket{1} & \dots & \bra{n}\Omega\ket{n} \\
\end{bmatrix}
\begin{bmatrix}
v_1 \\ \vdots \\ v_n 
\end{bmatrix}.
\end{align}
The elements of the first column are simply the components of the first transformed basis vector $\ket{1'} = \Omega\ket{1}$ in the given basis. Likewise, the elements of the j$^{\text{th}}$ column represent the image of the j$^\text{th}$ basis vector after $\Omega$ acts on it. 


\subsection{Matrix Elements of Products of Operators}

To get the matrix elements of a product of two operators, we do the following. Suppose we have operators $\Omega$ and $\Lambda$, then
\begin{align}
(\Omega\Lambda)_{ij} = \bra{i}\Omega\Lambda\ket{j} = \bra{i}\Omega \mathcal{I} \Lambda \ket{j}.
\end{align}
Now, we observe that 
\begin{align}
\mathcal{I} = \sum_k \ket{k}\bra{k}. 
\end{align}
So, 
\begin{align}
(\Omega \Lambda)_{ij} = \sum_k \bra{i}\Omega\ket{k}\bra{k}\Lambda \ket{j} = \sum_k \Omega_{ik}\Lambda_{kj}.
\end{align}



\subsection{The Adjoint of an Operator}
Recall that for a scalar $\alpha$
\begin{align}
\bra{\alpha v} = \bra{v}\alpha^*,
\end{align}
then we have a similar thing with operators if
\begin{align}
\Omega\ket{v} = \ket{v'}
\end{align}
then 
\begin{align}
\bra{\Omega v} = \bra{v}\Omega^\dagger,
\end{align}
where $\Omega^\dagger$ is \textit{the} adjoint of $\Omega$. The relationship between $\Omega^\dagger$ and $\Omega$ can be seen in a basis. We consider the matrix elements of $\Omega^\dagger$ in a basis:
\begin{align}
(\Omega^\dagger)_{ij} = \bra{i}\Omega^\dagger \ket{j} = \bra{\Omega i}\ket{j} = \bra{j}\ket{\Omega i}^* = \bra{j} \Omega \ket{i}^* = \Omega_{ji}^*.
\end{align}
We see that
\begin{align}
\Omega^\dagger_{ij} = \Omega^*_{ji},
\end{align}
i.e., in matrix form, $\Omega^\dagger$ is the conjugate transpose of $\Omega$. 

The rule for taking adjoins of equations is rather simple: When a product of operators, bras, kets, ad explicit numerical coefficients is encountered, reverse the order of all factors and make the substitution $\Omega \leftrightarrow \Omega^\dagger$, $\ket{} \leftrightarrow \bra{}$, $a \leftrightarrow a^*$. 


\subsection{Hermitian, Anti-Hermitian, and Unitary Operators}

\begin{defn}
	An operator $\Omega$ is Hermitian $\iff \Omega = \Omega^\dagger$.
\end{defn}


\begin{defn}
	An operator $\Omega$ is anti-Hermitian $\iff \Omega = -\Omega^\dagger$.
\end{defn}

Shankar's book ignores a bigger class of operators called \textit{normal} operators. Normal operators commute with their adjoints. In a sense, normal operators act \textit{like numbers}. Hermitian (or self-adjoint) operators are a subset of normal operators. So, the number-likeness of normal operators carries over to Hermitian operators and anti-Hermitian operators as well. Hermitian and anti-Hermitian operators are like pure real and pure imaginary numbers. Just as every number maybe be decomposed into a sum of pure real and pure imaginary parts, it turns out that we can decompose every operator into its Hermitian and anti-Hermitian parts. 
\begin{align}
\Omega = \f{\Omega + \Omega^\dagger}{2} + \f{\Omega - \Omega^\dagger}{2}.
\end{align}
One can verify that the first terms is Hermitian, and the second term is anti-Hermitian. 




\begin{defn}
	An operator $\mathcal{U}$ is unitary $\iff \mathcal{U}\mathcal{U}^\dagger = \mathcal{I}$. 
\end{defn}

Unitary operators are like complex numbers of unit modulus. 


\begin{thm}
	Unitary operators preserves the inner product between the vectors they act on.
	\begin{proof}
		Suppose
		\begin{align}
		\ket{v'} &= \mathcal{U}\ket{v}\\
		\ket{w'} &= \mathcal{U}\ket{w}.
		\end{align}
		Then
		\begin{align}
		\braket{v'}{w'} = \braket{\mathcal{U}v}{\mathcal{U}w} = \bra{v}\mathcal{U}^\dagger\mathcal{U}\ket{w} = \braket{v}{w}.
		\end{align}
	\end{proof}
\end{thm} 



\begin{thm}
	The columns (or rows) of a unitary matrix form an ONB.
	\begin{proof}
		Refer to a linear algebra text. The key is to consider an inner product between any two columns/rows.
	\end{proof}
\end{thm}


\subsection{Active and Passive Transformation}

Suppose all $\ket{v}$ is unitarily transformed to $\ket{v'}$:
\begin{align}
\ket{v} \to \U \ket{v}.
\end{align}
Then under this transformation, the matrix elements of any operator $\Omega$ are modified as follows:
\begin{align}
\ket{v'}\Omega\ket{v} \to \ket{\U v'} \Omega \ket{\U v} = \bra{v'}\U^\dagger \Omega \U \ket{v}.
\end{align}
It is clear that the same change is equivalent to leaving the vectors alone and subjecting all operators to the change
\begin{align}
\Omega \to \U^\dagger \Omega \U.
\end{align}

\textit{Active transformation} refers to changing the vectors, while \textit{passive transformation} refers to changing the operators.





\subsection{The Eigenvalue Problem}
I won't say much about what eigenvectors and eigenvalues are because we should be familiar with these concepts at this point. But just to introduce some terminology, each operator has certain kets of its own called \textit{eigenkets}, on which its action is simply that of scaling. So, eigenkets are just a different word for eigenvectors of an operator:
\begin{align}
\Omega\ket{v} = \omega\ket{v}.
\end{align}

Shankar's book talks about the characteristic equation and characteristic polynomial. While these are legitimate ways to find eigenvalues and eigenvectors, it is often very difficult. I'd prefer Leo Livshits' and Sheldon Axler's way and use minimal polynomials instead. I would steer away from determinants and characteristic polynomials at this point. 


\begin{thm}
	Eigenvalues of a Hermitian operator are real. 
	\begin{proof}
		Suppose 
		\begin{align}
		\Omega\ket{w} = a\ket{w},
		\end{align}
		then
		\begin{align}
		\bra{w}\Omega\ket{w} = a\braket{w},
		\end{align}
		and thus
		\begin{align}
		a^* \braket{w} = \bra{w}\Omega^\dagger\ket{w} = \bra{w}\Omega\ket{w} = a\braket{w}.
		\end{align}
		So we have
		\begin{align}
		(a - a^*)\braket{w} = 0.
		\end{align}
		Because $\ket{w}$ are eigenkets, they are cannot be the zero vector. This means $a = a^*$.
	\end{proof}
\end{thm}


Some might worry about the existence of eigenvalues of Hermitian operators. But worry no more, because Hermitian operators are a subclass of normal operators, which are a subclass of diagonalizable operators. This simply says Hermitian matrices are diagonalizable, and all its eigenvalues are real. But it turns out there is a little bit more to this. 


\begin{thm}
	For every Hermitian operator $\Omega$, there exists an ONB comprised entirely of the eigenvectors of $\Omega$. 
\end{thm}

Once again, this should be no surprise if one has studied normal operators. Hermitian operators inherit this property from its normalness. This property of normal operators are called the Spectral Theorem (for normal operators, of course). The proof of all this can be found in many linear algebra texts. 


\begin{thm}
	The eigenvalues of a unitary operator are complex numbers of unit modulus. 
	\begin{proof}
		The key to the proof is using inner products. 
	\end{proof}
\end{thm}



\subsubsection{Simultaneous Diagonalization of Two Hermitian Operators}

I would say the topic of simultaneous diagonalizability is covered quite well in Leo Livshits' course and hence in my \href{https://huanqbui.com/LaTeX\%20projects/Matrix_Analysis/HuanBui_MatrixAnalysis.pdf}{Matrix Analysis} notes. But here I will just give the most important results. 

\begin{thm}
	If $\Omega$ and $\Lambda$ are two commuting Hermitian operators, there exists a basis of common eigenvectors that diagonalizes them both. 
\end{thm} 

This result is not too surprising if we have studied simultaneous diagonalizability before. A more general theorem says that 
\begin{align}
\text{Simultaneous diagonalizbility} \iff \text{Individual diagonalizability} + \text{Commutativity}.
\end{align}
It is clear that because all Hermitian operators are diagonalizable, if two Hermitian operators commute, they are simultaneously diagonalizable.



\subsubsection{The Propagator}

In quantum mechanics (and classical mechanics of course), it is quite common to have some final state vector be obtained from an initial state vector multiplied by some matrix, which is independent of the initial state. We call this matrix the \textit{propagator}. 


The central problem in quantum mechanics is finding the state of a quantum system $\ket{\psi}$, which obeys the Schr\"{o}dinger equation:
\begin{align}
i\hbar \ket{\dot{\psi}} = J\ket{\psi}
\end{align}
where the Hermitian operator $H$ is called the \textit{Hamiltonian}. We will see much more of this as we move on.



\subsection{Functions of Operators and Related Concepts}

























\newpage


\section{Review of Classical Mechanics}


\newpage




\section{All is Not Well with Classical Mechanics}



\newpage



\section{The Postulates \textendash a General Discussion}


\newpage



\section{Simple Problems in One Dimension}


\newpage



\section{The Classical Limit}



\newpage


\section{The Harmonic Oscillator}



\newpage



\section{The Path Integral Formulation of Quantum Theory}



\newpage



\section{The Heisenberg Uncertainty Relation}



\newpage



\section{Systems with $N$ Degrees of Freedom}



\newpage



\section{Symmetries and Their Consequences}


\newpage



\section{Rotational Invariance and Angular Momentum}




\newpage




\section{The Hydrogen Atom}


\newpage



\section{Spin}



\newpage



\section{Additional of Angular Momentum}


\newpage



\section{Variational and WKB Methods}



\newpage


\section{Time-Independent Perturbation Theory}


\newpage



\section{Time-Dependent Perturbation Theory}


\newpage


\section{Scattering Theory}


\newpage


\section{The Dirac Equation}


\newpage



\section{Path Integrals\textendash II}


\newpage






















\section{Selected Problems and Solutions}


\subsection{Mathematical Introduction}



\newpage


\subsection{Review of Classical Mechanics}


\newpage




\subsection{All is Not Well with Classical Mechanics}



\newpage



\subsection{The Postulates \textendash a General Discussion}


\newpage



\subsection{Simple Problems in One Dimension}


\newpage



\subsection{The Classical Limit}



\newpage


\subsection{The Harmonic Oscillator}



\newpage



\subsection{The Path Integral Formulation of Quantum Theory}



\newpage



\subsection{The Heisenberg Uncertainty Relation}



\newpage



\subsection{Systems with $N$ Degrees of Freedom}



\newpage



\subsection{Symmetries and Their Consequences}


\newpage



\subsection{Rotational Invariance and Angular Momentum}




\newpage




\subsection{The Hydrogen Atom}


\newpage



\subsection{Spin}



\newpage



\subsection{Additional of Angular Momentum}


\newpage



\subsection{Variational and WKB Methods}



\newpage


\subsection{Time-Independent Perturbation Theory}


\newpage



\subsection{Time-Dependent Perturbation Theory}


\newpage


\subsection{Scattering Theory}


\newpage


\subsection{The Dirac Equation}


\newpage



\subsection{Path Integrals\textendash II}












\end{document}
