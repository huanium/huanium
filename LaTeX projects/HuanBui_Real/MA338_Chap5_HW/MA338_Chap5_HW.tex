\documentclass[11pt]{article}
\usepackage{amsmath}
\usepackage{physics}
\usepackage{amssymb}
\usepackage{graphicx}
\usepackage{hyperref}
\usepackage{amsfonts}
\usepackage{cancel}
\usepackage{xcolor}
\hypersetup{
	colorlinks,
	linkcolor={black!50!black},
	citecolor={blue!50!black},
	urlcolor={blue!80!black}
}
\usepackage{newpxtext,newpxmath}
\usepackage[left=1.25in,right=1.25in,top=0.9in,bottom=0.9in]{geometry}


%\newcommand{\fig}[1]{figure #1}
%\newcommand{\explain}{appendix?}
%\newcommand{\rat}{\mathbb{Q}}
%
%\newcommand{\mathbb{R}}{\mathbb{R}}
%\newcommand{\nat}{\mathbb{N}}
%\newcommand{\inte}{\mathbb{Z}}
%\newcommand{\M}{{\cal{M}}}
%\newcommand{\sss}{{\cal{S}}}
%\newcommand{\rrr}{{\cal{R}}}
%\newcommand{\uu}{2pt}
%\newcommand{\vv}{\vec{v}}
%\newcommand{\comp}{\mathbb{C}}
%\newcommand{\field}{\mathbb{F}}
%\newcommand{\f}[1]{ \hspace{.1in} (#1) }
%\newcommand{\set}[2]{\mbox{$\left\{ \left. #1 \hspace{3pt}
%\right| #2 \hspace{3pt} \right\}$}}
%\newcommand{\integral}[2]{\int_{#1}^{#2}}
%\newcommand{\ba}{\hookrightarrow}
%\newcommand{\ep}{\varepsilon}
%\newcommand{\limit}{\operatornamewithlimits{limit}}
%\newcommand{\ddd}{.1in}
%\newcommand{\ccc}{2in}
%\newcommand{\aaa}{1.5in}
%\newcommand{\B}{{\cal B}}
%\newcommand{\C}{{\cal C}}
%\newcommand{\D}{{\cal D}}
%\newcommand{\FF}{{\cal F}}


%\usepackage{epstopdf}
%\DeclareGraphicsRule{.tif}{png}{.png}{`convert #1 `basename #1 .tif`.png}
%\usepackage{graphics}
%\usepackage{array}
%\def\set#1#2{\left\{\left.\;#1\;\right| #2 \; \right\}}
%\def\Sum{\sum}
%\def\me{.05in}















\begin{document}
\begin{center}
{\Large\bf  Differentiation:  5.1, 2, 3, 6, 9, 12, 22, Baby Rudin}\\
$\,$\\
{\Large  Huan Q. Bui}
\end{center}




\noindent \textbf{5.1}
\noindent \textit{Proof:}  Let $f$ be defined for all reals such that $\abs{f(x) - f(y)} \leq (x-y)^2$ for all $x,y \in \mathbb{R}$. Let $\delta > 0$ be given. Pick $0 < x - y < \delta$,  which gives $0 \leq \abs{f(x) - f(y)}/(x-y) \leq x-y < \delta$.  This holds for any $\delta > 0$, so $f'(x) = 0$ for all $x \in \mathbb{R}$. This means $f$ is constant. \hfill $\square$\\


\noindent \textbf{5.2}
\noindent \textit{Proof:} Suppose $f'(x) > 0$ on $(a,b)$. Pick $x,y \in (a,b)$ such that $y > x$. Then by the MVT $(f(y) - f(x)) / (y-x) = f'(c) > 0$ for some $c \in (x,y)$. Since $y>x$, this holds if $f(y) > f(x)$, so $f$ is a strictly increasing function.\\

\noindent Let $g$ be its inverse function, so $g(f(x)) = x$. 
\begin{align*}
\frac{g(y) - g(x)}{y-x} = \frac{g(f(u)) - g(f(v))}{f(u) - f(v)} = \frac{u-v}{f(u) - f(v)}
\end{align*} 
where $y = f(u), x = f(v), y\neq x$, and $f(u) \neq f(v)$ (because $f$ is strictly increasing hence is one-to-one). And so
\begin{align}
\lim_{y\to x} \frac{g(y) - g(x)}{y-x} = \lim_{u\to v}\frac{u-v}{f(u) - f(v)} = \frac{1}{f'(v)}.
\end{align}
The limit exists because $f'(v) > 0$, and so $g'(x = f(v)) = 1/f'(v)$.   \hfill $\square$\\ 





\noindent \textbf{5.3}
\noindent \textit{Proof:}  Suppose $g$ is a real function on $\mathbb{R}$, with $\abs{g'} \leq M$. Fix $\epsilon >0$ and define $f(x) = x + \epsilon g(x) $. We want to show $f$ is one-to-one if $\epsilon$ is small enough. Let $a< b$ be given. Then $f(b) - f(a) = (b-a) + \epsilon (g(b) - g(a))$. By MVT, $\exists c \in (a,b)$ such that $(b-a)g'(c) = g(b) - g(a)$. With this, $f(b) - f(a) = (b-a)(1 - \epsilon g'(c))$. Pick $\epsilon < 1/M$, then $\abs{\epsilon g'(c)} < 1$, which means $f(b) - f(a) \neq 0$. So, $f$ is one-to-one. \hfill $\square$\\


\noindent \textbf{5.6}
\noindent \textit{Proof:} Suppose $f$ is continuous for $x \geq 0$, $f'$ exists for $x > 0$, $f(0) = 0$, and $f'$ is monotonically increasing. We want to show $g(x)= f(x)/x, (x>0)$ is increasing. Well we know that $g(x)$ is differentiable for $x>0$. So it suffices to show $g'(x) > 0$ for all $x>0$. Well, $g'(x) = -f(x)/x^2+f'(x)/x$. Now it comes down to showing the function $h(x) = xf'(x) - f(x)$ is positive for all $x > 0$. Now, $h(0) = 0f'(x) - 0 = 0$. And so $h(x)$ is positive for all $x>0$ if $h'(x) > 0$ for all $x>0$. Well, $h'(x) = f'(x) + xf''(x) - f'(x) = xf''(x) > 0$ for all $x > 0$ because $f'$ is monotonically increasing. So we're done.  \hfill $\square$\\



\noindent \textbf{5.9}
\noindent \textit{Proof:} $f$ is a continuous real function on $\mathbb{R}$. $f'$ exists for all $x\neq 0$ and $f'(x) \to 3$ as $x\to 0$. It DOES follow that $f'(0) = 3$. Consider the function $h(x) = f(x) - f(0)$ and $g(x) = x-0$. Then both approach zero as $x\to 0$. This means by l'Hopital's rule:
\begin{align*}
\frac{h(x)}{g(x)} = \frac{f(x) - f(0)}{x} \to \frac{f'(x)}{1} = 3 \mbox{ as } x\to 0^{\pm}.
\end{align*}
Thus, 
\begin{align*}
\lim_{x\to 0} \frac{f(x) - f(0)}{x-0} = 3 = f'(0). 
\end{align*}
\hfill $\square$\\



\noindent \textbf{5.12}
\noindent \textit{Proof:}  Suppose 
\begin{align*}
f(x) = \abs{x}^3 = \begin{cases}
x^3,\quad x \geq 0\\
-x^3, \quad x <0.
\end{cases}
\end{align*} 
It is not difficult to see that 
\begin{align}
f'(x) = \begin{cases}
3x^2, \quad x > 0\\
-3x^2, \quad x < 0
\end{cases}.
\end{align}
When $x=0$, we look at the limits:
\begin{align*}
&\lim_{x \to 0+}\frac{f(x) - f(0)}{x-0} = \lim_{x\to 0+}\frac{\abs{x}^3}{x} = +0^2 = 0\\
&\lim_{x \to 0-}\frac{f(x) - f(0)}{x-0} = \lim_{x\to 0+}\frac{\abs{x}^3}{x} = -0^2 = 0.
\end{align*}
So, $f'(0) = 0$. Next, it is also not difficult to see that 
\begin{align*}
f''(x) = \begin{cases}
6x, \quad x > 0\\
-6x, \quad x < 0
\end{cases}.
\end{align*}
When $x=0$, we look at the limits:
\begin{align*}
&\lim_{x \to 0+}\frac{f'(x) - f'(0)}{x-0} = \lim_{x\to 0+}\frac{3x^2}{x} = 0\\
&\lim_{x \to 0-}\frac{f'(x) - f'(0)}{x-0} = \lim_{x\to 0+}\frac{-3x^2}{x} = 0.
\end{align*}
So, $f''(0) = 0$. However, $f'''(0)$ does not exist because
\begin{align*}
&\lim_{x \to 0+}\frac{f''(x) - f''(0)}{x-0} = \lim_{x\to 0+}\frac{6x}{x} = 6\\
&\lim_{x \to 0-}\frac{f''(x) - f''(0)}{x-0} = \lim_{x\to 0+}\frac{-6x}{x} = -6 \neq \lim_{x \to 0+}\frac{f''(x) - f''(0)}{x-0},
\end{align*}
i.e., the limit as $x\to 0$ of the difference quotient does not exist. \hfill $\square$\\



\noindent \textbf{5.22}
\noindent \textit{Proof:} Suppose $f$ is a real function $(-\infty, \infty)$. $x$ is a fixed point if $f(x) = x$. 
\begin{enumerate}
	\item $f$ is differentiable and $f'(t) \neq 1$ for every real $t$. We want to show $f$ has at most one fixed point.  Suppose $f$ has at least two fixed points $a$ and $b$, then 
	\begin{align*}
	\frac{f(b) - f(a)}{b-a} = 1.
	\end{align*}
	By MVT, there exists $c\in (a,b)$ such that $f'(c) = 1$, which is a contradiction. So, $f$ at most one fixed point. 
	
	\item We want to show $f(t) = t + (1+e^t)^{-1}$ has no fixed point, although $0 < f'(t)< 1$ for all real $t$, i.e., we want to show $t \neq t + (1+e^t)^{-1} \forall t \in \mathbb{R}$. Obviously, $1/(1+e^t) \neq 0$ for all $t\in \mathbb{R}$. Now, 
	\begin{align*}
	f'(t) = 1 - \frac{e^t}{(1+e^t)^2} = 1 - \frac{e^t}{1+ 2e^t + e^{2t}}.
	\end{align*}
	The quantity $\frac{e^t}{(1+e^t)^2} >0$, so $f'(t) < 1$. Also, $\frac{e^t}{(1+e^t)^2} < 1$ because $0 < e^{t/2} < < e^t < 1+e^t$ (as $e^t$ is an increasing function). So, $0 < f'(t) < 1$. So, even though $0 < f'(t) < 1$, $f$ does not have any fixed point. 
	
	
	
	\item Suppose $\abs{f'(t)} \leq A$ for all real $t$ for some constant $A<1$ . We want to show $f$ has a fixed point $x$ and that $x = \lim x_n$ where $x_1$ is an arbitrary real number and $x_{n+1} = f(x_n)$ for $n=1,2,3,\dots$. Well, if $x_{n} = x_{n+1}$ then $\{ x_n\}$ is identically the sequence $\{ x\}$ and $x_n$ is a fixed point of $f$ for any $n$. Otherwise, MVT says
	\begin{align*}
	f(x_{n+1}) - f(x_n) = f'(t)(x_{n+1} - x_n)
	\end{align*} 
	for some $t$ between $x_n$ and $x_{n+1}$. Since $\abs{f'(t)} \leq A < 1$ and $f(x_{n+1}) = x_{n+2}$, we have that
	\begin{align*}
	\abs{x_{n+2} - x_{n+1}} \leq A \abs{x_{n+1} - x_n}  \leq \dots \leq A^{n-1}\abs{x_2 - x_1}.
	\end{align*}
	With this, for any positive $n>m$, 
	\begin{align*}
	\abs{x_n - x_m} &\leq \abs{x_n - x_{n-1}} + \dots + \abs{x_{m+1} - x_m}\\
	&\leq \abs{x_2 - x_1} \left( A^{n-2} + A^{n-3} + \dots + A^{m} + A^{m-1} \right)\\
	&= \abs{x_2 - x_1}A^{m-1}\left( A^{n-m-1} +\dots + 1 \right)\\
	&= \abs{x_2 - x_1}A^{m-1}\frac{1- A^{n-m}}{1-A}\\
	&\leq \abs{x_2  - x_1} \frac{A^{m-1}}{1-A}.
	\end{align*}
	For $\epsilon > 0$, there exists $N = 2 + \log_A \epsilon(1-A)/\abs{x_2 - x_1}$ such that whenever $n> m > N $, we have $\abs{x_n - x_m} < \epsilon$. So, $\{x_n\} \to x$, and so because $f$ is continuous,
	\begin{align*}
	x = \lim_{n\to \infty} x_{n+1} = f\left( \lim_{n\to \infty} x_n\right) = f(x).
	\end{align*} 
	This means $x = f(x)$, or $x$ is a fixed point of $f$. Why does this choice of $N$ work? Well, if $n > m > N$ then 
	\begin{align*}
	\abs{x_n - x_m} &\leq \abs{x_2 - x_1}\frac{A^{m-1}}{1-A}\\
	&\leq \abs{x_2 - x_1}\frac{A^{N-1}}{1-A}\\
	&= \abs{x_2 - x_1}\frac{A\epsilon(1-A)}{A(1-A)\abs{x_2 - x_1}} \\
	&= \epsilon.
	\end{align*}
	
	
	\item We want to show that the process described in item 3. can be visualized by the zig-zag path
	\begin{align*}
	(x_1,x_2) \to (x_2,x_2) \to (x_2,x_3) \to (x_3,x_3) \to (x_3,x_4) \to \dots
	\end{align*}
	We start with the $x$-coordinate $x_1$, then $f(x_1) = x_2$, so we get the first point $(x_1,x_2)$. If $x_1$ is a fixed point of $f$ then $x_1 = x_2$, so we move to $(x_2,x_2)$. If $(x_2,x_2) \neq (x_1,x_2)$, then we keep going by repeating: look at point $(x_2, f(x_2) = x_3)$, and so on.
	
	
	
	
	
\end{enumerate}



\hfill $\square$

  
\end{document}




