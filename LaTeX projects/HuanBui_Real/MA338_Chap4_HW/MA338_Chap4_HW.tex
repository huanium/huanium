\documentclass[11pt]{article}
\usepackage{amsmath}
\usepackage{physics}
\usepackage{amssymb}
\usepackage{graphicx}
\usepackage{hyperref}
\usepackage{amsfonts}
\usepackage{cancel}
\usepackage{xcolor}
\hypersetup{
	colorlinks,
	linkcolor={black!50!black},
	citecolor={blue!50!black},
	urlcolor={blue!80!black}
}
\usepackage{newpxtext,newpxmath}
\usepackage[left=1.25in,right=1.25in,top=0.9in,bottom=0.9in]{geometry}


%\newcommand{\fig}[1]{figure #1}
%\newcommand{\explain}{appendix?}
%\newcommand{\rat}{\mathbb{Q}}
%
%\newcommand{\mathbb{R}}{\mathbb{R}}
%\newcommand{\nat}{\mathbb{N}}
%\newcommand{\inte}{\mathbb{Z}}
%\newcommand{\M}{{\cal{M}}}
%\newcommand{\sss}{{\cal{S}}}
%\newcommand{\rrr}{{\cal{R}}}
%\newcommand{\uu}{2pt}
%\newcommand{\vv}{\vec{v}}
%\newcommand{\comp}{\mathbb{C}}
%\newcommand{\field}{\mathbb{F}}
%\newcommand{\f}[1]{ \hspace{.1in} (#1) }
%\newcommand{\set}[2]{\mbox{$\left\{ \left. #1 \hspace{3pt}
%\right| #2 \hspace{3pt} \right\}$}}
%\newcommand{\integral}[2]{\int_{#1}^{#2}}
%\newcommand{\ba}{\hookrightarrow}
%\newcommand{\ep}{\varepsilon}
%\newcommand{\limit}{\operatornamewithlimits{limit}}
%\newcommand{\ddd}{.1in}
%\newcommand{\ccc}{2in}
%\newcommand{\aaa}{1.5in}
%\newcommand{\B}{{\cal B}}
%\newcommand{\C}{{\cal C}}
%\newcommand{\D}{{\cal D}}
%\newcommand{\FF}{{\cal F}}


%\usepackage{epstopdf}
%\DeclareGraphicsRule{.tif}{png}{.png}{`convert #1 `basename #1 .tif`.png}
%\usepackage{graphics}
%\usepackage{array}
%\def\set#1#2{\left\{\left.\;#1\;\right| #2 \; \right\}}
%\def\Sum{\sum}
%\def\me{.05in}















\begin{document}
\begin{center}
{\Large\bf  Continuity:  Exercises 4.1 - 4.10, Baby Rudin}\\
$\,$\\
{\Large  Huan Q. Bui}
\end{center}


\noindent \textbf{4.1} 
\noindent \textit{Proof.} Let $f$ a real function on $\mathbb{R}$ which satisfies 
\begin{align*}
\lim_{h\to 0}[f(x+h) - f(x-h)] = 0
\end{align*}
for every $x \in \mathbb{R}$ be given. {To prove}: $f$ is not continuous. Consider this counterexample:
\begin{align*}
f(x) = \begin{cases}
0, \quad x \neq 0\\
1, \quad x = 0 
\end{cases}
\end{align*}
Clearly, $f$ satisfies the conditions above, but $f$ is not continuous at $0$.\hfill $\square$\\



\noindent \textbf{4.2} 
\noindent \textit{Proof.} Let $f$ a continuous mapping of a metric space $X$ into a metric space $Y$ be given. {To prove}: $f(\bar{E}) = \overline{f(E)}$ for every set $E \subset X$. Let subset $E \subset X$ be given. If $f(\bar{E}) = \emptyset$ then there's nothing to prove. If $f(\bar{E}) \neq \emptyset$, then pick $y \in f(\bar{E})$ and so there is some $x\in \bar{E}$ such that $y = f(x)$. Now, $x \in \bar{E} = E \cup E'$, so $x \in E'$ or $x \in E$. If $x\in E$ then $y = f(x) \in f(E) \subset \overline{f(E)} $. If $x \in E'$, then $x$ is a limit point of $E$. We now want to show $f(x)$ is a limit point of $f(E)$. Let $\epsilon > 0$ be given, then because $f$ is continuous, $\exists \delta > 0$ such that $d(f(x_0), f(x)) < \epsilon$ whenever $d(x_0 ,x) < \delta$, for all $x_0 \in X$. $x$ is a limit point of $E$, so for some $\delta > \delta' > 0$, there is $x_1 \in E$. This means $f(x_1) \in \mathcal{N}_\epsilon (x)$ for some $f(x_1) \in f(E)$. This means $f(p)$ is a limit point of $f(E)$. So, $f(p) \in \overline{f(E)}$. Therefore, $f(\bar{E}) \subset \overline{f(E)}$. \\

An example in which $ f(\bar{E}) \subsetneq \overline{f(E)} $. Let $E = \mathbb{N} \subsetneq \mathbb{R}$. We know that $\mathbb{N} = \overline{\mathbb{N}}$. Now, define $f: \mathbb{R} \to \mathbb{R}$ by $f(n) = \frac{1}{n}$. Obviously, $f(\mathbb{N}) = f(\bar{\mathbb{N}}) = \{ 1/n : n \in \mathbb{N}\setminus \{ 0 \} \}$ since the input sets are the same. Now, $\overline{f(\mathbb{N})} = \overline{ \{ 1/n : n\in \mathbb{N} \} } \cup \{ 0 \}$.  So, $f(\bar{\mathbb{N}}) \subsetneq \overline{f(\mathbb{N})}$.  \hfill $\square$\\


\noindent \textbf{4.3} 
\noindent \textit{Proof.} Let $f$ a continuous real function on a metric space $X$ be given. Consider the zero set $Z(f)$ of $f$. We want to show $Z(f)$ is closed. We notice that $Z(f) \equiv f^{-1}(\{ 0 \})$, where the set $\{ 0\}$ is closed. Theorem 4.8 says $f: X\to Y$ is continuous iff $f^{-1}(C)$ is closed in $X$ for every closed set $C$ in $Y$. In this problem, take $C = \{ 0 \} \subset X$. $C$ is closed, so $f^{-1}(C) = f^{-1}(\{ 0\}) = Z(f)$ is closed. \hfill $\square$ \\




\noindent \textbf{4.4}  
\noindent \textit{Proof.} Let $f,g : X \xrightarrow{\text{cont.}} Y$ and $E \stackrel{\text{dense}}{\subset} X$ be given. To prove: $f(E)$ dense in $f(X)$. Since $E$ dense in $X$, $\bar{E} = X$. Pick $y \in f(X)$. To show $f(E)$ dense in $f(X)$, we want to show that if $y \neq f(E)$, $y \in f(E)'$. Assume $y \in f(X) \setminus f(E)$, then there is an $x$ such that $y = f(x)$. If $x \in E$ then $y = f(x)\in f(E)$. This cannot happen, so $x \in X\setminus E$. $x \notin E$, which is dense in $X$, so there is a sequence $\{ x_n\} \subset E$ such that $x_n \to x \in X\setminus E$. Since $f$ is continuous, $f(x_n) \to f(x)$. If $f(x_n) = f(x) \in f(E)$ for some $n$, then we get a contradiction. So $f(x_n) \neq f(x)$ for all $n$. This means $y = f(x)$ is a limit point of $f(E)$, i.e., $y \in f(E)'$. So $f(E)$ is dense in $f(X)$.\\

Now, to prove: if $g(p) = f(p)$ for all $p \in E$, then $g(p) = f(p)$ for all $p \in X$. Well, if $p \in E$ then obviously, $g(p) = f(p)$. Consider $p \in E^c$. Since $E$ dense in $X$, there is a sequence $\{ p_n\}$ in $E$ such that $p_n \to p \in E^c$. Now, $f(p_n) = g(p_n)$ for all $n$ by hypothesis, so $f(p) = f(\lim_{n\to \infty} p_n) = \lim_{n\to \infty}f(p_n) =\lim_{n\to \infty}g(p_n) = g(\lim_{n\to \infty} p_n) = g(p)$. This means $f(p) = g(p)$ for all $p \in X$. \hfill $\square$\\ 







\noindent \textbf{4.5}  
\noindent \textit{Proof.}  Let $f$ be a real continuous function defined on a closed set $E \subset \mathbb{R}$. We want to construct a real function $g$ on $\mathbb{R}$ such that $f(x) = g(x)$ for all $x \in E$. Before we do this, we use a fact from Exercise 29, Chap 2 which says that because $E \subset X$ is closed,
\begin{align*}
E^c = \bigcup_{i=1}(a_i,b_i)
\end{align*}
where the union is at most countable and $a_i < b_i < a_{i+1} < b_{i+1}$ for any $i \in \mathbb{N}$. With this, $g(x)$ can be given by:
\begin{align*}
g(x) = \begin{cases}
f(x), \quad x \in E\\
f(a_i) + (x - a_i)\frac{f(b_i) - f(a_i)}{b_i - a_i}, \quad x \in E^c
\end{cases}
\end{align*}
Obviously $g(x)$ is a continuous function because $f(x)$ is continuous on $E$, and $g$ is just a linear function (hence continuous) on $E^c$ open. \\

When the word ``closed'' is omitted, we run into trouble. Consider $f(x) = 1/x$ on the open set $E = \mathbb{R} \setminus \{ 0\}$. Then there is no way for us to assign a real value to $g(0)$ and require that $g$ be continuous. \\

For vector-valued functions, the result is the following: for $f(x) = (f_1(x),\dots,f_d(x))$, where each $f_i(x)$ is a real continuous function on a closed set $E \subset \mathbb{R}$, we can extend each $f_i(x)$ by $g_i$ given by a similar definition above, to get an extension $g$ for $f$ given by $g(x) = (g_1(x),\dots,g_d(x))$. $g$ is continuous on $\mathbb{R}^d$ because each $g_i$ is continuous on $\mathbb{R}$.  \hfill $\square$\\


\noindent \textbf{4.6}   
\noindent \textit{Proof.} Let $f$ defined on $E$ be given. Assume $E \subset \mathbb{R}$ is compact. We want to show $f$ is continuous on $E$ iff its graph, $\mathcal{G} = \{ (x,f(x)): x\in E\}$ is compact.  \\

Before doing anything, we have to define the metric for the space $E\times f(E)$ in which the graph lives. For $x_1,x_2 \in E$ and $f(x_1), f(x_2) \in f(E)$, define
\begin{align*}
d((x_1,f(x_1),(x_2, f(x_2)) = \sqrt{d^2(x_1,x_2) + d^2(f(x_1), f(x_2))}.
\end{align*}
Okay with this we can start with the proof.\\

\noindent $(\rightarrow)$ Suppose $E$ is compact and $f$ is continuous. To show $\mathcal{G}$ is compact, we define a map $\mathcal{F} : E \to \mathcal{G}$ given by $\mathcal{F}(x) = (x,f(x))$. Since $E$ is compact, Theorem 4.14 tells us that if $\mathcal{F}$ is continuous on $E$ then $\mathcal{F}(E) = \mathcal{G}$ is compact. Well, let $\epsilon > 0$ be given. Pick a point $x_0 \in E$. Since $f$ is continuous, there is a $\delta > 0$ such that $d(f(x),f(x_0)) < \epsilon/\sqrt{2}$ whenever $d(x,x_0) < \delta$. Choose $\delta < \epsilon/\sqrt{2}$, then 
\begin{align}
d(\mathcal{F}(x), \mathcal{F}(x_0)) = \sqrt{d^2(x,x_0) + d^2(f(x),f(x_0))} < \sqrt{2\epsilon^2/2} = \epsilon.
\end{align}
So, $\mathcal{F}$ is continuous on $E$, and we're done. \\




\noindent $(\leftarrow)$  Suppose $\mathcal{G}$ and $E$ are compact. We want to show $f$ is continuous. Consider the function $\mathcal{F}$ given by $\mathcal{F}(x) = (x,f(x))$ like that defined above. To show $f$ is continuous, we can show $\mathcal{F}(x)$ is continuous, assuming that $\mathcal{G}, E$ are compact (since if $\mathcal{F}$ is continuous then its second component $f$ must also be continuous). The function $\bar{g}(x,f(x)) = x$ is 1-1 and continuous. It's inverse mapping is just $\mathcal{F}(x)$. By theorem 4.17, $\mathcal{F}$ is a continuous mapping from $E$ to $(E,f(E))$. It follows that $f$ is also continuous. \hfill $\square$\\








\noindent \textbf{4.7}   
\noindent \textit{Proof.} $f,g$ on $\mathbb{R}^2$ are given by $f(0,0) = g(0,0) = 0$, and if $(x,y)\neq 0$, $f(x,y) = xy^2/(x^2 + y^4)$, and $g(x,y) = xy^2/(x^2 + y^6)$. We want to show that $f$ is bounded on $\mathbb{R}^2$. By completing the square, we know that $x^2 + y^4 \geq 2xy^2 $, so $f(x,y) \leq 2$ for all $(x,y) \in \mathbb{R}^2$. So $f$ is bounded.\\


Next, to show $g$ is unbounded in every neighborhood of $(0,0)$, we look at sequences that converge to $(0,0)$. One such sequence is $\{ (x_n,y_n) =  \}  = \{ (1/n^3,1/n) \}$. Clearly, $g(x_n,y_n) = n^6/2n^5 = n/2 \to \infty$ as $n\to \infty$. So $g$ is unbounded in every neighborhood of $(0,0)$. \\

To show $f$ is not continuous at $(0,0)$ we look at where $\{ f(x_n,y_n)\}$ converges to when $(x_n,y_n) \to (0,0)$. Take the sequence $\{ (x_n,y_n) = (1/n^2,1/n) \}$. Then $f(x_n,y_n) = 1/2$ for all $n$. Obviously, $f(x_n,y_n) \to 1/2 \neq 0$ so $f$ is not continuous at $(0,0)$. \\

Now we want to show the restrictions of $f,g$ to any straight line in $\mathbb{R}^2$ are continuous. There are two cases: $x=c$ (the ``vertical'' line) and $y = ax+b$.  If $x=c$ constant, then if $c\neq 0$, then $f(x,y) = cy^2 / (x^2 + c^4)$ and $g(x,y) = cy^2 / (c^2 + y^6)$ are both continuous in $y$ and hence are continuous. If $c = 0$ then $f = g = 0$, also continuous. 

Consider straight lines: $y = ax + b$. If $b = 0$, then if for nonzero $(x,y)$, $f(x,y) = a^2x / (1 + a^4x^2)$ and $g = a^2x / (1 + a^6 x^4)$. As $x\to 0$, it is clear that $f \to 0$ and $g\to 0$, so $f,g$ are also continuous. If $b\neq 0$ then we don't have to worry because these lines don't pass the origin (which is where things can be bad). \hfill $\square$\\





\noindent \textbf{4.8}   
\noindent \textit{Proof.} Let $f$ a real uniformly continuous function on the bounded set $E \subset \mathbb{R}$. We want to show $f$ is bounded on $E$. Suppose $E$ is bounded by $M > 0$. Let $\epsilon > 0$ be given, then there is a $\delta > 0$ such that $\abs{f(p) - f(q)} < \epsilon$ for all $p,q\in E$ for which $\abs{p-q} < \delta$.  Since $E$ is bounded, we can find a finite cover for $E$:
\begin{align*}
E \subset \bigcup_{1\leq i \leq n} (x_i - \delta, x_i + \delta).
\end{align*}
where $x_i \in E$. Now we look at all the $f(x_i)$. For every $x \in E$, $x \in (x_i - \delta, x_i + \delta)$ for some $i$. By uniform continuity, $\abs{f(x) - f(x_i)} < \epsilon$. In other words, $\abs{f(x)} < \epsilon + \abs{f(x_i)}$. This holds for for all $x\in E$, so $f$ is bounded above by $\sup_i \{ f(x_i) \} + \epsilon$ and below by $\inf_i \{  f(x_i) \} - \epsilon$. Since $f$ is also continuous, it also makes sense to use $\max/\min$ instead of $\sup/\inf$.\\  
 

To show that the conclusion is false if boundedness of $E$ is omitted, we look at a counterexample. Look at the function $f(x) = x$ with $x\in \mathbb{R}$. $f$ is as uniformly continuous as one would like, but $f$ is not bounded. \hfill $\square$\\ 



\noindent \textbf{4.9}   
\noindent \textit{Proof.} We want to show that the definition of uniform continuity can be rephrased as: for every $\epsilon > 0$ there is a $\delta > 0$ such that $\text{diam} f(E) < \epsilon$ for all $E \subset X$ with $\text{diam}E < \delta$. To do this, we recall the definition of uniform continuity: $f: X \to Y$ is said to be \textit{uniformly continuous} if for every $\epsilon > 0$ there exists a $\delta > 0$ such that $d_Y(f(p), f(q)) < \epsilon$ for all $p,q \in X$ for which $d_X(q,p) < \delta$.\\

\noindent $(\rightarrow)$ Let $\epsilon > 0$ be given. Let $q,p\in X$ for which $d_X(q,p) < \delta$ for some $\delta$. Take $E = \{p,q \}$, then $\text{diam} E = d_X(q,p) < \delta$. By the new definition $\text{diam}f(E) < \epsilon$. Further, by the definition of the diameter of a set, $\text{diam}E \geq d_X(q,p)$, so $\text{diam} f(E) \geq d_Y (f(p), f(q))$, because we're taking the sup over more terms. This implies $d_Y(f(p), f(q)) < \epsilon$ whenever $d_X(q,p) < \delta$. New definition implies old definition. \\

\noindent $(\leftarrow)$  Let $E \subset X$ be given with $\text{diam}E < \delta$. By definition, for any $p,q \in E$, $d_X(q,p) \leq \text{diam} E < \delta $. The old definition says that $d_Y(f(p), f(q)) < \epsilon/2$, for any $p,q \in E$, and so $\text{diam} f(E) = \sup_{p,q\in E} d_Y(f(q), f(p)) \leq \epsilon/2 < \epsilon$. This means for every $\epsilon > 0$ there is a $\delta > 0$ such that $\text{diam} f(E) < \epsilon$ for all $E\subset X$ for which $\text{diam} E < \delta$. So, the old definition implies the new definition. \hfill $\square$\\  








\noindent \textbf{4.10}   
\noindent \textit{Proof.} Here we want to prove Theorem 4.19 in a different fashion. Theorem 4.19 says: If $f$ is a continuous mapping of a compact metric space $X$ into a metric space $Y$ then $f$ is uniformly continuous. The alternative proof goes by contradiction. Assume (to get a contradiction) that $f$ is $\cancel{\text{uniformly}}$ continuous. Since $f$ is not uniformly continuous, for some $\epsilon > 0$ there are sequences $\{ q_n\}$ and $\{ p_n\}$ in $X$ such that $d_X(p_n , q_n) \to 0$ but $d_Y(f(p_n), f(q_n)) > \epsilon$.   \\

Consider the sequences $\{ q_n\}$ and $\{ p_n\}$ above. Theorem 2.37 says that if $E$ is an infinite subset of a compact set $K$ then $E$ has a limit point in $K$. This means that the sequences $\{ p_n \}$ and $\{ q_n \}$ converge to points $p$ and $q$ in $E$, respectively. Now, since $d_X(p_n,q_n) \to 0$, $p=q$ which means that $f(p_{n}) \to f(p) = f(q) \leftarrow f(q_n)$ (because $f$ is continuous).  Also,
\begin{align*}
d_Y(f(p_{n}) , f(q_{n})) &\leq d_Y(f(p_{n}), f(p)) + d_Y(f(q_{n}), f(p))\\
&=  d_Y(f(p_{n}), f(p)) + d_Y(f(q_{n}), f(q)) \to 0 + 0 = 0.
\end{align*}
However, this contradicts $d_Y(f(p_n), f(q_n)) > \epsilon$. So, $f$ must be uniformly continuous. \hfill $\square$
  
\end{document}




