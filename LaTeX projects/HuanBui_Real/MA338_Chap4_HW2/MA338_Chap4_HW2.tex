\documentclass[11pt]{article}
\usepackage{amsmath}
\usepackage{physics}
\usepackage{amssymb}
\usepackage{graphicx}
\usepackage{hyperref}
\usepackage{amsfonts}
\usepackage{cancel}
\usepackage{xcolor}
\hypersetup{
	colorlinks,
	linkcolor={black!50!black},
	citecolor={blue!50!black},
	urlcolor={blue!80!black}
}
\usepackage{newpxtext,newpxmath}
\usepackage[left=1.25in,right=1.25in,top=0.9in,bottom=0.9in]{geometry}


%\newcommand{\fig}[1]{figure #1}
%\newcommand{\explain}{appendix?}
%\newcommand{\rat}{\mathbb{Q}}
%
%\newcommand{\mathbb{R}}{\mathbb{R}}
%\newcommand{\nat}{\mathbb{N}}
%\newcommand{\inte}{\mathbb{Z}}
%\newcommand{\M}{{\cal{M}}}
%\newcommand{\sss}{{\cal{S}}}
%\newcommand{\rrr}{{\cal{R}}}
%\newcommand{\uu}{2pt}
%\newcommand{\vv}{\vec{v}}
%\newcommand{\comp}{\mathbb{C}}
%\newcommand{\field}{\mathbb{F}}
%\newcommand{\f}[1]{ \hspace{.1in} (#1) }
%\newcommand{\set}[2]{\mbox{$\left\{ \left. #1 \hspace{3pt}
%\right| #2 \hspace{3pt} \right\}$}}
%\newcommand{\integral}[2]{\int_{#1}^{#2}}
%\newcommand{\ba}{\hookrightarrow}
%\newcommand{\ep}{\varepsilon}
%\newcommand{\limit}{\operatornamewithlimits{limit}}
%\newcommand{\ddd}{.1in}
%\newcommand{\ccc}{2in}
%\newcommand{\aaa}{1.5in}
%\newcommand{\B}{{\cal B}}
%\newcommand{\C}{{\cal C}}
%\newcommand{\D}{{\cal D}}
%\newcommand{\FF}{{\cal F}}


%\usepackage{epstopdf}
%\DeclareGraphicsRule{.tif}{png}{.png}{`convert #1 `basename #1 .tif`.png}
%\usepackage{graphics}
%\usepackage{array}
%\def\set#1#2{\left\{\left.\;#1\;\right| #2 \; \right\}}
%\def\Sum{\sum}
%\def\me{.05in}















\begin{document}
\begin{center}
{\Large\bf  Continuity:  Exercises 4.11, 14, 17, 18, 20, 21, 22, 23, Baby Rudin}\\
$\,$\\
\today\\
$\,$\\
{\Large  Huan Q. Bui}
\end{center}


\noindent \textbf{4.11}
\noindent \textit{Proof.}  Let $f: X \to Y$ be a uniformly continuous map. Let a Cauchy sequence $\{ x_n\} \subset X$ be given. \underline{To prove}: $\{ f(x_n)\}$ is Cauchy in $Y$. Let $\epsilon$ be given. We want to show that for sufficiently large $m,n$, $d_Y (f(x_n) - f(x_m)) < \epsilon$. Now, by uniform continuity of $f$, this holds whenever $d_X(x_n, x_m) < \delta$ for some $\delta > 0$. By the Cauchy-ness of $\{x_n\}$, this holds for any $\delta > 0$, provided sufficiently large $m,n$ (which we assumed). So the claim is proven.\\


We want to use this to prove the following statement in Exercise 13: for $E$ a dense subset of $X$ and $f$ a uniformly continuous \textit{real} function defined on $E$, that $f$ has a continuous extension from $E$ to $X$. To do this, let $E \subset X$ be given. $E$ is dense in $X$. $f: E \to \mathbb{R}$ is a uniformly continuous function. $E$ is dense in $X$ so for every $x \in X\setminus E$, there is a sequence $\{ x_n\} \subset E$ such that $x_n \to x$. From the proof above we know $\{ f(x_n)\}$ is Cauchy in $f(E) \subset \mathbb{R}$ and so $ f(x_n) \to \mathfrak{f} \in \mathbb{R}$. \\



We define the continuous extension as follows:
\begin{align*}
g(x) = \begin{cases}
f(x), \quad x \in E\\
\lim_{n\to \infty} f(x_n), \quad x \in X\setminus E, \{ x_n\} \subset E \mbox{ s.t. }  x_n \to x.
\end{cases}
\end{align*}

We claim that this is well-defined. To check this, we want to make sure $f(x_n)$ and $f(y_n)$ converge to the same value, provided the sequences $\{x_n\}$ and $\{ y_n\}$ converge to the same value. For $\{ x_n\}, \{ y_n\} \subset E$ such that $x_n, y_n \to x \in X\setminus E$, we want to show $f(x_n), f(y_n) \to f(x)$. Let $\epsilon > 0$ be given, there exists $\delta >0$ for which $\abs{f(x) - f(y)} < \epsilon$ whenever $d_X(x,y) < \delta$. For sufficiently large $n$, $d_X(x_n ,y_n) \leq d_X(x_n, x) + d(x,x) + d(x,y_n) < \delta $, which implies $\abs{f(x_n) - f(y_n)} < \epsilon$. And so $f(x_n),f(y_n) \to f(x)$. \\

Finally we want to show $g(x)$ is continuous on $X$. To do this, we consider a sequence $\{ p_n\}$ in $X$ that converges to some $p$ in $X$. For every $p_n \in X$ there is some $q_n \in E$ such that $d_X(p_n q_n) < d_X(p_n,p)$ (because $E$ is dense in $X$) and $\abs{g(p_n) = g(q_n)} < 1/n$. It follows that $d_X(q_n,p) \leq d_X(q_n,p_n) + d_X(p,p_n) < 2d_X(p_n,p) \to 0$ which means $q_n \to p$ as well. Now, because $\{ q_n\} \subset E$ converges to $p \in X$, we have that $g(q_n) \to g(p)$. We want to show $g(p_n) \to f(p)$. Well, $\abs{g(p) - g(p_n)} \leq \abs{g(p)-g(q_n)} + \abs{g(q_n) - g(p_n)} < \abs{g(p)-g(q_n)} + 1/n$. This goes to zero as $n\to\infty$. So, $g(p_n) \to g(p)$ as desired. So, $g$ is continuous  in $X$.  \hfill $\square$\\




\noindent \textbf{4.14} 
\noindent \textit{Proof.} Let $f$ be a continuous mapping from $I$ into $I$ where $I = [0,1]$ is the closed unit interval. We want to show $f(x) = x$ for at least one $x\in I$. Consider the function $g(x) = f(x) -x$. $g$ is continuous because $f$ and $\text{id}$ are continuous functions. $x, f(x) \in [0,1]$, and so $g(0) = f(0) - 0 \geq 0$ and  $g(1) = f(1) - 1 \leq 0$. If $g(0) = 0$ or $g(1) = 1$ then we have $f(1)= 1$ or $f(0) = 0$. Else, since $g$ is continuous, $g(1) < 0 < g(0)$ implies that there is some $x \in [0,1]$ such that $g(x) = f(x) -x = 0$ (Theorem 4.23, aka IVT). \hfill $\square$\\

\noindent \textbf{4.17}
\noindent \textit{Proof.} Let $f$ be a real function defined on $(a,b)$. We want to show that the set of points at which $f$ has a simple discontinuity is at most countable. \\



The first type of of simple discontinuity is where $f(x-) < f(x+)$. Let $E$ be the set on which $f(x-) < f(x+)$. With each point $x\in E$, we associate a triple $(p,q,r)$ of rational numbers such that
\begin{enumerate}
	\item $f(x-) < p < f(x+)$
	\item $a < q < t < x \implies f(t) < p$
	\item $x < t < r < b \implies f(t) > p$
\end{enumerate}
The first item is always possible be done because $\mathbb{Q}$ is dense in $\mathbb{R}$. The second item is possible because when $f(x-)$ exists, let $\epsilon  = p - f(x-)> 0$ be given, there is a $\delta > 0$ such that whenever $x - t < \delta$, $f(t) - f(x-) < \epsilon = p - f(x-)$, which implies $f(t) < p$. Now, we can always find a rational $q \in (x - \delta,x)$ such that for all $q < t < x$, $f(t) < p$. The third item follows from a similar argument. \\

Next we want to show the association is unique. Suppose we can also assign the same $(p,q,r)$ to $y\neq x$:
\begin{enumerate}
	\item $f(y-) < p < f(y+)$
	\item $a < q < t < y \implies f(t) < p$
	\item $y < t < r < b \implies f(t) > p$
\end{enumerate}
We want to get to a contradiction. WLOG, assume $y < x$, then there is a number $y < s < x$, we have 
\begin{enumerate}
	\item From $y$: $y < s < r < b \implies f(t) > p$
	\item From $x$:  $a < q < s < x \implies f(t) < p$
\end{enumerate}
which is a contradiction, since they cannot hold simultaneously. Thus, this association is unique. And because $\mathbb{Q}^3$ is still countable, there are countable such unique associations, and thus there  must be at most countable such simple discontinuities. \\

The simple discontinuity of type $f(x-) > f(x+)$ can be dealt with in a similar manner. So, let's consider the third type where $f(x-) = f(x+) = y$. For this type, the number $p$ in the association is no longer necessary, so we consider the following association with just two rational numbers $(q,r)$ where:
\begin{enumerate}
	\item $a < q < t < x \implies \abs{f(t) - z} > \abs{f(x) - z}$
	\item $x < t < r < b \implies \abs{f(t) - z} > \abs{f(x) - z}$
\end{enumerate}
Let's show this association is unique. Suppose $x < y$, then if we can have the same association for both $x,y$ then we must have
\begin{enumerate}
	\item $a < q < y < x \implies \abs{f(x) - z} > \abs{f(x) - z}$
	\item $x < y < r < b \implies \abs{f(y) - z} > \abs{f(x) - z}$
\end{enumerate}
which is a contradiction. So, the association is unique and thus the simple discontinuities of this type is at most countable.  \hfill $\square$\\




\noindent \textbf{4.18}
\noindent \textit{Proof.} Let the function $f$ defined on $\mathbb{R}$ be given by
\begin{align*}
f(x) = \begin{cases}
0, \quad x \text{ irrational}\\
1/n, \quad x = m/n
\end{cases}
\end{align*}
where $x$ in the second case is rational, with $m,n$ are integers with no nontrivial common divisor and $n > 0$. When $x = 0$, we take $n=1$. We want to show that $f$ is continuous at every irrational point, and that $f$ has a simple discontinuity at every rational point. \\

Let $x_0 \in \mathbb{R}$ be given. We claim that $\lim_{x\to x_0}f(x) = 0$. Let $\epsilon > 0$ be given. Take $q_0 \in \mathbb{N}$ such that $1/q_0 < \epsilon$. Now, for any interval $(x_0-x', x_0+x')$ for any $\infty> x' >0$, there are finitely rationals $p/q$ with denominator $ q \in (0, q_0]$. And so we can always find a $\delta > 0$ such that any rational $p/q$ in the interval $(x_0 - \delta, x_0 + \delta)$ has denominator $q > q_0$. Consider this $\delta$, then if $x \in (x_0-\delta, x_0 + \delta)$ is irrational then of course $f(x) = 0$, else if $x$ is rational then $f(x) = f(p/q) = 1/q < 1/q_0$, which means $\abs{f(x) - 0} < 1/q_0 < \epsilon$ for any $x \in (x_0 - \delta, x_0 + \delta)$. So, $\lim_{x\to x_0}f(x) = 0$ for all $x_0 \in \mathbb{R}$. \\

With this, if $x_0$ is irrational then $\lim_{x\to x_0} f(x) = 0 = f(x_0)$, so $f$ is continuous there. If $x_0$ is rational, then $\lim_{x\to x_0}f(x) = 0$ but $f(x) \neq 0$, which means $f$ has a simple discontinuity there. \hfill $\square$\\





\noindent \textbf{4.20}
\noindent \textit{Proof.} If $E$ is a nonempty subset of a metric space $X$, define the distance from $x\in X$ to $E$ by $\rho_E(x) = \inf_{z\in E} d(x,z)$.


\begin{enumerate}
	\item $\rho_E(x) = 0 \iff x\in \bar{E}$. Suppose $x\in \bar{E}$, then $x \in E \cup E'$. If $x\in E$ then obviously $\rho_E(x) = d(x,x) = 0$. If $x$ is a limit point of $E$ then for every $\epsilon > 0$ there is some $q \in E$ such that $d(x,q) < \epsilon$. This means $\rho_E(x) = 0$ as well. Suppose $\rho_E (x) = 0$. If $x \notin \bar{E} = E \cup E'$ then there exists $\epsilon > 0$ such that $\mathcal{N}_\epsilon(x)$ does not contain any point in $E$, which means $d(x,z) \geq \epsilon$ for every $z \in E$. This is clearly a contradiction.  
	
	
	
	\item Prove that $\rho_E$ is a uniformly continuous function on $X$, by showing that $\abs{\rho_E(x) - \rho_E(y)} \leq d(x,y)$ for all $x, y \in X$. Let $x,y\in X$ be given. Let $z \in E$ be given, then $\rho_E(x) \leq d(x,y) + d(y,z) \leq $. This holds for all $z$, so $\rho_E(x) \leq d(x,y) + \rho_E(y)$. And so, $\abs{\rho_E(x) - \rho_E(y)} \leq d(x,y)$. Thus, $\rho_E$ is a uniformly continuous function on $X$ because for any $\epsilon >0$, there is a $\delta = \epsilon$ such that for any $x,y \in X$,  whenever $d(x,y) < \delta = \epsilon$, $\abs{\rho_E(x)  - \rho_E(y)} \leq d(x,y) < \delta = \epsilon$ .
	
	
	
	
	
	
	 
\end{enumerate}

\hfill $\square$\\

\noindent \textbf{4.21}
\noindent \textit{Proof.} Suppose $K$ and $F$ are disjoint sets in a metric space $X$ and $K$ is compact, $F$ closed. We want to show that there exists $\delta > 0$ such that $d(p,q) > \delta$ if $p \in K, q\in F$. Well, from problem 20, $\rho_F(x) = 0 \iff x \in F$ since $F$ is closed. Also, from problem 20, we have that $d(p,q) \leq \abs{\rho_F(p) - \rho_F(q)} = \abs{\rho_F(p)}$.  Now, $\rho_F$ is a (uniformly) continuous function on the compact set $K$, so by Theorem 4.16 there is a point $p_0$ such that $\rho_F(p_0) = \inf_{t \in K} \rho_F(t) $. And so we have $d(p,q) \geq \abs{\rho_F(p)} \geq \abs{\rho_F(p_0)}$. So, if we let $\delta = \abs{\rho_F(p_0)}/2$ then clearly, $d(x,y) > \delta$.  \\


Suppose the ``compactness'' is dropped. Consider $X = \mathbb{R}$, $K = \mathbb{N}$ and $F = \{ n+1/2^n : n\in \mathbb{N}\}$. Then obviously $K,F$ are closed and disjoint, but some large elements on both sets can get arbitrarily close to each other, i.e., $d(n,n+1/2^n) \to 0$ as $n\to \infty$. \hfill $\square$\\ 


\noindent \textbf{4.22}
\noindent \textit{Proof.}  Let disjoint nonempty closed sets $A,B$ be given and define 
\begin{align*}
f(p) = \frac{\rho_A(p)}{\rho_A(p) + \rho_B(p)}, \quad p \in X.
\end{align*}
Obviously, $0 \leq f(p) \leq 1$ for all $p$ since it is a ratio of a nonnegative number to a larger positive number (which we know is positive because $A\cap B = \emptyset$). $\rho_A(p) = 0 \iff x\in \bar{A} = A$ (problem 20), so $f(p) = 0 \iff p \in A$. The same argument goes for $p \in B$, except that $p \in B \iff f(p) = \rho_A(p)/\rho_A(p) = 1$. Note that because $A \cap B = \emptyset$, this ratio is defined. We now want to show $f$ is continuous on $X$. This is easy because it just follows from the fact that both $\rho_A$ and $\rho_B$ are continuous on $X$. \\

This establishes a converse of Exercise 3: Every closed set $A \subset X$ is $Z(f)$ for some continuous real $f$ on $X$. Setting $V = f^{-1}([0,1/2))$ and $W = f^{-1}((1/2,1])$. We want to show $V,W$ are open and disjoint.\\

$f$ is a continuous function $X \to [0,1]$. By Theorem 4.8, because $[0,1/2)$ and $(1/2,1]$ are open sets in $[0,1]$, $V,W$ must be open in $X$. Further, $f(A) = \{ 0\} \subset [0,1/2)$ and $f(B) = \{1\} \subset (1/2,1]$, so $A\subset V$ and $B \subset W$. \hfill $\square$ \\




\noindent \textbf{4.23}
\noindent \textit{Proof.} A real-valued function $f$ defined in $(a,b)$ is \textit{convex} if 
\begin{align*}
f(\lambda x + (1-\lambda)y) \leq \lambda f(x) + (1-\lambda) f(y)
\end{align*} 
whenever $x,y \in (a,b)$, $0 < \lambda < 1$. We first want to show that every convex function is continuous. Next, we want to show that every increasing convex function of a convex function is convex. Finally, if $f$ is convex in $(a,b)$ and if $a<s<t<u<b$, we want to show that
\begin{align*}
\frac{f(t) - f(s)}{t-s} \leq \frac{f(u) - f(s)}{u-s} \leq \frac{f(u) - f(t)}{u-t}
\end{align*}

We will prove the last item first. Let $s,t,u \in (a,b)$ such that $s<t<u$. Then we can write
\begin{align*}
t = \frac{t-s}{u-s}u + \frac{u-t}{u-s}s.
\end{align*}
Obviously $\frac{t-s}{u-s} + \frac{u-t}{u-s} = 1$ and both are greater than 0. $f$ is convex, so
\begin{align*}
f(t) = f\left( \frac{t-s}{u-s}u + \frac{u-t}{u-s}s \right) \leq \frac{t-s}{u-s}f(u) + \frac{u-t}{u-s}f(s) = \frac{t-s}{u-s}f(u) + \left[1-\frac{t-s}{u-s}\right]f(s) 
\end{align*}
After some \textbf{nontrivial} rearranging (too much \LaTeX{}-ing here so I'll skip --- sorry) we get
\begin{align*}
\frac{f(t) - f(s)}{t-s} \leq \frac{f(u) - f(s)}{u-s} \leq \frac{f(u) - f(t)}{u-t}.
\end{align*}
Now we prove that $f$ is continuous. Let $\epsilon > 0$ be given. %Let $x_0 \in (a,b)$ be given. Let $u > x_0$ be given, then for any $s < x_0$, we have
%\begin{align*}
%\frac{f(x_0) - f(s)}{x_0 - s} \leq \frac{f(u) - f(s)}{u-s} \implies f(s) \geq f(x_0) - \frac{f(u) - f(x_0)}{u-x_0}(x_0 - s).
%\end{align*}
For any $x>y \in [x_1,x_2]$, there are also $x_0,x_3$ such that $x_0 < x_1 < x_2 < x_3$. By the inequalities we have
\begin{align*}
\frac{f(x) - f(y)}{x - y}  \leq  \frac{f(x_3)-f(y)}{x_3-y} \leq \frac{f(x_2)-f(y)}{x_2-y}
\end{align*}
and
\begin{align*}
\frac{f(x_1) - f(x_0)}{x_1 - x_0}  \leq  \frac{f(x)-f(x_0)}{x-x_0} \leq \frac{f(x)-f(y)}{x-y}.
\end{align*}
And so,
\begin{align*}
\abs{f(x) - f(y)} \leq \abs{x-y}\max\left\{ \frac{\abs{f(x_3) - f(x_2)}}{\abs{x_3 - x_2}}, \frac{\abs{f(x_1) - f(x_0)}}{\abs{x_1 - x_0}} \right\} \equiv C\abs{x-y}.
\end{align*}
Let $\delta = \min\{ \epsilon/C , \frac{x_2 - x_1}{2}\}$, then we have 
\begin{align*}
\abs{f(x) - f(y)} \leq C\frac{\epsilon}{C} = \epsilon.
\end{align*}
So $f$ is continuous on $(a,b)$. \\

Finally we want to show that every increasing convex function of a convex function is convex. Let $h(x) = g(f(x))$ where $g$ is an increasing convex function and $h$ is a convex function. For $x,y \in (a,b)$ and $\lambda \in (0,1)$, we have that
\begin{align*}
h(\lambda x + (1-\lambda)y) &= g(f(\lambda x + (1-\lambda)y))\\
&\leq g(\lambda f(x) + (1-\lambda)f(y))\\
&\leq \lambda g(f(x)) + (1-\lambda)g(f(y))\\
&= \lambda h(x) + (1-\lambda)h(y).
\end{align*}
So we're done. \hfill $\square$


\end{document}




