\documentclass[11pt]{article}
\usepackage{amsmath}
\usepackage{physics}
\usepackage{amssymb}
\usepackage{graphicx}
\usepackage{hyperref}
\usepackage{amsfonts}
\usepackage{cancel}
\usepackage{xcolor}
\hypersetup{
	colorlinks,
	linkcolor={black!50!black},
	citecolor={blue!50!black},
	urlcolor={blue!80!black}
}
\usepackage{newpxtext,newpxmath}
\usepackage[left=1.25in,right=1.25in,top=0.9in,bottom=0.9in]{geometry}


%\newcommand{\fig}[1]{figure #1}
%\newcommand{\explain}{appendix?}
%\newcommand{\rat}{\mathbb{Q}}
%
%\newcommand{\mathbb{R}}{\mathbb{R}}
%\newcommand{\nat}{\mathbb{N}}
%\newcommand{\inte}{\mathbb{Z}}
%\newcommand{\M}{{\cal{M}}}
%\newcommand{\sss}{{\cal{S}}}
%\newcommand{\rrr}{{\cal{R}}}
%\newcommand{\uu}{2pt}
%\newcommand{\vv}{\vec{v}}
%\newcommand{\comp}{\mathbb{C}}
%\newcommand{\field}{\mathbb{F}}
%\newcommand{\f}[1]{ \hspace{.1in} (#1) }
%\newcommand{\set}[2]{\mbox{$\left\{ \left. #1 \hspace{3pt}
%\right| #2 \hspace{3pt} \right\}$}}
%\newcommand{\integral}[2]{\int_{#1}^{#2}}
%\newcommand{\ba}{\hookrightarrow}
%\newcommand{\ep}{\varepsilon}
%\newcommand{\limit}{\operatornamewithlimits{limit}}
%\newcommand{\ddd}{.1in}
%\newcommand{\ccc}{2in}
%\newcommand{\aaa}{1.5in}
%\newcommand{\B}{{\cal B}}
%\newcommand{\C}{{\cal C}}
%\newcommand{\D}{{\cal D}}
%\newcommand{\FF}{{\cal F}}


%\usepackage{epstopdf}
%\DeclareGraphicsRule{.tif}{png}{.png}{`convert #1 `basename #1 .tif`.png}
%\usepackage{graphics}
%\usepackage{array}
%\def\set#1#2{\left\{\left.\;#1\;\right| #2 \; \right\}}
%\def\Sum{\sum}
%\def\me{.05in}















\begin{document}
\begin{center}
{\Large\bf  Continuity:  Exercises 4.11, 14, 17, 18, 20, 21, 22, 23, Baby Rudin}\\
$\,$\\
\today\\
{\Large  Huan Q. Bui}
\end{center}


\noindent \textbf{4.11}
\noindent \textit{Proof.}  Let $f: X \to Y$ be a uniformly continuous map. Let a Cauchy sequence $\{ x_n\} \subset X$ be given. \underline{To prove}: $\{ f(x_n)\}$ is Cauchy in $Y$. Let $\epsilon$ be given. We want to show that for sufficiently large $m,n$, $d_Y (f(x_n) - f(x_m)) < \epsilon$. Now, by uniform continuity of $f$, this holds whenever $d_X(x_n, x_m) < \delta$ for some $\delta > 0$. By the Cauchy-ness of $\{x_n\}$, this holds for any $\delta > 0$, provided sufficiently large $m,n$ (which we assumed). So the claim is proven.\\


We want to use this to prove the following statement in Exercise 13: for $E$ a dense subset of $X$ and $f$ a uniformly continuous \textit{real} function defined on $E$, that $f$ has a continuous extension from $E$ to $X$  \hfill $\square$

\noindent \textbf{4.14} 
\noindent \textit{Proof.}  

\noindent \textbf{4.17}
\noindent \textit{Proof.} 

\noindent \textbf{4.18}
\noindent \textit{Proof.} 

\noindent \textbf{4.20}
\noindent \textit{Proof.} 

\noindent \textbf{4.21}
\noindent \textit{Proof.} 

\noindent \textbf{4.22}
\noindent \textit{Proof.} 

\noindent \textbf{4.23}
\noindent \textit{Proof.} 



\end{document}




