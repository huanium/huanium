\documentclass[11pt]{article}
\usepackage{amsmath}
\usepackage{physics}
\usepackage{amssymb}
\usepackage{graphicx}
\usepackage{mathrsfs}
\usepackage{hyperref}
\usepackage{amsfonts}
\usepackage{cancel}
\usepackage{xcolor}
\hypersetup{
	colorlinks,
	linkcolor={black!50!black},
	citecolor={blue!50!black},
	urlcolor={blue!80!black}
}
\usepackage{newpxtext,newpxmath}
\usepackage[left=1.25in,right=1.25in,top=0.9in,bottom=0.9in]{geometry}


%\newcommand{\fig}[1]{figure #1}
%\newcommand{\explain}{appendix?}
%\newcommand{\rat}{\mathbb{Q}}
%
%\newcommand{\mathbb{R}}{\mathbb{R}}
%\newcommand{\nat}{\mathbb{N}}
%\newcommand{\inte}{\mathbb{Z}}
%\newcommand{\M}{{\cal{M}}}
%\newcommand{\sss}{{\cal{S}}}
%\newcommand{\rrr}{{\cal{R}}}
%\newcommand{\uu}{2pt}
%\newcommand{\vv}{\vec{v}}
%\newcommand{\comp}{\mathbb{C}}
%\newcommand{\field}{\mathbb{F}}
%\newcommand{\f}[1]{ \hspace{.1in} (#1) }
%\newcommand{\set}[2]{\mbox{$\left\{ \left. #1 \hspace{3pt}
%\right| #2 \hspace{3pt} \right\}$}}
%\newcommand{\integral}[2]{\int_{#1}^{#2}}
%\newcommand{\ba}{\hookrightarrow}
%\newcommand{\ep}{\varepsilon}
%\newcommand{\limit}{\operatornamewithlimits{limit}}
%\newcommand{\ddd}{.1in}
%\newcommand{\ccc}{2in}
%\newcommand{\aaa}{1.5in}
%\newcommand{\B}{{\cal B}}
%\newcommand{\C}{{\cal C}}
%\newcommand{\D}{{\cal D}}
%\newcommand{\FF}{{\cal F}}


%\usepackage{epstopdf}
%\DeclareGraphicsRule{.tif}{png}{.png}{`convert #1 `basename #1 .tif`.png}
%\usepackage{graphics}
%\usepackage{array}
%\def\set#1#2{\left\{\left.\;#1\;\right| #2 \; \right\}}
%\def\Sum{\sum}
%\def\me{.05in}















\begin{document}
\begin{center}
{\Large\bf  Sequences and Series of Functions:  7.1, 2, 3, 4, 5, 6, 7, Baby Rudin}\\
$\,$\\
{\Large  Huan Q. Bui}
\end{center}


\noindent \textbf{7.1}
\noindent \textit{Proof.} We WTS every uniformly convergent sequence of bounded functions is uniformly bounded. Let $\{f_n\}$ be a uniformly convergent sequence of bounded functions. $\forall \epsilon > 0, \exists N \in \mathbb{N} $ such that whenever $n \geq N$, $\abs{f_n - f} < \epsilon$ for all $ x \in E$. Further, because for each $n$, $f_n$ is bounded, we set $\abs{f_n} \leq M_n$ for all $x\in E$. Now, $\abs{f_n} \leq \abs{f_n - f} + \abs{f - f_N} + \abs{f_N} < 2\epsilon+ M_N  $. Obviously if we put $M = \max\{ M_1,\dots,M_{N-1}, 2\epsilon + M_{N} \}$ then $\abs{f_n} \leq M$ for all $x\in E$. This means $\{ f_n\}$ is uniformly bounded. \hfill $\square$\\ 



\noindent \textbf{7.2} 
\noindent \textit{Proof.} Suppose we have uniformly convergent sequences of functions $\{ f_n\}$ and $\{ g_n\}$. Let $\epsilon > 0$ be given, then by assumption, there is an $N \in \mathbb{N}$ sufficiently large such that whenever $n \geq N$ we have $\abs{f_n + g_n - f - g} \leq \abs{f_n - f} + \abs{g_n - g} < \epsilon/2 + \epsilon/2 = \epsilon$ for any $x\in E$. So, $\{ f_n + g_n\} \to f+g$ uniformly on $E$. Next, if $f_n$ and $g_n$ are bounded for each $n$ then by the previous problem $\{ f_n\}$ and $\{ g_n\}$ are uniformly bounded. This means $\abs{f_n} \leq \mathcal{F}$ and $\abs{g_n} \leq \mathcal{G}$ for any $n$ and $x \in E$. Also, because $g_n \to g$ and $f_n \to f$ uniformly, for $n$ sufficiently large, $\abs{g_n - g} < \epsilon$ and $\abs{f_n - f} \leq  \epsilon $. With this, for $n$ sufficiently large
\begin{align*}
\abs{f_ng_n - fg} 
&\leq \abs{f_n(g_n - g) = g_n(f_n - f)} \nonumber\\
&= \abs{f_n(g_n - g)} + \abs{g(f_n - f)}\nonumber\\
&= \abs{f_n}\abs{g_n - g} + \abs{g}\abs{f_n - f}\nonumber\\
&= \epsilon(\mathcal{F} + \mathcal{G})
\end{align*}
for any $x\in E$. Of course since $\mathcal{F}$ and $\mathcal{G}$ are fixed constants and $\epsilon$ can get arbitrarily small, $\{f_ng_n\} \to fg$ uniformly. \hfill $\square$\\


\noindent \textbf{7.3}
\noindent \textit{Proof.} We want to construct $\{f_n\}$ and $\{g_n\}$ such that both converge uniformly on some set $E$ but $\{f_ng_n\}$ does not converge uniformly. The key is to have one of original sequences be unbounded. To this end, we look at the set $E = (0,1)$ and define
\begin{align*}
f_n(x) = 1/n, \quad g_n(x) = 1/x.
\end{align*} 
Obviously $f_n \to 0$ uniformly on $E$.  $g_n \to 1/x$ on $E$ simply because the $g_n$'s don't depend on the index $n$. $\{f_ng_n\}$ doesn't converge uniformly on $E$ because for any $\epsilon < 1$, we can pick $x = 1/n$ so that $f_n(1/n)g_n(1/n) = 1 > \epsilon$ for all $n \in \mathbb{N}$. Of course, $\{f_ng_n\}$ converges point-wise on $E$ because $f_n(x)g_n(x) = 1/nx \to 0$ as $n\to \infty$. \hfill $\square$\\



\noindent \textbf{7.4}
\noindent \textit{Proof.} Consider 
\begin{align*}
f(x) = \sum^\infty_{n=1}\frac{1}{1+n^2x}.
\end{align*}
\begin{enumerate}
	\item For what values of $x$ does the series converge absolutely? The series diverges when $x=-1/n^2$ for any $n\in \mathbb{N}$ and when $x=0$. With this, we consider some cases:
	\begin{itemize}
		\item If $x>0$, then the absolute value of the summands are just the summands themselves, so we look at 
		\begin{align*}
		\sum^\infty_{n=1}\abs{\dots} = \sum^\infty_{i=1}{\frac{1}{1+n^2x}} \leq \sum^\infty_{n=1}{\frac{1}{n^2x}} = \frac{\pi^2}{6x} < \infty,
		\end{align*} 
		for any given value of $x$. So the series converges absolutely in this case. 
		
		\item If $x<0$ and $x\neq -1/n^2$, then let $x$ be fixed. Then $1+n^2x$ decreases as a function of $n$. When $n$ is large enough, $\abs{1+n^2x} \geq n^2$, which means for, say, $n \geq N$, we have $\abs{1/(1+n^2x)} \leq 1/n^2$. Because the series $\sum^\infty_{n=1}1/n^2$ converges, comparison test says that $\sum^\infty_{n=N} \abs{1/(1+n^2x)}$ converges. So, the entire series $\sum^\infty_{n=1}1/(1+n^2x)$ converges absolutely.
	\end{itemize} 
	So, the series converges absolutely for all $x\in \mathbb{R}\setminus (\{0\} \cup \{-1/n^2: n\in \mathbb{N}\})$.
	
	
	\item On what intervals does it converge uniformly? On what intervals does it fail to converge uniformly? We know that $f$ converges absolutely whenever $x>0$, so $f$ converges absolutely on any interval $[a,b]$ with $a>0$. In this case, we also have uniform convergence because $\abs{1/(1+n^2x)} \leq (1/a)(1/n^2)$ and $\sum^\infty_{n=1}(1/a)(1/n^2)$ converges (uniform convergence comes from the fact that the series $\sum^\infty_{n=1}(1/a)(1/n^2)$ satisfies the Cauchy criterion).\\

	We also know that if $x=-1/n^2$ then $f$ does not converge, so any interval $[a,b]$ such that $a\leq -1/n^2 \leq 0 \leq b$ for some $n\in \mathbb{N}$ won't work. Consider $[a,b]$ with $b<0$ which does not contain $-1/n^2$ for any natural $n$. Then we once again have uniform convergence because we once again compare $f$ to the series $\sum^\infty_{n=1}1/n^2$. 
	
	
	\item Is $f$ continuous wherever the series converges? YES. Theorem 7.12 says $f$ is continuous on $E$ if $f_n \to f$ uniformly on $E$. $f$ happens to converge wherever it converges uniformly, so YES $f$ is continuous wherever it converges.  
	
	
	\item Is $f$ bounded? NO. We note that when $x=0$ the series diverges. So, pick $x(t) = 1/t$, so that $x(t) \to 0$ as $t\to \infty$. Then
	\begin{align*}
	f(x(t)) = \sum^\infty_{n=1}\frac{t}{t + n^2} \to \infty, \quad t\to \infty
	\end{align*}
	since the denominator gets larger as $n\to \infty$. 
\end{enumerate}\hfill $\square$



\noindent \textbf{7.5} 
\noindent \textit{Proof.} We WTS a sequence $\{f_n\}$ where
\begin{align*}
f_n(x) = \begin{cases}
0, \quad x<1/(n+1)\\
\sin^2\frac{\pi}{x}, \quad 1/(n+1) \leq x \leq 1/n \\
0, 1/n < x
\end{cases}
\end{align*}
converges to a continuous function, but not uniformly so. Well, we see that $\lim_{n\to \infty} f_n(x) = 0$ which is a continuous function. This is because as $n\to \infty$, the support of $f$ shrinks and eventually vanish. However, the convergence is not uniform. For $1 > \epsilon > 0$ it is possible to take $x = 1/(N+1/2)$ in the support of $f_n$, $f_n(x) = \sin^2 [(N+1/2)\pi] = 1 > \epsilon$. We observe that the support of $f_n$ and $f_m$ intersect trivially when $m\neq n$, so $\sum^\infty_{n=1} \abs{f_n(x)} = \abs{f_m(x)}$ for some (single) value of $m$. So $\sum f_n$ converges absolutely. But each $f_n$ does not converge uniformly, so $\sum f_n$ does not converge uniformly either. \hfill $\square$\\




\noindent \textbf{7.6}
\noindent \textit{Proof.} We WTS the series 
\begin{align*}
\sum^\infty_{n=1} (-1)^n \frac{x^2+n}{n^2}
\end{align*}
converges uniformly in every bounded interval, but does not converge absolutely for any value of $x$. Well, 
\begin{align*}
\sum^\infty_{n=1} (-1)^n \frac{x^2+n}{n^2} = x^2\sum^\infty_{n=1}(-1)^n\frac{1}{n^2} + \sum^\infty_{n=1}(-1)^n\frac{1}{n}.
\end{align*}
Each of the series on the RHS converges uniformly on every bounded interval, so the original series also converges uniformly on every bounded interval (the first series does not depend on $x$ while the second series depends only on the max of the absolute values of the boundary points of the interval containing $x$). However, the original series does not converge absolutely because the second series on the RHS of 
\begin{align*}
\sum^\infty_{n=1}\abs{(-1)^n\frac{x^2+n}{n^2}} = \sum^\infty_{n=1}\frac{x^2}{n^2} + \sum^\infty_{n=1}\frac{1}{n}
\end{align*}
does not converge. \hfill $\square$\\




\noindent \textbf{7.7}
\noindent \textit{Proof.}  For $n=1,2,\dots, x$ real, put
\begin{align*}
f_n(x) = \frac{x}{1+nx^2}.
\end{align*}
We WTS $f_n \to f$ uniformly for some $f$ and $f'(x) = \lim_{n\to \infty}f'_n(x)$ is correct if $x\neq 0$ but not if $x = 0$. Well, by inspection we claim $f_n \to 0$ uniformly. Using calculus, we can find that
\begin{align*}
\abs{f_n(x)} = \abs{\frac{x}{1+n^2x}} \leq \frac{1}{2\sqrt{n}}
\end{align*}
where maximum is attained when $x = 1/\sqrt{n}$. This means for any $\epsilon > 0$, there is a sufficiently large $n$ such that $\abs{f_n(x)} < \epsilon$. So $\{ f_n\}$ converges uniformly. Next, if $x\neq 0$ then 
\begin{align*}
\lim_{n\to \infty}f'_n(x) = \lim_{n\to \infty} \frac{1-nx^2}{(1+nx^2)^2} = 0 = f'(x)
\end{align*}
where of course $f(x) = 0$. However, when $x=0$, 
\begin{align*}
\lim_{n\to \infty}f'_n(0) = \lim_{n\to \infty} \frac{1}{(1)^2} = 1 \neq 0 = f'(0).
\end{align*}
\hfill $\square$


  
\end{document}




