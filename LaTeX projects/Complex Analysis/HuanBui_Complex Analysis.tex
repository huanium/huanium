\documentclass{book}
\usepackage{physics}
\usepackage{amsmath}
\usepackage{authblk}
\usepackage{amsfonts}
\usepackage{esint}
\usepackage{bbold}
\usepackage{mathtools}
\usepackage{dsfont}
\usepackage{amsthm}
\usepackage{bbm}
\usepackage{amssymb}
\theoremstyle{definition}
\newtheorem{defn}{Definition}[section]
\newtheorem{prop}{Properties}[section]
\newtheorem{rmk}{Remark}[section]
\newtheorem{exmp}{Example}[section]
\newtheorem{prob}{Problem}[section]
\newtheorem{sln}{Solution}[section]
\newtheorem{thm}{Theorem}[section]
\newtheorem*{prob*}{Problem}
\newtheorem*{sln*}{Solution}
\usepackage{empheq}
\usepackage{tensor}



\usepackage{hyperref}
\usepackage{xcolor}
\hypersetup{
	colorlinks,
	linkcolor={black!50!black},
	citecolor={blue!50!black},
	urlcolor={blue!80!black}
}
%\usepackage{mathabx}

\newcommand{\R}{\mathbb{R}}

\newcommand{\F}{\mathcal{F}}
\newcommand{\p}{\partial}

\newcommand{\Arg}{\text{Arg}}

\newcommand{\f}[2]{\frac{#1}{#2}}

\newcommand{\G}{\mathcal{G}}
\newcommand{\C}{\mathbb{C}}
\newcommand{\Uni}{\mathcal{U}}

\newcommand{\V}{\mathbf{V}}
\newcommand{\W}{\mathbf{W}}
\newcommand{\Z}{\mathbf{Z}}
\newcommand{\Y}{\mathbf{Y}}
\newcommand{\U}{\mathbf{U}}
\newcommand{\X}{\mathbf{X}}

\newcommand*{\operp}{\perp\mkern-20.7mu\bigcirc}

\newcommand{\A}{\mathcal{A}}
\newcommand{\B}{\mathcal{B}}

\newcommand{\xpan}{\text{span}}

\newcommand{\lag}{\mathcal{L}}

\newcommand{\J}{\mathbf{J}}

\newcommand{\M}{\mathcal{M}}

\newcommand{\K}{\mathcal{K}}

\newcommand{\N}{\mathcal{N}}

\newcommand{\E}{\mathcal{E}}

\newcommand{\ima}{\text{Im}}
\newcommand{\lin}{\overset{\text{linear}}{\longrightarrow}}
\newcommand{\T}{\mathcal{T}}
\newcommand{\poly}{\mathbb{P}}
\newcommand{\s}{\mathcal{S}}

\newcommand{\jor}{\mathcal{J}}
\newcommand{\FF}{\mathfrak{F}}
\newcommand{\LL}{\mathfrak{L}}
\newcommand{\lat}{\mathfrak{Lat}}

\newcommand{\gives}{\rotatebox[origin=c]{180}{$\Rsh$}	}

\newcommand{\la}{\langle}
\newcommand{\ra}{\rangle}

\newcommand{\lp}{\left(}
\newcommand{\rp}{\right)}

\newcommand{\lb}{\left[}
\newcommand{\rb}{\right]}


\newcommand{\id}{\mathcal{I}}

\newcommand{\bigzero}{\mbox{\normalfont\Large\bfseries 0}}
\newcommand{\rvline}{\hspace*{-\arraycolsep}\vline\hspace*{-\arraycolsep}}



\usepackage{subfig}
\usepackage{listings}
\captionsetup[lstlisting]{margin=0cm,format=hang,font=small,format=plain,labelfont={bf,up},textfont={it}}
\renewcommand*{\lstlistingname}{Code \textcolor{violet}{\textsl{Mathematica}}}
\definecolor{gris245}{RGB}{245,245,245}
\definecolor{olive}{RGB}{50,140,50}
\definecolor{brun}{RGB}{175,100,80}
\lstset{
	tabsize=4,
	frame=single,
	language=mathematica,
	basicstyle=\scriptsize\ttfamily,
	keywordstyle=\color{black},
	backgroundcolor=\color{gris245},
	commentstyle=\color{gray},
	showstringspaces=false,
	emph={
		r1,
		r2,
		epsilon,epsilon_,
		Newton,Newton_
	},emphstyle={\color{olive}},
	emph={[2]
		L,
		CouleurCourbe,
		PotentielEffectif,
		IdCourbe,
		Courbe
	},emphstyle={[2]\color{blue}},
	emph={[3]r,r_,n,n_},emphstyle={[3]\color{magenta}}
}


\begin{document}
	\begin{titlepage}\centering
		\clearpage
		\title{\textsc{\bf{COMPLEX ANALYSIS}}\\\smallskip - A Quick Guide -\\}
		\author{\bigskip Huan Q. Bui}
		 \affil{Colby College\\$\,$\\ PHYSICS \& MATHEMATICS\\ Statistics \\$\,$\\Class of 2021\\}
		\date{\today}
		\maketitle
		\thispagestyle{empty}
	\end{titlepage}

\newpage

\section*{Preface}
\addcontentsline{toc}{subsection}{Preface}

Greetings,\\

\textit{Complex Analysis: A Quick Guide to} is compiled based on my MA352: Complex Analysis notes with professor Evan Randles. This guide is almost entirely based on \textit{Complex Variables and Applications, Eighth edition} by Churchill and Brown. \\

Enjoy!


\newpage
\tableofcontents
\newpage

\chapter{COMPLEX NUMBERS}

\section{Sums and Products}

Let $z \in \C$, it is customary to write
\begin{align}
z = x + iy = (x,y)
\end{align}
where
\begin{align}
x = \Re(z) \in \R \hspace{0.5cm} y = \Im(z) \in \R.
\end{align}
For $z_1, z_2 \in \C$, 
\begin{align}
z_1 = z_2 \iff \Re(z_1) = \Re(z_2) \wedge \Im(z_1) = \Im(z_2).
\end{align}
Addition works as we expect
\begin{align}
z_1 + z_2 = (x_1, y_1) + (x_2, y_2) = (x_1 + x_2 , y_1 + y_2).
\end{align}
So does multiplication
\begin{align}
z_1z_2 = (x_1,y_1)(x_2,y_2) = (x_1x_2 - y-1y_2, y_1x_2 + x_1y_2).
\end{align}
Of course,
\begin{align}
i^2 = -1 = (-1,0).
\end{align}


\section{Basic Algebraic Properties}

It is easy to see that complex number multiplication and addition are both commutative and associative:
\begin{align}
z_1 + z_2 = z_2 + z_1, &\hspace{0.5cm} z_1z_2 = z_2z_1\\
(z_1 + z_2) + z_3 = z_1 + (z_2 + z_3), &\hspace{0.5cm} (z_1z_2)z_3 = z_1(z_2z_3). 
\end{align}

The additive identity is $0 = (0,0)$. The multiplicative identity is $1 = (1,0)$. For $z = (x,y)\in \C$, the additive inverse is 
\begin{align}
-z = (-x,-y).
\end{align}

For any nonzero complex number $z = (x,y)$, there exists an multiplicative inverse $z^{-1}$ such that $zz^{-1} = z^{-1}z = 1$. We can find that
\begin{align}
z^{-1} = \lp \frac{x}{x^2 + y^2}, \frac{-y}{x^2 + y^2} \rp.
\end{align}

The existence of the multiplicative inverse allows us to show that if a product of two complex numbers is zero, then at least one of them is zero. And of course, if two complex numbers are nonzero, then so is their product. \\

Subtraction and division are defined in terms of addition and multiplication. For $z_1 = (x_1,y_2)$ and $z_2 = (x_2,y_2) \neq 0$, 
\begin{align}
&z_1 - z_2 = (x_1 - x_2, y_1 - y_2)\\
&\frac{z_1}{z_2} = z_1z_2^{-1} = (x_1,y_2)   \lp \frac{x_2}{x_2^2 + y_2^2}, \frac{-y_2}{x_2^2 + y_2^2} \rp = \lp \frac{x_1x_2 + y_1y_2}{x_2^2 + y_2^2} , \frac{y_1x_2 - x_1y_2}{x_2^2 + y_2^2} \rp.
\end{align}
This formula can be difficult to remember, so here's way to obtain it:
\begin{align}
\f{z_1}{z_2} = \f{(x_1 + iy_1)(x_2 - iy_2)}{(x_2 + iy_2)(x_2 - iy_2)}.
\end{align}






\section{Further Properties}

By the distributive law, we can show that
\begin{align}
\f{z_1 + z_2}{z_3} = (z_1 + z_2)z_3^{-1} = z_1z_3^{-1} + z_2z_3^{-1}  =\f{z_1}{z_3} + \f{z_2}{z_3}.
\end{align}

Beside some other expected properties involving quotients that follow, we also have the binomial formula. If $z_1, z_2$ are any two nonzero complex numbers, then
\begin{align}
(z_1 + z_2)^{n} = \sum^{n}_{k=0} {n\choose k} z_1^{k}z_2{n-k} \hspace{0.5cm} (n=1,2,\dots)
\end{align}







\section{Vectors and Moduli}

It is natural to associate $z = (x,y)$ to a point of a plane with coordinates $(x,y)$. The modulus of $z$ is defined as 
\begin{align}
\abs{z} = \sqrt{x^2 + y^2}.
\end{align}

The distance between two points $z_1, z_2$ is the same as the modulus of $z_1 - z_2$:
\begin{align}
\abs{z_2 - z_1} = \sqrt{(x_1 - x^2)^2 + (y_1 + y_2)^2}.
\end{align}

It is easy to see that 
\begin{align}
\abs{z}^2 = \Re(z)^2 + \Im(z)^2
\end{align}
so that
\begin{align}
&\Re(z) \leq \abs{\Re(z)} \leq \abs{z}\\
&\Im(z) \leq \abs{\Im(z)} \leq \abs{z}.
\end{align}


Next, we have the triangle inequality:
\begin{align}
\abs{z_1 + z_2} \leq \abs{z_1} + \abs{z_2}.
\end{align}

An immediate consequence of this inequality is another inequality:
\begin{align}
\abs{z_1 + z_2} \geq \abs{\abs{z_1} - \abs{z_2}}.
\end{align}
To prove this, we simply write $\abs{z_1} = \abs{(z_1 + z_2) - z_2}$. The triangle inequality takes care of the rest of the proof. \\

In summary, we have
\begin{align}
\abs{\abs{z_1} - \abs{z_2}} \leq \abs{z_1\pm z_2} \leq \abs{z_1} + \abs{z_2}.
\end{align}

The triangle inequality can be generalized by induction to sums involving any \textit{finite} number of terms:
\begin{align}
\abs{z_1 + z_2 + \dots + z_n} \leq \abs{z_1}+ \abs{z_2} + \dots + \abs{z_n}.
\end{align}








\section{Complex Conjugates}

For $z = (x,y)\in \C$, the complex conjugate of $z$, denoted $\bar{z}$, is
\begin{align}
\bar{z} = (x,-y). 
\end{align}
We note
\begin{align}
\bar{\bar{z}} = z, \hspace{0.5cm} \abs{\bar{z}} = \abs{z}.
\end{align}
We can show that
\begin{align}
\bar{z_1 + z_2} &= \bar{z_1} + \bar{z_2}\\
\bar{z_1z_2} &= \bar{z_1}\bar{z_2}\\
\bar{\lp \f{z_1}{z_2} \rp} &= \f{\bar{z_1}}{\bar{z_2}}, \hspace{0.5cm} (z_2 \neq 0)\\
\Re(z) &= \f{z + \bar{z}}{2}\\
\Im(z) &= \f{z - \bar{z}}{2i}\\
z\bar{z} &= \abs{z}^2\\
\abs{z_1 z_2} &= \abs{z_1}\abs{z_2}\\
\abs{\f{z_1}{z_2}} &= \f{\abs{z_1}}{\abs{z_2}}, \hspace{0.5cm} (z_2 \neq 0).
\end{align}





\section{Exponential Form}

For any nonzero complex number $z = (x,y)$, the polar form is 
\begin{align}
z = x + iy = r\cos\theta + ir\sin\theta,
\end{align}
where $r = \abs{z} \geq 0$. Note that for $z=0$, the angle $\theta$ is not defined. Each value of $\theta$ is called an argument of $z$, denoted $\arg(z)$. However, because $\arg(z)$ is ``multiple-valued,'' we define the \textit{principal value} of $\arg(z)$, $\Arg(z)$ as
\begin{align}
\arg(z) = \Arg(z) + 2n\pi, \hspace{0.5cm} (n = 0,\pm 1,\pm 2,\dots)
\end{align}

Note that when $z$ is a negative real number, $\Arg(z) = \pi$, not $-\pi$.\\

The polar form can also be re-written in a different way using Euler's formula:
\begin{align}
e^{i\theta} = \cos\theta + i\sin\theta.
\end{align}
With this,
\begin{align}
z = re^{i\theta} = \abs{z}e^{i\theta}.
\end{align}


\section{Products and Powers in Exponential Forms}

With a simple trigonometry identity, we can show that 
\begin{align}
e^{i\theta_1}e^{i\theta_2} = e^{i(\theta_1 + \theta_2)}.
\end{align}
So,
\begin{align}
z_1 z_2 = (r_1r_2)e^{i(\theta_1 + \theta_2)}.
\end{align}
Similarly,
\begin{align}
\f{z_1}{z_2} = \f{r_1}{r_2}e^{i(\theta_1 - \theta_2)}.
\end{align}

It is then easy to see that for $z\neq 0$,
\begin{align}
z^{-1} = \f{1}{z} = \f{1}{r}e^{-i\theta}.
\end{align}

And of course, we can see that 
\begin{align}
z^n = r^n e^{in\theta}, \hspace{0.5cm} (n=0,\pm 1,\pm 2,\dots).
\end{align}
This can be verified by induction. 


\section{Arguments of Products and Quotients}

For $z_1 = r_1 e^{i\theta_1}$ and $z_2 = r_2 e^{i\theta_2}$, 
\begin{align}
z_1z_2 = (r_1r_2)e^{i(\theta_1 + \theta_2)}.
\end{align}
So,
\begin{align}
\arg(z_1 + z_2) = \arg(z_1) + \arg(z_2).
\end{align}
We also have
\begin{align}
\arg\lp \f{z_1}{z_2} \rp = \arg(z_1) - \arg(z_2).
\end{align}







\section{Roots of Complex Numbers}

Let $z_1 = r_1 e^{i\theta_1}$ and $z_2 = r_2 e^{i\theta_2}$. By imagining $z_2$ and $z_2$ as points on a circle, we can see that 
\begin{align}
z_1 = z_2 \iff r_1 = r_2 \wedge \theta_1 = \theta_2 + 2n\pi
\end{align}
where $n$ is some integer $n = 0,\pm 1, \pm 2,\dots$\\

Next, it is useful to find roots of any nonzero complex number $z_0 = r_0 e^{i\theta_0}$. Well, suppose
\begin{align}
r^n e^{in\theta} = r_0 e^{i\theta_0},
\end{align}
where $n=2,3,\dots$ Then 
\begin{align}
r^n = r_0, \hspace{0.5cm} n\theta = \theta_0 + 2k\pi.
\end{align}
So,
\begin{align}
r &= \sqrt[n]{r_0} \\ 
\theta = \f{\theta_0}{n} + \f{2k\pi}{n}.
\end{align}
So, the complex numbers
\begin{align}
z = \sqrt[n]{r_0}\exp\lb i \lp \f{\theta_0}{n} + \f{2k\pi}{n} \rp \rb ,
\end{align}
where all the distinct roots are obtained when $k = 0,1,2,\dots,n-1$. These distinct roots are called $c_k$:
\begin{align}
c_k = \sqrt[n]{r_0} \exp\lb i\lp \f{\theta_0}{n} + \f{2k\pi}{n} \rp \rb
\end{align}
where $k = 0,1,2,\dots,n-1$.\\



$z_0^{1/n}$ denotes the \textit{set} of $n$th roots of $z_0$, while $\sqrt[n]{z_0}$ is reserved for the one positive root. If $\theta_0$ happens to be the principal argument of $z_0$ then $c_0$ is called the \textit{principal root}. This makes sense because when $z_0$ is a positive real number $r_0$ its principal root is nothing but $\sqrt[n]{r_0}$.\\

More compactly, we can also write
\begin{align}
c_k = c_0 \omega_n^k
\end{align}
where of course
\begin{align}
c_0 = \sqrt[n]{r_0}\exp\lp i\f{\theta_0}{n} \rp
\end{align}
and
\begin{align}
\omega_k = \exp\lp \f{2k\pi}{n} \rp.
\end{align}
A much more convenient way to remember all this is to first write 
\begin{align}
z_0 = r_0 e^{i(\theta_0 + 2k\pi)}.
\end{align}
Then from there, using the laws of exponential multiplication, we can get the \textit{set} of roots of $z_0$
\begin{align}
z_0^{1/n} = \lb r_0 e^{i(\theta_0 + 2k\pi)} \rb^{1/n} = \sqrt[n]{r_0}\exp\lb i\lp \f{\theta_0}{n} + \f{2k\pi}{n} \rp\rb
\end{align}
where $k = 0,1,2,\dots n-1$.



\section{Regions in the Complex Plane}

An $\epsilon$ \textit{neighborhood} 
\begin{align}
\abs{z - z_0} < \epsilon
\end{align}
of a given point $z_0$ contains all points $z$ lying inside but NOT on a circle centered at $z_0$ with a specified radius $\epsilon$. \\

Occasionally it's convenient to speak of a \textit{deleted neighborhood} (or punctured disk)
\begin{align}
0 < \abs{z - z_0} < \epsilon
\end{align}
consisting of all points $z$ in an $e\epsilon$ neighborhood of $z_0$ except for $z_0$ itself. \\

A point $z_0$ is an \textit{interior point} of a set $S$ when there is some neighborhood of $z_0$ that contains only points of $S$. \\

A point $z_0$ is an \textit{exterior point} of a set $S$ when there is some neighborhood of $z_0$ that contains no points of $S$. \\

If $z_0$ is neither of these then it is called a \textit{boundary point} of $S$. In this case, any neighborhood of $z_0$ contains at least one point in $S$ and at least one point not in $S$. \\

The \textit{totality} of all boundary points is called the \textit{boundary} of $S$. \\

A set is \textit{open} if it contains none of its boundary points. In fact, a set is open $\iff$ each of its points is an interior point. \\

A set is \textit{closed} if it contains all of its boundary points. \\

The \textit{closure} of a set $S$ is the closed set consisting of all points in $S$ together with the boundary of $S$. \\

Some sets are neither open nor closed. For example, the punctured disk $0 < \abs{z - z_0} \leq 1$ is neither an open nor a closed set because (1) it contains at least one boundary point and (2) the boundary point $z_0$ is NOT contained in the set. \\


An open set $S$ is \textit{connected} if each pair of points in it can be joined by a \textit{polygonal line}, consisting of a finite number of line segments joined end to end, that lies entirely in $S$. \\

A nonempty open set that is connected is called a \textit{domain}. Note: any neighborhood is a domain. \\

A \textit{region} is a domain together with some/none/all of its boundary points. \\

A set is \textit{bounded} if every point in it lies inside some circle. Otherwise, it is \textit{unbounded}. \\

A point $z_0$ is said to be an \textit{accumulation point} of a set $S$ if each deleted neighborhood of $z_0$ contains at least one point in $S$. Note: if a set is closed, then it contains each of its accumulation points. \\

A set is closed $\iff$ it contains all of its accumulation points. \\

A point $z_0$ is NOT an accumulation point of $S$ when there exists a deleted neighborhood which contains no points in $S$. 






\newpage



\chapter{Analytic Functions}





\section{Functions of a Complex Variable}

A function $f$ defined on a set $S$ of complex numbers assigns each complex number $z \in S$ to a complex number $w$, which is called the value of $f$. We write $f(z) = w$. $S$ is called the \textit{domain of definition} of $f$.  



\section{Mappings}
Because both $z$ and $w = f(z)$ both live on the complex plane, we can't simply draw a graph to represent $f$. Instead, we draw the $z$ and $w$ planes separately. When $f$ is thought of this way, we refer to $f$ as a \textit{mapping}. The set of images of points in a set $T$ is called the \textit{image} of $T$. The image of the entire domain of $f$ is called the \textit{range} of $f$. The \textit{inverse image} of a point $w$ is the set of all points $z$ whose $f(z) = w$. \\

Some examples of mappings include \textit{rotation, translation, reflection,} etc. 



\section{Mappings by the Exponential Function}

For $f(z) = e^z = e^x e^{iy} = \rho e^{i\theta}$, we see that vertical line segments are mapped to portions of circles, while horizontal line segments are mapped to portions of a ray. 




\section{Limits}

Let $f$ be defined at all points $z$ in some deleted neighborhood of $z_0$:
\begin{align}
\lim_{z\to z_0} f(z) = w_0
\end{align}
means that the point $w = f(z)$ can be made arbitrarily close to $w_0$ if we choose $z$ close enough to $z_0$ but distinct from it. More precisely, this statement means $\forall \epsilon >0, \exists \delta > 0 $ s.t. 
\begin{align}
\abs{f(z) - \omega_0} < \epsilon \hspace{0.5cm} \text{whenever} \hspace{0.5cm} 0 < \abs{z - z_0} < \delta.
\end{align}

We note that when a limit of a function $f(z)$ exists at a point $z_0$, it is unique. To show this, suppose that 
\begin{align}
\lim_{z \to z_0} f(z) &= w_0 \\
\lim_{z \to z_0} f(z) &= w_1.
\end{align} 
Applying the definition, we have that for any $\epsilon > 0$, there exists $\delta_1$ and $\delta_2$ such that
\begin{align}
&\abs{f(z) - w_0} < \epsilon \text{ whenever } 0 < \abs{z - z_0} < \delta_1 \\
&\abs{f(z) - w_0} < \epsilon \text{ whenever } 0 < \abs{z - z_0} < \delta_2.
\end{align}
Thus
\begin{align}
\abs{w_1 - w_0} = \abs{f(z) - w_0 - f(z) + w_1} \leq \abs{f(z) - w_1} + \abs{f(z) - w_0} < \epsilon + \epsilon  =2\epsilon.
\end{align}
But because $\abs{w_1 - w_0}$ is a nonnegative constant and $\epsilon$ can be chosen to be arbitrarily small, $w_1 = w_0$ necessarily. \qed





\section{Theorems on Limits}

\textbf{Theorem 1:} Suppose that for $z = x + iy$,
\begin{align}
f(z) = u(x,y) + iv(x,y)
\end{align}
and
\begin{align}
z_0 = x_0 + iy_0, \hspace{0.5cm} w_0 = u_0 + iv_0.
\end{align}
We have
\begin{align}
\lim_{z \to z_0}f(z) = w_0 \iff \lim_{(x,y)\to (x_0,y_0)} u(x,y) = u_0 \wedge \lim_{(x,y)\to (x_0,y_0)} v(x,y) = v_0.
\end{align}

To prove this, we first assume the latter two limits hold and show the first limit holds. The latter two limits tell us that $\forall \epsilon > 0, \exists \delta_1, \delta_2 > 0$ s.t. 
\begin{align}
&\abs{u - u_0} < \epsilon/2 \text{ whenever } 0 < \sqrt{(x - x_0)^2 + (y - y_0)^2} < \delta_1\\
&\abs{v - v_0} < \epsilon/2 \text{ whenever } 0 < \sqrt{(x - x_0)^2 + (y - y_0)^2} < \delta_2.
\end{align}
Let $\delta = \max (\delta_1, \delta_2)$. So, $\forall \epsilon > 0$, we have a $\delta > 0$ such that 
\begin{align}
\abs{(u + iv) - (u_0 + iv_0)} = \abs{(u - u_0) + i(v-v_0)} \leq \abs{u - u_0} + \abs{v - v_0} < \epsilon/2 + \epsilon/2 = \epsilon
\end{align}
whenever 
\begin{align}
0 < \sqrt{(x - x_0)^2 + (y - y_0)^2} = \abs{(x + iy) - (x_0 + iy_0)} < \delta.
\end{align}
This says the first limit holds.\\

Next, we assume that the first limit holds and show that the latter two follow. With this, $\forall \epsilon > 0, \exists \delta > 0$ such that 
\begin{align}
\abs{ (u + iv) - (u_0 + iv_0)} < \epsilon \text{ whenever } 0 < \abs{ (x + iy) - (x_0 + iy_0)} < \delta.
\end{align} 
But we also have that 
\begin{align}
&\abs{u - u_0} \leq \abs{(u - u_0) + i(v - v_0)} = \abs{ (u + iv) - (u_0 + iv_0)} < \epsilon\\
&\abs{v - v_0} \leq \abs{(u - u_0) + i(v - v_0)} = \abs{ (u + iv) - (u_0 + iv_0)} < \epsilon
\end{align}
and
\begin{align}
\abs{ (x + iy) - (x_0 + iy_0)} = \sqrt{(x - x_0)^2 + (y - y_0)^2}.
\end{align}
This says that $\forall \epsilon> 0, \exists \delta$ such that
\begin{align}
\abs{u - u_0} < \epsilon\\
\abs{v - v_0} < \epsilon
\end{align}
whenever
\begin{align}
0 < \sqrt{(x  -x_0)^2 + (y  - y_0)^2} < \delta,
\end{align}
i.e., the latter two limits hold. \qed



Next, we have a second theorem about sum, products, and quotients.\\

\noindent \textbf{Theorem 2:} Suppose that 
\begin{align}
\lim_{z \to z_0} f(z) = w_0 \hspace{0.5cm} \lim_{z \to z_0} F(z) = W_0.
\end{align}
Then 
\begin{align}
\lim_{z\to z_0} \lb  f(z) + F(z) \rb &= w_0 + W_0\\
\lim_{z \to z_0} \lb f(z)F(z) \rb &= w_0 W_0\\
\lim_{z \to z_0} \lb \f{f(z)}{F(z)} \rb &= \f{w_0}{W_0} \hspace{0.5cm} \text{if }W_0 \neq 0 .
\end{align}
The proof of these sub-theorems can be obtained from Theorem 1. \\

Next, we note that
\begin{align}
\lim_{z\to z_0} c = c, \hspace{0.5cm} \lim_{z \to z_0} z = z,
\end{align}
and so on. In fact, by induction, we can show that
\begin{align}
\lim_{z\to z_0} z^n = z_0^n
\end{align}
where $n=1,2,\dots$. Thus, it is not hard to see that
\begin{align}
\lim_{z\to z_0}P(z) = P(z_0)
\end{align}
where $P$ is some polynomial in $z$.


\section{Limits Involving the Point at Infinity}



The \textit{point at infinity} in the complex plane is denoted $\infty$. The complex plane with this point is called the \textit{extended} complex plane. More about visualizing the \textit{point at infinity} is in the book. For each small $\epsilon > 0$, the set $\abs{z} < 1/\epsilon$ an $\epsilon$ \textit{neighborhood} of $\infty$.\\

\noindent \textbf{Theorem:} If $z_0$ and $w_0$ are points in the $z$ and $w$ planes, respectively, then 
\begin{align}
\lim_{z\to z_0} f(z) = \infty \iff \lim_{z\to z_0} \f{1}{f(z)} = 0
\end{align} 
and
\begin{align}
\lim_{z\to \infty} f(z) = w_0 \iff \lim_{z\to 0} f\lp \f{1}{z} \rp = w_0
\end{align}
and
\begin{align}
\lim_{z\to \infty} f(z) = \infty \iff \lim_{z\to 0}\f{1}{f(1/z)} = 0.
\end{align}








\section{Continuity}

There are 3 criteria for $f$ to be continuous at a point $z_0$:
\begin{align}
&\lim_{z\to z_0} f(z) \text{ exists}\\
&f(z_0) \text{ exists}\\
&\lim_{z\to z_0} f(z) = f(z_0).
\end{align}
Note that the third statement is equivalent to saying $\forall \epsilon > 0, \exists \delta > 0$ such that
\begin{align}
\abs{f(z) - f(z_0)} < \epsilon \text{ whenever } 0 < \abs{z - z_0} < \delta.
\end{align}

If two functions are continuous at a point, then their sum and product are also continuous at that point. The quotient is also continuous at that point if the denominator is nonzero. \\


\noindent \textbf{Theorem 1.} A composition of continuous functions is continuous. 

\noindent \textbf{Theorem 2.} If a function $f(z)$ is continuous and nonzero at a point $z_0$, then $f(z) \neq 0$ throughout some neighborhood of that point. 

To prove this theorem we simply set $\epsilon = \abs{f(z_0)} / 2$. Because $f$ is continuous, there is a $\delta > 0$ such that 
\begin{align}
\abs{f(z) - f(z_0)} < \epsilon \text{ whenever } 0 < \abs{z - z_0} < \delta.
\end{align}
Now, if $f(z) = 0$ for some $z$ in this neighborhood, then we have a contradiction
\begin{align}
\abs{f(z_0)} < \f{\abs{f(z_0)}}{2}.
\end{align}
\qed


\noindent \textbf{Theorem 3.} If a function $f$ is continuous throughout a region $R$ that is both closed and bounded, there exists a nonnegative real number $M$ such that
\begin{align}
\abs{f(z)} \leq M \forall z \in R,
\end{align}
where equality holds for at least one such $z$. We say that $f$ is \textbf{bounded} on $R$. 




\section{Derivatives}





\section{Differentiation Formulas}
\section{Cauchy-Riemann Equations}
\section{Sufficient Conditions for Differentiability}
\section{Polar Coordinates}
\section{Analytic Functions}
\section{Examples}
\section{Harmonic Functions}
\section{Uniquely Determined Analytic Functions}
\section{Reflection Principle}


\newpage


















\end{document}
