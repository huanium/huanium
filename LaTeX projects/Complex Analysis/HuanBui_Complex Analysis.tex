\documentclass{book}
\usepackage{physics}
\usepackage{amsmath}
\usepackage{authblk}
\usepackage{amsfonts}
\usepackage{esint}
\usepackage{bbold}
\usepackage{mathtools}
\usepackage{dsfont}
\usepackage{amsthm}
\usepackage{bbm}
\usepackage{amssymb}
\theoremstyle{definition}
\newtheorem{defn}{Definition}[section]
\newtheorem{prop}{Properties}[section]
\newtheorem{rmk}{Remark}[section]
\newtheorem{exmp}{Example}[section]
\newtheorem{prob}{Problem}[section]
\newtheorem{sln}{Solution}[section]
\newtheorem{thm}{Theorem}[section]
\newtheorem*{prob*}{Problem}
\newtheorem*{sln*}{Solution}
\usepackage{empheq}
\usepackage{tensor}



\usepackage{hyperref}
\usepackage{xcolor}
\hypersetup{
	colorlinks,
	linkcolor={black!50!black},
	citecolor={blue!50!black},
	urlcolor={blue!80!black}
}
%\usepackage{mathabx}

\newcommand{\R}{\mathbb{R}}

\newcommand{\F}{\mathcal{F}}
\newcommand{\p}{\partial}

\newcommand{\Arg}{\text{Arg}}

\newcommand{\f}[2]{\frac{#1}{#2}}

\newcommand{\G}{\mathcal{G}}
\newcommand{\C}{\mathbb{C}}
\newcommand{\Uni}{\mathcal{U}}

\newcommand{\V}{\mathbf{V}}
\newcommand{\W}{\mathbf{W}}
\newcommand{\Z}{\mathbf{Z}}
\newcommand{\Y}{\mathbf{Y}}
\newcommand{\U}{\mathbf{U}}
\newcommand{\X}{\mathbf{X}}

\newcommand*{\operp}{\perp\mkern-20.7mu\bigcirc}

\newcommand{\A}{\mathcal{A}}
\newcommand{\B}{\mathcal{B}}

\newcommand{\xpan}{\text{span}}

\newcommand{\lag}{\mathcal{L}}

\newcommand{\J}{\mathbf{J}}

\newcommand{\M}{\mathcal{M}}

\newcommand{\K}{\mathcal{K}}

\newcommand{\N}{\mathcal{N}}

\newcommand{\E}{\mathcal{E}}

\newcommand{\ima}{\text{Im}}
\newcommand{\lin}{\overset{\text{linear}}{\longrightarrow}}
\newcommand{\T}{\mathcal{T}}
\newcommand{\poly}{\mathbb{P}}
\newcommand{\s}{\mathcal{S}}

\newcommand{\jor}{\mathcal{J}}
\newcommand{\FF}{\mathfrak{F}}
\newcommand{\LL}{\mathfrak{L}}
\newcommand{\lat}{\mathfrak{Lat}}

\newcommand{\gives}{\rotatebox[origin=c]{180}{$\Rsh$}	}

\newcommand{\la}{\langle}
\newcommand{\ra}{\rangle}

\newcommand{\lp}{\left(}
\newcommand{\rp}{\right)}

\newcommand{\lb}{\left[}
\newcommand{\rb}{\right]}


\newcommand{\id}{\mathcal{I}}

\newcommand{\bigzero}{\mbox{\normalfont\Large\bfseries 0}}
\newcommand{\rvline}{\hspace*{-\arraycolsep}\vline\hspace*{-\arraycolsep}}



\usepackage{subfig}
\usepackage{listings}
\captionsetup[lstlisting]{margin=0cm,format=hang,font=small,format=plain,labelfont={bf,up},textfont={it}}
\renewcommand*{\lstlistingname}{Code \textcolor{violet}{\textsl{Mathematica}}}
\definecolor{gris245}{RGB}{245,245,245}
\definecolor{olive}{RGB}{50,140,50}
\definecolor{brun}{RGB}{175,100,80}
\lstset{
	tabsize=4,
	frame=single,
	language=mathematica,
	basicstyle=\scriptsize\ttfamily,
	keywordstyle=\color{black},
	backgroundcolor=\color{gris245},
	commentstyle=\color{gray},
	showstringspaces=false,
	emph={
		r1,
		r2,
		epsilon,epsilon_,
		Newton,Newton_
	},emphstyle={\color{olive}},
	emph={[2]
		L,
		CouleurCourbe,
		PotentielEffectif,
		IdCourbe,
		Courbe
	},emphstyle={[2]\color{blue}},
	emph={[3]r,r_,n,n_},emphstyle={[3]\color{magenta}}
}


\begin{document}
	\begin{titlepage}\centering
		\clearpage
		\title{\textsc{\bf{COMPLEX ANALYSIS}}\\\smallskip - A Quick Guide -\\}
		\author{\bigskip Huan Q. Bui}
		 \affil{Colby College\\$\,$\\ PHYSICS \& MATHEMATICS\\ Statistics \\$\,$\\Class of 2021\\}
		\date{\today}
		\maketitle
		\thispagestyle{empty}
	\end{titlepage}

\newpage

\section*{Preface}
\addcontentsline{toc}{subsection}{Preface}

Greetings,\\

\textit{Complex Analysis: A Quick Guide to} is compiled based on my MA352: Complex Analysis notes with professor Evan Randles. This guide is almost entirely based on \textit{Complex Variables and Applications, Eighth edition} by Churchill and Brown. \\

Enjoy!


\newpage
\tableofcontents
\newpage

\chapter{COMPLEX NUMBERS}

\section{Sums and Products}

Let $z \in \C$, it is customary to write
\begin{align}
z = x + iy = (x,y)
\end{align}
where
\begin{align}
x = \Re(z) \in \R \hspace{0.5cm} y = \Im(z) \in \R.
\end{align}
For $z_1, z_2 \in \C$, 
\begin{align}
z_1 = z_2 \iff \Re(z_1) = \Re(z_2) \wedge \Im(z_1) = \Im(z_2).
\end{align}
Addition works as we expect
\begin{align}
z_1 + z_2 = (x_1, y_1) + (x_2, y_2) = (x_1 + x_2 , y_1 + y_2).
\end{align}
So does multiplication
\begin{align}
z_1z_2 = (x_1,y_1)(x_2,y_2) = (x_1x_2 - y-1y_2, y_1x_2 + x_1y_2).
\end{align}
Of course,
\begin{align}
i^2 = -1 = (-1,0).
\end{align}


\section{Basic Algebraic Properties}

It is easy to see that complex number multiplication and addition are both commutative and associative:
\begin{align}
z_1 + z_2 = z_2 + z_1, &\hspace{0.5cm} z_1z_2 = z_2z_1\\
(z_1 + z_2) + z_3 = z_1 + (z_2 + z_3), &\hspace{0.5cm} (z_1z_2)z_3 = z_1(z_2z_3). 
\end{align}

The additive identity is $0 = (0,0)$. The multiplicative identity is $1 = (1,0)$. For $z = (x,y)\in \C$, the additive inverse is 
\begin{align}
-z = (-x,-y).
\end{align}

For any nonzero complex number $z = (x,y)$, there exists an multiplicative inverse $z^{-1}$ such that $zz^{-1} = z^{-1}z = 1$. We can find that
\begin{align}
z^{-1} = \lp \frac{x}{x^2 + y^2}, \frac{-y}{x^2 + y^2} \rp.
\end{align}

The existence of the multiplicative inverse allows us to show that if a product of two complex numbers is zero, then at least one of them is zero. And of course, if two complex numbers are nonzero, then so is their product. \\

Subtraction and division are defined in terms of addition and multiplication. For $z_1 = (x_1,y_2)$ and $z_2 = (x_2,y_2) \neq 0$, 
\begin{align}
&z_1 - z_2 = (x_1 - x_2, y_1 - y_2)\\
&\frac{z_1}{z_2} = z_1z_2^{-1} = (x_1,y_2)   \lp \frac{x_2}{x_2^2 + y_2^2}, \frac{-y_2}{x_2^2 + y_2^2} \rp = \lp \frac{x_1x_2 + y_1y_2}{x_2^2 + y_2^2} , \frac{y_1x_2 - x_1y_2}{x_2^2 + y_2^2} \rp.
\end{align}
This formula can be difficult to remember, so here's way to obtain it:
\begin{align}
\f{z_1}{z_2} = \f{(x_1 + iy_1)(x_2 - iy_2)}{(x_2 + iy_2)(x_2 - iy_2)}.
\end{align}






\section{Further Properties}

By the distributive law, we can show that
\begin{align}
\f{z_1 + z_2}{z_3} = (z_1 + z_2)z_3^{-1} = z_1z_3^{-1} + z_2z_3^{-1}  =\f{z_1}{z_3} + \f{z_2}{z_3}.
\end{align}

Beside some other expected properties involving quotients that follow, we also have the binomial formula. If $z_1, z_2$ are any two nonzero complex numbers, then
\begin{align}
(z_1 + z_2)^{n} = \sum^{n}_{k=0} {n\choose k} z_1^{k}z_2{n-k} \hspace{0.5cm} (n=1,2,\dots)
\end{align}







\section{Vectors and Moduli}

It is natural to associate $z = (x,y)$ to a point of a plane with coordinates $(x,y)$. The modulus of $z$ is defined as 
\begin{align}
\abs{z} = \sqrt{x^2 + y^2}.
\end{align}

The distance between two points $z_1, z_2$ is the same as the modulus of $z_1 - z_2$:
\begin{align}
\abs{z_2 - z_1} = \sqrt{(x_1 - x^2)^2 + (y_1 + y_2)^2}.
\end{align}

It is easy to see that 
\begin{align}
\abs{z}^2 = \Re(z)^2 + \Im(z)^2
\end{align}
so that
\begin{align}
&\Re(z) \leq \abs{\Re(z)} \leq \abs{z}\\
&\Im(z) \leq \abs{\Im(z)} \leq \abs{z}.
\end{align}


Next, we have the triangle inequality:
\begin{align}
\abs{z_1 + z_2} \leq \abs{z_1} + \abs{z_2}.
\end{align}

An immediate consequence of this inequality is another inequality:
\begin{align}
\abs{z_1 + z_2} \geq \abs{\abs{z_1} - \abs{z_2}}.
\end{align}
To prove this, we simply write $\abs{z_1} = \abs{(z_1 + z_2) - z_2}$. The triangle inequality takes care of the rest of the proof. \\

In summary, we have
\begin{align}
\abs{\abs{z_1} - \abs{z_2}} \leq \abs{z_1\pm z_2} \leq \abs{z_1} + \abs{z_2}.
\end{align}

The triangle inequality can be generalized by induction to sums involving any \textit{finite} number of terms:
\begin{align}
\abs{z_1 + z_2 + \dots + z_n} \leq \abs{z_1}+ \abs{z_2} + \dots + \abs{z_n}.
\end{align}








\section{Complex Conjugates}

For $z = (x,y)\in \C$, the complex conjugate of $z$, denoted $\bar{z}$, is
\begin{align}
\bar{z} = (x,-y). 
\end{align}
We note
\begin{align}
\bar{\bar{z}} = z, \hspace{0.5cm} \abs{\bar{z}} = \abs{z}.
\end{align}
We can show that
\begin{align}
\bar{z_1 + z_2} &= \bar{z_1} + \bar{z_2}\\
\bar{z_1z_2} &= \bar{z_1}\bar{z_2}\\
\bar{\lp \f{z_1}{z_2} \rp} &= \f{\bar{z_1}}{\bar{z_2}}, \hspace{0.5cm} (z_2 \neq 0)\\
\Re(z) &= \f{z + \bar{z}}{2}\\
\Im(z) &= \f{z - \bar{z}}{2i}\\
z\bar{z} &= \abs{z}^2\\
\abs{z_1 z_2} &= \abs{z_1}\abs{z_2}\\
\abs{\f{z_1}{z_2}} &= \f{\abs{z_1}}{\abs{z_2}}, \hspace{0.5cm} (z_2 \neq 0).
\end{align}





\section{Exponential Form}

For any nonzero complex number $z = (x,y)$, the polar form is 
\begin{align}
z = x + iy = r\cos\theta + ir\sin\theta,
\end{align}
where $r = \abs{z} \geq 0$. Note that for $z=0$, the angle $\theta$ is not defined. Each value of $\theta$ is called an argument of $z$, denoted $\arg(z)$. However, because $\arg(z)$ is ``multiple-valued,'' we define the \textit{principal value} of $\arg(z)$, $\Arg(z)$ as
\begin{align}
\arg(z) = \Arg(z) + 2n\pi, \hspace{0.5cm} (n = 0,\pm 1,\pm 2,\dots)
\end{align}

Note that when $z$ is a negative real number, $\Arg(z) = \pi$, not $-\pi$.\\

The polar form can also be re-written in a different way using Euler's formula:
\begin{align}
e^{i\theta} = \cos\theta + i\sin\theta.
\end{align}
With this,
\begin{align}
z = re^{i\theta} = \abs{z}e^{i\theta}.
\end{align}


\section{Products and Powers in Exponential Forms}

With a simple trigonometry identity, we can show that 
\begin{align}
e^{i\theta_1}e^{i\theta_2} = e^{i(\theta_1 + \theta_2)}.
\end{align}
So,
\begin{align}
z_1 z_2 = (r_1r_2)e^{i(\theta_1 + \theta_2)}.
\end{align}
Similarly,
\begin{align}
\f{z_1}{z_2} = \f{r_1}{r_2}e^{i(\theta_1 - \theta_2)}.
\end{align}

It is then easy to see that for $z\neq 0$,
\begin{align}
z^{-1} = \f{1}{z} = \f{1}{r}e^{-i\theta}.
\end{align}

And of course, we can see that 
\begin{align}
z^n = r^n e^{in\theta}, \hspace{0.5cm} (n=0,\pm 1,\pm 2,\dots).
\end{align}
This can be verified by induction. 


\section{Arguments of Products and Quotients}

For $z_1 = r_1 e^{i\theta_1}$ and $z_2 = r_2 e^{i\theta_2}$, 
\begin{align}
z_1z_2 = (r_1r_2)e^{i(\theta_1 + \theta_2)}.
\end{align}
So,
\begin{align}
\arg(z_1 + z_2) = \arg(z_1) + \arg(z_2).
\end{align}






\section{Roots of Complex Numbers}





\newpage



\chapter{Analytic Functions}

\section{Functions of a Complex Variable}
\section{Mappings}
\section{Mappings by the Exponential Function}
\section{Limits}
\section{Theorems on Limits}
\section{Limits Involving the Point at Infinity}
\section{Continuity}
\section{Derivatives}
\section{Differentiation Formulas}
\section{Cauchy–Riemann Equations}
\section{Sufficient Conditions for Differentiability}
\section{Polar Coordinates}
\section{Analytic Functions}
\section{Examples}
\section{Harmonic Functions}
\section{Uniquely Determined Analytic Functions}
\section{Reflection Principle}


\newpage


















\end{document}
