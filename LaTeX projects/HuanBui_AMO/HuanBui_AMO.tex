\documentclass{book}
\usepackage{physics}
\usepackage{graphicx}
\usepackage{caption}
\usepackage{amsmath}
\usepackage[shortlabels]{enumitem}
\usepackage[left=1.25in,right=1.25in,top=1.25in,bottom=1.25in]{geometry}
\usepackage{bm}
\usepackage{authblk}
\usepackage{empheq}
\usepackage{amsfonts}
\usepackage{esint}
\usepackage[makeroom]{cancel}
\usepackage{dsfont}
\usepackage{centernot}
\usepackage{mathtools}
\usepackage{bigints}
\usepackage{amsthm}
\theoremstyle{definition}
\newtheorem{defn}{Definition}[section]
\newtheorem{prop}{Proposition}[section]
\newtheorem{rmk}{Remark}[section]
\newtheorem{thm}{Theorem}[section]
\newtheorem{exmp}{Example}[section]
\newtheorem{prob}{Problem}[section]
\newtheorem{sln}{Solution}[section]
\newtheorem*{prob*}{Problem}
\newtheorem{exer}{Exercise}[section]
\newtheorem*{exer*}{Exercise}
\newtheorem*{sln*}{Solution}
\usepackage{empheq}
\usepackage{hyperref}
\usepackage{tensor}
\usepackage{xcolor}
\hypersetup{
	colorlinks,
	linkcolor={black!50!black},
	citecolor={blue!50!black},
	urlcolor={blue!80!black}
}



\newcommand{\lambdabar}{{\mkern0.75mu\mathchar '26\mkern -9.75mu\lambda}}



\newcommand*\widefbox[1]{\fbox{\hspace{2em}#1\hspace{2em}}}

\newcommand{\p}{\partial}
\newcommand{\R}{\mathbb{R}}
\newcommand{\C}{\mathbb{C}}
\newcommand{\lag}{\mathcal{L}}
\newcommand{\nn}{\nonumber}
\newcommand{\ham}{\mathcal{H}}
\newcommand{\M}{\mathcal{M}}
\newcommand{\I}{\mathcal{I}}
\newcommand{\K}{\mathcal{K}}
\newcommand{\F}{\mathcal{F}}
\newcommand{\w}{\omega}
\newcommand{\lam}{\lambda}
\newcommand{\al}{\alpha}
\newcommand{\be}{\beta}
\newcommand{\x}{\xi}


\newcommand{\Else}{\text{else}}
\newcommand{\N}{\mathcal{N}}


\newcommand{\sig}{\bm\sigma}
\newcommand{\n}{\mathbf{n}}
\newcommand{\X}{\mathbf{X}}
\newcommand{\s}{\mathbf{S}}

\newcommand{\G}{\mathcal{G}}

\newcommand{\f}[2]{\frac{#1}{#2}}

\newcommand{\ift}{\infty}

\newcommand{\lp}{\left(}
\newcommand{\rp}{\right)}

\newcommand{\lb}{\left[}
\newcommand{\rb}{\right]}

\newcommand{\lc}{\left\{}
\newcommand{\rc}{\right\}}


\newcommand{\V}{\mathbf{V}}
\newcommand{\U}{\mathbf{U}}
\newcommand{\Id}{\mathbb{I}}
\newcommand{\D}{\mathcal{D}}
\newcommand{\Z}{\mathbf{Z}}
\newcommand{\had}{\mathbf{H}}
\newcommand{\Y}{\mathbf{Y}}
%\setcounter{chapter}{-1}


\makeatletter
\renewcommand{\@chapapp}{Chapter}
%\renewcommand\thechapter{$\bf{\ket{\arabic{chapter}}}$}
%\renewcommand\thesection{$\bf{\ket{\arabic{section}}}$}
%\renewcommand\thesubsection{$\bf{\ket{\arabic{subsection}}}$}
%\renewcommand\thesubsubsection{$\bf{\ket{\arabic{subsubsection}}}$}
\makeatother



\usepackage{subfig}
\usepackage{listings}
\captionsetup[lstlisting]{margin=0cm,format=hang,font=small,format=plain,labelfont={bf,up},textfont={it}}
\renewcommand*{\lstlistingname}{Code \textcolor{violet}{\textsl{Mathematica}}}
\definecolor{gris245}{RGB}{245,245,245}
\definecolor{olive}{RGB}{50,140,50}
\definecolor{brun}{RGB}{175,100,80}
\lstset{
	tabsize=4,
	frame=single,
	language=mathematica,
	basicstyle=\scriptsize\ttfamily,
	keywordstyle=\color{black},
	backgroundcolor=\color{gris245},
	commentstyle=\color{gray},
	showstringspaces=false,
	emph={
		r1,
		r2,
		epsilon,epsilon_,
		Newton,Newton_
	},emphstyle={\color{olive}},
	emph={[2]
		L,
		CouleurCourbe,
		PotentielEffectif,
		IdCourbe,
		Courbe
	},emphstyle={[2]\color{blue}},
	emph={[3]r,r_,n,n_},emphstyle={[3]\color{magenta}}
}


\begin{document}
	\begin{titlepage}\centering
		\clearpage
		\title{{\textsc{\textbf{ATOMIC, MOLECULAR, AND OPTICAL PHYSICS}}}\\ \smallskip - A Quick Guide - \\}
		\author{\bigskip Huan Q. Bui}
		\affil{Massachusetts Institute of Technology}
		\date{\today}
		\maketitle
		\thispagestyle{empty}
	\end{titlepage}

\subsection*{Preface}
\addcontentsline{toc}{subsection}{Preface}

Greetings,\\

This guide is based on 8.421 and 8.422 at MIT, two courses on atomic, molecular, and optical (AMO) physics, taught by Professor Martin Zwierlein between 2022 and 2023. The purpose of this guide is to provide a short summary of the most basic concepts in AMO physics and serve as a reference for PhD qualification exams in this field. This guide may not be treated as a textbook, and the reader may find much more pedagogical resources in the bibliography. Familiarity with undergraduate physics is recommended. \\

\noindent Enjoy! 

\newpage
\tableofcontents
\newpage



%%%%%%%%%%%%%%%%%%%%%%%%%%%%%%%%%%%%%%%%%%%%%%
%%%%%%%%%%%%%%%%%%%%%%%%%%%%%%%%%%%%%%%%%%%%%%



\chapter{Harmonic Oscillators and Resonance}


\section{The Lorentz Oscillator Model}

This model is incredibly powerful and is one of the most fundamental topics to know and understand when learning AMO physics. So let's get started. We will first consider a \textbf{classical harmonic oscillator} first. For non-trivial dynamics, we consider the case where this system is damped and driven. 
\begin{equation}\label{eq:lorentz}
\ddot x + \gamma \dot x + \omega_0^2 x = \f{F_0}{m}\cos(\omega t).
\end{equation}
Here $\gamma,\omega_0,m,F_0$ take their usual meanings. As a review, we shall introduce the general steady-state solution (we typically don't care about transient behavior) and write down the under-damping condition explicitly. To do this, we have to solve the ODE  \eqref{eq:lorentz}. Then, we will answer some questions that will improve our understanding of the model. \\



\noindent \textbf{\underline{What is the general solution?}}  \\


To make the forthcoming algebraic manipulations clearer, let us use complex notations, so that $\cos(\omega t) \to e^{i\omega t}$. Since the problem concerns only the steady-state solution, we shall ignore the transient behavior and consider the following ansatz oscillating at the drive frequency $\omega$:
\begin{align*}
x(t) = A e^{i\omega t } e^{i\phi}
\end{align*}
where $A \in \mathbb{R}$ is the amplitude and $\phi \in \mathbb{R}$ is the phase. Plugging the ansatz into the ODE, we find 
\begin{align*}
x(t) = A e^{i\omega t} e^{i\phi} \implies -A e^{i \omega t + i\phi} \lp \omega^2 -\omega_0^2 - i\gamma \omega \rp = \f{F_0}{m}e^{i\omega t} \implies A = \f{F_0/m}{-\omega^2 + \omega_0^2 + i\gamma \omega} e^{-i\phi}.
\end{align*}
Since $A$ is real, the denominator of $A$ must be a complex number of the form $\abs{A} e^{-i\phi}$ where
\begin{align*}
\abs{A} = \sqrt{{(\omega_0^2 - \omega^2)}^2 + (\gamma \omega)^2}.
\end{align*}
And the phase $\phi$ is\footnote{I should have used $e^{-i\phi}$ so that $\phi$ in this definition would be the phase \textit{lag}... but okay.}
\begin{align*}
\boxed{\phi = -\arctan(\f{\gamma\omega}{\omega_0^2 - \omega^2})}
\end{align*}
With these, we may write
\begin{align*}
\boxed{A = \f{F_0/m}{\sqrt{(\omega_0^2 - \omega^2)^2 + (\gamma \omega)^2}}}
\end{align*}
The oscillator is underdamped, so we want the roots of the associated chacteristic polynomial $\lambda^2 + \gamma \lambda + \omega_0^2 = 0$:
\begin{align*}
\lambda\pm = -\f{\gamma}{2} \pm \f{1}{2}\sqrt{\gamma^2 - 4\omega_0^2}
\end{align*}
to be complex (to get oscillations on top of an exponential decay). As a result, we require that 
\begin{align*}
\gamma^2 < 4\omega_0^2 \iff \boxed{\gamma < 2\omega_0}
\end{align*}
This is the standard underdamping condition. \\


\noindent \textbf{\underline{At which $\omega$ do we get maximal amplitude response?}} \\

Before proceeding, let us introduce two dimensionless quantities $\rho = \omega/\omega_0$ and $f = \gamma/\omega_0$ and rewrite $A$ as 
\begin{align*}
A =  \lp \f{F_0}{m \omega_0^2}\rp \f{1}{\sqrt{(1-\rho^2)^2 + (f\rho)^2}}
\end{align*}
Finding $\omega$ so that $A$ is maximal requires finding $\rho$ for which the denominator of $A$ is minimal:
\begin{align*}
\f{d}{d\rho}\lb (1-\rho^2)^2 + (f\rho)^2 \rb = 2\rho(-2 + f^2 + 2\rho^2).
\end{align*}
Setting the expression above to zero tells us that $\rho$ could be $0$ or $\sqrt{1-f^2/2}$ (we must also verify that $A$ has a global maximum, but I won't go into the details here). Since $f = \gamma/\omega_0 \in (0,2)$ we must consider two cases:
\begin{itemize}
	\item If ${0 < f \leq \sqrt{2}}$ then the solution $\rho = \sqrt{1-f^2/2}$ is real and we have 
	\begin{align*}
	A(0)= \f{F_0}{m\omega_0^2} <  \f{F_0}{m\omega_0^2} \lp \f{2}{f\sqrt{4-f^2}}\rp  = A\lp \sqrt{1-f^2/2}\rp
	\end{align*}
	because $f\sqrt{4-f^2} \leq 2$ for $f\in (0,2)$. We therefore see that $A$ attains its maximum at $$\boxed{\omega = \omega_0 \sqrt{1- \gamma^2/2\omega_0^2}}$$
	
	
	
	\item Otherwise, if ${\sqrt{2} < f < 2}$ then the solution $\rho = \sqrt{1-f^2/2}$ is imaginary and $A$ attains its maximum of $F_0/m\omega_0^2$ at $$\boxed{\omega = 0}$$
\end{itemize}
Contrary to what we might expect, the maximal amplitude is attained \textbf{NOT} at the resonance frequency $\omega_0$. Rather, depending on the ratio $\gamma/\omega_0$, we have different answers. \\



\noindent \textbf{\underline{At which $\omega$ does the phase lag between the response and the drive becomes $\pi/2$?}} \\



The phase \textit{lag} being $\pi/2$ means that $\phi = -\pi/2$, i.e.,
\begin{align*}
\arctan\lp \f{\gamma \omega}{\omega_0^2 - \omega^2} \rp = \f{\pi}{2} \implies \boxed{\omega = \omega_0}
\end{align*}
This means that the resonance condition is better characterized by the $\pi/2$ phase lag of the response relative to the drive, rather than the amplitude attaining its maximal value. \\




\noindent \textbf{\underline{When does the power delivered from the drive to the oscillator become maximal?}} \\

Here, by ``power'' we actually mean the power delivered from the drive averaged over one cycle. The power from the drive is given by $P = dW/dt$ where $dW$ is the work done by the drive over an infinitesimal $dx$ and is thus given by $dW = Fdx$.  Putting everything together we have $P = F(t)x'(t)$. The power delivered from the drive, averaged over one cycle of period $T = 2\pi / \omega$, is therefore
\begin{align*}
\langle P \rangle 
&= \f{\omega}{2\pi}\int_{0}^{2\pi/\omega} F_0 \cos(\omega t) \f{d}{dt} \Re{x(t)} \,dt \\
&= \f{F_0 A \omega}{2\pi}  \int_{0}^{2\pi/\omega} \cos(\omega t)  \f{d}{dt} \cos\lp \omega t + \phi \rp \,dt \\
&= -\f{F_0 A \omega^2}{2\pi}  \int_{0}^{2\pi/\omega} \cos(\omega t) \sin\lp \omega t + \phi \rp \,dt \\
&= -\f{1}{2}F_0 A \omega \sin \phi.
\end{align*} 
Plugging in the expressions for $\phi$ and $\omega$ we find 
\begin{align*}
\boxed{\langle P \rangle = \f{F_0^2 }{2m\gamma}\f{(\gamma\omega)^2}{(\omega_0^2 - \omega^2)^2 + (\gamma \omega)^2}}
\end{align*}
We recognize that $P$ has the form of a Lorentzian (or a Cauchy distribution) with FWHM $\gamma$ which attains the maximum $\langle P \rangle_\text{max} = F_0^2/2m\gamma$ at $\omega = \omega_0$.  \textcolor{blue}{One may also use standard calculus techniques to get this result.} \\



\noindent $\boxed{\textbf{!}}$ It makes sense that the power delivered by the drive to the oscillator, averaged over one cycle, is the same as the power dissipated by the damping, averaged one cycle, since the system is in equilibrium in steady state. This can be verified by repeating the calculation above explicitly, but for the damping force. The work done by drag force is $-m\gamma x'(t)$, from which we find the dissipated power is $P_\text{dis} = Fv = m\gamma v(t)^2$. From here, we have
\begin{align*}
\langle P_\text{dis}\rangle = \f{\omega}{2\pi}m\gamma A^2\int_0^{2\pi/\omega} \lp \f{d}{dt}\cos(\omega t + \phi)\rp^2\,dt = \f{1}{2}m\gamma A^2\omega^2.
\end{align*}
Plugging in the expression for $A$ we find that
\begin{align*}
\langle P_\text{dis}\rangle = \f{F_0^2}{2m\gamma} \f{(\gamma\omega)^2}{(\omega_0^2 - \omega^2)^2 + (\gamma \omega)^2} = \langle P \rangle \,\,\,\checkmark
\end{align*}



\noindent \textbf{\underline{Why does the dissipated power become maximal at that frequency?} } \\


When the drive $F \propto \cos (\omega t)$ is at $\omega = \omega_0$, the position $x(t) \propto \cos(\omega_0 t + \phi)$ of the oscillator has a $-\phi =\pi/2$ phase lag compared to the drive. However, the velocity $x'(t) \propto -\sin(\omega_0 t -\pi/2) = \cos(\omega_0 t)$ is now in phase with the drive. As a result, the drive is always doing positive work, and thus $\langle P \rangle$ is maximal.  \\







\noindent \textbf{\underline{Why does reducing $\gamma$ increase $P$ on resonance?}}\\

In the far off-resonance regime, we may assume that $\abs{\omega_0^2 - \omega^2}\gg \gamma \omega$, so that 
\begin{align*}
\langle P \rangle_\text{far off res.}  \approx   \f{F_0^2}{2m\gamma} \f{(\gamma \omega)^2}{(\omega_0^2 - \omega^2)^2} \propto \gamma,
\end{align*}
so $\langle P \rangle$ varies linearly in $\gamma$ for far off-resonance drive. In the near-resonance regime, we may ignore the term $\omega_0^2 - \omega^2$ in the denominator to find 
\begin{align*}
\langle P \rangle_\text{near res.} \approx \f{F_0}{2m\gamma} \propto \gamma^{-1}.
\end{align*}


As the damping is decreased, the amplitude of the motion increases, thereby increasing the dissipated power on resonance. In the limit of no damping $(\gamma = 0)$, the power dissipated on resonance becomes infinite because the amplitude blows up. The line shape of the response becomes that of a Dirac delta function. \\




\noindent \textbf{\underline{On resonance, what is the steady-state average energy stored and dissipated per cycle?}}\\

Now we have resonant driving, so $\omega = \omega_0$. The steady-state average energy stored in the oscillator is the sum of kinetic and potential energy. 
\begin{align*}
\langle KE\rangle &= \f{\omega_0}{2\pi} \int_0^{2\pi/\omega_0} \f{1}{2}m x'(t)^2\,dt=   \f{F_0^2 }{4m\gamma^2}\\
\langle PE \rangle &= \f{\omega_0}{2\pi}\int_0^{2\pi/\omega_0} \f{1}{2}m \omega_0^2 x(t)^2\,dt  = \f{F_0^2}{4m\gamma^2}
\end{align*}
where we have used $x(t) = (F_0/m\gamma \omega_0) \sin(\omega_0 t)$. From here, the total energy stored in the oscillator is 
\begin{align*}
\boxed{\langle E \rangle = \langle KE \rangle + \langle PE \rangle = \f{F_0^2}{2m\gamma^2}}
\end{align*}
On the other hand, the energy dissipated in one cycle may be calculated by integrating the dissipated power over a cycle $E_\text{lost} = \int P_\text{dis}\,dt$ where $P_\text{dis} = F_\text{damp} v(t)$: 
\begin{align*}
\boxed{E_\text{lost} = \int_0^{2\pi/\omega_0} \gamma m x'(t) x'(t)\,dt = \f{F_0^2 \pi}{m\gamma \omega_0}}
\end{align*}
Take the $2\pi$-adjusted ratio of these two results, we find 
\begin{align*}
2\pi \f{\langle E \rangle}{E_\text{lost}} = \f{\omega_0}{\gamma},
\end{align*}
which is nothing but the \textbf{quality factor} $\boxed{Q = \omega_0 / \gamma}$.\\










\noindent \textbf{\underline{What happens when the mass has charge $q$ and is driven by $F = e\mathcal{E}\cos(\omega t)$?}}\\


This is the \textbf{Lorentz Oscillator Model}. When $\gamma = 0$ (undamped -- we will think about what damping means in this case), the steady-state dipole moment $d(t) = ex(t)$ is given by:
\begin{align*}
d(t) = ex(t) = \f{e^2 E\cos(\omega t)}{\omega_0^2 - \omega^2}.
\end{align*}
This expression is related to a concept called \textbf{oscillator strength} which we will look at later. \\




Even in the absence of other kinds of damping, the motion will be damped because
of what is called \textbf{radiation damping}, $\gamma = \Gamma_{\text{rad}}$. From classical electrodynamics, we know that
any accelerated charge will emit radiation. The total power radiated by an accelerated
electron in the full solid angle of $4\pi$ (see \cite{griffiths2005introduction} for details) is 
\begin{align*}
P = \f{1}{6\pi \epsilon_0 c^3}|\ddot{d}|^2.
\end{align*}



\noindent \textbf{\underline{On resonance, what is the energy lost per orbital cycle?}}\\

Taking the amplitude of $x(t)$ to be $x_0$, the total power radiated in the full solid angle of $4\pi$ is 
\begin{align*}
P = \f{1}{6\pi \epsilon_0 c^3}\abs{\ddot d}^2 = \f{e^2 x_0^2 \omega_0^4 \cos^2(\omega_0 t)}{6 c^3 \pi \epsilon_0}.
\end{align*}
The energy lost per orbital cycle is thus
\begin{align*}
E_\text{lost} = \int_0^{2\pi/\omega_0} P\,dt = \f{e^2 x_0^2 \omega_0^3}{6c^3 \epsilon_0}.
\end{align*}



\noindent \textbf{\underline{From $E_\text{stored}$ and $E_\text{lost}$, what is $\Gamma_\text{rad}$? }}\\

The total energy is calculated in the same manner as before:
\begin{align*}
\langle KE\rangle &= \f{\omega_0}{2\pi} \int_0^{2\pi/\omega_0} \f{1}{2}m x'(t)^2\,dt= \f{1}{4}m \omega_0^2 x_0^2  \\
\langle PE \rangle &= \f{\omega_0}{2\pi}\int_0^{2\pi/\omega_0} \f{1}{2}m \omega_0^2 x(t)^2\,dt  = \f{1}{4}m \omega_0^2 x_0^2 \\
E_\text{stored} &= \langle KE \rangle + \langle PE \rangle = \f{1}{2}m \omega_0^2 x_0^2
\end{align*}
From here, we get
\begin{align*}
Q = 2\pi \f{E_\text{stored}}{E_\text{lost}} = \f{6c^3 m\pi \epsilon_0}{e^2 \omega_0}.
\end{align*}
Since $Q = \omega_0 / \Gamma_{\text{rad}}$, we find
\begin{align*}
\boxed{\Gamma_\text{rad} = \f{\omega_0}{Q} = \f{e^2 \omega_0^2}{6c^3 m \pi \epsilon_0}}
\end{align*} 

\noindent \textbf{\underline{What is $Q$ in terms of the classical radius of the electron and $\lambdabar $ of the emission?}}\\


In terms of the classical radius of the electron
\begin{align*}
r_0 = \f{e^2}{4\pi \epsilon_0 mc^2}, 
\end{align*}
we have
\begin{align*}
Q = \f{3\lambdabar}{2r_0} \quad\quad \text{and}\quad\quad \Gamma_\text{rad} = \f{\omega_0}{Q} = \f{2r_0c}{3\lambdabar^2}
\end{align*}




\noindent \textbf{\underline{Estimate $Q,\Gamma_\text{rad}$ for the Sodium D2 line.}}\\



With $\lambda = 589 $ nm, we have
\begin{align*}
Q &\approx 5.0 \times 10^7 \\
\Gamma_\text{rad} &\approx 2\pi \times 10.2 \text{ MHz}
\end{align*}
which is in remarkable agreement with the experimentally measured natural line width for the D2 line of Na which is $2\pi \times 9.795(11)$ MHz. The calculation is within $\sim 5$\% of the measured value.  





\subsection{The Signal-to-Noise ratio and the quality factor}



Let's look at the expression for the power delivered to the oscillator again:
\begin{align*}
{\langle P \rangle = \f{F_0^2 }{2m\gamma}\f{(\gamma\omega)^2}{(\omega_0^2 - \omega^2)^2 + (\gamma \omega)^2}}
\end{align*}
Near resonance, we may make the following approximation:
\begin{align*}
\omega_0^2 - \omega^2 \approx 2\omega(\omega_0  - \omega),
\end{align*}
so that 
\begin{align*}
\langle P \rangle = \f{F_0^2 }{2m\gamma} \f{1}{1 + \lp \f{\omega - \omega_0}{\gamma/2}\rp ^2}.
\end{align*}
This is a Lorentzian with FWHM $\Delta \omega = \gamma$. The time constant for the decay constant $\gamma$ is simply $\tau = 1/\gamma$. The quality factor is again $Q = \omega_0  / \gamma = \omega_0 / \Delta \omega$. From here, we see that
\begin{align*}
\tau \Delta \omega = 1
\end{align*}
which can be treated as an uncertainty relation, which characterizes \textbf{individual measurements}. This means we can actually find $\omega_0$ to better than $\Delta \omega$ with multiple measurements. To be more precise: In the presence of noise, the frequency precision with which the center can be located, $\delta \omega$, depends on the signal-to-noise ratio via
\begin{align*}
\delta \omega = \f{\Delta \omega}{S/N},
\end{align*}
as a rule of thumb. This means that we can see that with high $S/N$, we can measure $\omega_0$ very precisely. This is what we mean by ``splitting the resonance line.'' Typically, experimenters can split the line by a factor of $10^3$. 


\section{The Quantum Harmonic Oscillator}


Consider a 1D harmonic oscillator of mass $m$ and frequency $\omega$ in a number state $\ket{n}$. \\




\noindent \textbf{\underline{How are the annihilation and creation operators $ a$ and $ a^\dagger$ relate to $ x$ and $ p$?}}\\


Let $a = A  x - i B  p$, where $A,B$ are real scalars, so that $ a^\dagger = A x+ i B  p$. We want the following to hold:
\begin{align*}
\ham = \f{p^2 }{2m} + \f{1}{2}m\omega^2 x^2 = \hbar \omega \lp a^\dagger a + \f{1}{2} \rp
\end{align*}
so we compute
\begin{align*}
\hbar \omega \lp a^\dagger a+ \f{1}{2}\rp = \hbar \omega A^2 x^2 + \hbar \omega B^2 p^2 + \hbar \omega AB + \f{\hbar \omega}{2}
\end{align*}
where we have used $[x,p] = i\hbar $. By setting 
\begin{align*}
A = \sqrt{\f{m\omega}{2\hbar}} \quad\quad B = -\sqrt{\f{1}{2\hbar m\omega}}
\end{align*}
the first equation is satisfied. We therefore conclude that
\begin{align*}
a = \sqrt{\f{m\omega}{2\hbar}} \lp x + \f{i}{m\omega} p \rp \quad\quad a^\dagger = \sqrt{\f{m\omega}{2\hbar}} \lp x - \f{i}{m\omega} p \rp.
\end{align*}
From here, we find 
\begin{align*}
\boxed{x = \sqrt{\f{\hbar}{2 m\omega}}(a^\dagger + a) \quad\quad\text{and}\quad\quad p = i\sqrt{\f{\hbar m\omega }{2}}(a^\dagger - a)}
\end{align*}




\noindent \textbf{\underline{What are the average and rms position and momentum?}}\\


From the results above, we get
\begin{align*}
\bra{n} x \ket{n} = \sqrt{\f{\hbar}{2 m\omega}}\bra{n} a^\dagger + a \ket{n} = 0
\end{align*}
\begin{align*}
\bra{n} p \ket{n} = i\sqrt{\f{\hbar m\omega }{2}}\bra{n} a^\dagger - a \ket{n} = 0
\end{align*}
since both $a^\dagger$ and $a$ respectively send $\ket{n}$ to $\ket{n+1}$ and $\ket{n-1}$ which are orthonormal to $\ket{n}$. Next, 


\begin{align*}
\sqrt{\bra{n}p^2 \ket{n}} &= \sqrt{-\f{\hbar m\omega}{2}\bra{n} (a^\dagger - a)^2 \ket{n} }\\
&= \sqrt{-\f{\hbar m\omega}{2} \bra{n} a^\dagger a^\dagger -a^\dagger a - aa^\dagger + a^2 \ket{n} }\\
&= \sqrt{\hbar m\omega \lp n+ \f{1}{2} \rp}.
\end{align*}

\begin{align*}
\sqrt{\bra{n}x^2 \ket{n}} &= \sqrt{\f{\hbar}{2 m\omega}\bra{n} (a^\dagger + a)^2 \ket{n} }\\
&= \sqrt{\f{\hbar}{2 m\omega}\bra{n} a^\dagger a^\dagger +a^\dagger a + aa^\dagger + a^2 \ket{n} }\\
&= \sqrt{\f{\hbar}{m\omega}\lp n+ \f{1}{2} \rp}.
\end{align*}







\noindent \textbf{\underline{Check the results above using energy and the virial theorem.}}\\



The total energy is 
\begin{align*}
\langle  E\rangle  = \f{\langle p^2\rangle }{2m} + \f{1}{2}m\omega^2 \langle x^2\rangle = \f{1}{2m}\hbar m\omega \lp n+ \f{1}{2} \rp + \f{1}{2}m\omega^2 \f{\hbar}{m\omega}\lp n+ \f{1}{2} \rp  = \hbar \omega\lp n+ \f{1}{2}\rp \,\,\, \checkmark
\end{align*}

Virial theorem: Since $V\propto x^2$, the virial theorem states that $\langle T \rangle  = \langle V \rangle$. From the calculation above, we see that this holds:
\begin{align*}
\langle T \rangle =  \f{1}{2m}\hbar m\omega \lp n+ \f{1}{2} \rp  = \f{1}{2}m\omega^2 \f{\hbar}{m\omega}\lp n+ \f{1}{2} \rp = \langle V \rangle \,\,\, \checkmark
\end{align*}






\noindent \textbf{\underline{What is the rms size and velocity for Na in $\ket{0,0,0}$ with $\omega = 2\pi \times 100$ Hz?}}\\


For a Na atom in the state $\ket{0,0,0}$ of a 3D harmonic potential, we have, by spherical symmetry:
\begin{align*}
\sqrt{\langle r^2\rangle} = \sqrt{3\langle x^2 \rangle } = \sqrt{3(2\cdot 0+1)\f{\hbar}{2m\omega}} = \sqrt{\f{3\hbar}{2m\omega}}.
\end{align*}
With $\omega = 2\pi \times 100 $ Hz and $m_\text{Na} \approx 23 \times 1.66054\times 10^{-27}$ kg, the rms size is
\begin{align*}
\sqrt{\langle r^2\rangle} \approx 2.57 \text{ \textmu m}.
\end{align*}




Similarly, we can find the rms velocity using spherical symmetry:
\begin{align*}
\sqrt{\langle v^2 \rangle} = \f{1}{m}\sqrt{\langle p^2 \rangle} = \f{1}{m}\sqrt{3\langle p_x^2 \rangle} = \f{1}{m}\sqrt{3(2\cdot 0 + 1) \f{\hbar m \omega}{2}} = \sqrt{\f{3\hbar \omega}{2m}}.
\end{align*}
The numerical value for this is
\begin{align*}
\sqrt{\langle v^2 \rangle} \approx 1.61 \text{ mm/s}.
\end{align*}







%%%%%%%%%%%%%%%%%%%%%%%%%%%%%%%%%%%%%%%%%%%%%%
%%%%%%%%%%%%%%%%%%%%%%%%%%%%%%%%%%%%%%%%%%%%%%





\chapter{The Two-Level System}








\section{Magnetic Resonance: Classical Spin in $\mathbf{{B}(t)}$}




\subsection{Classical spin in a magnetic field}

\subsection{Rotating coordinate transformation}


\subsection{Larmor's Theorem}


\subsection{Motion in a rotating magnetic field}

\subsection{Rapid Adiabatic Passage: Landau-Zener Crossing}


\subsection{Adiabatic Following in a Magnetic Trap}





\section{Resonance of a Quantized Spin}




\subsection{Expectation value of magnetic moment behaves classically}


\subsection{The Rabi Problem}

take this section from the HW


\subsection{Rapid Adiabatic Passage (Landau-Zener) in a QM treatment}


\subsection{Adiabatic Passage}



\section{Density Matrix Formalism}











\section{Atomic Units}





%%%%%%%%%%%%%%%%%%%%%%%%%%%%%%%%%%%%%%%%%%%%%%
%%%%%%%%%%%%%%%%%%%%%%%%%%%%%%%%%%%%%%%%%%%%%%


\chapter{The Basics of an Atom}






%%%%%%%%%%%%%%%%%%%%%%%%%%%%%%%%%%%%%%%%%%%%%%
%%%%%%%%%%%%%%%%%%%%%%%%%%%%%%%%%%%%%%%%%%%%%%


\chapter{Fine Structure and Lamb Shift}














%%%%%%%%%%%%%%%%%%%%%%%%%%%%%%%%%%%%%%%%%%%%%%
%%%%%%%%%%%%%%%%%%%%%%%%%%%%%%%%%%%%%%%%%%%%%%


\chapter{Effects of the Nucleus on Atomic Structure}










%%%%%%%%%%%%%%%%%%%%%%%%%%%%%%%%%%%%%%%%%%%%%%
%%%%%%%%%%%%%%%%%%%%%%%%%%%%%%%%%%%%%%%%%%%%%%




\chapter{Atoms in Magnetic Fields}









%%%%%%%%%%%%%%%%%%%%%%%%%%%%%%%%%%%%%%%%%%%%%%
%%%%%%%%%%%%%%%%%%%%%%%%%%%%%%%%%%%%%%%%%%%%%%



\chapter{Atoms in Electric Fields}





%%%%%%%%%%%%%%%%%%%%%%%%%%%%%%%%%%%%%%%%%%%%%%
%%%%%%%%%%%%%%%%%%%%%%%%%%%%%%%%%%%%%%%%%%%%%%





\chapter{Atoms in Electromagnetic Fields}











%%%%%%%%%%%%%%%%%%%%%%%%%%%%%%%%%%%%%%%%%%%%%%
%%%%%%%%%%%%%%%%%%%%%%%%%%%%%%%%%%%%%%%%%%%%%%



\chapter{Resonance Line Shapes}






%%%%%%%%%%%%%%%%%%%%%%%%%%%%%%%%%%%%%%%%%%%%%%
%%%%%%%%%%%%%%%%%%%%%%%%%%%%%%%%%%%%%%%%%%%%%%




\chapter{Miscellaneous}

This chapter is dedicated to miscellaneous topics that are typically not ``standard'' in an AMO course. These include experimental techniques, advanced processes, and so on. The topics covered here are added in no particular order. The reader may find the \texttt{Search} function useful when browsing through this chapter. 

\section{Atom interferometry and related topics}

\section{Avoided crossing}

\section{Feshbach resonance}











\bibliographystyle{abbrv} 
\bibliography{HuanBui_AMO_refs}% Produces the bibliography via BibTeX.




	
	
\end{document}