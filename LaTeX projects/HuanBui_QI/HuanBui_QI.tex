\documentclass{book}
\usepackage{physics}
\usepackage{graphicx}
\usepackage{caption}
\usepackage{amsmath}
\usepackage{bm}
\usepackage{authblk}
\usepackage{empheq}
\usepackage{amsfonts}
\usepackage{esint}
\usepackage[makeroom]{cancel}
\usepackage{dsfont}
\usepackage{centernot}
\usepackage{mathtools}
\usepackage{bigints}
\usepackage{amsthm}
\theoremstyle{definition}
\newtheorem{defn}{Definition}[section]
\newtheorem{prop}{Proposition}[section]
\newtheorem{rmk}{Remark}[section]
\newtheorem{thm}{Theorem}[section]
\newtheorem{exmp}{Example}[section]
\newtheorem{prob}{Problem}[section]
\newtheorem{sln}{Solution}[section]
\newtheorem*{prob*}{Problem}
\newtheorem{exer}{Exercise}[section]
\newtheorem*{exer*}{Exercise}
\newtheorem*{sln*}{Solution}
\usepackage{empheq}
\usepackage{hyperref}
\usepackage{tensor}
\usepackage{xcolor}
\hypersetup{
	colorlinks,
	linkcolor={black!50!black},
	citecolor={blue!50!black},
	urlcolor={blue!80!black}
}


\newcommand*\widefbox[1]{\fbox{\hspace{2em}#1\hspace{2em}}}

\newcommand{\p}{\partial}
\newcommand{\R}{\mathbb{R}}
\newcommand{\C}{\mathbb{C}}
\newcommand{\lag}{\mathcal{L}}
\newcommand{\nn}{\nonumber}
\newcommand{\ham}{\mathcal{H}}
\newcommand{\M}{\mathcal{M}}
\newcommand{\I}{\mathcal{I}}
\newcommand{\K}{\mathcal{K}}
\newcommand{\F}{\mathcal{F}}
\newcommand{\w}{\omega}
\newcommand{\lam}{\lambda}
\newcommand{\al}{\alpha}
\newcommand{\be}{\beta}
\newcommand{\x}{\xi}

\newcommand{\G}{\mathcal{G}}

\newcommand{\f}[2]{\frac{#1}{#2}}

\newcommand{\ift}{\infty}

\newcommand{\lp}{\left(}
\newcommand{\rp}{\right)}

\newcommand{\lb}{\left[}
\newcommand{\rb}{\right]}

\newcommand{\lc}{\left\{}
\newcommand{\rc}{\right\}}


\newcommand{\V}{\mathbf{V}}
\newcommand{\U}{\mathcal{U}}
\newcommand{\Id}{\mathcal{I}}
\newcommand{\D}{\mathcal{D}}
\newcommand{\Z}{\mathcal{Z}}

%\setcounter{chapter}{-1}


\makeatletter
\renewcommand{\@chapapp}{Part}
%\renewcommand\thechapter{$\bf{\ket{\arabic{chapter}}}$}
%\renewcommand\thesection{$\bf{\ket{\arabic{section}}}$}
%\renewcommand\thesubsection{$\bf{\ket{\arabic{subsection}}}$}
%\renewcommand\thesubsubsection{$\bf{\ket{\arabic{subsubsection}}}$}
\makeatother



\usepackage{subfig}
\usepackage{listings}
\captionsetup[lstlisting]{margin=0cm,format=hang,font=small,format=plain,labelfont={bf,up},textfont={it}}
\renewcommand*{\lstlistingname}{Code \textcolor{violet}{\textsl{Mathematica}}}
\definecolor{gris245}{RGB}{245,245,245}
\definecolor{olive}{RGB}{50,140,50}
\definecolor{brun}{RGB}{175,100,80}
\lstset{
	tabsize=4,
	frame=single,
	language=mathematica,
	basicstyle=\scriptsize\ttfamily,
	keywordstyle=\color{black},
	backgroundcolor=\color{gris245},
	commentstyle=\color{gray},
	showstringspaces=false,
	emph={
		r1,
		r2,
		epsilon,epsilon_,
		Newton,Newton_
	},emphstyle={\color{olive}},
	emph={[2]
		L,
		CouleurCourbe,
		PotentielEffectif,
		IdCourbe,
		Courbe
	},emphstyle={[2]\color{blue}},
	emph={[3]r,r_,n,n_},emphstyle={[3]\color{magenta}}
}


\begin{document}
\begin{titlepage}\centering
 \clearpage
 \title{{\textsc{\textbf{QUANTUM INFORMATION \& QUANTUM COMPUTATION}}}\\ \smallskip - A Quick Guide - \\}
 \author{\bigskip Huan Q. Bui}
  \affil{Colby College\\$\,$\\ PHYSICS \& MATHEMATICS\\ Statistics \\$\,$\\Class of 2021\\}
 \date{\today}
 \maketitle
 \thispagestyle{empty}
\end{titlepage}

\subsection*{Preface}
\addcontentsline{toc}{subsection}{Preface}

Greetings,\\

This guide is based on \textit{Quantum Computer Science, An Introduction} by N. David Mermin, and \textit{Quantum Computation and Quantum Information} by Isaac Chuang and Michael Nielsen. \\

The entire copy of this text can be found in Chapter 4 of \textit{Quantum Theories, A Quick Guide to}. While this text fits under the more general title of \textit{quantum theories}, the topics covered here are no longer physical phenomena as explained by quantum theories. Rather, we will pay much attention to what happens when computation and information theory meet quantumness. This guide thus deserves its status as a separate set of notes.\\

This text has two parts. Part 1 covers many topics in an introductory manner. These include the ``rules of the game'' and some simple applications and problems. Most of Part 1 will be based on \textit{Quantum Computer Science: An Introduction} by Mermin, even though I might pull some topics from Mike and Ike. Part 2 contains more problems and topics in greater and more advanced details. Most of this part will be based on \textit{Quantum Computation and Quantum Information} by Mike and Ike. Part 1 should serve as an introduction to part 2. \\

I will assume familiarity with linear algebra. There is a section on some potentially unfamiliar linear algebra, but I would like to keep it short.\\

Enjoy!

\newpage
\tableofcontents
\newpage



\chapter{An Introduction}

\newpage


\section{Linear Algebra}

\subsection{Bases \& Linear Independence}
\subsection{Linear Operators \& Matrices}
\subsection{The Pauli Matrices}
\subsection{Inner Products}
\subsection{Eigenvectors \& Eigenvalues}
\subsection{Adjoints \& Hermitian Operators}
\subsection{Tensor Products}
\subsection{Operator Functions}
\subsection{Commutators \& Anti-commutators}
\subsection{Polar \& Singular Value Decomposition}
\newpage
\section{Quantum Mechanics}
\subsection{The Postulates}
\subsection{State space}
\subsection{Evolution}
\subsection{Quantum Measurement}
\subsection{Distinguishing quantum states}
\subsection{Projective Measurements}
\subsection{POVM measurements}
\subsection{Phase}
\subsection{Composite Systems}
\subsection{A global view}
\subsection{Superdense coding}
\newpage
\section{The Density Operator}
\subsection{Ensembles of Quantum States}
\subsection{General Properties of the Density Operator}
\subsection{The Reduced Density Operator}
\subsection{The Schmidt Decomposition \& Purifications}
\subsection{EPR \& the Bell Inequality}

\newpage

\section{The Game}

\subsection{Quantum computer}
\newpage
\subsection{Cbits \& their states}
\newpage
\subsection{Reversible operations on Cbits}
\newpage
\subsection{Manipulating operations on Cbits}
\newpage
\subsection{Qbits \& their states}
\newpage
\subsection{Reversible operations on Qbits}
\newpage
\subsection{Circuit diagrams}
\newpage
\subsection{Measurement gates and the Born rule}
\newpage
\subsection{The generalized Born rule}
\newpage
\subsection{Measurement gates and state preparation}
\newpage
\subsection{Constructing arbitrary 1- and 2-Qbit state}
\newpage
\subsection{Summary}
\newpage







\chapter{Quantum Information \& Quantum Computation}
\newpage



\end{document}