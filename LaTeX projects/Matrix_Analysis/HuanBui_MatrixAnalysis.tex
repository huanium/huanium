\documentclass{article}
\usepackage{physics}
\usepackage{amsmath}
\usepackage{authblk}
\usepackage{amsfonts}
\usepackage{esint}
\usepackage{mathtools}
\usepackage{amsthm}
\theoremstyle{definition}
\newtheorem{defn}{Definition}[section]
\newtheorem{rmk}{Remark}[section]
\newtheorem{exmp}{Example}[section]
\usepackage{empheq}
\usepackage{tensor}


\begin{document}
	\begin{titlepage}\centering
		\clearpage
		\title{\textsc{\bf{MATRIX ANALYSIS}}\\\smallskip A Quick Guide\\}
		\author{\bigskip Huan Bui}
		\affil{Colby College\\Physics \& Statistics\\Class of 2021\\}
		\date{\today}
		\maketitle
		\thispagestyle{empty}
	\end{titlepage}

\newpage

\subsection*{Preface}
\addcontentsline{toc}{subsection}{Preface}

Greetings,\\

\textit{Matrix Analysis, A Quick Guide to} is compiled based on my MA353: Matrix Analysis notes with professor Leo Livshits. The sections are based on a number of resources: \textit{Linear Algebra Done Right} by Axler, \textit{A Second Course in Linear Algebra} by Horn and Garcia, \textit{Matrices and Linear Transformations} by Cullen, \textit{Matrices: Methods and Applications} by Barnett, \textit{Problems and Theorems in Linear Algebra} by Prasolov, \textit{Matrix Operations} by Richard Bronson, and professor Leo Livshits' own textbook (in the making).\\

The development of this text will come in layers. The first layer, one that I am working on during the course of S'19 MA353, will be an overview of the key topics listed in the table of contents. As the semester progresses, I will be constantly updating the existing notes, as well as adding prof. Livshits' problems and my solutions to the problems. The second layer will come after the course is over, when concepts will have hopefully ``come together.''\\ 

Enjoy!


\newpage
\tableofcontents
\newpage

\section{List of Special Matrices \& Their Properties}
\section{List of Operations}
\section{List of Algorithms}

\section{Complex Numbers}
\newpage
\section{Vector Spaces \& Linear Functions}
\subsection{Isomorphisms}
\subsection{Coordinatization \& matricial representation of linear functions}
\newpage
\section{Products of vector spaces \& Sums of subspaces}
\subsection{Nullspaces}
\subsection{Ranges of Operator Powers}
\newpage
\section{Idempotents \& Resolutions of Identity}
\newpage
\section{Block-representations of operators}
\subsection{Direct sums of operators}
\newpage
\section{Invariant subspaces}
\subsection{Reducing subspaces}
\newpage 
\section{Polynomials applied to operators}
\subsection{Minimal polynomials of block-$\Delta^r$ operators}
\subsection{Minimal polynomials at a vector}
\newpage 
\section{Eigentheory}
\subsection{Spectral Mapping Theorem}
\newpage 
\section{Triangularization}
\subsection{Compression to invariant subspaces}
\subsection{Simultaneously $\Delta$-ity of commuting families}
\newpage 
\section{Diagonalization}
\subsection{Spectral resolutions}
\subsection{Compressions to reducing subspaces}
\subsection{Simultaneous diagonalizability for commuting families}
\newpage
\section{Primary decomposition over $\mathbb{C}$ and generalized eigenspaces}
\newpage 
\section{Cyclic decomposition and Jordan form}
\subsection{Square roots of operators}
\subsection{Similarity of a matrix and its transpose}
\subsection{Similarity of a matrix and its conjugate}
\subsection{Jordan forms of $\mathcal{AB}$ and $\mathcal{BA}$}
\subsection{Power-convergent operators}
\subsection{Power-bounded operators}
\subsection{Row-stochastic matrices}
\newpage 
\section{Determinant \& Trace}
\subsection{Classical adjoints}
\subsection{Cayley-Hamilton theorem}


\newpage

\end{document}
