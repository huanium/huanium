\documentclass{beamer}
 
\usepackage[utf8]{inputenc}

\usetheme{Madrid}
\usecolortheme{default}

\usepackage[qm]{qcircuit}
\usepackage{bibentry}













\usepackage{physics}
\usepackage{amsmath}
\usepackage{amsfonts}
\usepackage{esint}
\usepackage{bbold}
\usepackage{mathtools}
\usepackage{dsfont}
\usepackage{amsthm}
\usepackage{bbm}
\usepackage{amssymb}
\theoremstyle{definition}
\newtheorem{defn}{Definition}[section]
\newtheorem{prop}{Properties}[section]
\newtheorem{rmk}{Remark}[section]
\newtheorem{exmp}{Example}[section]
\newtheorem{prob}{Problem}[section]
\newtheorem{sln}{Solution}[section]
\newtheorem{thm}{Theorem}[section]
\newtheorem*{prob*}{Problem}
\newtheorem*{sln*}{Solution}
\usepackage{empheq}
\usepackage{tensor}
\usepackage{hyperref}
\usepackage{xcolor}

\newcommand{\R}{\mathbb{R}}
\newcommand{\F}{\mathcal{F}}
\newcommand{\p}{\partial}

\newcommand{\V}{\mathbf{V}}
\newcommand{\W}{\mathbf{W}}
\newcommand{\Z}{\mathbf{Z}}
\newcommand{\Y}{\mathbf{Y}}
\newcommand{\U}{\mathbf{U}}
\newcommand{\X}{\mathbf{X}}

\newcommand{\A}{\mathcal{A}}

\newcommand{\xpan}{\text{span}}

\newcommand{\lag}{\mathcal{L}}

\newcommand{\J}{\mathbf{J}}

\newcommand{\M}{\mathcal{M}}

\newcommand{\K}{\mathcal{K}}

\newcommand{\N}{\mathcal{N}}

\newcommand{\E}{\mathcal{E}}

\newcommand{\ima}{\text{Im}}
\newcommand{\lin}{\overset{\text{linear}}{\longrightarrow}}
\newcommand{\T}{\mathcal{T}}
\newcommand{\poly}{\mathbb{P}}
\newcommand{\s}{\mathcal{S}}

\newcommand{\gives}{\rotatebox[origin=c]{180}{$\Rsh$}	}


\newcommand{\bigzero}{\mbox{\normalfont\Large\bfseries 0}}
\newcommand{\rvline}{\hspace*{-\arraycolsep}\vline\hspace*{-\arraycolsep}}




 
 
%Information to be included in the title page:
\title{Matrix Theory in a Simple Quantum Adder}
\author[Huan Q. Bui] % (optional)
{Huan Q. Bui}

\institute[Colby College] % (optional)
{
	
	Matrix Analysis
	\and
	Professor Leo Livshits
}
\date{CLAS, May 2, 2019}
 
\logo{\includegraphics[height=0.5cm]{colby.png}}
 
\begin{document}
 
\frame{\titlepage}
 
\begin{frame}
\frametitle{Presentation layout}
\tableofcontents
\end{frame}

\section{Quantum what?}

\begin{frame}
\frametitle{Quantum what?}
Some ideas about quantum mechanics.\\

Bits and qubits. Quantum states. Measurement. Collapsing. Reversible.\\

Quantum computation? Information?

The big picture.
\end{frame}

\begin{frame}
\frametitle{Terminology}

Physics terms and math terms.

\end{frame}











\section{Matrix Theory}

\begin{frame}
\frametitle{Discrete and Quantum Fourier Transform}
Constructing a DFT matrix.
Properties of this matrix.
Orthonormal basis.
\end{frame}

\begin{frame}
\frametitle{Control-phase gate}
Again?
\end{frame}












\section{Simulation on IBM-Q}

\begin{frame}
\frametitle{Simulation on IBM-Q}

\begin{center}
$\,$\Qcircuit @C=.7em @R=.4em  {
	\lstick{\ket{\psi}} & \qw & \qw & \ctrl{1} &
	\gate{H} & \meter & \control \cw\\
	\lstick{\ket{0}} & \qw & \targ & \targ & \qw &
	\meter & \cwx\\
	\lstick{\ket{0}} & \gate{H} & \ctrl{-1} & \qw &
	\qw & \gate{X} \cwx & \gate{Z} \cwx &
	\rstick{\ket{\psi}} \qw
}



\end{center}


\end{frame}















\section{Recap}

\begin{frame}
\frametitle{Recap}
What did we learn on the show tonight, Craig? 

Q-circuit user guide \cite{eastin2004q}

quantum addition of classical numbers \cite{cherkas2016quantum}


Mike and Ike \cite{nielsen2002quantum}

Handbook of Linear Algebra \cite{hogben2006handbook}

addition on quantum computer \cite{draper2000q}

\end{frame}

\begin{frame}
\frametitle{References}

\bibliographystyle{amsalpha}
\bibliography{references}{}



\end{frame}



 
\end{document}
