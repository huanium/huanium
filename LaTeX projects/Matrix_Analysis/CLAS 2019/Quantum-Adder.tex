\documentclass{beamer}
 
\usepackage[utf8]{inputenc}

\usetheme{Madrid}
\usecolortheme{default}

\usepackage[qm]{qcircuit}
\usepackage{bibentry}













\usepackage{physics}
\usepackage{amsmath}
\usepackage{amsfonts}
\usepackage{esint}
\usepackage{bbold}
\usepackage{mathtools}
\usepackage{dsfont}
\usepackage{amsthm}
\usepackage{bbm}
\usepackage{amssymb}
\theoremstyle{definition}
\newtheorem{defn}{Definition}[section]
\newtheorem{prop}{Properties}[section]
\newtheorem{rmk}{Remark}[section]
\newtheorem{exmp}{Example}[section]
\newtheorem{prob}{Problem}[section]
\newtheorem{sln}{Solution}[section]
\newtheorem{thm}{Theorem}[section]
\newtheorem*{prob*}{Problem}
\newtheorem*{sln*}{Solution}
\usepackage{empheq}
\usepackage{tensor}
\usepackage{hyperref}
\usepackage{xcolor}

\newcommand{\R}{\mathbb{R}}
\newcommand{\F}{\mathcal{F}}
\newcommand{\p}{\partial}

\newcommand{\V}{\mathbf{V}}
\newcommand{\W}{\mathbf{W}}
\newcommand{\Z}{\mathbf{Z}}
\newcommand{\Y}{\mathbf{Y}}
\newcommand{\U}{\mathbf{U}}
\newcommand{\X}{\mathbf{X}}

\newcommand{\A}{\mathcal{A}}
\newcommand{\B}{\mathcal{B}}

\newcommand{\xpan}{\text{span}}

\newcommand{\lag}{\mathcal{L}}

\newcommand{\J}{\mathbf{J}}

\newcommand{\M}{\mathcal{M}}

\newcommand{\K}{\mathcal{K}}

\newcommand{\N}{\mathcal{N}}

\newcommand{\E}{\mathcal{E}}

\newcommand{\ima}{\text{Im}}
\newcommand{\lin}{\overset{\text{linear}}{\longrightarrow}}
\newcommand{\T}{\mathcal{T}}
\newcommand{\poly}{\mathbb{P}}
\newcommand{\s}{\mathcal{S}}

\newcommand{\gives}{\rotatebox[origin=c]{180}{$\Rsh$}	}


\newcommand{\bigzero}{\mbox{\normalfont\Large\bfseries 0}}
\newcommand{\rvline}{\hspace*{-\arraycolsep}\vline\hspace*{-\arraycolsep}}




 
 
%Information to be included in the title page:
\title{Matrix Theory in a 2-Qubit Entangler}
\author[Huan Q. Bui] % (optional)
{Huan Q. Bui}

\institute[Colby College] % (optional)
{
	
	Matrix Analysis
	\and
	Professor Leo Livshits
}
\date{CLAS, May 2, 2019}
 
%\logo{\includegraphics[height=0.3cm]{colby.png}}
 
\begin{document}
 
\frame{\titlepage}
 
\begin{frame}
\frametitle{Presentation layout}
\tableofcontents
\end{frame}

\section{Quantum Entanglement}

\begin{frame}
\frametitle{Quantum Bits - Qubits}

\textit{Qubits:}
\begin{align*}
\ket{\psi} = a\ket{0} + b\ket{1},
\end{align*}
where $\abs{a}^2 + \abs{b}^2 = 1$. \\
$\,$\\
\textit{Measurement: } Probabilistic
\begin{align*}
P(\ket{\psi} \to \ket{0}) = \abs{a}^2\hspace{0.5cm}
P(\ket{\psi} \to \ket{1}) = \abs{b}^2
\end{align*}
\end{frame}


\begin{frame}
\frametitle{Entanglement}
When qubits ``coordinate'':


\end{frame}



\begin{frame}
\frametitle{Recipe}

What do we need to entangle two qubits?
\begin{itemize}
	\item Tensor products
	\item Hadamard gate
	\item CNOT gate
	\item Measure
\end{itemize}
\end{frame}








\section{Matrix Theory}

\begin{frame}
\frametitle{Tensor Products}
The \textit{tensor product} of $\V = \mathbb{C}^{\Sigma_1}$ and $\W = \mathbb{C}^{\Sigma_2}$ is
\begin{align*}
\V \otimes \W = \mathbb{C}^{\Sigma_1 \times \Sigma_2}.
\end{align*}
\textit{Elementary tensors} span $\V\otimes \W$. For $\ket{v} \in \V$ and $\ket{w} \in \W$, 
\begin{align*}
\ket{v}\otimes \ket{w}\equiv \ket{v}\ket{w} \equiv \ket{vw}  \in \V \otimes \W.
\end{align*} 

\underline{Example}: Representing the classical number ``1'' with two qubits:
\begin{align*}
1_2 \equiv \ket{01} = \ket{0} \otimes \ket{1} = \begin{bmatrix}
1\\0
\end{bmatrix}\otimes 
\begin{bmatrix}
0\\1
\end{bmatrix}
=
\begin{bmatrix}
1\begin{bmatrix}
0\\1
\end{bmatrix}\\
0\begin{bmatrix}
0\\1
\end{bmatrix}
\end{bmatrix} = \begin{bmatrix}
0\\1\\0\\0
\end{bmatrix}.
\end{align*}


\end{frame}











\begin{frame}
\frametitle{Tensor Products (cont.)}

$\xpan(\ket{00}, \ket{01}, \ket{10}, \ket{11}) = \V\otimes \W$, where
\begin{align*}
\ket{00} = \begin{bmatrix}
1&0&0&0
\end{bmatrix}^\top, \ket{10} = \begin{bmatrix}
0&0&1&0
\end{bmatrix}^\top, \ket{11} = \begin{bmatrix}
0&0&0&1
\end{bmatrix}^\top.
\end{align*}
A \textit{generic state}: For $\abs{a_{00}}^2 + \abs{a_{01}}^2 + \abs{a_{10}}^2 + \abs{a_{11}}^2 = 1$,
\begin{align*}
\ket{\psi} = a_{00}\ket{00} + a_{01}\ket{01} + a_{10}\ket{10} + a_{11}\ket{11}.
\end{align*}
Not every $\ket{\psi} \in \V\otimes \W$ is an elementary tensor. \\
$\,$\\
\underline{Example}: There are no states $\ket{c}, \ket{d}$ such that
\begin{align*}
\ket{c} \otimes \ket{d} = \begin{bmatrix}
\frac{1}{\sqrt{2}} & 0 & 0 & \frac{1}{\sqrt{2}}
\end{bmatrix}^\top \rightarrow \textbf{Entangled}.
\end{align*}

\end{frame}

\begin{frame}
\frametitle{Tensor Products (cont.)}
\textit{Bilinearity}:
\begin{align*}
&\ket{a} \otimes (\alpha\ket{v} + \beta\ket{w}) = \alpha\ket{av} + \beta\ket{aw}\\
&(\alpha\ket{v} + \beta\ket{w}) \otimes \ket{b} = \alpha\ket{vb} + \beta\ket{wb}
\end{align*}
\textit{Of operators:} $\A \in \mathfrak{\lag}(\V), \mathcal{B} \in \mathfrak{\lag}(\W)$, $\A\otimes \B \in \mathfrak{\lag}(\V \otimes \W)$ is defined by
\begin{align*}
(\A \otimes \B)(\ket{v}\otimes \ket{w}) = (\A\ket{v})\otimes (\B\ket{w}).
\end{align*}
But not all $C \in \mathfrak{\lag}(\V\otimes \W)$ can be written as $\A \otimes \B$, $\A \in \mathfrak{\lag}(\V), \mathcal{B} \in \mathfrak{\lag}(\W)$\\ $\rightarrow$ \textbf{Entangled}.
\end{frame}


\begin{frame}
\frametitle{Example: Entanglement}
\begin{center}
	$\,$\Qcircuit @C=.7em @R=.4em  {
		\lstick{a: \ket{0}} & \qw & \qw & \targ & \meter & \qw \\
		\lstick{b: \ket{0}} & \qw & \gate{H} & \ctrl{-1}& \meter & \qw 
	}
\end{center}
\begin{align*}
H\begin{bmatrix}
1\\0
\end{bmatrix}_b = \frac{1}{\sqrt{2}}\begin{bmatrix}
1&1\\1&-1
\end{bmatrix}\begin{bmatrix}
1\\0
\end{bmatrix}_b = \frac{1}{\sqrt{2}}\ket{0}_b + \frac{1}{\sqrt{2}}\ket{1}_b
\end{align*}
\begin{align*}
CNOT_b = C_b = \begin{bmatrix}
1&0&0&0\\
0&0&0&1\\
0&0&1&0\\
0&1&0&0
\end{bmatrix}
\end{align*}
$\rightarrow$ Unitary Operations
\end{frame}


\begin{frame}
\frametitle{Example: Entanglement (cont.)}
\begin{center}
	$\,$\Qcircuit @C=.7em @R=.4em  {
		\lstick{a: \ket{0}} & \qw & \qw & \targ & \meter & \qw \\
		\lstick{b: \ket{0}} & \qw & \gate{H} & \ctrl{-1}& \meter & \qw 
	}
\end{center}
\begin{align*}
C_b(I \otimes H)\left(\begin{bmatrix}
1\\0
\end{bmatrix}_a
\otimes
\begin{bmatrix}
1\\0
\end{bmatrix}_b
\right)
&= C_b\left(\begin{bmatrix}
1&0\\0&1
\end{bmatrix}\begin{bmatrix}
1\\0
\end{bmatrix}_a\otimes \frac{1}{\sqrt{2}}\begin{bmatrix}
1&1\\1&-1
\end{bmatrix}\begin{bmatrix}
1\\0
\end{bmatrix}_b\right)\\
&=
\begin{bmatrix}
1&0&0&0\\
0&0&0&1\\
0&0&1&0\\
0&1&0&0
\end{bmatrix}
\begin{bmatrix}
1/\sqrt{2}\\1/\sqrt{2}\\0\\0
\end{bmatrix}
= 
\begin{bmatrix}
1/\sqrt{2}\\0\\0\\1/\sqrt{2}
\end{bmatrix}\\ &= \frac{1}{\sqrt{2}}\ket{0}\otimes\ket{0} + \frac{1}{\sqrt{2}}\ket{1}\otimes\ket{1} \rightarrow \textbf{Entangled}
\end{align*}
\end{frame}






\begin{frame}
\frametitle{Tensor Products (cont.)}
Other properties:
\begin{itemize}
	\item Associative
	\item Distributive
	\item Not commutative
	\item $(\A \otimes \B)^\dagger = \A^\dagger \otimes \B^\dagger$.
	\item $\Tr(\A\otimes \B) = \Tr(\A)\cdot \Tr(\B)$.
	\item $\det(\A \otimes \B) = (\det(\A))^m\cdot \det(\B)^n$, where $m$ is the dimension of $\A$ and $n$ of $\B$. 
\end{itemize}
\end{frame}




\begin{frame}
\frametitle{Unitary Operations}
\begin{itemize}
	\item Quantum Fourier Transform
	\item Control-phase
	\item 
\end{itemize}
\end{frame}

\begin{frame}
\frametitle{Unitary Operations: QFT}

\end{frame}












\section{Simulation on IBM-Q}

\begin{frame}
\frametitle{Simulation on IBM-Q}

A sample quantum circuit.

\begin{center}
$\,$\Qcircuit @C=.7em @R=.4em  {
	\lstick{\ket{\psi}} & \qw & \qw & \ctrl{1} &
	\gate{H} & \meter & \control \cw\\
	\lstick{\ket{0}} & \qw & \targ & \targ & \qw &
	\meter & \cwx\\
	\lstick{\ket{0}} & \gate{H} & \ctrl{-1} & \qw &
	\qw & \gate{X} \cwx & \gate{Z} \cwx &
	\rstick{\ket{\psi}} \qw
}



\end{center}


\end{frame}















\section{Recap}

\begin{frame}
\frametitle{Recap}
What did we learn on the show tonight, Craig? 

Q-circuit user guide \cite{eastin2004q}

quantum addition of classical numbers \cite{cherkas2016quantum}


Mike and Ike \cite{nielsen2002quantum}

Handbook of Linear Algebra \cite{hogben2006handbook}

addition on quantum computer \cite{draper2000q}

QFT quick math \cite{baconQFT}

Matrix analysis (where I read about unitary matrices) \cite{horn1990matrix}

\end{frame}

\begin{frame}
\frametitle{References}

\bibliographystyle{amsalpha}
\bibliography{references}{}



\end{frame}



 
\end{document}
