\documentclass[11pt]{article}
%\usepackage{newpxtext,newpxmath}
\usepackage[left=1in,right=1in,top=1in,bottom=1in]{geometry}
\usepackage{graphicx, amsmath, amsthm, latexsym, amssymb, color,cite,enumerate, physics, framed}
\usepackage{caption,subcaption, empheq, hyperref}
\usepackage{mathtools}
\pagenumbering{arabic}
\newtheorem{theorem}{Theorem}[section]
\newtheorem{lemma}[theorem]{Lemma}
\newtheorem{definition}[theorem]{Definition}
\newtheorem{corollary}[theorem]{Corollary}
\newtheorem{proposition}[theorem]{Proposition}
\newtheorem{convention}[theorem]{Convention}
\newtheorem{conjecture}[theorem]{Conjecture}
\newtheorem{remark}{Remark}
\newtheorem{example}{Example}
\newtheorem{exercise}{Exercise}
\newcommand*{\myproofname}{Proof}
\newenvironment{subproof}[1][\myproofname]{\begin{proof}[#1]\renewcommand*{\qedsymbol}{$\mathbin{/\mkern-6mu/}$}}{\end{proof}}

\newcommand{\R}{\mathbb{R}}
\newcommand{\N}{\mathbb{N}}
\newcommand{\F}{\mathcal{F}}
\newcommand{\X}{\mathcal{X}}
\newcommand{\lp}{\left(}
\newcommand{\rp}{\right)}
\newcommand{\lb}{\left[}
\newcommand{\rb}{\right]}
\newcommand{\lc}{\left\{}
\newcommand{\rc}{\right\}}
\newcommand{\p}{\partial}
\newcommand{\f}[2]{\frac{#1}{#2}}




\begin{document}
\begin{center}
\begin{framed}
{\Large  MA439: Functional Analysis\\
	 Tychonoff Spaces:  Exercises 5, 6, 12, 13, 14 on p.31, Ben Mathes}\\
$\,$\\
{\Large \bf  Huan Q. Bui\\}
$\,$\\
{\Large Due: Wed, Sep 23, 2020}
\end{framed}
\end{center}

\begin{exercise}[Ex 5, p.31]
	In an arbitrary topological space $\X$ , we say that a sequence $(x_i)$ converges to $x$ (and write $x_i \to x$) if, for every open set $G$ containing $x$, the sequence
	$(x_i)$ is eventually in $G$. Prove that $x_i \to x$ if and only if, for every subbasic open set $S$ \textcolor{blue}{containing $x$}, $(x_i)$ is eventually in $S$.
	\begin{proof}
		($\implies$) Suppose that $x_i \to x$ in $\X$. Let some subbasic open set $S$ be given. Because $S$ is subbasic open, it is open. Thus, because $x_i \to x$, $(x_i)$ is eventually in $S$.
		
		($\impliedby$) Let a sequence $(x_i)$ be given. Suppose that for every subbasic open set $S \subseteq \X$ containing $x$, $(x_i)$ is eventually in $S$. And, let $O_x$ be an open set containing $x$. It follows that there is some $k \in \N_+$ for which $x\in \cap^k_{i=1}S_i \subseteq O_x$, where each $S_i$ is a subbasic open set containing $x$. From here, it is clear that there is some positive integer $N$ for which $x_n \in S_i$ for all $i = 1,\dots,k$ whenever $n \geq N$. Thus, $(x_i)$ is eventually in $\cap^k_{i=1}S_i \subseteq O_x$. So, $x_i \to x$. 
 	\end{proof}
\end{exercise}

\begin{exercise}[Ex 6, p.31]
	In arbitrary topological spaces, the neighborhood filter $\F_x$ of a point $x$ is
	defined to be the collection of all subsets that contain an open set containing
	$x$, and we again define $\F \to x$ to mean $\F_x \subseteq \F$. Prove that $F \to x$ if and only if every subbasic open set containing $x$ is in $\F$.
	\begin{proof}
		($\implies$) Suppose that $\F \to x$. Let $S$ be a subbasic open set containing $x$. $S$ necessarily contains an open subset containing $x$, so $S\in \F_x \subseteq \F$.
		
		($\impliedby$) Consider the neighborhood filter $\F_x$ and some $F\in \F_x$. $F$ contains an open set containing $x$. Thus, there is some $k \in \N_+$ for which $F\supseteq \cap^k_{i=1}S_i$ where $S_i$'s are subbasic open sets. Now, because each $S_i \in \F$, $\cap^k_{i=1}S_i \in \F$ (since $\F$ is a filter). So, $F \in \F$. Thus, $\F_x \subseteq \F$, i.e., $\F \to x$. 
	\end{proof}
\end{exercise}

\begin{exercise}[Ex 12, p.31]
	Assume that $S = p^{-1}_k(G)$ is a subbasic open set in a product space $\prod_i \X_i$. Prove that $S = p^{-1}_k(p_k(S))$, and if $p_k(E) \subseteq p_k(S)$, then $E \subseteq S$.
	\begin{proof}
		When $S = p_k^{-1}(G) = \dots \times G \times \dots$, we have that $p_k(S) = p_k(p_k^{-1}(G)) = G$. So $S = p_k^{-1}(G) = p_k^{-1}(p_k(S))$. Next, assume that $p_k(E) \subseteq p_k(S)$, then because $p_k$ is a projection, $p_k^{-1}(p_k(E)) \subseteq p_k^{-1}(p_k(S)) = S$. Also, because $p_k$ is a projection, $E\subseteq p_k^{-1}(p_k(E))$. So $E\subseteq S$.  
	\end{proof}
\end{exercise}

\begin{exercise}[Ex 13, p.31]
	Prove that a topological space is compact if and only if every open covering by basic open sets has a finite subcover.
	\begin{proof}
		($\implies$) Let $(\X,\tau)$ be a topological space. Assume that $\X$ is compact, then every open covering has a finite subcover.  In particular, every open covering with basic open sets has a finite subcover. 
		
		($\impliedby$) Let $(\X,\tau)$ be a topological space. Let a base $\mathcal{B}$ and an open covering $\mathcal{C}$ be given. For each $x\in \X$, pick $O_x \subseteq \mathcal{C}$ such that $x\in O_x$ and pick $B_x \in \mathcal{B}$ for which $x\in B_x \subseteq O_x$. Assume that every open covering with basic open sets has a finite subcover, then the collection $\mathcal{C}_\mathcal{B} = \{ B_x : x\in \X\}$ has a finite subcover $\{B_{x_1},\dots, B_{x_N} \}$. Consequently the collection $\{ O_{x_1},\dots,O_{x_N} \}$ is a finite subcover in $\mathcal{C}$. Thus, $\mathcal{C}$ has a finite subcover.   
	\end{proof}
\end{exercise}

\begin{exercise}[Ex 14, p.31, \textcolor{red}{Alexander's Subbase Theorem}]
	Prove that a topological space is compact if and only if every open covering
	by subbasic open sets has a finite subcover. (This requires the axiom of
	choice.)

	\begin{proof}
		($\implies$) Let $(\X,\tau)$ be a topological space. Assume that $\X$ is compact, then every open covering has a finite subcover.  In particular, every open covering with subbasic open sets has a finite subcover. 
		
		($\impliedby$) Let $(\X,\tau)$ be a topological space. Let a subbase $\mathcal{B}$ and an open covering $\mathcal{C}$ be given. Assume that $\X$ is not compact yet every subbasic cover from $\mathcal{B}$ has a finite subcover. By the axiom of choice, choose an open cover $\mathcal{C}$ without a finite subcover is that \textbf{maximal}. Observe that $\mathcal{B} \cap \mathcal{C}$ cannot cover $\X$, since otherwise there would be a finite subcover coming from $\mathcal{B}$. With this, pick an $x\in \X\setminus \bigcup (\mathcal{B} \cap \mathcal{C})$. Since $\mathcal{C}$ is a cover, choose an $O_x \in \mathcal{C}$ such that $x\in O_x$. Since $\mathcal{B}$ is a subbase, there are $\{B_1,\dots, B_k \} \subseteq \mathcal{B}$ for which $x\in \cap^k_{i=1}B_i \subseteq O_x$. Now, by the choice of $x$, $B_i \notin \mathcal{B}\cap \mathcal{C}$ for all $i=1,2,\dots,k$. 
		
		Since $\mathcal{C}$ is the maximal open cover for which there is no finite subcover, the collection $\mathcal{C}\cup \{ B_i \}$ must have a finite subcover for each $i=1,2,\dots,k$. Thus,l let this be $\{ C_1^i, C_2^i, \dots , C_{n^i}^i\} \cup \{B_i\}$. It follows that $\{ C_1^i, C_2^i, \dots , C_{n^i}^i\}_{i=1}^n \cup \{\cup^n_{i=1} B_i\}$ is a finite open covering of $\X$. As a result, $\{ C_1^i, C_2^i, \dots , C_{n^i}^i\}_{i=1}^n \cup \{ U\}$ is also a finite cover for $\X$. But notice that $U\in \mathcal{C}$, so this finite cover is made up entirely of elements of $\mathcal{C}$. This is a contradiction. So, $\X$ must be compact. \footnote{Source: \href{https://people.clas.ufl.edu/kees/files/AlexanderTychonoff.pdf}{James Keesling, Dept. of Mathematics, Univ. of Florida}}
	\end{proof}
\end{exercise}


\end{document}




