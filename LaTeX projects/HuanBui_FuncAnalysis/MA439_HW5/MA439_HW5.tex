\documentclass[11pt]{article}
%\usepackage{newpxtext,newpxmath}
\usepackage[left=1in,right=1in,top=1in,bottom=1in]{geometry}
\usepackage{graphicx, amsmath, amsthm, latexsym, amssymb, color,cite,enumerate, physics, framed}
\usepackage{caption,subcaption, empheq, hyperref}
\usepackage{mathtools}
\pagenumbering{arabic}
\newtheorem{theorem}{Theorem}[section]
\newtheorem{lemma}[theorem]{Lemma}
\newtheorem{definition}[theorem]{Definition}
\newtheorem{corollary}[theorem]{Corollary}
\newtheorem{proposition}[theorem]{Proposition}
\newtheorem{convention}[theorem]{Convention}
\newtheorem{conjecture}[theorem]{Conjecture}
\newtheorem{remark}{Remark}
\newtheorem{example}{Example}
\newtheorem{exercise}{Exercise}
\newcommand*{\myproofname}{Proof}
\newenvironment{subproof}[1][\myproofname]{\begin{proof}[#1]\renewcommand*{\qedsymbol}{$\mathbin{/\mkern-6mu/}$}}{\end{proof}}

\newcommand{\R}{\mathbb{R}}
\newcommand{\N}{\mathbb{N}}
\newcommand{\F}{\mathcal{F}}
\newcommand{\X}{\mathcal{X}}
\newcommand{\lp}{\left(}
\newcommand{\rp}{\right)}
\newcommand{\lb}{\left[}
\newcommand{\rb}{\right]}
\newcommand{\lc}{\left\{}
\newcommand{\rc}{\right\}}
\newcommand{\p}{\partial}
\newcommand{\f}[2]{\frac{#1}{#2}}




\begin{document}
\begin{center}
\begin{framed}
{\Large  MA439: Functional Analysis\\
	 Tychonoff Spaces:  Exercises 1-6 on p.36, Ben Mathes}\\
$\,$\\
{\Large \bf  Huan Q. Bui\\}
$\,$\\
{\Large Due: Wed, Oct 7, 2020}
\end{framed}
\end{center}

\begin{exercise}[Ex 1, p.36]
	Let $\X$ be a topological space. Prove that if $d$ is a continuous pseudometric, then the sets $\{ y \in \X : d(x,y) > \delta \}$ are open, where $x\in \X$ and $\delta \in \R$.  
	\begin{proof}
		Let $O = \{ y \in \X : d(x,y) > \delta \}$. We want to show that each $y\in O$ is an interior point of $O$. Let $y\in O$ be given, then $d(x,y) > \delta$. This means that $d(x,y) \geq \delta + \epsilon$ for some $\epsilon > 0$. $d$ is a continuous pseudometric, so every $d$-ball is an open subset of $\X$. In particular, $B_d(y,\epsilon/2)$ is an open subset of $\X$. By the triangle inequality, for any $z\in B_d(y,\epsilon/2)$, $z\in O$. Thus, $B_d(y,\epsilon/2) \subseteq O$. So, $O$ is open as desired.  
 	\end{proof}
\end{exercise}

\begin{exercise}[Ex 2, p.36]
	Let $\X$ be a topological space. Prove that $d$ is a continuous pseudometric on $\X$ if and only if the function $f_x^d = d(x,\cdot)$ is continuous for every $x\in \X$.
	\begin{proof}
		$(\implies)$ Suppose that $d$ is a continuous pseudometric on $\X$. Let $\epsilon > 0$ and $x\in \X$. $f^d_x$ is continuous at $y\in \X$ if and only if for every $\epsilon > 0 \exists f(y) \in G\subseteq \X$ open for which $\abs{f_x^d(y) - f^d_x(y')} < \epsilon$ whenever $y'\in G$. Note that $\abs{f^d_x(y) - f^d_x(y')} = \abs{d(x,y) - d(x,y')} \leq d(y,y')$. So, we just take $G = B_d(y,\epsilon)$. 
		
		$(\impliedby)$ Let $d$ be a pseudometric and suppose that $f_x^d = d(x,\cdot)$ is continuous for every $x\in \X$. We want to show that every $d$-ball is open in $\X$. To this end, let $x\in \X$ and $\delta > 0$ be given and consider $B_d(x,\delta) = \{ y\in \X : d(x,y) < \delta  \} = \{ y\in \X : f_x^d(y) < \delta \} = \{ y\in \X : f_x^d(y) \in (-\delta,\delta) \}$ which is open by continuity of $f_x^d$. So we're done.  
	\end{proof}
\end{exercise}

\begin{exercise}[Ex 3, p.36]
	Let $\X$ be a Tychonoff space whose topology is generated by the family of pseudometrics $\mathcal{G}$. Prove that the topology on $\X$ is the same as the weak topology induced by the family of functions $f_x^d$ where $x\in \X$, $d\in \mathcal{G}$.
	\begin{proof}
		One inclusion is trivial. It remains to show the other inclusion. A topological space is Tychonoff means that for every closed set $F\subseteq \X$ and every $x\in F$, there exists a continuous function $f: \X \to \R$ for which $f[F] = \{ 0 \}$ and $f(x) = 1$. From $\mathcal{G}$, we use open balls as a subbase and build the topology from those balls. Alternatively, we can build the functions $\{ f^d_x : x\in \X, d\in \mathcal{G} \}$ and build the (open-ball) topology by taking inverse images of open sets. From the previous exercise, we have that the weak topology $\implies$ $f^d_x$ are all continuous, which implies that all balls are open relative to the weak topology, which implies that the new (open-ball) topology is contained in the weak topology. Since the weak topology is by definition \textit{weak}, this open-ball topology must be the weak topology itself. 
	\end{proof}
\end{exercise}

\begin{exercise}[{Ex 4, p.36}]
	Assume $\X$ is a Tychonoff space with generating family $\mathcal{G}$. If $E$ is a subset of $\X$, let $\mathcal{G}_E$ denote the set of restrictions of elements of $\mathcal{G}$ to $E$. Prove that the resulting Tychonoff Topology on $E$ generated by the family $\mathcal{G}_E$ is the same as the topological \textbf{subspace topology} that $E$ inherits from the topology on $\X$. 
	\begin{proof}
	 (Ideas) Get base from finite intersection of balls. $G$ open iff for every $x\in G$ there exist finitely many $d_1,\dots, d_k \in \mathcal{G}$ and $\epsilon_1,\dots, \epsilon_k > 0$ such that $\cap^k_{i=1}B_{d_i}(x,\epsilon_i) \subseteq G$. Try: Let $\tau$ denote the topology on $\X$. The subspace topology on $E$ is given by $\tau_E = \{E\cap U : U\in \tau \}$. $\boxed{?}$
	\end{proof}
\end{exercise}

\begin{exercise}[Ex 5, p.36]
	Give an example of a continuous pseudometric on $(0, 1)$ that is not the restriction of a continuous pseudometric on $\R$ to $(0, 1)$. 
	\begin{proof}
		Consider the continuous function $f(x) = 1/x$ defined on $(0,1)$. This function induces a continuous pseudometric $d(x,y) = \abs{f(x) - f(y)} = \abs{1/x - 1/y}$ on $(0,1)$ since $d$-balls are open. Now, this cannot be a restriction of a continuous pseudometric on $\R$ to $(0,1)$ because $d(x,y)$ is undefined when $x$ or $y=0$. 
	\end{proof}
\end{exercise}

\begin{exercise}[\textcolor{blue}{Ex 6, p.36}]
	Prove that a bounded continuous pseudometric on $(0, 1)$ is the restriction of a continuous pseudometric on $\R$ to $(0, 1)$. (?CHECK?)
	\begin{proof}
		\textcolor{blue}{Ben said he found a counter-example to this?}
	\end{proof}
\end{exercise}

\begin{exercise}[Ex 7, p.36]
	If $d_1$ and $d_2$ are continuous relative to a topology on $\X$, prove that $d_1 + d_2$ is continuous also.
	\begin{proof}
		We want to show that any $(d_1+d_2)$-ball is open. To this end, let $x\in \X, \epsilon > 0$ and consider $B_{d_1 + d_2}(x,\epsilon) = \{ y \in \X : d_1(x,y) + d_2(x,y) < \epsilon \} = \{ y \in \X : d_1(x,y) \leq \delta \land d_2(x,y) \leq \epsilon - \delta : \forall \delta \in [0,\epsilon) \}$. We can write this set as
		\begin{equation*}
		B_{d_1 + d_2}(x,\epsilon) = \bigcup_{\delta\in [0,\epsilon)} \lb B_{d_1}(x,\delta) \cap B_{d_2}(x,\epsilon - \delta) \rb.
		\end{equation*} 
		Since $d_1,d_2$ are both continuous, any intersection between a $d_1$ ball and a $d_2$ ball is open. It follows that any arbitrary union of these balls is also open. So $d_1 + d_2$ is continuous. 
	\end{proof}

\end{exercise}
\begin{exercise}[Ex 8, p.36]
	Assume that the topology on $\X$ is generated by the family of pseudometrics $\mathcal{G}$, and let $\mathcal{G}'$ be the set of all finite sums of elements of $\mathcal{G}$. Show that the set
	of $d$-balls with $d \in \mathcal{G}'$ forms a base for the topology.
	\begin{proof}
		Let $d_1, d_2\in \mathcal{G'}$ be given. Assume to avoid triviality that $B_{d_1}(x,\epsilon_1)\cap B_{d_2}(y,\epsilon_2) \neq \varnothing$. Let $z\in B_{d_1}(x,\epsilon_1)\cap B_{d_2}(y,\epsilon_2)$. We want to show that there is some $d\in \mathcal{G'}$ and $\epsilon > 0$ such that $B_d(z,\epsilon)\subseteq B_d \subseteq B_{d_1}(x,\epsilon_1) \cap B_{d_2}(y,\epsilon_2)$ (the ball $B_d(z,\epsilon)$ obviously contains $z$, so these two conditions make the collection of $d$-ball a base for $\X$). Now, let $\epsilon = \min\{\epsilon_1,\epsilon_2\} - \max\{d_1(x,z), d_2(z,y)\}$ and $d = d_1 + d_2$, which is in $\mathcal{G}'$. For any $u\in B_d(z,\epsilon)$, we have
		\begin{equation*}
		d(u,z) = d_1(u,z) + d_2(u,z) < \epsilon = \min\{\epsilon_1,\epsilon_2\} - \max\{d_1(x,z), d_2(z,y)\} 
		\end{equation*}
		which implies that
		\begin{equation*}
		\begin{cases}
		d_1(u,x) < d_1(u,z) + d_1(z,x) + d_2(u,z) < \min\{\epsilon_1,\epsilon_2\}    \\
		d_2(u,y) < d_2(u,z) + d_2(z,y) + d_1(u,z) < \min\{\epsilon_1,\epsilon_2\} 
		\end{cases}
		\end{equation*}
		so $u\in B_{d_1}(x,\epsilon_1) \cap B_{d_2}(y,\epsilon_2) $. Thus, $B_d(z,\epsilon) \subseteq B_{d_1}(x,\epsilon_1) \cap B_{d_2}(y,\epsilon_2)$ as desired. So the collection of $d$-balls where $d\in \mathcal{G}'$ forms a base the given topology.
	\end{proof}
\end{exercise}



\begin{exercise}[Ex 9, p.36]
	Two pseudometrics are \textbf{topologically equivalent} if they give rise to the same open sets. Prove that two pseudometrics are topologically equivalent if and only if each is continuous relative to the topology generated by the other.
	\begin{proof}
		The forward direction is automatic by definition. It remains to show the converse. Let pseudometrics $d_1,d_2$ be given such that $d_1$ is continuous relative to the topology $\tau_2$ generated by $d_2$ and $d_2$ is continuous relative to the topology $\tau_1$ generated by $d_1$. By continuity, for any $x\in \X$ and  $\epsilon>0$,  $B_{d_1}(x,\epsilon)$ is $d_2$-open and $B_{d_2}(x,\epsilon)$ is $d_1$-open. Let $O_1$ be an open set generated by $d_1$. Then $O_1$ is some union of $d_1$-balls. But since each $d_1$-open ball is open in $d_2$, each of these balls is generated by $d_2$-balls. By symmetry, we see that, $d_1,d_2$ must generate the same open sets. 
	\end{proof}
\end{exercise}



\begin{exercise}[Ex 10, p.36]
	Assume $d$ is a pseudometric on a set $\X$ and $d(x, y) = 0$ for some $x, y \in \X$. Prove that $d(x,z) = d(y,z)$ for all $z\in \X$.
	\begin{proof}
		By the triangle inequality: $|d(x,z) - d(y,z)| \leq d(x,y) = 0 \quad \forall z\in \X$. So, $\abs{d(x,z) - d(y,z)} = 0$ for all $z\in \X$. Thus, $d(x,z) = d(y,z)$ for all $z\in \X$ as desired. 
	\end{proof}
\end{exercise}


\begin{exercise}[Ex 11, p.36]
	Assume $d$ is a pseudometric on $\X$, and define a relation by $x \sim y$ if and only if $d(x, y) = 0$. Verify that this defines an equivalence relation on $\X$, and show that the quotient topology on the quotient space is metrizable.
	\begin{proof}
		We first check that $\sim$ is an equivalence relation on $\X$:
		\begin{itemize}
			\item Symmetry follows automatically since $d$ is a pseudometric.
			\item Reflexivity follows because $d(x,x) = 0$ for all $x\in \X$
			\item Transitivity: follows from the previous exercise. 
		\end{itemize}
		Thus, $\sim$ is an equivalence relation on $\X$. To prove that $\X/\sim$ is metrizable, we want to show that the open sets in $\X/\sim$ are generated by a single metric. Consider the following function $\mathfrak{d}: \X/\sim \times \X/\sim \to [0,\infty)$ defined by 
		\begin{equation*}
		\mathfrak{d}([x],[y]) = d(x,y).
		\end{equation*}
		for $x,y\in \X$ (and of course $[x],[y]\in \X/\sim$). It is clear that this is a metric because not only it inherits properties of the pseudometric $d$ but also it satisfies the property that $\mathfrak{d}([x],[y]) = d(x,y) =  0 \iff x \sim y \iff [x] = [y]$. We also know that open sets of $\X/\sim$ are the subsets of $\X/\sim$ that have an open pre-image under the surjective map $q: x \to [x]$. As a result, because $d$-balls in $\X$ are open, we have that $\mathfrak{d}$-balls in $\X/\sim$ are also open. Putting the results together, we find that $\X/\sim$ is metrizable, as desired.
	\end{proof}
\end{exercise}
\begin{exercise}[{Ex 12, p.36}]
	A topological space $\X$ is called \textbf{Hausdorff} if every pair of distinct points in $\X$ are contained in disjoint open subsets of $\X$. Prove that every Tychonoff space is Hausdorff.

	\begin{proof}
		Let a Tychonoff space $\X$ be given. By definition, the topology of $\X$ is the weak topology generated by the $d$-balls of a separating family of pseudometrics. From here, it is clear that for any two distinct points $x,y$  in $\X$, there is always some pseudometric $d$ in the family for which $d(x,y) = \delta > 0$. Consider the open balls $B_d(x,\delta/4)$ and $B_d(y,\delta/4)$. Assume that some point $u\in \X$ is in the intersection, then $\delta d(x,y) \leq d(x,u) + d(u,y) < \delta/2$, which is a contradiction. So, these open balls cannot intersect. Therefore, $\X$ is Hausdorff.
	\end{proof}
\end{exercise}
\begin{exercise}[{Ex 14, p.36}]
	A topological space is \textbf{completely regular} if every pair consisting of a closed set and a point not in that set can be separated with a continuous function. Prove that every Tychonoff space is completely regular.
	\begin{proof}
		Let a $A\subseteq \X$ be closed and $x\in \X\setminus A$ be given. It suffices to define some function $f$ that separates $A$ and $x$. Choose $d$ a pseudometric generating $\X$. Define $\delta_{A,d}(x): \X \to [0,\infty)$ by $\delta_{A,d}(z) = \inf_{a\in A}d(z,a)$. By the triangle inequality property inherited from the pseudometric $d$, we can check that $\delta_{A,d}$ is continuous. Further, we see that $\delta_{A,d}(A) = 0$ and $\delta_{A,d}(x) \neq 0$ (since $A$ is closed). Thus, $\X$ must be completely regular. 
	\end{proof}
\end{exercise}



\end{document}




