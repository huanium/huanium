\documentclass[11pt]{article}
%\usepackage{newpxtext,newpxmath}
\usepackage[left=1in,right=1in,top=1in,bottom=1in]{geometry}
\usepackage{graphicx, amsmath, amsthm, latexsym, amssymb, color,cite,enumerate, physics, framed}
\usepackage{caption,subcaption, empheq, hyperref}
\usepackage{mathtools}
\pagenumbering{arabic}
\newtheorem{theorem}{Theorem}[section]
\newtheorem{lemma}[theorem]{Lemma}
\newtheorem{definition}[theorem]{Definition}
\newtheorem{corollary}[theorem]{Corollary}
\newtheorem{proposition}[theorem]{Proposition}
\newtheorem{convention}[theorem]{Convention}
\newtheorem{conjecture}[theorem]{Conjecture}
\newtheorem{remark}{Remark}
\newtheorem{example}{Example}
\newtheorem{exercise}{Exercise}
\newcommand*{\myproofname}{Proof}
\newenvironment{subproof}[1][\myproofname]{\begin{proof}[#1]\renewcommand*{\qedsymbol}{$\mathbin{/\mkern-6mu/}$}}{\end{proof}}

\newcommand{\R}{\mathbb{R}}
\newcommand{\N}{\mathbb{N}}
\newcommand{\F}{\mathcal{F}}
\newcommand{\X}{\mathcal{X}}
\newcommand{\lp}{\left(}
\newcommand{\rp}{\right)}
\newcommand{\lb}{\left[}
\newcommand{\rb}{\right]}
\newcommand{\lc}{\left\{}
\newcommand{\rc}{\right\}}
\newcommand{\p}{\partial}
\newcommand{\f}[2]{\frac{#1}{#2}}




\begin{document}
\begin{center}
\begin{framed}
{\Large  MA439: Functional Analysis\\
	 Tychonoff Spaces:  Exercises 1-6 on p.36, Ben Mathes}\\
$\,$\\
{\Large \bf  Huan Q. Bui\\}
$\,$\\
{\Large Due: Wed, Sep 30, 2020}
\end{framed}
\end{center}

\begin{exercise}[Ex 1, p.36]
	Let $\X$ be a topological space. Prove that if $d$ is a continuous pseudometric, then the sets $\{ y \in \X : d(x,y) > \delta \}$ are open, where $x\in \X$ and $\delta \in \R$.  
	\begin{proof}
		Let $O = \{ y \in \X : d(x,y) > \delta \}$. We want to show that each $y\in O$ is an interior point of $O$. Let $y\in O$ be given, then $d(x,y) > \delta$. This means that $d(x,y) \geq \delta + \epsilon$ for some $\epsilon > 0$. $d$ is a continuous pseudometric, so every $d$-ball is an open subset of $\X$. In particular, $B_d(y,\epsilon/2)$ is an open subset of $\X$. By the triangle inequality, for any $z\in B_d(y,\epsilon/2)$, $z\in O$. Thus, $B_d(y,\epsilon/2) \subseteq O$. So, $O$ is open as desired.  
 	\end{proof}
\end{exercise}

\begin{exercise}[Ex 2, p.36]
	Let $\X$ be a topological space. Prove that $d$ is a continuous pseudometric on $\X$ if and only if the function $f_x^d = d(x,\cdot)$ is continuous for every $x\in \X$.
	\begin{proof}
		$(\implies)$ Suppose that $d$ is a continuous pseudometric on $\X$. Let $\epsilon > 0$ and $x\in \X$. $f^d_x$ is continuous at $y\in \X$ if and only if for every $\epsilon > 0 \exists f(y) \in G\subseteq \X$ open for which $\abs{f_x^d(y) - f^d_x(y')} < \epsilon$ whenever $y'\in G$. Note that $\abs{f^d_x(y) - f^d_x(y')} = \abs{d(x,y) - d(x,y')} \leq d(y,y')$. So, we just take $G = B_d(y,\epsilon)$. 
		
		$(\impliedby)$ Let $d$ be a pseudometric and suppose that $f_x^d = d(x,\cdot)$ is continuous for every $x\in \X$. We want to show that every $d$-ball is open in $\X$. To this end, let $x\in \X$ and $\delta > 0$ be given and consider $B_d(x,\delta) = \{ y\in \X : d(x,y) < \delta  \} = \{ y\in \X : f_x^d(y) < \delta \} = \{ y\in \X : f_x^d(y) \in (-\delta,\delta) \}$ which is open by continuity of $f_x^d$. So we're done.  
	\end{proof}
\end{exercise}

\begin{exercise}[Ex 3, p.36]
	Let $\X$ be a Tychonoff space whose topology is generated by the family of pseudometrics $\mathcal{G}$. Prove that the topology on $\X$ is the same as the weak topology induced by the family of functions $f_x^d$ where $x\in \X$, $d\in \mathcal{G}$.\footnote{completely regular $\equiv$ Tychonoff}
	\begin{proof}
		One inclusion is trivial. It remains to show the other inclusion. Tychonoff: for every closed set $F\subseteq \X$ and every $x\in F$, there exists a continuous function $f: \X \to \R$ for which $f[F] = \{ 0 \}$ and $f(x) = 1$. From $\mathcal{G}$, use balls as a subbase and build the topology from those balls. Alternatively, we can build the functions $\{ f^d_x : x\in \X, d\in \mathcal{G} \}$ and build the (open-ball) topology by taking inverse images of open sets. From the previous exercise, we have that weak topology $\implies$ $f^d_x$ are all continuous, which implies that all balls are open relative to the weak topology, which implies that the new (open ball) topology is contained in the weak topology. 
	\end{proof}
\end{exercise}

\begin{exercise}[Ex 4, p.36]
	Assume $\X$ is a Tychonoff space with generating family $\mathcal{G}$. If $E$ is a subset of $\X$, let $\mathcal{G}_E$ denote the set of restrictions of elements of $\mathcal{G}$ to $E$. Prove that the resulting Tychonoff Topology on $E$ generated by the family $\mathcal{G}_E$ is the same as the topological \textbf{subspace topology} that $E$ inherits from the topology on $\X$. 
	\begin{proof}
		get base from finite intersection of balls. $G$ open iff for every $x\in G$ there exist finitely many $d_1,\dots, d_k \in \mathcal{G}$ and $\epsilon_1,\dots, \epsilon_k > 0$ such that $\cap^k_{i=1}B_{d_i}(x,\epsilon_i) \subseteq G$.  
	\end{proof}
\end{exercise}

\begin{exercise}[Ex 5, p.36]
	Give an example of a continuous pseudometric on $(0, 1)$ that is not the restriction of a continuous pseudometric on $\R$ to $(0, 1)$. 
	\begin{proof}
		blah
	\end{proof}
\end{exercise}

\begin{exercise}[Ex 6, p.36]
	Prove that a bounded continuous pseudometric on $(0, 1)$ is the restriction of
	a continuous pseudometric on $\R$ to $(0, 1)$. (?CHECK?)

	\begin{proof}
		blah
	\end{proof}
\end{exercise}



\end{document}




