\documentclass[11pt]{article}
%\usepackage{newpxtext,newpxmath}
\usepackage[left=1in,right=1in,top=1in,bottom=1in]{geometry}
\usepackage{graphicx, amsmath, amsthm, latexsym, amssymb, color,cite,enumerate, physics, framed}
\usepackage{caption,subcaption, empheq, hyperref}
\usepackage{mathtools}
\pagenumbering{arabic}
\newtheorem{theorem}{Theorem}[section]
\newtheorem{lemma}[theorem]{Lemma}
\newtheorem{definition}[theorem]{Definition}
\newtheorem{corollary}[theorem]{Corollary}
\newtheorem{proposition}[theorem]{Proposition}
\newtheorem{convention}[theorem]{Convention}
\newtheorem{conjecture}[theorem]{Conjecture}
\newtheorem{remark}{Remark}
\newtheorem{example}{Example}
\newtheorem{exercise}{Exercise}
\newcommand*{\myproofname}{Proof}
\newenvironment{subproof}[1][\myproofname]{\begin{proof}[#1]\renewcommand*{\qedsymbol}{$\mathbin{/\mkern-6mu/}$}}{\end{proof}}

\newcommand{\R}{\mathbb{R}}
\newcommand{\N}{\mathbb{N}}
\newcommand{\F}{\mathcal{F}}
\newcommand{\X}{\mathcal{X}}
\newcommand{\lp}{\left(}
\newcommand{\rp}{\right)}
\newcommand{\lb}{\left[}
\newcommand{\rb}{\right]}
\newcommand{\lc}{\left\{}
\newcommand{\rc}{\right\}}
\newcommand{\p}{\partial}
\newcommand{\f}[2]{\frac{#1}{#2}}




\begin{document}
\begin{center}
\begin{framed}
{\Large  MA439: Functional Analysis\\
	 Tychonoff Spaces: Extra Problem \& 1, 2, 5, 7 pg. 51, Ben Mathes}\\
$\,$\\
{\Large \bf  Huan Q. Bui\\}
$\,$\\
{\Large Due: Wed, Oct 28, 2020}
\end{framed}
\end{center}



\begin{exercise}
	$C(X) = \{ f: X\to \mathbb{C} : f \text{ unif. cont., bdd} \}$ and uniform norm $|| f || = \sup_{x\in X} \abs{f(x)}$. Consider $B(X) = \{ f: X\to \mathbb{C}, \text{ bdd} \}$. Show that $C(X) \subseteq B(X)$ is closed, i.e. a uniform limit of uniformly continuous function is uniformly continuous.
	\begin{proof}
  		Let us prove the ``easier'' case on metric spaces first. Suppose that $f_n \to f$ uniformly where $\{f_n\}$ is a sequence of uniformly continuous functions. We claim that $f$ must also be uniformly continuous. To see this, let $\epsilon > 0$. We first have that 
		\begin{equation*}
		||f_n(x), f(x)|| < \f{\epsilon}{3}
		\end{equation*}
		for all $x$ whenever $n$ is sufficiently large. Now, each $f_n$ is uniformly continuous, so there is a $\delta$ for which $d(x,y) < \delta$ implies $|f_n(x) - f_n(y)| < \f{\epsilon}{3}$. Finally, consider $x,y\in X$ such that $d(x,y) < \delta$, then 
		\begin{equation*}
		||f(x) - f(y)|| \leq ||f(x) - f_n(x)|| + ||f_n(x) - f_n(y)|| + ||f_n(y) - f(y)|| < \epsilon.
		\end{equation*}
		so $f$ is uniformly continuous. This implies that the space $C(X)$ of all uniformly continuous functions from $X$ to $\mathbb{C}$ is a closed subset of $B(X)$. 
	\end{proof}
\end{exercise}

\begin{exercise}[Ex. 1, pg. 51]
	 Prove that a closed subset of a complete uniform space is complete
	 \begin{proof}
	 	Let a $C$ be a closed subset of $(\X,\mathcal{U})$ a complete uniform space be given. Consider a Cauchy net $\{ x_i \}$ in $C$. Since $C\subseteq X$, $\{ x_i\}$ is also a Cauchy net in $\X$. Thus, $\{ x_i\}$ converges because $\X$ is complete. This limit must belong to the closed set $C$, so $C$ is complete. 
	 \end{proof}
\end{exercise}

\begin{exercise}[Ex. 2, pg. 51]
	If $\F$ is a Cauchy filter and $\F \subseteq \F_0$, prove that $\F_0$ is a Cauchy.

	\begin{proof}
		Let $\F$ be a Cauchy filter in $(\X,\mathcal{U})$ and $\F \subseteq \F_0$. It is clear that $\F_0$ is also a filter. Now, let $\epsilon > 0$ and $d\in \mathcal{U}$ be given. $\F$ is Cauchy, so there exists an $x\in \X$ for which $B_d(x,\epsilon) \in \F \subset \F_0$. So $\F_0$ is also Cauchy. 
	\end{proof}
\end{exercise}

\begin{exercise}[Ex. 5, pg. 51]
	An element $x$ of a Tychonoff space is a cluster point of a net $\{x_i\}$ if the net is frequently in every neighborhood of $x$. Prove that a Cauchy net converges to any of its cluster points.
	\begin{proof}
		Let a Cauchy net $\{ x_i\}_{i\in I}$ be given. Consider a cluster point $x\in \{ x_i\}_{i\in I}$. We want to show that $\{x_i\}_{i\in I} \to x$. By the hypothesis, any neighborhood $\mathcal{U}$ containing $x$ must also contain infinitely many elements of $\{x_i\}_{i\in I}$. The Cauchyness of $\{x_i\}_{i\in I}$ guarantees that $\mathcal{U}$ contains a tail of $\{x_i\}_{i\in I}$. This implies the convergence of $\{x_i\}_{i\in I}$.  
	\end{proof}
\end{exercise}

\begin{exercise}[Ex. 7, pg. 51]
	Any filter $\F$ is a directed set, and if, for $F \in \F$ we choose $x_F \in F$, we obtain a \textbf{net based on the filter} (there are many of them). Prove that the filter converges to $x$ if and only if the net $\{x_F\}_{F\in \F}$ converges to $x$.
	\begin{proof}
		$(\implies)$ Suppose that a filter $\F\to x$. Consider a net based on $\F$ denoted by $\{x_F\}_{F\in \F}$. $\F \to x$ iff $\F_x \subseteq \F$. Consider a neighborhood $F_x \in \F_x$ of $x$. We have that $F_x$ contains a tail of $\{x_F\}_{F\in \F}$. To see this, fix an $F_x$. Because any $F$ in $\F$ meets $F_x$, $F' = F\cap F_x \subseteq F_x \in \F$. It is now clear that $\{ x_{F}\}_{F\in \F}$ for all $F \geq F'$ is contained in $F_x$. Therefore, $\{x_F\}_{F\in \F} \to x$.  
		
		$(\impliedby)$ Let a filter $\F$ and $\{x_F\}_{F\in \F} \to x$ be given. Assume to get a contradiction that $\F \not\to x$. This means that there is some set $O\subseteq \X$ containing an open set $F_x \ni x$ such that $O \notin \F$. This means that $F_x \notin \F$. Now, consider the net $\{ y_F\}_{F\in \F}$ which does not intersect $F_x$. Clearly, this net cannot converge to $x$, which contradicts the fact that all nets $\{x_F\}_{F\in \F} \to x$.
	\end{proof}
\end{exercise}


%\begin{exercise}[Ex. 8, pg. 51]
%	If $\{x_F\}_{F \in \F} $ is a net based on the filter $\F$, prove that $\{x_F\}_{F \in \F} $ is a Cauchy net if and only if $\F$ is a Cauchy filter.

%\end{exercise}



\end{document}




