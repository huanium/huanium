\documentclass[11pt]{article}
%\usepackage{newpxtext,newpxmath}
\usepackage[left=1in,right=1in,top=1in,bottom=1in]{geometry}
\usepackage{graphicx, amsmath, amsthm, latexsym, amssymb, color,cite,enumerate, physics, framed}
\usepackage{caption,subcaption, empheq, hyperref}
\usepackage{mathtools}
\pagenumbering{arabic}
\newtheorem{theorem}{Theorem}[section]
\newtheorem{lemma}[theorem]{Lemma}
\newtheorem{definition}[theorem]{Definition}
\newtheorem{corollary}[theorem]{Corollary}
\newtheorem{proposition}[theorem]{Proposition}
\newtheorem{convention}[theorem]{Convention}
\newtheorem{conjecture}[theorem]{Conjecture}
\newtheorem{remark}{Remark}
\newtheorem{example}{Example}
\newtheorem{exercise}{Exercise}
\newcommand*{\myproofname}{Proof}
\newenvironment{subproof}[1][\myproofname]{\begin{proof}[#1]\renewcommand*{\qedsymbol}{$\mathbin{/\mkern-6mu/}$}}{\end{proof}}

\newcommand{\R}{\mathbb{R}}
\newcommand{\N}{\mathbb{N}}
\newcommand{\F}{\mathcal{F}}
\newcommand{\X}{\mathcal{X}}
\newcommand{\lp}{\left(}
\newcommand{\rp}{\right)}
\newcommand{\lb}{\left[}
\newcommand{\rb}{\right]}
\newcommand{\lc}{\left\{}
\newcommand{\rc}{\right\}}
\newcommand{\p}{\partial}
\newcommand{\f}[2]{\frac{#1}{#2}}




\begin{document}
\begin{center}
\begin{framed}
{\Large  MA439: Functional Analysis\\
	 Tychonoff Spaces: 1, 2, 5, 7 pg. 51, Ben Mathes}\\
$\,$\\
{\Large \bf  Huan Q. Bui\\}
$\,$\\
{\Large Due: Wed, Oct 28, 2020}
\end{framed}
\end{center}



\begin{exercise}
	$C(X) = \{ f: X\to \mathbb{C} : f \text{ unif. cont., bdd} \}$ and uniform norm $|| f || = \sup_{x\in X} \abs{f(x)}$. Consider $B(X) = \{ f: X\to \mathbb{C}, \text{ bdd} \}$ Show that $C(X) \subseteq B(X)$ is closed, i.e. a uniform limit of unif. cont. fn is unif. cont.
	\begin{proof}
		This is the full generality. To make this easier, prove this example: consider a metric space $(X,d)$ and $f_n: X\to \mathbb{C}$ bdd, unif. cont. fns. and $|| f_n -f || \to 0$ uniformly where $|| h || = \sup_{x\in \X}\abs{h(x)}$. This implies that $f$ is unif. cont.  
	\end{proof}
\end{exercise}

\begin{exercise}[Ex. 8]
	
\end{exercise}



\end{document}




