\documentclass[11pt]{article}
%\usepackage{newpxtext,newpxmath}
\usepackage[left=1in,right=1in,top=1in,bottom=1in]{geometry}
\usepackage{graphicx, amsmath, amsthm, latexsym, amssymb, color,cite,enumerate, physics, framed}
\usepackage{caption,subcaption, empheq}
\usepackage{mathtools}
\pagenumbering{arabic}
\newtheorem{theorem}{Theorem}[section]
\newtheorem{lemma}[theorem]{Lemma}
\newtheorem{definition}[theorem]{Definition}
\newtheorem{corollary}[theorem]{Corollary}
\newtheorem{proposition}[theorem]{Proposition}
\newtheorem{convention}[theorem]{Convention}
\newtheorem{conjecture}[theorem]{Conjecture}
\newtheorem{remark}{Remark}
\newtheorem{example}{Example}
\newtheorem{exercise}{Exercise}
\newcommand*{\myproofname}{Proof}
\newenvironment{subproof}[1][\myproofname]{\begin{proof}[#1]\renewcommand*{\qedsymbol}{$\mathbin{/\mkern-6mu/}$}}{\end{proof}}

\newcommand{\R}{\mathbb{R}}
\newcommand{\N}{\mathbb{N}}
\newcommand{\F}{\mathcal{F}}
\newcommand{\X}{\mathcal{X}}
\newcommand{\lp}{\left(}
\newcommand{\rp}{\right)}
\newcommand{\lb}{\left[}
\newcommand{\rb}{\right]}
\newcommand{\lc}{\left\{}
\newcommand{\rc}{\right\}}
\newcommand{\p}{\partial}
\newcommand{\f}[2]{\frac{#1}{#2}}




\begin{document}
\begin{center}
\begin{framed}
{\Large  MA439: Functional Analysis\\
	 Tychonoff Spaces:  Exercises 10-16 on p.25-26, Ben Mathes}\\
$\,$\\
{\Large \bf  Huan Q. Bui\\}
$\,$\\
{\Large Due: Wed, Sep 16, 2020}
\end{framed}
\end{center}

\begin{exercise}[Ex 10 p.25]
Given a sequence $(x_i)$, one constructs the family $\mathcal{F}$ that consists of all sets that contain a tail of the sequence. (A \textbf{tail} of a sequence is a set of the form $\{x_i : i \geq n \}$ for some $n \in \mathbb{N}$). Prove that $\mathcal{F}$ is a filter, and $\mathcal{F} \to x$ if and only if $x_i \to x$.
	\begin{proof}
	$(\implies)$ By construction, it is clear that $\F$ is non-empty. Further, take $F_1,F_2\in \F$. $F_1\cap F_2$ is a set that contains a tail of $(x_i)$, thus belongs to $\F$. Finally, If $G\supseteq F$ where $F\in \F$, then $G$ is a set that contains a tail of $(x_i)$, so $G\in \F$. Thus, $\F$ is a filter. 
	
	Now, suppose $\F \to x$. By definition, this means $\F_x \subseteq \F$, where $\F_x$ is a family of sets defined by $\F_x = \{ H : \exists \epsilon > 0, B_d(x,\epsilon) \subseteq H \}$. It follows that for all $\epsilon > 0$, $B_d(x,\epsilon) \in \F$, and thus $B_d(x,\epsilon)$ always contains a tail of $(x_i)$. As a result, there is some $N \in \mathbb{N}$ for which $\abs{x_n - x} < \epsilon$ whenever $n \geq N$. So, $x_i \to x$.
	
	$(\impliedby)$ Conversely, suppose $\F \not\to x$. Then $\F_x \not\subseteq \F$, i.e., there exists some set $H\in \F_x$ but $H \not\in \F$. So, $H$ does not contain a tail of $(x_i)$. If follows that there is some $\epsilon > 0$ for which $B_d(x,\epsilon) \subseteq H$ but $B_d(x,\epsilon)$ does not contain a tail of $(x_i)$. This means that $\abs{x_n - x} \geq \epsilon$ for all $n\in \N$, which implies that $x_i \not\to x$.
	\end{proof}
\end{exercise}

\begin{exercise}[Ex 11 p.25]
If $\mathcal{F}$ is a filter in a metric space $(\X , d)$, we say that $\F$ is a \textbf{Cauchy filter} if $\F$ contains balls of all radii; i.e. if $\epsilon > 0$ is given, then there exists $x \in \X$ such that $B(x, \epsilon) \in \F.$ Prove that, if $\F$ is a Cauchy filter in a compact space $(\X , d)$, then $\F$ converges.

	\begin{proof}
	Assume that $(\X,d)$ is compact. In view of Theorem 17.4 of Willard's \textit{General Topology}, we have that each filter in $\X$ has an accumulation point. Let $\F$ be a Cauchy filter in $\X$. We claim that if $x$ is an accumulation point for $\F$ then $\F \to x$. By definition, $x$ is an accumulation point for $\F$ if and only if each $F\in \F$ meets each $U\in \mathcal{U}_x$, where $\mathcal{U}_x$ is the set of all neighborhoods of $x$ (and hence a filter in $\X$). In view of Definition 12.3, we want to show that $\mathcal{U}_x \subseteq \F$. 
	
	By definition, $\mathcal{U}_x$ is a collection of sets each containing an open set containing $x$. Let $U\in \mathcal{U}_x$, then there is an open set $O \subseteq U$ such that $x\in O$. Thus, $U$ contains $B_d(x,\epsilon)$ for some $\epsilon$. Now, $B_d(x,\epsilon) \in \F$ by definition, so $U\in \F$. Thus, we say that if $\F$ is a Cauchy filter in a compact metric space $(\X,d)$ then $\F$ converges.  
	\end{proof}
\end{exercise}

\begin{exercise}[Ex 12 p.25]
	Prove that a metric space $(\X , d)$ is totally bounded if and only if every filter is contained in a Cauchy filter. 
	\begin{proof}
		$(\implies)$ Suppose that the metric space $(\X,d)$ is totally bounded. Since every filter $\F$ is contained in an ultrafilter $\mathcal{U}$, it suffices to show that $\mathcal{U}$ is Cauchy. Let $\epsilon > 0$ be given. By definition, any open over $C_\epsilon$ for $\X$ has a finite subcover, so that $\X = \bigcup_{i=1}^N B_d(x_i,\epsilon)$.  Take $U\in \mathcal{U}$. Then 
		\begin{equation*}
		\mathcal{U}\ni U = U\cap \X = U \cap \bigcup^N_{i=1} B_d(x_i,\epsilon) = (U \cap B_d(x_1,\epsilon)) \cup \dots \cup (U \cap B_d(x_N,\epsilon)) \neq \varnothing.
		\end{equation*}
		This means that $B_d(x_i, \epsilon) \supseteq (U \cap B_d(x_i,\epsilon)) \neq \varnothing$ for some $i$. Since $(U\cap B_d(x_i,\epsilon)) \subseteq U \in \mathcal{U}$, we have that  $B_d(x_i,\epsilon) \in \mathcal{U}$. Since this holds for all $\epsilon >0$, $\mathcal{U}$ is Cauchy. 
		
		
		$(\impliedby)$ Assume that every filter is contained in a Cauchy filter. We want to show that $(\X,d)$ is totally bounded, i.e., that every sequence as a Cauchy subsequence\footnote{\textcolor{red}{I need a reference here. The statement can be found in various course notes online, but weirdly neither Rudin, Willard, Folland, nor Wade contains it -- It's also possible that I overlooked it.}}. Let a sequence $(x_i)$ be given. In view of Exercise 1, consider the \textbf{tail filter} $\F$ for this sequence. From our hypothesis, $\F$ is contained in a Cauchy filter $\mathcal{U}$. Thus, for any $n > 0$, there is some $x'_n$ for which $B_d(x'_n,1/n)\in \mathcal{U}$. Since $\mathcal{U}$ is a filter containing $\F$, $B_d(x'_n,1/n)$ must also contain some $x_{i_n}$ in the sequence $(x_i)$. From here, it is clear that $(x_{i_n})$ is a Cauchy, hence is a Cauchy subsequence of $(x_i)$. Thus, $(\X,d)$ is totally bounded.
	\end{proof}
\end{exercise}

\begin{exercise}[Ex 13 p.25]
	Prove that a metric space is complete if and only if every Cauchy filter converges.
	\begin{proof}
		$(\implies)$ Assuming that $\X$ is complete, or equivalently that every Cauchy sequence converges. Let a Cauchy filter $\F$ be given. Then for every $i\in N_+$, there exists an $ x_i$ for which $B_d(x_i,1/i)\in \F$. We claim that the sequence $(x_i)$ is Cauchy and thus converges to some $x\in \X$ (since $\X$ is complete). It is clear that because $\F$ is a filter, $B(x_i, 1/i) \cap B(x_j,1/j) \neq \varnothing$, and so $d(x_i, x_j) \leq 2/j$ (assuming $j < i$). It follows that for any $\epsilon > 0$, there is a sufficiently large $N\in \N_+$ for which $d(x_i,x_j) < \epsilon$ whenever $i,j \geq N$. Thus, $(x_i)$ is Cauchy, so $x_i \to x$ for some $x\in \X$. It remains to show that $\F \to x$. To this end, we want to show that $\F_x \subseteq \F$, i.e., that $\F$ contains all balls centered at $x$ of all radii. Since $(x_i) \to x$ for which $\F \ni B_d(x_i,1/i) \subseteq B_d(x,\delta)$. So $B_d(x,\delta)\in \F$. Therefore, $\F_x \subseteq  \F$, which implies that $\F\to x$.   	
		
		$(\impliedby)$ Conversely, suppose that every Cauchy filter converges. Let a Cauchy sequence $(x_i)$ be given. We want to show that $x_i\to x$ for some $x$. In view of Exercise 1, we consider the \textbf{tail filter} $\F$ for the sequence $(x_i)$. We claim that $\F$ is a Cauchy filter. Let $\epsilon > 0$ be given. By the Cauchyness of the sequence $(x_i)$, there exists an $i\in \N_+$ for which the ball $B_d(x_i,1/i) \in \F$ is contained in the ball $B_d(x_i,\epsilon)$. Since $\F$ is a filter, it must follow that $B_d(x_i,\epsilon) \in \F$ as well. Thus, $\F$ is a Cauchy filter. By the hypothesis, $\F \to x$ for some $x\in \X$, so we have that $\F_x\subseteq \F$. This implies that $\F$ contains all balls centered at $x$ of any radii. In particular, let $\delta > 0$ be given, then $B_d(x,\delta)\cap B(x_i,1/i) \neq \varnothing$ for all $i\in \N_+$ since $\F$ is a filter. This means that there exists $0 < \delta' < \delta$ and sufficiently large $i$ for which $d(x_i,x) \leq \delta+ 1/i < \delta$. So, $x_i \to x$.  
 	\end{proof}
\end{exercise}

\begin{exercise}[Ex 14 p.25]
	Prove that a metric space is compact if and only if every ultrafilter converges.
	(When passing from metric spaces to the arbitrary topological spaces introduced in the next section, the equivalent conditions presented in Theorem 7 are no longer equivalent! The equivalence of the first two conditions can be recovered within the setting of uniform spaces, as we will see in the next chapter, but the third condition becomes a new property, and gives rise to the definition of a \textit{sequentially compact} space. We wish to note that the ultrafilter statement in this problem remains equivalent to compactness in general topological spaces.)
	\begin{proof}
		$(\implies)$ Suppose than an ultrafilter $\F$ does not converge to any $x\in \X$. This implies that for each $x\in \X$, there is an neighborhood $N_x$ that contains no element of $\F$. Consider the open cover $\{ N_x : x\in \X \}$ for $\X$. By compactness, there is a finite subcover $\{ N_{x_1},\dots, N_{x_k} \}$. Let $F\in \F$ be given. Then we have
		\begin{equation*}
		\F \ni F\cap \X = F \cap \lp N_{x_1} \cup \dots \cup N_{x_k} \rp = (F \cap N_{x_1}) \cup \dots \cup (F\cap N_{x_k}) .
		\end{equation*}
		So, $F\cap N_{x_i} \in F$ for some $i$. But $N_{x_i} \not\in \F$, so this is a contradiction. Thus, $\F$ must converge to some $x\in \X$.
		
		$(\impliedby)$ Conversely, assume that $(\X,d)$ is not compact. Let $\{N_i : i\in I \}$ be a cover with no finite subcover. It follows that for a finite subset $\{ x_1,x_2,\dots, x_n\} \subseteq \X$, 
		\begin{equation*}
		(N_{x_1} \cup N_{x_2} \cup \dots \cup N_{x_n})^c = N_{x_1}^c \cap \dots \cap N_{x_n}^c \neq \varnothing,
		\end{equation*} 
		Now, generate a filter $\F'$ from the collection $\{ N_{x_i}^c : i \leq n \}$. Extend $\F'$ to an ultrafilter $\F$ of $\X$. Now assume that $\F \to x$. Choose some $i\leq n$ and $x\in N_{x_i}$. Since $\F \to x$, we can find $F \in \F$ such that $vF \subseteq N_{x_i}$. This implies that $F \cap N_{x_i}^c = \varnothing$. But $N_{x_i}^c \in \F$. Thus, $F, N_{x_i}^c \in \F$ but their intersection is empty. So, this must be a contradiction. Thus, $\F$ does not converge. 
	\end{proof}
\end{exercise}

\begin{exercise}[Ex 15 p.26]
	Show that a closed subset $S$ of a complete metric space $(\X,d)$ is complete.
	\begin{proof}
		Let $(s_i)$ be a Cauchy sequence in $S$. Since $S\subseteq \X$, $(s_i)$ converges to some $s\in \X$. Since $s$ is a limit point of $S$, $s\in \overline{S}$. But of course, $\overline{S} =S$ (since $S$ is closed), so $s\in S$. So $S$ is complete. 
	\end{proof}
\end{exercise}

\begin{exercise}[Ex 16 p.26]
	Recall that $B(\X)$ denotes the set of all bounded complex valued functions
	on $\X$ with the norm $\norm{f}_\infty = \sup \{|f(x)| : x \in \X \}$. Prove that $B(\X)$ is a complete metric space.

	\begin{proof}
		Let $(f_i)$ be a Cauchy sequence in $B(\X)$. We want to show that $(f_i)$ converges in $B(\X)$. Since $(f_i)$ is Cauchy, for every $\epsilon > 0$, there is an $N\in \N_+$ for which 
		\begin{equation*}
		\abs{f_n(x) - f_m(x)} \leq \sup\{ \abs{f_n(x) - f_m(x)} : x\in \X  \} = \norm{f_n - f_m}_\infty  < \epsilon, \quad \mbox{for all } x\in \X
		\end{equation*}
		whenever $n,m \geq N$. Thus, for each $x\in \X$, the sequence $(f_i(x)) \subseteq \mathbb{C}$ is Cauchy. Since $\mathbb{C}$ is complete, $(f_i(x))$ converges for all $x\in \X$. Let $f(x) = \lim_{i\to \infty} f_i(x)$. We also see that $f$ is bounded, because the Cauchy sequence $(f_i(x)) \to f(x)$ is bounded for all $x\in \X$. Thus, $f\in B(\X)$. It follows that for $n > N$,
		\begin{equation*}
		\abs{f_n(x) - f_(x)} = \lim_{m\to \infty}\abs{f_n(x) - f_m(x)} < \epsilon.
		\end{equation*} 
		Thus, 
		\begin{equation*}
		\norm{f_n - f}_\infty = \sup\{ \abs{f_n(x) - f(x)} : x\in \X \} < \epsilon.
		\end{equation*}
		Thus, $f_i \to f$ in $B(\X)$. So, $B(\X)$ is complete.
		
		
	\end{proof}
\end{exercise}




\end{document}




