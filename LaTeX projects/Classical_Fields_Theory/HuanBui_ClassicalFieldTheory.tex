\documentclass[a4paper,11pt]{article}
\usepackage{physics}
\usepackage{amsmath}
\numberwithin{equation}{section}
\usepackage{siunitx}
\usepackage{braket}
\usepackage{authblk}
\usepackage[a4paper, margin=1.5in]{geometry}
\usepackage{parskip}
\setlength{\parindent}{15pt}
\usepackage{tikz}
\usetikzlibrary{arrows, calc, patterns, angles, quotes}
\usetikzlibrary{decorations.pathmorphing}
\usepackage{amssymb}
\usetikzlibrary{shapes.geometric}
\usepackage[section]{placeins}
\usepackage{pgfplots}
\usepackage{clock}
\usepackage[clock]{ifsym}
\usetikzlibrary{shapes}
%\usepgfplotslibrary{external}
%\tikzexternalize

\begin{document}
\begin{titlepage}\centering
 \clearpage
 \title{\textsc{\bf{CLASSICAL FIELD THEORY}}\\\smallskip A Quick Guide\\}
 \author{\bigskip Huan Q. Bui}
 \affil{Colby College\\Physics \& Mathematical Sciences: Statistics \\Class of 2021\\}
 \date{\today}
 \maketitle
 \thispagestyle{empty}
\end{titlepage}

\newpage

\subsection*{Preface}
\addcontentsline{toc}{subsection}{Preface}

Greetings,\\

\textit{Classical Field Theory, A Quick Guide to} is compiled based on my independent study PH492: Topics in Classical Field Theory notes with professor Robert Bluhm. \\

This text is a continuation of \textit{General Relativity and Cosmology, A Quick Guide to}. Basic knowledge of general relativity, special relativity, linear algebra, and vector calculus is assumed.\\ 

Enjoy!


\newpage
\tableofcontents
\newpage

\section{???}

\end{document}
