\documentclass{report}
\usepackage{physics}
\usepackage{amsmath}
\usepackage{authblk}
\usepackage{amsfonts}
\usepackage{esint}
\usepackage{mathtools}
\usepackage{amsthm}
\theoremstyle{definition}
\newtheorem{defn}{Definition}[section]
\newtheorem{rmk}{Remark}[section]
\newtheorem{exmp}{Example}[section]
\newtheorem{sln}{Solution}[section]
\newtheorem{prop}{Proposition}[section]
\newtheorem{thm}{Theorem}[section]

\setcounter{chapter}{-1}

\usepackage{empheq}
\usepackage{tensor}

\begin{document}
	\begin{titlepage}\centering
		\clearpage
		\title{\textsc{\bf{Summer Assistantship\\
			at the\\
		JOINT QUANTUM INSTITUTE}}\\\smallskip NIST \& University of Maryland, College Park\\}
		\author{\bigskip Huan Q. Bui}
		 \affil{Colby College\\$\,$\\ PHYSICS \& MATHEMATICS\\ Statistics \\$\,$\\Class of 2021\\}
		\date{\today}
		\maketitle
		\thispagestyle{empty}
	\end{titlepage}

\newpage

\subsection*{Preface}
\addcontentsline{toc}{subsection}{Preface}

Greetings,\\

This journal is about my appointment at the Joint Quantum Institute at College Park in Maryland up to June 4, 2019. This journey truly deserves a proper, detailed, documentation not only for my own learning and research purposes but also for all twists and turns along the way.  \\

Enjoy, I guess?


\newpage
\tableofcontents
\newpage

\chapter{Week $-\infty$: Think JQI}

\section{Pre\textendash JQI}

\subsection{Conover Lab \& Summer 2018}

\subsection{October, 2018}


\subsection{February, 2018}

\subsection{OPT application, 2018: The wait begins}






\section{Towards JQI}

\subsection{May, 2019: Absolute chaos}

\subsection{DAMOP19 in Milwaukee, Wisconsin}

\subsection{Heading to JQI, May 31\textendash June 2, 2019}


\chapter{Week 1: A peek of JQI}

\section{Monday, June 3, 2019}

\begin{enumerate}
	\item I couldn't start working yet but professor Steven Rolston was kind enough to let me tour one of his labs and meet Hyok Sang Han, a post-doc I would be assisting over the summer.  
	\item I decided to start a \LaTeX project on Quantum Mechanics, based on my reading of Shankar's Quantum Mechanics book. 
\end{enumerate}








\section{Tuesday, June 4, 2019}

\begin{enumerate}
	\item Met Prof. Rolston again. Prof. Rolston introduced me to the post-doc I would be working with: Hyok Sang Han and his lab.
	\item Hyok Sang Han and prof. Rolston gave me a little bit of a tour of the lab and the nanofiber experiment Hyok has been working on. 
	\item Prof. Rolston then left. Hyok and I hung out for a bit more. He bought me lunch at Pho D'Lite nearby. We then talked for a bit about ourselves and what brought us to JQI.
	\item I of course explained in detail I still couldn't work but I would be more than happy to look at papers and past theses.
	\item Later that day Hyok sent me three PhD theses to read before I could actually work. 
	\item I got an OPT update. The congresswoman must have successfully contact USCIS about my case. The update sent by USCIS was very vague, and there was nothing I could make out of the message they sent me. 
	\item My friend got his OPT approved.
\end{enumerate}





\section{Wednesday, June 5, 2019}
\begin{enumerate}
	\item Hyok told me of a weekly journal club meeting where everybody often get together to talk about papers they have been reading - exactly what the name of the club suggests. I told Hyok I might join the next Wednesday.
	\item I began reading the PhD theses. I hope to be able to summarize some of them somewhere in this journal.
	\item I noticed that my case got updated, based on what the USCIS website said. But I had no idea what it was.  
	\item Later in the day I asked Hyok if I could come to the lab the next day to look at what he is current working on and if there was something I could do for the summer. He said I could come and look at the demonstration of the nanofiber pulling process (he was going to demonstrate this process to another post-doc), and told me to come to his lab after that to discuss some possible summer projects. 
\end{enumerate}



\section{Thursday, June 6, 2019}
	\begin{enumerate}
	\item I came into the lab at around 2 when Hyok was heading over the lab where the fiber-pulling process takes place. I met Matt, the other post-doc. We both watched Hyok operate the fiber-pulling system. The fiber pulling was unsuccessful (the fiber broke, which is what can happen quite often) but it a good learning experience. The system definitely worked, although as I could tell it is not fully automatic. A lot of care is needed when operating.
	
	
	\item Hyok and I then headed back to the lab where we discussed what Hyok has been up to. Hyok told me he was trying to recreate of the earlier experiments to make sure his apparatus worked. He then explained to me some of the physics involved and showed me a few papers (that I decided that I will read - to get to the bottom of things). 
	
	
	\item So according to Hyok there are maybe 2-3 things I can try to do over the summer. The first thing is to help him with setting up the pulse laser setup that is used for the selective excitation of the MOT cloud. This might not be an ideal thing for me to work on, because there would be some overlap between me and him. The second thing I could do was setting up the lasers that will be used for a \textbf{dipole trap} (Hyok also explained to me what a dipole trap is). The third I could do, which would be very important to JQI if I succeeded, was rewriting everything in LabView (the language Hyok and a lot of people at JQI use for controlling of the experimental apparatus) into something else more flexible. The problem with LabView, according to Hyok, was that different versions of LabView don't work on different versions of Windows, drivers of old LabView might not work on newer versions, and so on. He said it would be very nice if everything was written in a consistent and open-source language like Python, so that his systems would be useless when older versions of LabView no longer got support. He showed me the Github of JQI, and gave me the name of PhD candidate: Zach Smith, who was also working on translating everything into Python. He said he could introduce me to him.
	
	
	\item These all seem like very interesting things to work on. I'd say building the 1064 nm lasers for the dipole trap would be good for learning physics and understanding how the ``real'' optics work. It would also be nice to help Hyok move away from the temporary (but pretty good) solution of using a heating laser on the nanofiber and instead to using the dipole trap (which will keep the atoms very close to but not touching the nanofiber - touching the nanofiber is bad because the fiber has to be clean so that the atoms can couple with the electromagnetic field guided by the fiber). That, working directly with the experiment, and reading papers will definitely give me a very clear sense of what is going on in the lab and help me appreciate a little of what people at JQI do. 
	
	
	\item The LabView - Python translation project would also be interesting to work on too, even though it doesn't sound to involve a lot of physics. Of course, to be able to do this, I will have to be very comfortable with the apparatus in the lab and understand the current LabView procedures very well. I'd say this is a more ambitious project, but I think it will be worth a try. Besides, Hyok said if I could successfully translate everything in LabView into Python then not only I get to show Hyok how his own system works but people at JQI will also come to me for help (which is kind of a crazy idea) with changing their system as well. It seems that this LabView system is quite a bug for a number of people at JQI. 
	
	
	\item Hyok said I could do both, perhaps focusing a little more on the experimental side (that is building the dipole trap), but \textit{both}. So that's what I think I will do: experimental physics by day and some theory/computer science by night. I think those will take up my entire summer. I hope I have enough time to get something done. With me current work status (or lack thereof, to be precise) I can only read papers and think about what to do. I can go to the lab, but it is not a good idea for a number of reasons.
	
	
	\item Of course I won't forget about the \LaTeX projects I am current working on for myself. A few hours every week on those will be good. There is also time for that. 
	
	\item And of course I'm still waiting desperately for my OPT to get approved. \textit{Just a few days more}, I hope. 
	
	
	\end{enumerate}
\end{document}
