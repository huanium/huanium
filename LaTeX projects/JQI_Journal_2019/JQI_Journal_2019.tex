\documentclass{report}
\usepackage{physics}
\usepackage{amsmath}
\usepackage{authblk}
\usepackage{amsfonts}
\usepackage{esint}
\usepackage{mathtools}
\usepackage{graphicx}
\usepackage{amsthm}
\theoremstyle{definition}
\newtheorem{defn}{Definition}[section]
\newtheorem{rmk}{Remark}[section]
\newtheorem{exmp}{Example}[section]
\newtheorem{sln}{Solution}[section]
\newtheorem{prop}{Proposition}[section]
\newtheorem{thm}{Theorem}[section]

\setcounter{chapter}{-1}


\makeatletter
\renewcommand{\@chapapp}{Week}
\makeatother


\newcommand{\p}{\partial}
\newcommand{\R}{\mathbb{R}}
\newcommand{\C}{\mathbb{C}}
\newcommand{\lag}{\mathcal{L}}
\newcommand{\I}{\mathcal{I}}
\newcommand{\K}{\mathcal{K}}
\newcommand{\F}{\mathcal{F}}
\newcommand{\w}{\omega}
\newcommand{\lam}{\lambda}
\newcommand{\al}{\alpha}
\newcommand{\be}{\beta}
\newcommand{\x}{\xi}

\newcommand{\f}[2]{\frac{#1}{#2}}

\newcommand{\ift}{\infty}

\newcommand{\lp}{\left(}
\newcommand{\rp}{\right)}

\newcommand{\lb}{\left[}
\newcommand{\rb}{\right]}

\newcommand{\lc}{\left\{}
\newcommand{\rc}{\right\}}


\newcommand{\V}{\mathbf{V}}
\newcommand{\U}{\mathcal{U}}
\newcommand{\Id}{\mathcal{I}}


\usepackage{empheq}
\usepackage{tensor}
\usepackage{hyperref}


\usepackage{xcolor}
\hypersetup{
	colorlinks,
	linkcolor={black!50!black},
	citecolor={blue!50!black},
	urlcolor={blue!80!black}
}

\begin{document}
	\begin{titlepage}\centering
		\clearpage
		\title{\textsc{\bf{Summer 2019 at JQI}}\\\smallskip NIST \& University of Maryland, College Park\\}
		\author{\bigskip Huan Q. Bui}
		 \affil{Colby College\\$\,$\\ PHYSICS \& MATHEMATICS\\ Statistics \\$\,$\\Class of 2021\\}
		\date{\today}
		\maketitle
		\thispagestyle{empty}
	\end{titlepage}

\newpage

\subsection*{Preface}
\addcontentsline{toc}{subsection}{Preface}

Greetings,\\

This journal is about my appointment at the Joint Quantum Institute at College Park in Maryland up to June 4, 2019. This journey truly deserves a proper, detailed, documentation not only for my own learning and research purposes but also for all twists and turns along the way.  \\

Enjoy, I guess?


\newpage
\tableofcontents
\newpage

\chapter{Think JQI}

\section{Pre\textendash JQI}

\subsection{Conover Lab \& Summer 2018}

\subsection{October, 2018}


\subsection{February, 2018}

\subsection{OPT application, 2018: The wait begins}






\section{Towards JQI}

\subsection{May, 2019: Absolute chaos}

\subsection{DAMOP19 in Milwaukee, Wisconsin}

\subsection{Heading to JQI, May 31\textendash June 2, 2019}


\chapter{A peek of JQI}

\section{Monday, June 3, 2019}

\begin{itemize}
	\item I couldn't start working yet but professor Steven Rolston was kind enough to let me tour one of his labs and meet Hyok Sang Han, a post-doc I would be assisting over the summer.  
	\item I decided to start a \LaTeX project on Quantum Mechanics, based on my reading of Shankar's Quantum Mechanics book. 
\end{itemize}








\section{Tuesday, June 4, 2019}

\begin{itemize}
	\item Met Prof. Rolston again. Prof. Rolston introduced me to the post-doc I would be working with: Hyok Sang Han and his lab.
	\item Hyok Sang Han and prof. Rolston gave me a little bit of a tour of the lab and the nanofiber experiment Hyok has been working on. 
	\item Prof. Rolston then left. Hyok and I hung out for a bit more. He bought me lunch at Pho D'Lite nearby. We then talked for a bit about ourselves and what brought us to JQI.
	\item I of course explained in detail I still couldn't work but I would be more than happy to look at papers and past theses.
	\item Later that day Hyok sent me three PhD theses to read before I could actually work. 
	\item I got an OPT update. The congresswoman must have successfully contact USCIS about my case. The update sent by USCIS was very vague, and there was nothing I could make out of the message they sent me. 
	\item My friend got his OPT approved.
\end{itemize}





\section{Wednesday, June 5, 2019}
\begin{itemize}
	\item Hyok told me of a weekly journal club meeting where everybody often get together to talk about papers they have been reading - exactly what the name of the club suggests. I told Hyok I might join the next Wednesday.
	\item I began reading the PhD theses. I hope to be able to summarize some of them somewhere in this journal.
	\item I noticed that my case got updated, based on what the USCIS website said. But I had no idea what it was.  
	\item Later in the day I asked Hyok if I could come to the lab the next day to look at what he is current working on and if there was something I could do for the summer. He said I could come and look at the demonstration of the nanofiber pulling process (he was going to demonstrate this process to another post-doc), and told me to come to his lab after that to discuss some possible summer projects. 
\end{itemize}



\section{Thursday, June 6, 2019}
	\begin{itemize}
	\item I came into the lab at around 2 when Hyok was heading over the lab where the fiber-pulling process takes place. I met Matt, the other post-doc. We both watched Hyok operate the fiber-pulling system. The fiber pulling was unsuccessful (the fiber broke, which is what can happen quite often) but it a good learning experience. The system definitely worked, although as I could tell it is not fully automatic. A lot of care is needed when operating.
	
	
	\item Hyok and I then headed back to the lab where we discussed what Hyok has been up to. Hyok told me he was trying to recreate of the earlier experiments to make sure his apparatus worked. He then explained to me some of the physics involved and showed me a few papers (that I decided that I will read - to get to the bottom of things). 
	
	
	\item So according to Hyok there are maybe 2-3 things I can try to do over the summer. The first thing is to help him with setting up the pulse laser setup that is used for the selective excitation of the MOT cloud. This might not be an ideal thing for me to work on, because there would be some overlap between me and him. The second thing I could do was setting up the lasers that will be used for a \textbf{dipole trap} (Hyok also explained to me what a dipole trap is). The third I could do, which would be very important to JQI if I succeeded, was rewriting everything in LabView (the language Hyok and a lot of people at JQI use for controlling of the experimental apparatus) into something else more flexible. The problem with LabView, according to Hyok, was that different versions of LabView don't work on different versions of Windows, drivers of old LabView might not work on newer versions, and so on. He said it would be very nice if everything was written in a consistent and open-source language like Python, so that his systems would be useless when older versions of LabView no longer got support. He showed me the Github of JQI, and gave me the name of PhD candidate: Zach Smith, who was also working on translating everything into Python. He said he could introduce me to him.
	
	
	\item These all seem like very interesting things to work on. I'd say building the 1064 nm lasers for the dipole trap would be good for learning physics and understanding how the ``real'' optics work. It would also be nice to help Hyok move away from the temporary (but pretty good) solution of using a heating laser on the nanofiber and instead to using the dipole trap (which will keep the atoms very close to but not touching the nanofiber - touching the nanofiber is bad because the fiber has to be clean so that the atoms can couple with the electromagnetic field guided by the fiber). That, working directly with the experiment, and reading papers will definitely give me a very clear sense of what is going on in the lab and help me appreciate a little of what people at JQI do. 
	
	
	\item The LabView - Python translation project would also be interesting to work on too, even though it doesn't sound to involve a lot of physics. Of course, to be able to do this, I will have to be very comfortable with the apparatus in the lab and understand the current LabView procedures very well. I'd say this is a more ambitious project, but I think it will be worth a try. Besides, Hyok said if I could successfully translate everything in LabView into Python then not only I get to show Hyok how his own system works but people at JQI will also come to me for help (which is kind of a crazy idea) with changing their system as well. It seems that this LabView system is quite a bug for a number of people at JQI. 
	
	
	\item Hyok said I could do both, perhaps focusing a little more on the experimental side (that is building the dipole trap), but \textit{both}. So that's what I think I will do: experimental physics by day and some theory/computer science by night. I think those will take up my entire summer. I hope I have enough time to get something done. With me current work status (or lack thereof, to be precise) I can only read papers and think about what to do. I can go to the lab, but it is not a good idea for a number of reasons.
	
	
	\item Of course I won't forget about the \LaTeX projects I am current working on for myself. A few hours every week on those will be good. There is also time for that. 
	
	\item And of course I'm still waiting desperately for my OPT to get approved. \textit{Just a few days more}, I hope. 
	
	
	\end{itemize}



\section{Friday, June 7, 2019}
\begin{itemize}
	\item I stayed in my room for most of the day and worked on random ideas I've been having in mind. I spend a few hours writing the QM notes, then looked at a few bikes on craigslist. 
	
	\item Later in the day Hyok and I discussed a little bit over email a new computer he was going to purchase for the lab. The currently lab computer situation was quite strange. He had two different computers controlling two parts of the same experiment, the reason being the incompatibility issues of LabView. Some experimental control programs he had on one computer could not be opened on another. He was still able to run the experiments, but things could get a little messy. 
	
	\item A few hours later I was looking at the experimental programs Hyok was using and tried to figure out how they work. I tried to install a trial version of LabView, but not surprisingly it didn't support the .vi file Hyok sent me. This means to look at the files I will have to go to the lab and turn on the computers there. That's fine, except I can't do that without my work permit.
	
	\item A few moments later I decided to uninstall LabView. 
	
	\item I decided to start reading about optical dipole traps. 
\end{itemize}



\section{Saturday, June 8, 2019}
\begin{itemize}
	\item I wrote this from my aunt's house in Rockville. My family and I decided that I would stay here for the weekend.
	
	\item Here's some notes on the optical dipole trap paper I was reading. Here's the \href{https://arxiv.org/pdf/physics/9902072.pdf}{link}.
	
	\begin{itemize}
		\item Introduction: Traps for \textit{neutral} and \textit{charged} atoms are of course different. In traps for charged particles, a commonly used trapping force is the Coulomb interaction. Interactions used for trapping neutral atoms are often much weaker than the Coulomb interaction. Traps for neutral atoms can be realized on the basis of three different interactions. This basis gives three kinds of traps: \textit{radiation-pressure traps, magnetic traps,} and \textit{optical dipole traps}.
		
		\item I will go directly to the principles of optical dipole trapping. Optical dipole traps rely on the electric dipole interaction with far-detuned light (red and/or blue). This interaction is the weakest of the three interactions for trapping neutral atoms. Under appropriate conditions, the trapping mechanisms is independent of the particular sub-level of the electronic ground state. Moreover, a great variety of different trapping geometries can be realized as, e.g., highly anisotropic or multi-well potentials.

		
		\item So it seems like optical dipole traps is very advantageous: it avoids involving too much with the energy structure of the particles being trapped. It is also flexible, as many geometries/configurations of the trap can be realized.
		
		\item So how does optical dipole traps work? Here, we will consider only traps with far-detuned light.  In these traps an ultracold ensemble
		of atoms is confined in a nearly conservative potential well with very weak influence from spontaneous photon scattering. 
		
		
		\item We first consider an ensemble of two-level atoms. We assume that the optical excitation is very low and radiation force due to photon scattering is negligible compared to the dipole force. We will assume that the atom is a classical oscillator or quantum-mechanical oscillator to derive the main equations for the optical dipole interaction. 
		
		
		\item Oscillator model: Now, because the dipole interaction force is conservative, it can be derived from a gradient of a potential, whose minima are used for atom trapping. In the next steps, we will derive the equations for the dipole potential and the scattering rate by considering the atom as a simple oscillator subject to a classical radiation field. 
		
		
		\item When we shine laser light to an atom, we essentially place the atom into an electric field $\textbf{E}$, which induces an atomic dipole moment $\textbf{p}$ that oscillates at frequency $\omega$, the (driving) frequency of the laser. Assume that the electric field can be written as
		\begin{align}
		\textbf{E}(\textbf{r},t) = \mathbf{\hat{e}}\tilde{E}(\textbf{r})e^{-i\omega t} + c.c.
		\end{align}
		then the dipole moment is
		\begin{align}
		\textbf{p}(\textbf{r},t) = \mathbf{\hat{e}}\tilde{p}(\textbf{r})e^{-i\omega t} + c.c.
		\end{align}
		where $\mathbf{\hat{e}}$ is the polarization unit vector. The amplitudes $\tilde{E}$ and $\tilde{p}$ are linearly related:
		\begin{align}
		\tilde{p} = \alpha \tilde{E},
		\end{align}
		where $\alpha$ is the \textit{complex polarizability}, which depends on the driving frequency $\omega$. The interaction potential of the induced dipole moment $\textbf{p}$ in the driving field $\textbf{E}$ is given by
		\begin{align}
		U_{\text{dipole}} = -\frac{1}{2}\langle \textbf{pE} \rangle,
		\end{align}
		where the brackets denote the time average. We want to write this in terms of the field intensity $I = 2\epsilon_0 c\abs{\tilde{E}}^2$ and the polarizability. Replacing $\tilde{p}$ with $\alpha\tilde{E}$, we get
		\begin{align}
		U_{\text{dipole}} = -\frac{1}{2\epsilon_0 c}\Re(\alpha) I.
		\end{align}
		The real part of the polarizability describes the in-phase component of the dipole oscillation being responsible for the dispersive properties of the interaction. Now, the dipole force is the gradient of the interaction potential:
		\begin{align}
		\textbf{F}_\text{dipole} = -\grad{U_\text{dipole}}(\textbf{r}) = \frac{1}{2\epsilon_0 c}\Re(\alpha)\grad{I}(\textbf{r}).
		\end{align}
		As we have said before, this is a conservative force, and it is proportional to the gradient of the intensity of the field. 
		
		
		The power absorbed by the oscillator from the driving field is given by
		\begin{align}
		P_\text{abs} = \langle \dot{\textbf{p}\textbf{E}}  \rangle = \frac{\omega}{\epsilon_0 c}\Im(\alpha)I.
		\end{align}
		The absorption results from the imaginary part of the polarizability, which describes the out-of-phase component of the dipole oscillation. Let us assume that the light field is a stream of photons with momentum $\hbar \omega$. This gives the scattering rate:
		\begin{align}
		\Gamma_\text{scat}(\textbf{r}) = \frac{P_\text{abs}}{\hbar \omega} = \frac{1}{\hbar \epsilon_0 c}\Im(\alpha) I(\textbf{r}). 
		\end{align}
		
		\item What we have done so far is obtaining expressions for the interaction potential and scattered radiation power in terms of the position-dependent field intensity $I(\textbf{r})$ and the polarizability $\alpha(\omega)$. 
		
		\item To calculate $\alpha(\omega)$, we consider the atom in Lorentz's model of a classical oscillator. In this model, an electron with mass $m_e$, charge $e$ is considered to be bounded to the core and to have oscillation frequency of $\omega_0$, which is also the optical transition frequency. There's also damping due to dipole radiation of oscillating electron (Larmor's formula). This gives the equation of motion:
		\begin{align}
		\ddot{x} + \Gamma_\omega \dot{x} + \omega_0^2 x = -\frac{eE(t)}{m_e}.
		\end{align}
		Solving this equation and solving for $\alpha(\omega)$ gives
		\begin{align}
		\alpha(\omega) = \frac{e^2}{m_e}\frac{1}{\omega_0^2 - \omega^2 -i\omega \Gamma_\omega},
		\end{align}
		where
		\begin{align}
		\Gamma_\omega = \frac{e^2\omega^2}{6\pi \epsilon_0 m_e c^3}
		\end{align}
		is the classical damping rate due to radiative energy loss. Writing $e^2/m_e$ in terms of $\Gamma_\omega$ and $\omega$ and introducing on-resonant damping rate $\Gamma \equiv \Gamma_\omega = (\omega_0/\omega)^2 \Gamma_\omega$ gives
		\begin{align}
		\alpha(\omega) = 6\pi \epsilon_0 c^3 \frac{\Gamma/\omega_0^2}{\omega_0^2 - \omega^2 - i(\omega^3/\omega_0^2)\Gamma}.
		\end{align}
		We note that this is a purely classical approach. There's also a semiclassical approach where the atom is a two-level quantum mechanical system. It turns out that the results are effectively the same. 
		
		\item An important difference between the quantum mechanical and the classical oscillator is the possible occurrence of saturation. At too high intensities of the driving field, the excited state gets strongly populated and the above result is no longer valid. For dipole trapping, however, we are essentially interested in the far-detuned case with very low saturation and thus very
		low scattering rates ($\Gamma_\text{scat} \ll \Gamma$). Thus our result is an excellent approximation. 
		
		
		\item From the expression for the polarizability of the atomic oscillator, we can derive the dipole potential and scattering rate for large detunings and negligible saturation:
		\begin{align}
		&U_\text{dipole}(\mathbf{r}) = -\f{3\pi c^2}{2\omega_0^3}\lp \f{\Gamma}{\omega_0 - \omega} + \f{\Gamma}{\omega_0 + \omega}\rp I(\mathbf{r})\\
		&\Gamma_\text{scat}(\mathbf{r}) = \f{3\pi c^2}{2\hbar \omega_0^3}\lp \f{\omega}{\omega_0} \rp^3 \lp \f{\Gamma}{\omega_0 - \omega} + \f{\Gamma}{\omega_0 + \omega}\rp^2 I(\mathbf{r}).
		\end{align}
		
		
		
		\item Now, in most experiments, the laser is tuned very close to $\omega_0$, such that $\abs{\Delta} \ll \omega_0$. In this case, we have that $\omega/\omega_0 \approx 0$. Also, under the rotating wave approximation, we can neglect the fast term $\omega+ \omega_0$. These simply our expressions for the dipole and scattering rate to
		\begin{align}
		&U_\text{dipole}(\mathbf{r}) = \f{3\pi c^2}{2\omega_0^3}\f{\Gamma}{\Delta}I(\mathbf{r})\\
		&\Gamma_\text{scat} = \f{3\pi c^2}{2\hbar \omega_0^3}\lp \f{\Gamma}{\Delta} \rp^2I(\mathbf{r}).
		\end{align}
		
		\item We can also derive a simple relation between the scattering rate and the dipole potential
		\begin{align}
		\hbar \Gamma_\text{scat} = \f{\Gamma}{\Delta}U_\text{dipole}.
		\end{align}
		
		
		\item These equations we just derived show two very essential points for dipole trapping:
		\begin{itemize}
			\item \textit{Sign of detuning:} For red detuning ($\Delta < 0$), the dipole potential is negative and the interaction thus attracts atoms into the light field. Potential minima are therefore found at positions with maximum intensity. For blue detuning ($\Delta > 0$), the dipole interaction repels atoms out of the field, and potential minima correspond to minima of the intensity. And so dipole traps can be divided into red-detuned and blue-detuned traps. 
			\item \textit{Scaling with intensity and detuning:} The dipole potential scales as $I/\Delta$, while the scattering rate scales as $I/\Delta^2$. Therefore, optical dipole traps usually use large detunings and high intensities to keep the scattering rate as low as possible at a certain potential depth.
		\end{itemize}
	
	
	\item Below are some nice slides taken from this \href{https://www.mpq.mpg.de/5020867/0515b_atom_traps.pdf}{link}:
	\begin{figure}[!htb]
		\centering
		\includegraphics[scale=0.3]{slide1}
		\includegraphics[scale=0.3]{slide2}
		\includegraphics[scale=0.3]{slide3}
		\includegraphics[scale=0.3]{slide4}
		\includegraphics[scale=0.3]{slide5}
		\includegraphics[scale=0.3]{slide6}
	\end{figure}
		
	\end{itemize}





\end{itemize}



\section{Sunday, June 9, 2019}



\begin{itemize}
	\item It turns out that there are many kinds of optical dipole traps. There are red-detuned and blue-detuned, and among these are focused-beam, crossed-beam, standing-wave, and a couple of other kinds of traps. I'm interested in the standing-wave approach because of its geometry. 
	
	\item Essentially, what we want is to dipole-trap the atoms along the optical nanofiber with both red-detuned and blue-detuned methods, so as to both keep the atoms close to the nanofiber (red-detuned traps attract atoms to maximal intensity peaks) but not too close that the atoms stick to the fiber (blue-detuned traps repel atoms from the maximal intensity peaks). Explanations to any of this will eventually come as I read more, but what I can tell immediately is that the slides seem to be heavily inspired by the very same \href{https://arxiv.org/pdf/physics/9902072.pdf}{paper} I have been reading. The structure follows very well. I also see some mentioning of scaling rules, etc. This and the paper make a very good reading combination. The slides really helps with skipping the unimportant details in the paper and focusing on the main results. 
	
	\item What I now need is not only theory but what people have done to make these standing-wave traps. I managed to find a very well-written experimental \href{http://quantum-technologies.iap.uni-bonn.de/de/component/publications/?task=download\&file=69\&token=de46701d9e5eb9e185abc9784c3f8313}{paper} and a nice \href{https://www.mpq.mpg.de/5020867/0515b_atom_traps.pdf}{slideshow} that explain things pretty well. I'll make sure to take notes on the paper to actually know how to build a dipole trap. A very nice thing about that paper is that they also use the 1064 nm laser, exactly the one we will be using in Hyok's lab.
	
	
	
	\item Here's some notes on the paper by ALT et al on standing-wave dipole traps plus extra material I read on wikipedia and other sources:
	\begin{itemize}
		\item So, a standing-wave dipole trap consists of two counter-propagating Gaussian beams with equal intensities and parallel linear polarizations. Now, the paper talks about transporting the atoms, so they allow for the relative frequencies of the two counter-propagating beams to differ ($\omega$ and $\omega + \Delta \omega$) where $\Delta \omega \ll \omega$. With these, they could produce a position and time-dependent dipole potential of the form:
		\begin{align}
		V(z,\rho,t,U_0) = U_0 \f{w_0^2}{w^2(z)}e^{-2\rho^2/w^2(z)}\cos^2\lp \f{\Delta \omega}{2}t - kz \rp.
		\end{align}
		where $\lambda = c/\omega$ is the optical wavelength, $w^2 = w_0^2(1 + z^2/z_0^2)$ is the beam radius with waist $w_0$. $z_0 = \pi w_0^2 / \lambda$ is the Raleigh length. All of these quantities are shown schematically below:
		\begin{figure}[!htb]
			\centering
			\includegraphics[scale=0.05]{gauss.png}
		\end{figure}
	
		\item Fortunately, the dipole trap used in this paper is derived from a Nd:YAG laser $\lambda = 1064$ nm, which is far red-detuned from $D_1$ and $D_2$ Cesium lines. The maximum trap depth $U_0$ is given by
		\begin{align}
		U_0 = \f{\hbar \Gamma}{2}\f{P}{\pi w_0^2 I_0}\f{\Gamma}{\Delta}.
		\end{align}
		$\Gamma = 2\pi \times \omega'$ is the natural linewidth of some transition. $I_0$ is the corresponding saturation intensity. $P$ is the total power of both laser beams. For red-detunings, $\Delta < 0$, the dipole potential provides a three-dimensional confinement with a trap depth of $\abs{U_0}$. For alkalies, the effective detuning $\Delta$ is given by
		\begin{align}
		\f{1}{\Delta} = \f{1}{3}\lp \f{1}{\Delta_1} + \f{2}{\Delta_2} \rp
		\end{align}
		where $\Delta_i$ is the detuning from the $D_i$ line. With the rest of the parameters known, we can calculate the potential depth $U_0$ in Kelvin.
		 
		\begin{figure}[!htb]
			\centering
			\includegraphics[scale=0.5]{dipole.png}
		\end{figure}
	
	
		\item An atom of mass $m$ trapped in such a standing-wave potential oscillates (in harmonic approximation) with frequencies
		\begin{align}
		&\Omega_z = 2\pi \sqrt{\f{2U_0}{m\lambda^2}},\\
		&\Omega_\text{rad} = \sqrt{\f{4U_0}{mw_0^2}}.
		\end{align}
		If we know the mass of the atoms and beam waist $w_0$ and other beam parameters we can find these frequencies.
		
 	\end{itemize}
 
 
 
 	\item I also found an excellent \href{https://www.phas.ubc.ca/~qdg/publications/GraduateTheses/thesis_hon_Gao.pdf}{paper} by a student from the University of British Columbia on building a standing-wave optical dipole trap. This is an almost-perfect combination of theory and experiments. From a quick read-through I found that a lot of the theory came from Grimm's \href{https://arxiv.org/pdf/physics/9902072.pdf}{paper}, which is a good thing for me as I'm sure the new knowledge from this paper is consistent with what I have read before. The experimental procedures are also very well laid out. This paper is probably something I can refer to when I run into trouble. The paper also seems to be using some kind of combination of Python and LabView - which is pretty good considering my plans for building a dipole trap and moving the lab from LabView to Python.
\end{itemize}





\chapter{Bureaucratic Limbo}

\section{Monday, June 10, 2019}

\begin{itemize}
	\item I'm back on UMD campus, still waiting for my OPT application to get approved. If it doesn't get approved soon, I won't get my EAD card this week, which means I'll lose another week of the program.
	
	
	\item It's the end of the day. I heard nothing. Three people from Colby have gotten their OPT approved so far. One got approved today. I hate to say this but I'm losing hope I'm getting anything by the end of this week. 
	
	
	\item I can't do any work in this kind of limbo. I'd rather be out of a job, or not having a job at all, than being stuck in this indeterminate state. Will the EAD come? It might. OK. But when? This week? A week from now? Two weeks from now? When I'm on the plane back home? Can I even BE on a plane back home while my application is still processing?
	
	\item I can't concentrate. 
	
	\item I emailed Hyok this morning for more resources on the two-color optical dipole trap. It turned out that both Grover and Solan somewhat mentioned how to implement this in their papers. I will take notes on these in the near future. 
	
	\item Solano's ONF \href{https://arxiv.org/pdf/1703.10533.pdf}{review} on the arXiv, page 28. 
	
	\item Grover's \href{https://drum.lib.umd.edu/handle/1903/16638}{thesis}, Chapter 3. 
	
	\item If these link are broken, these papers can be found on my GitHub repository. 

\end{itemize}



\section{Tuesday, June 11, 2019}

\begin{itemize}
	\item It's my father's birthday, and all I have for him is bad news after bad news of this bureaucratic idiocy that is screwing my future over. And there's nothing I can do about it. Let's see what today brings (or not).
	
	\item Nothing new and I'm tired. 
	
	
	\item \textit{Grover's thesis, chapter 3:} N/A
	
	
	\item \textit{Solano's ONF review, page 28:} N/A
	
	\item Hyok sent me a link to a nice seminar at JQI a few years ago. Here's the \href{https://www.youtube.com/watch?v=IFsB554TCZY}{link} to the talk. It was good, and I learned something.
	
	\item I failed to read the articles I intended to read today. For me the whole OPT situation is going to be decided this week, in a not-too-good or bad way. I've already lost a big chunk of my time here, and I'm pretty sure I'm not going to have enough time to work on translating LabView to Python. Now I'm afraid I won't even have enough to build the dipole trap. At this point I basically have a month-and-a-half or less to work. 
	
	\item To get anything done in this short amount of time I might have to work over time. Honestly, I don't care. I'm willing to work past supper, and I'll willing to work on weekends. I know I can do anything in any amount of space and time if I put all my energy to it. I know I'm capable of getting the job done. The only problem right now is a piece of plastic that should have been in the mail 10 days ago.
	
	\item I really want to just go to the labs and actually start with something. But I know I can't, and that the only thing I can do to make the best use of my time now is reading papers. But it's so hard when I don't even know if it's worth it anymore. All my effort for the past year and a half to get here and work at JQI... it means nothing, of course, to the people who are making the decision to whether to process my case and/or to approve/deny my case. And this is not just about me. I'm suppose to be there, right now, to help a post-doc who has been working by himself for the past year. I made promise to myself and to him and to Steve Rolston. Last year, I was in Maine, working my butt off in class and in the lab to learn as much as I could so that I could secure a spot at JQI this summer. Now I'm sitting here, on UMD campus; and JQI is a 10-minute walk away, yet I can't go near it just because of a piece of plastic I need to physically have. 
	
	\item This really hurts. This whole process is really \textit{fucking} with me, and it's progressing for the worse. Everyday I check my case status about a hundred times. And everyday past without any news I recalculate how much time I have left and what I should be doing next. I start to think everyday pass without a response to a new strike against me, and I obsess over the next best-response: what's nest to do next. Sometimes I would come out of this obsession and read and take notes on some quantum mechanics. Sometimes I feel good enough to read and take notes on the dipole trap experiment. But most of the time I feel so claustrophobic that I can't force myself to do anything. I have so many questions (physics and non-physics) that I can't answer. Will I get to work? When will I get to work? Will I make it if I get to work? Can I extend my stay or can I go home? When can I go home? How should I explain my situation to people? My professors? My family? My friends? Myself? Is this summer wasted? Could I have done better? What do I do now? 
	
	\item Is this part of what being an international student is like? Is this fair? No and No. I certainly feel cheated by the administrative people who are and/or should be involved in all this work permit application mess. I really do. I couldn't apply more than 90 days from the start date, yet it's already taking them 90 days and counting to process the damn case. 
	This whole process very much messes with my initially not-too-negative view on bureaucracy here. I won't say much more, except that I'm very disappointed. 
	
	\item Should I hope for anything tomorrow? Well, I don't know. I will always be expecting news. But I don't think I'll be any more hopeful than yesterday or the day before or last week. Am I desperate? I don't know either. I've done all I could, so it doesn't make much of a difference whether I'm desperate or not.
	
	\item It's Wednesday tomorrow. There's going to be a lab group meeting. But I don't think I'll be there as I don't think I can call myself a group member with 2 days in the lab.
	
	\item There's going to be a JQI seminar on Thursday which I plan to attend. 
	
	
\end{itemize}


\section{Wednesday, June 12, 2019}
\begin{itemize}
	\item \textit{Grover's thesis, chapter 3:} N/A
	
	
	\item \textit{Solano's ONF review, page 28:} N/A
\end{itemize}




\end{document}
