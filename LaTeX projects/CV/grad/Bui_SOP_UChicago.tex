\documentclass[12pt]{article}
\usepackage[left=1in, right=1in,top=1.25in,bottom=1in]{geometry}

%\usepackage{newpxtext,newpxmath}
%\usepackage{amsfonts}
\usepackage{tgtermes, amsmath}
\usepackage{color}
\usepackage{bold-extra}
\usepackage{setspace}
\setstretch{1.1}
\usepackage{fancyhdr}
\pagestyle{fancy}
\lhead{Huan Q. Bui}
\rhead{Application for Chicago Physics}

\begin{document}
\begin{center}
	\textbf{Candidate Statement}
\end{center}
I am captivated by the strong interplay between theory and experiment in condensed matter and atomic, molecular, and optical physics. In particular, I am interested in experiments with ultracold gases for quantum science and fundamental physics. In May 2021, I will complete my undergraduate studies at Colby College with two Honors Theses on experimental atomic physics and mathematical analysis. My next objective is to expand and subsequently apply my research training to address a challenging problem in quantum science. To that end, the Physics Ph.D. program at the University of Chicago is a fantastic option.  \\ 


Since my first year at Colby College, I have been working on ultracold potassium experiments with Professor Charles Conover, my advisor and mentor. Currently, I am working towards a Physics Honors Thesis under his supervision on precision lifetime measurement of potassium's 5P$_{\text{3/2}}$ quantum state by counting photons emitted from an excited, magneto-optically trapped, atomic cloud of this species. In previous years, I constructed a variety of laboratory apparatus from external-cavity diode lasers to electronics for laser frequency stabilization. I also carried out precision spectroscopy on potassium in Rydberg states to determine its quantum defects and absolute energy levels. At our level of precision, energy shifts due to the millimeter-wave source are significant and thus require data extrapolation to obtain unbiased measurements. Applying Ramsey's separated oscillatory field method, I eliminated this necessity and gave an alternative measurement scheme with comparable precision. I presented this work at the Colby Undergraduate Summer Research Retreat in 2018 and at DAMOP 19. Through working in Professor Conover's lab, I was introduced to many fascinating experimental and theoretical techniques in atomic physics, which strengthened my initial interest in quantum science and inspired me to pursue graduate studies in this field.  \\


To broaden my experience in atomic physics, I joined Professor Steven Rolston's group at the Joint Quantum Institute (JQI) in Summer 2019 to work on probing the long-range interaction among rubidium atoms magneto-optically trapped around an optical nanofiber. There, I built an imaging system for optimizing light polarization in optical nanofibers, which often introduce birefringence and undesirable longitudinal polarizations. I also developed a stand-alone Python program for controlling the entire experiment, removing the group's reliance on the less compatible LabView program. In January 2020, Dr. Hyok Sang Han and I observed a transient decay flash in the rubidium 5P$_{\text{3/2}}$ population that was much faster than even the fastest superradiance mode of the system.  Since then, the group at JQI has been developing a theory for this phenomenon, for it is currently not well-understood in the one-dimensional geometry of our nanofiber experiment. \\ 


Following the time at JQI, my fascination with quantum information science led to my current research project with Dr. Timothy Hsieh at the Perimeter Institute for Theoretical Physics. The project explores efficient variational simulation of non-trivial quantum states that are not adiabatically connected to unentangled product states. Over this past summer, we aimed to accelerate Dr. Hsieh's simulation protocol, which according to numerical results could target the ground state of the uniform transverse-field Ising model with perfect fidelity via a low-depth ansatz based on the Quantum Approximate Optimization Algorithm (QAOA). However, after I found that this protocol could also target the ground state of any non-uniform Ising model with random transverse field and couplings, our focus shifted towards understanding how QAOA provides such a reliable ansatz. Since then, I have obtained additional results related to targeting eigenstates of more general Ising Hamiltonians using similar schemes. I am now studying the free-fermion representation of the non-uniform transverse-field Ising model. Ultimately, we aim to prove that the QAOA-based ansatz can indeed target any eigenstate of this model with perfect fidelity, in accordance with numerical results.  \\

I also enjoy exploring other areas of mathematics and physics and working on personal projects. Last year, I wrote a program to evaluate iterative convolutions for Professor Evan Randles at Colby and found numerical evidence for a conjecture in his mathematical analysis research. This led us to develop new results on which I will write my Mathematics Honors Thesis. We will also present at the Joint Mathematics Meetings in January 2021 and eventually publish these results. In 2018, I built a website to archive my reading and class notes. Since then, I have developed a selection of these notes into undergraduate-level ``guides'' to various topics in physics and mathematics, which include my exposition of massive gravity from the independent studies with Professor Robert Bluhm. The website now also holds my digital and film photography and, soon, my classical guitar recordings. Through these personal explorations, I not only get to work on a variety of interesting problems from proving mathematical theorems to media production, but also find great joy in inspiring my peers to start their own projects such as building a research blog or initiating an independent study. As a result, I always feel motivated to find newer and more challenging pursuits. \\

At Chicago, I would like to focus my efforts on quantum science research. I have corresponded with Professor David DeMille and am interested in working on his electron's EDM search experiment. I am also attracted to his new project on assembling ultracold polar molecules for making entanglement and potentially precision measurement of the proton's EDM. In addition, I am interested in Professor Jonathan Simon's research on photonic materials. Through meeting with him, I learn that his graduate students get to work on exploratory projects besides the main research, which is very exciting. Upon receiving my Ph.D., I aim to continue research in quantum science and eventually teach at a research-oriented university. Ultimately, I would like to make impactful contributions to the continually growing body of work in this field. I believe that admission to the Physics Ph.D. program at the University of Chicago is an ideal step towards this goal. \\

\noindent Thank you for your consideration. \\

\noindent \today






	
	
	
	
	
\end{document}