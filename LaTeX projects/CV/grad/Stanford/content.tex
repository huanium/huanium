At the moment, I am continuing the research started this summer with Dr. Timothy Hsieh at the Perimeter Institute for Theoretical Physics on efficient variational simulation of quantum states that are not adiabatically connected to unentangled product states. We initially intended to accelerate Dr. Hsieh's recently-developed protocol, which was motivated by the Quantum Approximate Optimization Algorithm (QAOA) and, based on numerics, could target the ground state of the uniform transverse-field Ising model with perfect fidelity. However, after I found that this protocol could also target the ground state of any non-uniform Ising model with random transverse field and couplings, our research focus shifted towards understanding how QAOA provides such a reliable ansatz. Since then, I have obtained additional results related to targeting excited states of this model and eigenstates of any disordered Ising Hamiltonian using similar schemes. My next goal is to study the free-fermion representation of the non-uniform transverse-field Ising model to better understand the structure of its eigenstates. Ultimately, we aim to provide a proof that the QAOA-based ansatz can target any eigenstate of this model with perfect fidelity in accordance with our numerical results.  \\


My interest in quantum information science originated with research experiences in ultracold atom experiments at the Joint Quantum Institute (JQI) and Colby College. In Summer 2019, I joined the Rolston group at JQI to study the long-range interaction between rubidium atoms magneto-optically trapped around an optical nanofiber. There, I built an imaging system for optimizing light polarization in optical nanofibers, which often introduce birefringence and undesirable longitudinal polarizations. I also developed a stand-alone Python program for controlling the entire experiment, removing the group's reliance on the less compatible LabView program. In January 2020, Dr. Hyok Sang Han and I observed a mysterious transient decay flash in the rubidium population that was much faster than even the fastest superradiance mode of the system.  The group at JQI is developing a theory for this phenomenon, for it was only recently discovered and is not well-understood in the one-dimensional geometry of our nanofiber experiment.   \\ 

Since my first year at Colby, I have been working on ultracold potassium experiments under Professor Charles Conover, my advisor and mentor. Applying the experimental techniques I learned at JQI, I am currently working towards a Physics Honors Thesis on lifetime measurements of quantum states in potassium by counting photons emitted from an excited, magneto-optically trapped, atomic cloud of this species. In previous years, I constructed a variety of laboratory apparatus from external-cavity diode lasers to electronics for laser frequency stabilization. I also carried out precision spectroscopy on potassium in Rydberg states to determine its quantum defects and absolute energy levels. At our level of precision, energy shifts due to the millimeter-wave source are significant and thus require data extrapolation to obtain unbiased measurements. Applying Ramsey's separated oscillatory field method, I eliminated this necessity and gave an alternative measurement scheme with comparable precision. I presented this work at Colby's summer research symposium in 2018 and at DAMOP 19.  \\

Besides physics research, I also explore other areas of physics and mathematics. For four semesters with Professor Robert Bluhm, I studied general relativity and reviewed massive gravity along with its nonlinear effects. This resulted in my detailed expositions of these topics, which are available on my website. Now, I am self-studying quantum field theory with an emphasis on applications in condensed matter physics. Last year, intrigued by their application in partial differential equations, I began my Mathematics Honors Thesis under Professor Evan Randles on convolution powers of complex-valued functions. Motivated by my numerical evidence, we developed a generalization of the polar-coordinate integration formula which aids in the understanding of convolution powers and semigroups of pseudo-differential operators. We are preparing a manuscript summarizing this result and will present it at the Joint Mathematics Meeting in January 2021. These projects, along with my role as a physics and mathematics teaching assistant in which I develop teaching and communication skills, enrich my academic experience outside of coursework and physics research.  \\ 