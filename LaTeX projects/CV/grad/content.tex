Since my first year at Colby College, I have been working on ultracold potassium experiments with Professor Charles Conover, my advisor and mentor. Currently, I am working towards a Physics Honors Thesis under his supervision on precision lifetime measurement of potassium's 5P$_{\text{3/2}}$ quantum state by counting photons emitted from an excited, magneto-optically trapped, atomic cloud of this species. In previous years, I constructed a variety of laboratory apparatus from external-cavity diode lasers to electronics for laser frequency stabilization. I also carried out precision spectroscopy on potassium in Rydberg states to determine its quantum defects and absolute energy levels. At our level of precision, energy shifts due to the millimeter-wave source are significant and thus require data extrapolation to obtain unbiased measurements. Applying Ramsey's separated oscillatory field method, I eliminated this necessity and gave an alternative measurement scheme with comparable precision. I presented this work at the Colby Undergraduate Summer Research Retreat in 2018 and at DAMOP 19. Through working in Professor Conover's lab, I was introduced to many fascinating experimental and theoretical techniques in atomic physics, which strengthened my initial interest in quantum science and inspired me to pursue graduate studies in this field.  \\


To broaden my experience in experimental atomic physics, I joined Professor Steven Rolston's group at the Joint Quantum Institute (JQI) in Summer 2019 to work on probing the long-range interaction among rubidium atoms magneto-optically trapped around an optical nanofiber. There, I built an imaging system for optimizing light polarization in optical nanofibers, which often introduce birefringence and undesirable longitudinal polarizations. I also developed a stand-alone Python program for controlling the entire experiment, removing the group's reliance on the less compatible LabView program. In January 2020, Dr. Hyok Sang Han and I observed a mysterious transient decay flash in the rubidium 5P$_{\text{3/2}}$ population that was much faster than even the fastest superradiance mode of the system.  Since then, the group at JQI has been developing a theory for this phenomenon, for it is currently not well-understood in the one-dimensional geometry of our nanofiber experiment. \\ 


Following the time at JQI, my fascination with quantum information science led to my current research project with Dr. Timothy Hsieh at the Perimeter Institute for Theoretical Physics. The project explores efficient variational simulation of non-trivial quantum states that are not adiabatically connected to unentangled product states. Over this past summer, we aimed to accelerate Dr. Hsieh's simulation protocol, which according to numerical results could target the ground state of the uniform transverse-field Ising model with perfect fidelity via a low-depth ansatz based on the Quantum Approximate Optimization Algorithm (QAOA). However, after I found that this protocol could also target the ground state of any non-uniform Ising model with random transverse field and couplings, our focus shifted towards understanding how QAOA provides such a reliable ansatz. Since then, I have obtained additional results related to targeting eigenstates of more general Ising Hamiltonians using similar schemes. I am now studying the free-fermion representation of the non-uniform transverse-field Ising model. Ultimately, we aim to prove that the QAOA-based ansatz can indeed target any eigenstate of this model with perfect fidelity, in accordance with numerical results.  \\

Besides quantum physics, I also explore other areas of physics and mathematics. For four semesters with Professor Robert Bluhm, I studied general relativity and massive gravity and wrote detailed expositions of these topics on my website. Now, I am self-studying quantum field theory with an emphasis on applications in condensed matter physics. Last year, intrigued by their application in partial differential equations, I began research on convolution powers of complex functions with Professor Evan Randles. Motivated by my numerical evidence, we recently developed a generalization of the polar-coordinate integration formula which aids in the understanding of convolution powers. We will present this result at the Joint Mathematics Meetings in January 2021 and are preparing a manuscript for publication. Moreover, I will write my Mathematics Honors Thesis on this work. These projects, along with my role as a physics and mathematics teaching assistant in which I develop valuable pedagogical skills, enrich my academic experience beyond coursework. \\ 



