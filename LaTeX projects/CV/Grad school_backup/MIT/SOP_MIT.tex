\documentclass[12pt]{article}
\usepackage[left=1in, right=1in,top=1.25in,bottom=1.25in]{geometry}

%\usepackage{newpxtext,newpxmath}
%\usepackage{amsfonts}
\usepackage{tgtermes}
\usepackage{color}
\usepackage{bold-extra}
\usepackage{setspace}
%\pagenumbering{gobble}
\usepackage{setspace}
\setstretch{1.1}
%\onehalfspacing

%\fontsize{12pt}{15pt}\selectfont

\usepackage{fancyhdr}
\pagestyle{fancy}
\lhead{Huan Q. Bui}
\rhead{Application for MIT Physics}
%\cfoot{center of the footer!}
%\renewcommand{\headrulewidth}{0.4pt}
%\renewcommand{\footrulewidth}{0.4pt}

\begin{document}
\begin{center}
	\textbf{Statement of Objectives}
\end{center}
My name is Huan Bui, and I study physics and mathematics with a statistics minor at Colby College. I am drawn to the interplay between theory and experiment in quantum information and condensed-matter physics.  In May 2021, I will complete my undergraduate studies at Colby College with two Honors Theses on experimental atomic physics and mathematical physics. The natural next step for me is to apply my research training to address a fundamental open problem in quantum science, and I believe that the Physics Ph.D. program at MIT is a fantastic option to that end.  \\ 

My most current interest is in simulation of quantum many-body systems. I have been researching efficient variational simulation of non-trivial quantum states using the quantum approximate optimization algorithm (QAOA) ansatz with Dr. Timothy Hsieh at the Perimeter Institute for Theoretical Physics. Recently, Dr. Hsieh has developed one such protocol to target with perfect fidelity a class of non-trivial states using an L-deep circuit, where L is the system size. We initially aimed to speedup this protocol using aspects of measurement-based quantum computation. However, after I discovered numerically that the current protocol could perfectly simulating the ground state of any transverse-field Ising model with random field and couplings, the research focus shifted towards understanding how the QAOA ansatz consistently gives perfect fidelity. Since then, I have further shown that an (L+1)-deep QAOA ansatz could also perfectly simulate excited states of such Ising models. My next goal is to explore Dr. Hsieh's conjecture on the correspondence between the spin picture and the free-fermion picture of the problem, and to target disordered Ising models, which cannot be put into free-fermion form. Ultimately, Dr. Hsieh and I aim to establish relationships between this protocol and the ability to target many-body localized states.\\

My interest in quantum simulation stems from my background in ultracold atom experiments. In Summer 2019, I joined the Rolston group at the Joint Quantum Institute (JQI) to study the long-range interactions between rubidium (Rb) atoms magneto-optically trapped around an optical nanofiber. There, I built an imaging system for optimizing light polarization in nanofibers, whose medium often introduces birefringence and undesirable longitudinal polarizations. I also developed a stand-alone Python program for controlling the entire Rb experiment, removing the group's reliance on the less compatible LabView program. In January 2020, Dr. Hyok Sang Han and I observed a mysterious transient decay flash in the Rb population that was much faster than even the fastest superradiance mode of the system. Since this phenomenon was only recently discovered and not well-understood in the 1-dimensional geometry of our nanofiber experiment, the group at JQI is constructing a model for this behavior.   \\ 

At Colby, I work on experiments with ultracold potassium (K) under Professor Charles Conover, my advisor and mentor. Applying the experimental techniques I learned at JQI, I am working towards a Physics Honors Thesis on lifetime measurements of quantum states in potassium. In previous years, I have constructed a variety of laboratory apparatus from external cavity diode lasers to laser frequency-stabilization electronics and carried out millimeter-wave precision measurements on Rydberg states of K to determine quantum defects and absolute energy levels. At our level of precision, energy shifts due to the excitation source are significant and thus require data extrapolation to obtain an unbiased measurement. By applying Ramsey's separated oscillatory field method, I eliminated this necessity and gave an alternative measurement scheme with comparable precision. I presented this work at Colby's summer research symposium in 2018 and at DAMOP 19.  \\

Besides physics research, I have been developing my communication skills through working as a tutor and teaching assistant for a wide range of courses from Single-variable Calculus to Quantum Mechanics and exploring other areas of physics and mathematics. For four semesters with Professor Robert Bluhm I actively studied general relativity and classical field theory, and reviewed massive gravity along with its nonlinear effects. This resulted in my detailed exposition of the subjects, which is available on my website. Last year, I began my Mathematics Honors Thesis under Professor Evan Randles at Colby on convolution powers of complex-valued functions. The correspondence between the convolution and the Fourier transform makes convolution powers central for generating solutions for partial differential equations. Motivated by my numerical evidence for a local limit conjecture, we recently constructed from measure theory an integration formula for estimating the Fourier transform of special surface-carried measures. We are preparing a manuscript summarizing this result and will present it at the Joint Mathematics Meeting in 2021. \\ 

At MIT, I would like to continue and focus my efforts on researching quantum information and condensed-matter physics. I have contacted Professor Soonwon Choi, who will be starting at MIT in July 2021. His research on quantum many-body systems and quantum information dynamics aligns directly with my current interests. I also contacted Professor Martin Zwierlein and am interested in his upcoming experiment on rotating quantum gases to investigate quantum Hall physics. Finally, I am attracted to both theoretical and experimental aspects of Professor Isaac Chuang's research. Upon receiving my Ph.D., I aim to continue research in these fields and eventually teach at a research-oriented university. I believe that admission to the Physics Ph.D. program at MIT is an excellent first step towards this goal. Thank you for your consideration.
	

















	
	
	
	
	
\end{document}