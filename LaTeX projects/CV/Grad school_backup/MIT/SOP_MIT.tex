\documentclass[12pt]{article}
\usepackage[left=1in, right=1in,top=1.25in,bottom=1in]{geometry}

%\usepackage{newpxtext,newpxmath}
%\usepackage{amsfonts}
\usepackage{tgtermes}
\usepackage{color}
\usepackage{bold-extra}
\usepackage{setspace}
%\pagenumbering{gobble}
\usepackage{setspace}
\setstretch{1.1}
%\onehalfspacing

%\fontsize{12pt}{15pt}\selectfont

\usepackage{fancyhdr}
\pagestyle{fancy}
\lhead{Huan Q. Bui}
\rhead{Application for MIT Physics}
%\cfoot{center of the footer!}
%\renewcommand{\headrulewidth}{0.4pt}
%\renewcommand{\footrulewidth}{0.4pt}

\begin{document}
%\noindent Huan Quang Bui \hfill Application for MIT Physics
\begin{center}
	\textbf{Statement of Objectives}
\end{center}
I am captivated by the interplay between theory and experiment in quantum information science, condensed matter, and atomic, molecular, and optical physics. In particular, I am interested in theoretical quantum simulation and applications of ultracold gases in quantum science and fundamental physics. In May 2021, I will complete my undergraduate studies at Colby College with two Honors Theses on experimental atomic physics and mathematical analysis. My next objective is to expand my research training and subsequently apply my skills to address an important open problem in quantum science. To that end, the Physics Ph.D. program at MIT is a fantastic option.  \\ 

My most recent interest is in simulation of quantum many-body systems. With Dr. Timothy Hsieh at the Perimeter Institute for Theoretical Physics, I am researching efficient variational simulation of quantum states that are not adiabatically connected to unentangled product states. Motivated by variational quantum eigensolver algorithms and the Quantum Approximate Optimization Algorithm (QAOA), Dr. Hsieh recently developed a protocol that could target a class of such non-trivial  states with perfect fidelity using an L/2-deep QAOA ansatz, where L is the system size. We initially aimed to accelerate this protocol; however, after I discovered that it could also target the ground state of the transverse-field Ising model with random field and couplings, the research focus shifted towards understanding how QAOA provides such a reliable ansatz. Since then, I have further found numerically that an (L+1)- and a modified L/2-deep ansatz could simulate with near-perfect fidelity excited states of this model and the ground state of any disordered Ising Hamiltonian, respectively. My next goal is to explore Dr. Hsieh's conjectures related to this fascinating behavior of the QAOA ansatz. Ultimately, we aim to establish relationships between the current protocol and the ability to target many-body localized states.\\

My interest in quantum information science stems from my research experiences in ultracold atom experiments at the Joint Quantum Institute (JQI) and Colby College. In Summer 2019, I joined the Rolston group at JQI to study the long-range interaction between rubidium atoms magneto-optically trapped around an optical nanofiber. There, I built an imaging system for optimizing light polarization in optical nanofibers, which often introduce birefringence and undesirable longitudinal polarizations. I also developed a stand-alone Python program for controlling the entire experiment, removing the group's reliance on the less compatible LabView program. In January 2020, Dr. Hyok Sang Han and I observed a mysterious transient decay flash in the rubidium population that was much faster than even the fastest superradiance mode of the system. Since this phenomenon was only recently discovered and not well-understood in the one-dimensional geometry of our nanofiber experiment, the group at JQI is currently constructing a model to study it in more detail.   \\ 

Since my first year at Colby, I have been working on ultracold potassium experiments under Professor Charles Conover, my advisor and mentor. Applying the experimental techniques I learned at JQI, I am currently working towards a Physics Honors Thesis on lifetime measurements of quantum states in potassium by counting photons emitted from an excited, magneto-optically trapped, atomic cloud of this species. In previous years, I have constructed a variety of laboratory apparatus from external-cavity diode lasers to laser frequency-stabilization electronics and carried out precision spectroscopy on potassium in Rydberg states to determine its quantum defects and absolute energy levels. At our level of precision, energy shifts due to the millimeter-wave source are significant and thus require data extrapolation to obtain unbiased measurements. Applying Ramsey's separated oscillatory field method, I eliminated this necessity and gave an alternative measurement scheme with comparable precision. I presented this work at Colby's summer research symposium in 2018 and at DAMOP 19.  \\

Besides physics research, I also explore other areas of physics and mathematics. For four semesters with Professor Robert Bluhm, I studied general relativity, classical, and quantum field theory and reviewed massive gravity along with its nonlinear effects. This resulted in my detailed expositions of these topics, which are available on my website. Last year, I began my Mathematics Honors Thesis under Professor Evan Randles on convolution powers of complex-valued functions, which are a central object for generating numerical solutions for partial differential equations. Motivated by my numerical evidence, we have developed a generalization of the polar-coordinate integration formula which aids in the understanding of convolution powers and semigroups of pseudo-differential operators. We are preparing a manuscript summarizing this result and will present it at the Joint Mathematics Meeting in January 2021. These projects, along with my role as a physics and mathematics teaching assistant in which I develop teaching and communication skills, enrich my academic experience outside of coursework and physics research.  \\ 

At MIT, I would like to focus my efforts on conducting research at the intersection of quantum information science, condensed matter, and atomic, molecular, and optical physics as well as expanding my teaching capacity. I have contacted Professor Soonwon Choi, who will be starting at MIT in July 2021. His research on quantum many-body systems and quantum information dynamics aligns directly with my current interests. I also contacted Professor Martin Zwierlein and am interested in his upcoming experiment on rotating quantum gases to investigate quantum Hall physics. Finally, I am attracted to both theoretical and experimental aspects of Professor Isaac Chuang's work. Upon receiving my Ph.D., I aim to continue research in these fields and eventually teach at a research-oriented university. In my future work, I would like to make contributions to the growing body of work in quantum science. I believe that admission to the Physics Ph.D. program at MIT is an excellent first step towards this goal. \\

\noindent Thank you for your consideration. \\

\noindent \today
	











	
	
	
	
	
\end{document}