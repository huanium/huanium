\documentclass[12pt]{article}
\usepackage[left=1in, right=1in,top=1in,bottom=1in]{geometry}

%\usepackage{newpxtext,newpxmath}
\usepackage{amsfonts}
\usepackage{tgtermes}
\usepackage{color}
\usepackage{bold-extra}
\usepackage{setspace}
\pagenumbering{gobble}

\usepackage{fancyhdr}
\pagestyle{fancy}
\lhead{Huan Q. Bui}
\rhead{\today \thepage}
%\cfoot{center of the footer!}
%\renewcommand{\headrulewidth}{0.4pt}
%\renewcommand{\footrulewidth}{0.4pt}

\begin{document}
\begin{center}
	\textbf{Statement of Objectives (MIT)}
\end{center}
I am drawn to the interplay between theory and experiment in quantum information and condensed-matter physics. My research training at Colby, the Joint Quantum Institute (JQI), and the Perimeter Institute has shaped my interests in table-top experiments as well as theoretical quantum simulations. I will soon complete my undergraduate studies with two Honors Theses on experimental atomic physics and mathematical physics. The natural next step for me is to apply my training to address a fundamental open problem in quantum science. I believe that MIT Physics is a fantastic option to that end because of the range of novel problems currently under investigation and the highly collaborative environment between theorists and experimentalists.   \\ 

Under the supervision of Dr. Timothy Hsieh at Perimeter Institute, I have been researching efficient variational simulation of non-trivial quantum states using the quantum approximate optimization algorithm (QAOA) ansatz in a quantum-classical hybrid setting. Recently, Dr. Hsieh and colleagues have developed one such protocol to target with perfect fidelity a class of non-trivial states using an L-deep circuit, where L is the system size. My project initially explored the possibility of improving Dr. Hsieh's protocol to achieve a sublinear-depth circuit by incorporating aspects of measurement-based quantum computation such as the cluster state and projective measurements. Surprisingly, while benchmarking the original algorithm, I discovered numerically that it is capable of perfectly simulating the ground state of not only any uniform transverse-field Ising model (TFIM) as found by Dr. Hsieh, but also any TFIM with random transverse field and couplings. This finding steered my research towards understanding how the QAOA ansatz consistently gives perfect fidelity. To this end, I will explore Dr. Hsieh's conjecture on the correspondence between the wavefunction picture and free-fermion picture of the problem under the Jordan-Wigner transformation. Regarding extensions and applications of the current protocol, I recently showed that an (L+1)-deep QAOA ansatz could perfectly target a large class of excited states for any TFIM with random field and couplings. From here, I will be targeting disordered TFIM's, which cannot be put into a free-fermion form. Ultimately, Dr. Hsieh and I aim to establish relationships between this protocol and the ability to target many-body localized states.\\

In the summer of 2019, I joined the Rolston group at JQI to experimentally study the long-range interactions between Rb atoms magneto-optically trapped (MOT) around an optical nanofiber. With Dr. Hyok Sang Han's guidance, I created an imaging system for optimizing light polarization in nanofibers. The medium inside a nanofiber often introduces birefringence and a non-uniform longitudinal polarization; thus, this system was necessary to guarantee at least  quasi-linearly polarized light. Additionally, I developed a Python program using the NI-DAQmx libraries for controlling the entire Rb experiment, removing the group's reliance on the less compatible LabView. In January 2020, while working on the experiment, Dr. Han and I observed a mysterious transient decay flash that was much faster than even the fastest superradiance mode of the system. Since this phenomenon was only recently discovered for a 3-dimensional atom cloud and is not yet understood particularly in the 1-dimensional geometry of our nanofiber experiment, the team at JQI is constructing a model to describe and explain this behavior. This promises new insights into the phenomenon.  \\ 

At Colby, I work on ultracold atom experiments under Professor Charles Conover. Applying the experimental techniques I learned at JQI, I am currently working towards a Physics Honors Thesis on lifetime measurements of certain quantum states in potassium by counting photons fluorescing from a MOT cloud of excited K atoms. In previous years, I have constructed a variety of laboratory apparatus from external cavity diode lasers to laser frequency-stabilization electronics and carried out millimeter-wave precision measurements of energy levels and quantum defects on K in Rydberg states. From summer 2018 to spring 2019, I measured the $\mbox{nd}_{j} \to \mbox{(n+1)d}_{j}$ two-photon transitions for $\mbox{30} \leq \mbox{n} \leq \mbox{35}$ to determine d-state quantum defects and a range of absolute energy levels of K. At our level of precision, AC Stark shifts due to the millimeter-wave source are significant, thus requiring data extrapolation to obtain an unbiased measurement. However, by applying Ramsey's separated oscillatory field method, I eliminated this necessity and gave an alternative measurement scheme with comparable precision. I presented this work at Colby's summer research symposium CUSRR 2018 and at DAMOP 19 in Milwaukee, Wisconsin jointly with Professor Conover. Our group also presented measurements of the p-state fine structure and quantum defects at DAMOP 19 and f-, g-, and h-state quantum defects at DAMOP 20. \\ 

Besides pursuing my primary interests, I have been developing my communication skills by working as a teaching assistant for a range of mathematics and physics courses from Linear Algebra to Quantum Mechanics. I also self-study quantum field theory and actively explore general relativity and mathematical physics. Having enjoyed General Relativity with Professor Robert Bluhm at Colby in Fall 2018, I subsequently took three independent studies with him on classical field theory, specifically to better our understanding of massive gravity. I reviewed the theory's development and the origins of nonlinear behaviors such as the Vainshtein radius and screening mechanisms. The project resulted in my 150-page exposition of this topic, which is available on my website. Following original works, I gave explicit derivations of important results and filled in knowledge gaps to make the document accessible to the undergraduate level. These gaps range from elementary quantum field theory to how to use Mathematica \texttt{xPert} and \texttt{xAct} packages for perturbative relativity. \\ 

Last year, I began my Mathematics Honors Thesis under Professor Evan Randles at Colby on convolution powers of certain complex-valued functions. The correspondence between the convolution and the Fourier transform makes convolution powers a central object for generating solutions for partial differential equations. Using Python for computing convolution powers and Mathematica for Fourier transforms, I provided numerical evidence for a new local limit conjecture related to this problem. Motivated by this, we recently made progress towards proving the conjecture by constructing a measure-theoretic quasi-polar-coordinate integration formula for estimating the Fourier transform of a class of surface-carried measures. We are preparing a manuscript summarizing this new result and will present it at the Joint Mathematics Meeting in January 2021. \\ 

MIT offers unmatched research opportunities in quantum information and condensed-matter physics: I have corresponded with Professor Soonwon Choi, who will be starting at MIT in July 2021. His research on quantum many-body systems and quantum information dynamics aligns directly with my current interests. I also contacted Professor Martin Zwierlein and am interested in his upcoming experiment on rotating quantum gases to investigate quantum Hall physics. Finally, I am attracted to both theoretical and experimental aspects of Professor Isaac Chuang's research. Upon receiving my Ph.D., my goal is to continue research in quantum information and condensed-matter physics and eventually teach at a research-oriented university. I believe that admission to the Physics Ph.D. program at MIT will make this possible. Thank you for your consideration.
	

















	
	
	
	
	
\end{document}