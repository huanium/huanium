\documentclass[11pt]{article}
\usepackage{amsmath}
\usepackage{physics}
\usepackage{amssymb}
\usepackage{graphicx}
\usepackage{hyperref}
\usepackage{amsfonts}
\usepackage{cancel}
\usepackage{xcolor}
\hypersetup{
	colorlinks,
	linkcolor={black!50!black},
	citecolor={blue!50!black},
	urlcolor={blue!80!black}
}
\newcommand{\f}[2]{\frac{#1}{#2}}
\usepackage{newpxtext,newpxmath}
\usepackage[left=1.25in,right=1.25in,top=1.25in,bottom=1.25in]{geometry}
\usepackage{framed}
\usepackage{caption}
\usepackage{subcaption}





\begin{document}
\begin{center}
{\large \bf PH312: Physics of Fluids (Prof. McCoy) -- Reflection}\\
{ Huan Q. Bui}\\
April 1, 2021
\end{center}




I don't have much to say for this week. I just think that it's very cool and satisfying to see how familiar laws like the parabolic flow profile or the linear drag relation come out of the Navier-Stokes equation. I also like our discussion of creeping flow around a spherical object. I think that technique used to solve the original PDE (taking the curl of both sides) is very neat. I don't think I would have come up with that myself either. I also look forward to working through this week's HW. The PDE-related stuff looks very interesting, and I would love to see for myself how the linear drag equation was derived.  



  
\end{document}




