\documentclass[11pt]{article}
\usepackage{amsmath}
\usepackage{physics}
\usepackage{amssymb}
\usepackage{graphicx}
\usepackage{hyperref}
\usepackage{amsfonts}
\usepackage{cancel}
\usepackage{xcolor}
\hypersetup{
	colorlinks,
	linkcolor={black!50!black},
	citecolor={blue!50!black},
	urlcolor={blue!80!black}
}
\newcommand{\f}[2]{\frac{#1}{#2}}
\usepackage{newpxtext,newpxmath}
\usepackage[left=1.25in,right=1.25in,top=0.9in,bottom=0.9in]{geometry}
\usepackage{framed}
\usepackage{caption}
\usepackage{subcaption}





\begin{document}
\begin{center}
{\large \bf PH312: Physics of Fluids (Prof. McCoy) -- Reflection}\\
{ Huan Q. Bui}\\
Mar 11 2021
\end{center}

\begin{framed}
	\noindent Reflection on the readings:
	\begin{enumerate}
		\item Kundu \& Cohen 1.6, 1.7
	\end{enumerate}
\end{framed}

\noindent I really like the historical aspects of our discussion in PH312 this week. While the events we went over in class (Torricelli, Galileo, Pascal, etc.), it was still nice to hear about how the idea of ``pressure'' came to be. \\

\noindent I also find the derivation of the equation of hydrostatic quite neat. Intuitively, it isn't very clear in the beginning why the pressure gradient is proportional to the density of the fluid and the gravitational field. But this turns out to just be a result of Newton's 2nd law and how the stress tensor works (for fluids at rest). \\

\noindent The discussion on surface tension was interesting. Before, I couldn't quite imagine how surface tension is called ``line force'' until I saw Figure 1.4(a) in Kundu and Cohen, which was very helpful. I would like to see how the derivation for the pressure jump at a non-spherical interface (I should just look this up later).  

  
\end{document}




