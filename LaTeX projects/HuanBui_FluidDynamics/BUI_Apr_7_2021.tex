\documentclass[11pt]{article}
\usepackage{amsmath}
\usepackage{physics}
\usepackage{amssymb}
\usepackage{graphicx}
\usepackage{hyperref}
\usepackage{amsfonts}
\usepackage{cancel}
\usepackage{xcolor}
\hypersetup{
	colorlinks,
	linkcolor={black!50!black},
	citecolor={blue!50!black},
	urlcolor={blue!80!black}
}
\newcommand{\f}[2]{\frac{#1}{#2}}
\usepackage{newpxtext,newpxmath}
\usepackage[left=1.25in,right=1.25in,top=1.25in,bottom=1.25in]{geometry}
\usepackage{framed}
\usepackage{caption}
\usepackage{subcaption}





\begin{document}
\begin{center}
{\large \bf PH312: Physics of Fluids (Prof. McCoy) -- Reflection}\\
{ Huan Q. Bui}\\
April 8, 2021
\end{center}



I found many aspects in our ``Life at low Re'' discussion very surprising, especially the example we did where we compared skin versus pressure drag for different orientations of a flat plate. It is very interesting that the relative strength of these drag forces depends on how large Re is, which really emphasizes the omnipresent nonlinearity in fluid dynamics. The Scallop Theorem is also surprising but makes intuitive sense. I think it is also the reason why we shouldn't wiggle too fast when stuck in quicksand: at such low Re, wiggling becomes a time-reversible process, and you get nowhere trying to escape with this technique. \\

The vortex lines/tubes discussion draws a lot of parallels to electrostatics, which I think helps with visualizing and understanding that these seemingly fictitious objects are actually material (which is a bit odd because imagining E\&M fields should be less intuitive than fluid flow). The Kelvin-Helmholtz vortex theorems sort of bring together many concepts such as vorticity, conservation of circulation, irrotationality of flow, etc.  which is very nice and has surprising consequences in inviscid flow. 

  
\end{document}




