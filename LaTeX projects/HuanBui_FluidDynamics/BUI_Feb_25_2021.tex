\documentclass[11pt]{article}
\usepackage{amsmath}
\usepackage{physics}
\usepackage{amssymb}
\usepackage{graphicx}
\usepackage{hyperref}
\usepackage{amsfonts}
\usepackage{cancel}
\usepackage{xcolor}
\hypersetup{
	colorlinks,
	linkcolor={black!50!black},
	citecolor={blue!50!black},
	urlcolor={blue!80!black}
}
\newcommand{\f}[2]{\frac{#1}{#2}}
\usepackage{newpxtext,newpxmath}
\usepackage[left=1in,right=1in,top=0.9in,bottom=0.9in]{geometry}
\usepackage{framed}
\usepackage{enumerate}

\usepackage{caption}
\usepackage{subcaption}

%\newcommand{\fig}[1]{figure #1}
%\newcommand{\explain}{appendix?}
%\newcommand{\rat}{\mathbb{Q}}
%
%\newcommand{\mathbb{R}}{\mathbb{R}}
%\newcommand{\nat}{\mathbb{N}}
%\newcommand{\inte}{\mathbb{Z}}
%\newcommand{\M}{{\cal{M}}}
%\newcommand{\sss}{{\cal{S}}}
%\newcommand{\rrr}{{\cal{R}}}
%\newcommand{\uu}{2pt}
%\newcommand{\vv}{\vec{v}}
%\newcommand{\comp}{\mathbb{C}}
%\newcommand{\field}{\mathbb{F}}
%\newcommand{\f}[1]{ \hspace{.1in} (#1) }
%\newcommand{\set}[2]{\mbox{$\left\{ \left. #1 \hspace{3pt}
%\right| #2 \hspace{3pt} \right\}$}}
%\newcommand{\integral}[2]{\int_{#1}^{#2}}
%\newcommand{\ba}{\hookrightarrow}
%\newcommand{\ep}{\varepsilon}
%\newcommand{\limit}{\operatornamewithlimits{limit}}
%\newcommand{\ddd}{.1in}
%\newcommand{\ccc}{2in}
%\newcommand{\aaa}{1.5in}
%\newcommand{\B}{{\cal B}}
%\newcommand{\C}{{\cal C}}
%\newcommand{\D}{{\cal D}}
%\newcommand{\FF}{{\cal F}}
\usepackage{amssymb}% http://ctan.org/pkg/amssymb
\usepackage{pifont}% http://ctan.org/pkg/pifont
\newcommand{\cmark}{\ding{51}}%
\newcommand{\xmark}{\ding{55}}%






\begin{document}
\begin{center}
{\large \bf PH312: Physics of Fluids (Prof. McCoy) -- Reflection}\\
{ Huan Q. Bui}\\
Feb 26 2021
\end{center}

\noindent \textbf{1.} Explain why particle paths, streamlines, and streaklines are identical for \textit{steady flow}.
\begin{enumerate}[(a)]
	\item In steady flow, the velocity field is  time-independent, i.e., $\mathbf{u}(\mathbf{r},t) = \mathbf{u}(\mathbf{r})$, and thus streamlines (which are just velocity field lines) are also time-independent. Let a particle $p$ be moving instantaneously along some streamline $S$. Because $S$ is unchanging and that the tangent of $S$ at $p$ is always along the velocity of $p$, $p$ must move along $S$.   
	
	\item Let $p$ be moving along $S$. Suppose $p$ moves onto some streamline $S_1 \neq S$. From (a), we know that $S_1$ and $S$ must cross, but this is impossible because $p$ will then have velocities in two direction. Two streamlines cannot intersect except at a position of zero velocity.
	
	\item Let particles $\{p\}$ make up some streakline. By definition, each of $\{p\}$ has passed through one point belonging to some streamline $S$. From (b), we know that none of $\{p\}$ leaves $S$, and so all of $\{p\}$ must follow $S$. 
\end{enumerate}

  
  
  
\noindent \textbf{2.} To transform the flow field in Figure 6.3(a) where flow is viewed from the bank to Figure 6.3(b) where flow is viewed from the ship, we add $-\mathbf{U}$ (which points Left-Right) to each velocity vector $\mathbf{u}'$ in (a). For example, the point $p$ in (a) with $\mathbf{u}'$ pointing Up-Left becomes $\mathbf{u} = $ Up-Left + Right $\approx$ Up-Right. The velocity may change direction in some cases depending on how $\mathbf{U}$ to compares to $\mathbf{u}'$ in direction and amplitude. For example, the regions in (a) where the flow is Right-to-Left are where it was previously moving downstream in (b) \textbf{more slowly} than $\mathbf{U}$.  \\






\noindent \textbf{3.} 
\begin{enumerate}[(a)]
	\item $a = b_i c_{ij} d_j$ \cmark 
	\item $a = b_i c_i  + d_j$ \xmark 
	
	The LHS has no free index, but the RHS does. So, this is not allowed according to the convention (p.28, K\&C: the free index must appear on both sides). If we want to say that the component $d_j$ is equal to the scalar $a - b_i c_i$ for all $j$ then we have to be explicit to avoid confusion. 
	
	\item $a_i = \delta_{ij} b_i + c_i$ \xmark  
	
	The contraction $\delta_{ij}b_i$ leaves $b_j$ with free index $j$, which is incompatible with the free index $i$ on $a_i$ and $c_i$. 
	
	\item $a_k = b_k c + d_i e_{ik} = cb_k +  d_i e_{ik}$ \cmark 
	\item $a_i = b_i + c_{ij} d_{ij} e_i$ \xmark 
	
	The index $i$ appears 3 times in the second term on the RHS. This can cause confusion.
\end{enumerate}
$\,$


\noindent \textbf{4.} Let $A$ be a second-order tensor and $R$ be a rotation matrix. Because $R^\top R = \mathbb{I}$, i.e., $(R^\top R)_{mn} = \delta_{mn}$, we have 
\begin{equation*}
A'_{ii} = R_{im}R_{in}A_{mn} = {R^\top}_{mi}R_{in} A_{mn} = (R^\top R)_{mn}A_{mn} = \delta_{mn}A_{mn} = A_{mm} = A_{ii}.
\end{equation*}
So, $A_{ii}$, or the trace of the matrix of $A$, is invariant under coordinate rotations. \\


\noindent \textbf{5.} Let $R$ be a rotation matrix. Because $R^\top R = \mathbb{I} = RR^\top$, i.e., $(RR^\top)_{ij} = \delta_{ij}$ , we have
\begin{equation*}
\delta'_{ij} = R_{im}R_{jn} \delta_{mn} = R_{im} R_{jm} = R_{im} {R^\top}_{mj} = (RR^\top)_{ij} = \delta_{ij}.
\end{equation*}
So, $\delta_{ij}$ is an isotropic tensor. 
\end{document}






