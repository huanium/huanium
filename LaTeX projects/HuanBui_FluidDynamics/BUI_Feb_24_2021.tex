\documentclass[11pt]{article}
\usepackage{amsmath}
\usepackage{physics}
\usepackage{amssymb}
\usepackage{graphicx}
\usepackage{hyperref}
\usepackage{amsfonts}
\usepackage{cancel}
\usepackage{xcolor}
\hypersetup{
	colorlinks,
	linkcolor={black!50!black},
	citecolor={blue!50!black},
	urlcolor={blue!80!black}
}
\newcommand{\f}[2]{\frac{#1}{#2}}
\usepackage{newpxtext,newpxmath}
\usepackage[left=1.25in,right=1.25in,top=0.9in,bottom=0.9in]{geometry}
\usepackage{framed}
\usepackage{caption}
\usepackage{subcaption}





\begin{document}
\begin{center}
{\large \bf PH312: Physics of Fluids (Prof. McCoy) -- Reflection}\\
{ Huan Q. Bui}\\
Feb 25 2021
\end{center}

\begin{framed}
	\noindent Reflection on the readings:
	\begin{enumerate}
		\item Tritton 5.1 - 5.3, 5.5, 6.1, 6.2; Kundu \& Cohen 3.1 - 3.5
		\item Kundu \& Cohen 2.1 - 2.4, 3.6, 3.7
		\item Kundu \& Cohen 2.5, 2.7 - 2.9
	\end{enumerate}
\end{framed}


\noindent There are two main ideas I learned this week from the readings and lectures: \textit{Lagrangian} versus \textit{Eulerian} description of fluid motion, and the \textit{material/substantial/total derivative} (which actually came up because of the distinction between the Lagrangian and Eulerian language). \\

The main distinction between the Lagrangian and Eulerian specifications are essentially relativity and what we choose to focus on. The Lagrangian view describes particle trajectories, while the Eulerian view focuses how fields (of fluid property) change for any given point in space-time. I find the distinction between these views very subtle. Even though it is easy to imagine how these views are different, translating one description to another more involved than expected. \\

An example is of course the introduction of the substantive/material/total derivative, which basically tells us the rate of change of some fluid property $F$ \textit{along the flow}, i.e., the material derivative of $F$ gives how $F$ changes not just in time $t$ when subjected to a velocity field that is spatially and temporally dependent. Mathematically, this can be seen from the fact that $F$ depends not just on $t$ but also on intermediate variables $x_i = x_i(t)$. The chain rule gives us formula we saw in class:
\begin{equation*}
\frac{DF}{Dt} = \frac{\partial F}{\partial t} + \frac{\partial F}{\partial x_i} \f{\partial x_i}{\partial t} = \frac{\partial F}{\partial t} + u_i\frac{\partial F}{\partial x_i}  = \left( \partial_t + u \cdot \grad \right) [F],
\end{equation*}
where we can interpret $\partial F/\partial t$ as the local rate of change of $F$ at a given point, and the second term adds a ``correction'' to this change as we go along the flow of the fluid element. \\

The readings also focused heavily on streamlines, streaklines, and particle path and when they coincide. I find this discussion very subtle as well, because steady flow in one frame need not be steady in a different frame. The example both books give was flow around an obstacle in the moving versus rest frame of the fluid. Figure 3.8(a) in Kundu \& Cohen's and the pair Figures 6.3-4 of Tritton's actually took me some figuring out, since (I think) we're much more familiar with the frame in which the obstacle (ship/boat/etc) is at rest. 


  
\end{document}




