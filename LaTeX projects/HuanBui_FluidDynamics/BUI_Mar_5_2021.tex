\documentclass[11pt]{article}
\usepackage{amsmath}
\usepackage{physics}
\usepackage{amssymb}
\usepackage{graphicx}
\usepackage{hyperref}
\usepackage{amsfonts}
\usepackage{cancel}
\usepackage{xcolor}
\hypersetup{
	colorlinks,
	linkcolor={black!50!black},
	citecolor={blue!50!black},
	urlcolor={blue!80!black}
}
\newcommand{\f}[2]{\frac{#1}{#2}}
\usepackage{newpxtext,newpxmath}
\usepackage[left=1.25in,right=1.25in,top=1.25in,bottom=1.25in]{geometry}
\usepackage{framed}
\usepackage{enumerate}

\usepackage{caption}
\usepackage{subcaption}

%\newcommand{\fig}[1]{figure #1}
%\newcommand{\explain}{appendix?}
%\newcommand{\rat}{\mathbb{Q}}
%
%\newcommand{\mathbb{R}}{\mathbb{R}}
%\newcommand{\nat}{\mathbb{N}}
%\newcommand{\inte}{\mathbb{Z}}
%\newcommand{\M}{{\cal{M}}}
%\newcommand{\sss}{{\cal{S}}}
%\newcommand{\rrr}{{\cal{R}}}
%\newcommand{\uu}{2pt}
%\newcommand{\vv}{\vec{v}}
%\newcommand{\comp}{\mathbb{C}}
%\newcommand{\field}{\mathbb{F}}
%\newcommand{\f}[1]{ \hspace{.1in} (#1) }
%\newcommand{\set}[2]{\mbox{$\left\{ \left. #1 \hspace{3pt}
%\right| #2 \hspace{3pt} \right\}$}}
%\newcommand{\integral}[2]{\int_{#1}^{#2}}
%\newcommand{\ba}{\hookrightarrow}
%\newcommand{\ep}{\varepsilon}
%\newcommand{\limit}{\operatornamewithlimits{limit}}
%\newcommand{\ddd}{.1in}
%\newcommand{\ccc}{2in}
%\newcommand{\aaa}{1.5in}
%\newcommand{\B}{{\cal B}}
%\newcommand{\C}{{\cal C}}
%\newcommand{\D}{{\cal D}}
%\newcommand{\FF}{{\cal F}}
\usepackage{amssymb}% http://ctan.org/pkg/amssymb
\usepackage{pifont}% http://ctan.org/pkg/pifont
\newcommand{\cmark}{\ding{51}}%
\newcommand{\xmark}{\ding{55}}%
\newcommand{\p}{\partial}%
%\usepackage{MnSymbol,wasysym}

\begin{document}
\begin{center}
{\large \bf PH312: Physics of Fluids (Prof. McCoy)}\\
{ Huan Q. Bui}\\
Mar 05 2021
\end{center}

\noindent \textbf{1.} We find the concept of \textit{density} more useful when talking about fluids because a fluid, unlike rigid bodies, can change shape/contract/expand or flow from one place to another and therefore the fluid mass at any local region may vary. This is not the case with density, which is independent of the spatial distribution of the fluid. \\ 




\noindent \textbf{2.} \textit{Stress} has units of pressure, which is force per area. It is easier to talk about stress in fluid dynamics because it is independent of the amount of material upon which forces act.  \\


\noindent \textbf{3.} [P3 in Ch. 3 of K\&C] Transforming into polar coordinates, we have $\vec{u} = (ay,0) = (ar\sin\theta,0)$. An infinitesimal line element on the unit circle has the form $d\vec{s} = (-\sin\theta,\cos\theta)\,d\theta$, and so an infinitesimal contribution to the line integral is $\vec{u} \cdot d\vec{s} = (ar\sin\theta,0)\cdot (-\sin\theta,\cos\theta) \, d\theta= -ar\sin^2\theta\,d\theta$. With $C:$ $r\equiv 1$ and $\theta = [0,2\pi]$, we compute $\Gamma$:
\begin{equation*}
\Gamma = \oint_C \vec{u}\cdot d\vec{s} = -a\int^{2\pi}_0  \sin^2\theta \,d\theta = -a\left( \frac{\theta}{2} - \frac{1}{4}\sin2\theta \right)\bigg\vert_0^{2\pi} = -a\pi. 
\end{equation*}
Alternatively, Sir Stokes tells us that
\begin{equation*}
\Gamma = \int_A \vec{\omega} \cdot d\vec{A} = \int_A \left(\grad \times \vec{u}\right)\cdot d\vec{A} = \int_A (0,0,-a)\cdot (0,0,1) dA = -a\int_A \,dA = -a\pi,
\end{equation*}
where we have used $d\vec{A} = \hat{n}\,dA = (0,0,1)\,dA$, and $A = \pi$, the area of the unit disk. \\







\noindent \textbf{4.} 
\begin{enumerate}[(a)]
	\item Since $e_{ij}$ is a symmetric tensor, it is orthogonally diagonalizable with real spectrum, i.e., we can orthogonally transform the standard frame into one in which $e_{ij} \to e'_{ij}$ is diagonal. In this frame, the original sphere in stretched/contracted along its (orthogonal) principal axes, resulting in an ellipsoid. 
	
	\item In class, we put
	\begin{equation*}
	e_{ij}\,dx_j = \f{1}{3}e_{kk}\delta_{ij}\,dx_j + \left( e_{ij} - \f{1}{3}e_{kk}\delta_{ij}\right)\,dx_j.
	\end{equation*}
	The first term is a scalar times the unit tensor, so it represents isotropic expansion and contraction and is responsible for \textit{all} volume change (if there is any).  The second term describes volume-preserving deformation. We can infer the volume-preserving property from the fact that 
	\begin{equation*}
	 \tr\left( e_{ij} - \f{1}{3}e_{kk}\delta_{ij}\right) =  e_{ii} - \f{1}{3}e_{kk}\delta_{ii} = \grad \cdot \vec{u} - \f{3}{3} \grad\cdot \vec{u} = 0.
	\end{equation*}
	
	\item Write $\p_i = \p/\p x^i$, then we have $\omega_i = (\grad \times \vec{u})_i = \epsilon_{ijk}\p_j u_k$, and so $\epsilon_{abi}\omega_i = \epsilon_{abi}\epsilon_{ijk}\p_j u_k$. The only non-trivial terms are those with $jk = ab$ and $jk = ba$. If $jk = ab$ then $\epsilon_{abi}\epsilon_{iab} = 1$ (to see this, fix $i=1$, and any summand with $ab$ containing $1$ vanishes). Similarly, if $jk = ba$ then $\epsilon_{abi}\epsilon_{iba} = -1$. Putting everything together, we find the desired result
	\begin{equation*}
	\epsilon_{abi}\omega_i = \p_a u_b - \p_b u_a = -\left( \p_b u_a - \p_a u_b \right)  = -r_{ab}.
	\end{equation*}
	Alternatively, we can follow K\&C's approach to verify the equality above.\\
	
	
	\noindent Geometrically, $r_{ij}$ is a rotation tensor. To see this, we let $r_{ij}$ act on some $v_j$
	\begin{equation*}
	r_{ij}v_j = -\epsilon_{ijk}\omega_kv_j = \epsilon_{ikj}\omega_k v_j = (\vec{\omega}\times \vec{v})_i.
	\end{equation*} 
	This is the $i$th component of the velocity at a distance $\vec{v}$ from the axis of rotation of a body rotating rigidly at angular velocity $\omega$.  
\end{enumerate}
$\,$


\noindent \textbf{5.} We have $\grad\cdot  \vec{u} = 0$ and $\grad \times \vec{u} = \vec{0}$ everywhere. 
\begin{enumerate}[(a)]
	\item The given flow field is divergence-free, so there is no volume change, locally. The flow field is also curl-free, so it is irrotational (no vorticity). We do not know whether there is (macroscopic) rotation or translation (both the non-zero constant and the zero flow field satisfy the above two constraints). 
	
	\item Suppose $\vec{u} = \grad \phi$, where $\grad^2 \phi =0$. Then we have
	\begin{equation*}
	\grad \cdot \vec{u} = \p_i u_i = \p_i \p_i \phi = \grad^2 \phi = 0
	\end{equation*}
	and
	\begin{equation*}
	\left(\grad \times \vec{u}\right)_i = \epsilon_{ijk}\p_j u_k = \epsilon_{ijk}\p_j \p_k \phi = 0.
	\end{equation*}
	The second fact doesn't require $\grad^2 \phi = 0$, since $\grad \phi$ is a conservative vector field. In electrostatics (with no source), $\phi = -V$, where $V$ is the electric potential, and $\vec{u}$ is the electric field. 
\end{enumerate}
  
\end{document}




