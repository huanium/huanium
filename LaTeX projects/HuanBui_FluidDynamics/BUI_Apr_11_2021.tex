\documentclass[11pt]{article}
\usepackage{amsmath}
\usepackage{physics}
\usepackage{amssymb}
\usepackage{graphicx}
\usepackage{hyperref}
\usepackage{amsfonts}
\usepackage{cancel}
\usepackage{xcolor}
\hypersetup{
	colorlinks,
	linkcolor={black!50!black},
	citecolor={blue!50!black},
	urlcolor={blue!80!black}
}
\newcommand{\f}[2]{\frac{#1}{#2}}
\usepackage{newpxtext,newpxmath}
\usepackage[left=1in,right=1in,top=1.25in,bottom=1.25in]{geometry}
\usepackage{framed}
\usepackage{enumerate}

\usepackage{caption}
\usepackage{subcaption}

%\newcommand{\fig}[1]{figure #1}
%\newcommand{\explain}{appendix?}
%\newcommand{\rat}{\mathbb{Q}}
%
%\newcommand{\mathbb{R}}{\mathbb{R}}
%\newcommand{\nat}{\mathbb{N}}
%\newcommand{\inte}{\mathbb{Z}}
%\newcommand{\M}{{\cal{M}}}
%\newcommand{\sss}{{\cal{S}}}
%\newcommand{\rrr}{{\cal{R}}}
%\newcommand{\uu}{2pt}
%\newcommand{\vv}{\vec{v}}
%\newcommand{\comp}{\mathbb{C}}
%\newcommand{\field}{\mathbb{F}}
%\newcommand{\f}[1]{ \hspace{.1in} (#1) }
%\newcommand{\set}[2]{\mbox{$\left\{ \left. #1 \hspace{3pt}
%\right| #2 \hspace{3pt} \right\}$}}
%\newcommand{\integral}[2]{\int_{#1}^{#2}}
%\newcommand{\ba}{\hookrightarrow}
%\newcommand{\ep}{\varepsilon}
%\newcommand{\limit}{\operatornamewithlimits{limit}}
%\newcommand{\ddd}{.1in}
%\newcommand{\ccc}{2in}
%\newcommand{\aaa}{1.5in}
%\newcommand{\B}{{\cal B}}
%\newcommand{\C}{{\cal C}}
%\newcommand{\D}{{\cal D}}
%\newcommand{\FF}{{\cal F}}
\usepackage{amssymb}% http://ctan.org/pkg/amssymb
\usepackage{pifont}% http://ctan.org/pkg/pifont
\newcommand{\cmark}{\ding{51}}%
\newcommand{\xmark}{\ding{55}}%
\newcommand{\p}{\partial}%
%\usepackage{MnSymbol,wasysym}



\usepackage{listings}
\captionsetup[lstlisting]{margin=0cm,format=hang,font=small,format=plain,labelfont={bf,up},textfont={it}}
\renewcommand*{\lstlistingname}{Code \textcolor{violet}{\textsl{Mathematica}}}
\definecolor{gris245}{RGB}{245,245,245}
\definecolor{olive}{RGB}{50,140,50}
\definecolor{brun}{RGB}{175,100,80}
\lstset{
	tabsize=4,
	frame=single,
	language=mathematica,
	basicstyle=\scriptsize\ttfamily,
	keywordstyle=\color{black},
	backgroundcolor=\color{gris245},
	commentstyle=\color{gray},
	showstringspaces=false,
	emph={
		r1,
		r2,
		epsilon,epsilon_,
		Newton,Newton_
	},emphstyle={\color{olive}},
	emph={[2]
		L,
		CouleurCourbe,
		PotentielEffectif,
		IdCourbe,
		Courbe
	},emphstyle={[2]\color{blue}},
	emph={[3]r,r_,n,n_},emphstyle={[3]\color{magenta}}
}






\begin{document}
\begin{center}
{\large \bf PH401: Stochastic Resonance.  Reflection \#3}\\
{ Huan Q. Bui}\\
April 11, 2021
\end{center}

\noindent \textbf{Wiesenfeld and Moss:} How noise can affect detection of weak periodic signals.  Noise can enhance the detection of weak signals via stochastic resonance. Here, ``weak'' means that the signal is small enough to not excite the system. SR happens when some non-zero noise level induces a transition in the system. Note that stochastic resonance is not the same as ``dither,'' a technique wherein periodic or random forcing is intentionally introduced to overcome regions of ``dead'' dynamical behavior in self-regulating systems. \\


Q: In the simple differential equation in the paper. What is $\omega$, the frequency of the noise? Does it have to match or be close to the ``natural'' frequency of the system? Or does it not matter?  \\




\noindent \textbf{Pikovsky and Kurths:} How noise can affect the spiking dynamics of neurons and other excitable systems. They study the dynamics of the excitable FHN system under external noisy driving. The noise excites the system, shown by a production of a sequence of pulses. At certain noise amplitudes, the coherence of the excitation reaches a maximum. Here, the FHN with noise is numerically integrated. For small or large noise amplitudes, the noise-excited oscillations are irregular. However, for some moderate noise level, relatively coherent oscillations are observed. The paper calls this phenomenon \textit{coherent resonance}, which resembles stochastic resonance. \\

The difference between SR and CR is that in SR both periodic and noisy forces drive the system, with the periodic response maximizing at some noise amplitude. In CR, however, no periodic force and no discrete component appear in the spectrum, but still at some noise amplitude the regularity maximizes. \\



\noindent \textbf{Kadar, Wang, and Showalter:} How noise influences the emergence of wave patterns in the BZ reaction. The paper reports the positive influence of noise on wave propagation in a photosensitive BZ reaction. The chemical medium is not excitable nor can it sustain wave propagation. However, with noisy light, wave propagation is enhanced and is achieved at an optimal level below/above which wave propagation weakens. The figures in the paper nicely illustrate this.  \\










\end{document}




