\documentclass[11pt]{article}
\usepackage{amsmath}
\usepackage{physics}
\usepackage{amssymb}
\usepackage{graphicx}
\usepackage{hyperref}
\usepackage{amsfonts}
\usepackage{cancel}
\usepackage{xcolor}
\hypersetup{
	colorlinks,
	linkcolor={black!50!black},
	citecolor={blue!50!black},
	urlcolor={blue!80!black}
}
\newcommand{\f}[2]{\frac{#1}{#2}}
\usepackage{newpxtext,newpxmath}
\usepackage[left=1.25in,right=1.25in,top=1.25in,bottom=1.25in]{geometry}
\usepackage{framed}
\usepackage{enumerate}

\usepackage{caption}
\usepackage{subcaption}

%\newcommand{\fig}[1]{figure #1}
%\newcommand{\explain}{appendix?}
%\newcommand{\rat}{\mathbb{Q}}
%
%\newcommand{\mathbb{R}}{\mathbb{R}}
%\newcommand{\nat}{\mathbb{N}}
%\newcommand{\inte}{\mathbb{Z}}
%\newcommand{\M}{{\cal{M}}}
%\newcommand{\sss}{{\cal{S}}}
%\newcommand{\rrr}{{\cal{R}}}
%\newcommand{\uu}{2pt}
%\newcommand{\vv}{\vec{v}}
%\newcommand{\comp}{\mathbb{C}}
%\newcommand{\field}{\mathbb{F}}
%\newcommand{\f}[1]{ \hspace{.1in} (#1) }
%\newcommand{\set}[2]{\mbox{$\left\{ \left. #1 \hspace{3pt}
%\right| #2 \hspace{3pt} \right\}$}}
%\newcommand{\integral}[2]{\int_{#1}^{#2}}
%\newcommand{\ba}{\hookrightarrow}
%\newcommand{\ep}{\varepsilon}
%\newcommand{\limit}{\operatornamewithlimits{limit}}
%\newcommand{\ddd}{.1in}
%\newcommand{\ccc}{2in}
%\newcommand{\aaa}{1.5in}
%\newcommand{\B}{{\cal B}}
%\newcommand{\C}{{\cal C}}
%\newcommand{\D}{{\cal D}}
%\newcommand{\FF}{{\cal F}}
\usepackage{amssymb}% http://ctan.org/pkg/amssymb
\usepackage{pifont}% http://ctan.org/pkg/pifont
\newcommand{\cmark}{\ding{51}}%
\newcommand{\xmark}{\ding{55}}%
\newcommand{\p}{\partial}%
%\usepackage{MnSymbol,wasysym}

\begin{document}
\begin{center}
{\large \bf PH401: Reflection \#2}\\
{ Huan Q. Bui}\\
Mar 13 2021
\end{center}
\noindent \textbf{1.} \textit{Phase behaviour of concentrated suspensions of nearly hard colloidal spheres} by Pusey and van Megen.
\begin{itemize}
	\item Studied ``colloidal fluids,'' which from my understanding are fluids with tiny particles suspended in them (like mint flakes in some toothpastes). 
	\item They studied specifically the phase diagram. Here the ``phase diagram'' is a plot of the volume of crystals in the fluid versus the concentration of the colloidal particles. 
	\item What they found was a ``phase transition'' in the sense that: as the concentration increases, the colloidal fluid goes from being a colloidal fluid to coexisting fluid + crystals to fully crystallized sample.  But when the concentration is really high, crystallization did not occur. Rather, they ended up with viscous ``colloidal glass.'' 
	\item I like the fact that they set the fluid up so that it had the same index of refraction as the particles', which allowed for light-scattering measurements. 
	\item The experiments took quite long! They had to wait for months on some of the samples.
	\item Sample 8 is really beautiful... It's hard to believe that that crystalline structure is made completely out of tiny, suspended ``spheres.'' This shows that crystallization doesn't happen only at the atomic/molecular level.
\end{itemize}


\noindent \textbf{2.}  \textit{Real-Space Imaging of Nucleation and Growth in Colloidal Crystallization} by Gasser et al. 
\begin{itemize}
	\item The imaging technique here gave the authors better insights into the crystallization process in concentrated colloidal suspension. 
	\item With this technique, they were able to find the critical nuclei, determine nucleation rates, and measured the average surface tension of the crystal-liquid interface. 
	\item A lot of this turns out to be thermodynamics (which is a bit surprising at first sight). Here, crystallization is a competition between surface and bulk energy, with the end goal to minimize the total energy (to put very roughly). 
	\item Turns out that when the crystallites reach a certain size, called the critical size, crystal growth becomes more thermodynamically favorable. 
\end{itemize}




\noindent \textbf{3.} \textit{Local Melting Attracts Grain Boundaries in Colloidal Polycrystals}, by Gerbode et al. 
\begin{itemize}
	\item They found that by shining a focused beam of laser to locally melting a 2D colloidal crystal they could attract and deform grain boundaries. 
	\item They also demonstrated that with this, it is possible to fabricate materials with custom crystal grain boundaries for arbitrary purposes.
	\item Q: How ``arbitrary'' can we get with forming these grain boundaries? i.e. What is limiting our ability to make the crystal melt and recrystallize the way we want? Is this ``melting attracts boundaries'' phenomenon universal in all crystalline structures? 
\end{itemize}
 
  
\end{document}




