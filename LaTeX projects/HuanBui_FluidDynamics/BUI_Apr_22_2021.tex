\documentclass[11pt]{article}
\usepackage{amsmath}
\usepackage{physics}
\usepackage{amssymb}
\usepackage{graphicx}
\usepackage{hyperref}
\usepackage{amsfonts}
\usepackage{cancel}
\usepackage{xcolor}
\hypersetup{
	colorlinks,
	linkcolor={black!50!black},
	citecolor={blue!50!black},
	urlcolor={blue!80!black}
}
\newcommand{\f}[2]{\frac{#1}{#2}}
\usepackage{newpxtext,newpxmath}
\usepackage[left=1in,right=1in,top=1.25in,bottom=1.25in]{geometry}
\usepackage{framed}
\usepackage{enumerate}

\usepackage{caption}
\usepackage{subcaption}

%\newcommand{\fig}[1]{figure #1}
%\newcommand{\explain}{appendix?}
%\newcommand{\rat}{\mathbb{Q}}
%
%\newcommand{\mathbb{R}}{\mathbb{R}}
%\newcommand{\nat}{\mathbb{N}}
%\newcommand{\inte}{\mathbb{Z}}
%\newcommand{\M}{{\cal{M}}}
%\newcommand{\sss}{{\cal{S}}}
%\newcommand{\rrr}{{\cal{R}}}
%\newcommand{\uu}{2pt}
%\newcommand{\vv}{\vec{v}}
%\newcommand{\comp}{\mathbb{C}}
%\newcommand{\field}{\mathbb{F}}
%\newcommand{\f}[1]{ \hspace{.1in} (#1) }
%\newcommand{\set}[2]{\mbox{$\left\{ \left. #1 \hspace{3pt}
%\right| #2 \hspace{3pt} \right\}$}}
%\newcommand{\integral}[2]{\int_{#1}^{#2}}
%\newcommand{\ba}{\hookrightarrow}
%\newcommand{\ep}{\varepsilon}
%\newcommand{\limit}{\operatornamewithlimits{limit}}
%\newcommand{\ddd}{.1in}
%\newcommand{\ccc}{2in}
%\newcommand{\aaa}{1.5in}
%\newcommand{\B}{{\cal B}}
%\newcommand{\C}{{\cal C}}
%\newcommand{\D}{{\cal D}}
%\newcommand{\FF}{{\cal F}}
\usepackage{amssymb}% http://ctan.org/pkg/amssymb
\usepackage{pifont}% http://ctan.org/pkg/pifont
\newcommand{\cmark}{\ding{51}}%
\newcommand{\xmark}{\ding{55}}%
\newcommand{\p}{\partial}%
%\usepackage{MnSymbol,wasysym}



\usepackage{listings}
\captionsetup[lstlisting]{margin=0cm,format=hang,font=small,format=plain,labelfont={bf,up},textfont={it}}
\renewcommand*{\lstlistingname}{Code \textcolor{violet}{\textsl{Mathematica}}}
\definecolor{gris245}{RGB}{245,245,245}
\definecolor{olive}{RGB}{50,140,50}
\definecolor{brun}{RGB}{175,100,80}
\lstset{
	tabsize=4,
	frame=single,
	language=mathematica,
	basicstyle=\scriptsize\ttfamily,
	keywordstyle=\color{black},
	backgroundcolor=\color{gris245},
	commentstyle=\color{gray},
	showstringspaces=false,
	emph={
		r1,
		r2,
		epsilon,epsilon_,
		Newton,Newton_
	},emphstyle={\color{olive}},
	emph={[2]
		L,
		CouleurCourbe,
		PotentielEffectif,
		IdCourbe,
		Courbe
	},emphstyle={[2]\color{blue}},
	emph={[3]r,r_,n,n_},emphstyle={[3]\color{magenta}}
}






\begin{document}
\begin{center}
{\large \bf PH312: Reflection \#9}\\
{ Huan Q. Bui}\\
April 22, 2021
\end{center}


\noindent \textbf{1.} The \textit{shower curtain effect}

\begin{enumerate}
	\item \textit{Possible explanation 1: Bernoulli's principle.} This explanation basically says that when we turn on the shower, the spray creates a low-pressure zone due to the motion of water. Meanwhile, the pressure outside the shower curtain stays the same. This pressure difference makes the curtain move in. (Source: \href{https://www.npr.org/templates/story/story.php?storyId=6430581}{[NPR] Arggh, Why Does the Shower Curtain Attack Me?}). 
	
	\item \textit{Possible explanation 2: Horizontal vortex formation.} David Schmidt's computer simulation shows that the spray drives a vortex that is perpendicular to the shower curtain. The center of this vortex, as we discussed in class, is a low-pressure region. This pulls the shower curtain in. Since the shower doesn't stop, the vortex is sustained, and the curtain stays pulled in. (Source: \href{https://www.scientificamerican.com/article/why-does-the-shower-curta/}{[Scientific American] Why Does the Shower Curtain Move Toward the Water?})
	
	\item \textit{Possible explanation 3: Coanda effect.} There's a tendency of water in motion to adhere to a surface of the curtain.  This tendency pulls the curtain inward towards the water. (Source: \href{https://www.straightdope.com/21342516/why-does-the-shower-curtain-blow-in-despite-the-water-pushing-it-out-revisited}{[Straight dope] Why does the shower curtain blow in despite the water pushing it out (revisited)?}) 
\end{enumerate}

I think the shower-curtain effect is possibly a combination of these effects. To be completely honest with you I like none of these explanations. Why can't we do an experiment and settle the discussion? To test \#3, can we spray some hydrophobic coating onto the shower curtain and see if the pulling effect changes? To test \#2, how about varying the speed, direction, and the spread of the water coming of the shower head? If the horizontal vortex theory is true, then change the location and strength of the vortex should change how the curtain behaves. To test \#1, can we vary the speed at which the water comes out and see what happens? I also came across other explanations that have to do with convection due to water vs. air temperatures. Can we run experiments that take into account water temperature as well? \\



\noindent \textbf{2.} Bernoulli's principle, misused...
\begin{enumerate}
	\item \textit{Equal transit theory:} This theory assumes that the ``parcels of air'' that separate at the wing's leading edge must rejoin at the trailing edge, and so because the top parcel travels further due to the shape of the wing, it must go faster. The flaw here is of course the ``equal transit'' assumption, which is actually unfounded. There are experiments that show that this theory doesn't hold.  (Source: \href{https://www.scientificamerican.com/article/no-one-can-explain-why-planes-stay-in-the-air/#:~:text=According 20to 20the 20most 20common,time 2C 20it 20must 20go 20faster.}{[Scientific American]No One Can Explain Why Planes Stay in the Air})
	
	
	\item Incorrect explanation from some MIT 16.00: Introduction to Aerospace Engineering and Design. \href{https://web.mit.edu/16.00/www/aec/flight.html#:~:text=Air 20flowing 20over 20an 20airfoil 20will 20decrease 20in 20pressure.,This 20pressure 20force 20is 20lift.}{(Theory of Flight)}. 
	I also found this page by NASA \href{https://www.grc.nasa.gov/WWW/k-12/airplane/wrong1.html}{(Incorrect Theories)} which lists two other (false) explanations: the ``skipping stone'' explanation and ``Venturi theory''. \\
	
	
	So it's clear that there are a lot of misconceptions about the origins of lift. Every now and then there is an article which basically says ``nobody knows how planes fly:'' \textit{The secret to airplane flight. No one really knows.} in the National Newspaper, 2012. \textit{There's No One Way to Explain How Flying Works} in Wired Magazine, 2018.
	\textit{No One Can Explain Why Planes Stay in the Air.} in the Scientific American magazine, 2020, etc. 
	
	
	\item 
	\begin{itemize}
		\item \textit{How do wings work?} Babinsky, Phys. Educ. (2003).\\
		
		This paper uses the analysis of pressure gradients and the curvature of streamlines to give a more correct explanation of lift. The authors begin with discussing some possible (false) explanations for lift. These include the equal transit theory, but also the Bernoulli explanation using speed differential. I think in class we said that Bernoulli's principle is part of the explanation, but apparently this is really true either: a speed differential across a straight piece of paper hanging vertically induces a pressure doesn't induce any movement. The author says: ``It is false to make a connection between the flow
		on the two sides of the paper using Bernoulli's
		equation.'' The reason is that Bernoulli's equation applies only along a streamline. There is no explicit relationship between the
		pressure and velocity of neighboring streamlines. However, if a streamline is
		curved, then there is a pressure gradient across the streamline, with the pressure increasing in the direction away from the center of curvature. This induces a force. \\
		
		
		\item Understanding wing lift. Silva and Soares, Phys. Educ. (2010).\\
		
		This paper adds air friction to the explanation given in the previous paper to make things more precise, i.e., the author wants to include boundary layer effects. Including this effects means that the air viscosity, Coanda effect and
		aerofoil contours change air
		flow direction and accelerate air down around the aerofoil. This creates a centripetal force, which creates a pressure gradient and thus a net force. The paper then outlines how the shape o aerofoils allow these effects to occur. 
		
		
		
	\end{itemize}
\end{enumerate}

\end{document}




