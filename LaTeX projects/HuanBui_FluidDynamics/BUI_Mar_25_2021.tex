\documentclass[11pt]{article}
\usepackage{amsmath}
\usepackage{physics}
\usepackage{amssymb}
\usepackage{graphicx}
\usepackage{hyperref}
\usepackage{amsfonts}
\usepackage{cancel}
\usepackage{xcolor}
\hypersetup{
	colorlinks,
	linkcolor={black!50!black},
	citecolor={blue!50!black},
	urlcolor={blue!80!black}
}
\newcommand{\f}[2]{\frac{#1}{#2}}
\usepackage{newpxtext,newpxmath}
\usepackage[left=0.9in,right=0.9in,top=0.9in,bottom=0.9in]{geometry}
\usepackage{framed}
\usepackage{enumerate}

\usepackage{caption}
\usepackage{subcaption}

%\newcommand{\fig}[1]{figure #1}
%\newcommand{\explain}{appendix?}
%\newcommand{\rat}{\mathbb{Q}}
%
%\newcommand{\mathbb{R}}{\mathbb{R}}
%\newcommand{\nat}{\mathbb{N}}
%\newcommand{\inte}{\mathbb{Z}}
%\newcommand{\M}{{\cal{M}}}
%\newcommand{\sss}{{\cal{S}}}
%\newcommand{\rrr}{{\cal{R}}}
%\newcommand{\uu}{2pt}
%\newcommand{\vv}{\vec{v}}
%\newcommand{\comp}{\mathbb{C}}
%\newcommand{\field}{\mathbb{F}}
%\newcommand{\f}[1]{ \hspace{.1in} (#1) }
%\newcommand{\set}[2]{\mbox{$\left\{ \left. #1 \hspace{3pt}
%\right| #2 \hspace{3pt} \right\}$}}
%\newcommand{\integral}[2]{\int_{#1}^{#2}}
%\newcommand{\ba}{\hookrightarrow}
%\newcommand{\ep}{\varepsilon}
%\newcommand{\limit}{\operatornamewithlimits{limit}}
%\newcommand{\ddd}{.1in}
%\newcommand{\ccc}{2in}
%\newcommand{\aaa}{1.5in}
%\newcommand{\B}{{\cal B}}
%\newcommand{\C}{{\cal C}}
%\newcommand{\D}{{\cal D}}
%\newcommand{\FF}{{\cal F}}
\usepackage{amssymb}% http://ctan.org/pkg/amssymb
\usepackage{pifont}% http://ctan.org/pkg/pifont
\newcommand{\cmark}{\ding{51}}%
\newcommand{\xmark}{\ding{55}}%
\newcommand{\p}{\partial}%
%\usepackage{MnSymbol,wasysym}

\begin{document}
\begin{center}
{\large \bf PH312: Physics of Fluids (Prof. McCoy)}\\
{ Huan Q. Bui}\\
\today
\end{center}
\noindent \textbf{1.}
\begin{enumerate}[(a)]
	\item Bolster et al's: This article talks about dynamical similarity: using dimensional analysis to create parallels between physical systems, which allows us insights into systems that are too difficult to fabricate in the laboratory setting. The article goes over a wide range of famous physical problems to which the key insights can be drawn from dimensional analysis alone. The article also shows that from dimensional analysis one can model real world systems in the lab so long as these models are \textit{dynamically similar} to their to-scale counterpart. This idea is supported by a number of examples, most of which from fluid dynamics. Despite some of its limitations, dynamical similarity can be used to solve problems not just in physics but also in biology, economics, etc. 
 	
	
	
	\item Hecksher's: That dimensional analysis is a very neat tool physicists use not just to check their work but also to derive new theories or work out scaling laws. To make this point clear, the articles brings up many classic examples including the Reynolds number (which we all know and love) and Planck's constant, which sets the scale of the quantum world. 
\end{enumerate}





\noindent \textbf{2.} 
To nondimensionalize the magnetohydrodynamic equations, we remove all units by defining 
\begin{equation*}
\mathbf{u}' = \mathbf{u}/U \quad \mathbf{x'} = \mathbf{x}/L \quad \quad \mathbf{B}' = \mathbf{B}/B_0 \quad P' = P/\rho U^2 \quad \Re = UL/\nu
\end{equation*}
Since these are linear scaling, the first two equations look the same:
\begin{equation*}
\boxed{\nabla' \cdot \mathbf{u}' = 0} \quad \boxed{\nabla' \cdot \mathbf{B}' = 0}
\end{equation*}
It remains to nondimensionalize the third and fourth equations. For the third equation:
\begin{align*}
\mathbf{u} \cdot \nabla \mathbf{u} &= -\f{1}{\rho}\nabla P + \f{1}{\rho \mu}(\nabla \times \mathbf{B})\times \mathbf{B} + \nu \nabla^2 \mathbf{u} \\
\f{U^2}{L}  \mathbf{u}' \cdot \nabla' \mathbf{u}'
&= -\f{\rho U^2}{L\rho}\nabla' P' + \f{B_0^2}{ L \rho \mu} (\nabla' \times \mathbf{B}')\times \mathbf{B}' + \f{\nu U }{L^2} \nabla'^2 \mathbf{u}',
\end{align*}
which becomes
\begin{equation*}
\boxed{
\mathbf{u}' \cdot \nabla' \mathbf{u}'
= -\nabla' P' + \f{B_0^2 }{  \rho \mu U^2} (\nabla' \times \mathbf{B}')\times \mathbf{B}' + \f{1}{\Re} \nabla'^2 \mathbf{u}'
}
\end{equation*}
On to the fourth equation. After defining the magnetic Reynolds number $\Re_m = UL/\eta$ and transforming the variables we find
\begin{align*}
\f{UB_0}{L}\mathbf{u}' \cdot \nabla' \mathbf{B} 
&= \f{UB_0}{L}\mathbf{B}' \cdot \nabla' \mathbf{u}' + \f{\eta B_0}{L^2}  \nabla'^2\mathbf{B}',
\end{align*}
which gives
\begin{equation*}
\boxed{
\mathbf{u}' \cdot \nabla' \mathbf{B}'
= \mathbf{B}' \cdot \nabla' \mathbf{u}' + \f{1}{\Re_m} \nabla'^2\mathbf{B}'
}
\end{equation*}
We know that $\Re$ tells us the strength of the viscous term (momentum diffusion) relative to the inertial term. Similarly, $\Re_m$ tells us about the relative strength of magnetic diffusion relative to the transport of the magnetic field by the flow. The term $B_0^2/\rho\mu U^2$ involves both the magnetic energy density and fluid density and velocity scale, so it must represent the relative strength of the \textit{interaction} between magnetic and mechanical flow effects. 





  
\end{document}




