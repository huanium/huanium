\documentclass[11pt]{article}
\usepackage{amsmath}
\usepackage{physics}
\usepackage{amssymb}
\usepackage{graphicx}
\usepackage{hyperref}
\usepackage{amsfonts}
\usepackage{cancel}
\usepackage{xcolor}
\hypersetup{
	colorlinks,
	linkcolor={black!50!black},
	citecolor={blue!50!black},
	urlcolor={blue!80!black}
}
\newcommand{\f}[2]{\frac{#1}{#2}}
\usepackage{newpxtext,newpxmath}
\usepackage[left=1.25in,right=1.25in,top=0.9in,bottom=0.9in]{geometry}
\usepackage{framed}
\usepackage{caption}
\usepackage{subcaption}





\begin{document}
\begin{center}
{\large \bf PH312: Physics of Fluids (Prof. McCoy) -- Reflection}\\
{ Huan Q. Bui}\\
Mar 04 2021
\end{center}

\begin{framed}
	\noindent Reflection on the readings:
	\begin{enumerate}
		\item Kundu \& Cohen 2.10 - 2.12, 3.7 - 3.9
		\item Tritton 6.4, Kundu \& Cohen 3.8, 3.10 - 3.12
		\item Kundu \& Cohen 2.6, 4.5, 4.6
	\end{enumerate}
\end{framed}

For me, one big takeaway from this week's lectures and readings is the \textit{physical interpretation} of some of the mathematical objects we've seen so far in the (Eulerian) language of fluid dynamics. I remember back in sophomore year when I was taking Vector Calculus with Professor Otto K. Bretscher how I used to spend many hours in his office trying to make sense of what the Big Trio (div, grad, and curl) meant physically and going through example after example of how tricky things can be sometimes. I remember particularly a vector field of the form
\begin{equation*}
\vec{u}(x,y,z) = \left( \frac{-y}{x^2+y^2}, \f{x}{x^2+y^2} ,0 \right),
\end{equation*}
of which you have kindly reminded me in Wednesday's lecture. I learned that this is a classically sneaky vector field, because it is an irrotational field that is not conservative (the positive closed loop integral around the $z$-axis gives $2\pi$ and 0 otherwise), but never quite understood why it is called ``irrotational'' when clearly the field lines are counter-clockwise concentric circles. Later, I learned that the curl of $\vec{u}$ actually only describes the flow field \textit{locally}, and it all made sense. As a result, the distinction we made between vorticity (local) and vortices (global) on Wednesday came very naturally to me. \\

\noindent Two other aspects that I find interesting from this week are (1) the decomposition of the stress and velocity gradient tensors and (2) the discussion around the symmetric stress tensor. It is still very surprising to me that the simple action of breaking the velocity gradient tensor up into a symmetric and anti-symmetric piece actually gives us useful information about how a fluid element changes in volume or deforms. Parallel to this is the decomposition of the symmetric stress tensor into an isotropic piece plus a traceless piece, which curiously tells us something about pressure exerted on a fluid element and the element's tendency to deform. This is something I did not see in my math classes. I remember taking Advanced Linear Algebra with Professor Leo Livshits, where I learned all about symmetric (or more generally, self-adjoint) matrices and showed that they are orthogonally (or more generally, unitarily) diagonalizable with real spectrum, but never quite thought about viewing the matrix as a combination of these ``actions'' until I took Quantum Mechanics and, now, Physics of Fluids. Consequently, it feels very nice whenever I could get some physical interpretation out of an abstract mathematical concept, because the physics and the math now go hand-in-hand, forever understood by me.  


  
\end{document}




