\documentclass[11pt]{article}
\usepackage{amsmath}
\usepackage{physics}
\usepackage{amssymb}
\usepackage{graphicx}
\usepackage{hyperref}
\usepackage{amsfonts}
\usepackage{cancel}
\usepackage{xcolor}
\hypersetup{
	colorlinks,
	linkcolor={black!50!black},
	citecolor={blue!50!black},
	urlcolor={blue!80!black}
}
\usepackage{newpxtext,newpxmath}
\usepackage[left=0.75in,right=0.75in,top=0.75in,bottom=0.9in]{geometry}
\usepackage{framed}
\usepackage{caption}
\usepackage{subcaption}

%\newcommand{\fig}[1]{figure #1}
%\newcommand{\explain}{appendix?}
%\newcommand{\rat}{\mathbb{Q}}
%
%\newcommand{\mathbb{R}}{\mathbb{R}}
%\newcommand{\nat}{\mathbb{N}}
%\newcommand{\inte}{\mathbb{Z}}
%\newcommand{\M}{{\cal{M}}}
%\newcommand{\sss}{{\cal{S}}}
%\newcommand{\rrr}{{\cal{R}}}
%\newcommand{\uu}{2pt}
%\newcommand{\vv}{\vec{v}}
%\newcommand{\comp}{\mathbb{C}}
%\newcommand{\field}{\mathbb{F}}
%\newcommand{\f}[1]{ \hspace{.1in} (#1) }
%\newcommand{\set}[2]{\mbox{$\left\{ \left. #1 \hspace{3pt}
%\right| #2 \hspace{3pt} \right\}$}}
%\newcommand{\integral}[2]{\int_{#1}^{#2}}
%\newcommand{\ba}{\hookrightarrow}
%\newcommand{\ep}{\varepsilon}
%\newcommand{\limit}{\operatornamewithlimits{limit}}
%\newcommand{\ddd}{.1in}
%\newcommand{\ccc}{2in}
%\newcommand{\aaa}{1.5in}
%\newcommand{\B}{{\cal B}}
%\newcommand{\C}{{\cal C}}
%\newcommand{\D}{{\cal D}}
%\newcommand{\FF}{{\cal F}}


%\usepackage{epstopdf}
%\DeclareGraphicsRule{.tif}{png}{.png}{`convert #1 `basename #1 .tif`.png}
%\usepackage{graphics}
%\usepackage{array}
%\def\set#1#2{\left\{\left.\;#1\;\right| #2 \; \right\}}
%\def\Sum{\sum}
%\def\me{.05in}















\begin{document}
\begin{center}
{\large \bf PH312: Physics of Fluids (Prof. McCoy) -- Reflection}\\
{ Huan Q. Bui}\\
Feb 20 2021
\end{center}




\begin{enumerate}
	\item \textit{Big whorls, little whorls}. 
	\begin{enumerate}
		\item It turns out that there is a connection between turbulence and predator-prey behaviors. Scientists have long had this intuitive understanding by suggesting ``small whorls'' \textit{feed on} ``big whorls'' which \textit{feed on} ``bigger whorls,'' and so on. But how we elucidate this link and make precise our intuition?
		
		\item Numerical results by HY Shih \textit{et al.}'s, followed by two experimental findings by M. Sano \&  K. Tamai and Lemoult \textit{et al} recently contributed substantially towards making this link clear. 
		
		\item HY Shih \textit{et al.}: There is a link between transition to turbulence and the \textbf{universality class} describing \textbf{directed percolation}. This is related to Y. Pomeau's conjecture (30 years ago) that the transition to turbulence in shear flows might be understood in terms of an \textbf{absorbing phase transition}. 
		
		\item The said two experiments, while very different, supported Pomeau's conjecture. They found evidence for \textbf{critical exponents} consistent with directed percolation, which puts transition to turbulence into a well-known universality class. This of course allows us to further our understanding of this phenomenon.

	
	\item \textbf{Questions/Thoughts} 
	\begin{itemize}
		\item I like how research on the link between transition to turbulence and predator-prey dynamics came out of our intuitive understanding of the phenomenon and how we describe it.
		
		\item There are terms here which I don't fully understand but I can see that scientists try to understanding transition to turbulence by ``mapping'' the problem to a more familiar and well-known system. I like how ubiquitous this strategy is in solving hard problems.
		
		\item I'd like to understand what the following: \textbf{Directed percolation, Universality class, Critical exponent, Phase transition} (in the sense of fluid dynamics and general thermodynamics).
	\end{itemize}
	 
	\end{enumerate}
	
	
	\item Mark Buchanan, \textit{Transition to turbulence}.
	\begin{enumerate}
		\item We still don't know (from first principles) how laminar flow transitions to turbulence, but we have a few pieces of information.
		
		\item There is a critical Reynolds numbers, Re$_\text{c}$ below which turbulence die out and above which turbulence persists. How does this transition happen? 
		
		\begin{itemize}
			\item When Re $\ll$ Re$_\text{c}$, turbulence exists as \textbf{localized puffs} separated by \textbf{laminar zones}. The lifetime of these puffs grows super-exponentially in Re. 
			
			\item When Re $\gg$ Re$_\text{c}$, turbulence persists, but puffs can \textbf{split} and spread turbulence. The rate at which these puffs split also grows super-exponentially in Re. The \textit{combination} of the lifetime and the rate at which the puffs split may be key to transition to turbulence. 
		\end{itemize}
	
		\item Relation to excitable systems: D. Barkley gave a model in which turbulent puffs in a pipe acts like action potentials in nerve axons. The model turns out to agree with Pomeau's conjectures. 
		
		\item Relation to directed percolation: Goldenfeld \textit{et al.} found that  zonal flows compete with
		turbulence following predator-prey interaction. Here, \textbf{laminar flow $=$ nutrient}, which
		\textbf{turbulence (prey)} feeds on and spreads. Turbulence can be fed on by \textbf{zonal flows (predator)}. This predator-prey model then maps onto a statistical model in the well-known directed percolation class.
		
		\item \textbf{Questions/Thoughts:} This article makes it clearer how transition to turbulence can be talked about in the language of predator and prey. But just what is \textbf{zonal flow} (for more see next article)? What does mean for a puff to \textbf{split}?
	\end{enumerate}
	
	
	\item Johannes Knebel, Markus F. Weber and Erwin Frey, \textit{In pursuit of turbulence}.
	\begin{enumerate}
		\item  Transition from laminar to turbulent flow sets in without a linear instability of the NS equation: small perturbations to the laminar flow decay for all Re. This
		is unlike the self-organization of Rayleigh-B\'{e}nard cells, which arise out of a linear instability of the NS equations.
		
		\item HY Shih observed new patterns when solving NS equation for pipe flow. This is \textbf{zonal flow}. They form in the transitional regime between
		laminar and turbulent flow. Zonal flow \textbf{is activated by anisotropic turbulent fluctuations, which in turn are inhibited by zonal flow.} This is why transition to turbulence resembles predator-prey dynamics.
		
		\item Many puzzles remain. The NS equation provides a complete description of fluids, but there are still mysteries. Why do critical exponents occur, and why/how are they universal? 
		
		
		\item \textbf{Questions/Thoughts:} The bit on zonal flow sheds some light on what it means. I tried to read the original article by Shih and Goldenfeld but couldn't see what it means \textit{physically}. 
	\end{enumerate}
	
	
	\item Yves Pomeau, \textit{The long and winding road}.
	\begin{enumerate}
		\item Linear stability theory alone can't explain transition to turbulence (as we have seen explained in the previous article). There were also experimental observations unexplained by theorists.
		
		\item Lev Landau followed Poincar\'{e} and proposed a theory for transition to turbulence in terms of \textbf{amplitude theory} rather than standard fluid-mechanical description using velocity fields. There arise some connections between Landau's language and thermodynamics. Two kinds of instability: subcritical and supercritical. 
		
		\item This view helps explain Reynolds'
		observation of ``turbulent'' domains growing
		or decaying in an otherwise linearly
		stable flow. Experiments also verified the connection to directed percolation.
		
		\item HY Shih then found some results that were consistent with the directed percolation class, as mentioned before. 
		
		
		\item \textbf{Questions/Thoughts:} This article is rather difficult to follow, since I'm very unfamiliar with the vocabulary. I can see that some mechanisms are being explained here and there, but in general I don't really \textit{understand} them. What is \textbf{subcritical bifurcation}? 
	\end{enumerate}



\item Hong-Yan Shih \textit{et al.} \textit{Ecological collapse and the emergence of traveling waves at the onset of shear turbulence}
\begin{enumerate}
	\item From the abstract: ``Our work demonstrates that a fluid on the edge of turbulence exhibits the same transitional scaling behavior as a predator-prey ecosystem on the edge of extinction, and establishes a precise connection with the DP universality class.''
	
	\item The main idea of this paper has been discussed in previous commentaries. Basically, what they did here is simulating flow in a pipe using NS equation. Then, they gave a predator-prey model based on some hypothesis, and showed that the predator-prey model actually behaves very much like the NS simulation. The task then is to quantify the comparison. 
	
	\item \textbf{Questions/Thoughts:} 
	\begin{itemize}
		\item I like how Figure 3 (which I'm assuming is a snapshot of Figure 2) shows how the fluid dynamics stuff is connected to the predator-prey behavior. 
		\item It's clear that the idea of \textbf{zonal flow} plays a very big role. The paper has mathematical definition of it, but I really would like to know what it means \textit{physically}. It would also nice to know how Shih recognized the importance of it from the research standpoint. 
	\end{itemize}
\end{enumerate}
\end{enumerate}






  
\end{document}




