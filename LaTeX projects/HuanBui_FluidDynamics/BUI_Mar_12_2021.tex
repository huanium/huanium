\documentclass[11pt]{article}
\usepackage{amsmath}
\usepackage{physics}
\usepackage{amssymb}
\usepackage{graphicx}
\usepackage{hyperref}
\usepackage{amsfonts}
\usepackage{cancel}
\usepackage{xcolor}
\hypersetup{
	colorlinks,
	linkcolor={black!50!black},
	citecolor={blue!50!black},
	urlcolor={blue!80!black}
}
\newcommand{\f}[2]{\frac{#1}{#2}}
\usepackage{newpxtext,newpxmath}
\usepackage[left=1.25in,right=1.25in,top=1.25in,bottom=1.25in]{geometry}
\usepackage{framed}
\usepackage{enumerate}

\usepackage{caption}
\usepackage{subcaption}

%\newcommand{\fig}[1]{figure #1}
%\newcommand{\explain}{appendix?}
%\newcommand{\rat}{\mathbb{Q}}
%
%\newcommand{\mathbb{R}}{\mathbb{R}}
%\newcommand{\nat}{\mathbb{N}}
%\newcommand{\inte}{\mathbb{Z}}
%\newcommand{\M}{{\cal{M}}}
%\newcommand{\sss}{{\cal{S}}}
%\newcommand{\rrr}{{\cal{R}}}
%\newcommand{\uu}{2pt}
%\newcommand{\vv}{\vec{v}}
%\newcommand{\comp}{\mathbb{C}}
%\newcommand{\field}{\mathbb{F}}
%\newcommand{\f}[1]{ \hspace{.1in} (#1) }
%\newcommand{\set}[2]{\mbox{$\left\{ \left. #1 \hspace{3pt}
%\right| #2 \hspace{3pt} \right\}$}}
%\newcommand{\integral}[2]{\int_{#1}^{#2}}
%\newcommand{\ba}{\hookrightarrow}
%\newcommand{\ep}{\varepsilon}
%\newcommand{\limit}{\operatornamewithlimits{limit}}
%\newcommand{\ddd}{.1in}
%\newcommand{\ccc}{2in}
%\newcommand{\aaa}{1.5in}
%\newcommand{\B}{{\cal B}}
%\newcommand{\C}{{\cal C}}
%\newcommand{\D}{{\cal D}}
%\newcommand{\FF}{{\cal F}}
\usepackage{amssymb}% http://ctan.org/pkg/amssymb
\usepackage{pifont}% http://ctan.org/pkg/pifont
\newcommand{\cmark}{\ding{51}}%
\newcommand{\xmark}{\ding{55}}%
\newcommand{\p}{\partial}%
%\usepackage{MnSymbol,wasysym}

\begin{document}
\begin{center}
{\large \bf PH312: Physics of Fluids (Prof. McCoy)}\\
{ Huan Q. Bui}\\
Mar 12 2021
\end{center}

\noindent \textbf{1.} 
\begin{enumerate}[(a)]
	\item In a liquid, we have linearly-vary pressure as a function of the liquid depth: 
	\begin{equation*}
	p(z) = p_0 + \rho g z.
	\end{equation*}
	Let the maximum height that the water can reach be $z=h$ (and $z=0$ be the ground).  At $z=0$, the pressure must equal atmospheric pressure, so $p_0 = 1$ atm $ = 101325$ Pa. At $z=h$, the $p(h) = 0$ since the height is maximal. Thus we have
	\begin{equation*}
	101325 \mbox{ Pa} = 1000 \mbox{ kg/m$^3$} \times 9.8 \mbox{ m/s$^2$} \times h \implies h \approx 10.3 \mbox{ m}.
	\end{equation*}
	
	\item Mercury's density is $13.6$ g/cm$^3$, which is $13.6$ times that of water, so $h_{\mbox{\tiny{Hg}}} = h/13.6 \approx 760$ mm (or mmHg).
	
	\item Assuming standard conditions ($T = $ 298 K, $M = 28.8\times 10^{-3}$ kg/mol), we have
	\begin{equation*}
	p(50)/p_0 \approx e^{-1.244 \times 10^{-4} \times 50} \approx 0.9937.
	\end{equation*}
	This pressure drop is about $0.6\%$ of 1 atm, corresponds to a drop of about 4-5 mm in the Hg level, which is measurable. 
\end{enumerate}


\noindent \textbf{2.} The answer is the ratio of the densities: $f = 917/1024 \approx 89.6\%$. To see this, let the volume of the iceberg be $1$, which gives $m_\text{ice} = \rho_\text{ice}$. Call $f$ the fraction submerged. Then by Archimedes' principle we have $m_\text{ice} g = \rho_\text{ice} g = (f\times 1)\rho_\text{w}g$, and so $f = \rho_\text{ice}/\rho_\text{w}$. \\

\noindent \textbf{3.} 
\begin{enumerate}[(a)]
	\item Let $\rho_f = 1.05$ g/cm$^3$ be the density of the fish and $\rho = 1$ g/cm$^3$ that of water (fish don't (can't) usually live in pure water, but we'll just assume this). Let the volume of the fish when collapsed be $V_c$, and the volume when inflated be $V_i$. Since $g$ will eventually cancel out, let us just set $g=1$. Archimedes says $\rho_f V_c = V_i \rho $ for neutral buoyancy, so $V_i/V_c = \rho_f/\rho$. Thus, the ratio between the bladder and total inflated volume is 
	\begin{equation*}
	\f{V_i - V_c}{V_i} = 1 - \f{\rho}{\rho_f} = 1 - \f{1}{1.05} \approx 4.76\%.
	\end{equation*}
	
	
	\item The pressure on the fish bladder changes linearly as a function of depth. So, even if the density of water remains constant at different depths, the ideal gas law may require it to add/remove gas from its bladder so that $V_i/V_c$ remains constant for neutral buoyancy. 
	
	\item Not all fish have to gulp air to add gas to their bladder. These fish may have \textit{gas glands} which produce chemicals that release some of the oxygen in their blood into the bladder. I think this is pretty interesting.\\
	
	If deeper-dwelling fish with closed bladders surface too quickly, the gases trapped in their bladder or dissolved in their tissues may expand/escape too quickly, which can be lethal. This is also why after diving we must wait a day before traveling on an airplane.
\end{enumerate}


\noindent \textbf{4.} 
\begin{enumerate}[(a)]
	\item 
	\begin{equation*}
	h = \f{2\sigma \sin \alpha}{\rho g R} = \f{2\times 0.073 \text{ N/m} \times \sin 90}{1000 \text{ kg/m}^{3} \times 9.8 \text{ m/s}^2 \times 1.5\times 10^{-3} \text{ m}} = 0.0099 \text{ m} \approx 0.99 \text{ cm}.
	\end{equation*}
	
	\item The radius ratio is $25 \times 10^{-6} /  1.5\times 10^{-3} = 0.0167$, so the height in this case is 0.99 cm/0.0167 $\approx$ 59.3 cm. 
	
	\item This answer is much smaller than expected (consider sequoias...). Capillary action plays a role, but it is not enough to bring water to hundreds of meters above ground. In turns out that water is also moved up the tree through \textit{transpiration}: the pressure differential created when water leaves leaves (sorry) creates a tension which pulls water from the ground. 
\end{enumerate}



\noindent \textbf{5.} Stout beer bubbles in a freshly poured Guinness pint  \href{https://www.youtube.com/watch?v=dSAqCahJyPE\&ab_channel=i20d1}{{appear}} to sink rather than rise. How can this be, when bubbles in a liquid must rise? This was an Internet mystery for a quite some time until William Lee, Professor of Mathematics at the University of Huddersfield, put matters to rest with a complete explanation in  \href{https://aapt.scitation.org/doi/full/10.1119/1.5021361}{\textit{Sinking bubbles in stout beers}}, which appeared in the American Journal of Physics in 2018. For this work, Lee received extensie press coverage, from World Science to BBC and MIT Technology Review, Phys.org, TIME,...\\

\noindent The Internet tends to make things seem more serious than they really are, and it is no exception here. With basic fluid dynamics, we can immediately rule out the possibility that the bubbles actually rise in the beer. In fact, other scientists have done experiments and simulations to check this and concluded that it is actually the downward current near the wall of the pint that carries the bubbles downward, which is possible due to the small size of the bubbles. The only question left is the mechanism through which these currents are formed; this is where William Lee comes in.  \\

\noindent Lee showed that the currents are due to the shape of the glass. The Guinness glass slopes inwards from top to bottom, and so as the bubbles rise, denser and bubble-free beer accumulates at the wall, which flows down due to gravity, dragging some bubbles with it. To confirm this theory, Lee used outward-sloping glasses and observed the opposite: as bubbles rise, they accumulate on the outward-sloping wall, which reduces the density of the beer in this region, which rises due to its buoyancy and carries more bubbles with it to the top. Here's the \href{https://www.youtube.com/watch?v=2_V2MRNlZOs\&ab_channel=MathematicalModelling}{demo} of this test.
  
\end{document}




